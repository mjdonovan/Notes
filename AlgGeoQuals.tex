% !TEX root = z_output/_AlgGeoQuals.tex
%%%%%%%%%%%%%%%%%%%%%%%%%%%%%%%%%%%%%%%%%%%%%%%%%%%%%%%%%%%%%%%%%%%%%%%%%%%%%%%%
%%%%%%%%%%%%%%%%%%%%%%%%%%% 80 characters %%%%%%%%%%%%%%%%%%%%%%%%%%%%%%%%%%%%%%
%%%%%%%%%%%%%%%%%%%%%%%%%%%%%%%%%%%%%%%%%%%%%%%%%%%%%%%%%%%%%%%%%%%%%%%%%%%%%%%%
\documentclass[11pt]{article}
\usepackage{fullpage}
\usepackage{amsmath,amsthm,amssymb}
\usepackage{mathrsfs,nicefrac}
\usepackage{amssymb}
\usepackage{epsfig}
\usepackage[all]{xy}
\usepackage{sseq}
\usepackage{tocloft}
\usepackage{cancel}
\usepackage[strict]{changepage}
\usepackage{color}
\usepackage{tikz}
\usepackage{extpfeil}
\usepackage{version}
%\usepackage{ifthen}
%Used for disabling hyperref
\ifx\dontloadhyperref\undefined
%\usepackage[pdftex,pdfborder={0 0 0 [1 1]}]{hyperref}
\usepackage[pdftex,pdfborder={0 0 .5 [1 1]}]{hyperref}
\else
\providecommand{\texorpdfstring}[2]{#1}
\fi

%>>>>>>>>>>>>>>>>>>>>>>>>>>>>>>
%<<<       Better ToC       <<<
%>>>>>>>>>>>>>>>>>>>>>>>>>>>>>>
\setlength{\cftbeforesecskip}{0.5ex}

%>>>>>>>>>>>>>>>>>>>>>>>>>>>>>>
%<<<      Hyperref mod      <<<
%>>>>>>>>>>>>>>>>>>>>>>>>>>>>>>

%needs more testing
\newcounter{dummyforrefstepcounter}
\newcommand{\labelRIGHTHERE}[1]
{\refstepcounter{dummyforrefstepcounter}\label{#1}}


%>>>>>>>>>>>>>>>>>>>>>>>>>>>>>>
%<<<  Theorem Environments  <<<
%>>>>>>>>>>>>>>>>>>>>>>>>>>>>>>
\ifx\dontloaddefinitionsoftheoremenvironments\undefined
\theoremstyle{plain}
\newtheorem{thm}{Theorem}[section]
\newtheorem*{thm*}{Theorem}
\newtheorem{lem}[thm]{Lemma}
\newtheorem*{lem*}{Lemma}
\newtheorem{prop}[thm]{Proposition}
\newtheorem*{prop*}{Proposition}
\newtheorem{cor}[thm]{Corollary}
\newtheorem*{cor*}{Corollary}
\newtheorem{defprop}[thm]{Definition-Proposition}
\newtheorem*{punchline}{Punchline}

\theoremstyle{definition}
\newtheorem{defn}{Definition}[section]
\newtheorem*{defn*}{Definition}
\newtheorem{exmp}{Example}[section]
\newtheorem*{exmp*}{Example}
\newtheorem{asspt}{Assumption}[section]
\newtheorem{notation}{Notation}[section]
\newtheorem{exercise}{Exercise}[section]
\newtheorem*{fact*}{Fact}
\newtheorem*{rmk*}{Remark}
\newtheorem{fact}{Fact}
\newtheorem*{aside}{Aside}
\newtheorem*{question}{Question}
\newtheorem*{answer}{Answer}

\else\relax\fi

%>>>>>>>>>>>>>>>>>>>>>>>>>>>>>>
%<<<      Fields, etc.      <<<
%>>>>>>>>>>>>>>>>>>>>>>>>>>>>>>
\DeclareSymbolFont{AMSb}{U}{msb}{m}{n}
\DeclareMathSymbol{\N}{\mathbin}{AMSb}{"4E}
\DeclareMathSymbol{\Octonions}{\mathbin}{AMSb}{"4F}
\DeclareMathSymbol{\Z}{\mathbin}{AMSb}{"5A}
\DeclareMathSymbol{\R}{\mathbin}{AMSb}{"52}
\DeclareMathSymbol{\Q}{\mathbin}{AMSb}{"51}
\DeclareMathSymbol{\PP}{\mathbin}{AMSb}{"50}
\DeclareMathSymbol{\I}{\mathbin}{AMSb}{"49}
\DeclareMathSymbol{\C}{\mathbin}{AMSb}{"43}
\DeclareMathSymbol{\A}{\mathbin}{AMSb}{"41}
\DeclareMathSymbol{\F}{\mathbin}{AMSb}{"46}
\DeclareMathSymbol{\Quaternions}{\mathbin}{AMSb}{"48}


%>>>>>>>>>>>>>>>>>>>>>>>>>>>>>>
%<<<       Operators        <<<
%>>>>>>>>>>>>>>>>>>>>>>>>>>>>>>
\DeclareMathOperator{\ad}{\textbf{ad}}
\DeclareMathOperator{\coker}{coker}
\renewcommand{\ker}{\textup{ker}\,}
\DeclareMathOperator{\End}{End}
\DeclareMathOperator{\Aut}{Aut}
\DeclareMathOperator{\Hom}{Hom}
\DeclareMathOperator{\Maps}{Maps}
\DeclareMathOperator{\Mor}{Mor}
\DeclareMathOperator{\Gal}{Gal}
\DeclareMathOperator{\Ext}{Ext}
\DeclareMathOperator{\Tor}{Tor}
\DeclareMathOperator{\Map}{Map}
\DeclareMathOperator{\Der}{Der}
\DeclareMathOperator{\Rad}{Rad}
\DeclareMathOperator{\rank}{rank}
\DeclareMathOperator{\ArfInvariant}{Arf}
\DeclareMathOperator{\KervaireInvariant}{Ker}
\DeclareMathOperator{\im}{im}
\DeclareMathOperator{\coim}{coim}
\DeclareMathOperator{\trace}{tr}
\DeclareMathOperator{\supp}{supp}
\DeclareMathOperator{\ann}{ann}
\DeclareMathOperator{\spec}{Spec}
\DeclareMathOperator{\proj}{Proj}
\DeclareMathOperator{\fiber}{F}
\DeclareMathOperator{\cofiber}{C}
\DeclareMathOperator{\cone}{cone}
\DeclareMathOperator{\Skel}{Sk}
\DeclareMathOperator{\conn}{conn}
\DeclareMathOperator{\colim}{colim}
\DeclareMathOperator{\limit}{lim}

%>>>>>>>>>>>>>>>>>>>>>>>>>>>>>>
%<<<   Cohomology Theories  <<<
%>>>>>>>>>>>>>>>>>>>>>>>>>>>>>>
\DeclareMathOperator{\KR}{{K\R}}
\DeclareMathOperator{\KO}{{KO}}
\DeclareMathOperator{\K}{{K}}
\DeclareMathOperator{\OmegaO}{{\Omega_{\Octonions}}}

%>>>>>>>>>>>>>>>>>>>>>>>>>>>>>>
%<<<   Algebraic Geometry   <<<
%>>>>>>>>>>>>>>>>>>>>>>>>>>>>>>
\DeclareMathOperator{\Spec}{Spec}
\DeclareMathOperator{\Proj}{Proj}
\DeclareMathOperator{\Sing}{Sing}
\DeclareMathOperator{\shfHom}{\mathscr{H}\textit{\!\!om}}
\newcommand{\WeilDivisors}{\textup{Div}}
\newcommand{\CartierDivisors}{\textup{CaDiv}}
\newcommand{\PrincipalWeilDivisors}{\textup{PrDiv}}
\newcommand{\LocallyPrincipalWeilDivisors}{\textup{LPDiv}}
\newcommand{\PrincipalCartierDivisors}{\textup{PrCaDiv}}
\newcommand{\DivisorClass}{\textup{Cl}}
\newcommand{\CartierClass}{\textup{CaCl}}
\newcommand{\Picard}{\textup{Pic}}
\DeclareMathOperator{\Frob}{Frob}


%>>>>>>>>>>>>>>>>>>>>>>>>>>>>>>
%<<<  Mathematical Objects  <<<
%>>>>>>>>>>>>>>>>>>>>>>>>>>>>>>
\newcommand{\sll}{\mathfrak{sl}}
\newcommand{\gl}{\mathfrak{gl}}
\newcommand{\GL}{\mbox{GL}}
\newcommand{\PGL}{\mbox{PGL}}
\newcommand{\SL}{\mbox{SL}}
\newcommand{\Mat}{\mbox{Mat}}
\newcommand{\Gr}{\textup{Gr}}
\newcommand{\Squ}{\textup{Sq}}
\newcommand{\catSet}{\textit{Sets}}
\newcommand{\RP}{{\R\PP}}
\newcommand{\CP}{{\C\PP}}
\newcommand{\Steen}{\mathscr{A}}
\newcommand{\Orth}{\textup{\textbf{O}}}

%>>>>>>>>>>>>>>>>>>>>>>>>>>>>>>
%<<<  Mathematical Symbols  <<<
%>>>>>>>>>>>>>>>>>>>>>>>>>>>>>>
\newcommand{\DASH}{\textup{---}}
\newcommand{\op}{\textup{op}}
\newcommand{\ob}{\textup{ob}\,}
\newcommand{\ho}{\textup{ho}}
\newcommand{\st}{\textup{st}}
\newcommand{\id}{\textup{id}}
\newcommand{\Bullet}{\ensuremath{\bullet} }

%>>>>>>>>>>>>>>>>>>>>>>>>>>>>>>
%<<<      Some Arrows       <<<
%>>>>>>>>>>>>>>>>>>>>>>>>>>>>>>
\let\shortmapsto\mapsto
\let\mapsto\longmapsto
\newcommand{\mapsfrom}{\,\reflectbox{$\mapsto$}\ }
\newcommand{\bigrightsquig}{\scalebox{2}{\ensuremath{\rightsquigarrow}}}
\newcommand{\bigleftsquig}{\reflectbox{\scalebox{2}{\ensuremath{\rightsquigarrow}}}}

%\newcommand{\cofibration}{\xhookrightarrow{\phantom{\ \,{\sim\!}\ \ }}}
%\newcommand{\fibration}{\xtwoheadrightarrow{\phantom{\sim\!}}}
%\newcommand{\acycliccofibration}{\xhookrightarrow{\ \,{\sim\!}\ \ }}
%\newcommand{\acyclicfibration}{\xtwoheadrightarrow{\sim\!}}
%\newcommand{\leftcofibration}{\xhookleftarrow{\phantom{\ \,{\sim\!}\ \ }}}
%\newcommand{\leftfibration}{\xtwoheadleftarrow{\phantom{\sim\!}}}
%\newcommand{\leftacycliccofibration}{\xhookleftarrow{\ \ {\sim\!}\,\ }}
%\newcommand{\leftacyclicfibration}{\xtwoheadleftarrow{\sim\!}}
%\newcommand{\weakequiv}{\xrightarrow{\ \,\sim\,\ }}
%\newcommand{\leftweakequiv}{\xleftarrow{\ \,\sim\,\ }}

\newcommand{\cofibration}
{\xhookrightarrow{\phantom{\ \,{\raisebox{-.3ex}[0ex][0ex]{\scriptsize$\sim$}\!}\ \ }}}
\newcommand{\fibration}
{\xtwoheadrightarrow{\phantom{\raisebox{-.3ex}[0ex][0ex]{\scriptsize$\sim$}\!}}}
\newcommand{\acycliccofibration}
{\xhookrightarrow{\ \,{\raisebox{-.55ex}[0ex][0ex]{\scriptsize$\sim$}\!}\ \ }}
\newcommand{\acyclicfibration}
{\xtwoheadrightarrow{\raisebox{-.6ex}[0ex][0ex]{\scriptsize$\sim$}\!}}
\newcommand{\leftcofibration}
{\xhookleftarrow{\phantom{\ \,{\raisebox{-.3ex}[0ex][0ex]{\scriptsize$\sim$}\!}\ \ }}}
\newcommand{\leftfibration}
{\xtwoheadleftarrow{\phantom{\raisebox{-.3ex}[0ex][0ex]{\scriptsize$\sim$}\!}}}
\newcommand{\leftacycliccofibration}
{\xhookleftarrow{\ \ {\raisebox{-.55ex}[0ex][0ex]{\scriptsize$\sim$}\!}\,\ }}
\newcommand{\leftacyclicfibration}
{\xtwoheadleftarrow{\raisebox{-.6ex}[0ex][0ex]{\scriptsize$\sim$}\!}}
\newcommand{\weakequiv}
{\xrightarrow{\ \,\raisebox{-.3ex}[0ex][0ex]{\scriptsize$\sim$}\,\ }}
\newcommand{\leftweakequiv}
{\xleftarrow{\ \,\raisebox{-.3ex}[0ex][0ex]{\scriptsize$\sim$}\,\ }}


%>>>>>>>>>>>>>>>>>>>>>>>>>>>>>>
%<<<     Greek Letters      <<<
%>>>>>>>>>>>>>>>>>>>>>>>>>>>>>>
%\newcommand{\oldphi}{\phi}
%\renewcommand{\phi}{\varphi}
\let\oldphi\phi
\let\phi\varphi
\renewcommand{\to}{\longrightarrow}
\newcommand{\eps}{\varepsilon}

%>>>>>>>>>>>>>>>>>>>>>>>>>>>>>>
%<<<  1st-4th & parentheses <<<
%>>>>>>>>>>>>>>>>>>>>>>>>>>>>>>
\newcommand{\first}{^\text{st}}
\newcommand{\second}{^\text{nd}}
\newcommand{\third}{^\text{rd}}
\newcommand{\fourth}{^\text{th}}
\newcommand{\ZEROTH}{$0^\text{th}$ }
\newcommand{\FIRST}{$1^\text{st}$ }
\newcommand{\SECOND}{$2^\text{nd}$ }
\newcommand{\THIRD}{$3^\text{rd}$ }
\newcommand{\FOURTH}{$4^\text{th}$ }
\newcommand{\iTH}{$i^\text{th}$ }
\newcommand{\jTH}{$j^\text{th}$ }
\newcommand{\nTH}{$n^\text{th}$ }

%>>>>>>>>>>>>>>>>>>>>>>>>>>>>>>
%<<<    upright commands    <<<
%>>>>>>>>>>>>>>>>>>>>>>>>>>>>>>
\newcommand{\upcol}{\textup{:}}
\newcommand{\upsemi}{\textup{;}}
\providecommand{\lparen}{\textup{(}}
\providecommand{\rparen}{\textup{)}}
\renewcommand{\lparen}{\textup{(}}
\renewcommand{\rparen}{\textup{)}}
\newcommand{\Iff}{\emph{iff} }

%>>>>>>>>>>>>>>>>>>>>>>>>>>>>>>
%<<<     Environments       <<<
%>>>>>>>>>>>>>>>>>>>>>>>>>>>>>>
\newcommand{\squishlist}
{ %\setlength{\topsep}{100pt} doesn't seem to do anything.
  \setlength{\itemsep}{.5pt}
  \setlength{\parskip}{0pt}
  \setlength{\parsep}{0pt}}
\newenvironment{itemise}{
\begin{list}{\textup{$\rightsquigarrow$}}
   {  \setlength{\topsep}{1mm}
      \setlength{\itemsep}{1pt}
      \setlength{\parskip}{0pt}
      \setlength{\parsep}{0pt}
   }
}{\end{list}\vspace{-.1cm}}
\newcommand{\INDENT}{\textbf{}\phantom{space}}
\renewcommand{\INDENT}{\rule{.7cm}{0cm}}

\newcommand{\itm}[1][$\rightsquigarrow$]{\item[{\makebox[.5cm][c]{\textup{#1}}}]}

\newcommand{\rednote}[1]{{\color{red}#1}\scalebox{.1}{rednote}}
\newcommand{\bluenote}[1]{{\color{blue}#1}\scalebox{.1}{rednote}}
\newcommand{\funcdef}[4]{\begin{align*}
#1&\to #2\\
#3&\mapsto#4
\end{align*}}

%\newcommand{\comment}[1]{}

%>>>>>>>>>>>>>>>>>>>>>>>>>>>>>>
%<<<       Categories       <<<
%>>>>>>>>>>>>>>>>>>>>>>>>>>>>>>
\newcommand{\Ens}{{\mathscr{E}ns}}
\DeclareMathOperator{\Sheaves}{{\mathsf{Shf}}}
\DeclareMathOperator{\Presheaves}{{\mathsf{PreShf}}}
\DeclareMathOperator{\Varieties}{{\mathsf{Var}}}
\DeclareMathOperator{\Schemes}{{\mathsf{Sch}}}
\DeclareMathOperator{\Rings}{{\mathsf{Rings}}}
\DeclareMathOperator{\AbGp}{{\mathsf{AbGp}}}
\DeclareMathOperator{\Modules}{{\mathsf{\!-Mod}}}
\DeclareMathOperator{\QuasiCoherent}{{\mathsf{QCoh}}}
\DeclareMathOperator{\Coherent}{{\mathsf{Coh}}}
\DeclareMathOperator{\GSW}{{\mathcal{SW}^G}}
\DeclareMathOperator{\Burnside}{{\mathsf{Burn}}}
\DeclareMathOperator{\GSet}{{G\mathsf{Set}}}
\DeclareMathOperator{\FinGSet}{{G\mathsf{Set}^\textup{fin}}}
\DeclareMathOperator{\HSet}{{H\mathsf{Set}}}
\DeclareMathOperator{\Cat}{{\mathsf{Cat}}}
\DeclareMathOperator{\Orb}{{\mathsf{Orb}}}
\DeclareMathOperator{\Set}{{\mathsf{Set}}}
\DeclareMathOperator{\sSet}{{\mathsf{sSet}}}
\DeclareMathOperator{\Top}{{\mathsf{Top}}}
\DeclareMathOperator{\GSpectra}{{G-\mathsf{Spectra}}}

%>>>>>>>>>>>>>>>>>>>>>>>>>>>>>>
%<<<     Script Letters     <<<
%>>>>>>>>>>>>>>>>>>>>>>>>>>>>>>
\newcommand{\scrQ}{\mathscr{Q}}
\newcommand{\scrW}{\mathscr{W}}
\newcommand{\scrE}{\mathscr{E}}
\newcommand{\scrR}{\mathscr{R}}
\newcommand{\scrT}{\mathscr{T}}
\newcommand{\scrY}{\mathscr{Y}}
\newcommand{\scrU}{\mathscr{U}}
\newcommand{\scrI}{\mathscr{I}}
\newcommand{\scrO}{\mathscr{O}}
\newcommand{\scrP}{\mathscr{P}}
\newcommand{\scrA}{\mathscr{A}}
\newcommand{\scrS}{\mathscr{S}}
\newcommand{\scrD}{\mathscr{D}}
\newcommand{\scrF}{\mathscr{F}}
\newcommand{\scrG}{\mathscr{G}}
\newcommand{\scrH}{\mathscr{H}}
\newcommand{\scrJ}{\mathscr{J}}
\newcommand{\scrK}{\mathscr{K}}
\newcommand{\scrL}{\mathscr{L}}
\newcommand{\scrZ}{\mathscr{Z}}
\newcommand{\scrX}{\mathscr{X}}
\newcommand{\scrC}{\mathscr{C}}
\newcommand{\scrV}{\mathscr{V}}
\newcommand{\scrB}{\mathscr{B}}
\newcommand{\scrN}{\mathscr{N}}
\newcommand{\scrM}{\mathscr{M}}

%>>>>>>>>>>>>>>>>>>>>>>>>>>>>>>
%<<<     Fractur Letters    <<<
%>>>>>>>>>>>>>>>>>>>>>>>>>>>>>>
\newcommand{\frakQ}{\mathfrak{Q}}
\newcommand{\frakW}{\mathfrak{W}}
\newcommand{\frakE}{\mathfrak{E}}
\newcommand{\frakR}{\mathfrak{R}}
\newcommand{\frakT}{\mathfrak{T}}
\newcommand{\frakY}{\mathfrak{Y}}
\newcommand{\frakU}{\mathfrak{U}}
\newcommand{\frakI}{\mathfrak{I}}
\newcommand{\frakO}{\mathfrak{O}}
\newcommand{\frakP}{\mathfrak{P}}
\newcommand{\frakA}{\mathfrak{A}}
\newcommand{\frakS}{\mathfrak{S}}
\newcommand{\frakD}{\mathfrak{D}}
\newcommand{\frakF}{\mathfrak{F}}
\newcommand{\frakG}{\mathfrak{G}}
\newcommand{\frakH}{\mathfrak{H}}
\newcommand{\frakJ}{\mathfrak{J}}
\newcommand{\frakK}{\mathfrak{K}}
\newcommand{\frakL}{\mathfrak{L}}
\newcommand{\frakZ}{\mathfrak{Z}}
\newcommand{\frakX}{\mathfrak{X}}
\newcommand{\frakC}{\mathfrak{C}}
\newcommand{\frakV}{\mathfrak{V}}
\newcommand{\frakB}{\mathfrak{B}}
\newcommand{\frakN}{\mathfrak{N}}
\newcommand{\frakM}{\mathfrak{M}}

\newcommand{\frakq}{\mathfrak{q}}
\newcommand{\frakw}{\mathfrak{w}}
\newcommand{\frake}{\mathfrak{e}}
\newcommand{\frakr}{\mathfrak{r}}
\newcommand{\frakt}{\mathfrak{t}}
\newcommand{\fraky}{\mathfrak{y}}
\newcommand{\fraku}{\mathfrak{u}}
\newcommand{\fraki}{\mathfrak{i}}
\newcommand{\frako}{\mathfrak{o}}
\newcommand{\frakp}{\mathfrak{p}}
\newcommand{\fraka}{\mathfrak{a}}
\newcommand{\fraks}{\mathfrak{s}}
\newcommand{\frakd}{\mathfrak{d}}
\newcommand{\frakf}{\mathfrak{f}}
\newcommand{\frakg}{\mathfrak{g}}
\newcommand{\frakh}{\mathfrak{h}}
\newcommand{\frakj}{\mathfrak{j}}
\newcommand{\frakk}{\mathfrak{k}}
\newcommand{\frakl}{\mathfrak{l}}
\newcommand{\frakz}{\mathfrak{z}}
\newcommand{\frakx}{\mathfrak{x}}
\newcommand{\frakc}{\mathfrak{c}}
\newcommand{\frakv}{\mathfrak{v}}
\newcommand{\frakb}{\mathfrak{b}}
\newcommand{\frakn}{\mathfrak{n}}
\newcommand{\frakm}{\mathfrak{m}}

%>>>>>>>>>>>>>>>>>>>>>>>>>>>>>>
%<<<  Caligraphic Letters   <<<
%>>>>>>>>>>>>>>>>>>>>>>>>>>>>>>
\newcommand{\calQ}{\mathcal{Q}}
\newcommand{\calW}{\mathcal{W}}
\newcommand{\calE}{\mathcal{E}}
\newcommand{\calR}{\mathcal{R}}
\newcommand{\calT}{\mathcal{T}}
\newcommand{\calY}{\mathcal{Y}}
\newcommand{\calU}{\mathcal{U}}
\newcommand{\calI}{\mathcal{I}}
\newcommand{\calO}{\mathcal{O}}
\newcommand{\calP}{\mathcal{P}}
\newcommand{\calA}{\mathcal{A}}
\newcommand{\calS}{\mathcal{S}}
\newcommand{\calD}{\mathcal{D}}
\newcommand{\calF}{\mathcal{F}}
\newcommand{\calG}{\mathcal{G}}
\newcommand{\calH}{\mathcal{H}}
\newcommand{\calJ}{\mathcal{J}}
\newcommand{\calK}{\mathcal{K}}
\newcommand{\calL}{\mathcal{L}}
\newcommand{\calZ}{\mathcal{Z}}
\newcommand{\calX}{\mathcal{X}}
\newcommand{\calC}{\mathcal{C}}
\newcommand{\calV}{\mathcal{V}}
\newcommand{\calB}{\mathcal{B}}
\newcommand{\calN}{\mathcal{N}}
\newcommand{\calM}{\mathcal{M}}

%>>>>>>>>>>>>>>>>>>>>>>>>>>>>>>
%<<<<<<<<<DEPRECIATED<<<<<<<<<<
%>>>>>>>>>>>>>>>>>>>>>>>>>>>>>>

%%% From Kac's template
% 1-inch margins, from fullpage.sty by H.Partl, Version 2, Dec. 15, 1988.
%\topmargin 0pt
%\advance \topmargin by -\headheight
%\advance \topmargin by -\headsep
%\textheight 9.1in
%\oddsidemargin 0pt
%\evensidemargin \oddsidemargin
%\marginparwidth 0.5in
%\textwidth 6.5in
%
%\parindent 0in
%\parskip 1.5ex
%%\renewcommand{\baselinestretch}{1.25}

%%% From the net
%\newcommand{\pullbackcorner}[1][dr]{\save*!/#1+1.2pc/#1:(1,-1)@^{|-}\restore}
%\newcommand{\pushoutcorner}[1][dr]{\save*!/#1-1.2pc/#1:(-1,1)@^{|-}\restore}









\usepackage{makeidx}

\usepackage{multicol}
\makeatletter
\renewenvironment{theindex}
  {\if@twocolumn
      \@restonecolfalse
   \else
      \@restonecoltrue
   \fi
   \setlength{\columnseprule}{0pt}
   \setlength{\columnsep}{10pt}% I CHANGED THIS from 35pt
   \begin{multicols}{3}[\section*{\indexname}]
   \markboth{\MakeUppercase\indexname}%
            {\MakeUppercase\indexname}%
   \thispagestyle{plain}
   \setlength{\parindent}{0pt}
   \setlength{\parskip}{0pt plus 0.3pt}
   \relax
   \let\item\@idxitem
   \small}% I ADDED THIS
  {\end{multicols}\if@restonecol\onecolumn\else\clearpage\fi}
\makeatother


\makeindex
\newcommand{\INDENT}{\textbf{}\hspace{4mm}}
\newcommand{\Index}[1]{\index{#1}#1}
\newcommand{\indexThm}[1]{\index{Hartshorne!Theorems, etc.!#1}}

\begin{document}
\section*{Hartshorne I.1 --- Affine Varieties}
\index{Hartshorne!I.1: Affine Varieties}
\begin{itemise}
\item A nonempty open subset of an irreducible\index{irreducible} space is
irreducible and dense.\index{dense}
\item The closure of an irreducible subspace is irreducible.
\item For $\fraka\subset A$, $I(Z(\fraka))=\sqrt\fraka$, by the
Nullstellensatz.\index{Nullstellensatz} $Z(I(Y))=\overline Y$.
\item Closed sets correspond to radical\index{radical} ideals. A closed set is
irreducible\index{irreducible} iff it corresponds to a prime\index{prime}
ideals.
\item Affine varieties\index{variety} correspond to finitely generated
$k$-algebras which are domains.
\item In a noetherian\index{noetherian} topological space, closed sets are
uniquely finite unions of irreducible ones.
\item The dimension\index{dimension} of a space is one less than the length of
the longest chain of distinct irreducible closed subsets. This coincides with
the dimension of the coordinate ring.
\item \textbf{Theorem 1.8A:}\indexThm{I.1.8A} For $k$ a field, $B$ a f.g.\
$k$-algebra which is a domain:\\ \Bullet $\dim(B)$ is the \Index{transcendence
degree} of $K(B)/k$. \ \ \Bullet $\text{height}\,\frakp+\dim B/\frakp=\dim B$.
\item \textbf{Theorem 1.11A:}\indexThm{I.1.11A}
(Hauptidealsatz)\index{Hauptidealsatz} Let $A$ be noetherian, and $f\in A$ be
neither zero divisor nor unit. Then if $\scrI$ is the set of prime ideals
containing $f$, any minimal ideal in $\scrI$ in fact has height one.

\INDENT\emph{Interpretation if $A$ is a UFD} --- $Z(f)$ can be written uniquely
as a union of hypersurfaces $Z(f_i)$. Any maximal closed irreducible subset of
$Z(f)$ has codimension one in $\Spec A$ --- the only larger irreducible is all
of $A$.
\item \textbf{Proposition 1.12A:}\indexThm{I.1.12A} A noetherian domain $A$ is a
UFD iff every prime ideal of height one is principal.

\INDENT\emph{Interpretation} --- a noetherian domain $A$ is a UFD iff every
maximal proper closed irreducible subset of $\Spec A$ is a hypersurface, the
zero set of one element.

\item A variety $Y\in \A^n$ has dimension $n-1$ iff it is a
\Index{hypersurface}: $Z(f)$ for some nonconstant irreducible $f$. The same
holds for a projective variety in $\PP^n$ of dimension $n-1$.
\end{itemise}
\section*{Hartshorne I.2 --- Projective Varieties}
\index{Hartshorne!I.2: Projective Varieties}
\begin{itemise}
\item To talk about $\PP^n$, introduce graded ring $S=k[x_0,\ldots,x_n]$, zero
sets of homogeneous polynomials,\index{homogeneous} and of homogeneous ideals.
\item Constructed homeomorphism $U_i\to \A^n$. Note that if $Y$ is a
(quasi)-projective variety, it is covered by the (quasi)-affine varieties $Y\cap
U_i$.
\item Closed subsets of $\PP^n$ correspond to homogeneous radical ideals of $S$
not equal to $S_+$, the irrelevant\index{irrelevant} maximal ideal.
\item Given a variety $Y\subset A^n$, the closure $\overline Y$ of $Y$ in
$\PP^n$ is called the projective closure.\index{projective closure} Its ideal is
generated by $\beta(I(Y))$, where $\beta:A\to S$ maps $x_i$ to $x_i/x_0$.
\item Have $d$-uple embedding\index{d-uple} $\PP^n\to\PP^N$ sending
$[x_0:\cdots:x_n]$ to all the degree $d$ monomails. The image of the 2-uple
embedding $\PP^2\to\PP^5$ is called the Veronese surface\index{Veronese
surface}. The image of the $3$-uple embedding of $\PP^1$ in $\PP^3$ is the
twisted cubic\index{twisted cubic} curve.
\item Have the \Index{Segre embedding} $\PP^r\times\PP^s\to\PP^{rs+r+s}$. The
\Index{quadric surface} in $\PP^3$, defined by $xy=zw$, is the Segre embedding
$\PP^1\times\PP^1\to\PP^3$.
\end{itemise}
\section*{Hartshorne I.3 --- Morphisms}
\index{Hartshorne!I.3: Morphisms}
\begin{itemise}
\item A function $f:Y\to k$ is regular\index{regular function} if it is locally
a quotient of (homogeneous in projective case) polynomials of the same degree
with nonvanishing denominator. A regular function is continuous.
\item A \Index{morphism} of varieties is a continuous map $f:X\to Y$ such that
if $\phi:V\to k$ is a regular function for open $V\subset Y$, then $\phi\circ f$
is regular of $f^{-1}(V)$. That is, $f$ must map the sheaf $\calO_Y$ of
\Index{regular functions} on $Y$ into $\calO_X$ (as subsheaves of the sheaves of
discontinuous functions).
\item For $p\in Y$, have the \Index{local ring} $\calO_{p,Y}$, the stalk of
$\calO_Y$ at $p$. The \Index{function field} is the local ring at the generic
point --- here we should define it to be the direct limit of the $\calO_Y(U)$
over all nonempty open subsets $U$ of $Y$. Elements of the function field are
called \Index{rational functions}. Note that by restriction, we have injections
$\calO_Y(Y)\to\calO_{p,Y}\to K(Y)$.
\item \textbf{Theorem 3.2:}\indexThm{I.3.2} For $Y$ affine,
$\calO_Y(Y)=A(Y)$. $\calO_{p,Y}\cong A(Y)_{\frakm_p}$, and $\dim\calO_{p,y}=\dim
Y$. $p\mapsto\frakm_p$ gives a bijection between points of $Y$ and
\Index{maximal ideals} of $A(Y)$. Finally, $K(Y)\cong A(Y)_{(0)}$, so that
$K(Y)$ is a finitely generated \index{field extension}extension of $k$, of
\Index{transcendence degree} $\dim Y$.
\item \textbf{Theorem 3.4:}\indexThm{I.3.4}
For $Y$ projective, $\calO_Y(Y)=k$, $\calO_{p,Y}=S(Y)_{(\frakm_p)}$, and
$K(Y)=S(Y)_{((0))}$. Here we are using \Index{degree zero localization}s.
\item \textbf{Proposition 3.5:}\indexThm{I.3.5}
There is a natural isomorphism, for varieties $X$ and affine varieties $Y$,
$\Hom(X,Y)\overset{\sim}{\to}\Hom(A(Y),\calO(X))$. In fact, if $\calV$ is the
category of varieties over $k$, and $\calD$ is the category of finitely
generated integral domains over $k$, there is a contravariant \Index{adjunction}
$\calO:\calV\longleftrightarrow\calD^\op:\Spec$, which induces an isomorphism of
categories between $\calD^\op$ and the full subcategory of $\calV$ consisting of
affine varieties.
\end{itemise}
\section*{Hartshorne I.4 --- Rational Maps}
\index{Hartshorne!I.4: Rational Maps}
\begin{itemise}
\item Morphisms which agree on a nonempty open subset of a variety are equal.
\item A \Index{rational map} $X\to Y$ is an equivalence class of morphisms
defined on a nonempty open subset of $X$. A rational map is \Index{dominant} if
its image is dense (i.e.\ for some (equivalently any) choice of open subset). A
\Index{birational} map is a rational map with an inverse as rational maps. A
variety is simply \Index{rational} if it is birational to a projective space.
\item The complement of a hypersurface $Z(f)\subset \A^n$ is a hypersurface
$Z(x_{n+1}f)\subset \A^{n+1}$ with coordinate ring $k[x_1,\ldots,x_n]_f$.
\item A variety has a \Index{base} for its topology consisting of open affine
subsets.
\item \textbf{Theorem 4.4:}\indexThm{I.4.4}
A dominant rational map induces a map of function fields, and the resulting
correspondence is bijective, giving an contravariant equivalence of categories
from ``varieties with DRMs'' to ``finitely generated field extensions of
$k$''.\item \textbf{Corollary 4.5:}\indexThm{I.4.5}
Varieties are birational \Iff they have isomorphic open subsets \Iff they have
isomorphic function fields.
\item The blowup is defined --- to blow up $0\in Y\subset A^n$, use composite
$\phi:X\to\A^n\times\PP^n\to\A^n$, where $X$ is defined by equations
$x_iy_j=x_jy_i$. Take $\widetilde Y=\overline{\phi^{-1}(Y-0)})$.
\item Rational fuunctions and more generally rational maps have a maximum
\Index{domain of definition}.
\item If $p\in X$ and $q\in Y$ are such that $\calO_{p,X}$ and $\calO_{q,Y}$ are
isomorphic $k$-algebras, then there is a birational map $X\to Y$ taking $p$ to
$q$.
\item If $X\subset\PP^n$ is a projective variety with $\dim X<n-1$, then for
some $p\notin X$, projection from $p$ induces a birational morphism from $X$ to
it's image. Repeating this process, we end up with a \Index{hypersurface} in
$A^{\dim X+1}$ birational to $X$.
\end{itemise}
\section*{Hartshorne I.5 --- Nonsingular Varieties}
\index{Hartshorne!I.5: Nonsingular Varieties}
\begin{itemise}
\item If $Y\in\A^n$ is an affine variety with ideal generated by
$f_1,\ldots,f_t$, then $Y$ is \Index{nonsingular} at $p\in Y$ if the rank of the
\Index{Jacobian} $\|(\partial f_i/\partial x_i)(p)\|$ is $n-\dim Y$. (It may be
lower).
\item If $A$ is a noetherian local ring, then say $A$ is a \Index{regular local
ring} if $\dim_{A/\frakm}\frakm/\frakm^2=\dim A$. Note that $\geq$ holds always.
\item \textbf{Theorem 5.1:}\indexThm{I.5.1}
$Y\in\A^n$ is nonsingular at $p$ \Iff $\calO_{p,Y}$ is a regular local ring.
Thus the notion of nonsigularity is intrinsic and extends to all varieties.
\item \textbf{Theorem 5.3:}\indexThm{I.5.3}
For any variety $Y$, the set $\Sing Y$ of singular points is a proper closed
subset.
\item To study very local behaviour, we take \Index{completion}s. The completion
of a local ring $(A,\frakm)$ is $\hat A:=\varprojlim
A/\frakm^n=\varprojlim\{\cdots\rightarrow A/\frakm^3\rightarrow
A/\frakm^2\rightarrow A/ \frakm\}$. 
\item \textbf{Theorem 5.4A:}\indexThm{I.5.4A} Suppose $A$ is noetherian. $\hat
A$ is local with maximal ideal $\frakm A$, and $A\to\hat A$ is injective. For
finitely generated $A$-modules $M$, $M\otimes_A\hat A$ is the $\frakm$-adic
completion of $M$. $\dim \hat A=\dim A$ and $\hat A$ is regular \Iff $A$ is
regular.
\item $p\in X$ and $q\in Y$ are \Index{analytically isomorphic} if
$\hat\calO_{p,X}\cong\hat\calO_{q,Y}$ as $k$-algebras.
\item \textbf{Theorem 5.5A:}\indexThm{I.5.5A}(\Index{Cohen Structure Theorem}) 
The only complete regular local ring of dimension $n$ containing a field $k$ is
the power series ring in $n$ variables over $k$. In particular, any two
nonsingular points of the same dimension are analytically isomorphic.
\item \textbf{Theorem 5.7A:}\indexThm{I.5.7A}(\Index{elimination theory}) 
Given homogeoneous polynomials $f_1,\ldots,f_r\in k[x_0,\ldots,x_n]$ with
indeterminate coefficients $a_{ij}$, we want to know when they have a common
root other than $(0,0,\ldots,0)$. There a polynomials
$g_1,\ldots,g_t\in\Z[a_{ij}]$, homogeneous in the coefficients of each $f_i$
separately, such that there is a common nonzero root for the $f_i$ \Iff there is
a common root for the $g_t$.
\end{itemise}
\section*{Hartshorne I.6 --- Nonsingular Curves}
\index{Hartshorne!I.6: Nonsingular Curves}
\begin{itemise}
\item There is a unique nonsingular projective curve in each birational
equivalence class of curves. For each finitely generated field extension $K/k$
of transcendence degree one, there is then a unique nonsingular projective curve
$C_K$ with function field $K$.
\item Moreover, and homomorphism $K_2\to K_1$ over $k$ is represented by a
morphism $C_{K_1}\to C_{K_2}$.
\item Let $K$ be a field and $G$ be a totally ordered abelian group. A
\Index{valuation} on $K$ with values in $G$ is homomorphism $v:K^\times\to G$
such that for all nonzero $x,y$: $v(x+y)\geq\min\{v(x),v(y)\}$. The
\Index{valuation ring} of $v$ is then $\{0\}\cup v^{-1}({\geq0})$, a local ring
whose units are $\ker v$. The valuation ring always has quotient
field $K$. Call $v$ a valuation of $K/k$ if $v(k^\times)=\{0\}$.
\item If $A,B$ are local rings in a field $K$, then $B$ dominates $A$ if
$A\subseteq B$ and $\frakm_B\cap A=\frakm_A$.
\item \textbf{Theorem 6.1A:}\indexThm{I.6.1A}
A local ring $R$ contained in a field $K$ is a valuation ring of $K$ \Iff it is
a maximal local ring w.r.t.\ dominanation. Every local ring inside $K$ is
dominated by some valuation ring of $K$.
\item \textbf{Theorem 6.2A:}\indexThm{I.6.2A}
Let $A$ be a noetherian local domain of dimension one. TFAE: 
\begin{enumerate}\squishlist
\item $A$ is a discrete valuation ring (i.e.\ comes from a valuation with values
in $\Z$);
\item $A$ is integrally closed (i.e.\ normal);
\item $A$ is a regular local ring (implied by 2, even in higher dimensions);
\item The maximal ideal of $A$ is principal.
\end{enumerate}
\item A \Index{Dedekind domain} is an integrally closed noetherian domain of
dimension one. As integral closure\index{Integrally closed} is a local property,
every localisation of a Dedekind domain is a DVR.
\item For $K/k$ a finitely generated extension of transcendence degree one, let
$C_K$\index{abstract nonsingular curve} be the set of all discrete valuation
rings of $K/k$.
\item Given a point $y$ on a nonsingular curve $Y$ with function field $K$,
there's an \emph{injective} function $Y\to C_K$ given by $p\mapsto\calO_{p}\in
C_K$.
\item \textbf{Theorem 6.5:}\indexThm{I.6.5}
For any $x\in K$, all but finitely map $R\in C_K$ contain $x$.

\INDENT\emph{Interpretation} --- $x\notin R$ means that $x$ will have a pole at
$R$ (as an element of the `curve' $C_K$). As $K$ is one dimensional, each $x$
has only finitely many poles.
\item \textbf{Corollary 6.6:}\indexThm{I.6.6}
Any DVR of $K/k$ is the local ring of a point on a nonsingular affine curve.
\item Make $C_K$ a topological space using the cofinite topology. If $U\subseteq
C_K$ is open, define $\calO(U):=\cap_{P\in U}R_P$. An element $f\in\calO(U)$
defines a function $U\to k$ by taking $P$ to the image of $f$ in the residue
field of $\calO_P$, which must be $k$ by 6.6.
\item $f\in\calO(U)$ can be recovered from its function $U\to k$, as if $0\neq
f\mapsto 0$, $f\in\frakm_p$ for all $p$, so $f^{-1}\notin R_p$ for all $p$,
contradicting 6.5, as there must be infinitely many $R_p$ by 6.6. By 6.5, any
$f\in K$ is a regular function on some open $U$. Thus the quotient field of
$C_K$ is $K$. 
\item An \Index{abstract nonsingular curve} is an open subset of $C_K$ with the
induced topology and sheaf of regular functions. A morphism of such is a
continuous map such that regular functions pull back to regular functions.
\item \textbf{Proposition 6.7:}\indexThm{I.6.7} Every nonsigular
quasi-projective curve is isomorphic to an abstract nonsingular curve.
\item \textbf{Theorem 6.9:}\indexThm{I.6.9} $C_K$ is isomorphic to a nonsingular
projective curve.
\item \textbf{Corollary 6.10:}\indexThm{I.6.10} Every abstract nonsingular curve
is isomorphic to a quasi-projective curve. Every nonsingular quasi-projective
curve is isomorphic to an open subset of a nonsingular projective curve.
\item \textbf{Corollary 6.11:}\indexThm{I.6.11} Every curve is birationally
equivalent to a nonsingular projective curve.
\item \textbf{Corollary 6.12:}\indexThm{I.6.12} The following categories are
equivalent:
\begin{enumerate}\squishlist
\item[(i)] Nonsingular projective curves and \emph{dominant morphisms};
\item[(ii)] Quasi-projective curves and \emph{dominant rational maps};
\item[(iii)] function fields of dimension 1 over $k$ and $k$-homomorphisms.
\end{enumerate}
\item \textbf{Proposition 6.8:}\indexThm{I.6.8} Let $X$ be a nonsingular curve,
$p\in X$, let $Y$ be a projective variety, and let $\phi:X-\{p\}\to Y$ be a
morphism. Then there is a unique extension of $\phi$ to a morphism $X\to Y$.
{\small (This fails if $Y$ is not projective, or if $\dim X>1$.)}
\item Using proposition 6.8, one can show that every automorphism of $\PP^1$ is
in $\PGL(1)$.
\end{itemise}
\section*{Hartshorne I.7 --- Intersections in Projective Space}
\index{Hartshorne!I.7: Intersections in Projective Space}
\section*{Hartshorne II.1 --- Sheaves}
\index{Hartshorne!II.1: Sheaves}
\begin{itemise}
\item A \Index{presheaf} on $X$ is a contravariant functor from the category of
open subsets of $X$ to abelian groups. There is some contention (Bjorn) as to
whether to require $\emptyset\mapsto0$. It's a sheaf if:
\begin{enumerate}\squishlist
\item[(3)] If $U$ is open with open cover $\{V_i\}$, and $s\in\scrF(U)$ has $s|_{V_i}=0$ for all $i$, then $s=0$.
\item[(4)] If $s_i\in\scrF(V_i)$ agree on overlaps, then they glue (uniquely).
\end{enumerate}
%\item A presheaf is a \Index{sheaf} if ``being zero can be determined using an
%open cover'' and if ``gluing sections is possible''.
\item A morphism of sheaves is an isomorphism iff it is an isomorphism on the
stalks.
\item The presheaf kernel, cokernel and image are exactly as one would hope.
\item Given a presheaf $\scrF$ there is a universal morphism $\scrF\to\scrF^+$
into a sheaf $\scrF^+$ called the \Index{sheafification}. $\scrF$ and $\scrF^+$
have the same stalks. The sections of $\scrF^+(U)$ can be viewed as sections of
$\cup_{p\in U} \scrF_p\downarrow U$ which are locally induced by sections of
$\scrF$. If $\scrF$ is a sub-presheaf of a sheaf $\scrG$, then
its sheafification is the intersection of all subsheaves of $\scrG$ containing
$\scrF$. Any presheaf with (3) embeds in its sheaf of discontinuous sections,
explaining the initial definition.
\item Given a morphism of sheaves $\scrF\to\scrG$, the kernel is a sheaf. The
image need not be a sheaf, so we define the \Index{sheaf image} to be the
smallest subsheaf of $\scrG$ containing the presheaf image of $\scrF$. We are
thus already equipped to discuss exact sequences of sheaves, by demanding that
kernel equals \emph{sheaf} image.
\item If $\scrF'\subset\scrF$ is an inclusion of sheaves, the \Index{quotient sheaf} $\scrF/\scrF'$ is defined to be the sheafification of the quotient presheaf. The quotient presheaf has (3) but not (4), as if $[f_i]\in\scrF(V_i)/\scrF'(V_i)$ agree on $V_i\cap V_j$, this only means that they $f_i-f_j\in \scrF'(V_i\cap V_j)$, and this element may not extend to $\scrF'(V_i\cup V_j)$.

\INDENT Sections of $(\scrF/\scrF')(U)$ are equivalence classes of collections of sections of the presheaf on coverings of $U$ which agree on overlaps.
\item A sequence of sheaves is \Index{exact} \Iff it is exact on the stalks.
\end{itemise}
\printindex
\end{document}
















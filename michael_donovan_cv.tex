% !TEX root = z_output/_michael_donovan_cv.tex

%______________________________________________________________________________________________________________________
% @brief    LaTeX2e Resume for Kamil K Wojcicki
\documentclass[margin,line]{resume}
\usepackage{mathrsfs}
\usepackage{amsmath, amsthm, amssymb, graphicx}
\usepackage[bookmarks=false,pdftex,pdfborder={0 0 0 [1 1]}]{hyperref}
\usepackage{cancel}

\headheight=-2pt
\topmargin=8pt
\topmargin=2pt
\textheight=634pt
\textheight=654pt
%\textwidth=440pt
%\oddsidemargin=-12pt
%\evensidemargin=8pt

%______________________________________________________________________________________________________________________


\newcommand{\CVsection}[1]{\section{\mysidestyle #1}}
\newcommand{\entry}[3]{\textbf{#1} #2 \hfill {#3}
           
\vspace{-2.7mm}}
\newcommand{\twolineentry}[4]{\textbf{#1} #2 \hfill {#4}\\%
#3
           
\vspace{-2.7mm}}



\newcommand{\FINALentry}[3]{\textbf{#1} #2 \hfill {#3}}
\newcommand{\FINALtwolineentry}[4]{\textbf{#1} #2 \hfill {#4}\\%
#3}



%- change the font on your email & web address to something less nerdy, it's best to use just one font for everything & use one size except for the header
%- there's a lot of bold text. Maybe the years don't have to be bold? Maybe something else can go. It's hard on the eyes if everything is bold
%- only show programming experience if you think it's plausible the employer would need it. For a business application, you might do a section on skills and include 1 line on programming amongst a couple of fluffier things like mentoring, facilitation etc
%- if you need more space, perhaps you don't need to list where the AMSI summer schools were, that could save 3 lines
%- a lot of Aus cvs have one line on recreation at the end. It goes something like 'in my free time I enjoy soccer, squash, baking fresh bread, socialising with uni club XYZ and travel' or similar. I don't know if that's the go in the US but we sometimes like to see it on cvs that are otherwise pretty academic only 


\begin{document}
\name{\Large Michael Donovan}
\begin{resume}

\CVsection{Contact\\Information}

75 Pleasant St, Apt \#2 \hfill {mdono@math.mit.edu}\\
Cambridge, MA  02139 \hfill {math.mit.edu/~mdono/}\\
USA \hfill +1 857 891 2805


%\CVsection{Research\\Interests}

% Speech processing, speech enhancement...\\ 
% machine learning and pattern recognition.


\CVsection{Education}

\twolineentry{Ph.D.\ in Mathematics}{Massachusetts Institute of Technology, USA}{%
\phantom{space}$-$ Advisor: Prof. Haynes Miller\\
\phantom{space}$-$ Thesis topic: The Adams spectral sequence in the homotopy theory of simplicial algebras%
}{May 2015 (expected)}
\twolineentry{B.Sc in Pure Mathematics}{University of Sydney, Australia}{%
\phantom{space}$-$ with First Class Honours and a University Medal\\
\phantom{space}$-$ Advisor: Prof. Gustav Lehrer\\
\phantom{space}$-$ Thesis title: \emph{The Cellularity of Certain Hecke Quotients}
}{2010}
\FINALentry{B.Com (Actuarial Studies) with B.Sc (Mathematics)}{Macquarie University, Australia}{2009}


\CVsection{Preprints in preparation}

\entry{\hspace{-.275em}}{\emph{On the Adams spectral sequence for commutative algebras}}{}
\FINALentry{\hspace{-.275em}}{\emph{Operations in the spectral sequence of a mixed simplicial $\mathbb{F}_2$-algebra}}{}


\CVsection{Work in \\progress}

\entry{\hspace{-.275em}}{\emph{The homology of unstable Lie coalgebras}}{}
\FINALentry{\hspace{-.275em}}{\emph{The Adams spectral sequence for a simplicial commutative algebra
sphere}}{}


\CVsection{Talks}

\twolineentry{Johns Hopkins Topology Seminar}{}{\emph{Calculating the Adams spectral sequence for a simplicial algebra sphere}}{9/15/2014}
\twolineentry{Union College Mathematics Conference}{}{\emph{The Adams spectral sequence for simplicial algebras}}{10/20/2013}
\twolineentry{MIT Juvitop topology seminar}{}{\emph{The construction of the EHP sequence}}{9/18/2013}
\twolineentry{MIT Babytop topology seminar}{}{\emph{The EHP spectral sequence}}{8/10/2012}
\FINALtwolineentry{MIT Pure Mathematics Graduate Student Seminar}{}{\emph{The Robinson-Schensted correspondence and the Kazhdan-Lusztig Basis}}{2/25/2011}


\CVsection{Conferences \\ \& Workshops}

\entry{Algebraic Topology semester long program}{MSRI, Berkeley}{2014}
\entry{MIT Talbot Workshop --- Calculus of Functors}{Garden City, Utah}{2012}
\entry{Young Topologists Meeting}{University of Copenhagen, Denmark}{2012}
\entry{Graduate Student Topology Conference}{Indiana University}{2012}
\entry{Union College Mathematics Conference}{Algebraic Topology session}{2011,2013}
\FINALentry{AMSI/ICE-EM Mathematics Summer Schools}{}%{%
%\phantom{space}$-$ University of Sydney (2007)\\
%\phantom{space}$-$ Monash University, Melbourne (2008)\\
%\phantom{space}$-$ University of Wollongong (2009)}
{2007-2009}


\CVsection{Awards and\\Scholarships} 

\entry{Charles and Holly Housman Award for Excellence in Teaching}{(MIT)}{2014}
\entry{Vice-Chancellor's Research Scholarship}{(declined, USyd)}{2010}
\entry{University of Sydney Medal}{}{2009}
\entry{University of Sydney Honours Scholarship}{}{2009}
\entry{MJ and M Ashby Prize}{for Mathematics in Science (USyd)}{2009}
\entry{Barker Prize}{for Mathematics Honours (USyd)}{2009}
\twolineentry{Alf and Pearl Pollard Memorial Prize}{for the most outstanding undergraduate}{student in the division of Economic and Financial Studies (MqU)}{2008}
\entry{Doris Wallent Scholarship}{for 300 level mathematics, division of ICS (MqU)}{2008}
\entry{Frederic Chong Mathematics Prize}{for proficiency in 300 level math (MqU)}{2008}
\entry{Alan McIntosh Analysis Prize}{for proficiency in 200 and 300 level analysis (MqU)}{2008}
\entry{Hubert-Vaughn Prize}{for proficiency in ACST354: Survival Models (MqU)}{2007}
\entry{2006-2007 Vacation Scholarship}{at USyd (AMSI/ICE-EM)}{2006}
\entry{SUMS problem competition}{prizewinner (USyd)}{2006-2008}
\entry{Frederic Chong Mathematics Prize}{for proficiency in 200 level math (MqU)}{2006}
\entry{Merit Certificate}{Division of EFS (MqU)}{2005-2008}
\FINALentry{Assoc Prize}{for proficiency in ACST211: Combinatorial Probability (MqU)}{2005}


\CVsection{Teaching and Work Experience}

\twolineentry{Recitation Instructor}{Department of Mathematics (MIT)}{\phantom{space}$-$ 18.03: Ordinary Differential Equations}{2012-present}
\twolineentry{Grader}{Department of Mathematics (MIT)}{%
\phantom{space}$-$ 18.725: Algebraic Geometry I\\
%\phantom{space}$-$ 18.726: Algebraic Geometry II\\
\phantom{space}$-$ 18.906: Algebraic Topology II%
}{2011-2012}
\twolineentry{Recitation Instructor}{Department of Mathematics (USyd)}{%
\phantom{space}$-$ MATH1903: Integral Calculus and Modelling (Adv)\\
\phantom{space}$-$ MATH1002: Linear Algebra\\
\phantom{space}$-$ MATH1014: Introduction to Linear Algebra%
}{2009}
\twolineentry{Casual Academic Staff}{Department of Actuarial Studies (MqU)}{%
\phantom{space}$-$ exam grading and preparation of teaching material\\
\phantom{space}$-$ PR representative for the department of EFS%
}{2008-2009}
\FINALtwolineentry{Recitation Instructor}{Department of Actuarial Studies (MqU)}{%
\phantom{space}$-$ ACST211: Combinatorial Probability (2006-2008)\\
\phantom{space}$-$ ACST200: Mathematics of Finance (2007-2008)\\
\phantom{space}$-$ ACST255: Contingent Payments I (2008)\\
\phantom{space}$-$ ACST355: Contingent Payments II (2008)\\
\phantom{space}$-$ ACST356: Mathematical Theory of Risk (2008)%
}{2006-2008}


\CVsection{Undergraduate \\ Research}

\entry{The cellularity of certain Hecke quotients}{Supervisor: G.\ I.\ Lehrer, (USyd)}{2009}
\FINALentry{The permuation degrees of finite groups}{Supervisor: A.\ Henderson, (USyd)}{2006-2007}


\CVsection{Service}

\entry{MIT Algebraic Topology Seminar}{Organiser}{2014}
\entry{Directed Reading Program}{Mentor (MIT)}{2011,2013}
\entry{MIT Pure Math Graduate Student Seminar}{Organiser}{2011-2012}
\entry{MIT Math Department Retreat}{Planning Committee Member}{2012-2013}
\FINALentry{Macquarie University Soccer Club}{Team Manager \& Club Secretary}{2008}


\CVsection{Programming}

\entry{Languages:}{\texttt{Maple} (adv.), \LaTeX\ (adv.), \texttt{C++} (int.).}{}
\entry{COMP225.5 Macquarie University Programming competition}{Runner Up}{2005}
\FINALentry{ACM South Pacific Programming Contest}{Contestant}{2005}


\CVsection{Recreation}

In my free time I enjoy playing soccer, squash and street hockey, cycling, baking fresh bread and cooking with friends.

%\CVsection{Referees} 
%    {\sl Available on request.}

%\CVsection{References} 
%
%\textbf{Scott Sheffield}, Professor of Mathematics, Department of Mathematics, MIT. \vspace{2mm}\\%
%\textsl{e-mail:} \texttt{sheffield@math.mit.edu} \vspace{1mm}\\%
%\textsl{phone:} +1 617 253 4350
%
%\textbf{Vojkan Jaksic}, Professor of Mathematics, Department of Mathematics and Statistics, McGill University. \vspace{2mm}\\%
%\textsl{e-mail:} \texttt{jaksic@math.mcgill.ca} \vspace{1mm}\\%
%\textsl{phone:} +1 514 398 3827

% \begin{tabular}{@{}p{6cm}p{6cm}}
% \textbf{Professor Kuldip Paliwal}       &  \textbf{Dr Stephen So}                   \\
% Professor                               &  Associate Lecturer                       \\
% Griffith University                     &  Griffith University                      \\
% Nathan, Queensland, Australia           &  Gold Coast, Queensland, Australia        \\
% phone: \textsl{available on request}    &  phone: \textsl{available on request}     \\
% e-mail: \textsl{available on request}   &  e-mail: \textsl{available on request}    \\
% \end{tabular}

% \begin{tabular}{@{}p{6cm}p{6cm}}
% \textbf{Dr Conrad Sanderson}            &  \textbf{Mr Sean Loye}                    \\
% Researcher                              &  Systems Engineer                         \\
% National ICT Australia                  &  Hewlett Packard                          \\
% St Lucia, Queensland, Australia         &  Milton, Queensland, Australia            \\
% phone: \textsl{available on request}    &  phone: \textsl{available on request}     \\
% e-mail: \textsl{available on request}   &  e-mail: \textsl{available on request}    \\
% \end{tabular}



%______________________________________________________________________________________________________________________
\end{resume}
\end{document}


%______________________________________________________________________________________________________________________
% EOF


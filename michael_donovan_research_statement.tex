% !TEX root = z_output/michael_donovan_research_statement.tex
\documentclass[11pt]{article}
\usepackage{fullpage}
\usepackage{amsmath,amsthm,amssymb}
\usepackage{mathrsfs,nicefrac}
\usepackage{amssymb}
\usepackage{epsfig}
\usepackage[all,2cell]{xy}
\usepackage{sseq}
\usepackage{tocloft}
\usepackage{cancel}
\usepackage[strict]{changepage}
\usepackage{color}
\usepackage{tikz}
\usepackage{extpfeil}
\usepackage{version}
\usepackage{framed}
\definecolor{shadecolor}{rgb}{.925,0.925,0.925}

%\usepackage{ifthen}
%Used for disabling hyperref
\ifx\dontloadhyperref\undefined
%\usepackage[pdftex,pdfborder={0 0 0 [1 1]}]{hyperref}
\usepackage[pdftex,pdfborder={0 0 .5 [1 1]}]{hyperref}
\else
\providecommand{\texorpdfstring}[2]{#1}
\fi
%>>>>>>>>>>>>>>>>>>>>>>>>>>>>>>
%<<<        Versions        <<<
%>>>>>>>>>>>>>>>>>>>>>>>>>>>>>>
%Add in the following line to include all the versions.
%\def\excludeversion#1{\includeversion{#1}}

%>>>>>>>>>>>>>>>>>>>>>>>>>>>>>>
%<<<       Better ToC       <<<
%>>>>>>>>>>>>>>>>>>>>>>>>>>>>>>
\setlength{\cftbeforesecskip}{0.5ex}

%>>>>>>>>>>>>>>>>>>>>>>>>>>>>>>
%<<<      Hyperref mod      <<<
%>>>>>>>>>>>>>>>>>>>>>>>>>>>>>>

%needs more testing
\newcounter{dummyforrefstepcounter}
\newcommand{\labelRIGHTHERE}[1]
{\refstepcounter{dummyforrefstepcounter}\label{#1}}


%>>>>>>>>>>>>>>>>>>>>>>>>>>>>>>
%<<<  Theorem Environments  <<<
%>>>>>>>>>>>>>>>>>>>>>>>>>>>>>>
\ifx\dontloaddefinitionsoftheoremenvironments\undefined
\theoremstyle{plain}
\newtheorem{thm}{Theorem}[section]
\newtheorem*{thm*}{Theorem}
\newtheorem{lem}[thm]{Lemma}
\newtheorem*{lem*}{Lemma}
\newtheorem{prop}[thm]{Proposition}
\newtheorem*{prop*}{Proposition}
\newtheorem{cor}[thm]{Corollary}
\newtheorem*{cor*}{Corollary}
\newtheorem{defprop}[thm]{Definition-Proposition}
\newtheorem*{punchline}{Punchline}
\newtheorem*{conjecture}{Conjecture}
\newtheorem*{claim}{Claim}

\theoremstyle{definition}
\newtheorem{defn}{Definition}[section]
\newtheorem*{defn*}{Definition}
\newtheorem{exmp}{Example}[section]
\newtheorem*{exmp*}{Example}
\newtheorem*{exmps*}{Examples}
\newtheorem*{nonexmp*}{Non-example}
\newtheorem{asspt}{Assumption}[section]
\newtheorem{notation}{Notation}[section]
\newtheorem{exercise}{Exercise}[section]
\newtheorem*{fact*}{Fact}
\newtheorem*{rmk*}{Remark}
\newtheorem{fact}{Fact}
\newtheorem*{aside}{Aside}
\newtheorem*{question}{Question}
\newtheorem*{answer}{Answer}

\else\relax\fi

%>>>>>>>>>>>>>>>>>>>>>>>>>>>>>>
%<<<      Fields, etc.      <<<
%>>>>>>>>>>>>>>>>>>>>>>>>>>>>>>
\DeclareSymbolFont{AMSb}{U}{msb}{m}{n}
\DeclareMathSymbol{\N}{\mathbin}{AMSb}{"4E}
\DeclareMathSymbol{\Octonions}{\mathbin}{AMSb}{"4F}
\DeclareMathSymbol{\Z}{\mathbin}{AMSb}{"5A}
\DeclareMathSymbol{\R}{\mathbin}{AMSb}{"52}
\DeclareMathSymbol{\Q}{\mathbin}{AMSb}{"51}
\DeclareMathSymbol{\PP}{\mathbin}{AMSb}{"50}
\DeclareMathSymbol{\I}{\mathbin}{AMSb}{"49}
\DeclareMathSymbol{\C}{\mathbin}{AMSb}{"43}
\DeclareMathSymbol{\A}{\mathbin}{AMSb}{"41}
\DeclareMathSymbol{\F}{\mathbin}{AMSb}{"46}
\DeclareMathSymbol{\G}{\mathbin}{AMSb}{"47}
\DeclareMathSymbol{\Quaternions}{\mathbin}{AMSb}{"48}


%>>>>>>>>>>>>>>>>>>>>>>>>>>>>>>
%<<<       Operators        <<<
%>>>>>>>>>>>>>>>>>>>>>>>>>>>>>>
\DeclareMathOperator{\ad}{\textbf{ad}}
\DeclareMathOperator{\coker}{coker}
\renewcommand{\ker}{\textup{ker}\,}
\DeclareMathOperator{\End}{End}
\DeclareMathOperator{\Aut}{Aut}
\DeclareMathOperator{\Hom}{Hom}
\DeclareMathOperator{\Maps}{Maps}
\DeclareMathOperator{\Mor}{Mor}
\DeclareMathOperator{\Gal}{Gal}
\DeclareMathOperator{\Ext}{Ext}
\DeclareMathOperator{\Tor}{Tor}
\DeclareMathOperator{\Map}{Map}
\DeclareMathOperator{\Der}{Der}
\DeclareMathOperator{\Rad}{Rad}
\DeclareMathOperator{\rank}{rank}
\DeclareMathOperator{\ArfInvariant}{Arf}
\DeclareMathOperator{\KervaireInvariant}{Ker}
\DeclareMathOperator{\im}{im}
\DeclareMathOperator{\coim}{coim}
\DeclareMathOperator{\trace}{tr}
\DeclareMathOperator{\supp}{supp}
\DeclareMathOperator{\ann}{ann}
\DeclareMathOperator{\spec}{Spec}
\DeclareMathOperator{\SPEC}{\textbf{Spec}}
\DeclareMathOperator{\proj}{Proj}
\DeclareMathOperator{\PROJ}{\textbf{Proj}}
\DeclareMathOperator{\fiber}{F}
\DeclareMathOperator{\cofiber}{C}
\DeclareMathOperator{\cone}{cone}
\DeclareMathOperator{\skel}{sk}
\DeclareMathOperator{\coskel}{cosk}
\DeclareMathOperator{\conn}{conn}
\DeclareMathOperator{\colim}{colim}
\DeclareMathOperator{\limit}{lim}
\DeclareMathOperator{\ch}{ch}
\DeclareMathOperator{\Vect}{Vect}
\DeclareMathOperator{\GrthGrp}{GrthGp}
\DeclareMathOperator{\Sym}{Sym}
\DeclareMathOperator{\Prob}{\mathbb{P}}
\DeclareMathOperator{\Exp}{\mathbb{E}}
\DeclareMathOperator{\GeomMean}{\mathbb{G}}
\DeclareMathOperator{\Var}{Var}
\DeclareMathOperator{\Cov}{Cov}
\DeclareMathOperator{\Sp}{Sp}
\DeclareMathOperator{\Seq}{Seq}
\DeclareMathOperator{\Cyl}{Cyl}
\DeclareMathOperator{\Ev}{Ev}
\DeclareMathOperator{\sh}{sh}
\DeclareMathOperator{\intHom}{\underline{Hom}}
\DeclareMathOperator{\Frac}{frac}



%>>>>>>>>>>>>>>>>>>>>>>>>>>>>>>
%<<<   Cohomology Theories  <<<
%>>>>>>>>>>>>>>>>>>>>>>>>>>>>>>
\DeclareMathOperator{\KR}{{K\R}}
\DeclareMathOperator{\KO}{{KO}}
\DeclareMathOperator{\K}{{K}}
\DeclareMathOperator{\OmegaO}{{\Omega_{\Octonions}}}

%>>>>>>>>>>>>>>>>>>>>>>>>>>>>>>
%<<<   Algebraic Geometry   <<<
%>>>>>>>>>>>>>>>>>>>>>>>>>>>>>>
\DeclareMathOperator{\Spec}{Spec}
\DeclareMathOperator{\Proj}{Proj}
\DeclareMathOperator{\Sing}{Sing}
\DeclareMathOperator{\shfHom}{\mathscr{H}\textit{\!\!om}}
\DeclareMathOperator{\WeilDivisors}{{Div}}
\DeclareMathOperator{\CartierDivisors}{{CaDiv}}
\DeclareMathOperator{\PrincipalWeilDivisors}{{PrDiv}}
\DeclareMathOperator{\LocallyPrincipalWeilDivisors}{{LPDiv}}
\DeclareMathOperator{\PrincipalCartierDivisors}{{PrCaDiv}}
\DeclareMathOperator{\DivisorClass}{{Cl}}
\DeclareMathOperator{\CartierClass}{{CaCl}}
\DeclareMathOperator{\Picard}{{Pic}}
\DeclareMathOperator{\Frob}{Frob}


%>>>>>>>>>>>>>>>>>>>>>>>>>>>>>>
%<<<  Mathematical Objects  <<<
%>>>>>>>>>>>>>>>>>>>>>>>>>>>>>>
\newcommand{\sll}{\mathfrak{sl}}
\newcommand{\gl}{\mathfrak{gl}}
\newcommand{\GL}{\mbox{GL}}
\newcommand{\PGL}{\mbox{PGL}}
\newcommand{\SL}{\mbox{SL}}
\newcommand{\Mat}{\mbox{Mat}}
\newcommand{\Gr}{\textup{Gr}}
\newcommand{\Squ}{\textup{Sq}}
\newcommand{\catSet}{\textit{Sets}}
\newcommand{\RP}{{\R\PP}}
\newcommand{\CP}{{\C\PP}}
\newcommand{\Steen}{\mathscr{A}}
\newcommand{\Orth}{\textup{\textbf{O}}}

%>>>>>>>>>>>>>>>>>>>>>>>>>>>>>>
%<<<  Mathematical Symbols  <<<
%>>>>>>>>>>>>>>>>>>>>>>>>>>>>>>
\newcommand{\DASH}{\textup{---}}
\newcommand{\op}{\textup{op}}
\newcommand{\CW}{\textup{CW}}
\newcommand{\ob}{\textup{ob}\,}
\newcommand{\ho}{\textup{ho}}
\newcommand{\st}{\textup{st}}
\newcommand{\id}{\textup{id}}
\newcommand{\Bullet}{\ensuremath{\bullet} }
\newcommand{\sprod}{\wedge}

%>>>>>>>>>>>>>>>>>>>>>>>>>>>>>>
%<<<      Some Arrows       <<<
%>>>>>>>>>>>>>>>>>>>>>>>>>>>>>>
\newcommand{\nt}{\Longrightarrow}
\let\shortmapsto\mapsto
\let\mapsto\longmapsto
\newcommand{\mapsfrom}{\,\reflectbox{$\mapsto$}\ }
\newcommand{\bigrightsquig}{\scalebox{2}{\ensuremath{\rightsquigarrow}}}
\newcommand{\bigleftsquig}{\reflectbox{\scalebox{2}{\ensuremath{\rightsquigarrow}}}}

%\newcommand{\cofibration}{\xhookrightarrow{\phantom{\ \,{\sim\!}\ \ }}}
%\newcommand{\fibration}{\xtwoheadrightarrow{\phantom{\sim\!}}}
%\newcommand{\acycliccofibration}{\xhookrightarrow{\ \,{\sim\!}\ \ }}
%\newcommand{\acyclicfibration}{\xtwoheadrightarrow{\sim\!}}
%\newcommand{\leftcofibration}{\xhookleftarrow{\phantom{\ \,{\sim\!}\ \ }}}
%\newcommand{\leftfibration}{\xtwoheadleftarrow{\phantom{\sim\!}}}
%\newcommand{\leftacycliccofibration}{\xhookleftarrow{\ \ {\sim\!}\,\ }}
%\newcommand{\leftacyclicfibration}{\xtwoheadleftarrow{\sim\!}}
%\newcommand{\weakequiv}{\xrightarrow{\ \,\sim\,\ }}
%\newcommand{\leftweakequiv}{\xleftarrow{\ \,\sim\,\ }}

\newcommand{\cofibration}
{\xhookrightarrow{\phantom{\ \,{\raisebox{-.3ex}[0ex][0ex]{\scriptsize$\sim$}\!}\ \ }}}
\newcommand{\fibration}
{\xtwoheadrightarrow{\phantom{\raisebox{-.3ex}[0ex][0ex]{\scriptsize$\sim$}\!}}}
\newcommand{\acycliccofibration}
{\xhookrightarrow{\ \,{\raisebox{-.55ex}[0ex][0ex]{\scriptsize$\sim$}\!}\ \ }}
\newcommand{\acyclicfibration}
{\xtwoheadrightarrow{\raisebox{-.6ex}[0ex][0ex]{\scriptsize$\sim$}\!}}
\newcommand{\leftcofibration}
{\xhookleftarrow{\phantom{\ \,{\raisebox{-.3ex}[0ex][0ex]{\scriptsize$\sim$}\!}\ \ }}}
\newcommand{\leftfibration}
{\xtwoheadleftarrow{\phantom{\raisebox{-.3ex}[0ex][0ex]{\scriptsize$\sim$}\!}}}
\newcommand{\leftacycliccofibration}
{\xhookleftarrow{\ \ {\raisebox{-.55ex}[0ex][0ex]{\scriptsize$\sim$}\!}\,\ }}
\newcommand{\leftacyclicfibration}
{\xtwoheadleftarrow{\raisebox{-.6ex}[0ex][0ex]{\scriptsize$\sim$}\!}}
\newcommand{\weakequiv}
{\xrightarrow{\ \,\raisebox{-.3ex}[0ex][0ex]{\scriptsize$\sim$}\,\ }}
\newcommand{\leftweakequiv}
{\xleftarrow{\ \,\raisebox{-.3ex}[0ex][0ex]{\scriptsize$\sim$}\,\ }}

%>>>>>>>>>>>>>>>>>>>>>>>>>>>>>>
%<<<    xymatrix Arrows     <<<
%>>>>>>>>>>>>>>>>>>>>>>>>>>>>>>
\newdir{ >}{{}*!/-5pt/@{>}}
\newcommand{\xycof}{\ar@{ >->}}
\newcommand{\xycofib}{\ar@{^{(}->}}
\newcommand{\xycofibdown}{\ar@{_{(}->}}
\newcommand{\xyfib}{\ar@{->>}}
\newcommand{\xymapsto}{\ar@{|->}}

%>>>>>>>>>>>>>>>>>>>>>>>>>>>>>>
%<<<     Greek Letters      <<<
%>>>>>>>>>>>>>>>>>>>>>>>>>>>>>>
%\newcommand{\oldphi}{\phi}
%\renewcommand{\phi}{\varphi}
\let\oldphi\phi
\let\phi\varphi
\renewcommand{\to}{\longrightarrow}
\newcommand{\from}{\longleftarrow}
\newcommand{\eps}{\varepsilon}

%>>>>>>>>>>>>>>>>>>>>>>>>>>>>>>
%<<<  1st-4th & parentheses <<<
%>>>>>>>>>>>>>>>>>>>>>>>>>>>>>>
\newcommand{\first}{^\text{st}}
\newcommand{\second}{^\text{nd}}
\newcommand{\third}{^\text{rd}}
\newcommand{\fourth}{^\text{th}}
\newcommand{\ZEROTH}{$0^\text{th}$ }
\newcommand{\FIRST}{$1^\text{st}$ }
\newcommand{\SECOND}{$2^\text{nd}$ }
\newcommand{\THIRD}{$3^\text{rd}$ }
\newcommand{\FOURTH}{$4^\text{th}$ }
\newcommand{\iTH}{$i^\text{th}$ }
\newcommand{\jTH}{$j^\text{th}$ }
\newcommand{\nTH}{$n^\text{th}$ }

%>>>>>>>>>>>>>>>>>>>>>>>>>>>>>>
%<<<    upright commands    <<<
%>>>>>>>>>>>>>>>>>>>>>>>>>>>>>>
\newcommand{\upcol}{\textup{:}}
\newcommand{\upsemi}{\textup{;}}
\providecommand{\lparen}{\textup{(}}
\providecommand{\rparen}{\textup{)}}
\renewcommand{\lparen}{\textup{(}}
\renewcommand{\rparen}{\textup{)}}
\newcommand{\Iff}{\emph{iff} }

%>>>>>>>>>>>>>>>>>>>>>>>>>>>>>>
%<<<     Environments       <<<
%>>>>>>>>>>>>>>>>>>>>>>>>>>>>>>
\newcommand{\squishlist}
{ %\setlength{\topsep}{100pt} doesn't seem to do anything.
  \setlength{\itemsep}{.5pt}
  \setlength{\parskip}{0pt}
  \setlength{\parsep}{0pt}}
\newenvironment{itemise}{
\begin{list}{\textup{$\rightsquigarrow$}}
   {  \setlength{\topsep}{1mm}
      \setlength{\itemsep}{1pt}
      \setlength{\parskip}{0pt}
      \setlength{\parsep}{0pt}
   }
}{\end{list}\vspace{-.1cm}}
\newcommand{\INDENT}{\textbf{}\phantom{space}}
\renewcommand{\INDENT}{\rule{.7cm}{0cm}}

\newcommand{\itm}[1][$\rightsquigarrow$]{\item[{\makebox[.5cm][c]{\textup{#1}}}]}


%\newcommand{\rednote}[1]{{\color{red}#1}\makebox[0cm][l]{\scalebox{.1}{rednote}}}
%\newcommand{\bluenote}[1]{{\color{blue}#1}\makebox[0cm][l]{\scalebox{.1}{rednote}}}

\newcommand{\rednote}[1]
{{\color{red}#1}\makebox[0cm][l]{\scalebox{.1}{\rotatebox{90}{?????}}}}
\newcommand{\bluenote}[1]
{{\color{blue}#1}\makebox[0cm][l]{\scalebox{.1}{\rotatebox{90}{?????}}}}


\newcommand{\funcdef}[4]{\begin{align*}
#1&\to #2\\
#3&\mapsto#4
\end{align*}}

%\newcommand{\comment}[1]{}

%>>>>>>>>>>>>>>>>>>>>>>>>>>>>>>
%<<<       Categories       <<<
%>>>>>>>>>>>>>>>>>>>>>>>>>>>>>>
\newcommand{\Ens}{{\mathscr{E}ns}}
\DeclareMathOperator{\Sheaves}{{\mathsf{Shf}}}
\DeclareMathOperator{\Presheaves}{{\mathsf{PreShf}}}
\DeclareMathOperator{\Psh}{{\mathsf{Psh}}}
\DeclareMathOperator{\Shf}{{\mathsf{Shf}}}
\DeclareMathOperator{\Varieties}{{\mathsf{Var}}}
\DeclareMathOperator{\Schemes}{{\mathsf{Sch}}}
\DeclareMathOperator{\Rings}{{\mathsf{Rings}}}
\DeclareMathOperator{\AbGp}{{\mathsf{AbGp}}}
\DeclareMathOperator{\Modules}{{\mathsf{\!-Mod}}}
\DeclareMathOperator{\fgModules}{{\mathsf{\!-Mod}^{\textup{fg}}}}
\DeclareMathOperator{\QuasiCoherent}{{\mathsf{QCoh}}}
\DeclareMathOperator{\Coherent}{{\mathsf{Coh}}}
\DeclareMathOperator{\GSW}{{\mathcal{SW}^G}}
\DeclareMathOperator{\Burnside}{{\mathsf{Burn}}}
\DeclareMathOperator{\GSet}{{G\mathsf{Set}}}
\DeclareMathOperator{\FinGSet}{{G\mathsf{Set}^\textup{fin}}}
\DeclareMathOperator{\HSet}{{H\mathsf{Set}}}
\DeclareMathOperator{\Cat}{{\mathsf{Cat}}}
\DeclareMathOperator{\Fun}{{\mathsf{Fun}}}
\DeclareMathOperator{\Orb}{{\mathsf{Orb}}}
\DeclareMathOperator{\Set}{{\mathsf{Set}}}
\DeclareMathOperator{\sSet}{{\mathsf{sSet}}}
\DeclareMathOperator{\Top}{{\mathsf{Top}}}
\DeclareMathOperator{\GSpectra}{{G-\mathsf{Spectra}}}
\DeclareMathOperator{\Lan}{Lan}
\DeclareMathOperator{\Ran}{Ran}

%>>>>>>>>>>>>>>>>>>>>>>>>>>>>>>
%<<<     Script Letters     <<<
%>>>>>>>>>>>>>>>>>>>>>>>>>>>>>>
\newcommand{\scrQ}{\mathscr{Q}}
\newcommand{\scrW}{\mathscr{W}}
\newcommand{\scrE}{\mathscr{E}}
\newcommand{\scrR}{\mathscr{R}}
\newcommand{\scrT}{\mathscr{T}}
\newcommand{\scrY}{\mathscr{Y}}
\newcommand{\scrU}{\mathscr{U}}
\newcommand{\scrI}{\mathscr{I}}
\newcommand{\scrO}{\mathscr{O}}
\newcommand{\scrP}{\mathscr{P}}
\newcommand{\scrA}{\mathscr{A}}
\newcommand{\scrS}{\mathscr{S}}
\newcommand{\scrD}{\mathscr{D}}
\newcommand{\scrF}{\mathscr{F}}
\newcommand{\scrG}{\mathscr{G}}
\newcommand{\scrH}{\mathscr{H}}
\newcommand{\scrJ}{\mathscr{J}}
\newcommand{\scrK}{\mathscr{K}}
\newcommand{\scrL}{\mathscr{L}}
\newcommand{\scrZ}{\mathscr{Z}}
\newcommand{\scrX}{\mathscr{X}}
\newcommand{\scrC}{\mathscr{C}}
\newcommand{\scrV}{\mathscr{V}}
\newcommand{\scrB}{\mathscr{B}}
\newcommand{\scrN}{\mathscr{N}}
\newcommand{\scrM}{\mathscr{M}}

%>>>>>>>>>>>>>>>>>>>>>>>>>>>>>>
%<<<     Fractur Letters    <<<
%>>>>>>>>>>>>>>>>>>>>>>>>>>>>>>
\newcommand{\frakQ}{\mathfrak{Q}}
\newcommand{\frakW}{\mathfrak{W}}
\newcommand{\frakE}{\mathfrak{E}}
\newcommand{\frakR}{\mathfrak{R}}
\newcommand{\frakT}{\mathfrak{T}}
\newcommand{\frakY}{\mathfrak{Y}}
\newcommand{\frakU}{\mathfrak{U}}
\newcommand{\frakI}{\mathfrak{I}}
\newcommand{\frakO}{\mathfrak{O}}
\newcommand{\frakP}{\mathfrak{P}}
\newcommand{\frakA}{\mathfrak{A}}
\newcommand{\frakS}{\mathfrak{S}}
\newcommand{\frakD}{\mathfrak{D}}
\newcommand{\frakF}{\mathfrak{F}}
\newcommand{\frakG}{\mathfrak{G}}
\newcommand{\frakH}{\mathfrak{H}}
\newcommand{\frakJ}{\mathfrak{J}}
\newcommand{\frakK}{\mathfrak{K}}
\newcommand{\frakL}{\mathfrak{L}}
\newcommand{\frakZ}{\mathfrak{Z}}
\newcommand{\frakX}{\mathfrak{X}}
\newcommand{\frakC}{\mathfrak{C}}
\newcommand{\frakV}{\mathfrak{V}}
\newcommand{\frakB}{\mathfrak{B}}
\newcommand{\frakN}{\mathfrak{N}}
\newcommand{\frakM}{\mathfrak{M}}

\newcommand{\frakq}{\mathfrak{q}}
\newcommand{\frakw}{\mathfrak{w}}
\newcommand{\frake}{\mathfrak{e}}
\newcommand{\frakr}{\mathfrak{r}}
\newcommand{\frakt}{\mathfrak{t}}
\newcommand{\fraky}{\mathfrak{y}}
\newcommand{\fraku}{\mathfrak{u}}
\newcommand{\fraki}{\mathfrak{i}}
\newcommand{\frako}{\mathfrak{o}}
\newcommand{\frakp}{\mathfrak{p}}
\newcommand{\fraka}{\mathfrak{a}}
\newcommand{\fraks}{\mathfrak{s}}
\newcommand{\frakd}{\mathfrak{d}}
\newcommand{\frakf}{\mathfrak{f}}
\newcommand{\frakg}{\mathfrak{g}}
\newcommand{\frakh}{\mathfrak{h}}
\newcommand{\frakj}{\mathfrak{j}}
\newcommand{\frakk}{\mathfrak{k}}
\newcommand{\frakl}{\mathfrak{l}}
\newcommand{\frakz}{\mathfrak{z}}
\newcommand{\frakx}{\mathfrak{x}}
\newcommand{\frakc}{\mathfrak{c}}
\newcommand{\frakv}{\mathfrak{v}}
\newcommand{\frakb}{\mathfrak{b}}
\newcommand{\frakn}{\mathfrak{n}}
\newcommand{\frakm}{\mathfrak{m}}

%>>>>>>>>>>>>>>>>>>>>>>>>>>>>>>
%<<<  Caligraphic Letters   <<<
%>>>>>>>>>>>>>>>>>>>>>>>>>>>>>>
\newcommand{\calQ}{\mathcal{Q}}
\newcommand{\calW}{\mathcal{W}}
\newcommand{\calE}{\mathcal{E}}
\newcommand{\calR}{\mathcal{R}}
\newcommand{\calT}{\mathcal{T}}
\newcommand{\calY}{\mathcal{Y}}
\newcommand{\calU}{\mathcal{U}}
\newcommand{\calI}{\mathcal{I}}
\newcommand{\calO}{\mathcal{O}}
\newcommand{\calP}{\mathcal{P}}
\newcommand{\calA}{\mathcal{A}}
\newcommand{\calS}{\mathcal{S}}
\newcommand{\calD}{\mathcal{D}}
\newcommand{\calF}{\mathcal{F}}
\newcommand{\calG}{\mathcal{G}}
\newcommand{\calH}{\mathcal{H}}
\newcommand{\calJ}{\mathcal{J}}
\newcommand{\calK}{\mathcal{K}}
\newcommand{\calL}{\mathcal{L}}
\newcommand{\calZ}{\mathcal{Z}}
\newcommand{\calX}{\mathcal{X}}
\newcommand{\calC}{\mathcal{C}}
\newcommand{\calV}{\mathcal{V}}
\newcommand{\calB}{\mathcal{B}}
\newcommand{\calN}{\mathcal{N}}
\newcommand{\calM}{\mathcal{M}}

%>>>>>>>>>>>>>>>>>>>>>>>>>>>>>>
%<<<<<<<<<DEPRECIATED<<<<<<<<<<
%>>>>>>>>>>>>>>>>>>>>>>>>>>>>>>

%%% From Kac's template
% 1-inch margins, from fullpage.sty by H.Partl, Version 2, Dec. 15, 1988.
%\topmargin 0pt
%\advance \topmargin by -\headheight
%\advance \topmargin by -\headsep
%\textheight 9.1in
%\oddsidemargin 0pt
%\evensidemargin \oddsidemargin
%\marginparwidth 0.5in
%\textwidth 6.5in
%
%\parindent 0in
%\parskip 1.5ex
%%\renewcommand{\baselinestretch}{1.25}

%%% From the net
%\newcommand{\pullbackcorner}[1][dr]{\save*!/#1+1.2pc/#1:(1,-1)@^{|-}\restore}
%\newcommand{\pushoutcorner}[1][dr]{\save*!/#1-1.2pc/#1:(-1,1)@^{|-}\restore}









\usepackage{framed}
%\usepackage{biblatex} 
\usepackage[style=alphabetic,citestyle=alphabetic,
url=false,doi=false,isbn=false,eprint=false]{biblatex}%
\newcommand{\NOFULLPAGE}{\relax}

\newcommand{\Sq}{\mathrm{Sq}}

\bibliography{../../Dropbox/logbook/_LOGBOOK/papers}
\begin{document}

\section*{\huge Research statement \hfill\normalsize Michael Donovan} 

In defining the notion of a model category, Daniel Quillen \cite{QuillenHomAlg.pdf} formulated the notion of homotopy theory in a variety of contexts. The context of primary interest in this statement is that of simplicial commutative rings. Quillen's advance allowed him to define a notion of the (co)homology of commutative rings, and put this definition in a common framework with the (co)homology of topological spaces. The resulting theory, now known as Andr\'e-Quillen (co)homology, is of inherent interest in algebraic topology, appearing for example in the proof of the Sullivan conjecture \cite{MillerSullivanConjecture.pdf}. It is of interest also to commutative algebraists, for example Michel Andr\'e, after whom the theory is also named for his development of a parallel theory of the simplicial resolution of rings.

My primary research goal has been to advance our understanding of the homotopy theory of simplicial commutative rings. By virtue of the fact that this homotopy theory fits into the same framework as the homotopy theory of spaces, one expects our understanding of one homotopy theory to shed light on the other. However, our understanding of simplicial algebras lags behind our knowledge of topological spaces.

My research has so far focused on the unstable Adams spectral sequence for simplicial commutative algebras. The \emph{stable} Adams spectral sequence for topological spaces  was introduced by Adams in 1958 \cite{MR0096219}, and its unstable counterpart was introduced by [who] in [when?]. On the other hand, the unstable Adams spectral sequence for simplicial commutative algebras was constructed by Radelescu-Banu only in 2006 \cite{radelescuBanu.pdf}. Further study of the classical case was performed by Bousfield and Kan in a series \cite{BousKanSSeq.pdf,BK_pairings.pdf, BK_pairings_products.pdf} of papers, while such study of the simplicial algebras case had not been performed.

Motivated and inspired by this and other work, I have taken on the task of performing all of this for the Adams spectral sequence for simplicial algebras. The next section will be a summary of the main results I have proved, and conjectures I have made, about the Adams spectral sequence. Following this section, I will remind the reader of the definition of \emph{spectral sequences}, and state some known properties of the classical unstable Adams spectral sequence for comparison with my work. Following this I will discuss operations in the Adams spectral sequence for simplicial algebras. The final section will discuss the structures I have had to study in order to gain access to the Adams $E^2$ page --- an infinite regress of composite functor spectral sequences.
\subsection*{Summary of results}
Fix a connected simplicial $\F_2$-algebra $X$.
Goerss [ref] showed that the Andr\'e-Quillen homology $H_*X$ of $X$ is a Lie coalgebra with an unstable left coaction of the dual of a certain algebra $\calP$ of cohomology operation. One might say that Goerss calculated the Koszul dual comonad to the commutative monad over $\F_2$; we will denote this comonad $\calL\hat{\,}\calP\hat{\,}$.

Radelescu-Banu has constructed in [ref] an Adams spectral sequence for $X$:
\[E^2_{-s,t}=\Ext^s_{\calL\hat{\,}\calP\hat{\,}}(\F_2[t],H_*^\textup{AQ}X)\implies \pi_{t-s}(\hat X_\textup{AQ}),\]
where $\hat X_\textup{AQ}$ is the completion of $X$ with respect to Andr\'e-Quillen homology. For a definition of the notion of spectral sequence, see the following section. A key example of this spectral sequence is when $X=S^d$ is the free $\F_2$-algebra on a standard simplicial $d$-sphere, but experience from classical topology would suggest that the $E^2$ page would be impossible to calculate. %Spectral sequences are a common tool in algebraic topology. For now, it is enough to think of a spectral sequence is a sequence $E^r$ of graded chain complexes such that $E^{r+1}$ is the homology of $E^r$. It will always happen in these examples that the groups $E^r$ stabilise degreewise for $r\gg0$, and we say that a spectral sequence calculates a vector space $W$ if $W$ is filtered and 

 A main result of my work will be to complete the proof of the following
\begin{conjecture}
There is a monad $\calC\calA\Delta$ on the category of bigraded $\F_2$-vectorspaces for which the $E^2$ page of the Adams spectral sequence for a connected simplicial algebra $X$ is an algebra, and such that the $E^2$ page for $S^d$ ($d>0$) is the free $\calC\calA\Delta$-algebra on the fundamental class $\imath\in E^2_{0,d}$.
\end{conjecture}
This conjecture is an analogue for the Adams spectral sequence of the results from [Goerss, thm 6.2, prop 6.5] regarding the Quillen spectral sequence. It is the summary of a rather large body of work, and relies on one remaining ingredient, namely the final conjecture in this document. I do not expect this result to be particularly challenging, but I have not yet fully investigated it.

The monad $\calC\calA\Delta$ of the theorem has as algebras  the bigraded vector spaces $W_{-s,t}$ equipped with:
\begin{enumerate}\squishlist
\setlength{\parindent}{.25in}
\item a commutative product $W_{-s,t}\otimes W_{-s',t'}\to W_{-s-s'-1,t+t'+1}$;
\item `horizontal' Steenrod operations $\Sq^i_\textup{h}:W_{-s,t}\to W_{-s-i,2t+1}$, zero unless $3\leq i\leq s+1$; and
\item `vertical' $\delta$ operations $\delta_i^\textup{v}:W_{-s,t}\to W_{-s-1,t+i+1}$, defined for $i$ satisfying $2\leq i<t$,
\end{enumerate}
subject to certain compatibilities.
Now the Adams spectral sequence serves to calculate the homotopy groups $\pi_{t-s}(\hat X_\textup{AQ})$ of the  completion of $X$ with respect to Andr\'e-Quillen homology. These groups support a commutative algebra structure, and an unstable action of operations $\delta_i$ [dwyer,bousfield]. Drawing inspiration from further work of Dwyer [], I prove:
\begin{thm*}
The commutative product, $\delta^\textup{v}$-operations and $\Sq_\textup{h}$-operations defined on $E^2$ extend wherever appropriate to the higher pages of the Adams spectral sequence, and these operations are compatible with the commutative product and $\delta$-operations on the target $\pi_{t-s}(X)$ of the spectral sequence.
\end{thm*}
An equivalent conjecture to the first conjecture is:
\begin{conjecture}[equivalent]
The Koszul dual of the comonad $\calL\hat{\,}\calP\hat{\,}$ is the monad $\calC\calA\Delta$.
\end{conjecture}
This is a measurement of the failure of Koszul duality for the commutative monad over $\F_2$ --- rationally, the Koszul double dual of the commutative monad is again the commutative monad, but over $\F_2$, double dualisation creates new operations of two different kinds.









A corollary of this would be the existence of the following `May-Koszul' spectral sequence for calculation of the Adams $E^2$ page, formed by filtering the comonad $\calL{\hat\,}\calP{\hat\,}$.
\begin{cor*}
For $X$ a connected simplicial algebra, there is a multiplicative spectral sequence:
\[{^{\textup{K}\!}}E^1_{***}\implies E^2_{**}\]
whose $E^1$ page is the free $\calC\calA\Delta$-algebra on the underlying vectorspace of $H_*X$.
\end{cor*}


As such we can calculate the Adams filtration of the homotopy groups of the algebraic sphere $S^d$. One discovers that the Adams spectral sequence runs rather faster than Quillen's spectral sequence [], an echo of the results of [nice spaces] in classical topology, which show that for `nice' topological spaces, the unstable Adams spectral sequence is a sped-up version of the lower-central-series spectral sequence.

Next, I prove the following two convergence theorems. Firstly, a vanishing line theorem, which follows from the first conjecture, or may be proved independently:
\begin{thm*}\label{vanishing-line-theorem}
Suppose that $X$ is a connected simplicial commutative algebra of finite type. Then the Adams spectral sequence for $X$ has a vanishing line of `slope $1$':
\[E^2_{-s,t}=0\textup{\ unless $s\leq (t-s)+1$}.\]
\end{thm*}
Next, a completeness theorem, originally conjectured by Radelescu-Banu [], and the analogue of the classical fact that simply-connected spaces are $R$-complete:
\begin{thm*}
If $X$ is a connected simplicial algebra, then the completion map $X\to \hat X_\textup{AQ}$ is an equivalence, so that the target of the Adams spectral sequence is simply $\pi_*(X)$.
\end{thm*}
%Now for $d>0$, let $X=S^d$ be a `sphere' in the category of commutative $\F_2$-algebras, i.e.\ the levelwise free algebra on a standard simplicial sphere. The previous conjecture is equivalent to

Now given that we have conjectured the $E^2$ page of the Adams spectral sequence for $S^d$, and that we know the homotopy of $S^d\sim \hat{S^d}_\textup{AQ}$, we might hope to be able to calculate all of the differentials.
\begin{conjecture}
All of the differentials in the Adams spectral sequence for $S^d$ are determined by simple commutation relations between the product, Steenrod operations, $\delta$ operations and the differential.
\end{conjecture}
\noindent A proof of this conjecture would require a close examination of the operations in question at the chain level, which I have not yet completed.
\subsection*{Spectral sequences}
\emph{Spectral sequences} are a powerful theoretical and computational tool, common in algebraic topology. However, they have a reputation for being complicated, so we'll remind the reader of a definition. For us, a \emph{spectral sequence}%footnote:
\footnote{In detail, our spectral sequences will have one homological grading and multiple cohomological gradings, which we will write as $E^r_{-s_n,\ldots,s_1,t}$. The differential $d^r$ will always have degree $(-r,r-1,0,\ldots,0)$.} will be a sequence $\{E^r\,;\,2\leq r<\infty\}$ of graded $\F_2$-vector spaces with a differential $d^r:E^r\to E^r$ for each $r$ such that $E^{r+1}$ is the homology $H(E^r;d^r)=\ker(d^r)/\im(d^r)$.

All of the spectral sequences we deal with will have the property that as $r$ increases, the group $E^r$ stabilises in each grading, so there is a limiting page, written $E^\infty$. We will say that the spectral sequence \emph{converges} to a graded vector space $W$ if $W$ has a decreasing filtration, and $E^\infty$ is isomorphic to the associated graded object of $W$, namely $F^sW/F^{s+1}W$. In this case, we write $E^2\implies W$.

Whenever $E^2\implies W$, one may hope to calculate $W$ by first calculating $E^2$ and then calculating the sequence $\{d^r\,;\,r\geq2\}$ of differentials. However, $E^2$ is generally very difficult to calculate, and even where it can be calculated, the differentials $d^r$ will be difficult to find.
A situation in which the calculation of a spectral sequence is more tractable is when each page $E^r$ supports any of a variety of algebraic structures in a way that is compatible with the differential. This will be a theme of the research presented here.

\subsection*{The classical unstable Adams spectral sequence}
We'll explain further by the example of the unstable Adams spectral sequence for topological spaces, which indeed motivated the present work. It was introduced as a tool for computing the homotopy groups of spheres. Given a connected topological space $X$, the homotopy groups $\pi_*(\hat X_2)$ of the $2$-completion of $X$ have a natural decreasing filtration, and the $E^r_{-s,t}\implies \pi_{t-s}(\hat X_2)$, i.e.\ 
\[E^\infty_{-s,t}=F_s\pi_{t-s}(\hat X_2)/F_{s+1}\pi_{t-s}(\hat X_2).\]
It is standard to use the shorthand $E^2_{-s,t}\implies \pi_{t-s}(\hat X_2)$ to indicate this fact. Moreover, Bousfield and Kan describe the $E^2$ page homotopically, as the derived functors
\[E^2_{-s,t}=\Ext^s(\F_2[t],H_*X),\]
taken in the category of unstable coalgebras over the Steenrod algebra [ref]. The question of how the homotopy groups of the completion $\hat X_2$ determine those of $X$ has been answered in particular when $X$ is a nilpotent space with finitely generated homotopy groups, in which case $\pi_*(\hat X_2)=\pi_*(X)\hat{_2}$ is the group-theoretic $2$-completion, allowing us to recover the $2$-component of $\pi_*(X)$. In particular, a full calculation of this spectral sequence for each sphere $X=S^d$ would yield a full calculation of the 2-torsion in the homotopy groups of spheres.

Both of the standard obstructions to calculation with spectral sequences apply in this setting: the derived functors defining $E^2$ are notoriously difficult to calculate, and the determination of the differential $d^r$ is diabolically challenging. 





Miller [ref] approached the calculation of the Adams $E^2$ by developing a composite functor spectral sequence to calculate $E^2_{-s,t}$ itself, which one may denote $E^2_{-s_2,-s_1,t}\implies E^2_{-s_2-s_1,t}$. However useful this spectral sequence was in proving the Sullivan conjecture, the new, trigraded $E^2$ term, given in terms of Andre-Quillen homology, is very difficult to calculate, as may be the differentials in Miller's spectral sequence. Nevertheless, we may draw a `master diagram' consisting of the two spectral sequences:
\[\pi_{t-s_1-s_2}(\hat X_2)\overset{\textup{Adams}}{\impliedby}E^2_{-s_2-s_1,t}\overset{\textup{Miller}}{\impliedby} E^2_{-s_2,-s_1,t}.\]

For the problem of calculating the Adams differentials (in ranges where $E^2_{-s,t}$ is known), one useful tool is due to Bousfield and Kan [ref]. They observe that the spectral sequence is in fact a spectral sequence of Lie algebras, so that each $E^r$ supports a Lie bracket compatible with the differential, and the resulting bracket on $E^\infty$ is compatible with the Whitehead bracket on $\pi_*(\hat X_2)$.



\subsection*{Operations in the Adams spectral sequence for simplicial algebras}
Given the extreme difficulty of calculating any of the three terms in the master diagram for topological spaces, one might expect the calculation of this spectral sequence for simplicial algebras to be unfeasable. However, we have a significant advantage in this context, which is that the homotopy operations for simplicial algebras are completely determined, by work of Bousfield, Cartan and Dwyer [refs]. In particular, the homotopy of `simplicial algebra spheres' is \emph{known}. The work of these authors shows that there are homotopy operations $\delta_i:\pi_{n}\to\pi_{n+i}$ defined for $2\leq i\leq n$.
One goal must then be to define compatible operations on the whole Adams spectral sequence.

The following theorem is inspired by Dwyer's beautiful calculation in [ref], but differs from it in two ways. Firstly, Dwyer studies the opposite filtration on a cosimplicial simplicial algebra, so that his results, while formally dual, do not apply. Secondly, in this Adams spectral sequence, all of the operations must be adjusted up to homotopy into filtration one higher than one might anticipate.


\begin{thm*}\label{adamsOperationsOmnibus}
There is a product pairing, `horizontal' Steenrod operations $\Sq^i_\textup{h}:E^r_{-s,t}\to E^r_{-s-i,2t+1}$, zero unless $3\leq j\leq s+1$, and `vertical' $\delta$ operations $\delta_i^\textup{v}:E^r_{-s,t}\to E^r_{-s-1,t+i+1}$, defined for $i$ satisfying $2\leq i<t-(r-2)$. If $x\in E^2_{-s,t}$ detects $a\in\pi_{t-s}(X)$, then $\delta_i^\textup{v}x$ detects $\delta_ia$ whenever $2\leq i\leq t-s$. If $t-s<i<t$, and $x\in E^2_{-s,t}$ survives to $E^{t-i+1}$, then there is a differential $d^{t-i+1}:\delta_i^\textup{v}x\mapsto \Sq_\textup{h}^{t-i+2}x$.
\end{thm*}
% I define `vertical' $\delta$-operations $\delta_i^\textup{v}:E^2_{-s,t}\to E^2_{-s-1,t+i+1}$ for $2\leq i<t$, and show that those $\delta_i^{\textup{v}}$ for which $i\leq t-s$ extend to the whole spectral sequence and are compatible with the $\delta_i$ on homotopy. One notes immediately that there is a disagreement between the unstable conditions on $\pi_*$ and on $E^2$, with more $\delta_i^\textup{v}$ being defined on $E^2_{-s,t}$ than there are $\delta_i$ defined on $\pi_{t-s}$. The remedy here is that whenever $t-s<i<t$, there is a canonical differential $d^{t-i+1}:\delta_i^\textup{v}x\mapsto \Sq_\textup{h}^{t-i+2}x$ for each $x\in E^2_{-s,t}$. The image of this differential we also define, being an example of a `horizontal' Steenrod operation $\Sq^j_\textup{h}:E^2_{-s,t}\to E^2_{-s-j,2t+1}$. These are zero unless $3\leq j\leq s+1$.
These operations allow us to populate the $E^2$ page of the Adams spectral sequence for $X=S^d$ (a simplicial algebra sphere), given only the fundamental class $\imath\in E^2_{0,d}$. That is, $E^2_{**}$ contains products of terms $\Sq^{j_b}_\textup{h}\cdots \Sq^{j_1}_\textup{h}\delta_{i_a}^\textup{v}\cdots \delta_{i_1}^\textup{v}\imath$, with the sequences involved suitably unstable and admissible. Although this might lead one to conjecture that a full description of $E^2(S^d)$ is available, more work is required in order to show, for example, that a given set of such classes is linearly independent on $E^2$, in order to give proofs of the conjectures made above. We will discuss this work in the following section.
\subsection*{Composite functor spectral sequences}

It happens, that the calculations can be taken much further for simplicial algebras over $\F_2$ than they can for topological spaces. Indeed, I perform the following construction, for $X$ a connected simplicial algebra over $\F_2$:
\begin{thm*}
For each $n\geq1$, there is a composite functor spectral sequence $\{{^{n\!}}E^r_{*\cdots *},\ r\geq2\}$, with each page ${^{n\!}}E^r$ having $n+1$ gradings. If we write $\{{^{0\!}}E^r\}$ for the Adams spectral sequence for $X$, then for all $n\geq1$, the spectral sequence $\{{^{n\!}}E^r\}$ computes ${^{n-1\!}}E^2$, so that $\{{^{n\!}}E^\infty\}$ is the associated graded object for some filtration on ${^{n-1\!}}E^2$. That is, there is a master diagram:
\[\xymatrix@R=4mm{
\pi_{*}(\hat X_\textup{AQ})&%r1c1
\ar@{=>}[l]_-{\textup{Ass}}
{^{0\!}}E^2_{**}&%r1c2
\ar@{=>}[l]_-{\textup{CFss}}
{^{1\!}}E^2_{***}
&%r1c3
{^{2\!}}E^2_{****}
\ar@{=>}[l]_-{\textup{CFss}}&%r1c4
\cdots
\ar@{=>}[l]_-{\textup{CFss}}%r1c5
}\]
\end{thm*}
%That is to say that I construct the following greatly extended master diagram, for $X$ a simplicial $\F_2$-algebra
%That is, a sequence of composite functor spectral sequences, each calculating the $E^2$-term of the previous, and finally, the Adams spectral sequence in place of the `zeroth' composite functor spectral sequence.
Although I will not give the full details of the derivation of these spectral sequences here, they are generalized Grothendieck spectral sequences as introduced by Blanc and Stover [ref]. They all have essentially the same construction. For example, ${^{0\!}}E^2_{-s,t}=\mathbb{R}^s\textup{Prim}(H_*X)$ is the right derived funtors of the primitives functor $(\calL{\hat{\,}}\calP{\hat{\,}})\textup{-coalgs}\to \textup{vect}$, and we find that this functor factors as
\[(\calL{\hat{\,}}\calP{\hat{\,}})\textup{-coalgs}\overset{\textup{Prim}_{\calP{\hat{\,}}}}{\to}
\calL{\hat{\,}}\textup{-coalgs}\overset{\textup{Prim}_{\calL{\hat{\,}}}}{\to}
 \F_2\textup{-vect}\]
Then $(R^*\textup{Prim}_{\calP{\hat{\,}}})(H_*X)$ is the homotopy of a cosimplicial Lie coalgebra, and by the results of [6author] it is compatibly a Lie coalgebra and an unstable comodule over the dual of the $\Lambda$ algebra. Writing $\calL{\hat{\,}}\Lambda{\hat{\,}}$ for the comonad whose coalgebras have such structure, one may use the construction of Blanc and Stover [] to define a spectral sequence \[{^{1\!}}E^2_{***}= (\mathbb{R}^* \textup{Prim}_{\calL{\hat{\,}}\Lambda{\hat{\,}}})(R^*\textup{Prim}_{\calP{\hat{\,}}})(H_*X) \implies (\mathbb{R}^* \textup{Prim}_{\calL{\hat{\,}}\calP{\hat{\,}}})(H_*X)={^{0\!}}E^2_{**}\]

%Then Blanc and Stover give a construction of a spectral sequence ${^{1\!}}E^2_{***}\implies {^{0\!}}E^2_{**}$, with ${^{1\!}}E^2_{***}$ given by certain right derived functors of the primitives functor 
%\[\textup{Prim}:\calL{\hat{\,}}\textup{-$\Pi$-coalgs}\to \F_2\textup{-vect}\]
%Here, $\calL{\hat{\,}}\textup{-$\Pi$-coalgs}$ is the category of cohomotopy coalgebras of cosimplicial Lie coalgebras, which by [6author], is a category of coalgebras for a comonad $\calL{\hat{\,}}\Lambda{\hat{\,}}$ whose coalgebras are at once Lie coalgebras and unstable comodules over the dual of the $\Lambda$ algebra. Indeed, we have
%\[{^{1\!}}E^2_{***}= (\mathbb{R}^* \textup{Prim}_{\calL{\hat{\,}}\Lambda{\hat{\,}}})(R^*\textup{Prim}_{\calP{\hat{\,}}})(H_*X) \]



This pattern continues ad infinitum. It is unusual, even in topology, to see such an infinite sequence of spectral sequences, and to make them effective as a calculation tool, one must endow them with extra structure. Using directly the constructions of Singer [], along with pairings I construct on the Blanc-Stover resolutions defining the composite functor spectral sequences:
\begin{thm*}
There are operations
\begin{enumerate}\squishlist
\setlength{\parindent}{.25in}
\item a pairing ${^{n\!}}E^r_{-s_{n+1},\ldots,t}\otimes {^{n\!}}E^r_{-s_{n+1}',\ldots,t'}\to {^{n\!}}E^r_{-s_{n+1}-s_{n+1}'-1,\ldots,t+t'+1}$;
\item horizontal `Steenrod operations' $\underline{\Sq}^i_\textup{h}:{^{n\!}}E^r_{-s_{n+1},\ldots,t}\to {^{n\!}}E^r_{-s_{n+1}-i-1,-2s_{n},\ldots,2t+1}$; and
\item vertical `Steenrod operations' $\underline{\Sq}^i_\textup{v}:{^{n\!}}E^r_{-s_{n+1},\ldots,t}\to {^{n\!}}E^r_{-s_{n+1}-1,-s_{n}-i,-2s_{n-1},\ldots,2t+1}$.
\end{enumerate}
The pairing at ${^{n\!}}E^\infty$ corresponds to the pairing at ${^{n-1\!}}E^2$. Moreover, if $x\in {^{n\!}}E^2_{-s'',-s',\ldots}$ detects $y\in {^{n-1\!}}E^2_{-s''-s',\ldots}$, then $\Sq_\textup{v}^ix$ detects $\Sq_\textup{h}^iy$ for $i\leq s'$, and $\Sq_\textup{v}^ix$ detects $\Sq_\textup{h}^{i+s'}y$ for $i\leq s''+1$.
%, the operations $\Sq^i_\textup{v}$ on ${^{n\!}}E^\infty$ correspond to the operations $\Sq^i_\textup{h}$ on ${^{n-1\!}}E^2$, and the operations $\Sq^i_\textup{h}$ on ${^{n\!}}E^\infty$ correspond to the operations $\Sq^{i+\textup{shift}}_\textup{h}$ on ${^{n-1\!}}E^2$.
\end{thm*}
We may summarise the structure in the various spectral sequences in the following enrichment of the master diagram:
\[\xymatrix@R=2mm{
\pi_{*}(\hat X_\textup{AQ})&%r1c1
\ar@{=>}[l]_-{\textup{Ass}}
{^{0\!}}E^2_{**}&%r1c2
\ar@{=>}[l]_-{\textup{CFss}}
{^{1\!}}E^2_{***}
&%r1c3
{^{2\!}}E^2_{****}
\ar@{=>}[l]_-{\textup{CFss}}&%r1c4
\cdots
\ar@{=>}[l]_-{\textup{CFss}}%r1c5
\\
\textup{com}&
\textup{com}\ar@{=>}[l]&
\textup{com}\ar@{=>}[l]&
\textup{com}\ar@{=>}[l]&
\cdots \ar@{=>}[l]\\
\delta_i&
\Sq^i_\textup{h}&
\Sq^i_\textup{h}\ar@{=>}[l]_-{\ \textup{sh.}}&
\Sq^i_\textup{h}\ar@{=>}[l]_-{\ \textup{sh.}}&
\cdots \ar@{=>}[l]_-{\ \textup{sh.}}\\
&
\delta^\textup{v}_i\ar@{=>}[ul]\ar[u]_-d
&
\Sq_\textup{v}^i\ar@{=>}[ul]&
\Sq_\textup{v}^i\ar@{=>}[ul]&
\cdots \ar@{=>}[ul]&
}\]
We'll end with a conjecture which would suffice to prove the first conjecture above, when used with a little cunning and with the results of May's thesis [].
\begin{conjecture}
For $n\geq2$, the vertical Steenrod operations on ${^{n\!}}E^2$ coincide with the Steenrod operations which arise from the observation that ${^{n\!}}E^2$ is given by the derived functors of $\textup{Prim}_{\calL{\hat{\,}}\Lambda{\hat{\,}}}$, and the algebra $\Lambda$ is Koszul dual to the Steenrod algebra (in which $\Sq^0$ is set to zero).
\end{conjecture}
I expect this result to be true, and indeed, have proven the analogous result when $n=1$, using the fact that the algebra of $\delta$ operations is Koszul dual to $\calP$.

\printbibliography

\end{document}


\section*{OFFCUTS}

We'll explain this diagram one section at a time. The second row is intended to indicate that the homotopy $\pi_*$ and each of the infinite sequence of $E^2$ pages may be endowed with the structure of commutative algebra, that each of the spectral sequences is indeed a spectral sequence of commutative algebras, and indeed, the products are compatible from one spectral sequence to the next. The operations $\delta_i$ on homotopy, and the operations $\delta_i^\textup{v}$ and $\Sq_\textup{h}$ on $E^2_{**}$ were discussed above.




 Almost all of this work is completed, and I am close to proving:
\begin{conjecture}
The Adams spectral sequence for a simplicial algebra sphere has $E^2(S^d)$ a subquotient of the polynomial algebra on the classes $\Sq^{j_b}_\textup{h}\cdots \Sq^{j_1}_\textup{h}\delta_{i_a}^\textup{v}\cdots \delta_{i_1}^\textup{v}\imath$.
\end{conjecture}
I are very close to proving this conjecture, and it is only for the sake of brevity that I do not identify the precise subquotient. A little more difficult is the following:
\begin{conjecture}
All of the differentials in this spectral sequence may be derived from simple commutation relations between the product, Steenrod operations, $\delta$ operations and the differential.
\end{conjecture}
The rest of this document is devoted to setting out the program I have developed for proving these conjectures. Before doing so, however, we point out a corollary of the first conjecture:
\begin{cor*}
For $X$ a connected simplicial algebra, there is a multiplicative spectral sequence, the Koszul spectral sequence:
\[{^{\textup{K}\!}}E^1_{***}\implies E^2_{**}\]
whose $E^1$ page is the free $\calC\calA\Delta$-algebra on the underlying vectorspace of $H_*X$.
\end{cor*}
\begin{proof}
Following Priddy [ref] one can filter the cobar construction on $H_*X$ by `length', in which one thinks of quadratic cohomology operations as doubling length.
\end{proof} Finally, one conjectures:
\begin{conjecture}
Suppose that $X$ is a connected simplicial algebra over $\F_2$. Then the Adams spectral sequence has $E^2(X)$ a subquotient of the polynomial algebra on the classes $\Sq^{j_b}_\textup{h}\cdots \Sq^{j_1}_\textup{h}\delta_{i_a}^\textup{v}\cdots \delta_{i_1}^\textup{v}h$, for $h\in H_*X$. 
\end{conjecture}

%After constructing a commutative product on $E^2_{**}$, we have products of terms $\Sq^{j_b}_\textup{h}\cdots \Sq^{j_1}_\textup{h}\delta_{i_a}^\textup{v}\cdots \delta_{i_1}^\textup{v}\imath$ (with the sequences involved suitably unstable and admissible). I conjecture that further combinations are unneccesary due to relations between the $\Sq$ and $\delta$-operations, and I conjecture many differentials between the (products of) classes $\Sq^{j_b}_\textup{h}\cdots \Sq^{j_1}_\textup{h}\delta_{i_a}^\textup{v}\cdots \delta_{i_1}^\textup{v}\imath$. However, at this point one cannot show that a given set of such classes is linearly independent, nor does it give a sense of what proportion of the $E^2$ page is built of such classes.

This is the analogue for the Adams spectral sequence of the results from [Goerss, thm 6.2, prop 6.5]. He investigates Quillen's spectral sequence, a spectral sequence whose analogue in classical topology is the lower central series spectral sequence of [Rector ``an unstable Adams'']. As one might expect by analogy with the results of [BousfieldCurtisNiceSpaces], my results suggest that the Adams spectral sequence is a sped-up version of Quillen's spectral sequence --- the $\delta^\textup{v}$-operations in the Adams spectral sequence only shift Adams filtration by one, while the $\delta$ operations of [Goerss, prop 6.5] double filtration.




\subsection*{Background on spectral sequences}
\emph{Spectral sequences} are a powerful theoretical and computational tool, common in algebraic topology. However, they have a reputation for being complicated, and so we will record here the details of the kinds of spectral sequences needed in this note, namely multiply graded `second quadrant' spectral sequences.

Such a spectral sequence will have the following form, for some fixed $n\geq1$. It will be a sequence of \emph{pages}, $E^r$, for $2\leq r< \infty$, with each page an $(n+1)$-graded $\F_2$-vector space $E^r_{-s_n,\ldots,-s_1,t}$ which may only be nonzero when all of $s_n,\ldots,s_1,t$ are nonnegative. Moreover, $E^r$ has a differential:
\[d^r:E^r_{-s_n,-s_{n-1},\ldots,-s_1,t}\to E^r_{-s_n-r,-s_{n-1}+r-1,\ldots,-s_1,t}.\]
The relationship between the various pages is that $E^{r+1}$ is isomorphic to the homology of $E^r$ for all $r\geq2$. It will always happen that in each multidegree $(-s_n,\ldots,-s_1,t)$, the sequence $E^r_{-s_n,\ldots,-s_1,t}$ stabilizes, so that we may define an $E^\infty$ page by requiring that $E^\infty_{-s_n,\ldots,-s_1,t}:=E^r_{-s_n,\ldots,-s_1,t}$ for $r\gg0$.

Now the spectral sequence above \emph{computes} or \emph{converges to} an $n$-times graded vector space $W$, denoted $E^2\implies W$, if $W$ in fact has an exhaustive, decreasing filtration:
\[W=F^0W\supseteq F^1W\supseteq F^2W\supseteq \cdots\ \ \ \textup{(with $\textstyle\bigcap_{s\geq0}F^sW=0$)}\]
and there are isomorphisms between the terms of the $E^\infty$ page and the filtration quotients
\[E^\infty_{-s_n,-s_{n-1},\ldots,-s_1,t}=\begin{cases}
(F^{s_n}/F^{s_n+1})W_{-s_n-s_{n-1},-s_{n-2},\ldots,-s_1,t},&\textup{if }n>1;\\
(F^{s_n}/F^{s_n+1})W_{-s_1+t},&\textup{if }n=1.
%\\,&\textup{if }
\end{cases}
\]
There are almost always two main obstructions to the calculation of a spectral sequence $E^2\implies W$. Firstly, one must determine $E^2$ itself, which is typically given as a collection of derived functors. Secondly, one must determine the differentials $d^r$ for $r\geq2$ in whatever range of degrees $E^2$ can be determined.



%\subsection*{offcuts}


% of bigraded differential abelian groups, with $E^r$ referred to as the \emph{$E^r$ page}, such that $E^{r+1}$ is the homology of $E^r$. The $E^2$ page can typically be described as a homological object. Typically it happens that the sequence of pages $E^r_{-s,t}$ stabilizes in each bidegree $(-s,t)$, in which case we may define an $E^\infty$ page by requiring that $E^\infty_{-s,t}:=E^r_{-s,t}$ for $r\gg0$. The $E^\infty$ page typically calculates some homotopy groups of interest, in the sense that these homotopy groups are filtered, and $E^\infty$ is the associated graded object.




%is a sequence $E^r$ ($2\leq r< \infty$) of bigraded differential abelian groups, with $E^r$ referred to as the \emph{$E^r$ page}, such that $E^{r+1}$ is the homology of $E^r$. The $E^2$ page can typically be described as a homological object. Typically it happens that the sequence of pages $E^r_{-s,t}$ stabilizes in each bidegree $(-s,t)$, in which case we may define an $E^\infty$ page by requiring that $E^\infty_{-s,t}:=E^r_{-s,t}$ for $r\gg0$. The $E^\infty$ page typically calculates some homotopy groups of interest, in the sense that these homotopy groups are filtered, and $E^\infty$ is the associated graded object.

\subsection*{The unstable Adams spectral sequence for simplicial algebras}
Now Goerss [ref] gave a full calculation of the operations on the Andr\'e-Quillen cohomology of simplicial algebras over $\F_2$. Resulting from his description, one finds that the Andre-Quillen homology $HX:=H_*^\textup{AQ}(X)$ is compatibly a Lie coalgebra and an unstable comodule over a certain bigraded coalgebra $\calP\hat{\,}$.

Radelescu-Banu constructed in [ref] an Adams spectral sequence for a simplicial algebra $X$:
\[E^2_{-s,t}=\Ext^s_{\calL\hat{\,}\calP\hat{\,}}(\F_2[t],HX)\implies \pi_{t-s}(\hat X_\textup{AQ}).\]
Here, the simplicial algebra $\hat X_\textup{AQ}$ might be called the completion of $X$ with respect to Andr\'e-Quillen homology. I prove the following conjecture of Radelescu-Banu, the analogue of the classical fact that simply-connected spaces are $R$-complete:
\begin{thm*}
If $X$ is a connected simplicial algebra, then the completion map $X\to \hat X_\textup{AQ}$ is an equivalence.
\end{thm*}
Combining this with theorem \ref{vanishing-line-theorem} (to follow), one has:
\begin{cor*}
If $X$ is a connected simplicial $\F_2$-algebra, the Adams spectral sequence converges strongly to the homotopy $\pi_*(X)$ of $X$.
\end{cor*}

\pagebreak

One part of my research has been to produce analogous results for the uASSsa: completion and convergence results, along with spectral sequence operations when the base field is $\F_2$.

A further component of my research has been to develop a composite functor spectral sequence with target the $E^2$ page of the uASSsa. This spectral sequence is analogous to Miller's spectral sequence in ???, rather than the May spectral sequence \cite{MayRestLie.pdf}. In fact, I develop an infinite series of such composite functor spectral sequences, each converging to the $E^2$ page of the previous. Using these spectral sequences, and Koszul methods following \cite{PriddyKoszul.pdf}, we can produce effective upper bounds on the size of the Adams $E^2$ page, including a vanishing line result.

Now even the $E^2$ page of the uASSts is notoriously difficult to compute, say, with input a sphere, and the homotopy groups of a topological sphere are notoriously difficult to calculate. On the other hand, over $\F_2$, the homotopy groups of a simplicial algebra sphere are known \cite{DwyerHtpyOpsSimpComAlg.pdf}, and one might expect the uASSca to be simple. Calculations I have made of the $E^2$-page show that this is quite untrue, and the spectral sequence is very far from degenerate. Indeed, it serves at this point to include a diagram of the various spectral sequences I have studied.

The tools and techniques I have developed appear to put within reach the complete calculation of the uASSsa for a simplicial $\F_2$-algebra sphere. I have made certain computer calculations which have led to a system of conjectured differentials, which if correct, would be all of the differentials in the spectral sequence. The work on verifying that these differentials are correct is ongoing, and seems certain to depend on understanding the various operations which can be defined on the composite functor spectral sequences I have developed.

If this work is completed, it will be possible to combine the understanding we have obtained of the $E^2$ term for spheres, in order to define a single `Koszul' spectral sequence converging to the $E^2$ term for any connected simplicial algebra.

%Completion of this analysis of the ASS for a simplicial algebra sphere may shed some light
To be (improved and) continued...

%\end{document}
\textup{}

\noindent The following six sections detail the various aspects of my research so far.



\section{The unstable Adams spectral sequence for simplicial $k$-algebras}
In \cite{radelescuBanu.pdf}, Radelescu-Banu shows how to form a homotopically correct Adams resolution for a simplicial commutative algebra $X$. He equips the cofibrant replacement functor $c$ with the structure of a comonad, after which the sequence
\[\frakX^s:=c(KQc)^{s+1}X\]
is a coaugmented cosimplicial object --- the comonad diagonal map is needed in order to define the coface maps. Then we call $\hat X:=\textup{Tot}\frakX$ \emph{the completion of $X$ with respect to abelianization}, a homotopical functor of $X$. The Adams spectral sequence is the spectral sequence of the mixed simplicial vectorspace $\frakX^\bullet_\bullet$, and we may regard its target as $\pi_*(\hat X)$.

\subsection{Completed work}
We prove a conjecture of Radelescu-Banu:
\begin{thm}
If $X$ is a connected simplicial algebra, then the completion map $X\to \hat X$ is an equivalence.
\end{thm}
This is the analogue of the topological fact that simply-connected spaces are $R$-complete. Combining this with theorem \ref{vanishing-line-theorem} (to follow), one has:
\begin{cor}
If $X$ is a connected simplicial $\F_2$-algebra, the Adams spectral sequence converges strongly to the homotopy $\pi_*(X)$ of $X$.
\end{cor}
\section{The homology of unstable Lie coalgebras over $\F_2$}
We will soon be faced with the task of calculating $E^2_{-s,t}(X)=\Ext^s(\F_2[t],H_*^Q(X))$, the $E^2$ page of the unstable Adams spectral sequence for $X$. This $\Ext$ group is taken in the category $\calC^1$ which is the target of the Andre-Quillen homology functor $H_*^Q$, a category of `unstable Lie coalgebras'. Omitting certain details, objects of this category are graded coalgebras which are compatibly unstable comodules for the coalgebra dual to the algebra $P$ generated by $P^i$ ($i\geq2$), subject to the Adem relation of \cite[p.17]{MR1089001}. There are degree shifts in these (co)operations, so that the cobracket has degree $-1$, and the dual of $P^i$ has degree $i+1$.

Write $\calV^1$ for the category of graded vector spaces, and $\textup{Pr}:\calC^1\to \calV^1$ for the functor whose value in dimension $t$ is $\Hom(\F_2[t],\DASH)$, so that $E^2_{-s,t}(X)=\mathbb{R}^s\textup{Pr}_t(H_*X)$. We refer to these derived functors as the homology of the unstable Lie coalgebra $H_*^QX$. For simplicity, we restrict to the case of connected simplicial algebras $X$.

\subsection{Completed work}\label{PastWorkOnE2LevelStructure}
%For simplicity, I restrict to the case where $X$ is connected. Then, omitting certain details, work of Goerss [] shows that $H:=H^Q_*X$ is an object in a certain comonadic category $\calC^1$ of Lie coalgebras which are compatibly coalgebras for the algebra of $P$-operations [\textbf{or something}]. It's in this category that we're calculating $\Ext$. 
In general, let $W$ be any object of $\calC^1$. In the style of the construction of Goerss  \cite[\S5]{MR1089001}, one defines a function $\xi_\textup{alg}:((C^{s+1}W)^{\wedge2})_{t}\to (C^{s+2}W)_{t+1}$, writing $C$ for the cofree construction in $\calC^1$, and $\wedge $ for the cokernel of the natural map from coproduct to product. One may combine this with a natural map $j:(\textup{Pr}(C^{s+1}W))^{\otimes2}\to \textup{Pr}((C^{s+1}W)^{\wedge 2})$, to obtain a pairing
\[\mathbb{R}^s\textup{Pr}_t(W)\otimes \mathbb{R}^{s'}\textup{Pr}_{t'}(W)\to \mathbb{R}^{s+s'+1}\textup{Pr}_{t+t'+1}(W)\textup{ given by }[x]\otimes [y]\mapsto [\xi_{\textup{alg}}j(D^0(x\otimes y))]\]
along with `Steenrod operations'
\[\Sq^{i+1}:\mathbb{R}^s\textup{Pr}_t(W)\to \mathbb{R}^{s+i+1}\textup{Pr}_{2t+1}(W)\textup{ given by }[x]\mapsto [\xi_{\textup{alg}}j(D^{s-i}(x\otimes x))]\]
where $\{D^k\}$ is a \emph{special cosimplicial Eilenberg-Zilber map} (c.f.\ \cite[5.2]{turner_opns_and_sseqs_I.pdf}). Note that $\Sq^0$ is zero, $\Sq^{i+1}$ is the self-square operation coming from the pairing, and $\Sq^{i+1}$ is zero when $i>s$.

Moreover, by virtue of the algebra of $\delta$-operations being Koszul dual to the algebra $P$, the derived functors admit (again omitting certain details) a left action of the $\delta$-operations:
\[\delta_i:\mathbb{R}^s\textup{Pr}(W)_t\to \mathbb{R}^{s+1}\textup{Pr}(W)_{t+i+1},\textup{ for $2\leq i<t$}.\]
When attempting to calculate $\mathbb{R}^*\textup{Pr}_*(W)$, one should also attempt to describe the action of these operations.


\subsection{Future progress}
It remains to prove that:
\begin{conjecture}
The `Steenrod operations' $\Sq^{i+1}$ satisfy the Adem relations suggested by their name and notation.
\end{conjecture}
\noindent This conjecture is analogous to results \cite[5.3]{PriddySimplicialLie.pdf} of Priddy. Unfortunately we need another mode of proof, since in this context the Eilenberg-Mac Lane functor $\overline{W}$ is rather less useful.

It also remains to find a commutation relation between the $\Sq^i$ and the $\delta_i$.

\section{Pairings and operations in the Adams spectral sequence for $\F_2$-algebras}
If $X$ is a simplicial $\F_2$-algebra, then $\pi_*(X)$ is itself a commutative algebra, of which $\bigoplus_{t>0}\pi_{t}(X)$ is a divided power ideal. Moreover, $\pi_*(X)$ exhibits higher divided square operations $\delta_i:\pi_t(X)\to \pi_{t+i}(X)$, for $2\leq i \leq t$, which satisfy an Adem relation, and the unstability relation $\delta_ix=\gamma_2(x)$. One would hope to be able to find products and $\delta$-operations on the whole spectral sequence, compatible at $E^\infty$ with those on $\pi_*(X)$.

We may use the canonical double complex, $E^0_{-s,t}=\frakX^s_t=(c(KQc)^{s+1}X)_t$, in which case we have 
\[E^1_{-s,t}=\pi_t(c(KQc)^{s+1}X)\cong \textup{Pr}(H_t^Q(c(KQc)^{s+1}X))\cong \textup{Pr}(C^{s+1}H_*^Q(X))\cong \Hom(H_*^QS^t,C^{s+1}(H^Q_*(X))).\]
One should not forget that we are derive the Adams spectral sequence as the spectral sequence of a cosimplicial simplicial algebra, filtered by cosimplicial degree. Dwyer \cite{DwyerHigherDividedSquares.pdf} has investigated the products and unary operations available in the spectral sequence of a cosimplicial simplicial coalgebra, filtered by cosimplicial degree. Unfortunately, this is not the linear dual of the situation of interest for the Adams spectral sequence I investigate.
\subsection{Completed work}
Using analogous constructions to Dwyer's, I prove:
\begin{thm}\label{adamsOperationsOmnibus}
Denote by $E^r_{-s,t}$ the spectral sequence of a cosimplicial simplicial nonuital $\F_2$-algebra $X$. Then one can construct a product pairing, Steenrod operations $\Sq^i:E^r_{-s,t}\to E^r_{-s-i,2t}$ (for $i\geq0$), and $\delta$ operations $\delta_i:E^r_{-s,t}\to E^r_{-s,t+i}$ (for $i$ satisfying $2\leq i<t-(r-2)$). The Steenrod operations have indeterminacy vanishing by $E^{2r-2}$, survive to $E^{2r-1}$, and commute with differentials, in the sense that $d_{2r-1}(\Sq^ix)=\Sq^{i+r-1}(d_rx)$ modulo indeterminacy. The $\delta$-operations may be multivalued when $i\geq \textup{max}(t-2(r-2),n+1)$ (so that at $E^2$, all the operations are well defined except for $\delta_t$, which is not defined at all). The Steenrod operations vanish whenever $i<r$, after all:
\begin{enumerate}\squishlist
\setlength{\parindent}{.25in}
\item When $i<t-s$, $d_r\delta_i(x)=\delta_i(d_rx)$.
\item When $i=t-s$, $d_r\delta_i(x)=\begin{cases}
\delta_i(d_rx),&\textup{if }s>0;\\
\delta_i(d_rx)+x\times d_rx,&\textup{if }s=0.
%\\,&\textup{if }
\end{cases}$
\item When $i>t-s$, $d_r\delta_i(x)=\begin{cases}
\delta_i(d_rx),&\textup{if }r<t-i+1;\\
\delta_i(d_rx)+\Sq^{t-i+1}(x),&\textup{if }r=t-i+1;\\
\Sq^{t-i+1}(x),&\textup{if }r>t-i+1.\\
%\\,&\textup{if }
\end{cases}$
\end{enumerate}
Broadly speaking then, the $\delta_i$ operations for $i>t-s$ support a differential hitting either a Steenrod operation or another $\delta_i$ operation, and all Steenrod operations applied to permanent cycles are hit in this way. If $X$ admits a coaugmentation from a simplicial algebra $X$, then the $\delta$-operations with $i\leq n$ at $E^\infty$ agree with the $\delta$-operations on the target of the spectral sequence. The $\delta$-operations (resp.\ Steenrod operations)  at $E^2$ agree with those following from the observation that $E^1$ is the normalization of a cosimplicial $\textup{Com}$-$\Pi$-algebra (resp.\ algebra). The operations may be externalized to operations $E^r(X)\to E^r(X\otimes_{\Sigma_2} X)$.
\end{thm}
It happens, however, that in the Adams spectral sequence of interest, all of these operations can and must be lifted to higher filtration. We'll use the same $D^1$ construction as Bousfield and Kan use to construct Whitehead squares in the topological Adams spectral sequence in \cite{BK_pairings_products.pdf}.
\begin{prop}
If $\frakX$ is the Radelescu-Banu resolution of a simplicial algebra, the multiplication $\mu:\frakX\otimes \frakX\to \frakX$ factors as in the commuting diagram
\[\xymatrix@R=4mm{
\frakX\otimes_{\Sigma_2}\frakX
\ar[r]
\ar[drr]_-{\mu}
&
\overline{D}^1\frakX
\ar[dr]^-(.7){\overline{\delta}}
&%r1c1
&%r1c2
{D}^1\frakX\ar[ll]_-{\sim}
\ar[dl]^-{{\delta}}\\%r1c3
&&%r2c1
\frakX&%r2c2
%r2c3
}\]
\end{prop}
Here, $\overline{D}^1\frakX$ is a version of the $D^1$-construction modified using our knowledge of the structure of $\frakX$, the crucial point being that each cosimplicial level of $\frakX$ bears a weak equivalence to an abelian object in the category of simplicial algebras.
\begin{cor}\label{liftedOperations}
The externalized operations of theorem \ref{adamsOperationsOmnibus} may be lifted to:
\begin{enumerate}\squishlist
\setlength{\parindent}{.25in}
\item a pairing $E^r_{-s,t}\otimes E^r_{-s',t'}\to E^r_{-s-s'-1,t+t'+1}$;
\item horizontal `Steenrod operations' $\underline{\Sq}^i_\textup{h}:E^r_{-s,t}\to E^r_{-s-i-1,2t+1}$; and
\item vertical `$\delta$-operations' $\underline{\delta}_i^\textup{v}:E^r_{-s,t}\to E^r_{-s-1,t+i+1}$
\end{enumerate}
\end{cor}
These symbols representing these new operations have been underlined, as it's not clear at present what relation they bear to the relations mentioned above, nor what Adem relations they will satisfy. However:
\begin{thm}
There is a commuting diagram
\[\mathclap{\xymatrix@R=4mm{
\pi_t(X^s)\otimes \pi_{t'}(X^s)\ar[d]^-{h\otimes h}
\ar[r]^-{\textup{sEZ}}
&%r1c1
\pi_{\frakt}(X^s\otimes X^s)\ar[r]^{\textup{(zig-zag)}}
&%r1c2
\pi_{\frakt}({D}^1X^s)&%r1c3
\pi_{\frakt+1}(X^{s+1})
\ar[l]^-{\cong\textup{-ish}}_-{\partial_\textup{conn}}
\ar[d]^-{h}
%\\%r1c4
%&%r2c1
%&%r2c2
%\pi_{\frakt}(D^1X)\ar@{-}[u]^-{\textup{zig-zag}}_-{\cong}
%&%r2c3
%\pi_{\frakt+1}(VX)\ar[u]_-{\cong}
%\ar[l]^-{\cong\textup{-ish}}_-{\partial_\textup{conn}}
%\ar[d]^-{\cong}
\\%r2c4
\textup{Pr}_t(HX^s)\otimes \textup{Pr}_{t'}(HX^s)\ar[r]^-{j}
&%r3c1
\textup{Pr}_{\frakt+1}(HX^s\otimes HX^s)\ar[rr]^-{\xi_\textup{alg}}
&%r3c2
&%r3c3
\textup{Pr}_{\frakt+1}(HX^{s+1})%r3c4
}}\]
so that the pairing of \ref{liftedOperations} coincides with the pairing on $E^2$ described in \S\ref{PastWorkOnE2LevelStructure}, and the Steenrod operation $\underline{\Sq}^i$ equals the operation $\Sq^{i+1}$ of \S\ref{PastWorkOnE2LevelStructure}.
\end{thm}
\subsection{Future work}
It remains to prove
\begin{conjecture}\label{vert-equals-koszul-n=1}
The operations $\underline{\delta}_i^\textup{v}$ coincide at $E^2$ with the $\delta_i$ of \S\ref{PastWorkOnE2LevelStructure}.
\end{conjecture}
%I as yet do not know if $\underline{\delta}_i$ coincides with $\delta_i$, although I really hope that it does. It might be false, or difficult, since the previous commuting diagram might not be as useful in this case.

\section{Iterative calculation of the homology of unstable Lie coalgebras}
The work summarised so far does not give us sufficient power to calculate the $E^2$-page of interest. However, I observe that the calculation of the homology of unstable Lie coalgebras can be performed iteratively, in a sense I will now explain.
\subsection{Completed work}\label{Iterative-calculation}
\begin{prop}
The functor $\textup{Pr}=\textup{Pr}_{\calC^1}$ factors through the category $\calL^1$ of Lie coalgebras:
\[\xymatrix@R=4mm@1{
\calC^1\ar[r]^-{\textup{Pr}_P}
&%r1c1
\calL^1\ar[r]^-{\textup{Pr}_\calL}
&%r1c2
\calV^1%r1c3
}\]
\end{prop}
We may thus construct a (triply-graded) Grothendieck spectral sequence calculating $(\mathbb{R}^*\textup{Pr})W$, for $W\in\calC^1$, using the techniques of \cite{Blanc_Stover-Groth_SS.pdf}. It would be too great an oversimplification to say that the $E^2$-page of the spectral sequence is given by $(\mathbb{R}^{s_2}\textup{Pr}_{\scrL^1\textup{-$\Pi$-coalg}})(\mathbb{R}^{s_1}\textup{Pr}_P)(W)_t$. 
More carefully, denote by $\calC^2$ the target category for the cohomotopy functor with domain $c(\calC^1)$, the category of $\calC^1$-cohomotopy-coalgebras (dual to the notion of $\Pi$-algebras in \cite{Blanc_Stover-Groth_SS.pdf}). Objects of $\calC^2$ are doubly graded, having a homological grading $t$ like that of elements of $\calC^1$, and a cohomological grading $s$, coming from cosimplicial degree. Moreover, $\mathbb{R}^*\textup{Pr}_P(W)$ is an object of $\calC^2$. We are finally equipped to write:
\[E^2_{-s_2,-s_1,t}=((\mathbb{R}^{s_2}\textup{Pr}_{\calC^2})(\mathbb{R}^*\textup{Pr}_P)W)_{-s_1,t}\implies E^2_{-s_2-s_1,t}.\]
Now work of the six authors of \cite{6Author.pdf} serves to demonstrate that the category $\calC^2$ is itself a category of unstable Lie coalgebras, this time with (approximately) an unstable right coaction of the $\Lambda$-algebra. Moreover,
\begin{prop}
The functor $\textup{Pr}_{\calC^2}$ factors through a category $\calL^2$ of Lie coalgebras:
\[\xymatrix@R=4mm@1{
\calC^2\ar[r]^-{\textup{Pr}_\Lambda}
&%r1c1
\calL^2\ar[r]^-{\textup{Pr}_\calL}
&%r1c2
\calV^2%r1c3
}\]
yielding a Grothendieck spectral sequence 
\[E^2_{-s_3,-s_2,-s_1,t}=((\mathbb{R}^{s_3}\textup{Pr}_{\calC^2})(\mathbb{R}^*\textup{Pr}_\Lambda)(\mathbb{R}^*\textup{Pr}_P)W)_{-s_2,-s_1,t}\implies E^2_{-s_3-s_2,-s_1,t}.\]
\end{prop}
This is the sense in which an iterative calculation of $\mathbb{R}^*\textup{Pr}_{\calC^1}$ is possible. In order to simplify exposition, suppose that $Y$ is an object of $\calC^N$, with $\calC^N$ a category of Lie coalgebras with unstable right coaction of the dual $\Lambda$-algebra. 
\begin{prop}
There are sequences of categories $\calC^N$ and $\calL^N$ ($N\geq1$), such that for each $N\geq2$, 
\begin{enumerate}\squishlist
\setlength{\parindent}{.25in}
\item $\calC^N$ is a category of $N$-graded Lie coalgebras with unstable right coaction of the dual $\Lambda$-algebra;
\item $\calL^N$ is a category of $N$-graded Lie coalgebras;
\item $\textup{Pr}_{\calC^N}:\calC^N\to\calV^N$ factors as $\xymatrix@R=4mm@1{
\calC^N\ar[r]^-{\textup{Pr}_\Lambda}
&%r1c1
\calL^N\ar[r]^-{\textup{Pr}_\calL}
&%r1c2
\calV^N%r1c3
}$;
\item $\calC^N$ is the category of $\calL^{N-1}$-cohomotopy-coalgebras.
\end{enumerate}
For $Y\in\calC^N$ ($N\geq2$), $(\mathbb{R}^*\textup{Pr}_\Lambda)Y$ is an object of $\calC^{N+1}$, and there is a Grothendieck spectral sequence
\[E^2=((\mathbb{R}^{*}\textup{Pr}_{\calC^{N+1}})(\mathbb{R}^*\textup{Pr}_\Lambda)Y)_{-s_{N+1},-s_N,\ldots,-s_1,t}\implies ((\mathbb{R}^{*}\textup{Pr}_{\calC^{N}})Y)_{-s_{N+1}-s_N,-s_{N-1},\ldots,-s_1,t}.\]
%
%
%
%The functor $\textup{Pr}_{\calC^N}$ factors as $\xymatrix@R=4mm@1{
%\calC^N\ar[r]^-{\textup{Pr}_\Lambda}
%&%r1c1
%\calL^N\ar[r]^-{\textup{Pr}_\calL}
%&%r1c2
%\calV^N%r1c3
%}$ through a category $\calL^N$ of Lie coalgebras; $\mathbb{R}^*\textup{Pr}_\Lambda$ is an object of a category $\calC^{N+1}$ which is a new category of unstable Lie coalgebras with of $\calL^n$-$\Pi$-coalgebras:
%\[\xymatrix@R=4mm@1{
%\calC^N\ar[r]^-{\textup{Pr}_\Lambda}
%&%r1c1
%\calL^N\ar[r]^-{\textup{Pr}_\calL}
%&%r1c2
%\calV^N%r1c3
%}\]
%yielding a Grothendieck spectral sequence 
%\[E^2_{-s_{N+1},-s_N,\ldots,-s_1,t}=((\mathbb{R}^{s_3}\textup{Pr}_{\calC^2})(\mathbb{R}^*\textup{Pr}_\Lambda)(\mathbb{R}^*\textup{Pr}_P)W)_{-s_2,-s_1,t}\implies E^2_{-s_3-s_2,-s_1,t}.\]
%There is a Grothendieck spectral sequence
\end{prop}
As in \S\ref{PastWorkOnE2LevelStructure}, we can define a pairing on the $\mathbb{R}^*\textup{Pr}_{\calC^N}$:
\[\mathbb{R}^*\textup{Pr}_{\calC^N}(Y)_{-s_{N},\ldots,t}\otimes \mathbb{R}^*\textup{Pr}_{\calC^N}(Y)_{-s_{N}',\ldots,t'}\to \mathbb{R}^*\textup{Pr}_{\calC^N}(Y)_{-s_{N}-s_{N}'-1,-s_{N-1}-s_{N-1}',\ldots,-s_1-s_1',t+t'+1}\]
along with horizontal `Steenrod operations'
\[\Sq^{i+1}_\textup{h}:\mathbb{R}^*\textup{Pr}_{\calC^N}(Y)_{-s_{N},\ldots,t} \to \mathbb{R}^*\textup{Pr}_{\calC^N}(Y)_{-s_{N}-i-1,-2s_{N-1},\ldots,-2s_1,2t+1}\]
Moreover, as the opposite of the $\Lambda$-algebra is Koszul dual to the Steenrod algebra for Lie algebras (via $\lambda_i\leftrightarrow \Sq^{i+1}$, with $\Sq^0=0$), we may define vertical (left) Steenrod operations (satisfying the Adem relations):
\[\Sq^{i+1}_\textup{K}:\mathbb{R}^*\textup{Pr}_{\calC^N}(Y)_{-s_{N},\ldots,t} \to \mathbb{R}^*\textup{Pr}_{\calC^N}(Y)_{-s_{N}-1,-s_{N-1}-i,-2s_{N-2},\ldots,-2s_1,2t+1}\]
\subsection{Future work}
We would like to prove:
\begin{conjecture}
The horizontal steenrod operations $\Sq_\textup{h}^{i+1}$ satisfy the Adem relation suggested by their name and notation.
\end{conjecture}

\section{Pairings and operations in the Grothendieck spectral sequences for the homology of unstable Lie coalgebras}
We are now faced with the question of how the pairing and operations at the $E^2$ page of the above Grothendieck spectral sequence relate to those at $E^\infty$. In \cite{SingerSteen1.pdf} (etc.), Singer constructs both vertical and horizontal Steenrod operations in the spectral sequence of a bicosimplicial algebra, abutting to the Steenrod operations on the target. We would like to do something similar in the Grothendieck spectral sequence, explaining the operations described in the previous section.
\subsection{Completed work}
Now if $Y\in\calC^N$, one can form a cosimplicial fibrant replacement $Y_\textup{f}$ using the cobar construction $Y_\textup{f}^\bullet=C^{\bullet+1}Y$. Then Blanc and Stover produce a monad $W$ on $c\calL^N$, allowing us to produce a bicosimplicial object $\scrW(\textup{Pr}_{\Lambda}Y_\textup{f})^{\bullet\bullet}:= (W^{\bullet+1}Y_\textup{f})^\bullet$. It's the bicosimplicial vectorspace $\textup{Pr}_{\calL^N} \scrW(\textup{Pr}_{\Lambda}Y_\textup{f})^{\bullet\bullet}$ whose spectral sequence is the Grothendieck spectral sequence calculating $(\mathbb{R}^*\textup{Pr}_{\calC^N})Y$. We are able to produce extra structure on the functor $W$:
\begin{prop}
$WY$ is naturally a (strict) commutative group object, with the functor $\hom(\DASH,WY)$ taking values in $\ensuremath{\F_2}$-vectorspaces.
\end{prop}
Using this structure, we may again follow the protocol of \cite[\S5]{MR1089001} to define a pairing $\xi_\textup{ch}:(\scrW^{\bullet} (\textup{Pr}_\Lambda Y_\textup{f})^\bullet)^{\wedge 2}\to \scrW^{\bullet+1} (\textup{Pr}_\Lambda Y_\textup{f})^\bullet$, and one can prove:
\begin{prop}
The map $\textup{Pr}_{\calL^N}\xi_\textup{ch}$ induces a degree $(-1,0)$ chain map after taking double normalizations
\[ N_*N_*((\textup{Pr}_{\calL^N}\scrW\textup{Pr}_\Lambda Y)^{\otimes2}_{\Sigma_2})\to N_*N_*(\textup{Pr}_{\calL^N}\scrW\textup{Pr}_\Lambda Y).\]
\end{prop}
Now Singer's operations can be externalised, so that if $A^{\bullet\bullet}$ is any bicosimplicial vectorspace, one obtains horizontal and vertical operations $E^r(A)\to E^r(A\otimes_{\Sigma_2}A)$, and a pairing $E^r(A)\otimes_{\Sigma_2}E^r(A)\to E^r(A\otimes_{\Sigma_2}A)$. Thus
\begin{cor}\label{opns-on-whole-gss}
There are operations
\begin{enumerate}\squishlist
\setlength{\parindent}{.25in}
\item a pairing $E^r_{-s_{n+1},\ldots,t}\otimes E^r_{-s_{n+1}',\ldots,t'}\to E^r_{-s_{n+1}-s_{n+1}'-1,\ldots,t+t'+1}$;
\item horizontal `Steenrod operations' $\underline{\Sq}^i_\textup{h}:E^r_{-s_{n+1},\ldots,t}\to E^r_{-s_{n+1}-i-1,-2s_{n},\ldots,2t+1}$; and
\item vertical `Steenrod operations' $\underline{\Sq}^i_\textup{v}:E^r_{-s_{n+1},\ldots,t}\to E^r_{-s_{n+1}-1,-s_{n}-i,-2s_{n-1},\ldots,2t+1}$.
\end{enumerate}
\end{cor}


\begin{thm}
The pairings of \ref{opns-on-whole-gss} and \S\ref{Iterative-calculation} coincide. The horizontal operations $\underline{\Sq}_\textup{h}^i$ of \ref{opns-on-whole-gss} coincide on $E^2$ with the horizontal operations ${\Sq}_\textup{h}^{i+1}$ of \S\ref{Iterative-calculation}. The operations $\underline{\Sq}_\textup{h}^i$ and $\underline{\Sq}_\textup{v}^i$ converge, in the style of \cite{SingerSteen1.pdf}, to the horizontal operations ${\Sq}_\textup{h}^{i+1}$ on the target.
\end{thm}
\subsection{Future work}\label{work-on-kos-opns-n>1}
In the future we should understand the way that the vertical operations $\underline{\Sq}_\textup{v}^i$ relate with the Koszul operations $\Sq_\textup{v}^{i+1}$.
If we can succeed in this analysis, then we'll likely be able to prove:
\begin{conjecture}\label{basis-of-derived-prims}
Suppose that $Y\in\calC^N$ is a one-dimensional unstable Lie coalgebra with all cooperations zero. Then each spectral sequence in the hierarchy of Grothendieck spectral sequences calculating $\mathbb{R}\textup{Pr}_{\calC^N}Y$ collapses, and we can specify a basis for $\mathbb{R}\textup{Pr}_{\calC^N}Y$ consisting of certain monomials in the iterated vertical Steenrod operations on the `fundamental class'.
\end{conjecture}

\section{The Adams spectral sequence for a sphere; vanishing lines}
\subsection{Completed work}
It is possible to form a bound on the size of the $E^2$ page of the Adams spectral sequence, written in terms of the Koszul operations, even without proving conjectures \ref{vert-equals-koszul-n=1} and \ref{basis-of-derived-prims}, and without the work proposed in \S\ref{work-on-kos-opns-n>1}. From this observation follows
\begin{thm}\label{vanishing-line-theorem}
Suppose that $X$ is a connected simplicial commutative algebra of finite type. % which is connected, and is of finite type in dimension 1. That is, $H^0_Q(X)=0$ and $H^1_Q(X)$ is a finite dimensional $\F_2$-vector space. 
Then the Adams spectral sequence for $X$ has a vanishing line of slope $T-1$:
\[E^2_{s,t}=0\textup{\ unless $s\leq\tfrac{1}{T-1}(t-s)+1$,}\]
where $T$ is the constant $2$, or $9/4$ if $X$ is the sphere $S^d$ for $d\geq2$.
\end{thm}
Now assume that conjecture \ref{basis-of-derived-prims} holds with $Y=H^Q_* S^d$ (for $d\geq2$). Then using computer simulations, I was able to conjecture a (very large) system of differentials which, if correct, would be all of the differentials in the Adams spectral sequence for the sphere $S^d$ for $d\geq2$.

\subsection{Future work}
It remains to verify that these conjectured differentials are correct. For this, I will probably need to prove all of the above conjectures, and then perform a calculation using the differentials provided by \ref{adamsOperationsOmnibus}.

If this work is completed, it will be possible to combine the understanding we have obtained of the $E^2$ term for spheres, in order to define a single `Koszul' spectral sequence converging to the $E^2$ term for any connected simplicial algebra.



\printbibliography

\end{document}

Quillen \cite{QuillenHomAlg.pdf} has formulated the notion of homotopy theory in a variety of algebraic contexts. In particular, there is a homotopy theory of any given type of universal algebra. The homotopy theory of augmented \ensuremath{\F_2}-algebras is an important and interesting example. It is of interest both to topologists at least insofar as it appears in the proofs of such interesting results as the Sullivan conjecture \cite{MillerSullivanConjecture.pdf}. This homotopy theory is of interest in commutative algebra, as Andr\'e and Quillen's definition of the cohomology of an augmented $\F_2$-algebra coincides with the Quillen homology of said algebra using the homotopical definitions of \cite{QuillenHomAlg.pdf}.

We are thus amply justified in considering the homotopy theory of augmented \ensuremath{\F_2}-algebras worthy of study. Interestingly, however, it is not well understood at present --- there is much progress to be made, including a number of unproven potential theorems analogous to long-standing results in the homotopy theory of spaces. I thus propose an investigation into a number of elements of this homotopy theory, with scope for either broadening or specialisation.
Present research has focused on the definition and study of an analogue of the Adams spectral sequence, while the hope is to turn to applications as soon as possible.
%Recently, it was suggested by my advisor that I should develop and study an Adams spectral sequence in the category $s\calA$ of simplicial augmented $\F_2$-algebras. Since that time, I have enjoyed studying the work relevant to this research, and have found the various challenges associated with understanding the resulting spectral sequence to be of interest. Thus, I make the following proposal.

\section*{Background}
Let $\calA$ be any category of universal algebras. Quillen developed in \cite{QuillenHomAlg.pdf} a model structure on the category $s\calA$ of simplicial objects in $\calA$, and introduced the notion of the Quillen homology of objects of $s\calA$.
An \emph{abelian object of $s\calA$} is an object $A\in s\calA$ along with a natural abelian group structure on $s\calA(\DASH,A)$. There is an evident category $\mathsf{ab}(s\calA)$ of abelian objects, and a left adjoint to the forgetful functor yielding an adjunction $s\calA\rightleftarrows\mathsf{ab}(s\calA)$. Quillen homology is then defined to be the left derived functors of \emph{abelianization}, this left adjoint.

At this point we specialise, letting $\calA$ be the category of augmented commutative \ensuremath{\F_2}-algebras.
%Let $\calA$ be the category of augmented commutative \ensuremath{\F_2}-algebras (or more generally any category of universal algebras). Quillen developed in \cite{QuillenHomAlg.pdf} a model structure on the category $s\calA$ of simplicial objects in $\calA$, and introduced the notion of the Andre-Quillen (co)homology of objects of $s\calA$. Quillen (co)homology appears to be of interest to the algebraists.
%An \emph{abelian object of $s\calA$} is an object $A\in s\calA$ along with a natural abelian group structure on $s\calA(\DASH,A)$. There is an evident category $\mathsf{ab}(s\calA)$ of abelian objects, and a left adjoint to the forgetful functor yielding an adjunction $s\calA\rightleftarrows\mathsf{ab}(s\calA)$. Andre-Quillen homology is supposed (in some generality) to be the left derived functors of \emph{abelianisation}, this left adjoint.
There, an analysis \cite[\S4]{MR1089001} of $\mathsf{ab}(s\calA)$ reveals that the abelian objects of $s\calA$ are precisely those simplicial algebras with levelwise trivial multiplication, and we can model the above adjunction as 
\[Q:s\calA\rightleftarrows s(v\ensuremath{\F_2}):K\]
where $v\ensuremath{\F_2}$ is the category of $\F_2$-vector spaces, $Q$ prolongs the indecomposables functor $Q:\calA\to v\ensuremath{\F_2}$, and $K$ prolongs the `square-zero' functor $K:v\ensuremath{\F_2}\to \calA$, under which $V$ maps to $\ensuremath{\F_2}\oplus V$ with trivial multiplication. Thus in our context, the Quillen homology $H_*^Q(A)$ of an object $A\in s\calA$ is $\mathbb{L}_*Q(A)=\pi_*Q(cA)$, where $c$ denotes a cofibrant replacement in $s\calA$. This coincides with those older definitions of Andr\'e-Quillen homology of interest in commutative algebra, and will be central in the following exposition.

\subsection*{Structure on cohomology and homotopy}
Let $S$ be the 
`symmetric algebra' monad on $v\ensuremath{\F_2}$, arising from the adjunction $\calA\rightleftarrows v\ensuremath{\F_2}$. Then objects of $\calA$ are precisely algebras for this monad. Now Dold's theorem \cite{DoldHomologySPs.pdf} shows that there is an endofunctor $\frakS$ of $n\ensuremath{\F_2}$ creating a commuting diagram
\[\xymatrix@R=4mm{
s(v\ensuremath{\F_2})\ar[r]^-{S}
\ar[d]^-{\pi_*}
&%r1c1
s(v\ensuremath{\F_2})\ar[d]^-{\pi_*}
\\%r1c2
n\ensuremath{\F_2}\ar[r]^-{\frakS}
&%r2c1
n\ensuremath{\F_2}%r2c2
}\]
Moreover, the naturality of Dold's result shows that $\frakS$ is also a monad, and $\pi_*$ sends (simplicial) $S$-algebras to $\frakS$-algebras. Thus the homotopy of an object of $\ensuremath{s\calA}$ is naturally an $\frakS$-algebra, and the category of $\frakS$-algebras is the richest value category for $\pi_*$ on $\ensuremath{s\calA}$. To justify this last claim, one calculates the homotopy of the `spheres' in $s\calA$, which represent homotopy --- the sphere `$S(n)\in\ensuremath{s\calA}$' (representing $I\pi_n:\ensuremath{s\calA}\to v\ensuremath{\F_2}$) is the image under $S$ of a $K(\ensuremath{\F_2},n)$: the free simplicial $F_2$-vector space on the standard sphere $S^n\in s\mathsf{Set}_*$. One finds that an $\frakS$-algebra is precisely a graded vector space with an action of the divided square operations of Bousfield \cite{BousOpnsDerFun.pdf,BousHomogFunctors.pdf}, Cartan \cite{CartanDivSquares} and Dwyer \cite{DwyerHtpyOpsSimpComAlg.pdf}.

One may next wonder what natural algebraic structure exists on $H^Q_*(A)$. It is more convenient to discuss the AQ cohomology $H_Q^*(A):=(H^Q_*(A))^*$. There are Eilenberg-Mac Lane objects in $ s\calA$ which represent AQ cohomology, and an analysis of the cohomology of products thereof, as performed in \cite{MR1089001}, reveals that the richest value category for $H_Q^*$ is $\calW$, whose definition we now recall.
An object of $\calW$ is a non-negatively graded \ensuremath{\F_2}-vector space $W^*$, along with:
\begin{enumerate}\squishlist
\setlength{\parindent}{.25in}
\item a Lie bracket $[\DASH,\DASH]:{W^n}\otimes{W^m}\to W^{n+m+1}$;
\item linear operations $P^i:W^n\to W^{n+i+1}$, satisfying certain `unstableness' axioms and `Adem relations'; and
\item a quadratic operation $\beta:W^0\to W^1$, which is a (partially defined) restriction for $[\DASH,\DASH]$; it satisfies those axioms which `$x\mapsto x^2 = \frac{1}{2}[x,x]$' might, were any of this to make sense.
\end{enumerate}

\subsection*{Quillen's spectral sequence}
In \cite{QuillenHCommRings.pdf} Quillen introduced a spectral sequence whose $E^1$ page depends functorially on the homology $H_*^QA$, converging to the homotopy groups $\pi_*A$.  More precisely:
\[E^1_{st}=\frakS_s(H_*^QA)\Rightarrow\pi_tA,\]
where we use the decomposition $\frakS=\bigoplus_{s\geq0}{\frakS_s}$ corresponding to the decomposition $S=\bigoplus_{s\geq0}{S_s}$ of the free symmetric algebra into lengths $s$.

This spectral sequence is straightforward to define. Take $A$ to be cofibrant, and filter by powers of the augmentation ideal, obtaining a filtered simplicial vector space. The associated graded simplicial vector space can be identified with $S_*(QA)$, and the definitions of $\frakS_*$ and $H_*^Q$ show that the associated spectral sequence has the desired $E^1$-term.

This spectral sequence is convergent when $A$ is connected (i.e.\ $\pi_0A\simeq\ensuremath{\F_2}$), and supports divided square operations (modulo indeterminacy) which reflect both those on $\pi_*A$ and those on $\frakS_s(H^Q_*A)_t$. It is the direct analogue of Curtis' unstable lower central series spectral sequence \cite{MR0184231}, which leaves us the task of defining an Adams spectral sequence.

\section*{An Adams spectral sequence}
One can form a classical unstable Adams spectral sequence using the Bousfield-Kan completion tower. This is the homotopy spectral sequence (cf.\ \cite{Bousfield-htpySS.pdf})   of the cosimplicial space $\F_2^\bullet X$ \cite{MR0365573} whose totalization is the \ensuremath{\F_2}-completion of $X$.

One might first ask how to replace $A\in \ensuremath{s\calA}$ with an appropriate cosimplicial object in $\ensuremath{s\calA}$, in order to imitate this process.  One could use the monad of the adjunction $Q:\ensuremath{s\calA}\rightleftarrows s(v\ensuremath{\F_2}):K$, but as the image of $K$ does not land within the cofibrant objects of \ensuremath{s\calA}, we would not be able to identify any resulting $E_2$-page with a functor of the (co)homology. [This problem was not present for the BK tower, as every simplicial set is cofibrant.]

In order to produce a homotopically correct resolution, one must mix intelligent cofibrant replacements into the iteration of this monad. This process is the subject of recent work of Blumberg and Riehl \cite{BlumRiehlResolutions.pdf}. They show that if one can choose a functorial cofibrant replacement $c$ on $\ensuremath{s\calA}$ which has the additional structure of a comonad%footnote:
\footnote{As every object of $s(v\ensuremath{\F_2})$ is fibrant, we need not mention a monadic fibrant replacement on that category, although there is no such asymmetry in the treatment of \cite{BlumRiehlResolutions.pdf}.}, then one can give a homotopically correct cosimplicial resolution of the type we desire. Precisely, they produce an augmented cosimplicial endofunctor of $\ensuremath{s\calA}$ (writing $\overline{Q}$ for $Qc:\ensuremath{s\calA}\to s(v\ensuremath{\F_2})$):
\[
\vcenter{
\def\labelstyle{\scriptstyle}
\xymatrix@C=1.5cm@1{
c\,
\ar[r]
&
\,cK\overline{Q}\,
\ar[r];[]
&
\,c(K\overline{Q})^2\,
\ar@<-1ex>[l];[]
\ar@<+1ex>[l];[]
\ar@<+1ex>[r];[]
\ar@<-1ex>[r];[]
&
\,c(K\overline{Q})^3\,
\ar[l];[]
\ar@<-2ex>[l];[]
\ar@<+2ex>[l];[]
\ar[r];[]
\ar@<+2ex>[r];[]
\ar@<-2ex>[r];[]
&
\,c(K\overline{Q})^4\,\makebox[0cm][l]{\,$\cdots $}
\ar@<-3ex>[l];[]
\ar@<-1ex>[l];[]
\ar@<+1ex>[l];[]
\ar@<+3ex>[l];[]
}}\]
For $A\in\ensuremath{s\calA}$, write $\underline{A}^\bullet$ for the cosimplicial object $c(K\overline{Q})^\bullet A$.
The Blumberg-Riehl approach augments Bousfield's work on cosimplicial resolutions \cite{BousCosimpResnHtpySS.pdf},  allowing us to define a homotopical `derived completion' endofunctor on $\ensuremath{s\calA}$ by $A\mapsto\hat A:=\textup{Tot}((\underline{A}^\bullet)^{\textup{rf}})$, where `$\textup{rf}$' denotes a Reedy fibrant replacement.

Bousfield shows how to obtain a spectral sequence $\pi^s\pi_t(\underline{A}^\bullet;S^0)\Rightarrow\pi_{t-s}(\hat A;S^0)$ by applying the well established Bousfield-Kan spectral sequence to the evident cosimplicial space of Dwyer-Kan mapping complexes. It serves our purposes best to dualize, and rewrite:
\[\pi_s\pi^t((Ic(K\overline{Q})^\bullet A)^*)\Rightarrow \pi^{t-s}((I\hat A)^*)\]
A complete derivation would at this point include a detailed examination of the dual Hurewicz homomorphism $H^*_Q(X)\to \pi^*((IX)^*)$, in particular when $X$ is an abelian object. One shows that:
\begin{prop*}
When $H^*_Q(A)$ is of finite type (for example when $A$ is connected and of finite type) the spectral sequence has $E_2^{st}=(\mathbb{L}_s\textup{Ind}_\calW)(H_Q^*A)^t=(\Ext^s_{\calW}(H_Q^*A,H_Q^*S^t))^*$.
\end{prop*}
\noindent I have not yet considered whether or not this spectral sequence converges.

One should be clear what is meant by the derived functors $\mathbb{L}_s\textup{Ind}_\calW$. $\calW$ is an example of what Blanc-Stover call a \emph{category of universal graded commutative algebras}, or `CUGA' \cite{Blanc_Stover-Groth_SS.pdf}. One notes that there is a free-forgetful adjunction $n\mathsf{Set}_*\rightleftarrows \calW$ ($n\mathsf{Set}_*$ being graded pointed sets), and the free objects provide enough projectives, so that $s\calW$ is a Quillen model category. Moreover, there is an evident `indecomposables' functor
\[\textup{Ind}_\calW:\calW\to n\ensuremath{\F_2}\]
which sends an object of $\calW$ to its vector space quotient by the image therein of all nontrivial operations. We mean the derived functors%footnote:
\footnote{Note that as every object in a CUGA is fibrant, any prolongation will preserve weak equivalence between cofibrant objects (which are just simplicial homotopy equivalences), so that any functor from a CUGA is left derivable.} of this functor.


\section*{A Grothendieck spectral sequence}
It was first noted by Haynes Miller in \cite{MillerSullivanConjecture.pdf} that composite functor spectral sequences can be of some utility in contexts such as ours. Thus one is led to consider a factorisation of the functor $\textup{Ind}_\calW$ through a third category, $\calL$, of graded Lie algebras.%footnote:
\footnote{An object of $\calL$ is a non-negatively graded \ensuremath{\F_2}-vector space $W^*$, with a Lie bracket ${W^n}\otimes{W^m}\to W^{n+m+1}$ satisfying the (non-redundant) axiom `$[x,x]=0$'.}

It is simply a matter of examining the definition (cf.\ \cite[p.17]{MR1089001}) of $\calW$ to see that $\textup{Ind}_\calW$ factors as
\[\calW\overset{\textup{Ind}_\textbf{P$\beta$}}{\to}\calL\overset{\textup{Ind}_{[]}}{\to}n\F_2\]
where $\textup{Ind}_\textbf{P$\beta$}$ and $\textup{Ind}_{[]}$ are the evident `indecomposables' functors killing $P^i$ and $\beta$ operations and Lie brackets respectively.%footnote:
\footnote{The factorisation $\textup{Ind}_{[]}\circ\textup{Ind}_{\textbf{P$\beta$}}$ represents a choice that I made some time ago, which may not be the most natural or most useful. Another alternative is to factor as $\textup{Ind}_{[]\beta}\circ\textup{Ind}_{\textbf{P}}$ through a category of Lie algebras with partially defined restriction.}

The category $\calL$ is itself a CUGA, and $\textup{Ind}_\textbf{P$\beta$}$ sends free objects in $\calW$ to free objects in $\calL$. In this context, Blanc and Stover have developed in \cite{Blanc_Stover-Groth_SS.pdf} a Grothendieck spectral sequence calculating $\mathbb{L}_*\textup{Ind}_\calW$. Before stating this spectral sequence, we must describe a little more of the Blanc-Stover theory.

\subsection*{Categories of $\Pi$-algebras}
Given a CUGA $\calC$, Blanc-Stover define a category $\calC'$ (which they denote $\calC$-$\Pi$-$\textup{Alg}$), which is the natural value category for the functor $\pi_*$ on $s\calC$. We may be more explicit as follows.

Observe that the homotopy groups of $X\in s\calC$ are in fact \textbf{bi}graded --- write $G_k\pi_nX$ for the $k^{\textup{th}}$ internally graded part of the $n^{\textup{th}}$ homotopy group of $X\in s\calC$. Then $G_k\pi_n$ is represented (in $\ho(s\calC)$) by $F(S^n(k))$, where the `sphere' $S^n(k)\in n(s\mathsf{Set}_*)$ is the graded pointed simplicial set which has a standard $n$-sphere in grading $k$ and a point in all other gradings.

As ever, the homotopy groups of $r$-fold wedges of spheres dictate the natural $r$-ary operations on the homotopy groups of objects of $s\calC$. These operations are subject to relations dictated by the data of various equations of maps under various compositions. Now an object of $\calC'$ is a bigraded vector space with an $r$-ary operation for every homotopy class in an $r$-fold wedge of spheres. Of course, we demand that the operations satisfy the appropriate relations.

Now if $T: \calC\to\calB$ is a functor between CUGAs, one can define a functor $\overline{T}:\calC'\to\calB'$. This is, of course, analogous to Dold's theorem. In the case where $\calC=\calB=v\ensuremath{\F_2}$, $\calC'=\calB'=n\ensuremath{\F_2}$, and $\overline{T}$ is precisely the functor associated with $T$ by Dold's theorem. We are now in position to present the promised Grothendieck spectral sequence.
\begin{prop*}
For any $W\in \calW$ there is a convergent spectral sequence:
\[(\mathbb{L}_p\overline{\textup{Ind}_{[]}}_q)(\mathbb{L}_*\textup{Ind}_\textbf{P$\beta$})(W)\Rightarrow (\mathbb{L}_{p+q}\textup{Ind}_\calW)(W).\]
\end{prop*}
\noindent 
The RHS is bigraded, while the LHS is trigraded.
The indices may seem a little confusing, but they needn't be. When interpreting the LHS, one thinks ``we're deriving a functor $\overline{\textup{Ind}_{[]}}:\calL'\to nn\ensuremath{\F_2}$ between categories of bigraded objects''.

\subsection*{The functor $\overline{\textup{Ind}_{[]}}:\calL'\to nn\ensuremath{\F_2}$}
\newcommand{\uLLa}{\mathsf{u}\Lambda\!^{+\!}\mathsf{La}}
Before making sense of the above spectral sequence, we must understand $\calL'$, and the functor $\overline{\textup{Ind}_{[]}}$. For this we introduce the category $\uLLa$ of `unstable $\Lambda^+$-Lie algebras'. An object of this category is a graded Lie algebra $L_*$ over \ensuremath{\F_2} (with axiom $[[x,x]]=0$, and no $+1$ shift in grading), along with right operations $\lambda_i:L_n\to L_{n+i}$ for $1\leq i\leq n$ (written $x\mapsto x\lambda_i$), such that:
\begin{enumerate}\squishlist
\setlength{\parindent}{.25in}
\item $\lambda_{\textup{top}}:=\lambda_n:L_n\to L_{2n}$ is a restriction for ($n>0$):
\begin{enumerate}\squishlist
\setlength{\parindent}{.25in}
\item $[[x\lambda_{\textup{top}},z]]=[[x,[[x,z]]]]$ for $|x|>0$, $|z|\geq0$
\item $\lambda_{\textup{top}}(x+y)=\lambda_{\textup{top}}(x)+\lambda_{\textup{top}}(y)+[[x,y]]$ for $|x|=|y|>0$
\end{enumerate}
\item the $\lambda_i:L_n\to L_{n+i}$ for $1\leq i< n$ are linear;
\item the $[[x\lambda_i,z]]=0$ for $1\leq i<|x|$.
\item the $\lambda_i$ operations satisfy the Adem relations of \cite{6Author.pdf}.
\end{enumerate}
One is able to see that in fact there is a Hilton-Milnor theorem%footnote:
\footnote{This comes from an essentially formal Hilton-Milnor decomposition for the coproduct of a finite number of free Lie algebras, where by `Lie algebra' we mean `Lie algebra over $\ensuremath{\F_2}$ subject to the extra axiom $[x,x]=0$'.} for homotopy groups of wedges of spheres in $s\calL$. This, along with the description of the homotopy groups of the free unrestricted Lie algebra given in \cite{6Author.pdf} can be used to demonstrate that $\calL'$ is a category of $+1$-shifted graded objects in $\uLLa$ (thus bigraded \ensuremath{\F_2}-Lie algebras with one of the gradings yielding a $+1$ shift, etc.).

%Moreover, the homotopy groups of a given sphere are essentially calculated by the six authors of \cite{6Author.pdf}. In particular, the elements of the homotopy groups of wedges of spheres are of the form ``$\lambda_{i_1}\lambda_{i_2}\cdots\lambda_{i_r}\ell$'' for $\ell$ an iterated Lie bracket of fundamental classes. Here the Lie bracket is $[[\DASH,\DASH]]$, the Eilenberg-Zilber Lie bracket on the homotopy of a simplicial Lie algebra.
%
%The details here haven't been worked out precisely. In particular, I don't know what an admissible sequence is supposed to be yet, and I don't know how the $\lambda$ operations squeeze outside the Lie brackets. However, one knows enough already to say that $\calL'$ is a category of bigraded objects subject to a bigraded Lie bracket%footnote:
%\footnote{(with a $+1$ shift in the internal grading)} and certain $\lambda_i$ operations.

At this point, it is not completely obvious what $\overline{\textup{Ind}_{[]}}$ will turn out to be, as demanded by the definitions of \cite{Blanc_Stover-Groth_SS.pdf}. However, it can be calculated that it simply kills the image of $[[\DASH,\DASH]]$ and of each $\lambda_i$ operation.

\subsection*{A third spectral sequence computing $\mathbb{L}_*\overline{\textup{Ind}_{[]}}$}
We have seen that $\calL'$ is the category of unstable $\Lambda^+$-Lie algebras with an internal grading, and $\overline{\textup{Ind}_{[]}}$ is the functor $\calL'\to nn\ensuremath{\F_2}$ which kills all $\lambda$ operations and brackets. In particular, we can factor this functor as
\[\calL'\overset{\textup{Ind}_{\lambda}}{\to}\textup{gr}{\calL}\overset{\textup{Ind}_{[[,]]}}{\to}nn\F_2\]
where (for present lack of notation) $\textup{gr}{\calL}$ is the category of bigraded Lie algebras with one of the gradings $+1$-shifted. Again, since $\textup{Ind}_\lambda$ preserves free objects, we obtain a convergent spectral sequence
\[(\mathbb{L}_p\overline{\textup{Ind}_{[[,]]}}_q)(\mathbb{L}_*\textup{Ind}_\lambda)(\Omega)\Rightarrow (\mathbb{L}_{p+q}\overline{\textup{Ind}_{[]}})(\Omega),\]
for any $\Omega\in\calL'$, for example $\Omega=(\mathbb{L}_*\textup{Ind}_\textbf{P$\beta$})(W)$. Now $\overline{\textup{Ind}_{[[,]]}}$ is no more difficult than $\overline{\textup{Ind}_{[]}}$, and this process can be iterated. One wonders if this has any potential use.


%It seems just possible at this moment that there is another Blanc-Stover-Grothendieck spectral sequence calculating the derived functors of $\overline{\textup{Ind}_{[]}}$. After all, given that the $\lambda_i$ can move out the the left, $\overline{\textup{Ind}_{[]}}$ \emph{may} factor as a composite in which the $\Lambda$ operations are killed first, and then the Lie brackets. If so, there \emph{may} be another Grothendieck spectral sequence. One of the terms in this \emph{third} spectral sequence would be Quillen homology of Lie algebras (I suspect), and thus conceptually easy to calculate. I suppose that if this worked out, this could be iterated literally forever. It's a strange thought.
\section*{Future directions}
\subsection*{A Koszul approach to the Grothendieck spectral sequence}
There may be a second approach to defining the above composite functor spectral sequence.
%
%
%Another question to be investigated was raised by Haynes some weeks ago, but hasn't been allocated much processor time up to this point. The question is whether or not the following can be made reality.
There is a commuting diagram:
\[\xymatrix@R=4mm{
\calW
\ar[r]^-{\textup{Ind}_{\textbf{P$\beta$}}}
\ar[d]
&%r1c1
\calL
\ar[r]
\ar[d]
&%r1c2
n\ensuremath{\F_2}
\\%r1c3
\calP
\ar[r]^-{\textup{Ind}_{\calP}}
&%r2c1
n\ensuremath{\F_2}
&%r2c2
%r2c3
}\]
Here, the category $\calP$ is obtained from $\calW$ by forgetting the Lie structure, and supports $P$ operations and the $\beta$ operation.

Now the derived functors of $\textup{Ind}_\calP$ might be thought of as $\Tor_\calP$ with a trivial object in $\calP$, and although some structure operations in $\calP$ are quadratic, not linear, one may hope that there is a Koszul calculation of these derived functors. Further, that some Koszul resolution can be naturally enriched with a Lie bracket and filtered in such a way that the required spectral sequence is just that associated with said filtration.


\subsection*{Operations in the Adams spectral sequence}
The Adams spectral sequence defined here converges to $(\pi_*(I\hat A))^*$, which is (modulo augmentations) dual to the homotopy of $\hat A$. As such, at least the abutment contains operations dual to the divided square operations, and a coalgebra structure. One wonders whether this structure can be extended to the $E^r$ pages for $r\geq2$. If so, it would be rather aesthetic if the structure on $E^2$ could be induced by some natural structure on $\mathbb{L}_*\textup{Ind}_{\calW}$ groups. We have yet to perform this investigation.
%The question comes up of how the coalgebra structure and dual $\delta$ operations to be found on the abutment of the Adams spectral sequence might be reflected on the Adams $E_2$, and moreover how this reflection might be reflected in the Grothendieck spectral sequence! 
%
%Question of how the resulting $\Lambda$ operations act on the $E_2$ page --- they should exhibit some duality with the $\delta$ operations. Hrrrrrmmmmmm... That's funny... Don't we kill the $\Lambda$ operations by applying $IndP$? Maybe this works just fine if you use the $\Lambda$-algebra approach ------ that's probably the point: use a filtration of the $\Lambda$ thing which gives what you want in homotopy AND whose filtered parts also do the right thing.

\subsection*{Possible applications}
As indicated above, there are a number of possible directions for this research, many of which I am at present unaware. Here I'll indicate a few possibilities.

In \cite{MR0413089}, Kan and Thurston show that every connected simplicial complex admits a homology isomorphism from a $K(G,1)$. An analogue in our context would replace $K(G,1)$ with `constant object of \ensuremath{s\calA}', and is yet to be investigated.

In \cite{PiAlgModuli.pdf}, Blanc, Dwyer and Goerss give an obstruction theory for the realisation of a $\Pi$-algebra as the homotopy $\Pi$-algebra of a topological space. One may search in this context for a similar theory, either for the realisation of a $\frakS$-algebra as the homotopy, or of an element of $\calW$ as the cohomology, of an element of $\ensuremath{s\calA}$. Again, this is work yet to be performed.

Interest (c.f.\ \cite{MR896094,MR733698}) has been shown in non-vanishing results for the cohomology of objects of \ensuremath{s\calA} with homotopy in only finitely many simplicial degrees, but various questions remain open for investigation.

It is possible to form a stable model category of `spectra in \ensuremath{s\calA}'. Having done this, one may be able to define a stable Adams spectral sequence, and a stable version of Quillen's fundamental spectral sequence. The topological analogues coincide up to a grading shift in the stable case, although the relationship between the unstable spectral sequences is not well understood. The corresponding investigation in the \ensuremath{\F_2}-algebra context to be more straightforward, but said analysis is yet to be performed.

Recent work of Haugseng and Miller \cite{RuneMillerCohomLoopSpace.pdf} contains a spectral for the cohomology of an infinite loop space, starting with certain derived functors applied to the stable cohomology. This spectral sequence may be related to the Adams spectral sequence via a map from the Dyer-Lashov algebra to the $\Lambda$ algebra. It would be interesting to see if this analysis has analogues in our context.

\subsection*{To do}
\begin{enumerate}\squishlist
\setlength{\parindent}{.25in}
\item Do all of Goerss' book in a more general context, including:
\item Check out Quillen's fundamental spectral sequence --- its $(E_1,d_1)$ depends on the homology, and maybe can be dualised to get something more comparable to the Adams SS I'm working on. In particular, I need to understand the $\frakS$ construction's interactions with dualisation in order to rewrite Goerss' formula for $d_1$, and then think a little about what in $E_1$ is hit by $d_1$, especially in the case of a sphere, where the two abutments are so well known. \textbf{Question:} I would think that the Quillen SS is degenerate for a sphere, from $E_1$. What about the Adams SS? It would be AMAZING if it had no differentials.
\end{enumerate}

\subsection*{Simplicial Lie algebras}
Do all this for simplicial Lie algebras, of the type I like. There should be an Adams SS defined in basically the same way, and a Quillen SS to compare it to. Note that I'll need to calculate (like Goerss on simplicial commutative algebras) the cohomology of Eilenberg-Mac Lane objects, in order to talk meaningfully about factorisations of the Hurewicz.



\printbibliography

\end{document}















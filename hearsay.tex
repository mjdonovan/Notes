% !TEX root = z_output/_hearsay.tex
%%%%%%%%%%%%%%%%%%%%%%%%%%%%%%%%%%%%%%%%%%%%%%%%%%%%%%%%%%%%%%%%%%%%%%%%%%%%%%%%
%%%%%%%%%%%%%%%%%%%%%%%%%%% 80 characters %%%%%%%%%%%%%%%%%%%%%%%%%%%%%%%%%%%%%%
%%%%%%%%%%%%%%%%%%%%%%%%%%%%%%%%%%%%%%%%%%%%%%%%%%%%%%%%%%%%%%%%%%%%%%%%%%%%%%%%
\documentclass[11pt]{article}
\usepackage{fullpage}
\usepackage{amsmath,amsthm,amssymb}
\usepackage{mathrsfs,nicefrac}
\usepackage{amssymb}
\usepackage{epsfig}
\usepackage[all,2cell]{xy}
\usepackage{sseq}
\usepackage{tocloft}
\usepackage{cancel}
\usepackage[strict]{changepage}
\usepackage{color}
\usepackage{tikz}
\usepackage{extpfeil}
\usepackage{version}
\usepackage{framed}
\definecolor{shadecolor}{rgb}{.925,0.925,0.925}

%\usepackage{ifthen}
%Used for disabling hyperref
\ifx\dontloadhyperref\undefined
%\usepackage[pdftex,pdfborder={0 0 0 [1 1]}]{hyperref}
\usepackage[pdftex,pdfborder={0 0 .5 [1 1]}]{hyperref}
\else
\providecommand{\texorpdfstring}[2]{#1}
\fi
%>>>>>>>>>>>>>>>>>>>>>>>>>>>>>>
%<<<        Versions        <<<
%>>>>>>>>>>>>>>>>>>>>>>>>>>>>>>
%Add in the following line to include all the versions.
%\def\excludeversion#1{\includeversion{#1}}

%>>>>>>>>>>>>>>>>>>>>>>>>>>>>>>
%<<<       Better ToC       <<<
%>>>>>>>>>>>>>>>>>>>>>>>>>>>>>>
\setlength{\cftbeforesecskip}{0.5ex}

%>>>>>>>>>>>>>>>>>>>>>>>>>>>>>>
%<<<      Hyperref mod      <<<
%>>>>>>>>>>>>>>>>>>>>>>>>>>>>>>

%needs more testing
\newcounter{dummyforrefstepcounter}
\newcommand{\labelRIGHTHERE}[1]
{\refstepcounter{dummyforrefstepcounter}\label{#1}}


%>>>>>>>>>>>>>>>>>>>>>>>>>>>>>>
%<<<  Theorem Environments  <<<
%>>>>>>>>>>>>>>>>>>>>>>>>>>>>>>
\ifx\dontloaddefinitionsoftheoremenvironments\undefined
\theoremstyle{plain}
\newtheorem{thm}{Theorem}[section]
\newtheorem*{thm*}{Theorem}
\newtheorem{lem}[thm]{Lemma}
\newtheorem*{lem*}{Lemma}
\newtheorem{prop}[thm]{Proposition}
\newtheorem*{prop*}{Proposition}
\newtheorem{cor}[thm]{Corollary}
\newtheorem*{cor*}{Corollary}
\newtheorem{defprop}[thm]{Definition-Proposition}
\newtheorem*{punchline}{Punchline}
\newtheorem*{conjecture}{Conjecture}
\newtheorem*{claim}{Claim}

\theoremstyle{definition}
\newtheorem{defn}{Definition}[section]
\newtheorem*{defn*}{Definition}
\newtheorem{exmp}{Example}[section]
\newtheorem*{exmp*}{Example}
\newtheorem*{exmps*}{Examples}
\newtheorem*{nonexmp*}{Non-example}
\newtheorem{asspt}{Assumption}[section]
\newtheorem{notation}{Notation}[section]
\newtheorem{exercise}{Exercise}[section]
\newtheorem*{fact*}{Fact}
\newtheorem*{rmk*}{Remark}
\newtheorem{fact}{Fact}
\newtheorem*{aside}{Aside}
\newtheorem*{question}{Question}
\newtheorem*{answer}{Answer}

\else\relax\fi

%>>>>>>>>>>>>>>>>>>>>>>>>>>>>>>
%<<<      Fields, etc.      <<<
%>>>>>>>>>>>>>>>>>>>>>>>>>>>>>>
\DeclareSymbolFont{AMSb}{U}{msb}{m}{n}
\DeclareMathSymbol{\N}{\mathbin}{AMSb}{"4E}
\DeclareMathSymbol{\Octonions}{\mathbin}{AMSb}{"4F}
\DeclareMathSymbol{\Z}{\mathbin}{AMSb}{"5A}
\DeclareMathSymbol{\R}{\mathbin}{AMSb}{"52}
\DeclareMathSymbol{\Q}{\mathbin}{AMSb}{"51}
\DeclareMathSymbol{\PP}{\mathbin}{AMSb}{"50}
\DeclareMathSymbol{\I}{\mathbin}{AMSb}{"49}
\DeclareMathSymbol{\C}{\mathbin}{AMSb}{"43}
\DeclareMathSymbol{\A}{\mathbin}{AMSb}{"41}
\DeclareMathSymbol{\F}{\mathbin}{AMSb}{"46}
\DeclareMathSymbol{\G}{\mathbin}{AMSb}{"47}
\DeclareMathSymbol{\Quaternions}{\mathbin}{AMSb}{"48}


%>>>>>>>>>>>>>>>>>>>>>>>>>>>>>>
%<<<       Operators        <<<
%>>>>>>>>>>>>>>>>>>>>>>>>>>>>>>
\DeclareMathOperator{\ad}{\textbf{ad}}
\DeclareMathOperator{\coker}{coker}
\renewcommand{\ker}{\textup{ker}\,}
\DeclareMathOperator{\End}{End}
\DeclareMathOperator{\Aut}{Aut}
\DeclareMathOperator{\Hom}{Hom}
\DeclareMathOperator{\Maps}{Maps}
\DeclareMathOperator{\Mor}{Mor}
\DeclareMathOperator{\Gal}{Gal}
\DeclareMathOperator{\Ext}{Ext}
\DeclareMathOperator{\Tor}{Tor}
\DeclareMathOperator{\Map}{Map}
\DeclareMathOperator{\Der}{Der}
\DeclareMathOperator{\Rad}{Rad}
\DeclareMathOperator{\rank}{rank}
\DeclareMathOperator{\ArfInvariant}{Arf}
\DeclareMathOperator{\KervaireInvariant}{Ker}
\DeclareMathOperator{\im}{im}
\DeclareMathOperator{\coim}{coim}
\DeclareMathOperator{\trace}{tr}
\DeclareMathOperator{\supp}{supp}
\DeclareMathOperator{\ann}{ann}
\DeclareMathOperator{\spec}{Spec}
\DeclareMathOperator{\SPEC}{\textbf{Spec}}
\DeclareMathOperator{\proj}{Proj}
\DeclareMathOperator{\PROJ}{\textbf{Proj}}
\DeclareMathOperator{\fiber}{F}
\DeclareMathOperator{\cofiber}{C}
\DeclareMathOperator{\cone}{cone}
\DeclareMathOperator{\skel}{sk}
\DeclareMathOperator{\coskel}{cosk}
\DeclareMathOperator{\conn}{conn}
\DeclareMathOperator{\colim}{colim}
\DeclareMathOperator{\limit}{lim}
\DeclareMathOperator{\ch}{ch}
\DeclareMathOperator{\Vect}{Vect}
\DeclareMathOperator{\GrthGrp}{GrthGp}
\DeclareMathOperator{\Sym}{Sym}
\DeclareMathOperator{\Prob}{\mathbb{P}}
\DeclareMathOperator{\Exp}{\mathbb{E}}
\DeclareMathOperator{\GeomMean}{\mathbb{G}}
\DeclareMathOperator{\Var}{Var}
\DeclareMathOperator{\Cov}{Cov}
\DeclareMathOperator{\Sp}{Sp}
\DeclareMathOperator{\Seq}{Seq}
\DeclareMathOperator{\Cyl}{Cyl}
\DeclareMathOperator{\Ev}{Ev}
\DeclareMathOperator{\sh}{sh}
\DeclareMathOperator{\intHom}{\underline{Hom}}
\DeclareMathOperator{\Frac}{frac}



%>>>>>>>>>>>>>>>>>>>>>>>>>>>>>>
%<<<   Cohomology Theories  <<<
%>>>>>>>>>>>>>>>>>>>>>>>>>>>>>>
\DeclareMathOperator{\KR}{{K\R}}
\DeclareMathOperator{\KO}{{KO}}
\DeclareMathOperator{\K}{{K}}
\DeclareMathOperator{\OmegaO}{{\Omega_{\Octonions}}}

%>>>>>>>>>>>>>>>>>>>>>>>>>>>>>>
%<<<   Algebraic Geometry   <<<
%>>>>>>>>>>>>>>>>>>>>>>>>>>>>>>
\DeclareMathOperator{\Spec}{Spec}
\DeclareMathOperator{\Proj}{Proj}
\DeclareMathOperator{\Sing}{Sing}
\DeclareMathOperator{\shfHom}{\mathscr{H}\textit{\!\!om}}
\DeclareMathOperator{\WeilDivisors}{{Div}}
\DeclareMathOperator{\CartierDivisors}{{CaDiv}}
\DeclareMathOperator{\PrincipalWeilDivisors}{{PrDiv}}
\DeclareMathOperator{\LocallyPrincipalWeilDivisors}{{LPDiv}}
\DeclareMathOperator{\PrincipalCartierDivisors}{{PrCaDiv}}
\DeclareMathOperator{\DivisorClass}{{Cl}}
\DeclareMathOperator{\CartierClass}{{CaCl}}
\DeclareMathOperator{\Picard}{{Pic}}
\DeclareMathOperator{\Frob}{Frob}


%>>>>>>>>>>>>>>>>>>>>>>>>>>>>>>
%<<<  Mathematical Objects  <<<
%>>>>>>>>>>>>>>>>>>>>>>>>>>>>>>
\newcommand{\sll}{\mathfrak{sl}}
\newcommand{\gl}{\mathfrak{gl}}
\newcommand{\GL}{\mbox{GL}}
\newcommand{\PGL}{\mbox{PGL}}
\newcommand{\SL}{\mbox{SL}}
\newcommand{\Mat}{\mbox{Mat}}
\newcommand{\Gr}{\textup{Gr}}
\newcommand{\Squ}{\textup{Sq}}
\newcommand{\catSet}{\textit{Sets}}
\newcommand{\RP}{{\R\PP}}
\newcommand{\CP}{{\C\PP}}
\newcommand{\Steen}{\mathscr{A}}
\newcommand{\Orth}{\textup{\textbf{O}}}

%>>>>>>>>>>>>>>>>>>>>>>>>>>>>>>
%<<<  Mathematical Symbols  <<<
%>>>>>>>>>>>>>>>>>>>>>>>>>>>>>>
\newcommand{\DASH}{\textup{---}}
\newcommand{\op}{\textup{op}}
\newcommand{\CW}{\textup{CW}}
\newcommand{\ob}{\textup{ob}\,}
\newcommand{\ho}{\textup{ho}}
\newcommand{\st}{\textup{st}}
\newcommand{\id}{\textup{id}}
\newcommand{\Bullet}{\ensuremath{\bullet} }
\newcommand{\sprod}{\wedge}

%>>>>>>>>>>>>>>>>>>>>>>>>>>>>>>
%<<<      Some Arrows       <<<
%>>>>>>>>>>>>>>>>>>>>>>>>>>>>>>
\newcommand{\nt}{\Longrightarrow}
\let\shortmapsto\mapsto
\let\mapsto\longmapsto
\newcommand{\mapsfrom}{\,\reflectbox{$\mapsto$}\ }
\newcommand{\bigrightsquig}{\scalebox{2}{\ensuremath{\rightsquigarrow}}}
\newcommand{\bigleftsquig}{\reflectbox{\scalebox{2}{\ensuremath{\rightsquigarrow}}}}

%\newcommand{\cofibration}{\xhookrightarrow{\phantom{\ \,{\sim\!}\ \ }}}
%\newcommand{\fibration}{\xtwoheadrightarrow{\phantom{\sim\!}}}
%\newcommand{\acycliccofibration}{\xhookrightarrow{\ \,{\sim\!}\ \ }}
%\newcommand{\acyclicfibration}{\xtwoheadrightarrow{\sim\!}}
%\newcommand{\leftcofibration}{\xhookleftarrow{\phantom{\ \,{\sim\!}\ \ }}}
%\newcommand{\leftfibration}{\xtwoheadleftarrow{\phantom{\sim\!}}}
%\newcommand{\leftacycliccofibration}{\xhookleftarrow{\ \ {\sim\!}\,\ }}
%\newcommand{\leftacyclicfibration}{\xtwoheadleftarrow{\sim\!}}
%\newcommand{\weakequiv}{\xrightarrow{\ \,\sim\,\ }}
%\newcommand{\leftweakequiv}{\xleftarrow{\ \,\sim\,\ }}

\newcommand{\cofibration}
{\xhookrightarrow{\phantom{\ \,{\raisebox{-.3ex}[0ex][0ex]{\scriptsize$\sim$}\!}\ \ }}}
\newcommand{\fibration}
{\xtwoheadrightarrow{\phantom{\raisebox{-.3ex}[0ex][0ex]{\scriptsize$\sim$}\!}}}
\newcommand{\acycliccofibration}
{\xhookrightarrow{\ \,{\raisebox{-.55ex}[0ex][0ex]{\scriptsize$\sim$}\!}\ \ }}
\newcommand{\acyclicfibration}
{\xtwoheadrightarrow{\raisebox{-.6ex}[0ex][0ex]{\scriptsize$\sim$}\!}}
\newcommand{\leftcofibration}
{\xhookleftarrow{\phantom{\ \,{\raisebox{-.3ex}[0ex][0ex]{\scriptsize$\sim$}\!}\ \ }}}
\newcommand{\leftfibration}
{\xtwoheadleftarrow{\phantom{\raisebox{-.3ex}[0ex][0ex]{\scriptsize$\sim$}\!}}}
\newcommand{\leftacycliccofibration}
{\xhookleftarrow{\ \ {\raisebox{-.55ex}[0ex][0ex]{\scriptsize$\sim$}\!}\,\ }}
\newcommand{\leftacyclicfibration}
{\xtwoheadleftarrow{\raisebox{-.6ex}[0ex][0ex]{\scriptsize$\sim$}\!}}
\newcommand{\weakequiv}
{\xrightarrow{\ \,\raisebox{-.3ex}[0ex][0ex]{\scriptsize$\sim$}\,\ }}
\newcommand{\leftweakequiv}
{\xleftarrow{\ \,\raisebox{-.3ex}[0ex][0ex]{\scriptsize$\sim$}\,\ }}

%>>>>>>>>>>>>>>>>>>>>>>>>>>>>>>
%<<<    xymatrix Arrows     <<<
%>>>>>>>>>>>>>>>>>>>>>>>>>>>>>>
\newdir{ >}{{}*!/-5pt/@{>}}
\newcommand{\xycof}{\ar@{ >->}}
\newcommand{\xycofib}{\ar@{^{(}->}}
\newcommand{\xycofibdown}{\ar@{_{(}->}}
\newcommand{\xyfib}{\ar@{->>}}
\newcommand{\xymapsto}{\ar@{|->}}

%>>>>>>>>>>>>>>>>>>>>>>>>>>>>>>
%<<<     Greek Letters      <<<
%>>>>>>>>>>>>>>>>>>>>>>>>>>>>>>
%\newcommand{\oldphi}{\phi}
%\renewcommand{\phi}{\varphi}
\let\oldphi\phi
\let\phi\varphi
\renewcommand{\to}{\longrightarrow}
\newcommand{\from}{\longleftarrow}
\newcommand{\eps}{\varepsilon}

%>>>>>>>>>>>>>>>>>>>>>>>>>>>>>>
%<<<  1st-4th & parentheses <<<
%>>>>>>>>>>>>>>>>>>>>>>>>>>>>>>
\newcommand{\first}{^\text{st}}
\newcommand{\second}{^\text{nd}}
\newcommand{\third}{^\text{rd}}
\newcommand{\fourth}{^\text{th}}
\newcommand{\ZEROTH}{$0^\text{th}$ }
\newcommand{\FIRST}{$1^\text{st}$ }
\newcommand{\SECOND}{$2^\text{nd}$ }
\newcommand{\THIRD}{$3^\text{rd}$ }
\newcommand{\FOURTH}{$4^\text{th}$ }
\newcommand{\iTH}{$i^\text{th}$ }
\newcommand{\jTH}{$j^\text{th}$ }
\newcommand{\nTH}{$n^\text{th}$ }

%>>>>>>>>>>>>>>>>>>>>>>>>>>>>>>
%<<<    upright commands    <<<
%>>>>>>>>>>>>>>>>>>>>>>>>>>>>>>
\newcommand{\upcol}{\textup{:}}
\newcommand{\upsemi}{\textup{;}}
\providecommand{\lparen}{\textup{(}}
\providecommand{\rparen}{\textup{)}}
\renewcommand{\lparen}{\textup{(}}
\renewcommand{\rparen}{\textup{)}}
\newcommand{\Iff}{\emph{iff} }

%>>>>>>>>>>>>>>>>>>>>>>>>>>>>>>
%<<<     Environments       <<<
%>>>>>>>>>>>>>>>>>>>>>>>>>>>>>>
\newcommand{\squishlist}
{ %\setlength{\topsep}{100pt} doesn't seem to do anything.
  \setlength{\itemsep}{.5pt}
  \setlength{\parskip}{0pt}
  \setlength{\parsep}{0pt}}
\newenvironment{itemise}{
\begin{list}{\textup{$\rightsquigarrow$}}
   {  \setlength{\topsep}{1mm}
      \setlength{\itemsep}{1pt}
      \setlength{\parskip}{0pt}
      \setlength{\parsep}{0pt}
   }
}{\end{list}\vspace{-.1cm}}
\newcommand{\INDENT}{\textbf{}\phantom{space}}
\renewcommand{\INDENT}{\rule{.7cm}{0cm}}

\newcommand{\itm}[1][$\rightsquigarrow$]{\item[{\makebox[.5cm][c]{\textup{#1}}}]}


%\newcommand{\rednote}[1]{{\color{red}#1}\makebox[0cm][l]{\scalebox{.1}{rednote}}}
%\newcommand{\bluenote}[1]{{\color{blue}#1}\makebox[0cm][l]{\scalebox{.1}{rednote}}}

\newcommand{\rednote}[1]
{{\color{red}#1}\makebox[0cm][l]{\scalebox{.1}{\rotatebox{90}{?????}}}}
\newcommand{\bluenote}[1]
{{\color{blue}#1}\makebox[0cm][l]{\scalebox{.1}{\rotatebox{90}{?????}}}}


\newcommand{\funcdef}[4]{\begin{align*}
#1&\to #2\\
#3&\mapsto#4
\end{align*}}

%\newcommand{\comment}[1]{}

%>>>>>>>>>>>>>>>>>>>>>>>>>>>>>>
%<<<       Categories       <<<
%>>>>>>>>>>>>>>>>>>>>>>>>>>>>>>
\newcommand{\Ens}{{\mathscr{E}ns}}
\DeclareMathOperator{\Sheaves}{{\mathsf{Shf}}}
\DeclareMathOperator{\Presheaves}{{\mathsf{PreShf}}}
\DeclareMathOperator{\Psh}{{\mathsf{Psh}}}
\DeclareMathOperator{\Shf}{{\mathsf{Shf}}}
\DeclareMathOperator{\Varieties}{{\mathsf{Var}}}
\DeclareMathOperator{\Schemes}{{\mathsf{Sch}}}
\DeclareMathOperator{\Rings}{{\mathsf{Rings}}}
\DeclareMathOperator{\AbGp}{{\mathsf{AbGp}}}
\DeclareMathOperator{\Modules}{{\mathsf{\!-Mod}}}
\DeclareMathOperator{\fgModules}{{\mathsf{\!-Mod}^{\textup{fg}}}}
\DeclareMathOperator{\QuasiCoherent}{{\mathsf{QCoh}}}
\DeclareMathOperator{\Coherent}{{\mathsf{Coh}}}
\DeclareMathOperator{\GSW}{{\mathcal{SW}^G}}
\DeclareMathOperator{\Burnside}{{\mathsf{Burn}}}
\DeclareMathOperator{\GSet}{{G\mathsf{Set}}}
\DeclareMathOperator{\FinGSet}{{G\mathsf{Set}^\textup{fin}}}
\DeclareMathOperator{\HSet}{{H\mathsf{Set}}}
\DeclareMathOperator{\Cat}{{\mathsf{Cat}}}
\DeclareMathOperator{\Fun}{{\mathsf{Fun}}}
\DeclareMathOperator{\Orb}{{\mathsf{Orb}}}
\DeclareMathOperator{\Set}{{\mathsf{Set}}}
\DeclareMathOperator{\sSet}{{\mathsf{sSet}}}
\DeclareMathOperator{\Top}{{\mathsf{Top}}}
\DeclareMathOperator{\GSpectra}{{G-\mathsf{Spectra}}}
\DeclareMathOperator{\Lan}{Lan}
\DeclareMathOperator{\Ran}{Ran}

%>>>>>>>>>>>>>>>>>>>>>>>>>>>>>>
%<<<     Script Letters     <<<
%>>>>>>>>>>>>>>>>>>>>>>>>>>>>>>
\newcommand{\scrQ}{\mathscr{Q}}
\newcommand{\scrW}{\mathscr{W}}
\newcommand{\scrE}{\mathscr{E}}
\newcommand{\scrR}{\mathscr{R}}
\newcommand{\scrT}{\mathscr{T}}
\newcommand{\scrY}{\mathscr{Y}}
\newcommand{\scrU}{\mathscr{U}}
\newcommand{\scrI}{\mathscr{I}}
\newcommand{\scrO}{\mathscr{O}}
\newcommand{\scrP}{\mathscr{P}}
\newcommand{\scrA}{\mathscr{A}}
\newcommand{\scrS}{\mathscr{S}}
\newcommand{\scrD}{\mathscr{D}}
\newcommand{\scrF}{\mathscr{F}}
\newcommand{\scrG}{\mathscr{G}}
\newcommand{\scrH}{\mathscr{H}}
\newcommand{\scrJ}{\mathscr{J}}
\newcommand{\scrK}{\mathscr{K}}
\newcommand{\scrL}{\mathscr{L}}
\newcommand{\scrZ}{\mathscr{Z}}
\newcommand{\scrX}{\mathscr{X}}
\newcommand{\scrC}{\mathscr{C}}
\newcommand{\scrV}{\mathscr{V}}
\newcommand{\scrB}{\mathscr{B}}
\newcommand{\scrN}{\mathscr{N}}
\newcommand{\scrM}{\mathscr{M}}

%>>>>>>>>>>>>>>>>>>>>>>>>>>>>>>
%<<<     Fractur Letters    <<<
%>>>>>>>>>>>>>>>>>>>>>>>>>>>>>>
\newcommand{\frakQ}{\mathfrak{Q}}
\newcommand{\frakW}{\mathfrak{W}}
\newcommand{\frakE}{\mathfrak{E}}
\newcommand{\frakR}{\mathfrak{R}}
\newcommand{\frakT}{\mathfrak{T}}
\newcommand{\frakY}{\mathfrak{Y}}
\newcommand{\frakU}{\mathfrak{U}}
\newcommand{\frakI}{\mathfrak{I}}
\newcommand{\frakO}{\mathfrak{O}}
\newcommand{\frakP}{\mathfrak{P}}
\newcommand{\frakA}{\mathfrak{A}}
\newcommand{\frakS}{\mathfrak{S}}
\newcommand{\frakD}{\mathfrak{D}}
\newcommand{\frakF}{\mathfrak{F}}
\newcommand{\frakG}{\mathfrak{G}}
\newcommand{\frakH}{\mathfrak{H}}
\newcommand{\frakJ}{\mathfrak{J}}
\newcommand{\frakK}{\mathfrak{K}}
\newcommand{\frakL}{\mathfrak{L}}
\newcommand{\frakZ}{\mathfrak{Z}}
\newcommand{\frakX}{\mathfrak{X}}
\newcommand{\frakC}{\mathfrak{C}}
\newcommand{\frakV}{\mathfrak{V}}
\newcommand{\frakB}{\mathfrak{B}}
\newcommand{\frakN}{\mathfrak{N}}
\newcommand{\frakM}{\mathfrak{M}}

\newcommand{\frakq}{\mathfrak{q}}
\newcommand{\frakw}{\mathfrak{w}}
\newcommand{\frake}{\mathfrak{e}}
\newcommand{\frakr}{\mathfrak{r}}
\newcommand{\frakt}{\mathfrak{t}}
\newcommand{\fraky}{\mathfrak{y}}
\newcommand{\fraku}{\mathfrak{u}}
\newcommand{\fraki}{\mathfrak{i}}
\newcommand{\frako}{\mathfrak{o}}
\newcommand{\frakp}{\mathfrak{p}}
\newcommand{\fraka}{\mathfrak{a}}
\newcommand{\fraks}{\mathfrak{s}}
\newcommand{\frakd}{\mathfrak{d}}
\newcommand{\frakf}{\mathfrak{f}}
\newcommand{\frakg}{\mathfrak{g}}
\newcommand{\frakh}{\mathfrak{h}}
\newcommand{\frakj}{\mathfrak{j}}
\newcommand{\frakk}{\mathfrak{k}}
\newcommand{\frakl}{\mathfrak{l}}
\newcommand{\frakz}{\mathfrak{z}}
\newcommand{\frakx}{\mathfrak{x}}
\newcommand{\frakc}{\mathfrak{c}}
\newcommand{\frakv}{\mathfrak{v}}
\newcommand{\frakb}{\mathfrak{b}}
\newcommand{\frakn}{\mathfrak{n}}
\newcommand{\frakm}{\mathfrak{m}}

%>>>>>>>>>>>>>>>>>>>>>>>>>>>>>>
%<<<  Caligraphic Letters   <<<
%>>>>>>>>>>>>>>>>>>>>>>>>>>>>>>
\newcommand{\calQ}{\mathcal{Q}}
\newcommand{\calW}{\mathcal{W}}
\newcommand{\calE}{\mathcal{E}}
\newcommand{\calR}{\mathcal{R}}
\newcommand{\calT}{\mathcal{T}}
\newcommand{\calY}{\mathcal{Y}}
\newcommand{\calU}{\mathcal{U}}
\newcommand{\calI}{\mathcal{I}}
\newcommand{\calO}{\mathcal{O}}
\newcommand{\calP}{\mathcal{P}}
\newcommand{\calA}{\mathcal{A}}
\newcommand{\calS}{\mathcal{S}}
\newcommand{\calD}{\mathcal{D}}
\newcommand{\calF}{\mathcal{F}}
\newcommand{\calG}{\mathcal{G}}
\newcommand{\calH}{\mathcal{H}}
\newcommand{\calJ}{\mathcal{J}}
\newcommand{\calK}{\mathcal{K}}
\newcommand{\calL}{\mathcal{L}}
\newcommand{\calZ}{\mathcal{Z}}
\newcommand{\calX}{\mathcal{X}}
\newcommand{\calC}{\mathcal{C}}
\newcommand{\calV}{\mathcal{V}}
\newcommand{\calB}{\mathcal{B}}
\newcommand{\calN}{\mathcal{N}}
\newcommand{\calM}{\mathcal{M}}

%>>>>>>>>>>>>>>>>>>>>>>>>>>>>>>
%<<<<<<<<<DEPRECIATED<<<<<<<<<<
%>>>>>>>>>>>>>>>>>>>>>>>>>>>>>>

%%% From Kac's template
% 1-inch margins, from fullpage.sty by H.Partl, Version 2, Dec. 15, 1988.
%\topmargin 0pt
%\advance \topmargin by -\headheight
%\advance \topmargin by -\headsep
%\textheight 9.1in
%\oddsidemargin 0pt
%\evensidemargin \oddsidemargin
%\marginparwidth 0.5in
%\textwidth 6.5in
%
%\parindent 0in
%\parskip 1.5ex
%%\renewcommand{\baselinestretch}{1.25}

%%% From the net
%\newcommand{\pullbackcorner}[1][dr]{\save*!/#1+1.2pc/#1:(1,-1)@^{|-}\restore}
%\newcommand{\pushoutcorner}[1][dr]{\save*!/#1-1.2pc/#1:(-1,1)@^{|-}\restore}










\includeversion{FirstTen}


\title{Hearsay}
\author{}
\date{}
% >>>

%% >>> header

\begin{document}
\tableofcontents
%\newcommand{\HSsection}[2]
%{\section{#1 \protect\ifthenelse{\protect\equal{#2}{}}{}{\small{(#2)}}}}
%%%WHAT'S WRONG WITH THIS
\newcommand{\HSsection}[2]{\section{#1 }}
%\ifthenelse{\equal{#2}{}}{}{\small{(#2)}}}}

\section*{Preface}
This file is intended only as a repository for mathematical facts I picked up in
conversation, and found interesting. I keep this file so that I don't forget
\emph{everything} that I've ever been told.

Any given entry is likely to be incorrect --- things I've given thought to live in \verb|_notes.pdf|.

Anybody who happens to be reading this, upon finding an
error, should let me know. If you are the person who told me something in here,
and would like to be credited, let me know.\footnote{I'm not doing this
for now, since stuff in here is likely to be incorrectly transcribed, and nobody
wants their name on something that's wrong.}

\pagebreak
\begin{FirstTen}
\HSsection{\texorpdfstring{$H_*(X)=\pi_*(\textup{Sp}^\infty X)$ when $X$}
{HX=pi(SpX) when X} is connected}{29/6/11}
Let $X$ be a connected space. Let $\textup{Sp}^\infty X$ be the free topological
commutative monoid on $X$, where the basepoint of $X$ acts as the identity. This
is the quotient of the james construction $J(X)$ by the infinite permutation
group $\Sigma_\infty$. Let $\overline\Z X$ be the free topological abelian group
on $X$ with basepoint the identity. Then $\textup{Sp}^\infty X\to \overline\Z X$
is a weak equivalence (iff $X$ is connected).

Moreover, there is a homotopy equivalence between the simplicial abelian groups
$\textup{Sing}\overline\Z X$ and $\Z\textup{Sing}X$. Thus $\pi_*(\overline\Z
X)=\pi_*(\Z\textup{Sing}X)$. Finally, it is a fact that given a simplicial
abelian group, its homotopy groups and its homology groups coincide!


\HSsection{\texorpdfstring{$Q$}{Q} turns cofiber sequences into fiber sequences}
{29/6/11}
This corresponds to the fact that $\overline\pi_*^s=\pi_*Q$ is a reduced
homology theory. $\overline\pi_*^s$ should give you a LES for a cofiber
sequence, but $\pi_*$ likes fiber sequences! Here, $QX:=\bigcup\Omega^i\Sigma^i
X$.

Now suppose that $X\to Y\to Z$ is a cofiber sequence of $n$-connected spaces.
Let $F\to Y$ be the homotopy fiber of $Y\to Z$. We obtain the a Serre long exact
sequence from the Serre spectral sequence, which starts (approximately) at $2n$:
\[\xymatrix{
H_{2n}(F)\ar[r]&H_{2n}(Y)\ar[r]&H_{2n}(Z)\ar[r]^{\tau\ \ }&
H_{2n-1}(F)\ar[r]&\cdots
}\]
Now there is an induced map $\epsilon:X\to F$, so we have arrows between the two
long exact sequences:
\[\xymatrix{
H_{2n}(X)\ar[r]\ar[d]^\epsilon&H_{2n}(Y)\ar[r]\ar@{=}[d]&
H_{2n}(Z)\ar[r]^{\ \ }\ar@{=}[d]&H_{2n-1}(X)\ar[r]\ar[d]^\epsilon&\cdots\\
H_{2n}(F)\ar[r]&H_{2n}(Y)\ar[r]&H_{2n}(Z)\ar[r]^{\tau\ \ }&
H_{2n-1}(F)\ar[r]&\cdots
}\]
The five lemma will show that $\epsilon$ is an isomorphism on $H_{i}$ for $i$ at
most (around) $2n$, as long as we can show that the rightmost square 
commutes\,\ldots
\HSsection{Loads of papers online}{10/8/11}
\url{http://www.maths.ed.ac.uk/~aar/cat}
\HSsection{Perfect fields}{30/8/11}
A field is said to be perfect if all of its extensions are separable. Any
characteristic zero field is perfect (see below). All finite fields are
obviously perfect, as the frobenius is the identity on $\F_p$. An example of an
infinite perfect field of positive characteristic is
$\F_p(x,x^{1/p},x^{1/p^2},x^{1/p^3},\ldots)$.

\begin{fact*}If $k$ has characteristic $p>0$, then it
is perfect \Iff the Frobenius is an isomorphism.\end{fact*}
\begin{proof}
Suppose that the Frobenius fails to be surjective. Then some element of $k$ has
no $p^\text{th}$ root (any $p^\text{th}$ root is unique). Adjoining the root
gives a non-separable extension.

For the converse, suppose that $F/E$ is an inseparable extension, but that
$\Frob_E$ is an isomorphism, for a contradiction. Choose an element $\alpha$ of
$F$ whose minimal polynomial $f$ has a repeated root. Now unless $f'=0$, $f'$
and $f$ share a factor, and $\gcd(f',f)$ is a polynomial with root $\alpha$, of
strictly lower degree than $f$. This is impossible, so that $f'=0$, and we
conclude that $f(x)=g(x^p)$. (This is why any extension is separable in
characteristic zero).

Now, let $\Gamma=E[\alpha^p]$. Then $\alpha\notin\Gamma$, so that $\Frob_\Gamma$
is not surjective. To see that $\alpha\notin\Gamma$: if it were, there would be
a polynomial $h(x)\in E[x]$ such that $h(\alpha^p)=\alpha$. We can assume that
$\deg(h)$ is less than $\deg(g)$, since $g(\alpha^p)=0$, but then as $\alpha$ is
a root of $h(x^p)-x$, we must have $h(x^p)=x$, which is absurd.

Now as $\Frob_E$ is an isomorphism, the image of $\Frob_\Gamma$ is a finite
dimensional $E$-vector space. Moreover, although $\Frob_\Gamma$ is not
$E$-linear, it does preserve $E$-linearly independent sets. Taking a basis for
$\Gamma$, its image under $\Frob_\Gamma$ is linearly independent, so
$\Frob_\Gamma(\Gamma)$ has the same dimension as $\Gamma$. Thus
$\Frob_\Gamma=\Gamma$, contradicting the fact that $\Frob_\Gamma$ is not
surjective.
\end{proof}

\HSsection{Some number theory}{30/8/11}
Suppose you have a local field $F$ of characteristic zero. Then the integral
closure of $\Z\subset F$ is a ring $\calO_F$ called the ring of integers of $F$.
We'll work with local fields such that $\calO_F$ is a complete local ring.
We write $\frakm_F$ for the maximal ideal.

Now suppose that $E$ is an extension of $F$ such that $\calO_E$ is, like, a DVR
or something, so that I can understand how to calculate the ramification degree.
Then we have a commuting diagram:
\[\xymatrix{
E&\calO_E\ar@{_{(}->}[l]&\frakm_E\ar@{_{(}->}[l]&\calO_E/\frakm_E\\
F\ar[u]&\calO_F\ar@{_{(}->}[l]\ar[u]&\frakm_F\ar@{_{(}->}
[l]\ar[u]&\calO_F/\frakm_F\ar[u]
}\]
Now the image of $\frakm_F$ in $\calO_E$ generates an ideal, necessarily
contained in $\frakm_E$. As $\calO_E$ is a DVR, this ideal is $\frakm_E^r$ for
some $r$, which is the ramification degree. $E/F$ is unramified if $r=1$.

\begin{thm*}
Suppose that $k$ is a perfect field of characteristic $p$. Then there exists a
ring $R$, unique up to unique isomorphism \lparen over $k$\rparen, such that:
\begin{itemize}\squishlist
\item $R$ is local and complete with maximal ideal $\frakm$;
\item $\frakm$ is generated by $p$ \lparen the sum $1+\cdots+1$\rparen; and
\item $R/\frakm\cong k$.
\end{itemize}
This is the \textbf{Witt ring} $W(k)$ of $k$.

In fact, $R$ is an initial object in the category of local complete rings with a
map to $k$.
\end{thm*}
\noindent To interpret this, think of the spectrum of a local complete ring as
somehow a small yet complete neighbourhood of a single closed point. Then $\Spec
R$ is a universal deformation of a universal $k$-point. Any other small
neighbourhood of a $k$-point maps into $\Spec R$. That is, $\Spec R$ contains
every possible deformation of a $k$-point.
\begin{fact*}
Suppose that $E/\Q_p$ is a finite extension. Then $[E:\Q_p]$ is the product of
the residue degree and the ramification degree.
\end{fact*}
\begin{cor*}
There is a unique unramified extension $E$ of $\Q_p$ of a given degree.
\end{cor*}
\begin{proof}
If $E/\Q_p$ is unramified of degree $d$, we get a diagram:
\[\xymatrix{
E&\calO_E\ar@{_{(}->}[l]&\frakm_E=p\calO_E\ar@{_{(}->}[l]&\calO_E/\frakm_E\\
\Q_p\ar[u]&\Z_p\ar@{_{(}->}[l]\ar[u]&p\Z_p\ar@{_{(}->}
[l]\ar[u]&\Z/p\Z\ar[u]
}\]
Now $[\calO_E/\frakm_E:\Z/p\Z]=d$ (see fact), and $\calO_E/\frakm_E$ is a
perfect field. Thus its Witt ring $R$ must be $\calO_E$. Thus $E$ is determined
as the field of fractions of the Witt ring of the field of $p^d$ elements.
\end{proof}
\HSsection{On the \texorpdfstring{$\Proj$}{Proj} construction}{30/8/11}
It is pointed out that an action of $k^*$ on a $k$-algebra $A$ corresponds to a
grading, under some (as yet unknown) circumstances. Given a grading, $k^*$ acts
by $x\cdot s=x^{|s|}s$. Given an action, it splits up into representations, all
of which are of this form.

On taking quotients, given an action $a:G\times X\to X$, the (categorical)
quotient should be the coequaliser:
\[\xymatrix{
G\times X\ar@<.5ex>[r]^{\ \ \ a}\ar@<-.5ex>[r]_{\ \ \ \pi_X}&X\ar[r]&X/G
}\]
The point is, maybe if you're trying to take the quotient of $\A^{n+1}$ by the
action of $k^*=\Spec k[x^{\pm1}]$, you should throw away the origin. However,
maybe if you take this categorical quotient in the category of, say, separated
schemes, that happens automatically, as the nasty point just doesn't belong in
an object of this category. \textbf{Actually, this is not true --- where is the origin supposed to map to in the coequaliser?}

One should see a way that the $\Proj$ construction is doing all this for us.
Presumably, this has something to do with the fact that the homogeneous ideals
are exactly those which are $k^*$-invariant. Maybe that means that by
restricting to these ideals you get generic points of invariant closed
subspaces\ldots\ how would I know?
\HSsection{The Weil Conjectures}{1/9/11}
Suppose that $X$ is a variety over $\Z$, and $p$ is a prime. Then it has
finitely many $\F_{p^n}$-points for each $n$, so we can form the generating
function
\[\sum_{n=1}^\infty|X(\F_{p^n})|.\]
One might conjecture that this is a rational function. To prove such a
conjecture, somebody showed that this function is of the form:
\[\prod_{d}\left(\chi(\Frob,H^d(X))\right)^{(-1)^d}\]
Here, $\chi(\Frob,H^d(X))$ is the characteristic polynomial of the Frobenius
acting on $H^d(X)$, the \'etale cohomology. It's over some field of different
characteristic, or somesuch.

Anyway, somebody recently claimed to me that you still get a rational function
even if instead of counting $\F_{p^n}$ points, you count $\Z/p^n\Z$ points!
Weirder still!
\HSsection{Homotopy excision; the Blakers-Massey theorem}{4/9/11}
Given a CW-complex $X$ and a CW-decomposition $X=A\cup B$, there is a map of
pair $(A,A\cap B)\to (X,B)$. The excision axiom for an unreduced homology theory
$h_*$ states that $h_*(A,A\cap B)\to h_*(X,B)$ is an isomorphism.

Now $\pi_*^s$, stable homotopy, is a homology theory, and thus satisfies
excision. However, unstable homotopy does not satisfy excision. The fact that it
does so in a range is referred to as homotopy excision.

To set it up, we'll say that a map $f:Z\to Y$ is $n$-connected if one of the
following equivalent conditions holds:
\begin{itemise}
\item $\pi_i(Y,Z)=0$ for $i\leq n$
\item $\pi_i(Z)\to\pi_i(Y)$ is epi for $i\leq n$, iso for $i<n$.
\item The homotopy fiber $F(f)$ is $(n-1)$-connected ($\pi_i(F(f))=0$ for
$i<n$).
\end{itemise}
{\small (\emph{Maybe there's actually some weird complication. I should read it
all more carefully some time. This probably works when $Y,Z$ are both
connected.})} Anyway, Hatcher tells us the following, in theorem 4.23, which is
known as the Blakers-Massey theorem:
\begin{thm*}Let $X$ be a CW-complex decomposed as the union of subcomplexes $A$
and $B$ with nonempty \emph{connected} intersection. If $(„A,A\cap B)$… is
$m$-connected and $„(B,A\cap B)$… is $n$-connected, then the map
$\pi_i(„A,A\cap B)\to\pi_i(„X,B)$ is an isomorphism for $i<m+‚n$ and a
surjection for $i= m+n$. That is, the map on homotopy fibers is
$m+n-1$-connected.
\end{thm*}

In the following diagram, the underlined labels on certain maps refer to
connectivities. Specifically, a map labeled $\underline r$ is to be
$r$-connected. Then we summarise:
\[\xymatrix{
F\ar@{-->}[r]_{\underline{m+n-1}}\ar@{-->}[d]&F'\ar@{-->}[d]\\
A\cap B\ar[d]^{\underline{m}}\ar[r]_{\underline{{n}}}&B\ar[d]\\
A\ar[r]&X
}\]
Anyway, here's a generalisation of all this. Let $V$ be the homotopy pullback of
the maps $A\rightarrow X\leftarrow B$. Then we have a diagram as follows, and
the induced arrow is $m+n-1$-connected:
\[\xymatrix{
V\ar@/_1pc/[ddr]\ar@/^1.5pc/[rrd]\\
&A\cap B\ar[ul]_(.3){\underline{m+n-1}}\ar[d]^{\underline{m}}
\ar[r]_{\underline{{n}}}&B\ar[d]\\
&A\ar[r]&X
}\]
For an example, suppose that $Y$ is an $(n-1)$-connected space. Then $Y\to CY$
is an $n$-connected map, so that:
\[\xymatrix{
\Omega\Sigma Y\ar@/_1pc/[ddr]\ar@/^1.5pc/[rrd]\\
&Y\ar[ul]_(.3){\underline{2n-1}}\ar[d]^{\underline{n}}
\ar[r]_{\underline{{n}}}&CY\ar[d]\\
&CY\ar[r]&X
}\]
This is consistent with what we know from the Bott-Samelson theorem (when
everything is free over $\Z$). The cofiber of $Y\to \Omega\Sigma Y$ is
$(2n-1)$-connected (is this exactly right?) (By the standard trick --- the Serre
exact sequence from the SSS gives us $H_*$ isos that we need.) (it's simply
connected, just by noting the dimensions of cells), so that one obtains a
homology isomorphism up to degree $2n-1$, and an injection at degree $2n$.

By the way, this should all have been written in terms of the connectivity of
the fibers. Labeling a map $\overline r$ when its fibre is $r$-connected, we
have
\[
\xymatrix{
F\ar@{-->}[r]_{\overline{m+n}}\ar@{-->}[d]&F'\ar@{-->}[d]\\
A\cap B\ar[d]^{\overline{m}}\ar[r]_{\overline{{n}}}&B\ar[d]\\
A\ar[r]&X
}\raisebox{-1.26cm}{\qquad and\qquad }\xymatrix{
V\ar@/_1pc/[ddr]\ar@/^1.5pc/[rrd]\\
&A\cap B\ar[ul]_(.3){\overline{m+n}}\ar[d]^{\overline{m}}
\ar[r]_{\overline{{n}}}&B\ar[d]\\
&A\ar[r]&X
}\]
By the way, that one fact is the generalisation of the other is not clear. Note
that the homotopy pullback is the homotopy fibre of the map $A\times B\to B$.
Somebody claimed that one can use the resulting long exact sequence and the long
exact sequences for the pairs $(A,V)$ and $(X,B)$ to see that
$\pi_i(A,V)\cong\pi_i(X,B)$ for all $i$, and then the newer fact to see that
$\pi_i(A,A\cap B)\cong\pi_i(A,V)$ in a range. I haven't thought about this
\emph{at all}!



\HSsection{Serre class theory breaks down with nontrivial 
\texorpdfstring{$\pi_1$}{fundamental group}}{8/9/11}
$X=S^1\vee S^2$ has finitely generated homology. However $\pi_2(X)=\oplus_\Z\Z$.
To see this, note that the universal cover is $\R$ with a sphere attached at
each integer. This has $H_2(\widetilde X)=\oplus_\Z\Z$, and one can apply
Hurewicz.

This is interesting, since the homology of $X$ (in each degree) is finitely
generated, but the homotopy is not. But finitely generated abelian groups form a
Serre class!

\HSsection{Group homology}{9/9/11}
\[\xymatrix{
\Z[\Z_2]\ar[r]^{1+t}&
\Z[\Z_2]\ar[r]^{1-t}&
\Z[\Z_2]\ar[r]^{1\shortmapsto 1}_{t\shortmapsto1}&
\Z\ \ &\\
}\]
Tensor with the $\Z_2$-module, take homology.
The notion of $\Z_2$ module is the same as a set of twisted coeffs
on $B\Z_2$.

By the way, the normal way to do homology with twisted coefficients is as
follows.
Suppose $\scrM$ is a system of local coefficients on $X$. Form $C_*(\widetilde
X)$. This has an action of $\pi=\pi_1(X)$, and we form
$C_*(X;\scrM)=C_*(\widetilde X)\otimes_{\Z[\pi]}\scrM$. Take homology to get
$H_*(X,\scrM)$.

To get the cohomology, one takes $C^*(X;\scrM)=\Hom_{\pi}(C_*(\widetilde
X),\scrM)$, and take cohomology.

Finally, suppose that $\scrM$ is a trivial system isomorphic to $\Z$. Pick a
CW-structure on $X$ and lift it to one on $\widetilde X$. Then
$C^\text{CW}_*(\widetilde X)\otimes_{\Z[\pi]}\Z=C^\text{CW}_*(X;\Z)$.
\HSsection{A compatibility in the Serre long exact sequence}{9/9/11}
\[\xymatrix{
H_n(F)\ar[r]&H_n(E)\ar[r]&H_n(B)\ar[r]^\tau&H_{n-1}(F)\\
H_n(F)\ar[r]\ar@{=}[u]&H_n(E)\ar[r]\ar@{=}[u]&
H_n(E,F)\ar[r]\ar[u]^{p_*}_\cong&H_{n-1}(F)\ar@{=}[u]
}\]
The Serre exact sequence holds in a range, depending on connectivities. This
diagram commutes (by Serre's construction), and the 5 lemma points out that
$p_*$ is an isomorphism in (at least most of) this range.
\end{FirstTen}

\HSsection{Mixed Motives}{14/10/11}

Let $\calC$ be a category. We can form the category $\Delta^\op\Psh(\calC)$ of simplicial presheaves on $\calC$. We'll think of this as the functor category
\[[\calC^\op,\sSet]\text{ \ i.e.\ presheaves of simplicial sets on $\calC$.}\]
($\Delta^\op\Psh(\calC)$ is also the category of simplicial objects in $\Psh(\calC)$).

Now suppose that $\calC$ is a site. Then we can define what we mean by a homotopy simplicial sheaf $\scrF$ on $\calC$. It should be a simplicial presheaf which turns pushout diagrams into homotopy pushout diagrams \rednote{[actually, i reckon it should turn coequalisers of covering families into homotopy equalisers (i.e.\ limits), or something. I don't know.}

Anyway, in our application, we'll be having $\calC$ the Nisnevich site. That is, the site whose underlying category is the category of schemes, and whose covering families are the coverings by Nisnevich maps. [A map $X\to Y$ is Nisnevich if it is Etal\'e, and for all $y\in Y$, there is some $x\mapsto y$ such that the extension $k(y)\to k(x)$ is an isomorphism.]

Note that it is very hard for a cover to be Zariski, easier to be Nisnevich, easier still to be Etal\'e, and finally, very easy to be flat. Now the points of the Zariski site are exactly the local rings, however, the points of the Nisnevich site are the Henselian local rings: as there are more coverings, there are more open sets, and so we can get closer to any given point --- thus we get fewer `points'.

By the way, consider the point $0\in\A^1_k$. Then the Zariski local ring is $k[t][\text{nonconstant}]^{-1}$. The Nisnevich local ring is $k[[t]]$. The Etale local ring is $k^{sep}[[t]]$, the power series over the separable closure. That is, since any finite separable field extension is an etale cover, it must be that there are no finite separable field extensions of the residue field. This perhaps is what's better about the Nisnevich --- one doesn't get this effect.

Anyway, note now that $\Delta^\op\Psh(\calC)$ is a category of diagrams in $\sSet$. $\sSet$ is a model category, and we give $\Delta^\op\Psh(\calC)$ the injective model structure --- the cofibrations and weak equivalences are objectwise, and the fibrations are what they have to be. Our plan is to perform a left Bausfield localisation, to add in the weak equivalences $\scrW$, where:
\[\scrW:=\{\scrF\to\scrG\ |\ \text{for all points $x$, $x^*(\scrF)\weakequiv x^*(\scrG)$}\}.\]
[By the way: a (simplicial) point of a site is a pair of adjoint functors, representing stalks and skyscraper sheaves $x^*:[\calC^\op,\Set]\longleftrightarrow\sSet:x_*$, satisfying some more axioms.]

Anyway, call an object $\scrH\in\Delta^\op\Psh(\calC)$ $\scrW$-local if for all morphisms $\scrF\to\scrG$, one has $[\scrG,\scrH]\to[\scrF,\scrH]$ a weak equivalence \rednote{(as this map is a map of simplicial sets? What? I'm confused.)}. The $\scrW$-local objects are exactly the sheaves, probably because the Nisnevich site has \emph{enough points}. Anyway, when we Bausfield localise, the new fibrant objects are those which were already fibrant, and are $\scrW$-local, so all fibrant objects are sheaves.

What's called for now is that we perform another left Bausfield localisation --- this time we want to make $\scrW':=\{\A^1\times X\to X\}$ weak equivalences.

By now, we get some coll things, like $\Sigma(\G)=\PP^1$. This holds because we had a pushout diagram:
\[\xymatrix{\ar[d]\A^1\setminus\{0\}\ar[r]&\A^1\ar[d]\\
\A^1\ar[r]&\PP^1
}\]
and now the $\A^1$ have become contractible.

Now let the category of motivic spectra be the category of sequences $X_i$ of simplicial presheaves on the Nisnevich site, with morphisms $X_i\wedge\PP^i\to X_{i+1}$. Put the appropriate model category structure on this, so that the homotopy category is right. A (mixed) motive is then to be an $H\Z$-module spectrum, where $H\Z$ represents `motivic cohomology'.

Maybe one calls these $\PP^1$-spectra. Whatever. Anyway, we can think of a couple of different spheres. Let $S^{1,0}$ the the sheafification of the constant diagram which takes value a simplicial $S^1$. Let $S^{1,1}$ be $\G_m$, the mulitplicative group of the field. The first coordinate here represents the topological dimension (i.e.\ over $\C$, the dimension of the sphere to which the complex points are homotopic). The second coordinate is the motivic coordinate. 

Smashing these together, we get $S^{pq}$ for $p\geq q$. As $\PP^1=S^{2,1}$, when we stabilise, we get all of the spheres.

Some examples of $\PP^1$-spectra:
\begin{itemise}
\item $KGL$. Here, $(KGL)_n:=BGL_\infty(k)$. Note that Bott periodicity states that 
\[\Map(\PP^1,BGL_\infty)\sim BGL_\infty.\]
This represents algebraic $K$-theory!
\item $MGL$. $(MGL)_n=\text{Th}(BGL_n,\omega)$, where $\omega$ is the universal $n$-bundle. This represents algebraic cobordism.
\end{itemise}
\end{document}
















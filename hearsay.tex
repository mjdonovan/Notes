\documentclass[11pt]{article}
%\usepackage{cancel}
\usepackage{fullpage}
\usepackage{amsmath,amsthm,amssymb}
\usepackage{mathrsfs,nicefrac}
\usepackage{amssymb}
\usepackage{epsfig}
\usepackage[all]{xy}
\usepackage{sseq}
\usepackage[pdftex]{hyperref}
\usepackage{tocloft}


\newcommand{\II}{\mathcal{I}}

\DeclareSymbolFont{AMSb}{U}{msb}{m}{n}
\DeclareMathSymbol{\N}{\mathbin}{AMSb}{"4E}
\DeclareMathSymbol{\Z}{\mathbin}{AMSb}{"5A}
\DeclareMathSymbol{\R}{\mathbin}{AMSb}{"52}
\DeclareMathSymbol{\Q}{\mathbin}{AMSb}{"51}
\DeclareMathSymbol{\PP}{\mathbin}{AMSb}{"50}
\DeclareMathSymbol{\I}{\mathbin}{AMSb}{"49}
\DeclareMathSymbol{\C}{\mathbin}{AMSb}{"43}
\DeclareMathSymbol{\A}{\mathbin}{AMSb}{"41}
\DeclareMathSymbol{\F}{\mathbin}{AMSb}{"46}

\newcommand{\ad}{\textup{\textbf{ad}}}
\newcommand{\sll}{\mathfrak{sl}}
\newcommand{\gl}{\mathfrak{gl}}
\newcommand{\GL}{\mbox{GL}}
\newcommand{\SL}{\mbox{SL}}
\newcommand{\tr}{\mbox{tr\ }}
\newcommand{\Mat}{\mbox{Mat}}
\newcommand{\Lie}{\mbox{\textbf{Lie }}}
\newcommand{\Der}{\textup{Der}}
\newcommand{\End}{\mbox{End\ }}
\newcommand{\im}{\mbox{im}}
\newcommand{\Gr}{\textup{Gr}}
\newcommand{\pim}{\mbox{pim}}
\newcommand{\supp}{\mbox{supp\,}}
\newcommand{\Ker}{\mbox{ker\ }}
\renewcommand{\ker}{\textup{ker}\,}
\newcommand{\coker}{\textup{coker}\,}

\newcommand{\PPPP}{\mathcal{P}}

\newcommand{\aaa}{\mathfrak{a}}
\newcommand{\mmm}{\mathfrak{m}}
\newcommand{\qqq}{\mathfrak{q}}
\newcommand{\ppp}{\mathfrak{p}}
\newcommand{\g}{\mathfrak{g}}
\newcommand{\h}{\mathfrak{h}}
\newcommand{\m}{\mathfrak{m}}
\newcommand{\He}{\mathfrak{H}}
\newcommand{\shfF}{\mathscr{F}}
\newcommand{\shfG}{\mathscr{G}}
\newcommand{\shfH}{\mathscr{H}}
\newcommand{\shfL}{\mathscr{L}}
\newcommand{\sHOMFG}{\mathscr{H}\textit{\!\!om}(\shfF,\shfG)}
\newcommand{\sHOM}{\mathscr{H}\textit{\!\!om}}
\newcommand{\Hom}{\textup{Hom}}
\newcommand{\Ext}{\textup{Ext}}
\newcommand{\Tor}{\textup{Tor}}
\newcommand{\Map}{\textup{Map}}
\newcommand{\sk}{\vspace*{1em}}
\renewcommand{\phi}{\varphi}
%\newcommand{\ann}{\textup{ann}}
\DeclareMathOperator{\ann}{ann}
\newcommand{\Squ}{\textup{Sq}}
\newcommand{\DASH}{\textup{---}}


\theoremstyle{plain}
\newtheorem{thm}{Theorem}[section]
\newtheorem*{thm*}{Theorem}
\newtheorem{lem}[thm]{Lemma}
\newtheorem*{lem*}{Lemma}
\newtheorem{prop}[thm]{Proposition}
\newtheorem*{prop*}{Proposition}
%\newtheorem*{prop}{Proposition}
\newtheorem{cor}[thm]{Corollary}
\newtheorem*{cor*}{Corollary}
\newtheorem{defprop}[thm]{Definition-Proposition}


\theoremstyle{definition}
\newtheorem{defn}{Definition}[section]
\newtheorem{exmp}{Example}[section]
\newtheorem{asspt}{Assumption}[section]
\newtheorem{notation}{Notation}[section]
\newtheorem{exercise}{Exercise}[section]


% 1-inch margins, from fullpage.sty by H.Partl, Version 2, Dec. 15, 1988.
\topmargin 0pt
\advance \topmargin by -\headheight
\advance \topmargin by -\headsep
\textheight 9.1in
\oddsidemargin 0pt
\evensidemargin \oddsidemargin
\marginparwidth 0.5in
\textwidth 6.5in

\parindent 0in
\parskip 1.5ex
%\renewcommand{\baselinestretch}{1.25}

\newcommand{\RAD}[1]{\textup{rad}(#1)}
\newcommand{\SPC}[1]{\textup{sp}(#1)}
\newcommand{\RED}[1]{{#1}_\textup{red}}
\newcommand{\REDpsh}[1]{{#1}_\textup{red}^-}
\newcommand{\OO}{\mathcal{O}}
\newcommand{\spec}{\textup{spec}\,}
\newcommand{\spce}{\textup{sp}}
\newcommand{\proj}{\textup{proj}\,}
\newcommand{\HOMs}{\textup{Hom}_\mathfrak{Sch}}
\newcommand{\HOMr}{\textup{Hom}_\mathfrak{Rings}}

\renewcommand{\to}{\longrightarrow}
\renewcommand{\mapsto}{\longmapsto}
\newcommand{\eps}{\varepsilon}

\newcommand{\catC}{\mathcal{C}}
\newcommand{\catD}{\mathcal{D}}
\newcommand{\catSet}{\textit{Sets}}
\newcommand{\op}{\textup{op}}
\newcommand{\dash}{\textup{---}}
\newcommand{\id}{\textup{id}}
\newcommand{\mapsfrom}{\,\reflectbox{$\mapsto$}\ }
\newcommand{\cone}{\textup{cone}}
\newcommand{\COMMENT}[1]{}

\newcommand{\calL}{\mathcal{L}}
\newcommand{\RP}{{\R\PP}}
\newcommand{\CP}{{\C\PP}}
\newcommand{\Steen}{\mathscr{A}}
\newcommand{\Orth}{\textup{\textbf{O}}}
\newcommand{\bigrightsquig}{\scalebox{2}{\ensuremath{\rightsquigarrow}}}
\newcommand{\SerreClass}{\mathscr{C}}


\newcommand{\squishlist}{
  \setlength{\itemsep}{1pt}
  \setlength{\parskip}{0pt}
  \setlength{\parsep}{0pt}
}
\newenvironment{itemise}{
\begin{list}{\textup{$\rightsquigarrow$}}
   {
      \setlength{\topsep}{.1cm}
      \setlength{\itemsep}{1pt}
      \setlength{\parskip}{0pt}
      \setlength{\parsep}{0pt}
   }
}{\end{list}\vspace{-.2cm}}



\title{Hearsay}
\author{}
\date{}
% >>>

%% >>> header

\begin{document}
%\tableofcontents
\newcommand{\HSsection}[2]{\section*{#1 \hfill \small{#2}}}

\HSsection{Preface}{}
This file is intended only as a repository for mathematical facts I picked up in conversation, and found interesting. Any given entry is likely to be incorrect, nothing found in this document should be relied upon.

\HSsection{$H_*(X)=\pi_*(\textup{Sp}^\infty X)$ when $X$ is connected}{29/6/11}
Let $X$ be a connected space. Let $\textup{Sp}^\infty X$ be the free topological commutative monoid on $X$, where the basepoint of $X$ acts as the identity. This is the quotient of the james construction $J(X)$ by the infinite permutation group $\Sigma_\infty$. Let $\overline\Z X$ be the free topological abelian group on $X$ with basepoint the identity. Then $\textup{Sp}^\infty X\to \overline\Z X$ is a weak equivalence (iff $X$ is connected).

Moreover, there is a homotopy equivalence between the simplicial abelian groups $\textup{Sing}\overline\Z X$ and $\Z\textup{Sing}X$. Thus $\pi_*(\overline\Z X)=\pi_*(\Z\textup{Sing}X)$. Finally, it is a fact that given a simplicial abelian group, its homotopy groups and its homology groups coincide!


\HSsection{$Q$ turns cofiber sequences into fiber sequences}{29/6/11}
This corresponds to the fact that $\overline\pi_*^s=\pi_*Q$ is a reduced homology theory. It should give you a LES for a cofiber sequence, but $\pi_*$ likes fiber sequences!

Here, $QX:=\bigcup\Omega^i\Sigma^i X$. Now suppose that $X\to Y\to Z$ is a cofiber sequence of $n$-connected spaces. Let $F\to Y$ be the homotopy fiber of $Y\to Z$.  We obtain the a Serre long exact sequence from the Serre spectral sequence, which starts (approximately) at $2n$:
\[\xymatrix{
H_{2n}(F)\ar[r]&H_{2n}(Y)\ar[r]&H_{2n}(Z)\ar[r]^{\tau\ \ }&H_{2n-1}(F)\ar[r]&\cdots
}\]
Now there is an induced map $\epsilon:X\to F$, so we have arrows between the two long exact sequences:
\[\xymatrix{
H_{2n}(X)\ar[r]\ar[d]^\epsilon&H_{2n}(Y)\ar[r]\ar@{=}[d]&H_{2n}(Z)\ar[r]^{\ \ }\ar@{=}[d]&H_{2n-1}(X)\ar[r]\ar[d]^\epsilon&\cdots\\
H_{2n}(F)\ar[r]&H_{2n}(Y)\ar[r]&H_{2n}(Z)\ar[r]^{\tau\ \ }&H_{2n-1}(F)\ar[r]&\cdots
}\]
The five lemma will show that $\epsilon$ is an isomorphism on $H_{i}$ for $i$ at most (around) $2n$, as long as we can show that the rightmost square commutes...
\end{document}
















% !TEX root = z_output/_PRLieAlgs.tex
\documentclass[11pt]{article}
\usepackage{fullpage}
\usepackage{amsmath,amsthm,amssymb}
\usepackage{mathrsfs,nicefrac}
\usepackage{amssymb}
\usepackage{epsfig}
\usepackage[all,2cell]{xy}
\usepackage{sseq}
\usepackage{tocloft}
\usepackage{cancel}
\usepackage[strict]{changepage}
\usepackage{color}
\usepackage{tikz}
\usepackage{extpfeil}
\usepackage{version}
\usepackage{framed}
\definecolor{shadecolor}{rgb}{.925,0.925,0.925}

%\usepackage{ifthen}
%Used for disabling hyperref
\ifx\dontloadhyperref\undefined
%\usepackage[pdftex,pdfborder={0 0 0 [1 1]}]{hyperref}
\usepackage[pdftex,pdfborder={0 0 .5 [1 1]}]{hyperref}
\else
\providecommand{\texorpdfstring}[2]{#1}
\fi
%>>>>>>>>>>>>>>>>>>>>>>>>>>>>>>
%<<<        Versions        <<<
%>>>>>>>>>>>>>>>>>>>>>>>>>>>>>>
%Add in the following line to include all the versions.
%\def\excludeversion#1{\includeversion{#1}}

%>>>>>>>>>>>>>>>>>>>>>>>>>>>>>>
%<<<       Better ToC       <<<
%>>>>>>>>>>>>>>>>>>>>>>>>>>>>>>
\setlength{\cftbeforesecskip}{0.5ex}

%>>>>>>>>>>>>>>>>>>>>>>>>>>>>>>
%<<<      Hyperref mod      <<<
%>>>>>>>>>>>>>>>>>>>>>>>>>>>>>>

%needs more testing
\newcounter{dummyforrefstepcounter}
\newcommand{\labelRIGHTHERE}[1]
{\refstepcounter{dummyforrefstepcounter}\label{#1}}


%>>>>>>>>>>>>>>>>>>>>>>>>>>>>>>
%<<<  Theorem Environments  <<<
%>>>>>>>>>>>>>>>>>>>>>>>>>>>>>>
\ifx\dontloaddefinitionsoftheoremenvironments\undefined
\theoremstyle{plain}
\newtheorem{thm}{Theorem}[section]
\newtheorem*{thm*}{Theorem}
\newtheorem{lem}[thm]{Lemma}
\newtheorem*{lem*}{Lemma}
\newtheorem{prop}[thm]{Proposition}
\newtheorem*{prop*}{Proposition}
\newtheorem{cor}[thm]{Corollary}
\newtheorem*{cor*}{Corollary}
\newtheorem{defprop}[thm]{Definition-Proposition}
\newtheorem*{punchline}{Punchline}
\newtheorem*{conjecture}{Conjecture}
\newtheorem*{claim}{Claim}

\theoremstyle{definition}
\newtheorem{defn}{Definition}[section]
\newtheorem*{defn*}{Definition}
\newtheorem{exmp}{Example}[section]
\newtheorem*{exmp*}{Example}
\newtheorem*{exmps*}{Examples}
\newtheorem*{nonexmp*}{Non-example}
\newtheorem{asspt}{Assumption}[section]
\newtheorem{notation}{Notation}[section]
\newtheorem{exercise}{Exercise}[section]
\newtheorem*{fact*}{Fact}
\newtheorem*{rmk*}{Remark}
\newtheorem{fact}{Fact}
\newtheorem*{aside}{Aside}
\newtheorem*{question}{Question}
\newtheorem*{answer}{Answer}

\else\relax\fi

%>>>>>>>>>>>>>>>>>>>>>>>>>>>>>>
%<<<      Fields, etc.      <<<
%>>>>>>>>>>>>>>>>>>>>>>>>>>>>>>
\DeclareSymbolFont{AMSb}{U}{msb}{m}{n}
\DeclareMathSymbol{\N}{\mathbin}{AMSb}{"4E}
\DeclareMathSymbol{\Octonions}{\mathbin}{AMSb}{"4F}
\DeclareMathSymbol{\Z}{\mathbin}{AMSb}{"5A}
\DeclareMathSymbol{\R}{\mathbin}{AMSb}{"52}
\DeclareMathSymbol{\Q}{\mathbin}{AMSb}{"51}
\DeclareMathSymbol{\PP}{\mathbin}{AMSb}{"50}
\DeclareMathSymbol{\I}{\mathbin}{AMSb}{"49}
\DeclareMathSymbol{\C}{\mathbin}{AMSb}{"43}
\DeclareMathSymbol{\A}{\mathbin}{AMSb}{"41}
\DeclareMathSymbol{\F}{\mathbin}{AMSb}{"46}
\DeclareMathSymbol{\G}{\mathbin}{AMSb}{"47}
\DeclareMathSymbol{\Quaternions}{\mathbin}{AMSb}{"48}


%>>>>>>>>>>>>>>>>>>>>>>>>>>>>>>
%<<<       Operators        <<<
%>>>>>>>>>>>>>>>>>>>>>>>>>>>>>>
\DeclareMathOperator{\ad}{\textbf{ad}}
\DeclareMathOperator{\coker}{coker}
\renewcommand{\ker}{\textup{ker}\,}
\DeclareMathOperator{\End}{End}
\DeclareMathOperator{\Aut}{Aut}
\DeclareMathOperator{\Hom}{Hom}
\DeclareMathOperator{\Maps}{Maps}
\DeclareMathOperator{\Mor}{Mor}
\DeclareMathOperator{\Gal}{Gal}
\DeclareMathOperator{\Ext}{Ext}
\DeclareMathOperator{\Tor}{Tor}
\DeclareMathOperator{\Map}{Map}
\DeclareMathOperator{\Der}{Der}
\DeclareMathOperator{\Rad}{Rad}
\DeclareMathOperator{\rank}{rank}
\DeclareMathOperator{\ArfInvariant}{Arf}
\DeclareMathOperator{\KervaireInvariant}{Ker}
\DeclareMathOperator{\im}{im}
\DeclareMathOperator{\coim}{coim}
\DeclareMathOperator{\trace}{tr}
\DeclareMathOperator{\supp}{supp}
\DeclareMathOperator{\ann}{ann}
\DeclareMathOperator{\spec}{Spec}
\DeclareMathOperator{\SPEC}{\textbf{Spec}}
\DeclareMathOperator{\proj}{Proj}
\DeclareMathOperator{\PROJ}{\textbf{Proj}}
\DeclareMathOperator{\fiber}{F}
\DeclareMathOperator{\cofiber}{C}
\DeclareMathOperator{\cone}{cone}
\DeclareMathOperator{\skel}{sk}
\DeclareMathOperator{\coskel}{cosk}
\DeclareMathOperator{\conn}{conn}
\DeclareMathOperator{\colim}{colim}
\DeclareMathOperator{\limit}{lim}
\DeclareMathOperator{\ch}{ch}
\DeclareMathOperator{\Vect}{Vect}
\DeclareMathOperator{\GrthGrp}{GrthGp}
\DeclareMathOperator{\Sym}{Sym}
\DeclareMathOperator{\Prob}{\mathbb{P}}
\DeclareMathOperator{\Exp}{\mathbb{E}}
\DeclareMathOperator{\GeomMean}{\mathbb{G}}
\DeclareMathOperator{\Var}{Var}
\DeclareMathOperator{\Cov}{Cov}
\DeclareMathOperator{\Sp}{Sp}
\DeclareMathOperator{\Seq}{Seq}
\DeclareMathOperator{\Cyl}{Cyl}
\DeclareMathOperator{\Ev}{Ev}
\DeclareMathOperator{\sh}{sh}
\DeclareMathOperator{\intHom}{\underline{Hom}}
\DeclareMathOperator{\Frac}{frac}



%>>>>>>>>>>>>>>>>>>>>>>>>>>>>>>
%<<<   Cohomology Theories  <<<
%>>>>>>>>>>>>>>>>>>>>>>>>>>>>>>
\DeclareMathOperator{\KR}{{K\R}}
\DeclareMathOperator{\KO}{{KO}}
\DeclareMathOperator{\K}{{K}}
\DeclareMathOperator{\OmegaO}{{\Omega_{\Octonions}}}

%>>>>>>>>>>>>>>>>>>>>>>>>>>>>>>
%<<<   Algebraic Geometry   <<<
%>>>>>>>>>>>>>>>>>>>>>>>>>>>>>>
\DeclareMathOperator{\Spec}{Spec}
\DeclareMathOperator{\Proj}{Proj}
\DeclareMathOperator{\Sing}{Sing}
\DeclareMathOperator{\shfHom}{\mathscr{H}\textit{\!\!om}}
\DeclareMathOperator{\WeilDivisors}{{Div}}
\DeclareMathOperator{\CartierDivisors}{{CaDiv}}
\DeclareMathOperator{\PrincipalWeilDivisors}{{PrDiv}}
\DeclareMathOperator{\LocallyPrincipalWeilDivisors}{{LPDiv}}
\DeclareMathOperator{\PrincipalCartierDivisors}{{PrCaDiv}}
\DeclareMathOperator{\DivisorClass}{{Cl}}
\DeclareMathOperator{\CartierClass}{{CaCl}}
\DeclareMathOperator{\Picard}{{Pic}}
\DeclareMathOperator{\Frob}{Frob}


%>>>>>>>>>>>>>>>>>>>>>>>>>>>>>>
%<<<  Mathematical Objects  <<<
%>>>>>>>>>>>>>>>>>>>>>>>>>>>>>>
\newcommand{\sll}{\mathfrak{sl}}
\newcommand{\gl}{\mathfrak{gl}}
\newcommand{\GL}{\mbox{GL}}
\newcommand{\PGL}{\mbox{PGL}}
\newcommand{\SL}{\mbox{SL}}
\newcommand{\Mat}{\mbox{Mat}}
\newcommand{\Gr}{\textup{Gr}}
\newcommand{\Squ}{\textup{Sq}}
\newcommand{\catSet}{\textit{Sets}}
\newcommand{\RP}{{\R\PP}}
\newcommand{\CP}{{\C\PP}}
\newcommand{\Steen}{\mathscr{A}}
\newcommand{\Orth}{\textup{\textbf{O}}}

%>>>>>>>>>>>>>>>>>>>>>>>>>>>>>>
%<<<  Mathematical Symbols  <<<
%>>>>>>>>>>>>>>>>>>>>>>>>>>>>>>
\newcommand{\DASH}{\textup{---}}
\newcommand{\op}{\textup{op}}
\newcommand{\CW}{\textup{CW}}
\newcommand{\ob}{\textup{ob}\,}
\newcommand{\ho}{\textup{ho}}
\newcommand{\st}{\textup{st}}
\newcommand{\id}{\textup{id}}
\newcommand{\Bullet}{\ensuremath{\bullet} }
\newcommand{\sprod}{\wedge}

%>>>>>>>>>>>>>>>>>>>>>>>>>>>>>>
%<<<      Some Arrows       <<<
%>>>>>>>>>>>>>>>>>>>>>>>>>>>>>>
\newcommand{\nt}{\Longrightarrow}
\let\shortmapsto\mapsto
\let\mapsto\longmapsto
\newcommand{\mapsfrom}{\,\reflectbox{$\mapsto$}\ }
\newcommand{\bigrightsquig}{\scalebox{2}{\ensuremath{\rightsquigarrow}}}
\newcommand{\bigleftsquig}{\reflectbox{\scalebox{2}{\ensuremath{\rightsquigarrow}}}}

%\newcommand{\cofibration}{\xhookrightarrow{\phantom{\ \,{\sim\!}\ \ }}}
%\newcommand{\fibration}{\xtwoheadrightarrow{\phantom{\sim\!}}}
%\newcommand{\acycliccofibration}{\xhookrightarrow{\ \,{\sim\!}\ \ }}
%\newcommand{\acyclicfibration}{\xtwoheadrightarrow{\sim\!}}
%\newcommand{\leftcofibration}{\xhookleftarrow{\phantom{\ \,{\sim\!}\ \ }}}
%\newcommand{\leftfibration}{\xtwoheadleftarrow{\phantom{\sim\!}}}
%\newcommand{\leftacycliccofibration}{\xhookleftarrow{\ \ {\sim\!}\,\ }}
%\newcommand{\leftacyclicfibration}{\xtwoheadleftarrow{\sim\!}}
%\newcommand{\weakequiv}{\xrightarrow{\ \,\sim\,\ }}
%\newcommand{\leftweakequiv}{\xleftarrow{\ \,\sim\,\ }}

\newcommand{\cofibration}
{\xhookrightarrow{\phantom{\ \,{\raisebox{-.3ex}[0ex][0ex]{\scriptsize$\sim$}\!}\ \ }}}
\newcommand{\fibration}
{\xtwoheadrightarrow{\phantom{\raisebox{-.3ex}[0ex][0ex]{\scriptsize$\sim$}\!}}}
\newcommand{\acycliccofibration}
{\xhookrightarrow{\ \,{\raisebox{-.55ex}[0ex][0ex]{\scriptsize$\sim$}\!}\ \ }}
\newcommand{\acyclicfibration}
{\xtwoheadrightarrow{\raisebox{-.6ex}[0ex][0ex]{\scriptsize$\sim$}\!}}
\newcommand{\leftcofibration}
{\xhookleftarrow{\phantom{\ \,{\raisebox{-.3ex}[0ex][0ex]{\scriptsize$\sim$}\!}\ \ }}}
\newcommand{\leftfibration}
{\xtwoheadleftarrow{\phantom{\raisebox{-.3ex}[0ex][0ex]{\scriptsize$\sim$}\!}}}
\newcommand{\leftacycliccofibration}
{\xhookleftarrow{\ \ {\raisebox{-.55ex}[0ex][0ex]{\scriptsize$\sim$}\!}\,\ }}
\newcommand{\leftacyclicfibration}
{\xtwoheadleftarrow{\raisebox{-.6ex}[0ex][0ex]{\scriptsize$\sim$}\!}}
\newcommand{\weakequiv}
{\xrightarrow{\ \,\raisebox{-.3ex}[0ex][0ex]{\scriptsize$\sim$}\,\ }}
\newcommand{\leftweakequiv}
{\xleftarrow{\ \,\raisebox{-.3ex}[0ex][0ex]{\scriptsize$\sim$}\,\ }}

%>>>>>>>>>>>>>>>>>>>>>>>>>>>>>>
%<<<    xymatrix Arrows     <<<
%>>>>>>>>>>>>>>>>>>>>>>>>>>>>>>
\newdir{ >}{{}*!/-5pt/@{>}}
\newcommand{\xycof}{\ar@{ >->}}
\newcommand{\xycofib}{\ar@{^{(}->}}
\newcommand{\xycofibdown}{\ar@{_{(}->}}
\newcommand{\xyfib}{\ar@{->>}}
\newcommand{\xymapsto}{\ar@{|->}}

%>>>>>>>>>>>>>>>>>>>>>>>>>>>>>>
%<<<     Greek Letters      <<<
%>>>>>>>>>>>>>>>>>>>>>>>>>>>>>>
%\newcommand{\oldphi}{\phi}
%\renewcommand{\phi}{\varphi}
\let\oldphi\phi
\let\phi\varphi
\renewcommand{\to}{\longrightarrow}
\newcommand{\from}{\longleftarrow}
\newcommand{\eps}{\varepsilon}

%>>>>>>>>>>>>>>>>>>>>>>>>>>>>>>
%<<<  1st-4th & parentheses <<<
%>>>>>>>>>>>>>>>>>>>>>>>>>>>>>>
\newcommand{\first}{^\text{st}}
\newcommand{\second}{^\text{nd}}
\newcommand{\third}{^\text{rd}}
\newcommand{\fourth}{^\text{th}}
\newcommand{\ZEROTH}{$0^\text{th}$ }
\newcommand{\FIRST}{$1^\text{st}$ }
\newcommand{\SECOND}{$2^\text{nd}$ }
\newcommand{\THIRD}{$3^\text{rd}$ }
\newcommand{\FOURTH}{$4^\text{th}$ }
\newcommand{\iTH}{$i^\text{th}$ }
\newcommand{\jTH}{$j^\text{th}$ }
\newcommand{\nTH}{$n^\text{th}$ }

%>>>>>>>>>>>>>>>>>>>>>>>>>>>>>>
%<<<    upright commands    <<<
%>>>>>>>>>>>>>>>>>>>>>>>>>>>>>>
\newcommand{\upcol}{\textup{:}}
\newcommand{\upsemi}{\textup{;}}
\providecommand{\lparen}{\textup{(}}
\providecommand{\rparen}{\textup{)}}
\renewcommand{\lparen}{\textup{(}}
\renewcommand{\rparen}{\textup{)}}
\newcommand{\Iff}{\emph{iff} }

%>>>>>>>>>>>>>>>>>>>>>>>>>>>>>>
%<<<     Environments       <<<
%>>>>>>>>>>>>>>>>>>>>>>>>>>>>>>
\newcommand{\squishlist}
{ %\setlength{\topsep}{100pt} doesn't seem to do anything.
  \setlength{\itemsep}{.5pt}
  \setlength{\parskip}{0pt}
  \setlength{\parsep}{0pt}}
\newenvironment{itemise}{
\begin{list}{\textup{$\rightsquigarrow$}}
   {  \setlength{\topsep}{1mm}
      \setlength{\itemsep}{1pt}
      \setlength{\parskip}{0pt}
      \setlength{\parsep}{0pt}
   }
}{\end{list}\vspace{-.1cm}}
\newcommand{\INDENT}{\textbf{}\phantom{space}}
\renewcommand{\INDENT}{\rule{.7cm}{0cm}}

\newcommand{\itm}[1][$\rightsquigarrow$]{\item[{\makebox[.5cm][c]{\textup{#1}}}]}


%\newcommand{\rednote}[1]{{\color{red}#1}\makebox[0cm][l]{\scalebox{.1}{rednote}}}
%\newcommand{\bluenote}[1]{{\color{blue}#1}\makebox[0cm][l]{\scalebox{.1}{rednote}}}

\newcommand{\rednote}[1]
{{\color{red}#1}\makebox[0cm][l]{\scalebox{.1}{\rotatebox{90}{?????}}}}
\newcommand{\bluenote}[1]
{{\color{blue}#1}\makebox[0cm][l]{\scalebox{.1}{\rotatebox{90}{?????}}}}


\newcommand{\funcdef}[4]{\begin{align*}
#1&\to #2\\
#3&\mapsto#4
\end{align*}}

%\newcommand{\comment}[1]{}

%>>>>>>>>>>>>>>>>>>>>>>>>>>>>>>
%<<<       Categories       <<<
%>>>>>>>>>>>>>>>>>>>>>>>>>>>>>>
\newcommand{\Ens}{{\mathscr{E}ns}}
\DeclareMathOperator{\Sheaves}{{\mathsf{Shf}}}
\DeclareMathOperator{\Presheaves}{{\mathsf{PreShf}}}
\DeclareMathOperator{\Psh}{{\mathsf{Psh}}}
\DeclareMathOperator{\Shf}{{\mathsf{Shf}}}
\DeclareMathOperator{\Varieties}{{\mathsf{Var}}}
\DeclareMathOperator{\Schemes}{{\mathsf{Sch}}}
\DeclareMathOperator{\Rings}{{\mathsf{Rings}}}
\DeclareMathOperator{\AbGp}{{\mathsf{AbGp}}}
\DeclareMathOperator{\Modules}{{\mathsf{\!-Mod}}}
\DeclareMathOperator{\fgModules}{{\mathsf{\!-Mod}^{\textup{fg}}}}
\DeclareMathOperator{\QuasiCoherent}{{\mathsf{QCoh}}}
\DeclareMathOperator{\Coherent}{{\mathsf{Coh}}}
\DeclareMathOperator{\GSW}{{\mathcal{SW}^G}}
\DeclareMathOperator{\Burnside}{{\mathsf{Burn}}}
\DeclareMathOperator{\GSet}{{G\mathsf{Set}}}
\DeclareMathOperator{\FinGSet}{{G\mathsf{Set}^\textup{fin}}}
\DeclareMathOperator{\HSet}{{H\mathsf{Set}}}
\DeclareMathOperator{\Cat}{{\mathsf{Cat}}}
\DeclareMathOperator{\Fun}{{\mathsf{Fun}}}
\DeclareMathOperator{\Orb}{{\mathsf{Orb}}}
\DeclareMathOperator{\Set}{{\mathsf{Set}}}
\DeclareMathOperator{\sSet}{{\mathsf{sSet}}}
\DeclareMathOperator{\Top}{{\mathsf{Top}}}
\DeclareMathOperator{\GSpectra}{{G-\mathsf{Spectra}}}
\DeclareMathOperator{\Lan}{Lan}
\DeclareMathOperator{\Ran}{Ran}

%>>>>>>>>>>>>>>>>>>>>>>>>>>>>>>
%<<<     Script Letters     <<<
%>>>>>>>>>>>>>>>>>>>>>>>>>>>>>>
\newcommand{\scrQ}{\mathscr{Q}}
\newcommand{\scrW}{\mathscr{W}}
\newcommand{\scrE}{\mathscr{E}}
\newcommand{\scrR}{\mathscr{R}}
\newcommand{\scrT}{\mathscr{T}}
\newcommand{\scrY}{\mathscr{Y}}
\newcommand{\scrU}{\mathscr{U}}
\newcommand{\scrI}{\mathscr{I}}
\newcommand{\scrO}{\mathscr{O}}
\newcommand{\scrP}{\mathscr{P}}
\newcommand{\scrA}{\mathscr{A}}
\newcommand{\scrS}{\mathscr{S}}
\newcommand{\scrD}{\mathscr{D}}
\newcommand{\scrF}{\mathscr{F}}
\newcommand{\scrG}{\mathscr{G}}
\newcommand{\scrH}{\mathscr{H}}
\newcommand{\scrJ}{\mathscr{J}}
\newcommand{\scrK}{\mathscr{K}}
\newcommand{\scrL}{\mathscr{L}}
\newcommand{\scrZ}{\mathscr{Z}}
\newcommand{\scrX}{\mathscr{X}}
\newcommand{\scrC}{\mathscr{C}}
\newcommand{\scrV}{\mathscr{V}}
\newcommand{\scrB}{\mathscr{B}}
\newcommand{\scrN}{\mathscr{N}}
\newcommand{\scrM}{\mathscr{M}}

%>>>>>>>>>>>>>>>>>>>>>>>>>>>>>>
%<<<     Fractur Letters    <<<
%>>>>>>>>>>>>>>>>>>>>>>>>>>>>>>
\newcommand{\frakQ}{\mathfrak{Q}}
\newcommand{\frakW}{\mathfrak{W}}
\newcommand{\frakE}{\mathfrak{E}}
\newcommand{\frakR}{\mathfrak{R}}
\newcommand{\frakT}{\mathfrak{T}}
\newcommand{\frakY}{\mathfrak{Y}}
\newcommand{\frakU}{\mathfrak{U}}
\newcommand{\frakI}{\mathfrak{I}}
\newcommand{\frakO}{\mathfrak{O}}
\newcommand{\frakP}{\mathfrak{P}}
\newcommand{\frakA}{\mathfrak{A}}
\newcommand{\frakS}{\mathfrak{S}}
\newcommand{\frakD}{\mathfrak{D}}
\newcommand{\frakF}{\mathfrak{F}}
\newcommand{\frakG}{\mathfrak{G}}
\newcommand{\frakH}{\mathfrak{H}}
\newcommand{\frakJ}{\mathfrak{J}}
\newcommand{\frakK}{\mathfrak{K}}
\newcommand{\frakL}{\mathfrak{L}}
\newcommand{\frakZ}{\mathfrak{Z}}
\newcommand{\frakX}{\mathfrak{X}}
\newcommand{\frakC}{\mathfrak{C}}
\newcommand{\frakV}{\mathfrak{V}}
\newcommand{\frakB}{\mathfrak{B}}
\newcommand{\frakN}{\mathfrak{N}}
\newcommand{\frakM}{\mathfrak{M}}

\newcommand{\frakq}{\mathfrak{q}}
\newcommand{\frakw}{\mathfrak{w}}
\newcommand{\frake}{\mathfrak{e}}
\newcommand{\frakr}{\mathfrak{r}}
\newcommand{\frakt}{\mathfrak{t}}
\newcommand{\fraky}{\mathfrak{y}}
\newcommand{\fraku}{\mathfrak{u}}
\newcommand{\fraki}{\mathfrak{i}}
\newcommand{\frako}{\mathfrak{o}}
\newcommand{\frakp}{\mathfrak{p}}
\newcommand{\fraka}{\mathfrak{a}}
\newcommand{\fraks}{\mathfrak{s}}
\newcommand{\frakd}{\mathfrak{d}}
\newcommand{\frakf}{\mathfrak{f}}
\newcommand{\frakg}{\mathfrak{g}}
\newcommand{\frakh}{\mathfrak{h}}
\newcommand{\frakj}{\mathfrak{j}}
\newcommand{\frakk}{\mathfrak{k}}
\newcommand{\frakl}{\mathfrak{l}}
\newcommand{\frakz}{\mathfrak{z}}
\newcommand{\frakx}{\mathfrak{x}}
\newcommand{\frakc}{\mathfrak{c}}
\newcommand{\frakv}{\mathfrak{v}}
\newcommand{\frakb}{\mathfrak{b}}
\newcommand{\frakn}{\mathfrak{n}}
\newcommand{\frakm}{\mathfrak{m}}

%>>>>>>>>>>>>>>>>>>>>>>>>>>>>>>
%<<<  Caligraphic Letters   <<<
%>>>>>>>>>>>>>>>>>>>>>>>>>>>>>>
\newcommand{\calQ}{\mathcal{Q}}
\newcommand{\calW}{\mathcal{W}}
\newcommand{\calE}{\mathcal{E}}
\newcommand{\calR}{\mathcal{R}}
\newcommand{\calT}{\mathcal{T}}
\newcommand{\calY}{\mathcal{Y}}
\newcommand{\calU}{\mathcal{U}}
\newcommand{\calI}{\mathcal{I}}
\newcommand{\calO}{\mathcal{O}}
\newcommand{\calP}{\mathcal{P}}
\newcommand{\calA}{\mathcal{A}}
\newcommand{\calS}{\mathcal{S}}
\newcommand{\calD}{\mathcal{D}}
\newcommand{\calF}{\mathcal{F}}
\newcommand{\calG}{\mathcal{G}}
\newcommand{\calH}{\mathcal{H}}
\newcommand{\calJ}{\mathcal{J}}
\newcommand{\calK}{\mathcal{K}}
\newcommand{\calL}{\mathcal{L}}
\newcommand{\calZ}{\mathcal{Z}}
\newcommand{\calX}{\mathcal{X}}
\newcommand{\calC}{\mathcal{C}}
\newcommand{\calV}{\mathcal{V}}
\newcommand{\calB}{\mathcal{B}}
\newcommand{\calN}{\mathcal{N}}
\newcommand{\calM}{\mathcal{M}}

%>>>>>>>>>>>>>>>>>>>>>>>>>>>>>>
%<<<<<<<<<DEPRECIATED<<<<<<<<<<
%>>>>>>>>>>>>>>>>>>>>>>>>>>>>>>

%%% From Kac's template
% 1-inch margins, from fullpage.sty by H.Partl, Version 2, Dec. 15, 1988.
%\topmargin 0pt
%\advance \topmargin by -\headheight
%\advance \topmargin by -\headsep
%\textheight 9.1in
%\oddsidemargin 0pt
%\evensidemargin \oddsidemargin
%\marginparwidth 0.5in
%\textwidth 6.5in
%
%\parindent 0in
%\parskip 1.5ex
%%\renewcommand{\baselinestretch}{1.25}

%%% From the net
%\newcommand{\pullbackcorner}[1][dr]{\save*!/#1+1.2pc/#1:(1,-1)@^{|-}\restore}
%\newcommand{\pushoutcorner}[1][dr]{\save*!/#1-1.2pc/#1:(-1,1)@^{|-}\restore}














\def\excludeversion#1{\includeversion{#1}}
\includeversion{Frontmatter}
\excludeversion{Endmatter}
\excludeversion{SteenrodAlgebrasAndTheirKoszulDuals}
\excludeversion{CategoriesOfInterest}
\includeversion{DiagramOfFunctors}
\includeversion{GrothendieckSpectralSequences}
\excludeversion{KoszulComplexes2plus}
\excludeversion{KoszulComplexes1}
\excludeversion{DerivedFunctorsLowDimension}
\excludeversion{PRlieKoszulComplexCalculation}
\includeversion{LieLambdaStructureOnKoszul}

\newcommand{\GS}[1]{\scrE^{#1}}
\newcommand{\GpS}[1]{\scrE^{#1}_\star}
\renewcommand{\Set}{\scrE}

\newcommand{\RestLie}[1]{\mathsf{r}{\scrL}^{#1}}%{{\mu\scrL}^{#1}}
\newcommand{\GoodLie}[1]{\mathsf{g}{\scrL}^{#1}}%{{\mu\scrL}^{#1}}
\newcommand{\BadLie}[1]{\mathsf{b}{\scrL}^{#1}}%{{\mu\scrL}^{#1}}
\newcommand{\PRLie}[1]{\scrR^{#1}}%{{\xi\scrL}^{#1}}

\newcommand{\LL}[1]{{\scrK}^{#1}}%{{\mu\scrL}^{#1}}
\newcommand{\GR}[1]{\scrV^{#1}}%{\mathsf{gr}^{#1}}
\newcommand{\nontop}[1]{\scrU^{#1}}%{\mu_{\mathsf{nt}}^{#1}}
\newcommand{\PiAlg}[1]{#1\textup{-$\Pi$-alg}}
\newcommand{\produces}[3]{{#1}{#3}{#2}}

\newcommand{\admis}[1]{\mathrm{adm}(#1)}%{{\mu\scrL}^{#1}}
\newcommand{\restn}[1]{\xi{#1}}
\newcommand{\iteratedrestn}[2]{\xi^{#2}{#1}}
\newcommand{\Ind}[2][]{\mathbf{I}{#2}_{#1}}%[internal indices]{category name}
\newcommand{\forget}{\mathrm{fg}}
\newcommand{\Fr}[1]{#1}%{\mathrm{Fr}_{#1}}
\newcommand{\derived}{\mathbb{L}}
\newcommand{\LambdaOp}{Q}
\renewcommand{\Q}{Q}
\newcommand{\SqShift}{\Sq_{+}}
\newcommand{\Sq}{\mathrm{Sq}}
\newcommand{\LieSteen}{\calA(\calL)}
\newcommand{\CommSteen}{\calA(\calC)}
\newcommand{\deltaAlgebra}{\Delta}
\newcommand{\DyerLashov}{R}

\newcommand{\minDim}{m}
\newcommand{\minDimP}{\overline{m}}
\newcommand{\BarMonomial}[3]{\textup{mon}_{#1,#2,#3}}

\newcommand{\Shuffles}[2]{\textup{Sh}_{#1#2}}
\newcommand{\HalfShuffles}[2]{\textup{Sh}_{#1#2}/2}

\renewcommand{\produces}[3]{
{
\def\labelstyle{\scriptstyle}
\xymatrix@C=2em@1{
{#1}
\ar@{-}[r]|-{{\,#3\,}}
&%r1c1
{#2}%r1c2
}}}





\bibliography{../../Dropbox/logbook/_LOGBOOK/papers}
\begin{document}
\begin{Frontmatter}
\tableofcontents\newpage
\end{Frontmatter}







\begin{SteenrodAlgebrasAndTheirKoszulDuals}
\section{Some homogeneous Koszul algebras}
In what follows, we will make heavy use of four algebras, each of which is a `homogeneous Koszul algebra' in the sense of \cite{PriddyKoszul.pdf}. The algebras of interest are the following:
\begin{enumerate}\squishlist
\setlength{\parindent}{.25in}
\item The Steenrod algebra for simplicial commutative algebras, $\CommSteen$, with generators $P^i$ for $i\geq2$, subject to relations:
\[P^iP^jx=\sum_{s=i-j+1}^{i+j-2}{2s-i-1\choose s-j}P^{i+j-s}P^sx\textup{ for $i\geq2j$}.\]
That this is in fact a description of $\CommSteen$ follows from Goerss' determination \cite[p.14]{MR1089001} of the cohomology of Eilenberg-Mac Lane objects in simplicial augmented commutative $\F_2$-algebras.

\item The ``$\Delta$-algebra'' $\deltaAlgebra$, with generators $\delta_i$ for $i\geq2$, subject to relations:
\[\delta_i\delta_j:=\sum_{s=(i+1)/2}^{(i+j)/3}{j+s-i-1\choose j-s}\delta_{i+j-s}\delta_s\textup{ for $i<2j$.}\]
\item The Steenrod algebra for simplicial restricted Lie algebras, $\LieSteen$, with generators $\SqShift^i$ for $i\geq0$, subject to relations:
\[\SqShift^i\SqShift^j=\sum_{k=0}^{(i-1)/2}{j-k-1\choose i-2k-1}\SqShift^{i+j-k}\SqShift^k\textup{\ for $i\leq2j$.}\]
According to \cite[\S7]{PriddySimplicialLie.pdf}, $\LieSteen$ is the quotient of the standard Steenrod algebra by the two-sided ideal generated by $\Sq^0$, and our description follows after introducing the notation $\SqShift^i:=\Sq^{i+1}$.
%
%This algebra acts on the Andr\'e-Quillen cohomology of restricted Lie algebras, as follows. For $\frakg$ a simplicial Lie algebra,

\item The Dyer-Lashov algebra $\DyerLashov$, with generators $\Q^i$ for $i\geq0$, subject to relations:
\[\Q^i\Q^j=\sum_{k=0}^{(i-2j)/2-1}{i-2j-2-k\choose k}\Q^{2j+1+k}\Q^{i-j-1-k}\textup{\ for $i>2j$.}\]
This algebra is the opposite of the $\Lambda$-algebra, under an anti-isomorphism defined by $Q^i\longleftrightarrow \lambda_i$.
\end{enumerate}
\subsection{Unstable actions on homotopy and Quillen cohomology}
In what follows, let $A$ be a simplicial augmented commutative $\F_2$-algebra $A$, and let $\frakg$ be a simplicial restricted Lie algebra over $\F_2$.

\subsubsection{$\CommSteen$ acting on the cohomology of $\F_2$-algebras}
%The Andr\'e-Quillen cohomology groups $H^*_Q(A)$ of a simplicial $\F_2$-algebra $A$ support the following structure:
\begin{enumerate}\squishlist
\setlength{\parindent}{.25in}
\item $H^*_Q(A)$ is a (bad) Lie algebra, under a bracket $H^i_Q(A)\otimes H^j_Q(A)\to H^{i+j+1}_Q(A)$; the structure is restricted in dimension zero (via $\beta:H^0_Q(A)\to H^1_Q(A)$), good in dimension 1, and bad otherwise;
\item $H^*_Q(A)$ is a graded left $\CommSteen$-module, so that there are operations $P^i:H^n_Q(A)\to H^{n+i+1}_Q(A)$, defined for all $n\geq0$ and $i\geq2$, but zero when $i>n$. These operations satisfy the $P$-Adem relations.
\item Every operation $P^i$ is linear, including the top operation $P^n:H^n_Q(A)\to H^{2n+1}_Q(A)$, which equals the self-bracket:  $P^nx=[x,x]$ if $|x|=n$.
\item The $P$ operations satisfy the following Cartan formula:
$[x,P^iy]=0$.
\end{enumerate}

\subsubsection{$\LieSteen$ acting on the cohomology of restricted Lie algebras}
%The Andr\'e-Quillen cohomology groups $H^*_Q(\frakg)$ of $\frakg$ support the following structure:
\begin{enumerate}\squishlist
\setlength{\parindent}{.25in}
\item $H^*_Q(\frakg)$ is a commutative algebra without unit, under a cup product $H^i_Q(\frakg)\otimes H^j_Q(\frakg)\to H^{i+j+1}_Q(\frakg)$;
\item $H^*_Q(\frakg)$ is a graded left $\LieSteen$-module, so that there are operations $\SqShift^i:H^n_Q(\frakg)\to H^{n+i+1}_Q(\frakg)$, defined for all $n\geq0$ and $i\geq0$, but zero when $i>n$. These operations satisfy the $\SqShift$-Adem relations.
\item Every operation $\SqShift^i$ is linear, including the top operation $\SqShift^n:H^n_Q(A)\to H^{2n+1}_Q(A)$, which equals the squaring operation:  $\SqShift^nx=x^2$ if $|x|=n$.
\item The $\SqShift$ operations satisfy the following Cartan formula:
\[\SqShift^k(x\cdot y)=\sum_{k'+k''=k-1}\SqShift^{k'}(x)\cdot \SqShift^{k''}(y).\]
\end{enumerate}

\subsubsection{$\Delta$ operations on the homotopy of $\F_2$-algebras}
%The homotopy groups $\pi_*(A)$ of $A$ support the following structure:
\begin{enumerate}\squishlist
\setlength{\parindent}{.25in}
\item $\pi_*(A)$ is a commutative algebra under a product $\pi_i(A)\otimes \pi_j(A)\to \pi_{i+j}(A)$; it is exterior in dimension 1, and it is a divided power algebra in dimensions 2 and above;
\item there are operations $\delta_i:\pi_n(A)\to \pi_{n+i}(A)$, defined only for $2\leq i\leq n$. 
As the operations $\delta_i$ are only partially defined, $\pi_*(A)$ does \emph{not} become a left module over the $\Delta$-algebra.
 However, when these operations are written on the left, they satisfy the $\delta$-Adem relations whenever one can be applied.
\item The non-top operations, $\delta_i:\pi_n(A)\to \pi_{n+i}(A)$ for $i<n$, are linear. The top operation, $\delta_n:\pi_n(A)\to \pi_{2n}(A)$ equals the divided square, and is thus quadratic:
\[\textup{if $|x|=|y|=n\geq2$, then }\delta_n(x)=\gamma_2(x)\textup{ and }\delta_n(x+y)=\delta_n(x)+\delta_n(y)+xy.\]
\item The $\delta$ operations satisfy the following Cartan formula:
\[\delta_i(xy)=\begin{cases}
x^2\delta_i(y),&\textup{if }|x|=0\textup{ and }|y|\geq2;\\
y^2\delta_i(x),&\textup{if }|y|=0\textup{ and }|x|\geq2;\\
0,&\textup{otherwise}.
%\\,&\textup{if }
\end{cases}
\]
\end{enumerate}
\subsubsection{$\Q$ operations on the homotopy of restricted Lie algebras}
%The homotopy groups $\pi_*(\frakg)$ of $\frakg$ support the following structure:
\begin{enumerate}\squishlist
\setlength{\parindent}{.25in}
\item $\pi_*(\frakg)$ is a restricted Lie algebra under a bracket $\pi_i(\frakg)\otimes \pi_j(\frakg)\to \pi_{i+j}(\frakg)$;
\item there are operations $\Q^i:\pi_n(\frakg)\to \pi_{n+i}(\frakg)$, defined only for $0\leq i\leq n$. As the operations $\Q^i$ are only partially defined, $\pi_*(\frakg)$ does \emph{not} become a left module over the Dyer-Lashov algebra. However, when these operations are written on the left, they satisfy the $\Q$-Adem relations whenever one can be applied.
\item The non-top operations, $\Q^i:\pi_n(\frakg)\to \pi_{n+i}(\frakg)$ for $i<n$, are linear. The top operation, $\Q^n:\pi_n(\frakg)\to \pi_{2n}(\frakg)$ equals the restriction, and is thus quadratic:
\[\textup{if $|x|=|y|=n\geq0$, then }\Q^n(x)=\restn{x}\textup{ and }\Q^n(x+y)=\Q^n(x)+\Q^n(y)+[x,y].\]
\item The $\Q$ operations satisfy the following Cartan formula. For $|x|\leq |y|$:
\[[\Q^i(x),y]=\begin{cases}
[x,[x,y]],&\textup{if }i=|x|;\\
0,&\textup{otherwise}.
%\\,&\textup{if }
\end{cases}
\]
\end{enumerate}

Note that it is not immediately obvious that the $\delta$-Adem ($\Q$-Adem) relations are well defined when viewed as relations that hold between operations on the homotopy of commuatative algebras (Lie algebras). That is, we are supposed to have $\Q^i\Q^jx=\cdots $ whenever $i>2j$ and $\Q^i\Q^jx$ is defined, but we haven't checked that every term in the right hand side is defined. This does indeed happen, by a lemma to follow. 

\subsection{The combinatorics of admissible sequences}

Now each of these algebras, being homogeneous, is bigraded%footnote:
\footnote{I'm not sure what the point is in bigrading things, really, but there's some cool stuff happening. For example, if you make $P$ act on something that's only singly graded, it makes sense to demand that degrees change via the associtated singly graded algebra, in which $|P^i|=i+1$, and viewing $P$ as bigraded makes the unstable condition seem more natural. Moreover, whenever you see a $\delta_j$ in this context, it does change internal degree by $i$, and homological degree by $1$. What it does to deeper degrees is just spooky. I guess the same thing holds for $\SqShift^i$, which is intense, given that it equals $\Sq^{i+1}$! I don't know.}, setting (\textbf{I guess}) 
\[\|P^i\|=\|\delta_i\|=\|\SqShift^i\|=\|\Q^i\|=(1,i).\]
Now let $b$ be any of the symbols $P$, $\delta$, $\SqShift$, or $\Q$. For a sequence $I=(i_{\ell},\ldots,i_1)$ of positive integers (with $i_j\geq2$ if $b$ is $P$ or $\delta$), write $b^I$ for the product $b^{i_\ell}\cdots b^{i_1}$. Say that $I$ is $b$-admissible if no $b$-Adem relations can be applied directly to $b^I$, so that one of the four evident inequalities hold between adjacent indices in $I$.
\begin{lem*}
All of the Adem relations listed above express the inadmissible length 2 sequence on the left hand side as a sum of admissible length 2 sequences. All four of the algebras are homogeneous Koszul algebras. The algebras $\CommSteen$ and $\Delta$ are Koszul dual, under $P^i\longleftrightarrow\delta_i$. The algebras $\LieSteen$ and $R$ are Koszul dual, under $\SqShift^i\longleftrightarrow\Q^i$. In particular, for length 2 sequences $K$ and $L$:
\[\produces{K}{L}{P}\iff\produces{L}{K}{\delta}\textup{ \ and \ }\produces{K}{L}{\SqShift}\iff\produces{L}{K}{\Q}\]

\end{lem*}
\begin{proof}
The first statement is checked manually from the Adem relations in each case.

Priddy's PBW stuff, along with the fact that none of the above Adem relations do any more than expected, shows that everything is Koszul. To check that the algebras are indeed Koszul dual to each other, one just looks closely at the Adem relations and checks the above `$\iff$' statements
\end{proof}
Thus, for all four of these algebras, the $b$-Adem relations can be used iteratively to write any expression as a sum of terms $b^I$ with $I$ $b$-admissible. We'll write $\produces{I}{J}{b}$ if $I\neq J$, and when $b^I$ is written as a sum of $b$-admissible expressions, $b^J$ appears therein (an odd number of times). Trivially, if $\produces{I}{J}{b}$, then $J$ is $b$-admissible, and $I$ is $b$-inadmissible.

It will be important to have a criterion for when $\delta_Ix$ is defined for an element $x\in\pi_*(A)$, or when $P^Ix$ can be nonzero, for $x\in H^*_Q(A)$, etc.
To answer such questions, we make the following two definitions. For $I=(i_\ell,\ldots,i_1)$ a sequence of nonnegative integers, we define
\[\minDim(I):=\begin{cases}
-\infty,&\textup{if }I=\emptyset;\\
\max\{(i_1),\,(i_2-i_1),\,\ldots,\,(i_{\ell}-i_{\ell-1}-\cdots -i_1)\},&\textup{otherwise};
%\\,&\textup{if }
\end{cases}
\]
\[\minDimP(I):=\begin{cases}
-\infty,&\textup{if }I=\emptyset;\\
\max\{(i_1),\,(i_2-i_1-1),\,(i_3-i_2-i_1-2),\,\ldots,\,(i_{\ell}-\cdots-i_1-\ell+1)\},&\textup{otherwise}.
%\\,&\textup{if }
\end{cases}
\]
If $I=(i_{\ell},\ldots,i_1)$ with all $i_j\geq2$, then:
\begin{itemize}
\setlength{\parindent}{.25in}
\item $P^Ix$ can be nonzero for $x\in H^n_Q(A)$ iff $\minDimP(I)\leq n$
\item $\delta_Ix$ is defined for $x\in \pi_n(A)$ iff $\minDim(I)\leq n$
\end{itemize}
If $I=(i_{\ell},\ldots,i_1)$ with all $i_j\geq0$, then:
\begin{itemize}
\setlength{\parindent}{.25in}
\item $\SqShift^Ix$ can be nonzero for $x\in H^n_Q(\frakg)$ iff $\minDimP(I)\leq n$
\item $\delta_Ix$ is defined for $x\in \pi_n(\frakg)$ iff $\minDim(I)\leq n$
\end{itemize}

%With this notation in hand, we may finally discuss whether or not the Adem relations that the $\delta$ operations and and $\Delta$ are well defined.
\begin{lem*}
In the following statements, $K$ and $L$ are each length 2 sequences:%, with entries nonnegative in cases (i) and (ii), and entries at least 2 in cases (iii) and (iv).
%
%In the following statements, $K=(i,j)$ and $L=(i',j')$ are to be two length 2 sequences, with entries nonnegative in cases (i) and (ii), and entries at least 2 in cases (iii) and (iv).
\begin{enumerate}[i)]
\setlength{\parindent}{.25in}
\item Suppose that $\produces{K}{L}{\Q}$. Then $\minDim(L)<\minDim(K)$, and $L$ has positive entries.
%\item Suppose that $\produces{K}{L}{\SqShift}$. Then $\minDim(L)>\minDim(K)$, and $K$ has positive entries. (shrug)
\item Suppose that $\produces{K}{L}{P}$. Then $\minDimP(L) > \minDimP(K)$.
%\item Suppose that $\produces{K}{L}{\delta}$. Then $\minDimP(L) < \minDimP(K)$. (shrug)
\end{enumerate}
In (ii), the entries of $K$ and $L$ should be at least 2.
\end{lem*}
\begin{proof}
In both of the proofs, let $K=(i,j)$ and $L=(i',j')$.

For (i), we wish to show that $\max\{j',i'-j'\}<\max\{j,i-j\}$, but we must have that $i'\leq 2j'$, and $i>2j$. Thus, we need only show that $j'<i-j$, but this is immediate from the indexing of the Adem relation. That $L$ has positive entries is also immediate.

For (ii), we wish to show that $\max\{j',i'-j'-1\}>\max\{j,i-j-1\}$, but we must have that $i'> 2j'$, and $i\geq2j$. Thus $\max\{j,i-j-1\}$ at most $i-j$ (which happens when $i=2j$). Thus, we need only show that $j'>i-j$; again, easy.
\end{proof}
%\begin{lem*}
%Suppose that %$i>2j$, and 
%$\produces{(i,j)}{(i',j')}{\Q}$. Then $\minDim(i,j)>
%\minDim(i',j')$, and $i',j'>0$.
%
%In particular, if $\Q^i\Q^jx$ is defined, and $i>2j$, it can be rewritten as a sum of terms $\Q^{i'}\Q^{j'}x$ in which neither of the $\Q^{i'}$ and $\Q^{j'}$ that appear are top operations.
%\end{lem*}
%\begin{proof}
%It must be that $(i,j)$ is $\Q$-inadmissible, so that $i>2j$. Then,
%by inspection, $i>j+i/2\geq i'\geq2j+1$, and $i-j-1\geq j'\geq i/2>j$. These inequalities demonstrate (i). For (ii), note that (i) shows that we needn't worry about $\Q^0$ failing to be defined, as it won't appear. Our only concern is that either $i'$ or $j'$ will be too large. Suppose then that $\Q^i\Q^j$ is defined on an $n$-dimensional class, so that $j\leq n$ and $i\leq n+j$. Then one notes that $j'\leq i-j-1\leq n-1$, and $i'-j'\leq (j+i/2)-(i/2)<i/2\leq(n+j)/2\leq n$. The first chain gives $j'\leq n-1$, and the second $i'\leq n-1+j'$, as the quantities involved are integers. These inequalities demonstrate (ii) and the final comment.
%\end{proof}
%
%\begin{lem*}
%Suppose that $\produces{(i,j)}{(i',j')}{P}$. Then $\minDimP(i',j') > \minDimP(i,j)$.
%\end{lem*}
%\begin{proof}
%Since $i\geq2j$, 
%$j'\geq i-j+1=
%j+(i-2j+1)=
%i-j-1+(2)
%$, so that $j'>\max\{j,i-j-1\}$.
%\end{proof}


%
%\subsection{The Steenrod algebra for simplicial Lie algebras and its Koszul dual}
%Let  $\LieSteen$ be the Steenrod algebra for simplicial restricted Lie algebras \cite[\S7]{PriddySimplicialLie.pdf}. This is the quotient of the standard Steenrod algebra by the two-sided ideal generated by $\Sq^0$. We'll introduce the notation $\SqShift^i:=\Sq^{i+1}$, so that $\LieSteen$ is generated by $\SqShift^i$ for $i\geq-1$, subject to relations:
%\[\SqShift^{-1}=0\textup{ and }\SqShift^i\SqShift^j=\sum_{k=0}^{(i-1)/2}{j-k-1\choose i-2k-1}\SqShift^{i+j-k}\SqShift^k\textup{\ for $i\leq2j$.}\]
%This is a homogeneous Koszul algebra in the sense of \cite{PriddyKoszul.pdf}. Its cohomology algebra $H^*\LieSteen:=\Ext_{\LieSteen}^{**}(\F_2,\F_2)$ is also a homogeneous Koszul algebra, generated by the $\Q^i$ for $i\geq0$, where $\Q^i$ is dual to $\SqShift^i$ \cite[\S8]{PriddySimplicialLie.pdf}. These classes satisfy their own Adem relations%footnote:
%\footnote{Compare with \cite[p.143]{Miller-infDeloopSS.pdf}, and May-Cohen-Lada page 9.}:
%\[\Q^i\Q^j=\sum_{k=0}^{(i-2j)/2-1}{i-2j-2-k\choose k}\Q^{2j+1+k}\Q^{i-j-1-k}\textup{\ for $i>2j$.}\]
%Importantly, $H^*\LieSteen$ is isomorphic to the opposite of the $\Lambda$-algebra, and $\Q^i\longleftrightarrow\lambda_i$ under this anti-isomorphism. While it is not precisely true that the $\Lambda$-algebra has an unstable right action on the homotopy of any simplicial Lie algebra, there are various $\lambda$ operators acting thereupon, satisfying the Adem relations in the $\Lambda$-algebra when written on the right \cite[Prop 8.6]{CurtisSimplicialHtpy.pdf}. We prefer to think of this as a left action of various $\Q$ operators, in order to simplify notation. Details to follow.
%
%\subsection{The Steenrod algebra for simplicial commutative algebras and its Koszul dual}
%It follows from Goerss' determination \cite[p.14]{MR1089001} of the cohomology of Eilenberg-Mac Lane objects in simplicial augmented commutative $\F_2$-algebras that the corresponding Steenrod algebra, $\CommSteen$ may be described thus.
%Let $\CommSteen$ be the algebra generated by $P^i$ for $i\geq0$ subject to relations:
%\[P^0=0,\ P^1=0\textup{ and }P^iP^jx=\sum_{s=i-j+1}^{i+j-2}{2s-i-1\choose s-j}P^{i+j-s}P^sx\textup{ for $i\geq2j$}.\]
%The Quillen cohomology of a simplicial augmented $\F_2$-algebra has an unstable left action of this algebra. This is also a homogeneous Koszul algebra, as is its cohomology algebra $H^*\CommSteen=\deltaAlgebra$. The algebra $\Delta$ is generated by classes $\delta_i$ dual to $P^i$ for $i\geq 2$. These classes satisfy the following $\delta$-Adem relation:
%\[\delta_i\delta_j:=\sum_{s=(i+1)/2}^{(i+j)/3}{j+s-i-1\choose j-s}\delta_{i+j-s}\delta_s\textup{ for $i<2j$.}\]
%It is again not precisely true that $\deltaAlgebra$ acts on the homotopy of simplicial commutative algebras. There are, however, various $\delta$ operations acting on the homotopy of a simplicial commutative algebra, and they satisfy the $\delta$-Adem relation wherever it makes sense.
\end{SteenrodAlgebrasAndTheirKoszulDuals}

\begin{CategoriesOfInterest}
\section{Categories of interest}
In the following subsections we detail a number of algebraic categories relevant to our study.
\subsection{Categories of graded sets and vector spaces}
\begin{itemize}
\setlength{\parindent}{.25in}
\item For $n\geq1$, let $\GpS{n}$ be the category of $n$-times non-negatively graded pointed sets $X$ whose graded part $X_{a_n,\ldots,a_1}$ is a single point whenever $a_1=0$. That is, an object $X\in\GpS{n}$ is a collection $X_{a_n,\ldots,a_1}$ of pointed sets, one for each $n$-tuple $(a_n,\ldots,a_1)$ of non-negative integers, such that $X_{a_n,\ldots,a_2,0}=\{\star\}$ for all $a_n,\ldots,a_2$. For $x\in X_{a_n,\ldots,a_1}$, we write $|x|:=a_n$.
\item For $n\geq1$, let $\GS{n}$ be the category of $n$-times non-negatively graded sets $X$ whose graded part $X_{a_n,\ldots,a_1}$ is empty whenever $a_1=0$. For $x\in X_{a_n,\ldots,a_1}$, we write $|x|:=a_n$.

\item For $n\geq1$, let $\GR{n}$ be the category of $n$-times non-negatively graded $\F_2$-vector spaces $V$ such that $V_{a_n,\ldots,a_1}=0$ whenever $a_1=0$.  For a nonzero homogeneous element $v\in V_{a_n,\ldots,a_1}$, write $|v|:=a_n$.
\end{itemize}
All the categories $\mathsf{C}=\mathsf{C}^n$ to be defined below will consist of objects $X$ of $\GR{n}$ enriched with certain extra structure. For $y\in Y_{a_n,\ldots,a_1}$ a non-zero homogeneous element of $Y$, we will continue to write $|y|:=a_n$ in each of the following contexts.
Moreover, for any of the following categories there are free/forget adjunctions $\GS{n} \rightleftarrows \mathsf{C}$ and $\GpS{n} \rightleftarrows \mathsf{C}$. We'll denote either of these free functors simply by $X\mapsto\Fr{\mathsf{C}}(X)$.


\subsection{Categories of graded Lie algebras}

\begin{itemize}
\setlength{\parindent}{.25in}
\item For $n\geq1$, let $\BadLie{n}$ be the following category of $n$-graded Lie algebras over $\F_2$. An object of $\BadLie{n}$ is to be an object $L$ of $\GR{n}$ with (skew-)symmetric maps
\[[\DASH,\DASH]:L_{a_n,\ldots,a_1}\otimes L_{b_n,\ldots,b_1}\to L_{a_n+b_n,\ldots,a_2+b_2,a_1+b_1+1}\]
satisfying the Jacobi identity. Note that this Lie bracket has degree $(0,\ldots,0,1)$. As the ground field has characteristic 2, the concepts of symmetry and skew-symmetry coincide. More importantly, skew-symmetry does not imply the `alternating' relation, so that there may be $x\in L$ with $[x,x]$ nonzero. This explains the $\mathsf{b}$ in the notation --- we refer to such Lie algebras as \emph{bad} Lie algebras.

\item For $n\geq1$, let $\GoodLie{n}$ be the full subcategory of $\BadLie{n}$ consisting of those Lie algebras which are \emph{good}. That is, those $L\in\BadLie{n}$ for which $[x,x]=0$ for all $x\in L$.
\item For $n\geq1$, let $\RestLie{n}$ be the category of restricted Lie algebras in $\GoodLie{n}$. That is, an object of $\RestLie{n}$ is an object $L\in\GoodLie{n}$ along with `restriction' operations
\[\restn{}:L_{a_n,\ldots,a_1}\to L_{2a_n,2a_{n-1},\ldots,2a_2,2a_1+1}\]
which satisfy, for $x,y\in L_{a_n,\ldots,a_1}$ and $z\in L$: \[\restn{(x+y)}=\restn{(x)}+\restn{(y)}+[x,y]\textup{ and }[\restn{x},z]=[x,[x,z]].\]
\item For $n\geq1$, let $\PRLie{n}$ be the following category of partially restricted $n$-graded \emph{good} Lie algebras. An object of $\PRLie{n}$ consists is an object $L$ of $\GoodLie{n}$
%For $n\geq1$, let $\PRLie{n}$ be the following category of partially restricted $n$-graded Lie algebras. An object of $\PRLie{n}$ is a good Lie algebra $L=\bigoplus L_{a_n,\ldots,a_1}$, under a $(0,\ldots,0,1)$-shifted bracket as above
admitting restriction operations
\[\restn{}:L_{a_n,\ldots,a_1}\to L_{2a_n,2a_{n-1},\ldots,2a_2,2a_1+1}\]
defined whenever not all of $a_n,\ldots,a_{2}$ are zero. %The graded part $L_{a_n,\ldots,a_1}$ is assumed to be zero whenever $a_1=0$, and we write $|x|$ as above. 
For $x,y\in L_{a_n,\ldots,a_1}$ and $z\in L$ the restriction operations satisfy \[\restn{(x+y)}=\restn{(x)}+\restn{(y)}+[x,y]\textup{ and }[\restn{x},z]=[x,[x,z]].\]
Note that when $n=1$ the restriction is not defined anywhere.
\end{itemize}
We have defined the bracket and restrictions on homogeneous elements only, but there is no problem extending these operations definition to non-homogeneous elements using the bilinearity of the bracket and the above formula for the restriction of a sum.
%We can define $\restn{}$ on the the direct sum of those $L_{a_n,\ldots,a_1}$ for which not all of $a_n,\ldots,a_2$ are zero, using the above formula for $\restn{(x+y)}$. Then both of the above two formulae will still hold for non-homogeneous $x,y,z$. Thus, we could equally have defined $\PRLie{n}$ by giving a restriction on this entire ideal.
%\begin{prop*}
%If $X\in\GS{n}$ a graded set, the `Hall basis' of $\Fr{\GoodLie{n}}(X)$ \cite{MR0038336} consists of certain `standard monomials' $b$ in the elements of $X$. Then $\Fr{\BadLie{n}}(X)$ has a basis $\{b\}\cup\{[b,b]\}$.
%\end{prop*}

If $X\in\GS{n}$ a graded set, one may choose a `Hall basis' of $\Fr{\GoodLie{n}}(X)$ \cite{MR0038336} consisting of certain `standard monomials' $b$ in the elements of $X$.
\begin{prop*}
Let $X\in\GS{n}$ be a graded set, and let $B$ be a Hall basis for $\Fr{\GoodLie{n}}(X)$ derived from $X$. Note that $B\subset \Fr{\GoodLie{n}}(X)$ is actually a graded set. Then:
\begin{enumerate}[i)]\squishlist
\setlength{\parindent}{.25in}
\item $\Fr{\BadLie{n}}(X)$ has a basis $B\cup\{[b,b]\,:\,b\in B\}$;
\item $\Fr{\RestLie{n}}(X)$ has a basis $\{\iteratedrestn{b}{n}\,:\,b\in B,\,n\geq0\}$; and
\item $\Fr{\PRLie{n}}(X)$ has a basis $B\cup \{\iteratedrestn{b}{n}\,:\,b\in B_{a_{n},\ldots,a_1},\,n\geq1,\textup{ not all of $a_n,\ldots,a_2$ are zero}\}$.
\end{enumerate}
\end{prop*}
\begin{proof}
Part (ii) is standard and can be found in \cite[Proposition 14, p.66]{MR886063}.
For part (i), note that the Jacobi identity implies that $[x,[y,y]]$ is identically zero in any bad Lie algebra.
%For part (i), note that in any Lie algebra in characteristic 2, $[x,[y,y]]$ is zero, by anti-symmetry in odd or zero characteristic, and by the Jacobi identity in characteristic 2. 
In particular, the proposed basis must span $\Fr{\BadLie{n}}(X)$. On the other hand, as one easily defines a Lie algebra structure on the vector space spanned by the symbols $B\cup\{[b,b]\}$, they must be linearly independent in $\Fr{\BadLie{n}}(X)$, proving (i).

For (iii), the proposed basis again must span $\Fr{\PRLie{n}}(X)$, and can be seen to be linearly independent by considering the map $\Fr{\PRLie{n}}(X)\to \Fr{\RestLie{n}}(X)$ of partially restricted Lie algebras.
\end{proof}


\subsection{Homotopy operations for simplicial Lie algebras}
The goal of this section is to determine the natural operations on the homotopy of a simplicial object in the category $\mathsf{C}$, for $\mathsf{C}=\mathsf{C}^n$ any of the above categories of graded Lie algebras. This is equivalent to calculating the homotopy of `wedges of spheres'.

That is, suppose that $X\in\GS{n+1}$ is a graded set. Then one can form a simplicial pointed graded set $S(X)\in s\GpS{n}$ by the formula
\[(S(X))_{a_n,\ldots,a_1}=\bigvee_{x\in X_{d,a_n,\ldots,a_1}}S^d_{a_n,\ldots,a_1}\]
Here, $S^d_{a_n,\ldots,a_1}$ is the standard simplicial $d$-sphere $\Delta^d/\partial\Delta^n$ living in the $a_n,\ldots,a_1$ grading.
Then $\Fr{\mathsf{C}}S(X)\in s\mathsf{C}$ represents the following functor $s\mathsf{C}\to\AbGp$
\[C\mapsto \prod_{x\in X_{d,a_n,\ldots,a_1}}\pi_d(C)_{a_n,\ldots,a_1}\]
where $\pi_d(C)_{a_n,\ldots,a_1}$ is the grading $a_n,\ldots,a_1$ part of $\pi_d(C)$. In particular, by Yoneda's lemma:
\[\left\{\prod_{x\in X_{d,a_n,\ldots,a_1}}\pi_d(\DASH)_{a_n,\ldots,a_1}\overset{\textup{nat}}{\to}\pi_D(\DASH)_{A_n,\ldots,A_1}\right\}\cong\pi_D\left(\Fr{\mathsf{C}}S(X)\right)_{A_n,\ldots,A_1}\]
Thus we're interested in the homotopy groups of $\mathsf{C}S(X)$. These are well known when $\mathsf{C}$ is $\GoodLie{n}$ or $\RestLie{n}$, and from these calculations we will deduce the result when $\mathsf{C}=\PRLie{n}$. Before stating the results, we recall certain well known operations on the homotopy of good and restricted Lie algebras, as discussed in \cite[Thm 7.12, Prop 8.8]{CurtisSimplicialHtpy.pdf}:
\begin{prop*}
For $X$ a simplicial object either in $\GoodLie{n}$ or $\RestLie{n}$, $\pi_{*}(X)$ naturally becomes a good Lie algebra (in $\GoodLie{n+1}$) under a bracket induced by the shuffle map. Moreover, $\pi_{*}(X)$ has natural operations
\[\Q^i:\pi_{a_{n+1}}(X)_{a_n,\ldots,a_1}\to \pi_{a_{n+1}+i}(X)_{2a_n,\ldots,2a_2,2a_1+1}
\]
defined whenever $0<i\leq a_{n+1}$ and when $i=0$ if $X\in\RestLie{n}$. The following relations hold between the various operations (whenever the left hand side of each relation is defined).
%\[\textup{for $i>2j,$\ \ }\Q^i\Q^jx=\sum_{k=0}^{(i-2j)/2-1}{i-2j-2-k\choose k}\Q^{2j+1+k}\Q^{i-j-1-k}x.\]
%\[\Q^i{(x+y)}=\Q^i{x}+\Q^i{y}\textup{ and }[\Q^i{x},z]=0\textup{ for $i<|x|=|y|$ and any $z$}\]
%\[\Q^i(x+y)=\Q^ix+\Q^iy+[x,y]\textup{ and }[\Q^{i}x,z]=[x,[x,z]]\textup{ for $i=|x|=|y|$ and any $z$}.\]
The $\Q$ operations satisfy the $\Q$-Adem relations:%, so that if $i>2j$, and $\Q^i\Q^jx$ is defined, then:
\[\Q^i\Q^jx=\sum_{k=0}^{(i-2j)/2-1}{i-2j-2-k\choose k}\Q^{2j+1+k}\Q^{i-j-1-k}x\textup{ for }i>2j.\]
The non-top operations are linear and are killed by the bracket.
That is, for $x,y\in \pi_{a_{n+1}}(X)_{a_n,\ldots,a_1}$, $z\in \pi_*(X)$ and $i<a_{n+1}$: \[\Q^i{(x+y)}=\Q^i{x}+\Q^i{y}\textup{ and }[\Q^i{x},z]=0.\]
On the other hand, the top $\Q$ operation, 
acts as a (partially defined) restriction, so that, for $x,y\in \pi_i(X)_{ a_{n},\ldots,a_1}$ and $z\in \pi_*(X)$,
\[\Q^i(x+y)=\Q^ix+\Q^iy+[x,y]\textup{ and }[\Q^{i}x,z]=[x,[x,z]].\]

%
%
%
%
% The top $\Q^i$ acts as a restriction operation, so that $\Q^{|x|}(x)=\restn(x)$ whenever both sides are defined. The non-top operations are linear and are killed by the bracket.
%That is, for $x,y\in L_{a_n,\ldots,a_1}$, $i<a_n$ (with not all of $i,a_{n-1},\ldots,a_{2}$ being zero) and $z\in L$: \[\Q^i{(x+y)}=\Q^i{x}+\Q^i{y}\textup{ and }[\Q^i{x},z]=0.\]
%Finally, the $\Q$ operations are assumed to satisfy the following $\Q$-Adem relations wherever they make sense. If $i>2j$, and $\Q^i\Q^jx$ is defined (i.e.\ $j\leq|x|$, not all of $j,a_{n-1},\ldots,a_2$ are zero, and $i\leq|x|+j$):
%\[\Q^i\Q^jx=\sum_{k=0}^{(i-2j)/2-1}{i-2j-2-k\choose k}\Q^{2j+1+k}\Q^{i-j-1-k}x.\]
\end{prop*}

\noindent Again, it is not immediately obvious that $\Q$-Adem relations are well defined, but this follows from the same lemma as previously, even with the restrictions on $\Q^0$. 
%\begin{lem*}
%Suppose that %$i>2j$, and 
%$\produces{(i,j)}{(i',j')}{\Q}$. Then
%\begin{enumerate}[i)]\squishlist
%\setlength{\parindent}{.25in}
%\item Both $i'$ and $j'$ are nonzero.
%\item If $\Q^i\Q^j$ is defined on a class in $\pi_n (L)$ for $L$ a simplicial Lie algebra, then $\Q^{i'}\Q^{j'}$ is defined on all of $\pi_{n-1}(L)$, which is to say that $\minDim(i',j')<
%\minDim(i,j)$.
%\end{enumerate}
%In particular, if $\Q^i\Q^jx$ is defined, and $i>2j$, it can be rewritten as a sum of terms $\Q^{i'}\Q^{j'}x$ in which neither of the $\Q^{i'}$ and $\Q^{j'}$ that appear are top operations.
%\end{lem*}
With these operations in hand, we can express the homotopy groups of interest as follows:
\begin{prop*}
Suppose that $X\in\GS{n+1}$ is a graded set. Let $B$ be a Hall basis for the free Lie algebra $\Fr{\GoodLie{n+1}}(X)$. Note that $B\subset \Fr{\GoodLie{n}}(X)$ is in particular a graded set.
\begin{enumerate}[i)]\squishlist
\setlength{\parindent}{.25in}
\item $\pi_*(\Fr{\GoodLie{n}}S(X))$ has basis $\{\Q^Ib\,:\,b\in B,\,I\in\admis{\Q},\ 0<i_1\leq|b|\}$;
\item $\pi_*(\Fr{\RestLie{n}}S(X))$ has basis $\{\Q^Ib\,:\,b\in B,\,I\in\admis{\Q},\ 0\leq i_1\leq|b|\}$; and
\item $\pi_*(\Fr{\PRLie{n}}S(X))$ has basis $\{\Q^Ix\,:\,b\in B_{a_{n+1},\ldots,a_1},\,I\in\admis{\Q},\ i_1\leq|b|,\ i_1,a_{n},\ldots,a_2\textup{ not all zero}\}$.
\end{enumerate}
We always allow $I$ to be the empty sequence in the above bases, in which case the condition on $i_{1}$ is to be ignored. If $I$ is nonempty, we mean that $I=(i_\ell,\ldots,i_1)$, so $i_{1}$ is the rightmost entry of $I$.
\end{prop*}
The first two parts assert that the natural homotopy operations in the homotopy of a simplicial good or restricted Lie algebra are generated by the operations of the previous proposition, and are subject only to those relations stated. Part (iii) asserts that the natural homotopy operations for partially restricted Lie algebras interpolate between those for good and those for restricted Lie algebras in the obvious way.

Part (iii) demonstrates that $\PiAlg{\PRLie{n}}$, the natural target category for the functor $\pi_*$ on $s\PRLie{n}$, is in fact the category $\LL{n+1}$ that we'll define in the next section.
\begin{proof}[Proof of proposition]
See
\cite[Thm 8.8 and proof]{CurtisSimplicialHtpy.pdf} for the proof of parts (i) and (ii).

For any set $Y\in\GpS{n}$, write $Y^\textup{z}\in\GpS{n}$ for the object defined by
\[(Y^\textup{z})_{a_n,\ldots,a_1}=\begin{cases}
Y_{0,\ldots,0,a_1},&\textup{if }a_n=\cdots =a_2=0;\\
\star,&\textup{otherwise}.
%\\,&\textup{if }
\end{cases}
\]
This assignment is natural in $Y$, and there is a natural surjection $\Fr\RestLie{n}(Y)\to\Fr\RestLie{n}(Y^\textup{z})$ of restricted Lie algebras. The above calculation of bases for the free constructions in $\GoodLie{n}$, $\RestLie{n}$ and $\PRLie{n}$, shows that the kernel of the composite $\Fr\RestLie{n}(Y)\to\Fr\RestLie{n}(Y^\textup{z})\to\Fr\RestLie{n}(Y^\textup{z})/\Fr\GoodLie{n}(Y^\textup{z})$ is precisely $\Fr{\PRLie{n}}(Y)$.

Thus we have short exact sequences of simplicial vector spaces:
\[\xymatrix@R=4mm{
0
\ar[r]
&%r1c1
\Fr{\PRLie{n}}S(X)
\ar[r]^\gamma
&%r1c2
\Fr{\RestLie{n}}S(X)
\ar[r]^-(.4)\beta&%r1c3
\Fr{\RestLie{n}}S(X)/\Fr{\PRLie{n}}S(X)
\ar[r]
\ar[d]^-{\cong}
&%r1c4
0\\%r1c5
0
\ar[r]
&%r1c1
\Fr{\GoodLie{n}}S(X^\textup{z})
\ar[r]^{\alpha}
&%r1c2
\Fr{\RestLie{n}}S(X^\textup{z})
\ar[r]&%r1c3
\Fr{\RestLie{n}}S(X^\textup{z})/\Fr{\GoodLie{n}}S(X^\textup{z})
\ar[r]
&%r1c4
0
}\]
Now parts (i) and (ii) show that $\alpha_*$ is a monomorphism, so that the LES for the bottom row splits into short exact sequences, and $\beta_*$ maps the subspace of $\pi_*(\Fr{\RestLie{n}}S(X))$ spanned by
\[C=\{\Q^Ib\in\pi_*(\Fr{\RestLie{n}}S(X))\,:\,b\in B_{0,\ldots,0,a_1},\,I\in\admis{\Q},\ 0= i_1\leq|b|\}\]
isomorphically onto the homotopy of the final term. This shows that the LES for the top row splits. Finally, we can construct classes 
\[D=\{\Q^Ix\in\pi_*(\Fr{\PRLie{n}}S(X))\,:\,b\in B_{a_{n+1},\ldots,a_1},\,I\in\admis{\Q},\ i_1\leq|b|,\ i_1,a_{n},\ldots,a_2\textup{ not all zero}\}\]
in $\pi_*(\Fr{\PRLie{n}}S(X))$ using the same formulae as in $\pi_*(\Fr{\RestLie{n}}S(X))$. Thus $\gamma(D)$ and $C$ together form the basis of $\pi_*(\Fr{\RestLie{n}}S(X))$ given in part (ii), and $D$ must be a basis for $\pi_*(\Fr{\PRLie{n}}S(X))$.
% and the homotopy of the rightmost terms has basis
%\[C=\{\Q^Ib\,:\,b\in B_{0,\ldots,0,a_1},\,I\in\admis{\Q},\ 0= i_\ell\leq|b|\}.\]
%Moreover, for any $\Q^Ib$ appearing, $\beta_*(\Q^Ib)=\Q^Ib$, so that $\beta_*$ is an epimorphism, so that the LES for the top row splits, and $\ker(\beta_*)=\pi_*(\Fr{\PRLie{n}}S(X))$.
%Finally, we can construct the classes 
%\[\{\Q^Ix\,:\,b\in B_{a_{n+1},\ldots,a_1},\,I\in\admis{\Q},\ i_\ell\leq|b|,\ i_\ell,a_{n},\ldots,a_2\textup{ not all zero}\}\]
%in $\pi_*(\Fr{\PRLie{n}}S(X))$ using the same formulae as one does in $\pi_*(\Fr{\RestLie{n}}S(X))$, so that the image of $\gamma_*$ contains all the basis elements of $\pi_*(\Fr{\RestLie{n}}S(X))$ that
\end{proof}



%Then \cite[Thm 8.6, Thm 8.8]{CurtisSimplicialHtpy.pdf} shows that $\pi_*(\Fr{\RestLie{n}}(V))$ is an element of $\LL{n}$, and has a basis consisting of tensors $\Q^{I}\otimes b$, for $b\in B$, and $I\in\admis{\Q}$ with $i_\ell\leq|b|$.




\subsection{Categories of graded Lie algebras with unstable operations}
\begin{itemize}
\setlength{\parindent}{.25in}
%\item Let $\LL{1}$ be the full subcategory of connected objects in $\calW$. We already have the definition for these in Goerss' book, but I'll spell this out here, to emphasise certain interesting aspects of the character of $\LL{1}$.
%An object of $\LL{1}$ is a (potentially bad) graded Lie algebra $L$ in $\BadLie{1}$ with an unstable (linear) action of $\CommSteen$.
%That is, $L$ admits $P$ operations
%\[P^i:L_{a_1}\to L_{a_1+i+1}\]
%which are \emph{always} defined for $i\geq0$, but \emph{equal zero} unless $2\leq i\leq a_n$.
%%By `bad' Lie algebra, we mean that the bracket satisfies the Jacobi identity, and is skew symmetric (``$[x,y]=[y,x]$ for all $x,y$'').
%%The graded part $L_{0}$ is assumed to be zero, and for $x\in L_{a_1}$ we write $|x|:=a_1$. 
%The $P$ operations are assumed to satisfy the following $P$-Adem relations. If $i\geq 2j$:
%\[P^iP^jx=\sum_{s=i-j+1}^{i+j-2}{2s-i-1\choose s-j}P^{i+j-s}P^sx.\]
%Every $P$ operation is assumed linear, and $[x,P^iy]=0$ for all $i,x,y$. Additionally, the top $P$ operation is the `self-square': for $x\in L_{i}$, $P^{i}x=[x,x]$.
\item Let $\LL{1}$ be the following category of graded bad Lie algebras which are also unstable left $\Q$-modules.
An object of $\LL{1}$ is a (potentially) bad graded Lie algebra $L$ in $\BadLie{1}$ %with an unstable (linear) action of $\CommSteen$.
admitting linear operations
\[P^i:L_{a_1}\to L_{a_1+i+1}\]
which are \emph{always} defined for $i\geq0$, but \emph{equal zero} unless $2\leq i\leq a_n$. They satisfy the $P$-Adem operations when written on the left. Finally, $[x,P^iy]=0$ for all $i,x,y$, and the top $P$ operation is the `self-square': for $x\in L_{i}$, $P^{i}x=[x,x]$.

Note that this is just the full subcategory of connected objects in Goerss' category $\calW$. See \cite[p.14]{MR1089001}. \textbf{I might make them only defined for $i\geq2$ and stick $[x,x]=0$ on for $|x|=1$.}

\item Let $\nontop{1}$ be the following category of graded unstable $P$-modules. An object of $\nontop{1}$ is a graded vector space $M$ in $\GR{1}$,
admitting linear operations
\[P^i:M_{a_1}\to M_{a_1+i+1}\]
which are always defined, but are zero unless $2\leq i\leq a_n$.
%The graded part $M_{0}$ is assumed to be zero, and for $x\in M_{a_n,\ldots,a_1}$ we write $|x|:=a_n$. 
The $P$ operations are assumed to satisfy the same $P$-Adem relations as in $\LL{1}$. %Every operation $P^i$ is assumed linear.
\item For $n\geq2$, let $\LL{n}$ be the following category of $n$-graded partially restricted Lie algebras with unstable $\Q$ operators acting from the left. An object of $\LL{n}$ consists of an object $L$ of $\PRLie{n}$ 
admitting $\Q$ operations
\[\Q^i:L_{a_n,\ldots,a_1}\to L_{a_n+i,2a_{n-1},\ldots,2a_2,2a_1+1}\]
defined whenever $i\leq a_n$ and not all of $i,a_{n-1},\ldots,a_{2}$ are zero. The top $\Q^i$ \emph{equals} the restriction operation, so that $\Q^{|x|}(x)=\restn(x)$ whenever both sides are defined. The non-top operations are linear and are killed by the bracket.
That is, for $x,y\in L_{a_n,\ldots,a_1}$, $i<a_n$ (with not all of $i,a_{n-1},\ldots,a_{2}$ being zero) and $z\in L$: \[\Q^i{(x+y)}=\Q^i{x}+\Q^i{y}\textup{ and }[\Q^i{x},z]=0.\]
Finally, the $\Q$ operations are assumed to satisfy the following $\Q$-Adem relations wherever they make sense. If $i>2j$, and $\Q^i\Q^jx$ is defined (i.e.\ $j\leq|x|$, not all of $j,a_{n-1},\ldots,a_2$ are zero, and $i\leq|x|+j$):
\[\Q^i\Q^jx=\sum_{k=0}^{(i-2j)/2-1}{i-2j-2-k\choose k}\Q^{2j+1+k}\Q^{i-j-1-k}x.\]
%\item For $n\geq2$, let $\LL{n}$ be the following category of $n$-graded \emph{good} Lie algebras with unstable $\Q$ operators acting from the left. An object of $\LL{n}$ consists of an object $L$ of $\GoodLie{n}$
%%, under a $(0,\ldots,0,1)$-shifted bracket:
%%\[[\DASH,\DASH]:L_{a_n,\ldots,a_1}\otimes L_{b_n,\ldots,b_1}\to L_{a_n+b_n,\ldots,a_2+b_2,a_1+b_1+1}\]
%admitting $\Q$ operations
%\[\Q^i:L_{a_n,\ldots,a_1}\to L_{a_n+i,2a_{n-1},\ldots,2a_2,2a_1+1}\]
%defined whenever $i\leq a_n$ and not all of $i,a_{n-1},\ldots,a_{2}$ are zero. %By `good' Lie algebra, we mean that the bracket satisfies the Jacobi identity, and is alternating (``$[x,x]=0$ for all $x$'').
%%The graded part $L_{a_n,\ldots,a_1}$ is assumed to be zero whenever $a_1=0$, and for $x\in L_{a_n,\ldots,a_1}$ we write $|x|:=a_n$.
%The $\Q$ operations are assumed to satisfy the following $\Q$-Adem relations wherever they make sense. If $i>2j$, and $\Q^i\Q^jx$ is defined (i.e.\ $j\leq|x|$, not all of $j,a_{n-1},\ldots,a_2$ are zero, and $i\leq|x|+j$):
%\[\Q^i\Q^jx=\sum_{k=0}^{(i-2j)/2-1}{i-2j-2-k\choose k}\Q^{2j+1+k}\Q^{i-j-1-k}x.\]
%The non-top operations are linear and are killed by the bracket.
%That is, for $x,y\in L_{a_n,\ldots,a_1}$ and $z\in L$: \[\Q^i{(x+y)}=\Q^i{x}+\Q^i{y}\textup{ and }[\Q^i{x},z]=0.\]
%On the other hand, the top operation, 
%\[\Q^i:L_{i, a_{n-1},\ldots,a_1}\to L_{2i, 2a_{n-1},\ldots,2a_2,2a_1+1}\]
%acts as a partially defined restriction for $L$, satsfying, for $x,y\in L_{i, a_{n-1},\ldots,a_1}$ and $z\in L$,
%\[\Q^i(x+y)=\Q^ix+\Q^iy+[x,y]\textup{ and }[\Q^{i}x,z]=[x,[x,z]].\]

\item For $n\geq2$, let $\nontop{n}$ be the following category of $n$-graded vector spaces with unstable $\Q$ operators acting from the left. An object of $\nontop{n}$ is an $n$-graded vector space $M$ in $\GR{n}$,
admitting linear operations
\[\Q^i:M_{a_n,\ldots,a_1}\to M_{a_n+i,2a_{n-1},\ldots,2a_2,2a_1+1}\]
defined whenever $i< a_n$ (note the strict inequality) and not all of $i,a_{n-1},\ldots,a_{2}$ are zero.
%The graded part $M_{a_n,\ldots,a_1}$ is assumed to be zero whenever $a_1=0$, and for $x\in M_{a_n,\ldots,a_1}$ we write $|x|:=a_n$. 
The $\Q$ operations are assumed to satisfy the same $\Q$-Adem relations as in $\LL{n}$, wherever they make sense in $\nontop{n}$. %The same discussion shows that this too is a meaningful definition. %Every operation $\Q^i$ is assumed linear.
\end{itemize}


\begin{prop*}
Let $X\in\GS{n}$ be a graded set, and let $B$ be a Hall basis for $\Fr{\GoodLie{n}}(X)$ derived from $X$. Note that $B\subset \Fr{\GoodLie{n}}(X)$ is in particular a graded set. Then when $n=1$:
\begin{enumerate}[i)]\squishlist
\setlength{\parindent}{.25in}
\item $\Fr{\LL{1}}(X)$ has basis $\{P^Ib\,:\,b\in B,\,I=(i_\ell,\ldots,i_1)\textup{ is $P$-admissible with }i_1\leq|b| \}$;
\item $\Fr{\nontop{1}}(X)$ has basis $\{P^Ix\,:\,x\in X,\,I=(i_\ell,\ldots,i_1)\textup{ is $P$-admissible with }i_1\leq|x| \}$;
\end{enumerate}
and when $n\geq2$:
\begin{enumerate}[i)]\squishlist
\setlength{\parindent}{.25in}
\item[iii)] $\Fr{\LL{n}}(X)$ has basis $\{\Q^Ib\,:\,b\in B_{a_{n},\ldots,a_1},\,I\in\admis{\Q},\ i_1\leq|b|,\ i_1,a_{n-1},\ldots,a_2\textup{ not all zero}\}$;
\item[iv)] $\Fr{\nontop{n}}(X)$ has basis $\{\Q^Ix\,:\,x\in X_{a_{n},\ldots,a_1},\,I\in\admis{\Q},\ i_1<|b|,\ i_1,a_{n-1},\ldots,a_2\textup{ not all zero}\}$.
\end{enumerate}
We always allow $I$ to be the empty sequence in the above bases, in which case the condition on $i_{1}$ is to be ignored. If $I$ is nonempty, we mean that $I=(i_\ell,\ldots,i_1)$, so that $i_{1}$ is the rightmost entry of $I$.
Note that the inequalities involved differ according to whether $n=1$ or $n\geq2$.
\end{prop*}
%\begin{prop*}
%If I ever use this, I'll swap the `i_ell's
%For $n\geq2$, for $X\in\GS{n}$ a graded set, $\Fr{\LL{n}}(X)$ has basis the compositions $\Q^{I}b$, where $b$ runs over a homogeneous Hall basis for the free (unrestricted, good) Lie algebra $\Fr{\GoodLie{n}}(X)$, and $I=(i_1,\ldots,i_\ell)$ is a $\Q$-admissible sequence with $i_\ell\leq|b|$ and not all of $i_{\ell},a_{n-1},\ldots,a_2$ are zero if $b$ has grading $(a_n,\ldots,a_1)$. 
%
%$\Fr{\nontop{n}}(X)$ has basis the compositions $\Q^{I}x$ where $x\in X$ and $I=(i_1,\ldots,i_\ell)$ is a $\Q$-admissible sequence with $i_\ell<|x|$ and not all of $i_{\ell},a_{n-1},\ldots,a_2$ are zero if $x$ has grading $(a_n,\ldots,a_1)$.
%%Here, $L^uX$ is an object of $\GoodLie{n}$ if we stipulate that the bracket shifts degrees as appropriate.
%\end{prop*}
\begin{proof}
Part (i) is \cite[Thm F, p.15]{MR1089001}.
In each of the remaining three cases, it is clear that the proposed bases span the free constructions. It remains to show that the elements are linearly independent.
The proposed basis in part (ii) is seen to be linearly independent using the map $\Fr{\nontop{1}}(X)\to \Fr{\LL{1}}(X)$. Part (iv) follows from (iii) similarly.

For part (iii), note that we have calculated the object $\pi_*(\Fr{\PRLie{n-1}}S(X))\in\PiAlg{\PRLie{n-1}}=\LL{n}$, and that the elements of our proposed basis of $\Fr{\LL{n}}(X)$ map to a basis of $\pi_*(\Fr{\PRLie{n-1}}S(X))$ under the natural map. This proves the linear independence statement we require.
%sequences I here are back to front again
%For part (iii), choose a simplicial object of $\GR{n-1}$, say $V\in s(\GR{n-1})$, such that $\pi_{*}V=\Fr{\GR{n}}(X)$. Then \cite[Thm 8.6, Thm 8.8]{CurtisSimplicialHtpy.pdf} shows that $\pi_*(\Fr{\RestLie{n}}(V))$ is an element of $\LL{n}$, and has a basis consisting of tensors $\Q^{I}\otimes b$, for $b\in B$, and $I\in\admis{\Q}$ with $i_\ell\leq|b|$. The `not all zero' condition is dropped here, so that $\pi_*(\Fr{\RestLie{n}}(V))$ is larger than our proposed $\LL{n}(X)$.
%
%Consider then the morphism $\Fr{\LL{n}}(X)\to \pi_*(\Fr{\RestLie{n}}(V))$ induced by the universal property of the free construction. The images of the proposed basis for $\Fr{\LL{n}}(X)$ map to linearly independent elements in $\pi_*(\Fr{\RestLie{n}}(V))$, showing that $\Fr{\LL{n}}(X)$ is as described. The same argument will prove (iv), this time viewing $\pi_*(\Fr{\RestLie{n}}(V))$ as an element of $\nontop{n}$.
\end{proof}
\begin{cor*}
For all $n\geq 1$, the forgetful functor $\forget:\LL{n}\to\nontop{n}$ preserves free objects.
\end{cor*}
\begin{proof}
When $n=1$, $\Fr{\LL{1}}(X)$ is free on $B$ as an object of $\nontop{n}$. For $n\geq2$, $\Fr{\LL{n}}(X)$  is free on $B\cup \{\iteratedrestn{b}{n}\,:\,b\in B_{a_{n},\ldots,a_1},\,n\geq1,\textup{ not all of $a_n,\ldots,a_2$ are zero}\}$ as an object of $\nontop{n}$, since
\[\iteratedrestn{b}{n}=\Q^{(2^{n-1}|b|)}\cdots \Q^{(2|b|)}\Q^{|b|}b.\qedhere\]
\end{proof}
\end{CategoriesOfInterest}





\begin{DiagramOfFunctors}
\section{Grothendieck spectral sequences}
\subsection{A diagram of functors}
Now for each category $\mathsf{C}$ amongst $\LL{n}$, $\PRLie{n}$ and $\nontop{n}$, there is a functor $\Ind{\mathsf{C}}:\mathsf{C\overset{}{\to}\GR{n}}$, which takes the quotient of the underlying vector space by the subspace spanned by the image of all of the structure maps. These provide the solid horizontal arrows in the following diagram, while the vertical maps are the evident forgetful functors: %There are also free/forget adjunctions $\GS{n} \rightleftarrows \mathsf{C}$. %Finally, there's a forgetful functor $\forget:\LL{n}\to\nontop{n}$, which, importantly, preserves  free objects%footnote:
%\footnote{I think that the adjunction does need to come from graded pointed sets, but that that's no problem for Blanc-Stover. The only thing that matters is that $\forget$ preserves frees, but it does --- frees in $\nontop{n}$ are easy to describe.}. These functors give the solid maps in the following diagram:
\[\xymatrix@R=8mm@C=15mm@!C{
\LL{n}
\ar@/^1.5em/[rr]^-{\Ind{\LL{n}}}
\ar[d]_-{\forget}
\ar@{-->}[r]_-{}%\Ind{\nontop{n}}}
&%r1c1
\PRLie{n}
\ar[d]^-{\forget}
\ar[r]_-{\Ind{\PRLie{n}}}
&%r1c2
\GR{n}
\\%r1c3
\nontop{n}
\ar[r]_-{\Ind{\nontop{n}}}
&%r2c1
\GR{n}
&%r2c2
%r2c3
}\]
\begin{prop*}
For all $n\geq1$ there exists a functor $\LL{n}\to\PRLie{n}$ which preserves free objects, and completing the diagram above to a commuting square and triangle.
\end{prop*}
\begin{proof}
For all $n$, the bracket and restrictions that are required to define an object of $\nontop{n}$ are well defined after quotienting out the image of the operations in $\Ind{\nontop{n}}$.

When $n=1$ we are to quotient by the image of all of the $P$ operations. There are no restriction operations to be defined, and the bracket is well defined due to the axiom ``$[x,P^iy]=0$''. Although we started with a bad Lie algebra, the axiom ``$P^{|x|}x=[x,x]$'' shows that the quotient is a good lie algebra, as required.

When $n\geq2$  we are to quotient by the image of all of the non-top $\Q$ operations. Again, the bracket is well defined on the quotient as brackets kill non-top $\Q$ operations. To see that the restriction is well defined on the quotient, %we must check the following fact: for homogeneous $x,y\in L$, and $0\leq i<|y|$ such that $\Q^iy$ is defined and $i+|y|=|x|$, $\restn{(x)}$ and $\restn(x+\Q^iy)$ differ by a sum of images of nontop $\Q$ operations. This follows from the fact that $\Q$-Adem relations do not produce top $\Q$ operations, since \[\restn{(x+\Q^iy)}=\restn{(x)}+\restn{(\Q^iy)}+[x,\Q^iy]=\restn{(x)}+\Q^{|y|+i}\Q^iy+0,\] and $|x|=i+|y|>2i$, so that there's an Adem relation to apply to $\Q^{|x|}\Q^iy$.
suppose that $\Q^iy$ is the image of a well-defined non-top operation, so that $i<|y|$. Then
\[\restn{(x+\Q^iy)}=\restn{(x)}+\restn{(\Q^iy)}+[x,\Q^iy]=\restn{(x)}+\Q^{|y|+i}\Q^iy.\]
The rightmost term admits the application of an Adem relation, as $|y|+i>2i$, but we have seen that Adem relations never return top $\Q$ operations.

These arguments have shown that the quotient supports the structure of partially restricted Lie algebra. The assignment is clearly functorial, and makes the whole diagram commute. 
By examining our results on the bases of the various free constructions, one sees that $\Ind{\nontop{n}}$ sends $\Fr{\LL{n}}(X)$ to $\Fr{\PRLie{n}}(X)$, both when $n=1$ and when $n\geq2$.
\end{proof}
\noindent It should cause no confusion when we denote the functor of this proposition $\Ind{\nontop{n}}:\LL{n}\to \PRLie{n}$ as well, giving a commuting diagram
\[\xymatrix@R=8mm@C=15mm@!C{
\LL{n}
\ar@/^1.5em/[rr]^-{\Ind{\LL{n}}}
\ar[d]_-{\forget}
\ar@{->}[r]_-{\Ind{\nontop{n}}}
&%r1c1
\PRLie{n}
\ar[d]^-{\forget}
\ar[r]_-{\Ind{\PRLie{n}}}
&%r1c2
\GR{n}
\\%r1c3
\nontop{n}
\ar[r]_-{\Ind{\nontop{n}}}
&%r2c1
\GR{n}
&%r2c2
%r2c3
}\]
in which the functors $\forget$ and $\Ind{\nontop{n}}$ with domain $\LL{n}$ both preserve free objects.
%The point is that there is a functor $\LL{n}\to\PRLie{n}$ which completes this diagram to a commuting square and triangle, which we also name $\Ind{\nontop{n}}$. It will be important to note that $\Ind{\nontop{n}}$ sends free objects in $\LL{n}$ to free objects in $\PRLie{n}$.
%
%When $n=1$, $\Ind{\nontop{n}}$ quotients the underlying vector space by the images of all of the $P$ operations. The result inherits the structure of an element of $\PRLie{n}$ from the bracket in $\LL{1}$. This is well defined due to the axioms ``$[x,P^iy]=0$'' and ``$P^{|x|}x=[x,x]$''. 
%
%When $n\geq2$, $\Ind{\nontop{n}}$ quotients the underlying vector space by the images of all of the non-top $\Q$ operations. That is, by the span of the $\Q^ix$ for $i<|x|$. The result inherits the structure of an element of $\PRLie{n}$ as follows. The Lie bracket is induced by that in $\LL{n}$; well defined by the axiom ``$[\Q^i{x},z]=0$ for $i<|x|$''. The restriction is induced by the top $\Q$ operations, being defined by the formula $\restn{x}:=\Q^{|x|}x$. It is a little less obvious that this is in fact well defined. One must check, for homogeneous $x,y\in L$, and $0\leq i<|y|$ such that $\Q^iy$ is defined and $i+|y|=|x|$, that $\Q^{|x|}x$ and $\Q^{|x|}(x+\Q^iy)$ differ by a sum of images of non-top $\Q$ operations. But\[\Q^{|x|}(x+\Q^iy)-\Q^{|x|}x=\Q^{|x|}(\Q^iy)+[x,\Q^iy]=\Q^{|x|}\Q^iy,\]and the above calcuation regarding indices in the $\Q$-Adem relations shows that $\Q^{|x|}\Q^iy$ can be rewritten without the appearance of a top $\Q$.
\end{DiagramOfFunctors}
\begin{GrothendieckSpectralSequences}
\subsection{Grothendieck spectral sequences}
We are interested in the derived functors $(\derived\Ind{\LL{n}})X\in\GR{n+1}$ for $X\in\LL{n}$. It is convenient for us to omit the customary asterisk in $\derived_*$, and to think of the derived functors as a single object with a one more grading than $X$ has. We always write the new grading to the left of the previous gradings, as is customary. To study $(\derived\Ind{\LL{n}})(X)$ we use the Grothendieck spectral sequence derived in \cite{Blanc_Stover-Groth_SS.pdf}. The spectral sequence takes the form:
\[E^2_{st}=(\derived_s(\overline{\Ind{\PRLie{n}}}_t))(\derived\Ind{\nontop{n}})X\implies(\derived_{s+t}\Ind{\LL{n}})X\textup{ with }d^{r}:E_{st}\to E_{s-r,t+r-1}.\]
To explain this, note that $\derived\Ind{\nontop{n}}\in\PiAlg{\PRLie{n}}$, and $\overline{\Ind{\PRLie{n}}}$ is the functor $\PiAlg{\PRLie{n}}\to\GR{n+1}$ induced by $\Ind{\PRLie{n}}$. The $t$ subscript picks out the part with (new) grading $t$. It might be simpler to write:
\[E^2=(\derived\overline{\Ind{\PRLie{n}}})(\derived\Ind{\nontop{n}})X\implies(\derived\Ind{\LL{n}})X\textup{ with }\deg(d^r)=(-r,r-1,0,\ldots,0).\]
Here, each $E^r$ is $(n+2)$-graded, while $(\derived\Ind{\LL{n}})X$ is $(n+1)$-graded. Each page $E^r$ has two homological degrees in addition to the $n$ internal degrees of $X$, and these extra two degrees on $E^\infty$ sum to the single homological degree in $(\derived\Ind{\LL{n}})X$.
\begin{prop*}
There's a commuting diagram of functors:
\[\xymatrix@R=0mm@C=14mm{
\PiAlg{\PRLie{n}}
\ar[dr]^-{\overline{\Ind{\PRLie{n}}}}
\ar[dd]_-{\cong}
&%r1c1
\\%r1c2
&%r2c1
\GR{n+1}\\%r2c2
\LL{n+1}
\ar[ur]_-{\Ind{\LL{n+1}}}
&%r3c1
%r3c2
}\]
%Under the isomorphism $\PiAlg{\PRLie{n}}\cong\LL{n+1}$, the functor $\overline{\Ind{\PRLie{n}}}:\PiAlg{\PRLie{n}}\to \GR{n+1}$ is identified with 
\end{prop*}
\begin{proof}
As $\overline{\Ind{\PRLie{n}}}$ is a $0^\textup{th}$ left derived functor, and $\Ind{\LL{n+1}}$ preserves colimits, it is enough to check that this diagram commutes on free objects in $\PiAlg{\PRLie{n}}$. All such objects are of the form $\pi_*(\Fr{\PRLie{n}}SX)$ for some $X\in\GS{n+1}$, and
\[\overline{\Ind{\PRLie{n}}}(\pi_*(\Fr{\PRLie{n}}SX)):=\pi_*(\Ind{\PRLie{n}}(\Fr{\PRLie{n}}SX))=\pi_*(SX)=X=\Ind{\LL{n+1}}(\pi_*(\Fr{\PRLie{n}}SX)).\qedhere\]
%The functor $\overline{\Ind{\PRLie{n}}}$ is the $0^\textup{th}$ left derived functor of the functor $\textup{free}(\PiAlg{\PRLie{n}})\to\GR{n+1}$ defined by $\pi_*(\Fr{\PRLie{n}}SX)\mapsto \pi_*(\Ind{\PRLie{n}}(\Fr{\PRLie{n}}SX))$ for $X\in\GS{n+1}$. One calculates $\pi_*(\Ind{\PRLie{n}}(\Fr{\PRLie{n}}SX))=\pi_*(SX)=X$
\end{proof}
%Finally, we should recall that $\PiAlg{\PRLie{n}}\simeq\LL{n+1}$, and under this isomorphism, $\overline{\Ind{\PRLie{n}}}$ is identified with $\Ind{\LL{n+1}}$. 
\begin{cor*}
For $n\geq1$, and any $X\in\LL{n}$, there is a spectral sequence:
\[E^2=(\derived{\Ind{\LL{n+1}}})(\derived\Ind{\nontop{n}})X\implies(\derived\Ind{\LL{n}})X.\]
\begin{itemise}
\setlength{\parindent}{.25in}
\item For $r\leq\infty$, the $E^r$ page is an object of $\GR{n+2}$.
\item For $r<\infty$, $d^r:E^r\to E^r$ has degree $(-r,r-1,0,\ldots,0)$.
\item The target, $(\derived\Ind{\LL{n}})X$, is an object of $\GR{n+1}$.
\item There is an increasing filtration $F$ on $(\derived\Ind{\LL{n}})X$ such that
\[E^\infty_{a_{n+2},\ldots,a_1}=(F_{a_{n+2}}/F_{a_{n+2}-1})((\derived\Ind{\LL{n}})X)_{a_{n+2}+a_{n+1},a_n,\ldots,a_1}.\]
\end{itemise}
%The pages $E^r$ ($r\leq\infty$) are $(n+2)$-graded, with $\deg(d^r)=(-r,r-1,0,\ldots,0)$. The term $E^\infty_{a_{n+2},\ldots,a_1}$ is a subquotient of the degree $(a_{n+2}+a_{n+1},a_n,\ldots,a_1)$ part of the target.
\end{cor*}
%Thus, we may write the spectral sequence as:
%\[\fbox{$E^2=(\derived{\Ind{\LL{n+1}}})(\derived\Ind{\nontop{n}})X\implies(\derived\Ind{\LL{n}})X.$}\]
The following observation explains why there's little danger in having two functors named $\Ind{\nontop{n}}$.
\begin{lem*}
For $n\geq1$, and any $X\in\LL{n}$,
\[(\derived{\Ind{\nontop{n}}})X\cong(\derived{\Ind{\nontop{n}}})(\forget(X))\textup{ as a vector space.}\]
That is, as vector spaces, both $(\Ind{\nontop{n}})X$ and $(\derived{\Ind{\nontop{n}}})X$ are unambiguous.
\end{lem*}
\begin{proof}
As $\forget:\LL{n}\to\nontop{n}$ preserves free objects and weak equivalences, there's a one-line Grothendieck spectral sequence proving the result.
\end{proof}
\end{GrothendieckSpectralSequences}


\begin{KoszulComplexes2plus}
\section{Unstable Koszul complexes and Priddy's algorithm}
\subsection{Koszul complexes for $\derived\Ind{\nontop{n}}$ for $n\geq2$}
We will define Koszul complexes calculating $\derived\Ind{\nontop{n}}$ for all $n$. The techniques we will use originate in \cite{PriddyKoszul.pdf}. It will be useful to record here a customized version of Priddy's arguments, for three reasons: his results do not apply to our situation without minor modification to accomodate unstable conditions; we will require a technique from his proof of \cite[Thm 5.3]{PriddyKoszul.pdf} in what follows; we prefer to work with the Koszul complex, not the coKoszul complex.
%We Due to the fact that we are working with categories of unstable modules, there are straightforward changes to be made to Priddy's exposition. We make these changes here. The other benefit to covering this material is that we will
%Moreover, for certain calculations to follow we will need to refer to certain techniques found in the proof of \cite[Thm 5.3]{PriddyKoszul.pdf}. In order that our reader may understand these techniques, we 

%\textbf{remove }For $I=(i_\ell,\ldots,i_1)$ a sequence of nonnegative integers, define
%\[\minDim(I):=\begin{cases}
%-\infty,&\textup{if }I=\emptyset;\\
%\max\{(i_1),\,(i_2-i_1),\,\ldots,\,(i_{\ell}-i_{\ell-1}-\cdots i_1)\},&\textup{otherwise}.
%%\\,&\textup{if }
%\end{cases}
%\]
%Thus, for $n\geq2$:
%\begin{itemise}
%\setlength{\parindent}{.25in}
%\item  for $I\neq\emptyset$, $\minDim(I)$ is the lowest dimension $d\geq0$ such that $\Q^I$ may act on a $d$-dimensional element of an object in $\LL{n}$, if we ignore the extra restrictions on $\Q^0$; and
%\item $\minDim(I)$ is the unique element of $\{-\infty,0,1,2,\ldots\}$ such that for any $X\in\nontop{n}$ and any $x\in X$, $\Q^Ix$ is defined if and only if $\minDim(I)<|x|$, if we ignore the extra restrictions on $\Q^0$.
%\end{itemise}

%\subsubsection{The unstable bar construction and Koszul complex for $\nontop{n}$, $n\geq2$}
For any object $X\in\nontop{n}$, the bar construction ${B}_\bullet(\Fr{\nontop{n}},\Fr{\nontop{n}},X)=(\Fr{\nontop{n}})^{\bullet+1}X$ is a free simplicial resolution of $X$. Now there is an isomorphism of graded vector spaces $\Ind{\nontop{n}}(\Fr{\nontop{n}}(X))\cong\forget(X)$. Thus $(\derived\Ind{\nontop{n}})X$ may be calculated as the homotopy of the bar construction $B_\bullet(\forget,\Fr{\nontop{n}},X)=(\Fr{\nontop{n}})^\bullet X$. Normalizing this complex by taking the quotient by degenerate simplices, one obtains the reduced Bar construction $\overline{B}_*(\forget,\Fr{\nontop{n}},X)$, a chain complex calculating $(\derived\Ind{\nontop{n}})X$, which we now describe.
\begin{prop*}
As a vector space each $\overline{B}_r$ is graded, with $\overline{B}_{r,\ell}$
%For $r\geq0$, $\overline{B}_r(\forget,\Fr{\nontop{n}},X)$ is 
spanned by monomials
\[\BarMonomial{I}{\textbf{k}}{x}:=\left[\Q^{i_{k_r+\cdots +k_1}}\cdots \Q^{i_{k_{r-1}+\cdots +k_1+1}}
\middle|\cdots 
\middle|\Q^{i_{k_2+k_1}}\cdots \Q^{i_{k_1+1}}
\middle|\Q^{i_{k_1}}\cdots \Q^{i_1}\right]
x\]
where:
\begin{itemize}
\setlength{\parindent}{.25in}
\item $x\in X_{a_n,\ldots,a_1}$ is a homogeneous element of $X$;
\item $\textbf{k}=(k_r,k_{r-1},\ldots,k_{1})$ is a sequence of positive integers with sum $\ell$;
%\item $0<k_1<k_2<\cdots <k_{r-1}<\ell$ is increasing, so that each bar is nonempty;
\item $I=(i_\ell,\ldots,i_1)$ is a sequence of nonnegative integers with $\minDim(I)<a_n$; and
\item if all of $a_{n-1},\ldots,a_2$ equal zero, $0$ does not appear in $I$.
\end{itemize}
As Adem relations may be applied between each pair of bars, a basis of $\overline{B}_r(\forget,\Fr{\nontop{n}},X)$ consists of all the above monomials in which each of the $r$ subsequences of $I$ is $\Q$-admissible, as $x$ runs through a homogeneous basis for $X$. The formula for the differential is the standard one, making %under the $\ell$ grading on each $\overline{B}_r$ makes
 $\overline{B}_*$ a chain complex with increasing filtration, $F_p\overline{B}_r=\bigoplus_{\ell\geq p}\overline{B}_{r,\ell}$. 
\end{prop*}
\noindent We'll say that a monomial $\BarMonomial{I}{\textbf{k}}{x}\in\overline{B}_r$ is bar-admissible if each of the evident $r$ subsequences of $I$ are $\Q$-admissible. Thus a basis for $\overline{B}_{r,\ell}$ consists of the bar-admissible monomials on a basis of $X$. Note that for any fixed element $x$ and any fixed internal grading $\ell$, the condition on $\minDim(I)$ implies that there are only finitely many monomials of the form $\BarMonomial{I}{\textbf{k}}{x}$ in $\overline{B}_{r,\ell}$.
\begin{prop*}
Any cycle in $\overline{B}_{r,\ell}$, for $\ell>r$ is homologous to a cycle in $\overline{B}_{r,r}$.
\end{prop*}
\begin{proof}
First, a construction. For any bar-admissible monomial $\BarMonomial{I}{\textbf{k}}{x}\in\overline{B}_{r,\ell}$ with $\ell>r$, choose the unique integer $e$ such that $k_r=k_{r-1}=\cdots =k_{r-e+1}=1<k_{r-e}$, so that
\[\BarMonomial{I}{\textbf{k}}{x}=\left[\Q^{i_\ell}\middle|\Q^{i_{\ell-1}} \middle|\Q^{i_{\ell-2}}\middle|\cdots \middle|\Q^{i_{\ell-e+1}}\middle|\Q^{i_{\ell-e}}\Q^{i_{\ell-e-1}}\cdots \middle|\cdots \right]x.\]
That is, we've identified the longest initial segment of length one bars. Then we define
\[\Psi(\BarMonomial{I}{\textbf{k}}{x})=\begin{cases}
0,\textup{ if }(i_{\ell},\ldots,i_{\ell-e})\textup{ is not $\SqShift$-admissible};\\
\left[\Q^{i_\ell}\middle|\Q^{i_{\ell-1}} \middle|\Q^{i_{\ell-2}}\middle|\cdots \middle|\Q^{i_{\ell-e+1}}\middle|\Q^{i_{\ell-e}}\middle|\Q^{i_{\ell-e-1}}\cdots \middle|\cdots \right]x,\textup{ otherwise}.
%\\,&\textup{if }
\end{cases}
\]
Thus $\Psi$ either returns zero or adds a single bar to produce a bar-admissible monomial in $\overline{B}_{r+1,\ell}$. One may extend $\Psi$ by linearity to produce linear operators $\Psi:\overline{B}_{r,\ell}\to\overline{B}_{r+1,\ell}$, for $\ell>r$, and extend $\Psi$ by zero on $\overline{B}_{r,r}$, to produce $\Psi:\overline{B}_{r}\to\overline{B}_{r+1}$.


Put a total preorder on the bar-admissible monomials by saying that
\[\BarMonomial{I}{\textbf{k}}{x}\gtrsim \BarMonomial{I'}{\textbf{k}'}{x'}\textup{ if either } \ell>\ell'\textup{ or }\ell=\ell'\textup{ and }I\geq I'\textup{ lexicographically}.\]
Then, as remarked in \cite[proof of thm 5.3]{PriddyKoszul.pdf}, a straightforward verification shows that for any bar-admissible monomial $m\in\overline{B}_{r,\ell}$ with $r>\ell$, $\partial\Psi(m)+\Psi\partial(m)\equiv m$ modulo strictly lesser monomials.
%Now any element of $\overline{B}_r$ can be written in the form $z=\sum_{j=1}^N \BarMonomial{I^{(j)}}{\textbf{k}^{(j)}}{x^{(j)}}$ for some nonincreasing sequence $\left\{\BarMonomial{I^{(j)}}{\textbf{k}^{(j)}}{x^{(j)}}\right\}$ of monomials. Choose the largest $M$ such that $$

Now suppose that $z\in Z\overline{B}_r$ is a cycle which is not already in $\overline{B}_{r,r}$. Then write 
$z=\sum_i a_i + \sum_j b_j$ where the $a_i$ and $b_j$ are bar-admissible monomials, all of the $a_i$ are equivalent under $\gtrsim$, but $a_i\gnsim b_j$ for all $i$ and $j$. Then $z+\sum_i\partial\Psi(a_i)$ is a homologous cycle in which every monomial appearing (when written as a sum of bar-admissibles) is strictly less than the $a_i$. We may iterate this process, reducing the largest equivalence class of monomials appearing in $z$ at each step. As there are only finitely many possible sequences $I$ appearing, eventually a cycle in $\overline{B}_{r,r}$ is reached.
\end{proof}

\begin{prop*}
Suppose that $X\in\nontop{n}$ is trivial, so that all of the $\Q$ operations act as zero. Then %the differential $\partial $ of $\overline{B}_r$ respects the $\ell$ grading, so that 
$(\derived_r\Ind{\nontop{n}})X$ has a basis $\{\Q(I)x\}$, where
\begin{itemize}
\setlength{\parindent}{.25in}
\item $x\in X_{a_n,\ldots,a_1}$ is an element of a chosen homogeneous basis of $X$;
\item $I=(i_r,\ldots,i_1)$ a $\SqShift$-admissible sequence with $\minDim(I)<a_n$; and
\item if all of $a_{n-1},\ldots,a_2$ equal zero, $0$ does not appear in $I$.
\end{itemize}
These classes are defined by the following sum taken over certain non-negative integer sequences $J=(j_{r},\ldots,j_1)$:
\[\Q(I)x:=\left[\Q^{i_r}\middle|\cdots\middle|\Q^{i_1} \right]x+\sum_{\produces{J}{I}{\SqShift}}\left[\Q^{j_r} \middle|\cdots\middle|\Q^{j_1} \right]x.\]%\textup{ (sum over nonnegative sequences $J=(j_{r},\ldots,j_1)$)}\]
\end{prop*}
\noindent Note that if $\produces{J}{I}{\SqShift}$, then $\minDim(J)<\minDim(I)$, and also that if $0$ appears in $I$, it must also appear in $J$ [these facts are Koszul dual to those described in the lemma above for $\Q$]. Thus, all the terms in the definition of $\Q(I)x$ are indeed valid in the bar construction.
\begin{proof}
As $X$ is trivial, the differential of $\overline{B}_r$ respects the $\ell$ grading, so that even the derived functors $(\derived_r\Ind{\nontop{n}})X$ have the extra $\ell$ grading. By the previous proposition, they are concentrated along the diagonal $r=\ell$.

It appears to be easier at this stage to work with the duals $\overline{C}^{r,\ell}$ of the $\overline{B}_{r,\ell}$, and determine the cokernel of $\delta:\overline{C}^{r-1,r}\to\overline{C}^{r,r}$. There is no danger in dualizing twice: $X$, being trivial, is the union of its finite dimensional subobjects, so that we may assume that $X$ is finite dimensional. Fix a basis $\{x\}$ of $X$. %, inducing a dual basis $\{x^*\}$ of $X^*$. 
Then, we have a basis $\{\BarMonomial{I}{\textbf{k}}{x}^*\}$ for $\overline{C}^{*,*}$ dual to the above basis for $\overline{B}_{*,*}$.

Now $\delta:\overline{C}^{r-1,r}\to\overline{C}^{r,r}$ may be described as follows. A basis for $\overline{C}^{r-1,r}$ is given by elements of the form
\[\left(\left[\Q^{i_r}\middle|\cdots\middle|\Q^{i_{m+2}}  \middle|\Q^{i_{m+1}}\Q^{i_m}\middle|\Q^{i_{m-1}}\middle|\cdots \middle|\Q^{i_1} \right]x\right)^*\]
which are duals of bar-admissible monomials. The image under $\delta$ of this class is
\[
\left(\left[\Q^{i_r}\middle|\cdots \middle|\Q^{i_1} \right]x\right)^*
+
\sum_{\produces{(\alpha,\beta)}{(i_{m+1},i_m)}{\Q}}\left(\left[\Q^{i_r}\middle|\cdots\middle|\Q^{i_{m+2}}  \middle|\Q^{\alpha}\middle|\Q^{\beta}\middle|\Q^{i_{m-1}}\middle|\cdots \middle|\Q^{i_1} \right]x\right)^*\]
We are working in the cobar construction, which is naturally viewed as the \emph{quotient} of the standard cobar construction $C(\F_2,\DyerLashov,X)$ by those cobars not satisfying our unstable condition. Thus, in the right hand side of the above expression, any summands with $\minDim(I)\geq|x|$ or which fail the condition on $\Q^0$ are taken to be zero.
%
%Where, if $x\in X_{a_n,\ldots,a_1}$:
%\begin{itemise}
%\setlength{\parindent}{.25in}
%\item we omit those summands in which $\minDim(i_{r},\ldots,i_1)\geq a_n$; and
%\item if all of $a_{n-1},\ldots,a_2$ are zero, we omit any summands with $i_m=0$ or $i_{m+1}=0$.
%\end{itemise}
The term with $(i_{m+1},i_m)=(\alpha,\beta)$ is not omitted. Moreover, for length 2 sequences $K$ and $L$, as $\Q$ and $\SqShift$ are Koszul dual, $\produces{K}{L}{\Q}$ if and only if $\produces{L}{K}{\SqShift}$. Thus, we obtain a relation:
\[\left(\left[\Q^{i_r}\middle|\cdots \middle|\Q^{i_1} \right]x\right)^*
=
\sum_{\produces{(i_{m+1},i_m)}{(\alpha,\beta)}{\SqShift}}\left(\left[\Q^{i_r}\middle|\cdots\middle|\Q^{i_{m+2}}  \middle|\Q^{\alpha}\middle|\Q^{\beta}\middle|\Q^{i_{m-1}}\middle|\cdots \middle|\Q^{i_1} \right]x\right)^*\]
where we make the same omissions on the right hand side. Thus, the relation here is simply the $\SqShift$-Adem relation.
Using this relation, we see that a basis of $H^{r,r}\overline{C}$ is represented by the cocycles $\BarMonomial{I}{\textbf{k}}{x}^*$, where $I$, and $x$ are as in the statement of of this proposition (and $\textbf{k}=(1,\ldots,1)$).

As long as each $\Q(I)x$ is a cycle, they will form the dual basis for $H_{rr}\overline{B}$, as in the sum defining $\Q(I)x$, the only $\SqShift$-admissible $J$ appearing is $J=I$. In order to check that each $\Q(I)x$ is a cycle, it only remains to check that they pair to zero with each relation in the dual. But that is not hard to see. On $\overline{C}^{r,r}$, by definition of $\Q(I)x$, the operation `pair with $\Q(I)x$' \emph{equals} the operation `reduce to $\SqShift$-admissible form using $\SqShift$-Adem relations, and take the coefficient of $(\left[\Q^{i_\ell}\middle|\cdots\middle|\Q^{i_1} \right]x)^*$'. This second operation obviously returns zero on Adem relations, and these are precisely the image of $\delta$.
\end{proof}
\begin{prop*}
Suppose instead that $X$ is any object in $\nontop{n}$, the Koszul complex $\overline{K}_*X$ is the subcomplex of $\overline{B}_*(\forget,\Fr{\nontop{n}},X)$ with basis the classes $\Q(I)x$ (for the same $I$ and $x$ as above). The inclusion of $\overline{K}_*X$ into $\overline{B}_*(\forget,\Fr{\nontop{n}},X)$ is a homology equivalence, the restriction of the differential to $\overline{K}_*X$ is given by the following formula:
\[\partial(\Q(I)x)=\Q((i_{\ell},\ldots,i_2))(\Q^{i_1}x)+\sum_{\produces{J}{I}{\SqShift}} \Q((j_{\ell},\ldots,j_2))(\Q^{j_1}x),\]
where the sum is taken over those $J=(j_{\ell},\ldots,j_1)$ such that $(j_{\ell},\ldots,j_2)$ is $\SqShift$-admissible.
\end{prop*}
\begin{proof}
The spectral sequence associated to the filtered complex has $E^0_{r,\ell}=\overline{B}_{r,\ell}(\forget,\Fr{\nontop{n}},X^\textup{Triv})$, where $X^\textup{Triv}$ is a trivial object of $\nontop{n}$ on the same underlying vector space as $X$. The differential $d^0$ implements the differential on this reduced bar construction, so that $E^1_{r,\ell}=(\derived_{r,\ell}\Ind{\nontop{n}})X$. This is concentrated on $r=\ell$, so that $d^1$ is the last nonzero differential, after which the spectral sequence collapses.

The formula for $d^1:E^1_{r,r}\to E^1_{r-1,r-1}$ is simply:
\[\sum_j \left[\Q^{i_r^{(j)}}\middle|\cdots \middle|\Q^{i_1^{(j)}}\right]x^{(j)}
\mapsto
\sum_j \left[\Q^{i_r^{(j)}}\middle|\cdots \middle|\Q^{i_2^{(j)}}\right]\left(\Q^{i_1^{(j)}}x^{(j)}\right)\]
Now as $E^1_{r,r}\subseteq E^0_{r,r}$, in order to write this in terms of the $\Q(I)x$ we need only focus on the terms involving $\SqShift$-admissible sequences, and the formula for $\partial$ follows.
\end{proof}


\end{KoszulComplexes2plus}


\begin{KoszulComplexes1}
\subsection{Koszul complexes for $\derived\Ind{\nontop{1}}$}
We'll record here the necessary changes to the previous theory required when $n=1$.

%\textbf{remove} For $I=(i_\ell,\ldots,i_1)$ a sequence of integers with each $i_j\geq2$, define
%\[\minDimP(I):=\begin{cases}
%-\infty,&\textup{if }I=\emptyset;\\
%\max\{(i_1),\,(i_2-i_1-1)\,(i_3-i_2-i_1-2),\,\ldots,\,(i_{\ell}-i_{\ell-1}-\cdots i_1-\ell+1)\},&\textup{otherwise}.
%%\\,&\textup{if }
%\end{cases}
%\]
%Thus, for $I\neq\emptyset$, $\minDimP(I)$ is the lowest dimension $d\geq0$ such that $P^Ix$ may be nonzero, for $x$ a $d$-dimensional element of an object in either $\LL{1}$ or $\nontop{1}$.

%\subsubsection{The unstable bar construction and Koszul complex for $\nontop{1}$}
We are again interested in the reduced Bar construction $\overline{B}_*(\forget,\Fr{\nontop{1}},X)$, which computes $(\derived\Ind{\nontop{1}})X$.

%\begin{lem*}
%Suppose that $i\geq2j$, and $\produces{(i,j)}{(i',j')}{P}$. Then $\minDimP(i',j') > \minDimP(i,j)$.
%\end{lem*}
%\begin{proof}
%$j'\geq i-j+1=
%j+(i-2j+1)=
%i-j-1+(2)
%$, so that $j'>\max\{j,i-j-1\}$.
%\end{proof}

\begin{prop*}
As a vector space each $\overline{B}_r$ is graded, with $\overline{B}_{r,\ell}$
%For $r\geq0$, $\overline{B}_r(\forget,\Fr{\nontop{n}},X)$ is 
spanned by monomials
\[\BarMonomial{I}{\textbf{k}}{x}:=\left[P^{i_{k_r+\cdots +k_1}}\cdots P^{i_{k_{r-1}+\cdots +k_1+1}}
\middle|\cdots 
\middle|P^{i_{k_2+k_1}}\cdots P^{i_{k_1+1}}
\middle|P^{i_{k_1}}\cdots P^{i_1}\right]
x\]
where:
\begin{itemize}
\setlength{\parindent}{.25in}
\item $x\in X_{a_1}$ is a homogeneous element of $X$;
\item $\textbf{k}=(k_r,k_{r-1},\ldots,k_{1})$ is a sequence of positive integers with sum $\ell$; and
\item $I=(i_\ell,\ldots,i_1)$ is a sequence of integers with $i_j\geq2$ and $\minDimP(I)<a_1$;
\end{itemize}
As Adem relations may be applied between each pair of bars, a basis of $\overline{B}_r(\forget,\Fr{\nontop{1}},X)$ consists of all the above monomials in which each of the $r$ subsequences of $I$ is $P$-admissible, as $x$ runs through a homogeneous basis for $X$. The formula for the differential is the standard one, making %under the $\ell$ grading on each $\overline{B}_r$ makes
 $\overline{B}_*$ a chain complex with increasing filtration, $F_p\overline{B}_r=\bigoplus_{\ell\geq p}\overline{B}_{r,\ell}$. 
\end{prop*}


\begin{prop*}
Suppose that $X$ is any object in $\nontop{1}$, the Koszul complex $\overline{K}_*X$ is the subcomplex of $\overline{B}_*(\forget,\Fr{\nontop{1}},X)$ with basis the classes $P(I)x$, where 
\begin{itemize}
\setlength{\parindent}{.25in}
\item $x\in X_{a_1}$ is an element of a chosen homogeneous basis of $X$;  and
\item $I=(i_r,\ldots,i_1)$ a $\delta$-admissible sequence with $i_j\geq2$ and $\minDimP(I)\leq a_1$.
\end{itemize}
and where 
\[P(I)x:=\left[P^{i_r} \middle|\cdots\middle|P^{i_1} \right]x+\sum_{\substack{\produces{J}{I}{\delta} \\ \minDimP(J)\leq|x|}}\left[P^{j_r} \middle|\cdots\middle|P^{j_1} \right]x.\]
The inclusion of $\overline{K}_*X$ into $\overline{B}_*(\forget,\Fr{\nontop{1}},X)$ is a homology equivalence, the restriction of the differential to $\overline{K}_*X$ is given by the following formula:
\[\partial(P(I)x)=P((i_{\ell},\ldots,i_2))(P^{i_1}x)+\sum_{\substack{\produces{J}{I}{\delta} \\ \minDimP(J)\leq|x|}} P((j_{\ell},\ldots,j_2))(P^{j_1}x),\]
where the sum is taken over those $J=(j_{\ell},\ldots,j_1)$ such that $(j_{\ell},\ldots,j_2)$ is $\delta$-admissible.
\end{prop*}
\begin{proof}
The proof of this proposition is the same as the proof of the corresponding propositions in the $n\geq2$ case. We use the same algorithm to compress to the diagonal, and the same spectral sequence. There are two main differences between the two cases:

In the $n\geq2$ case, the structure of an element of $\nontop{n}$ only has non-top $\Q$ operations, so that the various constraints on $\minDim(I)$ were strict. On the other hand, elements of $\nontop{1}$ support all the $P$ operations, so that the inequalities are not strict.

In the $n\geq2$ case, the bar construction was the subobject of a stable bar construction, and the cobar was a quotient. Thus, our final formula didn't contain any extraneous terms, but our interim calculation with the cobar construction did. When $n=1$, this situation is reversed, so that our results land in a quotient, and certain summands therein are zero.
%
%Note that in the $n=1$ case, there are no summands to omit in the formula for $\delta:\overline{C}^{r-1,r}\to\overline{C}^{r,r}$. [Really, in the $n\geq2$ case, the formula for $\delta$ lands in the quotient by inadmissible terms, while in the $n=1$ case, $\delta$ lands in the subcomplex of admissible terms. This is, of course, dual to the situation in the Bar construction or in the free $\nontop{n}$ construction.] For example, in the $n=1$ case, if $\produces{J}{I}{\delta}$, then $\minDimP(J)\geq\minDimP(I)$, which is the opposite to the inequality in the $n\geq2$ case. However, all of the above summands are defined, even those with $\minDimP(J)>|x|$ --- it's just that those summands are zero, due to the unstable condition for $P$. Thus, when in the above, we include a second subscript on the sums, it's just that the sum would contain extra terms, all automatically zero, if we didn't.
\end{proof}
\end{KoszulComplexes1}


\begin{DerivedFunctorsLowDimension}
\section{Lie algebra homology and low dimensional calculations}
\subsection{Calculating $\derived\Ind{\LL{n}}$ in low internal dimension}
%\subsubsection{$\derived\Ind{\LL{1}}$}
Choose $n\geq 1$, and define 
\[d=\begin{cases}
4,&\textup{if }n=1;\\
1,&\textup{if }n=2;
\\0,&\textup{if }n\geq3.
\end{cases}
\]
We will here present a method to calculate the object $(\derived\Ind{\LL{n}})X=((\derived\Ind{\LL{n}})X)_{a_{n+1},\ldots,a_1}$ in dimensions with $a_n\leq d$. Consider the diagram:
\[\xymatrix@R=2mm@C=13mm{
&%r1c1
\ar[dd]^-{\Ind{\nontop{n}}}
\ar@<.5ex>[dl]%^-{\forget}
\LL{n}\\%r1c2
\ar@<.5ex>[ur]^-{\LL{n}}
\ar@<.5ex>[dr]^-{\PRLie{n}}
\GR{n}&%r2c1
\\%r2c2
&%r3c1
\PRLie{n}
\ar@<.5ex>[ul]%^-{\forget}
%r3c2
}\]
In this diagram, $\Ind{\nontop{n}}$ does not change the underlying vector space in dimensions $(a_n,\ldots,a_1)$ with $a_n\leq d$. Moreover, the free constructions $\Fr{\LL{n}}$ and $\Fr{\PRLie{n}}$ agree in these dimensions, as do the structure maps of the free/forget adjunctions. Thus, there is an isomorphism between $(\derived\Ind{\LL{n}})(X)$ and $(\derived\Ind{\PRLie{n}})(\Ind{\nontop{n}}(X))$ in dimensions with $a_n\leq d$. [Note that these are objects of $\GR{n+1}$, and so have gradings $a_{n+1},\ldots,a_1$ --- our inequality involves $a_n$, the outermost grading of an object of $\LL{n}$.]

There is a complex that computes $\derived\Ind{\PRLie{n}}$ --- it interpolates between the Cartan-Eilenberg resolution for the homology of good Lie algebras, and May's $\overline{X}$ complex for  restricted Lie algebra homology.
\end{DerivedFunctorsLowDimension}

\begin{PRlieKoszulComplexCalculation}

\subsection{The Cartan-Eilenberg-May complex for partially restricted Lie algebra homology}
The PBW theorem states that $E^0VL=A(L^u)\otimes E(L^r)$, and the homology of this algebra (see \cite[\S7]{PriddyKoszul.pdf}) is thus $E(L^u)\otimes \Gamma(L^r)$. We'll think of this as the sub-coalgebra of $\Gamma(L)$ in which there are no $\gamma_i$ for $i>1$ applied to unrestrictable elements. It's the shuffle product that produces representatives in $\overline{B}(E^0V(L))$ for classes in $E(L^u)\otimes \Gamma(L^r)$, and in order to calculate with these representatives, we first introduce some notation.
\begin{itemize}
\setlength{\parindent}{.25in}
\item $\Z^n:=\{\textbf{z}=(z_1,\ldots,z_n);\ z_i\in\Z\}$
\item $r_i:\Z^n\to\Z^n$ given by $(z_1,\ldots,z_n)\mapsto(z_1,\ldots,z_i-1,\ldots,z_n)$
\item $r_{ij}:=r_i\circ r_j:\Z^n\to\Z^n$
\item $\sigma:\Z^n\to\Z$ given by summing entries
\item for $\textbf{z}\in\Z^n$ write $\calP(\textbf{z})$ for the set of permutations of the multiset $\{i^{z_i};i=1,\ldots,n\}$, written as sequences $\textbf{p}=(p_1,\ldots,p_{\sigma \textbf{z}})$. This set is empty if $\textbf{z}$ has a negative entry. For such a $\textbf{p}$, write $p\sigma_i$ for $(p_1,\ldots,p_{i-1},p_{i+1},p_{i},p_{i+2},\ldots,p_{\sigma \textbf{z}})$.
\item $d_k:\calP(\textbf{z})\to\calP((r_{p_kp_{k+1}}\textbf{z})*(1))$ (the star denotes concatenation), in which $(p_1,\ldots,p_{\sigma \textbf{z}})$ is sent to $(p_1,\ldots,p_{k-1},\ell(\textbf{z})+1,p_{k+2},\ldots,p_{\sigma \textbf{z}})$. That is, two entries are removed, and in their position we place a single entry, $\ell(\textbf{z})+1$.
\end{itemize}
\begin{prop*}
Choose $\textbf{z}\in\Z^n$, and $i$ satisfying $1\leq i\leq n$. Then as $k$ ranges from $1$ to $\sigma \textbf{z}-1$, and $\textbf{p}$ ranges through those elements of $\calP(\textbf{z})$ satisfying $(p_k,p_{k+1})=(i,i)$, then $d_kp$ assumes each value in $\calP((r_{ii}\textbf{z})*(1))$ exactly once.
\end{prop*}
\begin{prop*}
Choose $\textbf{z}\in\Z^n$, and $i\neq j$ satisfying $1\leq i,j\leq n$. Then as $k$ ranges from $1$ to $\sigma \textbf{z}-1$, and $\textbf{p}$ ranges through those elements of $\calP(\textbf{z})$ satisfying $(p_k,p_{k+1})=(i,j)$, then $d_k\textbf{p}$ assumes each value in $\calP((r_{ij}\textbf{z})*(1))$ exactly once. This occurs symmetrically in $i$ and $j$, which is to say that for such a pair $(k,\textbf{p})$, the pair $(k,\textbf{p}\sigma_k)$ works with $i$ and $j$ swapped, and we have $d_k(\textbf{p}\sigma_k)=d_k(\textbf{p})$.
\end{prop*}


We can respresent a product $\prod_{i=1}^{n}\gamma_{z_i}(x_i)$ in $E(L^u)\otimes \Gamma(L^r)$ by a class $\left\langle \textbf{z},\textbf{x}\right\rangle$ in $\overline{B}(E^0V(L))$ which we now define:

\begin{itemize}
\setlength{\parindent}{.25in}
\item $(hL)^n:=\{\textbf{x}=(x_1,\ldots,x_n);\ x_i\textup{ homogeneous in }V(L)\}$
%\item $d_{ij}:(hL)^n\to(hL)^{n+1}$ given by $\textbf{x}\mapsto\textbf{x}*(x_ix_j)$
%\item $b_{ij}:(hL)^n\to(hL)^{n+1}$ given by $\textbf{x}\mapsto\textbf{x}*([x_i,x_j])$
\item Choose $\textbf{x}\in(hL)^n$, $\textbf{z}\in\Z^n$ and $\textbf{p}\in\calP(\textbf{z}).$ Then define $\left\langle \textbf{p},\textbf{x}\right\rangle\in\overline{B}_{\sigma\textbf{z},*}(E^0V(L))$ to be the bar $[x_{p_1},\ldots,x_{p_{\sigma\textbf{z}}}]$. Then define $\left\langle \textbf{z},\textbf{x}\right\rangle:= \sum_{\textbf{p}\in\calP(\textbf{z})}\left\langle \textbf{p},\textbf{x}\right\rangle$. 
\end{itemize}
Now $\overline{K}_*(V(L)):=H_{*=*}(E^0V(L))$ is spanned by the bars $\left\langle \textbf{z},\textbf{x}\right\rangle$ where each $x_i$ is a homogeneous element of $L$, which must be restrictable if $z_i>1$ (here, $\sigma\textbf{z}$ is the homological degree). In fact, there's a basis consisting of the $\left\langle \textbf{z},\textbf{x}\right\rangle$ where $\textbf{x}$ runs over (strictly) increasing sequences of elements of an ordered basis chosen for $L$, and the same restrictability condition holds.

Finally we may calculate the differential on $\overline{K}_*(V(L))$. Here goes:
\begin{alignat*}{2}
d\left\langle \textbf{z},\textbf{x}\right\rangle
&=
\sum_{k=1}^{\sigma\textbf{z}-1} \sum_{\textbf{p}\in\calP(\textbf{z})}d_k\left\langle \textbf{p},\textbf{x}\right\rangle%
\\
%&=
%\sum_{k}\sum_{\textbf{p}}\left\langle d_k\textbf{p},d_{p_kp_{k+1}}\textbf{x}\right\rangle%
%\\
&=
\sum_{i=1}^{n}\sum_{j=1}^{n}\sum_{k}\sum_{\textbf{p}:{p_k=i\atop p_{k+1}=j}}\left\langle d_k\textbf{p},\textbf{x}*(x_{p_k}x_{p_{k+1}})\right\rangle%
\\
&=
\sum_{i}\sum_{k}\sum_{\textbf{p}:{p_k=i\atop p_{k+1}=i}}\left\langle d_k\textbf{p},\textbf{x}*(x_ix_i)\right\rangle+\sum_{i<j} \sum_{k}\sum_{\textbf{p}:{p_k=i\atop p_{k+1}=j}}\left\langle d_k\textbf{p},\textbf{x}*(x_ix_j+x_jx_i)\right\rangle%
\\
&=
\sum_{i}\sum_{k}\sum_{\textbf{p}:{p_k=i\atop p_{k+1}=i}}\left\langle d_k\textbf{p},\textbf{x}*\left(\restn{(x_i)}\right)\right\rangle+ \sum_{i<j}\sum_{k}\sum_{\textbf{p}:{p_k=i\atop p_{k+1}=j}}\left\langle d_k\textbf{p},\textbf{x}*\left([x_i,x_j]\right)\right\rangle%
\\
&=
\sum_{i}\sum_{\textbf{q}\in\calP((r_{ii}\textbf{z})*(1))}\left\langle \textbf{q},\textbf{x}*\left(\restn{(x_i)}\right)\right\rangle+\sum_{i<j} \sum_{\textbf{q}\in\calP((r_{ij}\textbf{z})*(1))}\left\langle \textbf{q},\textbf{x}*\left([x_i,x_j]\right)\right\rangle%
\\
% Left hand side
% Relation
&=
% Right hand side
\sum_{i}\left\langle (r_{ii}\textbf{z})*(1),\textbf{x}*\left(\restn{(x_i)}\right)\right\rangle+\sum_{i<j} \left\langle (r_{ij}\textbf{z})*(1),\textbf{x}*\left([x_i,x_j]\right)\right\rangle%
% Comment
\end{alignat*}
This is the obvious restriction of May's $\overline{X}$, see page 141 of ``The cohomology of restricted Lie algebras and of Hopf algebras'' (J.\ Alg.\ 3). We use the subspace with fewer $\gamma_i$, and the differential he gives makes sense on and preserves this subspace, and coincides with that here.

He writes the differential on $f=\prod_{i=1}^{n}\gamma_{z_i}(y_i)$ as
\[d(f)=\left(\sum_{i=1}^{n}f_iy_i\right)+ \sum_{i=1}^{n}f_{i,i}\gamma_1(\restn{(y_i)})+\sum_{1\leq i<j\leq n}f_{i,j}\gamma_1([y_i,y_j]).\]
Here, the parenthetical term vanishes in $\overline{K}$ --- it's there in case we're using nontrivial coefficients.
\end{PRlieKoszulComplexCalculation}

\begin{LieLambdaStructureOnKoszul}
\section{Structure on $\derived\Ind{\nontop{n}}$}
\subsection{Constructions in the homotopy of simplicial Lie algebras}
We showed above that $\PiAlg{\PRLie{n}}$ is in fact $\LL{n+1}$, which is to say that the homotopy of a simplicial object of $\PRLie{n}$ in fact supports both $\Q$ operations and a Lie bracket, giving the structure of an element of $\LL{n+1}$. These constructions are standard, and we record them here for convenience:

Let $\frakg$ be a simplicial Lie algebra over $\F_2$. Let $N\frakg_*$ be the Moore complex of $\frakg$, defined as
\[N\frakg_n:=\bigcap_{i=1}^{n}\ker(d_i:\frakg_n\to\frakg_{n-1})\textup{ with differential }\partial:=d_0|_{N\frakg}.\]
Then $N\frakg_*$ is a differential graded Lie algebra, under the map
\[N\frakg_*\otimes N\frakg_*\overset{\nabla}{\to}N(\frakg_*\otimes \frakg_*)\to N\frakg_*.\]
Here, $\Delta$ is the Eilenberg-Zilber map, or `shuffle map'. Thus, when $|x|=n$ and $|y|=m$:
\[x\otimes y\overset{\nabla}{\mapsto}\sum_{(\alpha,\beta)\in\Shuffles{p}{q}}s_\beta x\otimes s_\alpha y\mapsto \sum_{(\alpha,\beta)}[s_\beta x, s_\alpha y].\]
Here, $\Shuffles{p}{q}$ is the set of $(p,q)$-shuffles, pairs $(\alpha,\beta)$ with $\alpha=(\alpha_p,\ldots,\alpha_1)$ and $\beta=(\beta_q,\ldots,\beta_1)$ a pair of monotonically decreasing sequences such that $\alpha$ and $\beta$ together partition the set $\{0,\ldots,p+q-1\}$.

Next, we have operations $\Q^i:N\frakg_n\to N\frakg_{n+i}$, defined for $0<i\leq n$, by the following formula:
\[x\mapsto \sum_{{(\alpha,\beta)\in\HalfShuffles{i}{i}}}[s_\beta x, s_\alpha x].\]
Here, $\HalfShuffles{i}{i}$ consists of those elements $(\alpha,\beta)$ of $\Shuffles{i}{i}$ such that $\alpha_i<\beta_i$.
One notes that as $i\leq n$, the expressions $s_\alpha x$ and $s_\beta x$ both still make sense, and that when $i=n$, this formula is one half of the formula for $[x,x]$ (in $N\frakg_*$).

Finally, if $\frakg$ supports a partial restriction on a subspace $\frakg^r$, we have an operation $\Q^0:N\frakg^r_n\to N\frakg_n$, defined simply by $x\mapsto\restn(x)$.

\subsection*{$\LL{n+1}$ structure on $(\derived\Ind{\nontop{n}})X$}
We have calculated 

\end{LieLambdaStructureOnKoszul}



\begin{Endmatter}
\printbibliography
\end{Endmatter}
\end{document}















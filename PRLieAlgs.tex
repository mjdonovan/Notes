% !TEX root = z_output/_PRLieAlgs.tex
\documentclass[11pt]{article}
\usepackage{fullpage}
\usepackage{amsmath,amsthm,amssymb}
\usepackage{mathrsfs,nicefrac}
\usepackage{amssymb}
\usepackage{epsfig}
\usepackage[all,2cell]{xy}
\usepackage{sseq}
\usepackage{tocloft}
\usepackage{cancel}
\usepackage[strict]{changepage}
\usepackage{color}
\usepackage{tikz}
\usepackage{extpfeil}
\usepackage{version}
\usepackage{framed}
\definecolor{shadecolor}{rgb}{.925,0.925,0.925}

%\usepackage{ifthen}
%Used for disabling hyperref
\ifx\dontloadhyperref\undefined
%\usepackage[pdftex,pdfborder={0 0 0 [1 1]}]{hyperref}
\usepackage[pdftex,pdfborder={0 0 .5 [1 1]}]{hyperref}
\else
\providecommand{\texorpdfstring}[2]{#1}
\fi
%>>>>>>>>>>>>>>>>>>>>>>>>>>>>>>
%<<<        Versions        <<<
%>>>>>>>>>>>>>>>>>>>>>>>>>>>>>>
%Add in the following line to include all the versions.
%\def\excludeversion#1{\includeversion{#1}}

%>>>>>>>>>>>>>>>>>>>>>>>>>>>>>>
%<<<       Better ToC       <<<
%>>>>>>>>>>>>>>>>>>>>>>>>>>>>>>
\setlength{\cftbeforesecskip}{0.5ex}

%>>>>>>>>>>>>>>>>>>>>>>>>>>>>>>
%<<<      Hyperref mod      <<<
%>>>>>>>>>>>>>>>>>>>>>>>>>>>>>>

%needs more testing
\newcounter{dummyforrefstepcounter}
\newcommand{\labelRIGHTHERE}[1]
{\refstepcounter{dummyforrefstepcounter}\label{#1}}


%>>>>>>>>>>>>>>>>>>>>>>>>>>>>>>
%<<<  Theorem Environments  <<<
%>>>>>>>>>>>>>>>>>>>>>>>>>>>>>>
\ifx\dontloaddefinitionsoftheoremenvironments\undefined
\theoremstyle{plain}
\newtheorem{thm}{Theorem}[section]
\newtheorem*{thm*}{Theorem}
\newtheorem{lem}[thm]{Lemma}
\newtheorem*{lem*}{Lemma}
\newtheorem{prop}[thm]{Proposition}
\newtheorem*{prop*}{Proposition}
\newtheorem{cor}[thm]{Corollary}
\newtheorem*{cor*}{Corollary}
\newtheorem{defprop}[thm]{Definition-Proposition}
\newtheorem*{punchline}{Punchline}
\newtheorem*{conjecture}{Conjecture}
\newtheorem*{claim}{Claim}

\theoremstyle{definition}
\newtheorem{defn}{Definition}[section]
\newtheorem*{defn*}{Definition}
\newtheorem{exmp}{Example}[section]
\newtheorem*{exmp*}{Example}
\newtheorem*{exmps*}{Examples}
\newtheorem*{nonexmp*}{Non-example}
\newtheorem{asspt}{Assumption}[section]
\newtheorem{notation}{Notation}[section]
\newtheorem{exercise}{Exercise}[section]
\newtheorem*{fact*}{Fact}
\newtheorem*{rmk*}{Remark}
\newtheorem{fact}{Fact}
\newtheorem*{aside}{Aside}
\newtheorem*{question}{Question}
\newtheorem*{answer}{Answer}

\else\relax\fi

%>>>>>>>>>>>>>>>>>>>>>>>>>>>>>>
%<<<      Fields, etc.      <<<
%>>>>>>>>>>>>>>>>>>>>>>>>>>>>>>
\DeclareSymbolFont{AMSb}{U}{msb}{m}{n}
\DeclareMathSymbol{\N}{\mathbin}{AMSb}{"4E}
\DeclareMathSymbol{\Octonions}{\mathbin}{AMSb}{"4F}
\DeclareMathSymbol{\Z}{\mathbin}{AMSb}{"5A}
\DeclareMathSymbol{\R}{\mathbin}{AMSb}{"52}
\DeclareMathSymbol{\Q}{\mathbin}{AMSb}{"51}
\DeclareMathSymbol{\PP}{\mathbin}{AMSb}{"50}
\DeclareMathSymbol{\I}{\mathbin}{AMSb}{"49}
\DeclareMathSymbol{\C}{\mathbin}{AMSb}{"43}
\DeclareMathSymbol{\A}{\mathbin}{AMSb}{"41}
\DeclareMathSymbol{\F}{\mathbin}{AMSb}{"46}
\DeclareMathSymbol{\G}{\mathbin}{AMSb}{"47}
\DeclareMathSymbol{\Quaternions}{\mathbin}{AMSb}{"48}


%>>>>>>>>>>>>>>>>>>>>>>>>>>>>>>
%<<<       Operators        <<<
%>>>>>>>>>>>>>>>>>>>>>>>>>>>>>>
\DeclareMathOperator{\ad}{\textbf{ad}}
\DeclareMathOperator{\coker}{coker}
\renewcommand{\ker}{\textup{ker}\,}
\DeclareMathOperator{\End}{End}
\DeclareMathOperator{\Aut}{Aut}
\DeclareMathOperator{\Hom}{Hom}
\DeclareMathOperator{\Maps}{Maps}
\DeclareMathOperator{\Mor}{Mor}
\DeclareMathOperator{\Gal}{Gal}
\DeclareMathOperator{\Ext}{Ext}
\DeclareMathOperator{\Tor}{Tor}
\DeclareMathOperator{\Map}{Map}
\DeclareMathOperator{\Der}{Der}
\DeclareMathOperator{\Rad}{Rad}
\DeclareMathOperator{\rank}{rank}
\DeclareMathOperator{\ArfInvariant}{Arf}
\DeclareMathOperator{\KervaireInvariant}{Ker}
\DeclareMathOperator{\im}{im}
\DeclareMathOperator{\coim}{coim}
\DeclareMathOperator{\trace}{tr}
\DeclareMathOperator{\supp}{supp}
\DeclareMathOperator{\ann}{ann}
\DeclareMathOperator{\spec}{Spec}
\DeclareMathOperator{\SPEC}{\textbf{Spec}}
\DeclareMathOperator{\proj}{Proj}
\DeclareMathOperator{\PROJ}{\textbf{Proj}}
\DeclareMathOperator{\fiber}{F}
\DeclareMathOperator{\cofiber}{C}
\DeclareMathOperator{\cone}{cone}
\DeclareMathOperator{\skel}{sk}
\DeclareMathOperator{\coskel}{cosk}
\DeclareMathOperator{\conn}{conn}
\DeclareMathOperator{\colim}{colim}
\DeclareMathOperator{\limit}{lim}
\DeclareMathOperator{\ch}{ch}
\DeclareMathOperator{\Vect}{Vect}
\DeclareMathOperator{\GrthGrp}{GrthGp}
\DeclareMathOperator{\Sym}{Sym}
\DeclareMathOperator{\Prob}{\mathbb{P}}
\DeclareMathOperator{\Exp}{\mathbb{E}}
\DeclareMathOperator{\GeomMean}{\mathbb{G}}
\DeclareMathOperator{\Var}{Var}
\DeclareMathOperator{\Cov}{Cov}
\DeclareMathOperator{\Sp}{Sp}
\DeclareMathOperator{\Seq}{Seq}
\DeclareMathOperator{\Cyl}{Cyl}
\DeclareMathOperator{\Ev}{Ev}
\DeclareMathOperator{\sh}{sh}
\DeclareMathOperator{\intHom}{\underline{Hom}}
\DeclareMathOperator{\Frac}{frac}



%>>>>>>>>>>>>>>>>>>>>>>>>>>>>>>
%<<<   Cohomology Theories  <<<
%>>>>>>>>>>>>>>>>>>>>>>>>>>>>>>
\DeclareMathOperator{\KR}{{K\R}}
\DeclareMathOperator{\KO}{{KO}}
\DeclareMathOperator{\K}{{K}}
\DeclareMathOperator{\OmegaO}{{\Omega_{\Octonions}}}

%>>>>>>>>>>>>>>>>>>>>>>>>>>>>>>
%<<<   Algebraic Geometry   <<<
%>>>>>>>>>>>>>>>>>>>>>>>>>>>>>>
\DeclareMathOperator{\Spec}{Spec}
\DeclareMathOperator{\Proj}{Proj}
\DeclareMathOperator{\Sing}{Sing}
\DeclareMathOperator{\shfHom}{\mathscr{H}\textit{\!\!om}}
\DeclareMathOperator{\WeilDivisors}{{Div}}
\DeclareMathOperator{\CartierDivisors}{{CaDiv}}
\DeclareMathOperator{\PrincipalWeilDivisors}{{PrDiv}}
\DeclareMathOperator{\LocallyPrincipalWeilDivisors}{{LPDiv}}
\DeclareMathOperator{\PrincipalCartierDivisors}{{PrCaDiv}}
\DeclareMathOperator{\DivisorClass}{{Cl}}
\DeclareMathOperator{\CartierClass}{{CaCl}}
\DeclareMathOperator{\Picard}{{Pic}}
\DeclareMathOperator{\Frob}{Frob}


%>>>>>>>>>>>>>>>>>>>>>>>>>>>>>>
%<<<  Mathematical Objects  <<<
%>>>>>>>>>>>>>>>>>>>>>>>>>>>>>>
\newcommand{\sll}{\mathfrak{sl}}
\newcommand{\gl}{\mathfrak{gl}}
\newcommand{\GL}{\mbox{GL}}
\newcommand{\PGL}{\mbox{PGL}}
\newcommand{\SL}{\mbox{SL}}
\newcommand{\Mat}{\mbox{Mat}}
\newcommand{\Gr}{\textup{Gr}}
\newcommand{\Squ}{\textup{Sq}}
\newcommand{\catSet}{\textit{Sets}}
\newcommand{\RP}{{\R\PP}}
\newcommand{\CP}{{\C\PP}}
\newcommand{\Steen}{\mathscr{A}}
\newcommand{\Orth}{\textup{\textbf{O}}}

%>>>>>>>>>>>>>>>>>>>>>>>>>>>>>>
%<<<  Mathematical Symbols  <<<
%>>>>>>>>>>>>>>>>>>>>>>>>>>>>>>
\newcommand{\DASH}{\textup{---}}
\newcommand{\op}{\textup{op}}
\newcommand{\CW}{\textup{CW}}
\newcommand{\ob}{\textup{ob}\,}
\newcommand{\ho}{\textup{ho}}
\newcommand{\st}{\textup{st}}
\newcommand{\id}{\textup{id}}
\newcommand{\Bullet}{\ensuremath{\bullet} }
\newcommand{\sprod}{\wedge}

%>>>>>>>>>>>>>>>>>>>>>>>>>>>>>>
%<<<      Some Arrows       <<<
%>>>>>>>>>>>>>>>>>>>>>>>>>>>>>>
\newcommand{\nt}{\Longrightarrow}
\let\shortmapsto\mapsto
\let\mapsto\longmapsto
\newcommand{\mapsfrom}{\,\reflectbox{$\mapsto$}\ }
\newcommand{\bigrightsquig}{\scalebox{2}{\ensuremath{\rightsquigarrow}}}
\newcommand{\bigleftsquig}{\reflectbox{\scalebox{2}{\ensuremath{\rightsquigarrow}}}}

%\newcommand{\cofibration}{\xhookrightarrow{\phantom{\ \,{\sim\!}\ \ }}}
%\newcommand{\fibration}{\xtwoheadrightarrow{\phantom{\sim\!}}}
%\newcommand{\acycliccofibration}{\xhookrightarrow{\ \,{\sim\!}\ \ }}
%\newcommand{\acyclicfibration}{\xtwoheadrightarrow{\sim\!}}
%\newcommand{\leftcofibration}{\xhookleftarrow{\phantom{\ \,{\sim\!}\ \ }}}
%\newcommand{\leftfibration}{\xtwoheadleftarrow{\phantom{\sim\!}}}
%\newcommand{\leftacycliccofibration}{\xhookleftarrow{\ \ {\sim\!}\,\ }}
%\newcommand{\leftacyclicfibration}{\xtwoheadleftarrow{\sim\!}}
%\newcommand{\weakequiv}{\xrightarrow{\ \,\sim\,\ }}
%\newcommand{\leftweakequiv}{\xleftarrow{\ \,\sim\,\ }}

\newcommand{\cofibration}
{\xhookrightarrow{\phantom{\ \,{\raisebox{-.3ex}[0ex][0ex]{\scriptsize$\sim$}\!}\ \ }}}
\newcommand{\fibration}
{\xtwoheadrightarrow{\phantom{\raisebox{-.3ex}[0ex][0ex]{\scriptsize$\sim$}\!}}}
\newcommand{\acycliccofibration}
{\xhookrightarrow{\ \,{\raisebox{-.55ex}[0ex][0ex]{\scriptsize$\sim$}\!}\ \ }}
\newcommand{\acyclicfibration}
{\xtwoheadrightarrow{\raisebox{-.6ex}[0ex][0ex]{\scriptsize$\sim$}\!}}
\newcommand{\leftcofibration}
{\xhookleftarrow{\phantom{\ \,{\raisebox{-.3ex}[0ex][0ex]{\scriptsize$\sim$}\!}\ \ }}}
\newcommand{\leftfibration}
{\xtwoheadleftarrow{\phantom{\raisebox{-.3ex}[0ex][0ex]{\scriptsize$\sim$}\!}}}
\newcommand{\leftacycliccofibration}
{\xhookleftarrow{\ \ {\raisebox{-.55ex}[0ex][0ex]{\scriptsize$\sim$}\!}\,\ }}
\newcommand{\leftacyclicfibration}
{\xtwoheadleftarrow{\raisebox{-.6ex}[0ex][0ex]{\scriptsize$\sim$}\!}}
\newcommand{\weakequiv}
{\xrightarrow{\ \,\raisebox{-.3ex}[0ex][0ex]{\scriptsize$\sim$}\,\ }}
\newcommand{\leftweakequiv}
{\xleftarrow{\ \,\raisebox{-.3ex}[0ex][0ex]{\scriptsize$\sim$}\,\ }}

%>>>>>>>>>>>>>>>>>>>>>>>>>>>>>>
%<<<    xymatrix Arrows     <<<
%>>>>>>>>>>>>>>>>>>>>>>>>>>>>>>
\newdir{ >}{{}*!/-5pt/@{>}}
\newcommand{\xycof}{\ar@{ >->}}
\newcommand{\xycofib}{\ar@{^{(}->}}
\newcommand{\xycofibdown}{\ar@{_{(}->}}
\newcommand{\xyfib}{\ar@{->>}}
\newcommand{\xymapsto}{\ar@{|->}}

%>>>>>>>>>>>>>>>>>>>>>>>>>>>>>>
%<<<     Greek Letters      <<<
%>>>>>>>>>>>>>>>>>>>>>>>>>>>>>>
%\newcommand{\oldphi}{\phi}
%\renewcommand{\phi}{\varphi}
\let\oldphi\phi
\let\phi\varphi
\renewcommand{\to}{\longrightarrow}
\newcommand{\from}{\longleftarrow}
\newcommand{\eps}{\varepsilon}

%>>>>>>>>>>>>>>>>>>>>>>>>>>>>>>
%<<<  1st-4th & parentheses <<<
%>>>>>>>>>>>>>>>>>>>>>>>>>>>>>>
\newcommand{\first}{^\text{st}}
\newcommand{\second}{^\text{nd}}
\newcommand{\third}{^\text{rd}}
\newcommand{\fourth}{^\text{th}}
\newcommand{\ZEROTH}{$0^\text{th}$ }
\newcommand{\FIRST}{$1^\text{st}$ }
\newcommand{\SECOND}{$2^\text{nd}$ }
\newcommand{\THIRD}{$3^\text{rd}$ }
\newcommand{\FOURTH}{$4^\text{th}$ }
\newcommand{\iTH}{$i^\text{th}$ }
\newcommand{\jTH}{$j^\text{th}$ }
\newcommand{\nTH}{$n^\text{th}$ }

%>>>>>>>>>>>>>>>>>>>>>>>>>>>>>>
%<<<    upright commands    <<<
%>>>>>>>>>>>>>>>>>>>>>>>>>>>>>>
\newcommand{\upcol}{\textup{:}}
\newcommand{\upsemi}{\textup{;}}
\providecommand{\lparen}{\textup{(}}
\providecommand{\rparen}{\textup{)}}
\renewcommand{\lparen}{\textup{(}}
\renewcommand{\rparen}{\textup{)}}
\newcommand{\Iff}{\emph{iff} }

%>>>>>>>>>>>>>>>>>>>>>>>>>>>>>>
%<<<     Environments       <<<
%>>>>>>>>>>>>>>>>>>>>>>>>>>>>>>
\newcommand{\squishlist}
{ %\setlength{\topsep}{100pt} doesn't seem to do anything.
  \setlength{\itemsep}{.5pt}
  \setlength{\parskip}{0pt}
  \setlength{\parsep}{0pt}}
\newenvironment{itemise}{
\begin{list}{\textup{$\rightsquigarrow$}}
   {  \setlength{\topsep}{1mm}
      \setlength{\itemsep}{1pt}
      \setlength{\parskip}{0pt}
      \setlength{\parsep}{0pt}
   }
}{\end{list}\vspace{-.1cm}}
\newcommand{\INDENT}{\textbf{}\phantom{space}}
\renewcommand{\INDENT}{\rule{.7cm}{0cm}}

\newcommand{\itm}[1][$\rightsquigarrow$]{\item[{\makebox[.5cm][c]{\textup{#1}}}]}


%\newcommand{\rednote}[1]{{\color{red}#1}\makebox[0cm][l]{\scalebox{.1}{rednote}}}
%\newcommand{\bluenote}[1]{{\color{blue}#1}\makebox[0cm][l]{\scalebox{.1}{rednote}}}

\newcommand{\rednote}[1]
{{\color{red}#1}\makebox[0cm][l]{\scalebox{.1}{\rotatebox{90}{?????}}}}
\newcommand{\bluenote}[1]
{{\color{blue}#1}\makebox[0cm][l]{\scalebox{.1}{\rotatebox{90}{?????}}}}


\newcommand{\funcdef}[4]{\begin{align*}
#1&\to #2\\
#3&\mapsto#4
\end{align*}}

%\newcommand{\comment}[1]{}

%>>>>>>>>>>>>>>>>>>>>>>>>>>>>>>
%<<<       Categories       <<<
%>>>>>>>>>>>>>>>>>>>>>>>>>>>>>>
\newcommand{\Ens}{{\mathscr{E}ns}}
\DeclareMathOperator{\Sheaves}{{\mathsf{Shf}}}
\DeclareMathOperator{\Presheaves}{{\mathsf{PreShf}}}
\DeclareMathOperator{\Psh}{{\mathsf{Psh}}}
\DeclareMathOperator{\Shf}{{\mathsf{Shf}}}
\DeclareMathOperator{\Varieties}{{\mathsf{Var}}}
\DeclareMathOperator{\Schemes}{{\mathsf{Sch}}}
\DeclareMathOperator{\Rings}{{\mathsf{Rings}}}
\DeclareMathOperator{\AbGp}{{\mathsf{AbGp}}}
\DeclareMathOperator{\Modules}{{\mathsf{\!-Mod}}}
\DeclareMathOperator{\fgModules}{{\mathsf{\!-Mod}^{\textup{fg}}}}
\DeclareMathOperator{\QuasiCoherent}{{\mathsf{QCoh}}}
\DeclareMathOperator{\Coherent}{{\mathsf{Coh}}}
\DeclareMathOperator{\GSW}{{\mathcal{SW}^G}}
\DeclareMathOperator{\Burnside}{{\mathsf{Burn}}}
\DeclareMathOperator{\GSet}{{G\mathsf{Set}}}
\DeclareMathOperator{\FinGSet}{{G\mathsf{Set}^\textup{fin}}}
\DeclareMathOperator{\HSet}{{H\mathsf{Set}}}
\DeclareMathOperator{\Cat}{{\mathsf{Cat}}}
\DeclareMathOperator{\Fun}{{\mathsf{Fun}}}
\DeclareMathOperator{\Orb}{{\mathsf{Orb}}}
\DeclareMathOperator{\Set}{{\mathsf{Set}}}
\DeclareMathOperator{\sSet}{{\mathsf{sSet}}}
\DeclareMathOperator{\Top}{{\mathsf{Top}}}
\DeclareMathOperator{\GSpectra}{{G-\mathsf{Spectra}}}
\DeclareMathOperator{\Lan}{Lan}
\DeclareMathOperator{\Ran}{Ran}

%>>>>>>>>>>>>>>>>>>>>>>>>>>>>>>
%<<<     Script Letters     <<<
%>>>>>>>>>>>>>>>>>>>>>>>>>>>>>>
\newcommand{\scrQ}{\mathscr{Q}}
\newcommand{\scrW}{\mathscr{W}}
\newcommand{\scrE}{\mathscr{E}}
\newcommand{\scrR}{\mathscr{R}}
\newcommand{\scrT}{\mathscr{T}}
\newcommand{\scrY}{\mathscr{Y}}
\newcommand{\scrU}{\mathscr{U}}
\newcommand{\scrI}{\mathscr{I}}
\newcommand{\scrO}{\mathscr{O}}
\newcommand{\scrP}{\mathscr{P}}
\newcommand{\scrA}{\mathscr{A}}
\newcommand{\scrS}{\mathscr{S}}
\newcommand{\scrD}{\mathscr{D}}
\newcommand{\scrF}{\mathscr{F}}
\newcommand{\scrG}{\mathscr{G}}
\newcommand{\scrH}{\mathscr{H}}
\newcommand{\scrJ}{\mathscr{J}}
\newcommand{\scrK}{\mathscr{K}}
\newcommand{\scrL}{\mathscr{L}}
\newcommand{\scrZ}{\mathscr{Z}}
\newcommand{\scrX}{\mathscr{X}}
\newcommand{\scrC}{\mathscr{C}}
\newcommand{\scrV}{\mathscr{V}}
\newcommand{\scrB}{\mathscr{B}}
\newcommand{\scrN}{\mathscr{N}}
\newcommand{\scrM}{\mathscr{M}}

%>>>>>>>>>>>>>>>>>>>>>>>>>>>>>>
%<<<     Fractur Letters    <<<
%>>>>>>>>>>>>>>>>>>>>>>>>>>>>>>
\newcommand{\frakQ}{\mathfrak{Q}}
\newcommand{\frakW}{\mathfrak{W}}
\newcommand{\frakE}{\mathfrak{E}}
\newcommand{\frakR}{\mathfrak{R}}
\newcommand{\frakT}{\mathfrak{T}}
\newcommand{\frakY}{\mathfrak{Y}}
\newcommand{\frakU}{\mathfrak{U}}
\newcommand{\frakI}{\mathfrak{I}}
\newcommand{\frakO}{\mathfrak{O}}
\newcommand{\frakP}{\mathfrak{P}}
\newcommand{\frakA}{\mathfrak{A}}
\newcommand{\frakS}{\mathfrak{S}}
\newcommand{\frakD}{\mathfrak{D}}
\newcommand{\frakF}{\mathfrak{F}}
\newcommand{\frakG}{\mathfrak{G}}
\newcommand{\frakH}{\mathfrak{H}}
\newcommand{\frakJ}{\mathfrak{J}}
\newcommand{\frakK}{\mathfrak{K}}
\newcommand{\frakL}{\mathfrak{L}}
\newcommand{\frakZ}{\mathfrak{Z}}
\newcommand{\frakX}{\mathfrak{X}}
\newcommand{\frakC}{\mathfrak{C}}
\newcommand{\frakV}{\mathfrak{V}}
\newcommand{\frakB}{\mathfrak{B}}
\newcommand{\frakN}{\mathfrak{N}}
\newcommand{\frakM}{\mathfrak{M}}

\newcommand{\frakq}{\mathfrak{q}}
\newcommand{\frakw}{\mathfrak{w}}
\newcommand{\frake}{\mathfrak{e}}
\newcommand{\frakr}{\mathfrak{r}}
\newcommand{\frakt}{\mathfrak{t}}
\newcommand{\fraky}{\mathfrak{y}}
\newcommand{\fraku}{\mathfrak{u}}
\newcommand{\fraki}{\mathfrak{i}}
\newcommand{\frako}{\mathfrak{o}}
\newcommand{\frakp}{\mathfrak{p}}
\newcommand{\fraka}{\mathfrak{a}}
\newcommand{\fraks}{\mathfrak{s}}
\newcommand{\frakd}{\mathfrak{d}}
\newcommand{\frakf}{\mathfrak{f}}
\newcommand{\frakg}{\mathfrak{g}}
\newcommand{\frakh}{\mathfrak{h}}
\newcommand{\frakj}{\mathfrak{j}}
\newcommand{\frakk}{\mathfrak{k}}
\newcommand{\frakl}{\mathfrak{l}}
\newcommand{\frakz}{\mathfrak{z}}
\newcommand{\frakx}{\mathfrak{x}}
\newcommand{\frakc}{\mathfrak{c}}
\newcommand{\frakv}{\mathfrak{v}}
\newcommand{\frakb}{\mathfrak{b}}
\newcommand{\frakn}{\mathfrak{n}}
\newcommand{\frakm}{\mathfrak{m}}

%>>>>>>>>>>>>>>>>>>>>>>>>>>>>>>
%<<<  Caligraphic Letters   <<<
%>>>>>>>>>>>>>>>>>>>>>>>>>>>>>>
\newcommand{\calQ}{\mathcal{Q}}
\newcommand{\calW}{\mathcal{W}}
\newcommand{\calE}{\mathcal{E}}
\newcommand{\calR}{\mathcal{R}}
\newcommand{\calT}{\mathcal{T}}
\newcommand{\calY}{\mathcal{Y}}
\newcommand{\calU}{\mathcal{U}}
\newcommand{\calI}{\mathcal{I}}
\newcommand{\calO}{\mathcal{O}}
\newcommand{\calP}{\mathcal{P}}
\newcommand{\calA}{\mathcal{A}}
\newcommand{\calS}{\mathcal{S}}
\newcommand{\calD}{\mathcal{D}}
\newcommand{\calF}{\mathcal{F}}
\newcommand{\calG}{\mathcal{G}}
\newcommand{\calH}{\mathcal{H}}
\newcommand{\calJ}{\mathcal{J}}
\newcommand{\calK}{\mathcal{K}}
\newcommand{\calL}{\mathcal{L}}
\newcommand{\calZ}{\mathcal{Z}}
\newcommand{\calX}{\mathcal{X}}
\newcommand{\calC}{\mathcal{C}}
\newcommand{\calV}{\mathcal{V}}
\newcommand{\calB}{\mathcal{B}}
\newcommand{\calN}{\mathcal{N}}
\newcommand{\calM}{\mathcal{M}}

%>>>>>>>>>>>>>>>>>>>>>>>>>>>>>>
%<<<<<<<<<DEPRECIATED<<<<<<<<<<
%>>>>>>>>>>>>>>>>>>>>>>>>>>>>>>

%%% From Kac's template
% 1-inch margins, from fullpage.sty by H.Partl, Version 2, Dec. 15, 1988.
%\topmargin 0pt
%\advance \topmargin by -\headheight
%\advance \topmargin by -\headsep
%\textheight 9.1in
%\oddsidemargin 0pt
%\evensidemargin \oddsidemargin
%\marginparwidth 0.5in
%\textwidth 6.5in
%
%\parindent 0in
%\parskip 1.5ex
%%\renewcommand{\baselinestretch}{1.25}

%%% From the net
%\newcommand{\pullbackcorner}[1][dr]{\save*!/#1+1.2pc/#1:(1,-1)@^{|-}\restore}
%\newcommand{\pushoutcorner}[1][dr]{\save*!/#1-1.2pc/#1:(-1,1)@^{|-}\restore}









\newcommand{\Aff}{\textup{Aff}}


\newcommand{\GoodLie}[1]{\mathsf{g}{\scrL}^{#1}}%{{\mu\scrL}^{#1}}
\newcommand{\BadLie}[1]{\mathsf{b}{\scrL}^{#1}}%{{\mu\scrL}^{#1}}


\newcommand{\LL}[1]{{\scrK}^{#1}}%{{\mu\scrL}^{#1}}
\newcommand{\PR}[1]{\scrR^{#1}}%{{\xi\scrL}^{#1}}
\newcommand{\GR}[1]{\scrG^{#1}}%{\mathsf{gr}^{#1}}
\newcommand{\GS}[1]{\scrG\!\scrS^{#1}}
\newcommand{\GpS}[1]{\scrG\!\scrS^{#1}_*}
\newcommand{\nontop}[1]{\scrU^{#1}}%{\mu_{\mathsf{nt}}^{#1}}
\newcommand{\PiAlg}[1]{#1\textup{-$\Pi$-alg}}
\newcommand{\produces}[1]{\overset{#1}{\rightsquigarrow}}
\newcommand{\restn}[1]{\xi{#1}}
\newcommand{\Ind}[2][]{\mathbf{I}{#2}_{#1}}%[internal indices]{category name}
\newcommand{\forget}{\mathrm{fg}}
\newcommand{\Fr}[1]{\mathrm{Fr}_{#1}}
\newcommand{\derived}{\mathbb{L}}
\newcommand{\LambdaOp}{\xi^{+}}
\newcommand{\SqShift}{\Sq_{+}}
\newcommand{\Sq}{\mathrm{Sq}}
\newcommand{\LieSteen}{\calA(\calL)}

\begin{document}

\section*{Grothendieck Spectral Sequences}
\subsection*{Some important algebras}
Let  $\LieSteen$ be the Steenrod algebra for simplicial restricted Lie algebras [priddy PrimHtpy \S7]. This is the quotient of the standard Lie algebra by the two-sided ideal generated by $\Sq^0$. We'll introduce the notation $\SqShift^i:=\Sq^{i+1}$, so that $\LieSteen$ is generated by $\SqShift^i$ for $i\geq0$, subject to relations:
\[\SqShift^i\SqShift^j=\sum_{k=0}^{(a-1)/2}{j-k-1\choose i-2k-1}\SqShift^{i+j-k}\SqShift^k\textup{\ for $i\leq2j$.}\]
This is a homogeneous Koszul algebra in the sense of [Priddy KoszulResns]. It's cohomology algebra $H^*\LieSteen:=\Ext_{\LieSteen}^{**}(\F_2,\F_2)$ is also a homogeneous Koszul algebra, generated by the $\LambdaOp_i:=\xi_{i+1}$ for $i\geq0$, where $\LambdaOp_i$ is dual to $\SqShift^i$ [see priddy PrimHtpy \S8]. These classes satisfy their own Adem relations:
\[\LambdaOp_i\LambdaOp_j=\sum_{k=0}^{(i-2j)/2-1}{i-2j-2-k\choose k}\LambdaOp_{2j+1+k}\LambdaOp_{i-j-1-k}\textup{\ for $i>2j$.}\]
Importantly, $H^*\LieSteen$ is isomorphic to the opposite of the $\Lambda$-algebra, and $\LambdaOp_i\longleftrightarrow\lambda_i$ under this anti-isomorphism. The $\Lambda$-algebra has an unstable right action on the homotopy of any simplicial Lie algebra (details to follow), so that $H^*\LieSteen$ has an unstable left action thereupon. We'll prefer the left action of $H^*\LieSteen$ in order to simplify notation.

\subsection*{Categories of interest}
\begin{itemize}
\setlength{\parindent}{.25in}
\item For $n\geq1$, let $\GpS{n}$ be the category of $n$-times non-negatively graded pointed sets $X$ whose graded part $X_{a_n,\ldots,a_1}$ is a single point whenever $a_1=0$. That is, an object $X\in\GpS{n}$ is a collection $X_{a_n,\ldots,a_1}$ of pointed sets, one for each $n$-tuple $(a_n,\ldots,a_1)$ of non-negative integers, such that $X_{a_n,\ldots,a_2,0}=\{\star\}$ for all $a_n,\ldots,a_2$. For $x\in X_{a_n,\ldots,a_1}$, we write $|x|:=a_n$.




%All the categories $\mathsf{C}$ to be defined have forgetful functors $\forget:\mathsf{C}\to\GpS{n}$, as in each, an object will be an 
\item For $n\geq1$, let $\GR{n}$ be the category of $n$-times non-negatively graded $\F_2$-vector spaces $V$ whose graded part $V_{a_n,\ldots,a_1}$ is zero whenever $a_1=0$.  For a homogeneous element $v\in V_{a_n,\ldots,a_1}$, write $|v|:=a_n$.

All the categories $\mathsf{C}$ to be defined below will consist of objects $X=\bigoplus X_{a_n,\ldots,a_1}$ of $\GR{n}$ enriched with certain extra structure (for some $n$). For $x\in X_{a_n,\ldots,a_1}$ a non-zero homogeneous element of $X$, we will continue to write $|x|:=a_n$ in each of the following contexts.

\item For $n\geq1$, let $\BadLie{n}$ be the following category of $n$-graded Lie algebras over $\F_2$. An object of $\BadLie{n}$ is to be an object $L$ of $\GR{n}$ with (skew-)symmetric maps
\[[\DASH,\DASH]:L_{a_n,\ldots,a_1}\otimes L_{b_n,\ldots,b_1}\to L_{a_n+b_n,\ldots,a_2+b_2,a_1+b_1+1}\]
satisfying the Jacobi identity. Note that this Lie bracket has degree $(0,\ldots,0,1)$. As the ground field has characteristic 2, the concepts of symmetry and skew-symmetry coincide. More impartantly, skew-symmetry does not imply the `alternating' relation, so that there may be $x\in L$ with $[x,x]$ nonzero. This explains the $\mathsf{b}$ in the notation --- we refer to such Lie algebras as \emph{bad} Lie algebras.
\item For $n\geq1$, let $\GoodLie{n}$ be the full subcategory of $\BadLie{n}$ consisting of those Lie algebras which are \emph{good}. That is, those $L\in\BadLie{n}$ for which $[x,x]=0$ for all $x\in L$.

\item For $n\geq2$, let $\LL{n}$ be the following category of $n$-graded \emph{good} Lie algebras with unstable left action of the $\LambdaOp$-algebra. An object of $\LL{n}$ consists of an object $L$ of $\GoodLie{n}$
%, under a $(0,\ldots,0,1)$-shifted bracket:
%\[[\DASH,\DASH]:L_{a_n,\ldots,a_1}\otimes L_{b_n,\ldots,b_1}\to L_{a_n+b_n,\ldots,a_2+b_2,a_1+b_1+1}\]
admitting $\LambdaOp$ operations
\[\LambdaOp_i:L_{a_n,\ldots,a_1}\to L_{a_n+i,2a_{n-1},\ldots,2a_2,2a_1+1}\]
defined whenever $i\leq a_n$ and not all of $i,a_{n-1},\ldots,a_{2}$ are zero. %By `good' Lie algebra, we mean that the bracket satisfies the Jacobi identity, and is alternating (``$[x,x]=0$ for all $x$'').
%The graded part $L_{a_n,\ldots,a_1}$ is assumed to be zero whenever $a_1=0$, and for $x\in L_{a_n,\ldots,a_1}$ we write $|x|:=a_n$.
The $\LambdaOp$ operations are assumed to satisfy the following $\LambdaOp$-Adem relations wherever they make sense. If $i>2j$, and $\LambdaOp_i\LambdaOp_jx$ is defined (i.e.\ $j\leq|x|$, not all of $j,a_{n-1},\ldots,a_2$ are zero, and $i\leq|x|+j$):
\[\LambdaOp_i\LambdaOp_jx=\sum_{k=0}^{(i-2j)/2-1}{i-2j-2-k\choose k}\LambdaOp_{2j+1+k}\LambdaOp_{i-j-1-k}x.\]
The non-top operations are linear and are killed by the bracket.
That is, for $x,y\in L_{a_n,\ldots,a_1}$ and $z\in L$: \[\LambdaOp_i{(x+y)}=\LambdaOp_i{x}+\LambdaOp_i{y}\textup{ and }[\LambdaOp_i{x},z]=0.\]
On the other hand, the top operation, 
\[\LambdaOp_i:L_{i, a_{n-1},\ldots,a_1}\to L_{2i, 2a_{n-1},\ldots,2a_2,2a_1+1}\]
acts as a partially defined restriction for $L$, satsfying, for $x,y\in L_{i, a_{n-1},\ldots,a_1}$ and $z\in L$,
\[\LambdaOp_i(x+y)=\LambdaOp_ix+\LambdaOp_iy+[x,y]\textup{ and }[\LambdaOp_{i}x,z]=[x,[x,z]].\]

\begin{shaded}\noindent
The $\LambdaOp$-Adem relations above make sense, as they preserve definedness:
\[\textup{if $(i,j)\produces{\LambdaOp}(i',j')$, then $i>j+i/2\geq i'\geq2j+1$, and $i-j-1\geq j'\geq i/2>j$.}\]
Thus, if $\LambdaOp_i\LambdaOp_jx$ is defined, so is $\LambdaOp_{i'}\LambdaOp_{j'}x$ --- neither of $i'$ or $j'$ are zero, and one notes that $j'\leq i-j-1<|x|$ and $i'-j'\leq (j+i/2)-(i/2)<i/2\leq(|x|+j)/2\leq |x|$, so that it makes sense to write $\LambdaOp_{j'}x$, and $\LambdaOp_{i'}\LambdaOp_{j'}x$ respectively. In fact, as each of these chains involved a strict inequality, the composite $\LambdaOp_{i'}\LambdaOp_{j'}$ makes sense on \emph{strictly more} degrees than $\LambdaOp_{i}\LambdaOp_{j}$ does.
\end{shaded}

\item Let $\LL{1}$ be the full subcategory of connected objects in $\calW$. We already have the definition for these in Goerss' book, but I'll spell this out here, to emphasise certain interesting aspects of the character of $\LL{1}$.
An object of $\LL{1}$ is a (potentially bad) graded Lie algebra $L$ in $\BadLie{1}$
admitting $P$ operations
\[P^i:L_{a_1}\to L_{a_1+i+1}\]
which are \emph{always} defined for $i\geq0$, but \emph{equal zero} unless $2\leq i\leq a_n$.
%By `bad' Lie algebra, we mean that the bracket satisfies the Jacobi identity, and is skew symmetric (``$[x,y]=[y,x]$ for all $x,y$'').
%The graded part $L_{0}$ is assumed to be zero, and for $x\in L_{a_1}$ we write $|x|:=a_1$. 
The $P$ operations are assumed to satisfy the following $P$-Adem relations. If $i\geq 2j$:
\[P^iP^jx=\sum_{s=i-j+1}^{i+j-2}{2s-i-1\choose s-j}P^{i+j-s}P^sx.\]
Every $P$ operation is assumed linear, and $[x,P^iy]=0$ for all $i,x,y$. Additionally, the top $P$ operation is the `self-square': for $x\in L_{i}$, $P^{i}x=[x,x]$.


\item For $n\geq2$, let $\nontop{n}$ be the following category of $n$-graded unstable $\LambdaOp$-modules. An object of $\nontop{n}$ is an $n$-graded vector space $M$ in $\GR{n}$,
admitting linear operations
\[\LambdaOp_i:M_{a_n,\ldots,a_1}\to M_{a_n+i,2a_{n-1},\ldots,2a_2,2a_1+1}\]
defined whenever $i< a_n$ (note the strict inequality) and not all of $i,a_{n-1},\ldots,a_{2}$ are zero.
%The graded part $M_{a_n,\ldots,a_1}$ is assumed to be zero whenever $a_1=0$, and for $x\in M_{a_n,\ldots,a_1}$ we write $|x|:=a_n$. 
The $\LambdaOp$ operations are assumed to satisfy the same $\LambdaOp$-Adem relations as in $\LL{n}$, wherever they make sense in $\nontop{n}$. The same discussion shows that this too is a meaningful definition. %Every operation $\LambdaOp_i$ is assumed linear.

\item Let $\nontop{1}$ be the following category of graded unstable $P$-modules. An object of $\nontop{1}$ is a graded vector space $M$ in $\GR{1}$,
admitting linear operations
\[P^i:M_{a_1}\to M_{a_1+i+1}\]
which are always defined, but are zero unless $2\leq i\leq a_n$ (note that the inequality is not strict).
%The graded part $M_{0}$ is assumed to be zero, and for $x\in M_{a_n,\ldots,a_1}$ we write $|x|:=a_n$. 
The $P$ operations are assumed to satisfy the same $P$-Adem relations as in $\LL{1}$. %Every operation $P^i$ is assumed linear.

\item 

For $n\geq1$, let $\PR{n}$ be the following category of partially restricted $n$-graded \emph{good} Lie algebras. An object of $\PR{n}$ consists is an object $L$ of $\GoodLie{n}$
%For $n\geq1$, let $\PR{n}$ be the following category of partially restricted $n$-graded Lie algebras. An object of $\PR{n}$ is a good Lie algebra $L=\bigoplus L_{a_n,\ldots,a_1}$, under a $(0,\ldots,0,1)$-shifted bracket as above
admitting restriction operations
\[\restn{}:L_{a_n,\ldots,a_1}\to L_{2a_n,2a_{n-1},\ldots,2a_2,2a_1+1}\]
defined whenever not all of $a_n,\ldots,a_{2}$ are zero. %The graded part $L_{a_n,\ldots,a_1}$ is assumed to be zero whenever $a_1=0$, and we write $|x|$ as above. 
For $x,y\in L_{a_n,\ldots,a_1}$ and $z\in L$ the restriction operations satisfy \[\restn{(x+y)}=\restn{x}+\restn{y}+[x,y]\textup{ and }[\restn{x},z]=[x,[x,z]].\]
Note that when $n=1$ the restriction is not defined anywhere.
\begin{shaded}\noindent
We can define $\restn{}$ on the the direct sum of those $L_{a_n,\ldots,a_1}$ for which not all of $a_n,\ldots,a_2$ are zero, using the above formula for $\restn{(x+y)}$. Then both of the above two formulae will still hold for non-homogeneous $x,y,z$. Thus, we could equally have defined $\PR{n}$ by giving a restriction on this entire ideal.
\end{shaded}

\end{itemize}

\subsection*{Free constructions}
For any of the above categories $\mathsf{C}$ there are free/forget adjunctions $\Fr{\mathsf{C}}:\GS{n} \rightleftarrows \mathsf{C}$. In this section we'll describe the free constructions.

\begin{prop*}
For $n\geq2$, for $X\in\GS{n}$ a graded set, $\Fr{\LL{n}}X$ has basis the compositions $\LambdaOp_{I}b$, where $b$ runs over a homogeneous Hall basis for the free (unrestricted) Lie algebra $L^uX$, and $I=(i_1,\ldots,i_\ell)$ is a $\LambdaOp$-admissible sequence with $i_\ell\leq|b|$. Here, $L^uX$ is an object of $\GoodLie{n}$ if we stipulate that the bracket shifts degrees as appropriate.
\end{prop*}
\begin{proof}
Choose a simplicial object of $\GR{n-1}$, say $V\in s(\GR{n-1})$, such that $\pi_{*}V=\Fr{\GR{n}}X$. Then [CurtisSimplHtpy thm 8.8] states that $\pi_*L^r(V)=L^r(\pi_*V)\hat\otimes \Lambda$. Here $\Lambda$ is the $\Lambda$-algebra, $L^r$ is the free restricted Lie algebra functor. Now the $\LambdaOp$-algebra [see PriddyPrimOpns, 9.1, and 
\end{proof}

$\calA(\calL)$ the Steenrod algebra for simplicial Lie algebras. $\LambdaOp_i$ (i.e.\ $\xi_{i+1}$) for $i\geq0$ generate $\Ext_{\calA(\calL)}$. [Priddy 8.1].

\end{document}

\subsection*{A diagram of functors}
Now for each category $\mathsf{C}$ amoungst $\LL{n}$, $\PR{n}$ and $\nontop{n}$, there is a functor $\Ind{\mathsf{C}}:\mathsf{C\overset{}{\to}\GR{n}}$, which takes the quotient of the underlying vector space by the subspace spanned by the image of all of the structure maps. There are also free/forget adjunctions $\GpS{n} \rightleftarrows \mathsf{C}$. Finally, there's a forgetful functor $\forget:\LL{n}\to\nontop{n}$, which, importantly, preserves  free objects%footnote:
\footnote{I think that the adjunction does need to come from graded pointed sets, but that that's no problem for Blanc-Stover. The only thing that matters is that $\forget$ preserves frees, but it does --- frees in $\nontop{n}$ are easy to describe.}. These functors give the solid maps in the following diagram:
\[\xymatrix@R=8mm@C=15mm@!C{
\LL{n}
\ar@/^1.5em/[rr]^-{\Ind{\LL{n}}}
\ar[d]_-{\forget}
\ar@{-->}[r]_-{\Ind{\nontop{n}}}
&%r1c1
\PR{n}
\ar[d]^-{\forget}
\ar[r]_-{\Ind{\PR{n}}}
&%r1c2
\GR{n}
\\%r1c3
\nontop{n}
\ar[r]_-{\Ind{\nontop{n}}}
&%r2c1
\GR{n}
&%r2c2
%r2c3
}\]
The point is that there is a functor $\LL{n}\to\PR{n}$ which completes this diagram to a commuting square and triangle, which we also name $\Ind{\nontop{n}}$. It will be important to note that $\Ind{\nontop{n}}$ sends free objects in $\LL{n}$ to free objects in $\PR{n}$.

When $n=1$, $\Ind{\nontop{n}}$ quotients the underlying vector space by the images of all of the $P$ operations. The result inherits the structure of an element of $\PR{n}$ from the bracket in $\LL{1}$. This is well defined due to the axioms ``$[x,P^iy]=0$'' and ``$P^{|x|}x=[x,x]$''. 

When $n\geq2$, $\Ind{\nontop{n}}$ quotients the underlying vector space by the images of all of the non-top $\LambdaOp$ operations. That is, by the span of the $\LambdaOp_ix$ for $i<|x|$. The result inherits the structure of an element of $\PR{n}$ as follows. The Lie bracket is induced by that in $\LL{n}$; well defined by the axiom ``$[\LambdaOp_i{x},z]=0$ for $i<|x|$''. The restriction is induced by the top $\LambdaOp$ operations, being defined by the formula $\restn{x}:=\LambdaOp_{|x|}x$. It is a little less obvious that this is in fact well defined. One must check, for homogeneous $x,y\in L$, and $0\leq i<|y|$ such that $\LambdaOp_iy$ is defined and $i+|y|=|x|$, that $\LambdaOp_{|x|}x$ and $\LambdaOp_{|x|}(x+\LambdaOp_iy)$ differ by a sum of images of non-top $\LambdaOp$ operations. But
\[\LambdaOp_{|x|}(x+\LambdaOp_iy)-\LambdaOp_{|x|}x=\LambdaOp_{|x|}(\LambdaOp_iy)+[x,\LambdaOp_iy]=\LambdaOp_{|x|}\LambdaOp_iy,\]
and the above calcuation regarding indices in the $\LambdaOp$-Adem relations shows that $\LambdaOp_{|x|}\LambdaOp_iy$ can be rewritten without the appearance of a top $\LambdaOp$.

\subsection*{Grothendieck spectral sequences}
We are interested in the derived functors $(\derived\Ind{\LL{n}})X\in\GR{n+1}$ for $X\in\LL{n}$. It is convenient for us to omit the customary asterisk in $\derived_*$, and to think of the derived functors as a single object with a one more grading than $X$ has. We always write the new grading to the left of the previous gradings, as is customary. To study $(\derived\Ind{\LL{n}})(X)$ we have a Grothendieck spectral sequence, thanks for Blanc-Stover, or some other approach. As Blanc and Stover write it, the spectral sequence takes the form:
\[E^2_{st}=(\derived_s(\overline{\Ind{\PR{n}}}_t))(\derived\Ind{\nontop{n}})X\implies(\derived_{s+t}\Ind{\LL{n}})X\textup{ with }d^{r}:E_{st}\to E_{s-r,t+r-1}.\]
To explain this, note that $\derived\Ind{\nontop{n}}\in\PiAlg{\PR{n}}$, and $\overline{\Ind{\PR{n}}}$ is the functor $\PiAlg{\PR{n}}\to\GR{n+1}$ induced by $\Ind{\PR{n}}$. The $t$ subscript picks out the part with (new) grading $t$. It might be simpler to write:
\[E^2=(\derived\overline{\Ind{\PR{n}}})(\derived\Ind{\nontop{n}})X\implies(\derived\Ind{\LL{n}})X\textup{ with }\deg(d^r)=(-r,r-1,0,\ldots,0).\]
Here, each $E^r$ is $(n+2)$-graded, while $(\derived\Ind{\LL{n}})X$ is $(n+1)$-graded. Each page $E^r$ has two homological degrees in addition to the $n$ internal degrees of $X$, and these extra two degrees on $E^\infty$ sum to the single homological degree in $(\derived\Ind{\LL{n}})X$.

The final piece of information here is that $\PiAlg{\PR{n}}\simeq\LL{n+1}$, and under this isomorphism, $\overline{\Ind{\PR{n}}}$ is identified with $\Ind{\LL{n+1}}$. Thus, we may write the spectral sequence as:
\[E^2=(\derived{\Ind{\LL{n+1}}})(\derived\Ind{\nontop{n}})X\implies(\derived\Ind{\LL{n}})X.\]

\subsection*{Koszul complexes}
\begin{itemise}
\setlength{\parindent}{.25in}
\item Calculate the free constructions.
\end{itemise}



%Then there's a spectral sequence... it has dr of degree $(?,?,0,\ldots,0)$, it converges to the associated graded via summing the front two indices....



\end{document}

\section*{The Koszul complex for partially restricted Lie algebras}

\subsection*{Set-theoretic preliminaries}

\begin{itemize}
\setlength{\parindent}{.25in}
\item $\Z^n:=\{\textbf{z}=(z_1,\ldots,z_n);\ z_i\in\Z\}$
\item $r_i:\Z^n\to\Z^n$ given by $(z_1,\ldots,z_n)\mapsto(z_1,\ldots,z_i-1,\ldots,z_n)$
\item $r_{ij}:=r_i\circ r_j:\Z^n\to\Z^n$
\item $\sigma:\Z^n\to\Z$ given by summing entries
\item for $\textbf{z}\in\Z^n$ write $\calP(\textbf{z})$ for the set of permutations of the multiset $\{i^{z_i};i=1,\ldots,n\}$, written as sequences $\textbf{p}=(p_1,\ldots,p_{\sigma \textbf{z}})$. This set is empty if $\textbf{z}$ has a negative entry. For such a $\textbf{p}$, write $p\sigma_i$ for $(p_1,\ldots,p_{i-1},p_{i+1},p_{i},p_{i+2},\ldots,p_{\sigma \textbf{z}})$.
\item $d_k:\calP(\textbf{z})\to\calP((r_{p_kp_{k+1}}\textbf{z})*(1))$ (the star denotes concatenation), in which $(p_1,\ldots,p_{\sigma \textbf{z}})$ is sent to $(p_1,\ldots,p_{k-1},\ell(\textbf{z})+1,p_{k+2},\ldots,p_{\sigma \textbf{z}})$. That is, two entries are removed, and in their position we place a single entry, $\ell(\textbf{z})+1$.
\end{itemize}
\begin{prop*}
Choose $\textbf{z}\in\Z^n$, and $i$ satisfying $1\leq i\leq n$. Then as $k$ ranges from $1$ to $\sigma \textbf{z}-1$, and $\textbf{p}$ ranges through those elements of $\calP(\textbf{z})$ satisfying $(p_k,p_{k+1})=(i,i)$, then $d_kp$ assumes each value in $\calP((r_{ii}\textbf{z})*(1))$ exactly once.
\end{prop*}
\begin{prop*}
Choose $\textbf{z}\in\Z^n$, and $i\neq j$ satisfying $1\leq i,j\leq n$. Then as $k$ ranges from $1$ to $\sigma \textbf{z}-1$, and $\textbf{p}$ ranges through those elements of $\calP(\textbf{z})$ satisfying $(p_k,p_{k+1})=(i,j)$, then $d_k\textbf{p}$ assumes each value in $\calP((r_{ij}\textbf{z})*(1))$ exactly once. This occurs symmetrically in $i$ and $j$, which is to say that for such a pair $(k,\textbf{p})$, the pair $(k,\textbf{p}\sigma_k)$ works with $i$ and $j$ swapped, and we have $d_k(\textbf{p}\sigma_k)=d_k(\textbf{p})$.
\end{prop*}

\subsection*{The Koszul complex}
The PBW theorem states that $E^0VL=A(L^u)\otimes E(L^r)$, and the homology of this algebra (see Priddy) is thus $E(L^u)\otimes \Gamma(L^r)$. (I think) we'll think of this as the sub-coalgebra of $\Gamma(L)$ in which there are no $\gamma_i$ for $i>1$ applied to unrestrictable elements. It's the shuffle product that produces representatives in $\overline{B}(E^0V(L))$ --- we may send a product $\prod_{i=1}^{n}\gamma_{z_i}(x_i)$ to $\left\langle \textbf{z},\textbf{x}\right\rangle$, a quantity we now define:

\begin{itemize}
\setlength{\parindent}{.25in}
\item $(hL)^n:=\{\textbf{x}=(x_1,\ldots,x_n);\ x_i\textup{ homogeneous in }V(L)\}$
%\item $d_{ij}:(hL)^n\to(hL)^{n+1}$ given by $\textbf{x}\mapsto\textbf{x}*(x_ix_j)$
%\item $b_{ij}:(hL)^n\to(hL)^{n+1}$ given by $\textbf{x}\mapsto\textbf{x}*([x_i,x_j])$
\item Choose $\textbf{x}\in(hL)^n$, $\textbf{z}\in\Z^n$ and $\textbf{p}\in\calP(\textbf{z}).$ Then define $\left\langle \textbf{p},\textbf{x}\right\rangle\in\overline{B}_{\sigma\textbf{z},*}(E^0V(L))$ to be the bar $[x_{p_1},\ldots,x_{p_{\sigma\textbf{z}}}]$. Then define $\left\langle \textbf{z},\textbf{x}\right\rangle:=\sum_{\textbf{p}\in\calP(\textbf{z})}\left\langle \textbf{p},\textbf{x}\right\rangle$. 
\end{itemize}
Now $\overline{K}_*(V(L)):=H_{*=*}(E^0V(L))$ is spanned by the bars $\left\langle \textbf{z},\textbf{x}\right\rangle$ where each $x_i$ is a homogeneous element of $L$, which must be restrictable if $z_i>1$ (here, $\sigma\textbf{z}$ is the homological degree). In fact, there's a basis consisting of the $\left\langle \textbf{z},\textbf{x}\right\rangle$ where $\textbf{x}$ runs over (stictly) increasing sequences of elements of an ordered basis chosen for $L$, and the same restrictability condition holds.

Finally we may calculate the differential on $\overline{K}_*(V(L))$. Here goes:
\begin{alignat*}{2}
d\left\langle \textbf{z},\textbf{x}\right\rangle
&=
\sum_{k=1}^{\sigma\textbf{z}-1}\sum_{\textbf{p}\in\calP(\textbf{z})}d_k\left\langle \textbf{p},\textbf{x}\right\rangle%
\\
%&=
%\sum_{k}\sum_{\textbf{p}}\left\langle d_k\textbf{p},d_{p_kp_{k+1}}\textbf{x}\right\rangle%
%\\
&=
\sum_{i=1}^{n}\sum_{j=1}^{n}\sum_{k}\sum_{\textbf{p}:{p_k=i\atop p_{k+1}=j}}\left\langle d_k\textbf{p},\textbf{x}*(x_{p_k}x_{p_{k+1}})\right\rangle%
\\
&=
\sum_{i}\sum_{k}\sum_{\textbf{p}:{p_k=i\atop p_{k+1}=i}}\left\langle d_k\textbf{p},\textbf{x}*(x_ix_i)\right\rangle+\sum_{i<j}\sum_{k}\sum_{\textbf{p}:{p_k=i\atop p_{k+1}=j}}\left\langle d_k\textbf{p},\textbf{x}*(x_ix_j+x_jx_i)\right\rangle%
\\
&=
\sum_{i}\sum_{k}\sum_{\textbf{p}:{p_k=i\atop p_{k+1}=i}}\left\langle d_k\textbf{p},\textbf{x}*\left(x_i^{[2]}\right)\right\rangle+\sum_{i<j}\sum_{k}\sum_{\textbf{p}:{p_k=i\atop p_{k+1}=j}}\left\langle d_k\textbf{p},\textbf{x}*\left([x_i,x_j]\right)\right\rangle%
\\
&=
\sum_{i}\sum_{\textbf{q}\in\calP((r_{ii}\textbf{z})*(1))}\left\langle \textbf{q},\textbf{x}*\left(x_i^{[2]}\right)\right\rangle+\sum_{i<j} \sum_{\textbf{q}\in\calP((r_{ij}\textbf{z})*(1))}\left\langle \textbf{q},\textbf{x}*\left([x_i,x_j]\right)\right\rangle%
\\
% Left hand side
% Relation
&=
% Right hand side
\sum_{i}\left\langle (r_{ii}\textbf{z})*(1),\textbf{x}*\left(x_i^{[2]}\right)\right\rangle+\sum_{i<j} \left\langle (r_{ij}\textbf{z})*(1),\textbf{x}*\left([x_i,x_j]\right)\right\rangle%
% Comment
\end{alignat*}
This is the obvious restriction of May's $\overline{X}$, see page 141 of ``The cohomology of restricted Lie algebras and of Hopf algebras'' (J.\ Alg.\ 3). We use the subspace with fewer $\gamma_i$, and the differential he gives makes sense on and preserves this subspace, and coincides with that here.

He writes the differential on $f=\prod_{i=1}^{n}\gamma_{z_i}(y_i)$ as
\[d(f)=\left(\sum_{i=1}^{n}f_iy_i\right)+ \sum_{i=1}^{n}f_{i,i}\gamma_1(\xi(y_i))+\sum_{1\leq i<j\leq n}f_{i,j}\gamma_1([y_i,y_j]).\]
Here, the parenthetical term vanishes in $\overline{K}$ --- it's there in case we're using nontrivial coefficients.
\end{document}

















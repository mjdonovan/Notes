% !TEX root = z_output/_PRLieAlgs.tex
\documentclass[10pt]{article}
%%%USINGsubsectionNUMBERS
\newcommand{\NOFULLPAGE}{\relax}
\usepackage{fullpage}
\usepackage{amsmath,amsthm,amssymb}
\usepackage{mathrsfs,nicefrac}
\usepackage{amssymb}
\usepackage{epsfig}
\usepackage[all,2cell]{xy}
\usepackage{sseq}
\usepackage{tocloft}
\usepackage{cancel}
\usepackage[strict]{changepage}
\usepackage{color}
\usepackage{tikz}
\usepackage{extpfeil}
\usepackage{version}
\usepackage{framed}
\definecolor{shadecolor}{rgb}{.925,0.925,0.925}

%\usepackage{ifthen}
%Used for disabling hyperref
\ifx\dontloadhyperref\undefined
%\usepackage[pdftex,pdfborder={0 0 0 [1 1]}]{hyperref}
\usepackage[pdftex,pdfborder={0 0 .5 [1 1]}]{hyperref}
\else
\providecommand{\texorpdfstring}[2]{#1}
\fi
%>>>>>>>>>>>>>>>>>>>>>>>>>>>>>>
%<<<        Versions        <<<
%>>>>>>>>>>>>>>>>>>>>>>>>>>>>>>
%Add in the following line to include all the versions.
%\def\excludeversion#1{\includeversion{#1}}

%>>>>>>>>>>>>>>>>>>>>>>>>>>>>>>
%<<<       Better ToC       <<<
%>>>>>>>>>>>>>>>>>>>>>>>>>>>>>>
\setlength{\cftbeforesecskip}{0.5ex}

%>>>>>>>>>>>>>>>>>>>>>>>>>>>>>>
%<<<      Hyperref mod      <<<
%>>>>>>>>>>>>>>>>>>>>>>>>>>>>>>

%needs more testing
\newcounter{dummyforrefstepcounter}
\newcommand{\labelRIGHTHERE}[1]
{\refstepcounter{dummyforrefstepcounter}\label{#1}}


%>>>>>>>>>>>>>>>>>>>>>>>>>>>>>>
%<<<  Theorem Environments  <<<
%>>>>>>>>>>>>>>>>>>>>>>>>>>>>>>
\ifx\dontloaddefinitionsoftheoremenvironments\undefined
\theoremstyle{plain}
\newtheorem{thm}{Theorem}[section]
\newtheorem*{thm*}{Theorem}
\newtheorem{lem}[thm]{Lemma}
\newtheorem*{lem*}{Lemma}
\newtheorem{prop}[thm]{Proposition}
\newtheorem*{prop*}{Proposition}
\newtheorem{cor}[thm]{Corollary}
\newtheorem*{cor*}{Corollary}
\newtheorem{defprop}[thm]{Definition-Proposition}
\newtheorem*{punchline}{Punchline}
\newtheorem*{conjecture}{Conjecture}
\newtheorem*{claim}{Claim}

\theoremstyle{definition}
\newtheorem{defn}{Definition}[section]
\newtheorem*{defn*}{Definition}
\newtheorem{exmp}{Example}[section]
\newtheorem*{exmp*}{Example}
\newtheorem*{exmps*}{Examples}
\newtheorem*{nonexmp*}{Non-example}
\newtheorem{asspt}{Assumption}[section]
\newtheorem{notation}{Notation}[section]
\newtheorem{exercise}{Exercise}[section]
\newtheorem*{fact*}{Fact}
\newtheorem*{rmk*}{Remark}
\newtheorem{fact}{Fact}
\newtheorem*{aside}{Aside}
\newtheorem*{question}{Question}
\newtheorem*{answer}{Answer}

\else\relax\fi

%>>>>>>>>>>>>>>>>>>>>>>>>>>>>>>
%<<<      Fields, etc.      <<<
%>>>>>>>>>>>>>>>>>>>>>>>>>>>>>>
\DeclareSymbolFont{AMSb}{U}{msb}{m}{n}
\DeclareMathSymbol{\N}{\mathbin}{AMSb}{"4E}
\DeclareMathSymbol{\Octonions}{\mathbin}{AMSb}{"4F}
\DeclareMathSymbol{\Z}{\mathbin}{AMSb}{"5A}
\DeclareMathSymbol{\R}{\mathbin}{AMSb}{"52}
\DeclareMathSymbol{\Q}{\mathbin}{AMSb}{"51}
\DeclareMathSymbol{\PP}{\mathbin}{AMSb}{"50}
\DeclareMathSymbol{\I}{\mathbin}{AMSb}{"49}
\DeclareMathSymbol{\C}{\mathbin}{AMSb}{"43}
\DeclareMathSymbol{\A}{\mathbin}{AMSb}{"41}
\DeclareMathSymbol{\F}{\mathbin}{AMSb}{"46}
\DeclareMathSymbol{\G}{\mathbin}{AMSb}{"47}
\DeclareMathSymbol{\Quaternions}{\mathbin}{AMSb}{"48}


%>>>>>>>>>>>>>>>>>>>>>>>>>>>>>>
%<<<       Operators        <<<
%>>>>>>>>>>>>>>>>>>>>>>>>>>>>>>
\DeclareMathOperator{\ad}{\textbf{ad}}
\DeclareMathOperator{\coker}{coker}
\renewcommand{\ker}{\textup{ker}\,}
\DeclareMathOperator{\End}{End}
\DeclareMathOperator{\Aut}{Aut}
\DeclareMathOperator{\Hom}{Hom}
\DeclareMathOperator{\Maps}{Maps}
\DeclareMathOperator{\Mor}{Mor}
\DeclareMathOperator{\Gal}{Gal}
\DeclareMathOperator{\Ext}{Ext}
\DeclareMathOperator{\Tor}{Tor}
\DeclareMathOperator{\Map}{Map}
\DeclareMathOperator{\Der}{Der}
\DeclareMathOperator{\Rad}{Rad}
\DeclareMathOperator{\rank}{rank}
\DeclareMathOperator{\ArfInvariant}{Arf}
\DeclareMathOperator{\KervaireInvariant}{Ker}
\DeclareMathOperator{\im}{im}
\DeclareMathOperator{\coim}{coim}
\DeclareMathOperator{\trace}{tr}
\DeclareMathOperator{\supp}{supp}
\DeclareMathOperator{\ann}{ann}
\DeclareMathOperator{\spec}{Spec}
\DeclareMathOperator{\SPEC}{\textbf{Spec}}
\DeclareMathOperator{\proj}{Proj}
\DeclareMathOperator{\PROJ}{\textbf{Proj}}
\DeclareMathOperator{\fiber}{F}
\DeclareMathOperator{\cofiber}{C}
\DeclareMathOperator{\cone}{cone}
\DeclareMathOperator{\skel}{sk}
\DeclareMathOperator{\coskel}{cosk}
\DeclareMathOperator{\conn}{conn}
\DeclareMathOperator{\colim}{colim}
\DeclareMathOperator{\limit}{lim}
\DeclareMathOperator{\ch}{ch}
\DeclareMathOperator{\Vect}{Vect}
\DeclareMathOperator{\GrthGrp}{GrthGp}
\DeclareMathOperator{\Sym}{Sym}
\DeclareMathOperator{\Prob}{\mathbb{P}}
\DeclareMathOperator{\Exp}{\mathbb{E}}
\DeclareMathOperator{\GeomMean}{\mathbb{G}}
\DeclareMathOperator{\Var}{Var}
\DeclareMathOperator{\Cov}{Cov}
\DeclareMathOperator{\Sp}{Sp}
\DeclareMathOperator{\Seq}{Seq}
\DeclareMathOperator{\Cyl}{Cyl}
\DeclareMathOperator{\Ev}{Ev}
\DeclareMathOperator{\sh}{sh}
\DeclareMathOperator{\intHom}{\underline{Hom}}
\DeclareMathOperator{\Frac}{frac}



%>>>>>>>>>>>>>>>>>>>>>>>>>>>>>>
%<<<   Cohomology Theories  <<<
%>>>>>>>>>>>>>>>>>>>>>>>>>>>>>>
\DeclareMathOperator{\KR}{{K\R}}
\DeclareMathOperator{\KO}{{KO}}
\DeclareMathOperator{\K}{{K}}
\DeclareMathOperator{\OmegaO}{{\Omega_{\Octonions}}}

%>>>>>>>>>>>>>>>>>>>>>>>>>>>>>>
%<<<   Algebraic Geometry   <<<
%>>>>>>>>>>>>>>>>>>>>>>>>>>>>>>
\DeclareMathOperator{\Spec}{Spec}
\DeclareMathOperator{\Proj}{Proj}
\DeclareMathOperator{\Sing}{Sing}
\DeclareMathOperator{\shfHom}{\mathscr{H}\textit{\!\!om}}
\DeclareMathOperator{\WeilDivisors}{{Div}}
\DeclareMathOperator{\CartierDivisors}{{CaDiv}}
\DeclareMathOperator{\PrincipalWeilDivisors}{{PrDiv}}
\DeclareMathOperator{\LocallyPrincipalWeilDivisors}{{LPDiv}}
\DeclareMathOperator{\PrincipalCartierDivisors}{{PrCaDiv}}
\DeclareMathOperator{\DivisorClass}{{Cl}}
\DeclareMathOperator{\CartierClass}{{CaCl}}
\DeclareMathOperator{\Picard}{{Pic}}
\DeclareMathOperator{\Frob}{Frob}


%>>>>>>>>>>>>>>>>>>>>>>>>>>>>>>
%<<<  Mathematical Objects  <<<
%>>>>>>>>>>>>>>>>>>>>>>>>>>>>>>
\newcommand{\sll}{\mathfrak{sl}}
\newcommand{\gl}{\mathfrak{gl}}
\newcommand{\GL}{\mbox{GL}}
\newcommand{\PGL}{\mbox{PGL}}
\newcommand{\SL}{\mbox{SL}}
\newcommand{\Mat}{\mbox{Mat}}
\newcommand{\Gr}{\textup{Gr}}
\newcommand{\Squ}{\textup{Sq}}
\newcommand{\catSet}{\textit{Sets}}
\newcommand{\RP}{{\R\PP}}
\newcommand{\CP}{{\C\PP}}
\newcommand{\Steen}{\mathscr{A}}
\newcommand{\Orth}{\textup{\textbf{O}}}

%>>>>>>>>>>>>>>>>>>>>>>>>>>>>>>
%<<<  Mathematical Symbols  <<<
%>>>>>>>>>>>>>>>>>>>>>>>>>>>>>>
\newcommand{\DASH}{\textup{---}}
\newcommand{\op}{\textup{op}}
\newcommand{\CW}{\textup{CW}}
\newcommand{\ob}{\textup{ob}\,}
\newcommand{\ho}{\textup{ho}}
\newcommand{\st}{\textup{st}}
\newcommand{\id}{\textup{id}}
\newcommand{\Bullet}{\ensuremath{\bullet} }
\newcommand{\sprod}{\wedge}

%>>>>>>>>>>>>>>>>>>>>>>>>>>>>>>
%<<<      Some Arrows       <<<
%>>>>>>>>>>>>>>>>>>>>>>>>>>>>>>
\newcommand{\nt}{\Longrightarrow}
\let\shortmapsto\mapsto
\let\mapsto\longmapsto
\newcommand{\mapsfrom}{\,\reflectbox{$\mapsto$}\ }
\newcommand{\bigrightsquig}{\scalebox{2}{\ensuremath{\rightsquigarrow}}}
\newcommand{\bigleftsquig}{\reflectbox{\scalebox{2}{\ensuremath{\rightsquigarrow}}}}

%\newcommand{\cofibration}{\xhookrightarrow{\phantom{\ \,{\sim\!}\ \ }}}
%\newcommand{\fibration}{\xtwoheadrightarrow{\phantom{\sim\!}}}
%\newcommand{\acycliccofibration}{\xhookrightarrow{\ \,{\sim\!}\ \ }}
%\newcommand{\acyclicfibration}{\xtwoheadrightarrow{\sim\!}}
%\newcommand{\leftcofibration}{\xhookleftarrow{\phantom{\ \,{\sim\!}\ \ }}}
%\newcommand{\leftfibration}{\xtwoheadleftarrow{\phantom{\sim\!}}}
%\newcommand{\leftacycliccofibration}{\xhookleftarrow{\ \ {\sim\!}\,\ }}
%\newcommand{\leftacyclicfibration}{\xtwoheadleftarrow{\sim\!}}
%\newcommand{\weakequiv}{\xrightarrow{\ \,\sim\,\ }}
%\newcommand{\leftweakequiv}{\xleftarrow{\ \,\sim\,\ }}

\newcommand{\cofibration}
{\xhookrightarrow{\phantom{\ \,{\raisebox{-.3ex}[0ex][0ex]{\scriptsize$\sim$}\!}\ \ }}}
\newcommand{\fibration}
{\xtwoheadrightarrow{\phantom{\raisebox{-.3ex}[0ex][0ex]{\scriptsize$\sim$}\!}}}
\newcommand{\acycliccofibration}
{\xhookrightarrow{\ \,{\raisebox{-.55ex}[0ex][0ex]{\scriptsize$\sim$}\!}\ \ }}
\newcommand{\acyclicfibration}
{\xtwoheadrightarrow{\raisebox{-.6ex}[0ex][0ex]{\scriptsize$\sim$}\!}}
\newcommand{\leftcofibration}
{\xhookleftarrow{\phantom{\ \,{\raisebox{-.3ex}[0ex][0ex]{\scriptsize$\sim$}\!}\ \ }}}
\newcommand{\leftfibration}
{\xtwoheadleftarrow{\phantom{\raisebox{-.3ex}[0ex][0ex]{\scriptsize$\sim$}\!}}}
\newcommand{\leftacycliccofibration}
{\xhookleftarrow{\ \ {\raisebox{-.55ex}[0ex][0ex]{\scriptsize$\sim$}\!}\,\ }}
\newcommand{\leftacyclicfibration}
{\xtwoheadleftarrow{\raisebox{-.6ex}[0ex][0ex]{\scriptsize$\sim$}\!}}
\newcommand{\weakequiv}
{\xrightarrow{\ \,\raisebox{-.3ex}[0ex][0ex]{\scriptsize$\sim$}\,\ }}
\newcommand{\leftweakequiv}
{\xleftarrow{\ \,\raisebox{-.3ex}[0ex][0ex]{\scriptsize$\sim$}\,\ }}

%>>>>>>>>>>>>>>>>>>>>>>>>>>>>>>
%<<<    xymatrix Arrows     <<<
%>>>>>>>>>>>>>>>>>>>>>>>>>>>>>>
\newdir{ >}{{}*!/-5pt/@{>}}
\newcommand{\xycof}{\ar@{ >->}}
\newcommand{\xycofib}{\ar@{^{(}->}}
\newcommand{\xycofibdown}{\ar@{_{(}->}}
\newcommand{\xyfib}{\ar@{->>}}
\newcommand{\xymapsto}{\ar@{|->}}

%>>>>>>>>>>>>>>>>>>>>>>>>>>>>>>
%<<<     Greek Letters      <<<
%>>>>>>>>>>>>>>>>>>>>>>>>>>>>>>
%\newcommand{\oldphi}{\phi}
%\renewcommand{\phi}{\varphi}
\let\oldphi\phi
\let\phi\varphi
\renewcommand{\to}{\longrightarrow}
\newcommand{\from}{\longleftarrow}
\newcommand{\eps}{\varepsilon}

%>>>>>>>>>>>>>>>>>>>>>>>>>>>>>>
%<<<  1st-4th & parentheses <<<
%>>>>>>>>>>>>>>>>>>>>>>>>>>>>>>
\newcommand{\first}{^\text{st}}
\newcommand{\second}{^\text{nd}}
\newcommand{\third}{^\text{rd}}
\newcommand{\fourth}{^\text{th}}
\newcommand{\ZEROTH}{$0^\text{th}$ }
\newcommand{\FIRST}{$1^\text{st}$ }
\newcommand{\SECOND}{$2^\text{nd}$ }
\newcommand{\THIRD}{$3^\text{rd}$ }
\newcommand{\FOURTH}{$4^\text{th}$ }
\newcommand{\iTH}{$i^\text{th}$ }
\newcommand{\jTH}{$j^\text{th}$ }
\newcommand{\nTH}{$n^\text{th}$ }

%>>>>>>>>>>>>>>>>>>>>>>>>>>>>>>
%<<<    upright commands    <<<
%>>>>>>>>>>>>>>>>>>>>>>>>>>>>>>
\newcommand{\upcol}{\textup{:}}
\newcommand{\upsemi}{\textup{;}}
\providecommand{\lparen}{\textup{(}}
\providecommand{\rparen}{\textup{)}}
\renewcommand{\lparen}{\textup{(}}
\renewcommand{\rparen}{\textup{)}}
\newcommand{\Iff}{\emph{iff} }

%>>>>>>>>>>>>>>>>>>>>>>>>>>>>>>
%<<<     Environments       <<<
%>>>>>>>>>>>>>>>>>>>>>>>>>>>>>>
\newcommand{\squishlist}
{ %\setlength{\topsep}{100pt} doesn't seem to do anything.
  \setlength{\itemsep}{.5pt}
  \setlength{\parskip}{0pt}
  \setlength{\parsep}{0pt}}
\newenvironment{itemise}{
\begin{list}{\textup{$\rightsquigarrow$}}
   {  \setlength{\topsep}{1mm}
      \setlength{\itemsep}{1pt}
      \setlength{\parskip}{0pt}
      \setlength{\parsep}{0pt}
   }
}{\end{list}\vspace{-.1cm}}
\newcommand{\INDENT}{\textbf{}\phantom{space}}
\renewcommand{\INDENT}{\rule{.7cm}{0cm}}

\newcommand{\itm}[1][$\rightsquigarrow$]{\item[{\makebox[.5cm][c]{\textup{#1}}}]}


%\newcommand{\rednote}[1]{{\color{red}#1}\makebox[0cm][l]{\scalebox{.1}{rednote}}}
%\newcommand{\bluenote}[1]{{\color{blue}#1}\makebox[0cm][l]{\scalebox{.1}{rednote}}}

\newcommand{\rednote}[1]
{{\color{red}#1}\makebox[0cm][l]{\scalebox{.1}{\rotatebox{90}{?????}}}}
\newcommand{\bluenote}[1]
{{\color{blue}#1}\makebox[0cm][l]{\scalebox{.1}{\rotatebox{90}{?????}}}}


\newcommand{\funcdef}[4]{\begin{align*}
#1&\to #2\\
#3&\mapsto#4
\end{align*}}

%\newcommand{\comment}[1]{}

%>>>>>>>>>>>>>>>>>>>>>>>>>>>>>>
%<<<       Categories       <<<
%>>>>>>>>>>>>>>>>>>>>>>>>>>>>>>
\newcommand{\Ens}{{\mathscr{E}ns}}
\DeclareMathOperator{\Sheaves}{{\mathsf{Shf}}}
\DeclareMathOperator{\Presheaves}{{\mathsf{PreShf}}}
\DeclareMathOperator{\Psh}{{\mathsf{Psh}}}
\DeclareMathOperator{\Shf}{{\mathsf{Shf}}}
\DeclareMathOperator{\Varieties}{{\mathsf{Var}}}
\DeclareMathOperator{\Schemes}{{\mathsf{Sch}}}
\DeclareMathOperator{\Rings}{{\mathsf{Rings}}}
\DeclareMathOperator{\AbGp}{{\mathsf{AbGp}}}
\DeclareMathOperator{\Modules}{{\mathsf{\!-Mod}}}
\DeclareMathOperator{\fgModules}{{\mathsf{\!-Mod}^{\textup{fg}}}}
\DeclareMathOperator{\QuasiCoherent}{{\mathsf{QCoh}}}
\DeclareMathOperator{\Coherent}{{\mathsf{Coh}}}
\DeclareMathOperator{\GSW}{{\mathcal{SW}^G}}
\DeclareMathOperator{\Burnside}{{\mathsf{Burn}}}
\DeclareMathOperator{\GSet}{{G\mathsf{Set}}}
\DeclareMathOperator{\FinGSet}{{G\mathsf{Set}^\textup{fin}}}
\DeclareMathOperator{\HSet}{{H\mathsf{Set}}}
\DeclareMathOperator{\Cat}{{\mathsf{Cat}}}
\DeclareMathOperator{\Fun}{{\mathsf{Fun}}}
\DeclareMathOperator{\Orb}{{\mathsf{Orb}}}
\DeclareMathOperator{\Set}{{\mathsf{Set}}}
\DeclareMathOperator{\sSet}{{\mathsf{sSet}}}
\DeclareMathOperator{\Top}{{\mathsf{Top}}}
\DeclareMathOperator{\GSpectra}{{G-\mathsf{Spectra}}}
\DeclareMathOperator{\Lan}{Lan}
\DeclareMathOperator{\Ran}{Ran}

%>>>>>>>>>>>>>>>>>>>>>>>>>>>>>>
%<<<     Script Letters     <<<
%>>>>>>>>>>>>>>>>>>>>>>>>>>>>>>
\newcommand{\scrQ}{\mathscr{Q}}
\newcommand{\scrW}{\mathscr{W}}
\newcommand{\scrE}{\mathscr{E}}
\newcommand{\scrR}{\mathscr{R}}
\newcommand{\scrT}{\mathscr{T}}
\newcommand{\scrY}{\mathscr{Y}}
\newcommand{\scrU}{\mathscr{U}}
\newcommand{\scrI}{\mathscr{I}}
\newcommand{\scrO}{\mathscr{O}}
\newcommand{\scrP}{\mathscr{P}}
\newcommand{\scrA}{\mathscr{A}}
\newcommand{\scrS}{\mathscr{S}}
\newcommand{\scrD}{\mathscr{D}}
\newcommand{\scrF}{\mathscr{F}}
\newcommand{\scrG}{\mathscr{G}}
\newcommand{\scrH}{\mathscr{H}}
\newcommand{\scrJ}{\mathscr{J}}
\newcommand{\scrK}{\mathscr{K}}
\newcommand{\scrL}{\mathscr{L}}
\newcommand{\scrZ}{\mathscr{Z}}
\newcommand{\scrX}{\mathscr{X}}
\newcommand{\scrC}{\mathscr{C}}
\newcommand{\scrV}{\mathscr{V}}
\newcommand{\scrB}{\mathscr{B}}
\newcommand{\scrN}{\mathscr{N}}
\newcommand{\scrM}{\mathscr{M}}

%>>>>>>>>>>>>>>>>>>>>>>>>>>>>>>
%<<<     Fractur Letters    <<<
%>>>>>>>>>>>>>>>>>>>>>>>>>>>>>>
\newcommand{\frakQ}{\mathfrak{Q}}
\newcommand{\frakW}{\mathfrak{W}}
\newcommand{\frakE}{\mathfrak{E}}
\newcommand{\frakR}{\mathfrak{R}}
\newcommand{\frakT}{\mathfrak{T}}
\newcommand{\frakY}{\mathfrak{Y}}
\newcommand{\frakU}{\mathfrak{U}}
\newcommand{\frakI}{\mathfrak{I}}
\newcommand{\frakO}{\mathfrak{O}}
\newcommand{\frakP}{\mathfrak{P}}
\newcommand{\frakA}{\mathfrak{A}}
\newcommand{\frakS}{\mathfrak{S}}
\newcommand{\frakD}{\mathfrak{D}}
\newcommand{\frakF}{\mathfrak{F}}
\newcommand{\frakG}{\mathfrak{G}}
\newcommand{\frakH}{\mathfrak{H}}
\newcommand{\frakJ}{\mathfrak{J}}
\newcommand{\frakK}{\mathfrak{K}}
\newcommand{\frakL}{\mathfrak{L}}
\newcommand{\frakZ}{\mathfrak{Z}}
\newcommand{\frakX}{\mathfrak{X}}
\newcommand{\frakC}{\mathfrak{C}}
\newcommand{\frakV}{\mathfrak{V}}
\newcommand{\frakB}{\mathfrak{B}}
\newcommand{\frakN}{\mathfrak{N}}
\newcommand{\frakM}{\mathfrak{M}}

\newcommand{\frakq}{\mathfrak{q}}
\newcommand{\frakw}{\mathfrak{w}}
\newcommand{\frake}{\mathfrak{e}}
\newcommand{\frakr}{\mathfrak{r}}
\newcommand{\frakt}{\mathfrak{t}}
\newcommand{\fraky}{\mathfrak{y}}
\newcommand{\fraku}{\mathfrak{u}}
\newcommand{\fraki}{\mathfrak{i}}
\newcommand{\frako}{\mathfrak{o}}
\newcommand{\frakp}{\mathfrak{p}}
\newcommand{\fraka}{\mathfrak{a}}
\newcommand{\fraks}{\mathfrak{s}}
\newcommand{\frakd}{\mathfrak{d}}
\newcommand{\frakf}{\mathfrak{f}}
\newcommand{\frakg}{\mathfrak{g}}
\newcommand{\frakh}{\mathfrak{h}}
\newcommand{\frakj}{\mathfrak{j}}
\newcommand{\frakk}{\mathfrak{k}}
\newcommand{\frakl}{\mathfrak{l}}
\newcommand{\frakz}{\mathfrak{z}}
\newcommand{\frakx}{\mathfrak{x}}
\newcommand{\frakc}{\mathfrak{c}}
\newcommand{\frakv}{\mathfrak{v}}
\newcommand{\frakb}{\mathfrak{b}}
\newcommand{\frakn}{\mathfrak{n}}
\newcommand{\frakm}{\mathfrak{m}}

%>>>>>>>>>>>>>>>>>>>>>>>>>>>>>>
%<<<  Caligraphic Letters   <<<
%>>>>>>>>>>>>>>>>>>>>>>>>>>>>>>
\newcommand{\calQ}{\mathcal{Q}}
\newcommand{\calW}{\mathcal{W}}
\newcommand{\calE}{\mathcal{E}}
\newcommand{\calR}{\mathcal{R}}
\newcommand{\calT}{\mathcal{T}}
\newcommand{\calY}{\mathcal{Y}}
\newcommand{\calU}{\mathcal{U}}
\newcommand{\calI}{\mathcal{I}}
\newcommand{\calO}{\mathcal{O}}
\newcommand{\calP}{\mathcal{P}}
\newcommand{\calA}{\mathcal{A}}
\newcommand{\calS}{\mathcal{S}}
\newcommand{\calD}{\mathcal{D}}
\newcommand{\calF}{\mathcal{F}}
\newcommand{\calG}{\mathcal{G}}
\newcommand{\calH}{\mathcal{H}}
\newcommand{\calJ}{\mathcal{J}}
\newcommand{\calK}{\mathcal{K}}
\newcommand{\calL}{\mathcal{L}}
\newcommand{\calZ}{\mathcal{Z}}
\newcommand{\calX}{\mathcal{X}}
\newcommand{\calC}{\mathcal{C}}
\newcommand{\calV}{\mathcal{V}}
\newcommand{\calB}{\mathcal{B}}
\newcommand{\calN}{\mathcal{N}}
\newcommand{\calM}{\mathcal{M}}

%>>>>>>>>>>>>>>>>>>>>>>>>>>>>>>
%<<<<<<<<<DEPRECIATED<<<<<<<<<<
%>>>>>>>>>>>>>>>>>>>>>>>>>>>>>>

%%% From Kac's template
% 1-inch margins, from fullpage.sty by H.Partl, Version 2, Dec. 15, 1988.
%\topmargin 0pt
%\advance \topmargin by -\headheight
%\advance \topmargin by -\headsep
%\textheight 9.1in
%\oddsidemargin 0pt
%\evensidemargin \oddsidemargin
%\marginparwidth 0.5in
%\textwidth 6.5in
%
%\parindent 0in
%\parskip 1.5ex
%%\renewcommand{\baselinestretch}{1.25}

%%% From the net
%\newcommand{\pullbackcorner}[1][dr]{\save*!/#1+1.2pc/#1:(1,-1)@^{|-}\restore}
%\newcommand{\pushoutcorner}[1][dr]{\save*!/#1-1.2pc/#1:(-1,1)@^{|-}\restore}









\DeclareFieldFormat{postnote}{#1}
\DeclareFieldFormat{multipostnote}{#1}
\excludeversion{Omitted}

\excludeversion{wedding}
\excludeversion{ConversationWithHaynes_InclusionOfBarConstructions}
\excludeversion{GrothendieckSpectralSequencestake2}
\excludeversion{convergenceOLD}
\excludeversion{hiltonMilnorQuestionable}
\excludeversion{Thoughts on Adams Multiplicativity}
\excludeversion{Thoughts on Adams Multiplicativity II}
\excludeversion{prereqs for Thoughts III}
\excludeversion{Thoughts on Adams Multiplicativity III}
%\def\excludeversion#1{\includeversion{#1}}
\excludeversion{Frontmatters}
\excludeversion{Endmatter}
\excludeversion{AdamsDerivation}
\excludeversion{TotalisationInSAlg}
\excludeversion{convergence}
\excludeversion{iteratedBarConstructions}
\excludeversion{SteenrodAlgebrasAndTheirKoszulDuals}
\excludeversion{CategoriesOfInterest}
\excludeversion{HomotopicalAlgebra}
\excludeversion{DiagramOfFunctors}
\excludeversion{GrothendieckSpectralSequences}
\excludeversion{Grothendieck Multiplicativity}
\excludeversion{Adams Muliplicativity}
\includeversion{letter to Dwyer}
\excludeversion{Adams sseq operations}
\excludeversion{KoszulComplexes_n>1}
\excludeversion{KoszulComplexes1}
\excludeversion{DerivedFunctorsLowDimension}
\excludeversion{PRlieKoszulComplexCalculationOriginalWithSSeq}
\excludeversion{Corestricted Lie coalgebras Executive Summary}
\excludeversion{PRlieKoszulComplexCalculation}
\excludeversion{LieLambdaStructureOnKoszul}
\excludeversion{SequenceOfSequencesIntro}
\excludeversion{KoszulSequenceCombinatorics}
\excludeversion{CalculatingRepeatedKoszul}
\excludeversion{DimZeroPart}
\excludeversion{AxisComputationSummary}
\excludeversion{VanishingLines}
\excludeversion{conjectured differentials}
\excludeversion{grothendieck collapse}

\newcommand{\GS}[1]{\scrE^{#1}}
\newcommand{\GpS}[1]{\scrE^{#1}_\star}
\renewcommand{\Set}{Set}

%\newcommand{\RestLie}[1]{\mathsf{r}{\scrL}^{#1}}%{{\mu\scrL}^{#1}}
%\newcommand{\GoodLie}[1]{\mathsf{g}{\scrL}^{#1}}%{{\mu\scrL}^{#1}}
%\newcommand{\BadLie}[1]{\mathsf{b}{\scrL}^{#1}}%{{\mu\scrL}^{#1}}
%\newcommand{\PRLie}[1]{\scrR^{#1}}%{{\xi\scrL}^{#1}}



\newcommand{\RestLie}[1]%
{\ifblank{#1}{\mathsf{r}{\scrL}}{\mathsf{r}{\scrL}^{#1}}}
\newcommand{\GoodLie}[1]%
{\ifblank{#1}{\mathsf{g}{\scrL}}{\mathsf{g}{\scrL}^{#1}}}
\newcommand{\BadLie}[1]%
{\ifblank{#1}{\mathsf{b}{\scrL}}{\mathsf{b}{\scrL}^{#1}}}
\newcommand{\PRLie}[1]%
{\ifblank{#1}{\scrR}{\scrR^{#1}}}


\newcommand{\LL}[1]{\ifblank{#1}{\scrK}{\scrK^{#1}}}
\newcommand{\GR}[1]{\ifblank{#1}{\scrV}{\scrV^{#1}}}
\newcommand{\nontop}[1]{\ifblank{#1}{\scrU}{\scrU^{#1}}}
\newcommand{\PiAlg}[1]{#1\textup{-$\Pi$-alg}}
\newcommand{\PiAlgebra}[1]{#1\textup{-$\Pi$-algebra}}
\newcommand{\produces}[3]{{#1}{#3}{#2}}

\newcommand{\admis}[1]{\mathrm{adm}(#1)}%{{\mu\scrL}^{#1}}

\newcommand{\iteratedrestn}[2]{\xi^{#2}{#1}}


\newcommand{\Boverline}{\smash{\overline{B}}\rule{0mm}{\heightof{\ensuremath{B}}}}
\newcommand{\Koverline}{\smash{\overline{K}}\rule{0mm}{\heightof{\ensuremath{K}}}}
\newcommand{\jmathbar}{\bar{\jmath}}


\newcommand{\Ind}[2][]{\ifblank{#1}{\mathbf{I}^{\smash{\mbox{\tiny $#2$}}}}{\mathbf{I}^{\mbox{\tiny $#2$}}_{#1}}}%[internal indices]{category name}
\newcommand{\forgetSymbol}{\mathrm{fg}}
\newcommand{\forget}[1]{\mathrm{fg}_{\smash{\mbox{\tiny $#1$}}}}
\newcommand{\BarConst}[1]{B^{\smash{\mbox{\tiny $#1$}}}}


\newcommand{\Fr}[2][]{\ifblank{#1}{#2}{#2_{#1}}}
\newcommand{\restn}[2][]{\ifblank{#1}{\xi{#2}}{\xi_{#1}{#2}}}%{\xi{#1}}


\newcommand{\algCat}{\calA}
\newcommand{\trip}[3]{{#1}_{\smash{#2}}^{\smash{#3}}}

\newcommand{\twist}{\sigma}

\DeclareMathOperator{\Constant}{con}

\newcommand{\derived}{\mathbb{L}}
%\newcommand{\LambdaOp}{Q}
\renewcommand{\Q}{Q}
\newcommand{\SqShift}{\Sq_{\smash{+}}}
\newcommand{\Sq}{\mathrm{Sq}}
\newcommand{\Comm}{\calC}
\newcommand{\LieSteen}{\calA(\calL)}
\newcommand{\CommSteen}{\calA(\Comm)}
\newcommand{\deltaAlgebra}{\Delta}
\newcommand{\DyerLashov}{R}

\newcommand{\dual}{\,\check{\,}}
\newcommand{\LieOperad}{\scrL}
\newcommand{\dualLieOperad}{\LieOperad\dual}
\newcommand{\CommOperad}{\scrC}
\newcommand{\dualCommOperad}{\CommOperad\dual}

\newcommand{\minDim}{m}
\newcommand{\minDimP}{\overline{m}}
\newcommand{\BarMonomial}[3]{\textup{mon}_{#1,#2,#3}}

\newcommand{\Shuffles}[2]{\textup{Sh}_{#1#2}}
\newcommand{\HalfShuffles}[2]{\textup{Sh}_{#1#2}/2}

\newcommand{\PuncCube}[1]{\calP_{0}[#1]}
\newcommand{\FullCube}[1]{\calP[#1]}
\newcommand{\TruncSimplex}[1]{{\Delta_{\leq#1}}}
\newcommand{\TruncAugSimplex}[1]{{\Delta^{\smash{+}}_{\smash{\geq#1}}}}

\newcommand{\ModDegeneracies}[1]{(#1/\textup{Degens})}

\renewcommand{\produces}[3]{
{
\def\labelstyle{\scriptstyle}
\xymatrix@C=2em@1{
{#1}
\ar@{-}[r]|-{{\,#3\,}}
&%r1c1
{#2}%r1c2
}}}





\bibliography{../../Dropbox/logbook/_LOGBOOK/papers}
\begin{document}
\begin{Frontmatters}
\tableofcontents%\newpage
\end{Frontmatters}








\begin{hiltonMilnorQuestionable}
\subsection{Hilton-Milnor theorems}
Suppose that $\algCat$ is a category of finitary universal algebras in $k$-modules, with no nullary or unary operations, for $k$ a field.
\begin{defn}
$\calA$ is \emph{graded by arity} if the natural filtration of $F(A)$ by total arity splits, naturally in the set $A$.
\end{defn}
\begin{defn}
$\algCat$ is \emph{Hilton-Milnor} if, for all finite sets $B$, there is a subset $S\subset F(B)$ with the following property. Whenever we are given a collection $\{A_\beta\}_{\beta\in B}$ of $k$-modules, the natural map
\[\bigoplus_{L\in S}L^{\otimes}(\{FA_\beta\})\to F(\oplus A_\beta)\]
is an isomorphism.
\end{defn}
\begin{prop}\label{HiltonMilnorTheoriesConverge}
If $\algCat$ is Hilton-Milnor and graded by arity, then whenever $X\in s\algCat$ is 0-connected, the power $X^s$ of $X$ is $s$-connected.
\end{prop}
\begin{exmp}
If $\algCat=\Comm$, then $S\subset F(B)$ is just $\{b_ib_j|i<j\}$. If $\algCat$ is the category of non-unital associative algebras over $k$, then $S=\{b_{i_1}\cdots b_{i_\ell}|i_j\neq i_{j+1}\textup{ for }1\leq j<\ell\}$. If $\algCat$ is the category of (good) Lie algebras, then $S$ is a basis of the free Lie algebra on $B$.
\end{exmp}
\begin{proof}[Proof of \ref{HiltonMilnorTheoriesConverge}]
Filter an almost free $X$. Then, $X^s/X^{s+1}$ is isomorphic to $F^{s/s+1}(QX)$, which has homotopy groups given by $\calF^{s/s+1}(H^Q_*X)$, where $\calF^{s/s+1}$ is the Dold functor for $F^{s/s+1}$. I guess that $H^Q_*X=0$ since $X$ can be replaced by a reduced object. Thus we only need to see that $\calF^{s/s+1}(W)$, for $W$ a connected graded $k$-module, starts at grading $s$.

By the proof Dold's theorem, it's enough to look at the homotopy of $F^{s/s+1}$ applied to wedges of positive spheres. These can be understood using the Hilton-Milnor condition, and the K\"unneth theorem. The point is that these homotopy groups are the sum, according to a Hilton-Milnor decomposition, of various $s$-ary operations on the homotopy of $F$ on a single sphere.
\end{proof}

\end{hiltonMilnorQuestionable}





\begin{GrothendieckSpectralSequencestake2}
%%%%\subsection{Grothendieck spectral sequence fiddles}
%%%%\begin{lem*}
%%%%Suppose that $X$ is a levelwise free object in $s\nontop{n}$. Let $A$ be the bisimplicial object which is the levelwise bar construction on $X$, defined by $A_{s,t}=U^{t+1}X_s$. Then the natural map $\Delta A\to X$ is a cofibrant replacement, and $\pi_*(I\Delta A)\to\pi_*(IX)$ is an isomorphism. That is, the complex $IX$ already calculates $(\derived I)X$.
%%%%\end{lem*}
%%%%\begin{proof}
%%%%The proof will proceed by factoring the map $\pi_*(I\Delta A)\to\pi_*(IX)$ as the composite of three isomorphisms:
%%%%\[\xymatrix@R=4mm{
%%%%\pi_*(I\Delta A)
%%%%\ar[r]_-{}
%%%%\ar[d]_-{\textup{AW}}
%%%%&%r1c1
%%%%\pi_*(IX)
%%%%\\%r1c2
%%%%H_*(\Tot CIA)\ar[r]^-{\textup{edge}}
%%%%&%r2c1
%%%%E^2_{*,0}\ar[u]_-{\tau}
%%%%%r2c2
%%%%}\]
%%%%The first will be homology isomorphism induced by the Alexander-Whitney map $C\Delta IA\to\Tot CIA$. The second map will be an edge homomorphism in the spectral sequence obtained by filtering $\Tot CIA$ by `$s$':
%%%%\begin{alignat*}{2}
%%%%E^0_{st}
%%%%&=
%%%%IU^{t+1}X_s%
%%%%\\
%%%%E^1_{st}
%%%%&=
%%%%(\derived_t I)(X_s)=\begin{cases}
%%%%IX_s,&\textup{if }t=0;\\
%%%%0,&\textup{otherwise}.
%%%%\end{cases}
%%%%&\qquad&\text{(as $X_s$ is free)}\\
%%%%E^2_{st}
%%%%&=
%%%%\begin{cases}
%%%%\pi_s(IX),&\textup{if }t=0;\\
%%%%0,&\textup{otherwise}.
%%%%\end{cases}
%%%%\end{alignat*
%%%%The isomorphism $\tau$ is induced, for each $s$, by the augmentation in the $t$ direction, $(I(\textup{id},d_0)):IUX_s\to IX_s$ %$I\rho:IUX_s\to IX_s$, where $\rho$ is the structure morphism $UX_s\to X_s$. 
%%%%The edge homomorphism sends the class of an $s$-cycle in $\Tot CIA$, say,
%%%%\[\sum_{i=0}^s z_{i,s-i},\textup{ where }z_{i,s-i}\in IA_{i,s-i}\]
%%%%to the class $z_{s,0}$ in $E^2_{s,0}$. Suppose then that $w\in IA_{ss}$ is a cycle in $CI\Delta A$. Then $\textup{edge}(\textup{AW}(w))$ is represented simply by $(I(\textup{id},(d_0)^s))(w)$. Thus, the effect of the composite of the three isomorphisms on a cycle is $w\mapsto (I(\textup{id},(d_0)^{s+1}))(w)$, but $(\textup{id},(d_0)^{s+1})$ is exactly the natural map $\Delta A\to X$, demonstrating that the above diagram commutes.
%%%%
%%%%\end{proof}
%%%%\begin{prop}
%%%%For $n\geq1$, and any $X\in\LL{n}$, there's an isomorphism of vector spaces
%%%%\[(\derived{\Ind{\nontop{}}})(\forget{\nontop{}}X)\cong\forget{\GR{}}(\derived{\Psi})X,\]
%%%%induced by the evident inclusion 
%%%%\[\Ind{\nontop{}}B_\bullet(\Fr{\nontop{n}},\Fr{\nontop{n}},\forget{\nontop{}}X)\to \forget{\GR{}}\Psi B_\bullet(\Fr{\LL{n}},\Fr{\LL{n}},X).\]
%%%%You might write this as $B_\bullet(\forgetSymbol,\Fr{\nontop{n}},\forget{\nontop{}}X)\to B_\bullet(\PRLie{n},\Fr{\LL{n}},X)$.
%%%%\end{prop}
%%%%\noindent In light of this result, we write $(\derived{\Ind{\nontop{}}})X$ for either $(\derived{\Ind{\nontop{}}})X$ or $(\derived{\Ind{\nontop{}}})\forgetSymbol(X)$, without ambiguity. \textbf{Discuss} the meaning of the fg and Rn in the bar construction.
%%%%\begin{proof}
%%%%As $\forgetSymbol:\LL{n}\to\nontop{n}$ preserves free objects and weak equivalences, there's a one-line Grothendieck spectral sequence producing the isomorphism, but this argument does not show that the isomorphism is induced by the inclusion of bar constructions. If $\forgetSymbol(B_\bullet(\LL{n},\LL{n},X))$ were almost free in $s\nontop{n}$, the map ${B}_\bullet(\nontop{n},\nontop{n},\forgetSymbol(X))\to{B}_\bullet(\LL{n},\LL{n},X)$ would be a map of cofibrant replacements of $\forgetSymbol(X)$ in $s\nontop{n}$, and thus 
%%%%
%%%%Or should I say: ``the obvious inclusion map $\Boverline_*(\nontop{n},\nontop{n},X)\to\Boverline_*(\LL{n},\LL{n},X)$ is a map of almost free objects realising the identity on $X$''? Is the latter actually almost free in $s\nontop{n}$?
%%%%\end{proof}
%%%%\begin{lem}
%%%%For any $X$ in $s\LL{n}$, $(\derived{\Ind{\nontop{}}})(\forget{\nontop{}}(X))$ may be calculated as the homotopy of the object 
%%%%\[\forgetSymbol(\Ind{\nontop{}}(B_\bullet(\Fr{\LL{n}},\Fr{\LL{n}},X)))=\forgetSymbol(B_\bullet(\Fr{\PRLie{n}},\Fr{\LL{n}},X))\in s\GR{n}.\]
%%%%\end{lem}
%%%%\begin{proof}
%%%%Although $B:=B_\bullet(\Fr{\LL{n}},\Fr{\LL{n}},X)$ is an acyclic levelwise free object in $s\nontop{n}$, with $\pi_0(B)=X$, it may fail to be `almost free'. The point here is that $B$ can be used to calculate the derived functors of $\Ind{\nontop{}}$ anyway. To see this, consider the bisimplicial object $A:=\Ind{\nontop{}}(B_\bullet(\Fr{\nontop{n}},\Fr{\nontop{n}},B))$ formed by applying $\Ind{\nontop{}}$ to the levelwise bar construction on $B$, so that
%%%%\[A_{s,t}=(\Ind{\nontop{}})((\Fr{\nontop{n}})^{s+1})B_t\cong (\Fr{\nontop{n}})^sB_t.\]
%%%%Now there are two first-quadrant spectral sequences converging to $\pi_*(\Delta A)$, one with ${E}^{2,I}=\pi_s\pi^e_tA$ and one with $E^{2,II}=\pi_t\pi^i_sA$. We'll show that they are both one-line spectral sequences by $E^2$, with ${E}^{2,I}_{s,0}=(\derived_s\Ind{\nontop{}})\forgetSymbol(X)$, and ${E}^{2,II}_{0,t}=\pi_t(\Ind{\nontop{}}B)$, proving the result.
%%%%
%%%%
%%%%To identify ${E}^{2,I}_{s,t}$, write $A_{s,\bullet}$ for the singly simplicial object obtained by fixing the inner coordinate $s$. Then by lemma \ref{lemIteratedFreeUobjectPreservesAcyclicity}, $A_{s,\bullet}$ is acyclic, with $\pi_0 A_{s,\bullet}=(\Fr{\nontop{n}})^s\pi_0(B_\bullet)= (\Fr{\nontop{n}})^sX$. Now $\pi_0 A_{s,\bullet}$ is simplicial in $s$, and can be identified with the bar construction computing $(\derived_s{\Ind{\nontop{}}})\forgetSymbol(X)$.
%%%%
%%%%On the other hand, for fixed $t$, $A_{\bullet,t}$ is simply the bar construction calculating $(\derived\Ind{\nontop{}})B_t$. As $B_t$ is free for each $t$, $(\derived_0\Ind{\nontop{}})B_t=\Ind{\nontop{}}(B_t)$, and the higher derived functors vanish, identifying $E^{2,II}$.
%%%%\end{proof}
%%%%\begin{lem}\label{lemIteratedFreeUobjectPreservesAcyclicity}
%%%%Suppose that $W\in s\nontop{n}$ is acyclic. Then for each $s\geq0$,  $(\Fr{\nontop{n}})^sW\in s\nontop{n}$ is acyclic, with $\pi_0((\Fr{\nontop{n}})^sW)=(\Fr{\nontop{n}})^s(\pi_0(W))$.
%%%%\end{lem}
%%%%\begin{proof}
%%%%By induction on $s$ it is enough to prove the statement when $s=1$. The calculation of $\pi_0$ is immediate from the fact that $\Fr{\nontop{n}}$, being a left adjoint, preserves colimits.
%%%%
%%%%To approach the acyclicity question, we'll use unreduced chain complexes, writing $\partial:W_r\to W_{r-1}$ for the sum of the face maps. Now suppose that $a\in(\Fr{\nontop{n}}W)_r$ is a cycle. Without loss of generality, we may write
%%%%$a:=\sum_j Q^{I_j}x_j$
%%%%where all of the $I_j$ are distinct, and the $x_j$ are (potentially inhomogeneous) elements of $W_r$ (such that $Q^{I_j}x_j$ is defined). Then $\partial (a)=\sum Q^{I_j}\partial(x_j)=0$, which implies that each $\partial(x_j)$ is zero, so that $x_j=\partial(y_j)$ for some $y_j\in W_{r+1}$. Note that the internal dimensions of the homogeneous summands of $y_j$ may be taken to be the same as those of $x_j$, in which case $\sum Q^{I_j}y_j$ is defined in $\Fr{\nontop{n}}W_{r+1}$, and has boundary $a$.
%%%%\end{proof}
\end{GrothendieckSpectralSequencestake2}























\begin{AdamsDerivation}
\section{The Adams Spectral sequence for commutative $\F_2$-algebras}
Let $\Comm$ be the category of augmented commutative $\F_2$-algebras. For $X\in \Comm$, one defines:
\[IX:=\ker(\epsilon:X\to\F_2)\subset X\textup{\quad and\quad }QX:=IX/(IX)^2.\]
Then the functor $Q:\Comm\to\Vect$ admits a right adjoint $K:\Vect\to\Comm$ defined by $V\mapsto \F_2\oplus V$, with multiplication characterised by $V^2=0$.

An analysis (see \cite[\S4]{MR1089001}) of the category of abelian objects in $s\Comm$ reveals that the abelian objects of are precisely the image of $K$, and the prolongation $Q:s\Comm\rightleftarrows s\Vect:K$ of the above adjunction models the adjunction between `abelianization' and `the inclusion of the abelian objects'. Thus we define the Quillen homology $H_*^Q(X_{\bullet})$ of $X_{\bullet}\in s\Comm$ to be $\mathbb{L}_*Q(X_{\bullet}):=\pi_*Q(cX_\bullet)$, where $c$ denotes a cofibrant replacement in $s\Comm$.





\subsection{A Bousfield-Kan construction in $s\Comm$}
In \cite{MR0365573}, Bousfield and Kan construct, for a space (i.e.\ simplicial set) $X$, the $\F_2$-completion $\hat X$ to be the totalisation of the cosimplicial resolution of $X$ arising from the monad of the adjunction $\F_2:s\Set_*\rightleftarrows s\Vect:\forgetSymbol$. By virtue of $\hat X$ having been constructed as a totalisation, there is a totalisation spectral sequence converging to $\pi_*(\hat X)$. This is known as the unstable Adams spectral sequence for $X$.

As pointed out by Blumberg and Riehl in \cite{BlumRiehlResolutions.pdf}, this process is aided by the fact that the construction of $\hat X$ is homotopically well behaved --- in $s\Set_*$, all objects are cofibrant, and in $s\Vect$, all objects are fibrant, so that the monad of the adjunction preserves weak equivalences, and fortunately, the Bousfield-Kan tower is already Reedy fibrant.

In our context, the right adjoint in $Q:s\Comm\rightleftarrows s\Vect:K$ fails to land in the subcategory of cofibrant simplicial algebras. In order to remedy this, Blumberg and Riehl show how to mix appropriate cofibrant replacements into the iteration of the monad\footnote{As every object of $s\Vect$ is fibrant, we need not mention a monadic fibrant replacement on that category, although there is no such asymmetry in the treatment of \cite{BlumRiehlResolutions.pdf}.}. That is, they show that if one can choose a functorial cofibrant replacement $c$ on $\ensuremath{s\calA}$ which additionally has structure of a homotopical comonad, then one can give a homotopically correct cosimplicial resolution of the type we desire. Writing $\overline{Q}$ for $Qc:s\Comm\to s\Vect$, their resolution takes the form:
\[
\vcenter{
\def\labelstyle{\scriptstyle}
\xymatrix@C=1.5cm@1{
cX_\bullet\,
\ar[r]
&
\,cK\overline{Q}X_\bullet\,
\ar[r];[]
&
\,c(K\overline{Q})^2X_\bullet\,
\ar@<-1ex>[l];[]
\ar@<+1ex>[l];[]
\ar@<+1ex>[r];[]
\ar@<-1ex>[r];[]
&
\,c(K\overline{Q})^3X_\bullet\,\makebox[0cm][l]{\,$\cdots $}
\ar[l];[]
\ar@<-2ex>[l];[]
\ar@<+2ex>[l];[]
}}\]
%For $X\in\Comm$, write $\underline{X}_\bullet^\bullet$ for the cosimplicial object $c(K\overline{Q})^\bullet A$.
The Blumberg-Riehl approach augments Bousfield's work on cosimplicial resolutions \cite{BousCosimpResnHtpySS.pdf},  allowing us to define a homotopical `derived completion' endofunctor on $\Comm$ by $X_\bullet\mapsto\hat X_\bullet:=\textup{Tot}((c(K\overline{Q})^\bullet X_\bullet)^{\textup{rf}})$, where `$\textup{rf}$' denotes a Reedy fibrant replacement.

Bousfield shows how to obtain a spectral sequence $\pi^s\pi_t(\underline{X}^\bullet;S^0)\Rightarrow\pi_{t-s}(\hat X;S^0)$ by applying the well established Bousfield-Kan spectral sequence to the evident cosimplicial space of Dwyer-Kan mapping complexes. It serves our purposes best to dualize, and rewrite:
\[\pi_s\pi^t((I(c(K\overline{Q})^\bullet X)^{\textup{rf}})^*)\Rightarrow \pi^{t-s}((I\hat X)^*)\]
A complete derivation would at this point include a detailed examination of the dual Hurewicz homomorphism $H^*_Q(X)\to \pi^*((IX)^*)$, in particular when $X$ is an abelian object. One shows that:
\begin{prop}
When $H^*_Q(X)$ is of finite type (for example when $X$ is connected and of finite type) the spectral sequence has $E_2^{st}=(\mathbb{L}_s\textup{Ind}_\calW)(H_Q^*X)^t=(\Ext^s_{\calW}(H_Q^*X,H_Q^*S^t))^*$.
\end{prop}
\end{AdamsDerivation}

\begin{wedding}
\begin{itemize}\squishlist
\setlength{\parindent}{.25in}
\item When do we set menu / do a tasting
\item\item\item What sound equipment do you have for sound / lighting. (upstairs \& downstairs \& for speeches, mics etc.)
\item\item\item When do we set a schedule for the day
\item\item\item when can we start set-up
\item\item\item Take-down? Do we need someone extra to set up particular decorations table cards etc.
\item\item\item Curtains
\item\item Chairs -- is there somewhere to put them if we rent two sets, one for ceremony one for dinner?
\item\item\item Table configurations?
\item\item\item desert table / cake
\item\item\item recommended DJ?
\item\item\item could we do a mix of round and square tables? How about putting square tables together for a banquet table? plan of terrasse?
\item\item\item square footage and plans for upstairs do determine arrangement for ceremony?
\item\item\item walk through the venue to determine plans for ceremony? rehearsal?
\item\item\item will you correspond with vendors on day of?
\end{itemize}

\noindent \textbf{Bice}: 514-937-6009

If you need a sound system with lightnings for the dance floor. I can provide a quality sound system with 2x 1000w speakers, 1x subwoofer, 1x wireless microphone and 2 lights effects for the dance floor. Up to 150 people I rent this kit for \$250 transport and installation included. If you are more then 150 people we should plan to add a extra pair of speakers for \$125.


\end{wedding}


\begin{TotalisationInSAlg}
\subsection{Cosimplicial objects in $s\Comm$}
\begin{prop}
Suppose that $X^\bullet$ is a cosimplicial object in $s\Comm$. Then for $0\leq m\leq\infty$, $I\Tot_m(X^\bullet)\cong\Tot_m(IX^\bullet)$, where $I$ is the right adjoint in the adjunction $\textup{Sym}:s\Vect\rightleftarrows s\Comm:I$, and the second totalization is that of the cosimplicial simplicial vector space $IX^\bullet$.
\end{prop}
\begin{proof}
The totalization of a cosimplicial object in $s\Comm$ is defined by the end
\[\Tot_m(X^\bullet):= \int_{[k]\in\Delta}\left(X^k\right)^{\textup{sk}_m\Delta[k]}\]
where for each $k$, the cotensor within the end is defined by taking various powers of the $X^k_t$ in each simplicial dimension $t$. As $I$ is the prolongation of the right adjoint in $\Vect\rightleftarrows\Comm$, it commutes with all of these limits.
\end{proof}
Following \cite{BousfieldHSSCS.pdf}, for $\textbf{B}$ a cosimplicial simplicial vector space, let $N^*N_*\textbf{B}$ be the normalized double complex. The total complex $T\textbf{B}$ is then defined by
\[(T\textbf{B})_n=\prod_{t-s=n}N^sN_t\textbf{B},\ \textup{with filtration }(F^mT\textbf{B})_n=\prod_{\substack{t-s=n\\t\geq m}}N^sN_t\textbf{B}.\]
Now if we write
\[(T_m\textbf{B})_n:=(T\textbf{B}/F^{m+1} T\textbf{B})_n\cong\prod_{\substack{t-s=n\\t\leq m}}N^sN_t\textbf{B}, \]
then $F^mT\textbf{B}$ is a decreasing filtration and the $T_m\textbf{B}$ form a tower of surjections of chain complexes with inverse limit $T_\infty\textbf{B}=T\textbf{B}$. 
\begin{lem}[{\cite[Lemma 2.2]{BousfieldHSSCS.pdf}}]
There are natural chain maps $\phi_m:N_*\Tot_m\textbf{B}\to T_m\textbf{B}$ for $m\leq\infty$ which induce isomorphisms $\pi_*\Tot_m\textbf{B}\to H_*T_m\textbf{B}$ and are compatible with the tower maps $\Tot_{s}\textbf{B}\to\Tot_m\textbf{B}$ and $T_{s}\textbf{B}\to T_m\textbf{B}$ for $s\geq m$.
\end{lem}
\begin{cor}
Suppose that $X^\bullet$ is a cosimplicial object in $s\Comm$. Then there is a map of spectral sequences, which is an isomorphism from $E^1$, from the homotopy spectral sequence of the $\Tot$ tower $\{I\Tot_mX^\bullet\}$ to the spectral sequence of the bicomplex underlying $IX^\bullet$. Moreover, there is an isomorphism $\pi_*I\Tot X^\bullet\to H_*TIX^\bullet$ (to the homology of the total complex of the bicomplex), which is compatible with the map of spectral sequences.
\end{cor}
\begin{cor}
Suppose that $X$ is a connected element of $s\Comm$. Then the Adams spectral sequence for $X$ is the spectral sequence of the bicomplex underlying the cosimplicial simplicial vector space $I((c(K\overline{Q})^\bullet X)^{\textup{rf}})$. This spectral sequence admits a vanishing line from $E^2$, and so converges strongly to its target, $\pi_{t-s}(I\hat X)$.
\end{cor}
\end{TotalisationInSAlg}

\begin{convergence}

\subsection{Cosimplicial objects and cubical diagrams}

There is a diagram of inclusions of categories
\[\xymatrix@R=4mm{
\Delta&%r1c1
\PuncCube{\infty}
\ar[l]_-{G_\infty}
\\%r1c2
{\vdots}
\ar[u]
&%r2c1
\vdots
\ar[u]
\\%r2c2
\TruncSimplex{n}
\ar[u]_-{\tau}
&%r3c1
\PuncCube{n}
\ar[u]_-{\tau}
\ar[l]_-{G_n}
\\%r3c2
\TruncSimplex{n-1}
\ar[u]_-{\tau}
&%r3c1
\PuncCube{n-1}
\ar[u]_-{\tau}
\ar[l]_-{G_{n-1}}
\\%r3c2
{\vdots}
\ar[u]
&%r2c1
\vdots
\ar[u]
}\textup{ including into }
\xymatrix@R=4mm{
\Delta^+&%r1c1
\FullCube{\infty}
\ar[l]_-{G_\infty^+}
\\%r1c2
{\vdots}
\ar[u]
&%r2c1
\vdots
\ar[u]
\\%r2c2
\TruncAugSimplex{n}
\ar[u]_-{\tau}
&%r3c1
\FullCube{n}
\ar[u]_-{\tau}
\ar[l]_-{G_n^+}
\\%r3c2
\TruncAugSimplex{n-1}
\ar[u]_-{\tau}
&%r3c1
\FullCube{n-1}
\ar[u]_-{\tau}
\ar[l]_-{G_{n-1}^+}
\\%r3c2
{\vdots}
\ar[u]
&%r2c1
\vdots
\ar[u]
}
\]
Here, $\TruncSimplex{n}$ is the truncation of the simplex category $\Delta=\TruncSimplex{\infty}$, and $\PuncCube{n}$ is the poset of finite nonempty subsets of $[n]=\{0,\ldots,n\}$, for $n\leq\infty$.
Sinha, in \cite{SinhaSpacesOfKnots.pdf}, shows that the functors $G_n$, for $n\leq\infty$, are homotopy final.

The Bousfield-Kan map see \cite{Hirschhorn.pdf} is a map $N(\Delta\downarrow\DASH)\to \Delta$ of cosimplicial spaces, where we write $N$ for the nerve of a category, and the seconds $\Delta$ here means the Yoneda embedding of the category $\Delta$, the standard cosimplicial simplex. Hirschhorn demonstrates that this is a weak equivalence of Reedy cofibrant cosimplicial spaces. One may, for any $n< m\leq\infty$, restrict $N(\TruncSimplex{m}\downarrow\DASH)\in(s\Set)^{\TruncSimplex{m}}$ to an object of $(s\Set)^\TruncSimplex{n}$, and there is a commuting diagram
\[\xymatrix@R=4mm{
N(\Delta\downarrow\DASH)|_{\TruncSimplex{n}}
\ar[r]
&%r1c1
\Delta|_{\TruncSimplex{n}}
\ar@{=}[d]\\%r1c2
N(\TruncSimplex{n}\downarrow\DASH)\ar[u]
\ar[r]
&%r2c1
\TruncSimplex{n}%r2c2
}\]
of objects in $(s\Set)^{\TruncSimplex{n}}$. In fact, by the proof in \cite{Hirschhorn.pdf}, all the objects in this diagram are cofibrant, and all the maps are weak equivalences.

Now, by results of \cite[{X.4.9}]{YellowMonster}, all (truncated) cosimplicial groups are Reedy fibrant. In particular, all of the following constructions are homotopical for any cosimplicial object in one of our categories of interest.

We may define the homotopy limit of a functor $X:\calD\to s\algCat$ to be the end $\hom^\calD(N(\calD\downarrow\DASH),X)$. The partial totalization of a functor $X:\TruncSimplex{n}\to s\algCat$ is defined as the end $\hom^\TruncSimplex{n}(\TruncSimplex{n},X)$. Suppose now that $X:\Delta^+\to s\algCat$ is augmented (and whenever you take a totalisation, restrict implicitly to nonempty). By \cite[X.4.9]{YellowMonster}, all (truncated) cosimplicial groups are Reedy fibrant, and so $X|_\Delta$ is Reedy fibrant. Thus we have the following diagram:
\[\xymatrix@R=4mm{
\hotfib(X|_{\FullCube{\infty}})\ar[d]\ar[r]&
X^{-1}\ar[r]\ar[d]&
\Tot X
\ar[d]
\ar[r]^-{\sim}
&%r1c1
\holim (X|_{\Delta})
\ar[d]
\ar[r]^-{\sim}
&%r1c2
\holim(X|_{\PuncCube{\infty}})
\ar[d]
\\%r1c3
\vdots\ar[d]&%r2c1
\vdots\ar[d]&%r2c1
\vdots\ar[d]&%r2c1
\vdots\ar[d]&%r2c2
\vdots\ar[d]
\\%r2c3
\hotfib(X|_{\FullCube{n}})\ar[d]\ar[r]&
X^{-1}\ar[r]\ar[d]&
\Tot_n X
\ar[d]
\ar[r]^-{\sim}
&%r1c1
\holim (X|_{\TruncSimplex{n}})
\ar[d]
\ar[r]^-{\sim}
&%r1c2
\holim(X|_{\PuncCube{n}})
\ar[d]
\\
\hotfib(X|_{\FullCube{n-1}})\ar[d]\ar[r]&
X^{-1}\ar[r]\ar[d]&
\Tot_{n-1} X
\ar[d]
\ar[r]^-{\sim}
&%r1c1
\holim (X|_{\TruncSimplex{n-1}})
\ar[d]
\ar[r]^-{\sim}
&%r1c2
\holim(X|_{\PuncCube{n-1}})
\ar[d]
\\
\cdots&\cdots&\cdots&\cdots&\cdots
}\]
Here, $\hotfib(X|_{\FullCube{n}})$ is defined to be the homotopy fiber of the map $X^{-1}\to \holim(X|_{\PuncCube{n}})$. This is the first definition of the homotopy fibre of a cube that Goodwillie gives. This definition is weakly equivalent (see \cite{LuisGoodwillie.pdf}), however, to an iterative definition which we'll give now. The total homotopy fiber of a $0$-cube should simply be its unique object. Suppose that one has defined the notion of total fibre for diagrams indexed by $\FullCube{n-1}$. Then for $C:\FullCube{n}\to s\algCat$, one defines
\[\hotfib(C):=\hofib(\hotfib(C|_{\FullCube{n-1}})\to \hotfib(\psi_n^*C))\]
where $\psi_n:\FullCube{n-1}\to\FullCube{n}$ is the functor $T\mapsto T\cup\{n\}$.

Interpreting the leftmost tower in the above diagram using the iterative definition of $\hotfib$, the defining fiber sequence for $\hotfib(X|_{\FullCube{n}})$ is
\[\hotfib(X|_{\FullCube{n}})\to \hotfib(X|_{\FullCube{n-1}})\to\hotfib(\psi_n^*(X|_{\FullCube{n-1}}))+,\]
and the tower map coincides with the fiber inclusion.

%%%%%%%%%%%%%%%%%%%%%%%%%%%%%%%%%%%%%%%%%%%%%%%%%%%%%\textbf{Maybe I should sweep the stuff with comparing the tot tower to the tower of holimits under the rug, given that it's well known. Instead, work with the injective model structure on diagrams.}


\newcommand{\dupdown}[2]{D^{\smash{#1}}_{\smash{#2}}}
\newcommand{\caldup}[1]{\calD^{\smash{#1}}}
\newcommand{\caldupdown}[2]{\calD^{\smash{#1}}_{\smash{#2}}}
\subsection{Functors derived with respect to Quillen homology}
Let $\mathsf{CR(s\Comm)}$ be the category of cofibrant replacement functors in $s\Comm$. That is, an object of $\mathsf{CR(s\Comm)}$ is a pair, $(c,\epsilon)$, such that $c:s\Comm\to s\Comm$ is a functor whose image consists only of cofibrant objects, and $\epsilon:c\Rightarrow I$ is a natural acyclic fibration. Morphisms in $\mathsf{CR(s\Comm)}$ are to be natural transformations which commute with the augmentations.

[Following \cite{BousKanSSeq.pdf}, Sinha, BR, RB, Harper] Fix a functor $F:s\Comm\to s\Comm$, and a cofibrant replacement functor $(c,\epsilon)\in\mathsf{CR(s\Comm)}$.

We'll define the $r^\textup{th}$ derived functor $(\dupdown{r}{b}F)$ of $F$ with respect to Andr\'e-Quillen homology. The definition is recursive: one sets $(\dupdown{0}{b}F)(X)=F(cX)$, and
\[(\dupdown{s}{c}F)(X):=\hofib((\dupdown{s-1}{c}F)(cX)\overset{(\dupdown{s-1}{c}F)(\eta_{cX})}{\to} (\dupdown{s-1}{c}F)(QcX)),\]
where $Q$ is the indecomposables functor, $\eta:I\Rightarrow Q$ is the natural surjection onto indecomposables, and $\hofib$ is any fixed  functorial construction of the homotopy fiber. These functors fit into a tower, via the following composite natural transformation:
\[(\dupdown{s}{c}F)(X)\to (\dupdown{s-1}{c}F)(cX)\overset{(\dupdown{s-1}{c}F)(\epsilon)}{\to} (\dupdown{s-1}{c}F)(X).\]
For clarity, we'll record the construction of $(\dupdown{1}{c}F)(X)$, $(\dupdown{2}{c}F)(X)$ and the map between them, in the following diagram. Note that every composable pair of  parallel arrows is defined to be a homotopy fiber sequence.


%@=50pt
\[\def\labelstyle{\scriptstyle}
\xymatrix@!0@R=30pt@C=50pt{
(\dupdown{2}{c}F)(X)\ar[dr]\ar[dd]\\
&(\dupdown{1}{c}F)(cX) \ar[rr]\ar[rd]\ar[dd]         &           &FcccX \ar[rr]\ar[rd]\ar'[d][dd]^{Fcc\epsilon}         &           &   FcQccX \ar'[d][dd]^{FcQc\epsilon}           \ar[dr]  &                  \\
(\dupdown{1}{c}F)(X)\ar[dr]&        &  (\dupdown{1}{c}F)(QcX) \ar[rr] \ar[dd]  &                     &  FccQcX \ar[rr] \ar[dd]  &             & FcQcQcX
         \ar[dd] \\
&(\dupdown{1}{c}F)(X) \ar'[r][rr] \ar[dr] &        &   FccX \ar'[r][rr] \ar[dr] &        &   FcQcX
\ar[dr] &                   \\
&        &   0 \ar[rr]      &               &   0 \ar[rr]      &                    &
0
}\]
We have thus constructed a tower
\[\xymatrix@R=4mm{
\cdots 
\ar[r]
&%r1c1
(\dupdown{2}{c}F)X
\ar[r]
&%r1c3
%r1c4
(\dupdown{1}{c}F)X
\ar[r]&%r1c5
(\dupdown{0}{c}F)X=FcX,
}\]
which is natural in $X$.
Moreover, this construction is natural in the functor $F$ and homotopical as long as $F$ preserves weak equivalences between cofibrant objects. Finally, given a morphism $(c,\epsilon)\to (c',\epsilon')$ of cofibrant replacements, one obtains natural weak equivalences $(\dupdown{t}{c})X\to(\dupdown{t}{c'})X$ for $t\geq0$.

Suppose that $(c,\epsilon)$ and $(c',\epsilon')$ are any two competing cofibrant replacement functors. Then we can form a third cofibrant replacement, $(cc',\epsilon\circ c\epsilon')$, which maps to each of $(c,\epsilon)$ and $(c',\epsilon')$, via the natural transformations $c\epsilon$ and $\epsilon'$ respectively. Thus we have natural weak equivalences of towers:
\[\xymatrix@R=4mm{
\cdots 
\ar[r]
&%r1c1
(\dupdown{2}{c}F)X
\ar[r]
&%r1c3
%r1c4
(\dupdown{1}{c}F)X
\ar[r]&%r1c5
(\dupdown{0}{c}F)X\makebox[0pt][l]{$\mbox{}=FcX$}\\
\cdots 
\ar[r]
&%r1c1
(\dupdown{2}{cc'}F)X
\ar[u]_-{\sim}
\ar[d]^-{\sim}
\ar[r]
&%r1c3
%r1c4
(\dupdown{1}{cc'}F)X
\ar[u]_-{\sim}
\ar[d]^-{\sim}
\ar[r]&%r1c5
(\dupdown{0}{cc'}F)X\makebox[0pt][l]{$\mbox{}=Fcc'X$}
\ar[u]_-{\sim}
\ar[d]^-{\sim}\\
\cdots 
\ar[r]
&%r1c1
(\dupdown{2}{c'}F)X
\ar[r]
&%r1c3
%r1c4
(\dupdown{1}{c'}F)X
\ar[r]&%r1c5
(\dupdown{0}{c'}F)X\makebox[0pt][l]{$\mbox{}=Fc'X$}
}\]
It is in this sense that the tower is independent of the cofibrant replacement used.



%A MAP INDUCED ON TFIBRES OF 2-CUBES
%\[\xymatrix@=20pt{
%f\ar[dr]\ar[dd]\\
%&X^{[-1]} \ar[rr]\ar[rd]\ar[dd]         &           &X^{[-1]} \ar[rr]\ar[rd]\ar'[d][dd]         &           &   X^{[0]} \ar'[d][dd]           \ar[dr]  &                  \\
%f\ar[dr]&        &  X^{[0]} \ar[rr] \ar[dd]  &                     &  X^{[0]} \ar[rr] \ar[dd]  &             & X^{[1]}
%         \ar[dd] \\
%&X^{[0]} \ar'[r][rr] \ar[dr] &        &   X^{[0]} \ar'[r][rr] \ar[dr] &        &   X^{[1]}
%\ar[dr] &                   \\
%&        &   X^{[1]} \ar[rr]      &               &   X^{[1]} \ar[rr]      &                    &
%X^{[2]}
%}\]

\subsection{Functors derived using the bar construction}
For this section, we will always use the standard bar construction $(b,\epsilon)$ as cofibrant replacement functor. That is, $(bX)_n:=T^{n+1}X_n$, where $T$ is the monad from the free/forget adjunction between vector spaces and nucas.

We'll now define a modification $(\caldup{s}F)(X)$ of $(\dupdown{s}{b}F)$. There will two differences: in the definition of $(\caldup{s}F)(X)$ there will be one fewer cofibrant replacement applied, and we will take \emph{strict} fibers, not \emph{homotopy} fibers. Again, the definition is inductive: one sets $(\caldup{0}F)=F$ and
\[(\caldup{s}F)(X):=\ker((\caldup{s-1}F)(bX)\to (\caldup{s-1}F)(QbX)).\]
Fortunately, there's a zigzag of towers
\[\xymatrix@R=4mm{
\cdots 
\ar[r]
&%r1c1
(\dupdown{2}{b}I)X
\ar[r]
&%r1c3
%r1c4
(\dupdown{1}{b}I)X
\ar[r]&%r1c5
(\dupdown{0}{b}I)X\makebox[0pt][l]{$\mbox{}=bX$}\\
\cdots 
\ar[r]
&%r1c1
(\caldup{2}b)X
\ar[u]_-{}
\ar[d]^-{}
\ar[r]
&%r1c3
%r1c4
(\caldup{1}b)X
\ar[u]_-{}
\ar[d]^-{}
\ar[r]&%r1c5
(\caldup{0}b)X\makebox[0pt][l]{$\mbox{}=bX$}
\ar[u]_-{}
\ar[d]^-{}\\
\cdots 
\ar[r]
&%r1c1
(\caldup{2}I)X
\ar[r]
&%r1c3
%r1c4
(\caldup{1}I)X
\ar[r]&%r1c5
(\caldup{0}I)X\makebox[0pt][l]{$\mbox{}=X$}
}\]
in which the upward maps result from mapping strict fibers to homotopy fibers, and the downward maps are induced by $\epsilon:b\to I$. [In the middle tower, we're taking the derived functors of the cofibrant replacement itself, $b$.]
%
%As these functors are particularly well behaved, we will be able to use a modified version of the derived functor construction, which we'll write $(\caldup{s}F)$, for $F$ either $I$ or $P^s$. 

%%%There are two differences between this definition and the definition of $(\dupdown{s}{b}F)$ --- there is one fewer application of the replacement $b$, and we take strict fibers (i.e.\ kernels) instead of homotopy fibers. 
\begin{prop}\label{verticalsEquivsInDvCALD}
The vertical maps in the above zigzag are all weak equivalences.
\end{prop}
\begin{proof}
That the maps between $0^\textup{th}$ derived functors are weak equivalences is clear. Suppose then, by induction on $r$, that the maps of $(r-1)^\textup{st}$ derived functors are weak equivalences. Then we may form the diagram:
\[\xymatrix@R=4mm{
&
(\dupdown{r}{b}I)X\ar[r]&
(\dupdown{r-1}{b}I)(bX)\ar[r]&
(\dupdown{r-1}{b}I)(QbX)&
\\
0\ar[r]&
(\caldup{r}b)X\ar[d]\ar[u]\ar[r]&
(\caldup{r-1}b)(bX)\ar[d]^{\sim}\ar[u]_{\sim}\ar[r]&
(\caldup{r-1}b)(QbX)\ar[d]^{\sim}\ar[u]_{\sim}\ar[r]&
0\\
0\ar[r]&
(\caldup{r}I)X\ar[r]&
(\caldup{r-1}I)(bX)\ar[r]&
(\caldup{r-1}I)(QbX)\ar[r]&
0
}\]
in which four of the vertical arrows are known to be weak equivalences by induction.
Now the bottom two rows are both short exact, using the fact (\ref{towerWithPowers}) that the functors $(\caldup{r}I)$ and $(\caldup{r}b)$ both preserve surjections. The top row is defined to be a homotopy fibre sequence. Thus each row has a homotopy long exact sequence, and we may apply the 5-lemma.
\end{proof}











Now let $P^s:s\Comm\to s\Comm$ be the ``$s^\textup{th}$ power'' functor, the prolongation of the endofunctor $Y\mapsto Y^s$ of $\Comm$.
%Our purpose in this section will be to prove the following, where we write $I$ for the identity functor $s\Comm\to s\Comm$:
\begin{prop}\label{convergenceProp}
%%%The functors $(\caldup{s}I):s\Comm\to s\Comm$ preserve weak equivalences.
%%%For $X\in s\Comm$ a connected simplicial algebra, consider the tower:
%%%\[\xymatrix@R=4mm{
%%%\cdots 
%%%\ar[r]
%%%&%r1c1
%%%(\caldup{t}I)X
%%%\ar[r]
%%%&%r1c3
%%%\cdots \ar[r]
%%%&%r1c4
%%%(\caldup{1}I)X
%%%\ar[r]&%r1c5
%%%(\caldup{0}I)X=X.
%%%}\]
For integers $t\geq2$ and $q\geq0$, the map $\pi_q(\caldup{2t+q-1}I)X\to\pi_q(\caldup{t}I)X$ is zero.
\end{prop}
\begin{proof}%[Proof of \ref{convergenceProp}]
%The first claim is \ref{towerIsHomotopical}. 
Apply $(\caldup{t}\DASH)X$ to the diagram of functors constructed in \ref{towerWithPowers}. In light of \ref{DsDt=Dt+s}, one obtains the following homotopy commuting diagram in $s\Comm$:
\[\xymatrix@R=4mm{
\cdots 
\ar[r]
&%r1c1
(\caldup{2t+q-1}I)X
\ar[r]
\ar[d]
&
\cdots \ar[r]
&%r1c4
(\caldup{t+1}I)X
\ar[r]
\ar[d]&%r1c5
(\caldup{t}I)X
\ar@{=}[d]
\\%r1c6
\cdots
\ar[r]
&%r2c1
(\caldup{t}P^{t+q})X
\ar[r]
&%r2c3
\cdots 
\ar[r]&%r2c4
(\caldup{t}P^2)X
\ar[r]
&%r2c5
(\caldup{t}I)X%r2c6
}\]
By \ref{connectivityOfDerivedPowers}, $(\caldup{t}P^{t+q})(A)$ is $q$-connected, and the result follows.
\end{proof}
The remainder of this section consists of the constructions required to complete the proofs of these two propositions.
%%%\begin{lem}\label{DsDt=Dt+s}
%%%For any functor $F$, there is a natural isomorphism $(\caldup{s}(\caldup{t}F))\cong (\caldup{s+t}F)$ so that the evident diagram commutes:
%%%\[\xymatrix@R=4mm{
%%%(\caldup{s}(\caldup{t}F))
%%%\ar[r]^-{\cong}
%%%\ar[d]^-{(\caldup{s}\delta)}
%%%&%r1c1
%%%(\caldup{s+t}F)
%%%\ar[d]^-{\delta}
%%%\\%r1c2
%%%(\caldup{s}(\caldup{t-1}F))
%%%\ar[r]^-{\cong}&%r1c1
%%%(\caldup{s+t-1}F)
%%%}\]
%%%\end{lem}
%\begin{lem}
%For $Y\in\Comm$, and integers $r,n\geq0$ and $s>0$:
%\begin{enumerate}
%\squishlist
%\setlength{\parindent}{.25in}
%\item[a)] $(\caldup{r}_nI)Y\subset (\calT_1T^n_1)\cdots (\calT_rT^n_r)Y$ is the subset consiting of those mega\-monomials which have a doubling in each of the $\calT_1,\ldots,\calT_r$ functors.
%\item[b)] $(\caldup{r}_nb)Y\subset (\calT T^n)(\calT_1T^n_1)\cdots (\calT_rT^n_r)Y$ is the subset consiting of those mega\-monomials which have a doubling in each of the $\calT_1,\ldots,\calT_r$ functors.
%\item[c)] $(\caldup{r}_nP^{s})Y\subset (\calT_1T^n_1)\cdots (\calT_rT^n_r)Y$ is the subset consiting of those mega\-monomials which have a doubling in each of the $\calT_2,\ldots,\calT_r$ functors, and a power of $s$ in the $\calT_1$ functor.
%\end{enumerate}
%
%\end{lem}
\begin{lem}\label{towerWithPowers}
The functors $(\caldup{r}I)$, $(\caldup{r}b)$ and $(\caldup{r}P^{s})$ preserve surjective maps and there is a commuting diagram of functors:
\[\xymatrix@R=4mm{
\cdots 
\ar[r]
&%r1c1
(\caldup{r}I)
\ar[r]
\ar[d]
&%r1c2
\cdots \ar[r]
&%r1c4
(\caldup{2}I)
\ar[r]
\ar[d]
&%r1c4
(\caldup{1}I)
\ar[r]
\ar[d]&%r1c5
(\caldup{0}I)
\ar@{=}[d]
\\%r1c6
\cdots
\ar[r]
&%r2c1
P^{r+1}
\ar[r]
&%r2c3
\cdots 
\ar[r]&%r2c4
P^3
\ar[r]
&P^2
\ar[r]
&%r2c5
I%r2c6
}\]
\end{lem}
\begin{proof}
Before we give this proof, we'll need a little notation.
If $F$ is the prolongation of a functor $\Comm\to\Comm$, one notes that $(\caldup{s}F)$ can be defined one level at a time. That is,  we may define functors $(\caldupdown{s}{n}F):\Comm\to\Comm$ such that $(\caldup{s}F)(X)_n=(\caldupdown{s}{n}F)(X_n)$, by defining $(\caldupdown{0}{n}F):=F$, and
\[(\caldupdown{s}{n}F)(Y):=\ker((\caldupdown{s-1}{n}F)(T^{n+1}Y)\to (\caldupdown{s-1}{n}F)(QT^{n+1}Y)).\]
The claims of the lemma follow from the following facts, which hold for any $Y\in\Comm$, and integers $r,n\geq0$ and $s>0$:
\begin{enumerate}
\squishlist
\setlength{\parindent}{.25in}
\item[a)] $(\caldupdown{r}{n}I)Y\subset (\calT_1T^n_1)\cdots (\calT_rT^n_r)Y$ is the subset consiting of those mega\-monomials which have a doubling in each of the $\calT_1,\ldots,\calT_r$ functors.
\item[b)] $(\caldupdown{r}{n}b)Y\subset (\calT T^n)(\calT_1T^n_1)\cdots (\calT_rT^n_r)Y$ is the subset consiting of those mega\-monomials which have a doubling in each of the $\calT_1,\ldots,\calT_r$ functors.
\item[c)] $(\caldupdown{r}{n}P^{s})Y\subset (\calT_1T^n_1)\cdots (\calT_rT^n_r)Y$ is the subset consiting of those mega\-monomials which have a doubling in each of the $\calT_2,\ldots,\calT_r$ functors, and a power of $s$ in the $\calT_1$ functor.
\end{enumerate}
(a) is trivial when $r=0$. Supposing an inductive hypothesis:
\begin{alignat*}{2}
(\caldupdown{r}{n}I)(Y)
&:=
\ker((\caldupdown{r-1}{n}I)(\calT_r T_r^n Y)\to (\caldupdown{r-1}{n}I) (Q\calT_rT_r^n X_n))%
\\
&=
(\caldupdown{r-1}{n}I)(\calT_r T_r^n Y)\cap\langle \textup{meganomials with a doubling in $\calT_r$}\rangle\\
&=
\langle \textup{meganomials with a doubling in $\calT_1,\ldots,\calT_{r-1}$}\rangle\cap\langle \textup{doubling in $\calT_r$}\rangle\\
&=
\langle \textup{meganomials with a doubling in $\calT_1,\ldots,\calT_{r-1},\calT_{r}$}\rangle
\end{alignat*}
The proofs of (b) and (c) are essentially the same as the proof of (a).
\end{proof}



%\begin{cor}\label{towerIsHomotopical}
%The functors $(\caldup{r}I):s\Comm\to s\Comm$ preserve all weak equivalences.
%\end{cor}
%\begin{proof}
%This is trivial when $r=0$. Suppose, by induction on $r$, that $(\caldup{r-1}I)$ preserves weak equivalences, and that $f:X\to X'$ is a weak equivalence in $s\Comm$. Then we have a commutative diagram in $s\Comm$:
%\[\xymatrix@R=4mm{
%0\ar[r]&
%(\caldup{r}I)X\ar[d]^{(\caldup{r}I)(f)}\ar[r]&
%(\caldup{r-1}I)(cX)\ar[d]^{(\caldup{r-1}I)(cf)}\ar[r]&
%(\caldup{r-1}I)(QcX)\ar[d]^{(\caldup{r-1}I)(Qcf)}\ar[r]&
%0\\0\ar[r]&
%(\caldup{r}I)X'\ar[r]&
%(\caldup{r-1}I)(cX')\ar[r]&
%(\caldup{r-1}I)(QcX')\ar[r]&
%0
%}\]
%Now the rows are short exact sequences, as $(\caldup{r-1}I)$ preserves surjective maps. The maps $cf$ and $Qcf$ are weak equivalences, as $c$ is a cofibrant replacement, and $Q$ is left Quillen. Thus, the induction hypothesis implies that the $(\caldup{r-1}I)(cf)$ and $(\caldup{r-1}I)(Qcf)$ are weak equivalences. The homotopy long exact sequence and the 5-lemma together imply that ${(\caldup{r}I)(f)}$ is a weak equivalence.
%\end{proof}

\begin{prop}\label{connectivityOfPowerQuillen}
For $A\in s\Comm$ almost free and connected, and any $s\geq0$, $P^{s}A$ is $(s-1)$-connected.
\end{prop}
\begin{proof}
Truncate Quillen's fundamental spectral sequence, as presented in \cite[thm 6.2]{MR1089001}.
\end{proof}
\begin{cor}\label{connectivityOfDerivedPowers}
For $A\in s\Comm$ connected, and any $t>0$ and $s\geq2$, $(\caldup{t}P^{s})(A)$ is $(s-t)$-connected.
\end{cor}
\begin{proof}
We'll prove this by induction on $t$. When $t=1$:
\[(\caldup{1}P^{s})A:=\ker(P^{s}(cA)\to P^{s}(QcA))=P^{s}(cA)\]
Thus $(\caldup{1}P^{s})A$ is $(s-1)$-connected, by \ref{connectivityOfPowerQuillen}.

Now suppose, by induction, that $(\caldup{t-1}P^{s})(B)$ is $(s-(t-1))$-connected for any connected $B$. Then by \ref{towerWithPowers}, there's a short exact sequence:
\[\xymatrix{
0\ar[r]&
(\caldup{t}P^{s})(A)\ar[r]&
(\caldup{t-1}P^{s})(cA)\ar[r]&
(\caldup{t-1}P^{s})(QcA)\ar[r]&
0
}\]
in which the rightmost two objects are each $(s-t+1)$-connected. The associated long exact sequence shows that $(\caldup{t}P^{s})(A)$ is $(s-t)$-connected.
\end{proof}

\begin{lem}\label{DsDt=Dt+s}
By construction, $(\caldup{s}(\caldup{t}I))(X)$ and $(\caldup{s+t}I){(X)}$ are equal. Moreover, the following two maps are naturally homotopic:
\begin{gather*}
(\caldup{s}\delta_t):(\caldup{s}(\caldup{t}I))(X)\to (\caldup{s}(\caldup{t-1}I))(X)\\
\delta_{s+t}:(\caldup{s+t}I){(X)}\to (\caldup{s+t-1}I){(X)}
\end{gather*}
%$(\caldup{s}\delta_t):(\caldup{s}(\caldup{t}I))(X)\to (\caldup{s}(\caldup{t-1}I))(X)$ and $\delta_{s+t}:(\caldup{s+t}I){(X)}\to (\caldup{s+t-2}I){(X)}$ are homotopic.so that the evident diagram commutes:
%\[\xymatrix@R=4mm{
%(\caldup{s}(\caldup{t}F))
%\ar[r]^-{\cong}
%\ar[d]^-{(\caldup{s}\delta)}
%&%r1c1
%(\caldup{s+t}F)
%\ar[d]^-{\delta}
%\\%r1c2
%(\caldup{s}(\caldup{t-1}F))
%\ar[r]^-{\cong}&%r1c1
%(\caldup{s+t-1}F)
%}\]
\end{lem}
\begin{proof}
It will be enough to construct an explicit simplicial homotopy between the maps $f:=(\caldup{s}\delta_t)$ and $g:=(\caldup{s-1}\delta_{t+1})$ for each fixed $s,t\geq0$. The maps $f$ and $g$ are just the restrictions of maps $\overline{f}$ and $\overline{g}$ defined on all of $c^{s+t}X$, as indicated by the following diagram:
\[\xymatrix@R=7mm{
c^{s+t}X
\ar@<.5ex>[r]^-{\overline{f}:=c^s\epsilon c^{t-1}}
\ar@<-.5ex>[r]_-{\overline{g}:=c^{s-1}\epsilon c^{t}}
&%r1c1
c^{s+t-1}X\\%r1c2
(\caldup{s+t}I)(X)
\ar@<.5ex>[r]^-{f}
\ar@<-.5ex>[r]_-{g}
\ar[u]
&%r2c1
(\caldup{s+t-1}I)(X)
\ar[u]
%r2c2
}\]
It is enough to note that the simplicial homotopy between $\overline{f}$ and $\overline{g}$ provided by \ref{IteratedBarConstructionHomotopy} descends to a homotopy between $f$ and $g$.
\end{proof}

\subsection{Fibres in the $\Tot$ tower}
Now we'll write $\sigma $ for the cofibrant replacement that Radelescu-Banu proves is a comonad. The following map of $2$-cubes (each occupying a horizontal plane) may be generalised to a map of $n$-cubes. It is a pointwise weak equivalence.
\[\def\labelstyle{\scriptstyle}
\xymatrix@!0@R=30pt@C=50pt{
\sigma X \ar[rr]\ar[rd]\ar[dd]         &           &   \sigma Q\sigma X \ar'[d][dd]           \ar[dr]  &                  \\
 &  \sigma Q\sigma X \ar[rr] \ar[dd]  &             & \sigma Q\sigma Q\sigma X
         \ar[dd] \\
\sigma \sigma \sigma X \ar'[r][rr] \ar[dr] &        &   \sigma Q\sigma \sigma X
\ar[dr] &                   \\
           &   \sigma \sigma Q\sigma X \ar[rr]      &                    &
\sigma Q\sigma Q\sigma X
}\]
The homotopy total fibers of these squares are $\widetilde{X}^2:=\hofib(X\to \Tot_1\overline{X}^\bullet)$ and $(\dupdown{2}{\sigma} I)(X)$ respectively. This allows us to construct the following diagram. It shows that the spectral sequence coming from the $\textup{Tot}$ tower has connectivity properties corresponding to those proven above. Of course, the top three rows here are in each entry a fibre sequence.
%%%\[\xymatrix@R=4mm{
%%%\cdots 
%%%\ar[r]
%%%&%r1c1
%%%\Tot_{t-1}(\overline{X}^\bullet)
%%%\ar[r]
%%%&%r1c3
%%%\cdots \ar[r]
%%%&%r1c4
%%%\Tot_0(\overline{X}^\bullet)
%%%\ar[r]&%r1c5
%%%0\makebox[0pt][l]{$\mbox{}=\sigma X$}
%%%\\
%%%\cdots
%%%\ar[r]
%%%&%r1c1
%%%X
%%%\ar[u]
%%%\ar[r]
%%%&%r1c3
%%%\cdots \ar[r]
%%%&%r1c4
%%%X
%%%\ar[u]
%%%\ar[r]&%r1c5
%%%X\makebox[0pt][l]{$\mbox{}=\sigma X$}\ar[u]
%%%\\
%%%\cdots 
%%%\ar[r]
%%%&%r1c1
%%%\widetilde{X}^t
%%%\ar[u]_-{}
%%%\ar[d]^-{\sim}
%%%\ar[r]
%%%&%r1c3
%%%\cdots \ar[r]
%%%&%r1c4
%%%\widetilde{X}^1
%%%\ar[u]_-{}
%%%\ar[d]^-{\sim}
%%%\ar[r]&%r1c5
%%%\widetilde{X}^0\makebox[0pt][l]{$\mbox{}=\sigma X$}
%%%\ar[u]_-{}
%%%\ar[d]^-{\sim}\\
%%%\cdots 
%%%\ar[r]
%%%&%r1c1
%%%(D^{t}_\sigma I)X
%%%\ar[r]
%%%&%r1c3
%%%\cdots \ar[r]
%%%&%r1c4
%%%(D^1_\sigma I)X
%%%\ar[r]&%r1c5
%%%(D^0_\sigma I)X\makebox[0pt][l]{$\mbox{}=\sigma X$}\\
%%%\cdots 
%%%\ar[r]
%%%&%r1c1
%%%(D^{t}_{b\sigma} I)X
%%%\ar[u]_-{\sim}
%%%\ar[d]^-{\sim}
%%%\ar[r]
%%%&%r1c3
%%%\cdots \ar[r]
%%%&%r1c4
%%%(D^1_{b\sigma} I)X
%%%\ar[u]_-{\sim}
%%%\ar[d]^-{\sim}
%%%\ar[r]&%r1b5
%%%(D^0_{b\sigma} I)X\makebox[0pt][l]{$\mbox{}={b\sigma} X$}
%%%\ar[u]_-{\sim}
%%%\ar[d]^-{\sim}\\
%%%\cdots 
%%%\ar[r]
%%%&%r1c1
%%%(D^{t}_bI)X
%%%\ar[r]
%%%&%r1c3
%%%\cdots \ar[r]
%%%&%r1c4
%%%(D^1_bI)X
%%%\ar[r]&%r1c5
%%%(D^0_bI)X\makebox[0pt][l]{$\mbox{}=bX$}\\
%%%\cdots 
%%%\ar[r]
%%%&%r1c1
%%%(\caldup{t}b)X
%%%\ar[u]_-{\sim}
%%%\ar[d]^-{\sim}
%%%\ar[r]
%%%&%r1c3
%%%\cdots \ar[r]
%%%&%r1c4
%%%(\caldup{1}b)X
%%%\ar[u]_-{\sim}
%%%\ar[d]^-{\sim}
%%%\ar[r]&%r1c5
%%%(\caldup{0}b)X\makebox[0pt][l]{$\mbox{}=bX$}
%%%\ar[u]_-{\sim}
%%%\ar[d]^-{\sim}\\
%%%\cdots 
%%%\ar[r]
%%%&%r1c1
%%%(\caldup{t}I)X
%%%\ar[r]
%%%&%r1c3
%%%\cdots \ar[r]
%%%&%r1c4
%%%(\caldup{1}I)X
%%%\ar[r]&%r1c5
%%%(\caldup{0}I)X\makebox[0pt][l]{$\mbox{}=X$}
%%%}\]
\[\xymatrix@R=4mm{
\cdots 
\ar[r]
&%r1c1
\Tot_{1}(\overline{X}^\bullet)
\ar[r]
&%r1c3
%r1c4
\Tot_0(\overline{X}^\bullet)
\ar[r]&%r1c5
0
\\
\cdots
\ar[r]
&%r1c1
X
\ar[u]
\ar[r]
&%r1c3
%r1c4
X
\ar[u]
\ar[r]&%r1c5
X\ar[u]
\\
\cdots 
\ar[r]
&%r1c1
\widetilde{X}^2
\ar[u]_-{}
\ar[d]^-{\sim}
\ar[r]
&%r1c3
%r1c4
\widetilde{X}^1
\ar[u]_-{}
\ar[d]^-{\sim}
\ar[r]&%r1c5
\widetilde{X}^0\makebox[0pt][l]{$\mbox{}=\sigma X$}
\ar[u]_-{}
\ar[d]^-{\sim}\\
\cdots 
\ar[r]
&%r1c1
(\dupdown{2}{\sigma} I)X
\ar[r]
&%r1c3
%r1c4
(\dupdown{1}{\sigma} I)X
\ar[r]&%r1c5
(\dupdown{0}{\sigma} I)X\makebox[0pt][l]{$\mbox{}=\sigma X$}\\
\cdots 
\ar[r]
&%r1c1
(\dupdown{2}{b\sigma} I)X
\ar[u]_-{\sim}
\ar[d]^-{\sim}
\ar[r]
&%r1c3
%r1c4
(\dupdown{1}{b\sigma} I)X
\ar[u]_-{\sim}
\ar[d]^-{\sim}
\ar[r]&%r1b5
(\dupdown{0}{b\sigma} I)X\makebox[0pt][l]{$\mbox{}={b\sigma} X$}
\ar[u]_-{\sim}
\ar[d]^-{\sim}\\
\cdots 
\ar[r]
&%r1c1
(\dupdown{2}{b}I)X
\ar[r]
&%r1c3
%r1c4
(\dupdown{1}{b}I)X
\ar[r]&%r1c5
(\dupdown{0}{b}I)X\makebox[0pt][l]{$\mbox{}=bX$}\\
\cdots 
\ar[r]
&%r1c1
(\caldup{2}b)X
\ar[u]_-{\sim}
\ar[d]^-{\sim}
\ar[r]
&%r1c3
%r1c4
(\caldup{1}b)X
\ar[u]_-{\sim}
\ar[d]^-{\sim}
\ar[r]&%r1c5
(\caldup{0}b)X\makebox[0pt][l]{$\mbox{}=bX$}
\ar[u]_-{\sim}
\ar[d]^-{\sim}\\
\cdots 
\ar[r]
&%r1c1
(\caldup{2}I)X
\ar[r]
&%r1c3
%r1c4
(\caldup{1}I)X
\ar[r]&%r1c5
(\caldup{0}I)X\makebox[0pt][l]{$\mbox{}=X$}
}\]


\end{convergence}

\begin{iteratedBarConstructions}
\subsection{Iterated simplicial bar constructions}
In this section we will be concerned with iterated simplicial bar constructions. Before getting into the details, let's establish a little notation. Fix a category of universal algebras $\algCat$. For any simplicial object $X$ in $\algCat$, we'll write 
\[\trip{d}{i,q}{X}:X_q\to X_{q-1}\textup{\ and\ }\trip{s}{i,q}{X}:X_q\to X_{q+1}\]
for the $i^\textup{th}$ face and degeneracy maps out of $X_q$. Suppose $F,G\in\algCat^\algCat$ are endofunctors, $\Phi:F\to G$ is a natural transformation, and $A,B\in\algCat$ are objects. Write $[\Phi]:\algCat(A,B)\to\algCat(FA,GB)$ for the operator sending $m:A\to B$ to the diagonal composite in the commuting square
\[\xymatrix@R=4mm{
FA
\ar[r]^-{\Phi_A}
\ar[d]_-{Fm}
\ar[dr]|-{[\Phi]m}
&%r1c1
GA
\ar[d]^-{Gm}
\\%r1c2
FB
\ar[r]_-{\Phi_B}
&%r2c1
GB
%r2c2
}\]
Write $T:\algCat\to \algCat$ for the comonad of the free/forgetful adjunction down to sets. Then there is an (augmented) simplicial endofunctor, $\frakb\in s\algCat^\algCat$, derived from the unit and counit of the adjunction:
\[\vcenter{
\def\labelstyle{\scriptstyle}
\xymatrix@C=1.65cm@1{
{\ I\,}
&
\,T^1\,
\ar@{..>}[l]|(.65){\frakd_{0,0}}
\ar[r]|(.65){\fraks_{0,0}}
&
\,T^2\,
\ar@<-1ex>[l]|(.65){\frakd_{0,1}}
\ar@<+1ex>[l]|(.65){\frakd_{1,1}}
\ar@<+1ex>[r]|(.65){\fraks_{0,1}}
\ar@<-1ex>[r]|(.65){\fraks_{1,1}}
&
\,T^3\,
\ar[l]|(.65){\frakd_{1,2}}
\ar@<-2ex>[l]|(.65){\frakd_{0,2}}
\ar@<+2ex>[l]|(.65){\frakd_{2,2}}
\ar[r]|(.65){\fraks_{1,2}}
\ar@<+2ex>[r]|(.65){\fraks_{0,2}}
\ar@<-2ex>[r]|(.65){\fraks_{2,2}}
&
\,T^4\,\makebox[0cm][l]{\,$\cdots $}
\ar@<-3ex>[l]|(.65){\frakd_{0,3}}
\ar@<-1ex>[l]|(.65){\frakd_{1,3}}
\ar@<+1ex>[l]|(.65){\frakd_{2,3}}
\ar@<+3ex>[l]|(.65){\frakd_{3,3}}
}}\]




The simplicial bar construction is the cofibrant replacement functor $(b,\epsilon)$ on $s\algCat$ which is the diagonal of the bisimplicial object obtained by application of $\frakb$ levelwise. That is, for $X\in s\algCat$, $bX$ is the simplicial object with $(bX)_q:=T^{q+1}X_q$, and with
\[\trip{d}{i,q}{bX}:=[\frakd_{i,q}]\trip{d}{i,q}{X}.\]
The augmentation $\epsilon:b\to I$ is defined on level $q$ by 
\[\epsilon_q=\frakd_{0,0}\frakd_{0,1}\cdots \frakd_{0,q}:T^{q+1}\to I\]


%\[\vcenter{
%\def\labelstyle{\scriptstyle}
%\xymatrix@C=1.65cm@1{
%\,T^1X_0\,
%\ar[r]|(.65){\fraks_{0,0}\trip{s}{0,0}{X}}
%&
%\,T^2X_1\,
%\ar@<-1ex>[l]|(.65){\frakd_{0,1}}
%\ar@<+1ex>[l]|(.65){\frakd_{1,1}}
%\ar@<+1ex>[r]|(.65){\fraks_{0,1}}
%\ar@<-1ex>[r]|(.65){\fraks_{1,1}}
%&
%\,T^3X_2\,
%\ar[l]|(.65){\frakd_{1,2}}
%\ar@<-2ex>[l]|(.65){\frakd_{0,2}}
%\ar@<+2ex>[l]|(.65){\frakd_{2,2}}
%\ar[r]|(.65){\fraks_{1,2}}
%\ar@<+2ex>[r]|(.65){\fraks_{0,2}}
%\ar@<-2ex>[r]|(.65){\fraks_{2,2}}
%&
%\,T^4X_3\,\makebox[0cm][l]{\,$\cdots $}
%\ar@<-3ex>[l]|(.65){\frakd_{0,3}}
%\ar@<-1ex>[l]|(.65){\frakd_{1,3}}
%\ar@<+1ex>[l]|(.65){\frakd_{2,3}}
%\ar@<+3ex>[l]|(.65){\frakd_{3,3}}
%}}\]
%
%
%\[(bX)_n:=(\frakb(X_n))_n\]
%where we may write the 
%, is obtained by applying the simplicial endofunctor above to $X\in s\algCat$ levelwise, and taking the diagonal.




\begin{prop}\label{IteratedBarConstructionHomotopy}
For any integers $s,t\geq1$, there is a simplicial homotopy $b^{s-1}\epsilon_{(b^{t}X)}\simeq b^{s}\epsilon_{(b^{t-1}X)}$ of maps $b^{s+t}X\to b^{s+t-1}X$, natural in $X\in s\algCat$.
\end{prop}
\begin{proof}
Write $K=b^{s+t}X$ and $L=b^{s+t-1}X$ for the source and target of these maps respectively. Noting the formulae
\[[\frakd_{iq}]^2=[\frakd_{q+i,2q}\circ\frakd_{i,2q+1}]\ \textup{and}\  [\fraks_{iq}]^2=[\fraks_{q+i+2,2q+2}\circ\fraks_{i,2q+1}],\]
we can describe the simplicial structure maps in $K$ and $L$ as follows:
\begin{alignat*}{2}
\trip{d}{iq}{L}&=[\frakd_{iq}]^{s+t-1}\trip{d}{iq}{X}\\
\trip{s}{iq}{L}&=[\fraks_{iq}]^{s+t-1}\trip{s}{iq}{X}\\
\trip{d}{iq}{K}&=[\frakd_{iq}]^{s-1}[\frakd_{q+i,2q}\circ\frakd_{i,2q+1}][\frakd_{iq}]^{t-1}\trip{d}{iq}{X}\\
\trip{s}{iq}{K}&=[\fraks_{iq}]^{s-1}[\fraks_{q+i+2,2q+2}\circ\fraks_{i,2q+1}][\fraks_{iq}]^{t-1}\trip{s}{iq}{X}
\end{alignat*}
We can now state an explicit simplicial homotopy between the two maps of interest. Using precisely the notation of \cite[\S5]{MaySimpObj.pdf}, we define $\trip{h}{jq}{}:K_q\to L_{q+1}$, for $0\leq i\leq q$, by the formula
\[\trip{h}{jq}{}:=[\fraks_{jq}]^{s-1}[\frakd_{j+1,q+2}\circ\cdots \circ\frakd_{j+1,2q+1}][\fraks_{jq}]^{t-1}\trip{s}{jq}{X}\]
If the $\trip{h}{jq}{}$ in fact form a valid homotopy $f\simeq g$, then: %$f_q=\trip{d}{0,q+1}{L}\trip{h}{0,q}{}$, and $f_q=\trip{d}{q+1,q+1}{L}\trip{h}{q,q}{}$. We may calculate these composites as follows:
\begin{alignat*}{2}
f_q&=\trip{d}{0,q+1}{L}\trip{h}{0,q}{}\\
&=
[\frakd_{0,q+1}\fraks_{0q}]^{s-1}[\frakd_{0,q+1}\frakd_{1,q+2}\circ\cdots \circ\frakd_{1,2q+1}][\frakd_{0,q+1}\fraks_{0q}]^{t-1}(\trip{d}{0,q+1}{X}\trip{s}{0q}{X})%
\\
&=
[\trip{\textup{id}}{T^{q+1}}{}]^{s-1}[\frakd_{0,q+1}\frakd_{0,q+2}\circ\cdots \circ\frakd_{0,2q+1}][\trip{\textup{id}}{T^{q+1}}{}]^{t-1}\trip{\textup{id}}{X_q}{}%
\end{alignat*}
which is exactly the action of  $b^{s-1}\epsilon_{(b^{t}X)}$ in level $q$. Similarly,
\begin{alignat*}{2}
g_q&=\trip{d}{q+1,q+1}{L}\trip{h}{q,q}{}\\
&=
[\frakd_{q+1,q+1}\fraks_{qq}]^{s-1}[\frakd_{q+1,q+1}\frakd_{q+1,q+2}\circ\cdots \circ\frakd_{q+1,2q+1}][\frakd_{q+1,q+1}\fraks_{qq}]^{t-1}(\trip{d}{q+1,q+1}{X}\trip{s}{qq}{X})
\\
&=
[\trip{\textup{id}}{T^{q+1}}{}]^{s-1}[\frakd_{q+1,q+1}\frakd_{q+1,q+2}\circ\cdots \circ\frakd_{q+1,2q+1}][\trip{\textup{id}}{T^{q+1}}{}]^{t-1}\trip{\textup{id}}{X_q}{}%
\end{alignat*}
is exactly the action of $b^{s}\epsilon_{(b^{t-1}X)}$ in level $q$.
All that remains is to verify that the $h_{i,q}$ satisfy the defining identities for the notion of simplicial homotopy. While forming each of the composites of interest, none of the groupings provided by the square brackets are disturbed. Thus, each identity (labeled 1-5) can be checked one piece (labeled a-b) at a time:
{\renewcommand{\circ}{\relax}
\begin{enumerate}\squishlist
\setlength{\parindent}{.25in}
\item $\trip{d}{i,q+1}{L}\circ \trip{h}{j,q}{}=\trip{h}{j-1,q-1}{}\circ \trip{d}{i,q}{K}$ whenever $0\leq i<j\leq q$
\begin{enumerate}\squishlist
\setlength{\parindent}{.25in}
\item $\trip{d}{i,q+1}{X}\circ \trip{s}{j,q}{X}=\trip{s}{j-1,q-1}{X}\circ \trip{d}{i,q}{X}$ and $\trip{\frakd}{i,q+1}{}\circ \trip{\fraks}{j,q}{}=\trip{\fraks}{j-1,q-1}{}\circ \trip{\frakd}{i,q}{}$
\item 
$\frakd_{i,q+1}\circ
\frakd_{j+1,q+2}\circ\cdots \circ\frakd_{j+1,2q+1}=
\frakd_{j,q+1}\circ\cdots \circ\frakd_{j,2q-1}\circ
\frakd_{q+i,2q}\circ\frakd_{i,2q+1}$
\end{enumerate}
\item $\trip{d}{j+1,q+1}{L}\circ \trip{h}{j,q}{}=\trip{d}{j+1,q+1}{L}\circ \trip{h}{j+1,q}{}$ whenever $0\leq j\leq q-1$
\begin{enumerate}\squishlist
\setlength{\parindent}{.25in}
\item $\trip{d}{j+1,q+1}{X}\circ \trip{s}{j,q}{X}=\trip{d}{j+1,q+1}{X}\circ \trip{s}{j+1,q}{X}$ and $\trip{\frakd}{j+1,q+1}{}\circ \trip{\fraks}{j,q}{}=\trip{\frakd}{j+1,q+1}{}\circ \trip{\fraks}{j+1,q}{}$
\item 
$\frakd_{j+1,q+1}\circ
\frakd_{j+1,q+2}\circ\cdots \circ\frakd_{j+1,2q+1}=
\frakd_{j+1,q+1}\circ
\frakd_{j+2,q+2}\circ\cdots \circ\frakd_{j+2,2q+1}$
\end{enumerate}
\item $\trip{d}{i,q+1}{L}\circ \trip{h}{j,q}{}=\trip{h}{j,q-1}{}\circ \trip{d}{i-1,q}{K}$ whenever $0\leq j<i-1\leq q$
\begin{enumerate}\squishlist
\setlength{\parindent}{.25in}
\item $\trip{d}{i,q+1}{X}\circ \trip{s}{j,q}{X}=\trip{s}{j,q-1}{X}\circ \trip{d}{i-1,q}{X}$ and $\trip{\frakd}{i,q+1}{}\circ \trip{\fraks}{j,q}{}=\trip{\fraks}{j,q-1}{}\circ \trip{\frakd}{i-1,q}{}$
\item 
$\frakd_{i,q+1}\circ
\frakd_{j+1,q+2}\circ\cdots \circ\frakd_{j+1,2q+1}=
\frakd_{j+1,q+1}\circ\cdots \circ\frakd_{j+1,2q-1}\circ
\frakd_{q+i-1,2q}\circ\frakd_{i-1,2q+1}$
\end{enumerate}
\item $\trip{s}{i,q+1}{L}\circ \trip{h}{j,q}{}=\trip{h}{j+1,q+1}{}\circ \trip{s}{i,q}{K}$ whenever $0\leq i\leq j\leq q$
\begin{enumerate}\squishlist
\setlength{\parindent}{.25in}
\item $\trip{s}{i,q+1}{X}\circ \trip{s}{j,q}{X}=\trip{s}{j+1,q+1}{X}\circ \trip{s}{i,q}{X}$ and $\trip{\fraks}{i,q+1}{}\circ \trip{\fraks}{j,q}{}=\trip{\fraks}{j+1,q+1}{}\circ \trip{\fraks}{i,q}{}$
\item 
$\fraks_{i,q+1}\circ
\frakd_{j+1,q+2}\circ\cdots \circ\frakd_{j+1,2q+1}=
\frakd_{j+2,q+3}\circ\cdots \circ\frakd_{j+2,2q+3}\circ
\fraks_{q+i+2,2q+2}\circ\fraks_{i,2q+1}$
\end{enumerate}
\item $\trip{s}{i,q+1}{L}\circ \trip{h}{j,q}{}=\trip{h}{j,q+1}{}\circ \trip{s}{i-1,q}{K}$ whenever $0\leq j<i\leq q+1$
\begin{enumerate}\squishlist
\setlength{\parindent}{.25in}
\item $\trip{s}{i,q+1}{X}\circ \trip{s}{j,q}{X}=\trip{s}{j,q+1}{X}\circ \trip{s}{i-1,q}{X}$ and $\trip{\fraks}{i,q+1}{}\circ \trip{\fraks}{j,q}{}=\trip{\fraks}{j,q+1}{}\circ \trip{\fraks}{i-1,q}{}$
\item 
$\fraks_{i,q+1}\circ
\frakd_{j+1,q+2}\circ\cdots \circ\frakd_{j+1,2q+1}=
\frakd_{j+1,q+3}\circ\cdots \circ\frakd_{j+1,2q+3}\circ
\fraks_{q+i+1,2q+2}\circ\fraks_{i-1,2q+1}$
\end{enumerate}
\end{enumerate}
}
\noindent Each of these equations follows from the simplicial identities.
\end{proof}
\end{iteratedBarConstructions}
%
%
%Before giving the proof, we'll introduce a little notation for the maps involved in these iterated bar constructions. Whenever $0\leq i\leq q$, there are natural transformations write $\frakd_{iq}:=(T^{i}\rho T^{q-i}):T^{q+1}\Rightarrow T^q$ and $\fraks_{iq}:=(T^{i+1}\imath T^{q-i}):T^{q+1}\Rightarrow T^{q+2}$.
%
%
%
%
%
%
%
%$K=b^{s+t}X$, $L=b^{s+t-1}X$, so that
%\[K_q=(T^{q+1})^{s+t}X_q\qquad L_q=(T^{q+1})^{s+t-1}X_q\]
%
%
%
%Write $\frakd_{iq}:=(T^{i}\rho T^{q-i}):T^{q+1}\Rightarrow T^q$ and $\fraks_{iq}:=(T^{i+1}\imath T^{q-i}):T^{q+1}\Rightarrow T^{q+2}$. Given natural transformations $\alpha:F\Rightarrow F'$ and $\beta:G\Rightarrow G'$, write $[\alpha][\beta]$ for horizontal composition $F\circ G\Rightarrow F'\circ G'$. Use the symbol $\circ$ for vertical composition, as ever.
%
%It will be useful to note the following two equations
%\[[\frakd_{iq}]^2=[\frakd_{q+i,2q}\circ\frakd_{i,2q+1}]\ \textup{and}\  [\fraks_{iq}]^2=[\fraks_{q+i+2,2q+2}\circ\fraks_{i,2q+1}]\]
%In light of these, we can write formulae for the differentials on the various bar constructions as follows:
%\[\xymatrix@R=5mm@C=65mm{
%\makebox[0cm][r]{$\trip{d}{iq}{K}:\,$}K_q
%\ar[r]^-{[\frakd_{iq}]^{s-1}[\frakd_{q+i,2q}\circ\frakd_{i,2q+1}][\frakd_{iq}]^{t-1}\trip{d}{iq}{X}}%^-{[\frakd_{iq}]^{s+t}\trip{d}{iq}{X}}
%&
%K_{q-1}\\
%\makebox[0cm][r]{$\trip{s}{iq}{K}:\,$}K_q
%\ar[r]^-{[\fraks_{iq}]^{s-1}[\fraks_{q+i+2,2q+2}\circ\fraks_{i,2q+1}][\fraks_{iq}]^{t-1}\trip{s}{iq}{X}}%^-{[\fraks_{iq}]^{s+t}\trip{s}{iq}{X}}
%&
%K_{q+1}\\
%\makebox[0cm][r]{$\trip{d}{iq}{L}:\,$}L_q
%\ar[r]^-{[\frakd_{iq}]^{s+t-1}\trip{d}{iq}{X}}
%&
%L_{q-1}\\
%\makebox[0cm][r]{$\trip{s}{iq}{L}:\,$}L_q
%\ar[r]^-{[\fraks_{iq}]^{s+t-1}\trip{s}{iq}{X}}
%&
%L_{q+1}\\
%\makebox[0cm][r]{$\trip{h}{jq}{}:\,$}K_q
%\ar[r]^-{[\fraks_{jq}]^{s-1}[\frakd_{j+1,q+2}\circ\cdots \circ\frakd_{j+1,2q+1}][\fraks_{jq}]^{t-1}\trip{s}{jq}{X}}%_-{([\fraks_{jq}]^{s-1}T^{j+1}\rho^{q}T^{q-j+1}[\fraks_{jq}]^{t-1}\trip{s}{jq}{X})}
%&
%L_{q+1}\\
%}\]
%
%\[\trip{d}{i,q}{K}=\frakd_{q+i,2q}\circ\frakd_{i,2q+1}\]
%\[\trip{d}{i-1,q}{K}=\frakd_{q+i-1,2q}\circ\frakd_{i-1,2q+1}\]
%
%
%
%\[\trip{s}{i,q}{K}=\fraks_{q+i+2,2q+2}\circ\fraks_{i,2q+1}\]
%\[\trip{s}{i-1,q}{K}=\fraks_{q+i+1,2q+2}\circ\fraks_{i-1,2q+1}\]
%
%\[\trip{d}{i,q}{L}=\frakd_{i,q}\]
%\[\trip{d}{i,q+1}{L}=\frakd_{i,q+1}\]
%\[\trip{d}{j+1,q+1}{L}=\frakd_{j+1,q+1}\]
%
%
%\[\trip{s}{iq}{L}=\fraks_{iq}\]
%\[\trip{s}{i,q+1}{L}=\fraks_{i,q+1}\]
%
%
%\[\trip{h}{j,q}{}=\frakd_{j+1,q+2}\circ\cdots \circ\frakd_{j+1,2q+1}\]
%\[\trip{h}{j,q-1}{}=\frakd_{j+1,q+1}\circ\cdots \circ\frakd_{j+1,2q-1}\]
%\[\trip{h}{j-1,q-1}{}=\frakd_{j,q+1}\circ\cdots \circ\frakd_{j,2q-1}\]
%\[\trip{h}{j+1,q}{}=\frakd_{j+2,q+2}\circ\cdots \circ\frakd_{j+2,2q+1}\]
%\[\trip{h}{j,q+1}{}=\frakd_{j+1,q+3}\circ\cdots \circ\frakd_{j+1,2q+3}\]
%\[\trip{h}{j+1,q+1}{}=\frakd_{j+2,q+3}\circ\cdots \circ\frakd_{j+2,2q+3}\]
%
%
%
%{\renewcommand{\circ}{\relax}
%\begin{enumerate}\squishlist
%\setlength{\parindent}{.25in}
%\item $\trip{d}{i,q+1}{L}\circ \trip{h}{j,q}{}=\trip{h}{j-1,q-1}{}\circ \trip{d}{i,q}{K}$ whenever $0\leq i<j\leq q$
%\begin{enumerate}\squishlist
%\setlength{\parindent}{.25in}
%\item $\trip{d}{i,q+1}{X}\circ \trip{s}{j,q}{X}=\trip{s}{j-1,q-1}{X}\circ \trip{d}{i,q}{X}$ and $\trip{\frakd}{i,q+1}{}\circ \trip{\fraks}{j,q}{}=\trip{\fraks}{j-1,q-1}{}\circ \trip{\frakd}{i,q}{}$
%\item 
%$\frakd_{i,q+1}\circ
%\frakd_{j+1,q+2}\circ\cdots \circ\frakd_{j+1,2q+1}=
%\frakd_{j,q+1}\circ\cdots \circ\frakd_{j,2q-1}\circ
%\frakd_{q+i,2q}\circ\frakd_{i,2q+1}$
%\end{enumerate}
%\item $\trip{d}{j+1,q+1}{L}\circ \trip{h}{j,q}{}=\trip{d}{j+1,q+1}{L}\circ \trip{h}{j+1,q}{}$ whenever $0\leq j\leq q-1$
%\begin{enumerate}\squishlist
%\setlength{\parindent}{.25in}
%\item $\trip{d}{j+1,q+1}{X}\circ \trip{s}{j,q}{X}=\trip{d}{j+1,q+1}{X}\circ \trip{s}{j+1,q}{X}$ and $\trip{\frakd}{j+1,q+1}{}\circ \trip{\fraks}{j,q}{}=\trip{\frakd}{j+1,q+1}{}\circ \trip{\fraks}{j+1,q}{}$
%\item 
%$\frakd_{j+1,q+1}\circ
%\frakd_{j+1,q+2}\circ\cdots \circ\frakd_{j+1,2q+1}=
%\frakd_{j+1,q+1}\circ
%\frakd_{j+2,q+2}\circ\cdots \circ\frakd_{j+2,2q+1}$
%\end{enumerate}
%\item $\trip{d}{i,q+1}{L}\circ \trip{h}{j,q}{}=\trip{h}{j,q-1}{}\circ \trip{d}{i-1,q}{K}$ whenever $0\leq j<i-1\leq q$
%\begin{enumerate}\squishlist
%\setlength{\parindent}{.25in}
%\item $\trip{d}{i,q+1}{X}\circ \trip{s}{j,q}{X}=\trip{s}{j,q-1}{X}\circ \trip{d}{i-1,q}{X}$ and $\trip{\frakd}{i,q+1}{}\circ \trip{\fraks}{j,q}{}=\trip{\fraks}{j,q-1}{}\circ \trip{\frakd}{i-1,q}{}$
%\item 
%$\frakd_{i,q+1}\circ
%\frakd_{j+1,q+2}\circ\cdots \circ\frakd_{j+1,2q+1}=
%\frakd_{j+1,q+1}\circ\cdots \circ\frakd_{j+1,2q-1}\circ
%\frakd_{q+i-1,2q}\circ\frakd_{i-1,2q+1}$
%\end{enumerate}
%\item $\trip{s}{i,q+1}{L}\circ \trip{h}{j,q}{}=\trip{h}{j+1,q+1}{}\circ \trip{s}{i,q}{K}$ whenever $0\leq i\leq j\leq q$
%\begin{enumerate}\squishlist
%\setlength{\parindent}{.25in}
%\item $\trip{s}{i,q+1}{X}\circ \trip{s}{j,q}{X}=\trip{s}{j+1,q+1}{X}\circ \trip{s}{i,q}{X}$ and $\trip{\fraks}{i,q+1}{}\circ \trip{\fraks}{j,q}{}=\trip{\fraks}{j+1,q+1}{}\circ \trip{\fraks}{i,q}{}$
%\item 
%$\fraks_{i,q+1}\circ
%\frakd_{j+1,q+2}\circ\cdots \circ\frakd_{j+1,2q+1}=
%\frakd_{j+2,q+3}\circ\cdots \circ\frakd_{j+2,2q+3}\circ
%\fraks_{q+i+2,2q+2}\circ\fraks_{i,2q+1}$
%\end{enumerate}
%\item $\trip{s}{i,q+1}{L}\circ \trip{h}{j,q}{}=\trip{h}{j,q+1}{}\circ \trip{s}{i-1,q}{K}$ whenever $0\leq j<i\leq q+1$
%\begin{enumerate}\squishlist
%\setlength{\parindent}{.25in}
%\item $\trip{s}{i,q+1}{X}\circ \trip{s}{j,q}{X}=\trip{s}{j,q+1}{X}\circ \trip{s}{i-1,q}{X}$ and $\trip{\fraks}{i,q+1}{}\circ \trip{\fraks}{j,q}{}=\trip{\fraks}{j,q+1}{}\circ \trip{\fraks}{i-1,q}{}$
%\item 
%$\fraks_{i,q+1}\circ
%\frakd_{j+1,q+2}\circ\cdots \circ\frakd_{j+1,2q+1}=
%\frakd_{j+1,q+3}\circ\cdots \circ\frakd_{j+1,2q+3}\circ
%\fraks_{q+i+1,2q+2}\circ\fraks_{i-1,2q+1}$
%\end{enumerate}
%\end{enumerate}
%}
%
%\begin{align*}
% d_id_j&=d_{j-1}d_i\quad \text{ if $i<j$}, \notag\\
% d_i s_j &= s_{j-1} d_i\quad\text{ if $i < j$},\notag \\
%d_j s_j &= d_{j+1} s_j=\text{id},\label{E: axioms}\\
% d_i s_j &= s_j d_{i-1}\quad  \text{ if $i > j + 1$},\notag\\
%  s_is_j&=s_{j+1}s_i\quad \text{ if $i\leq j$}.\notag
%\end{align*}
%
%
%
%
%
%
%
%
%
%%%%%%%%%%%%
%%%%%%%%%%%%
%%%%%%%%%%%%
%%%%%%%%%%%%
%%%%%%%%%%%%
%%%%%%%%%%%%
%%%%%%%%%%%%
%%%%%%%%%%%%
%%%%%%%%%%%%
%%%%%%%%%%%%
%%%%%%%%%%%%
%%%%%%%%%%%%
%%%%%%%%%%%%
%%%%%%%%%%%%
%%%%%%%%%%%%
%%%%%%%%%%%%
%%%%%%%%%%%%
%%%%%%%%%%%%
%%%%%%%%%%%%
%%%%%%%%%%%%
%%%%%%%%%%%%
%%%%%%%%%%%%
%%%%%%%%%%%%
%%%%%%%%%%%%
%%%%%%%%%%%%
%%%%%%%%%%%%\newpage
%%%%%%%%%%%%
%%%%%%%%%%%%
%%%%%%%%%%%%
%%%%%%%%%%%%
%%%%%%%%%%%%
%%%%%%%%%%%%$K=b^{s+t}X$, $L=b^{s+t-1}X$, so that
%%%%%%%%%%%%\[K_q=(T^{q+1})^{s+t}X_q\qquad L_q=(T^{q+1})^{s+t-1}X_q\]
%%%%%%%%%%%%writing $\frakd_{iq}$ for $(T^{i}\rho T^{q-i})$ and $\fraks_{iq}$ for $(T^{i+1}\imath T^{q-i})$,
%%%%%%%%%%%%\[\xymatrix@R=2mm@C=65mm{
%%%%%%%%%%%%\makebox[0cm][r]{$\trip{d}{iq}{K}:\,$}K_q
%%%%%%%%%%%%\ar[r]^-{(\frakd_{iq})^{s+t}\trip{d}{iq}{X}}
%%%%%%%%%%%%&
%%%%%%%%%%%%K_{q-1}\\
%%%%%%%%%%%%\makebox[0cm][r]{$\trip{s}{iq}{K}:\,$}K_q
%%%%%%%%%%%%\ar[r]^-{(\fraks_{iq})^{s+t}\trip{s}{iq}{X}}
%%%%%%%%%%%%&
%%%%%%%%%%%%K_{q+1}\\
%%%%%%%%%%%%\makebox[0cm][r]{$\trip{d}{iq}{L}:\,$}L_q
%%%%%%%%%%%%\ar[r]^-{(\frakd_{iq})^{s+t-1}\trip{d}{iq}{X}}
%%%%%%%%%%%%&
%%%%%%%%%%%%L_{q-1}\\
%%%%%%%%%%%%\makebox[0cm][r]{$\trip{s}{iq}{L}:\,$}L_q
%%%%%%%%%%%%\ar[r]^-{(\fraks_{iq})^{s+t-1}\trip{s}{iq}{X}}
%%%%%%%%%%%%&
%%%%%%%%%%%%L_{q+1}\\
%%%%%%%%%%%%\makebox[0cm][r]{$\trip{h}{jq}{}:\,$}K_q
%%%%%%%%%%%%\ar[r]^-{(\fraks_{jq})^{s-1}T^{j+1}\rho^{q}T^{q-j+1}(\fraks_{jq})^{t-1}\trip{s}{jq}{X}}
%%%%%%%%%%%%&
%%%%%%%%%%%%L_{q+1}\\
%%%%%%%%%%%%}\]
%%%%%%%%%%%%
%%%%%%%%%%%%\newpage
%%%%%%%%%%%%
%%%%%%%%%%%%writing $\frakd_{iq}$ for $(T^{i}\rho T^{q-i})$ and $\fraks_{iq}$ for $(T^{i+1}\imath T^{q-i})$,
%%%%%%%%%%%%
%%%%%%%%%%%%
%%%%%%%%%%%%\begin{align}
%%%%%%%%%%%% d_id_j&=d_{j-1}d_i\quad \text{ if $i<j$}, \notag\\
%%%%%%%%%%%% d_i s_j &= s_{j-1} d_i\quad\text{ if $i < j$},\notag \\
%%%%%%%%%%%%d_j s_j &= d_{j+1} s_j=\text{id},\label{E: axioms}\\
%%%%%%%%%%%% d_i s_j &= s_j d_{i-1}\quad  \text{ if $i > j + 1$},\notag\\
%%%%%%%%%%%%  s_is_j&=s_{j+1}s_i\quad \text{ if $i\leq j$}.\notag
%%%%%%%%%%%%\end{align}
%%%%%%%%%%%%
%%%%%%%%%%%%\subsection*{identities}
%%%%%%%%%%%%
%%%%%%%%%%%%
%%%%%%%%%%%%We must check, for $0\leq i<j\leq q$ that $\trip{d}{i,q+1}{L}\trip{h}{jq}{}=\trip{h}{j-1,q-1}{}\trip{d}{i,q}{K}$
%%%%%%%%%%%%\[\xymatrix@R=5mm@C=65mm{
%%%%%%%%%%%%K_q
%%%%%%%%%%%%\ar[r]^-{\trip{h}{jq}{}}_-{(\fraks_{jq})^{s-1}T^{j+1}\rho^{q}T^{q-j+1}(\fraks_{jq})^{t-1}\trip{s}{jq}{X}}
%%%%%%%%%%%%\ar[d]_-{(\frakd_{iq})^{s+t}\trip{d}{iq}{X}}
%%%%%%%%%%%%&
%%%%%%%%%%%%L_{q+1}
%%%%%%%%%%%%\ar[d]^-{(\frakd_{i,q+1})^{s+t-1}\trip{d}{i,q+1}{X}}
%%%%%%%%%%%%\\
%%%%%%%%%%%%K_{q-1}
%%%%%%%%%%%%\ar[r]_-{\trip{h}{j-1,q-1}{}}^-{(\fraks_{j-1,q-1})^{s-1}T^{j}\rho^{q-1}T^{q-j+1}(\fraks_{j-1,q-1})^{t-1}\trip{s}{j-1,q-1}{X}}
%%%%%%%%%%%%&
%%%%%%%%%%%%L_{q}\\
%%%%%%%%%%%%}\]
%%%%%%%%%%%%
%%%%%%%%%%%%
%%%%%%%%%%%%For $0\leq j\leq q-1$, we need $\trip{d}{j+1,q+1}{L}$ to coequalize $\trip{h}{j,q}{}$ and $\trip{h}{j+1,q}{}$.
%%%%%%%%%%%%\[\xymatrix@R=5mm@C=65mm{
%%%%%%%%%%%%K_q
%%%%%%%%%%%%\ar@<.5ex>[r]^-{\trip{h}{j,q}{}=(\fraks_{jq})^{s-1}T^{j+1}\rho^{q}T^{q-j+1}(\fraks_{jq})^{t-1}\trip{s}{jq}{X}}
%%%%%%%%%%%%\ar@<-.5ex>[r]_-{\trip{h}{j+1,q}{}=(\fraks_{j+1,q})^{s-1}T^{j+2}\rho^{q}T^{q-j}(\fraks_{j+1,q})^{t-1}\trip{s}{j+1,q}{X}}
%%%%%%%%%%%%&
%%%%%%%%%%%%L_{q+1}
%%%%%%%%%%%%\ar[d]^-{(\frakd_{j+1,q+1})^{s+t-1}\trip{d}{j+1,q+1}{X}}
%%%%%%%%%%%%\\
%%%%%%%%%%%%&
%%%%%%%%%%%%L_{q}\\
%%%%%%%%%%%%}\]
%%%%%%%%%%%%We must check, for $0\leq j<i-1\leq q$ that $\trip{d}{i,q+1}{L}\trip{h}{jq}{}=\trip{h}{j,q-1}{}\trip{d}{i-1,q}{K}$
%%%%%%%%%%%%\[\xymatrix@R=5mm@C=65mm{
%%%%%%%%%%%%K_q
%%%%%%%%%%%%\ar[r]^-{\trip{h}{jq}{}}_-{(\fraks_{jq})^{s-1}T^{j+1}\rho^{q}T^{q-j+1}(\fraks_{jq})^{t-1}\trip{s}{jq}{X}}
%%%%%%%%%%%%\ar[d]_-{(\frakd_{i-1,q})^{s+t}\trip{d}{i-1,q}{X}}
%%%%%%%%%%%%&
%%%%%%%%%%%%L_{q+1}
%%%%%%%%%%%%\ar[d]^-{(\frakd_{i,q+1})^{s+t-1}\trip{d}{i,q+1}{X}}
%%%%%%%%%%%%\\
%%%%%%%%%%%%K_{q-1}
%%%%%%%%%%%%\ar[r]_-{\trip{h}{j,q-1}{}}^-{(\fraks_{j,q-1})^{s-1}T^{j+1}\rho^{q-1}T^{q-j}(\fraks_{j,q-1})^{t-1}\trip{s}{j,q-1}{X}}
%%%%%%%%%%%%&
%%%%%%%%%%%%L_{q}\\
%%%%%%%%%%%%}\]
%%%%%%%%%%%%If $0\leq i\leq j\leq q$, then $\trip{s}{i,q+1}{L}\trip{h}{jq}{}=\trip{h}{j+1,q+1}{}\trip{s}{i,q}{K}$
%%%%%%%%%%%%\[\xymatrix@R=5mm@C=65mm{
%%%%%%%%%%%%K_q
%%%%%%%%%%%%\ar[r]^-{\trip{h}{jq}{}}_-{(\fraks_{jq})^{s-1}T^{j+1}\rho^{q}T^{q-j+1}(\fraks_{jq})^{t-1}\trip{s}{jq}{X}}
%%%%%%%%%%%%\ar[d]_-{(\fraks_{iq})^{s+t}\trip{s}{iq}{X}}
%%%%%%%%%%%%&
%%%%%%%%%%%%L_{q+1}
%%%%%%%%%%%%\ar[d]^-{(\fraks_{i,q+1})^{s+t-1}\trip{s}{i,q+1}{X}}
%%%%%%%%%%%%\\
%%%%%%%%%%%%K_{q+1}
%%%%%%%%%%%%\ar[r]_-{\trip{h}{j+1,q+1}{}}^-{(\fraks_{j+1,q+1})^{s-1}T^{j+2}\rho^{q+1}T^{q-j+1}(\fraks_{j+1,q+1})^{t-1}\trip{s}{j+1,q+1}{X}}
%%%%%%%%%%%%&
%%%%%%%%%%%%L_{q+2}\\
%%%%%%%%%%%%}\]
%%%%%%%%%%%%If $0\leq j<i\leq q+1$, then $\trip{s}{i,q+1}{L}\trip{h}{jq}{}=\trip{h}{j,q+1}{}\trip{s}{i-1,q}{K}$
%%%%%%%%%%%%\[\xymatrix@R=5mm@C=65mm{
%%%%%%%%%%%%K_q
%%%%%%%%%%%%\ar[r]^-{\trip{h}{jq}{}}_-{(\fraks_{jq})^{s-1}T^{j+1}\rho^{q}T^{q-j+1}(\fraks_{jq})^{t-1}\trip{s}{jq}{X}}
%%%%%%%%%%%%\ar[d]_-{(\fraks_{i-1,q})^{s+t}\trip{s}{i-1,q}{X}}
%%%%%%%%%%%%&
%%%%%%%%%%%%L_{q+1}
%%%%%%%%%%%%\ar[d]^-{(\fraks_{i,q+1})^{s+t-1}\trip{s}{i,q+1}{X}}
%%%%%%%%%%%%\\
%%%%%%%%%%%%K_{q+1}
%%%%%%%%%%%%\ar[r]_-{\trip{h}{j,q+1}{}}^-{(\fraks_{j,q+1})^{s-1}T^{j+1}\rho^{q+1}T^{q-j+2}(\fraks_{j,q+1})^{t-1}\trip{s}{j,q+1}{X}}
%%%%%%%%%%%%&
%%%%%%%%%%%%L_{q+2}\\
%%%%%%%%%%%%}\]

\begin{SteenrodAlgebrasAndTheirKoszulDuals}
\section{Some homogeneous Koszul algebras}
In what follows, we will make heavy use of four algebras, each of which is a `homogeneous Koszul algebra' in the sense of \cite{PriddyKoszul.pdf}. The algebras of interest are the following:
\begin{enumerate}\squishlist
\setlength{\parindent}{.25in}
\item The Steenrod algebra for simplicial commutative algebras, $\CommSteen$, with generators $P^i$ for $i\geq2$, subject to relations:
\[P^iP^jx=\sum_{s=i-j+1}^{i+j-2}{2s-i-1\choose s-j}P^{i+j-s}P^sx\textup{ for $i\geq2j$}.\]
That this is in fact a description of $\CommSteen$ follows from Goerss' determination \cite[p.14]{MR1089001} of the cohomology of Eilenberg-Mac Lane objects in simplicial augmented commutative $\F_2$-algebras.

\item The ``$\Delta$-algebra'' $\deltaAlgebra$, with generators $\delta_i$ for $i\geq2$, subject to relations:
\[\delta_i\delta_j:=\sum_{s=(i+1)/2}^{(i+j)/3}{j+s-i-1\choose j-s}\delta_{i+j-s}\delta_s\textup{ for $i<2j$.}\]
\item The Steenrod algebra for simplicial restricted Lie algebras, $\LieSteen$, with generators $\SqShift^i$ for $i\geq0$, subject to relations:
\[\SqShift^i\SqShift^j=\sum_{k=0}^{(i-1)/2}{j-k-1\choose i-2k-1}\SqShift^{i+j-k}\SqShift^k\textup{\ for $i\leq2j$.}\]
According to \cite[\S7]{PriddySimplicialLie.pdf}, $\LieSteen$ is the quotient of the standard Steenrod algebra by the two-sided ideal generated by $\Sq^0$, and our description follows after introducing the notation $\SqShift^i:=\Sq^{i+1}$.
%
%This algebra acts on the Andr\'e-Quillen cohomology of restricted Lie algebras, as follows. For $\frakg$ a simplicial Lie algebra,

\item The Dyer-Lashov algebra $\DyerLashov$, with generators $\Q^i$ for $i\geq0$, subject to relations:
\[\Q^i\Q^j=\sum_{k=0}^{(i-2j)/2-1}{i-2j-2-k\choose k}\Q^{2j+1+k}\Q^{i-j-1-k}\textup{\ for $i>2j$.}\]
This algebra is the opposite of the $\Lambda$-algebra, under an anti-isomorphism defined by $Q^i\longleftrightarrow \lambda_i$.
\end{enumerate}
\subsection{Unstable actions on homotopy and Quillen cohomology}
In what follows, let $A$ be a simplicial augmented commutative $\F_2$-algebra, and let $\frakg$ be a simplicial restricted Lie algebra over $\F_2$.

\subsubsection{$\CommSteen$ acting on the cohomology of $\F_2$-algebras}
%The Andr\'e-Quillen cohomology groups $H^*_Q(A)$ of a simplicial $\F_2$-algebra $A$ support the following structure:
\begin{enumerate}\squishlist
\setlength{\parindent}{.25in}
\item $H^*_Q(A)$ is a (bad) Lie algebra, under a bracket $H^i_Q(A)\otimes H^j_Q(A)\to H^{i+j+1}_Q(A)$; the structure is restricted in dimension zero (via $\beta:H^0_Q(A)\to H^1_Q(A)$), good in dimension 1, and bad otherwise;
\item $H^*_Q(A)$ is a graded left $\CommSteen$-module, so that there are operations $P^i:H^n_Q(A)\to H^{n+i+1}_Q(A)$, defined for all $n\geq0$ and $i\geq2$, but zero when $i>n$. These operations satisfy the $P$-Adem relations.
\item Every operation $P^i$ is linear, including the top operation $P^n:H^n_Q(A)\to H^{2n+1}_Q(A)$, which equals the self-bracket:  $P^nx=[x,x]$ if $|x|=n$.
\item The $P$ operations satisfy the following Cartan formula:
$[x,P^iy]=0$.
\end{enumerate}

\subsubsection{$\LieSteen$ acting on the cohomology of restricted Lie algebras}
%The Andr\'e-Quillen cohomology groups $H^*_Q(\frakg)$ of $\frakg$ support the following structure:
\begin{enumerate}\squishlist
\setlength{\parindent}{.25in}
\item $H^*_Q(\frakg)$ is a commutative algebra without unit, under a cup product $H^i_Q(\frakg)\otimes H^j_Q(\frakg)\to H^{i+j+1}_Q(\frakg)$;
\item $H^*_Q(\frakg)$ is a graded left $\LieSteen$-module, so that there are operations $\SqShift^i:H^n_Q(\frakg)\to H^{n+i+1}_Q(\frakg)$, defined for all $n\geq0$ and $i\geq0$, but zero when $i>n$. These operations satisfy the $\SqShift$-Adem relations.
\item Every operation $\SqShift^i$ is linear, including the top operation $\SqShift^n:H^n_Q(A)\to H^{2n+1}_Q(A)$, which equals the squaring operation:  $\SqShift^nx=x^2$ if $|x|=n$.
\item The $\SqShift$ operations satisfy the following Cartan formula:
\[\SqShift^k(x\cdot y)=\sum_{k'+k''=k-1}\SqShift^{k'}(x)\cdot \SqShift^{k''}(y).\]
\end{enumerate}

\subsubsection{$\Delta$ operations on the homotopy of $\F_2$-algebras}
%The homotopy groups $\pi_*(A)$ of $A$ support the following structure:
\begin{enumerate}\squishlist
\setlength{\parindent}{.25in}
\item $\pi_*(A)$ is a commutative algebra under a product $\pi_i(A)\otimes \pi_j(A)\to \pi_{i+j}(A)$; it is exterior in dimension 1, and it is a divided power algebra in dimensions 2 and above;
\item There are operations $\delta_i:\pi_n(A)\to \pi_{n+i}(A)$, defined only for $2\leq i\leq n$. 
As the operations $\delta_i$ are only partially defined, $\pi_*(A)$ does \emph{not} become a left module over the $\Delta$-algebra.
 However, when these operations are written on the left, they satisfy the $\delta$-Adem relations whenever one can be applied.
\item The non-top operations, $\delta_i:\pi_n(A)\to \pi_{n+i}(A)$ for $i<n$, are linear. The top operation, $\delta_n:\pi_n(A)\to \pi_{2n}(A)$ equals the divided square, and is thus quadratic:
\[\textup{if $|x|=|y|=n\geq2$, then }\delta_n(x)=\gamma_2(x)\textup{ and }\delta_n(x+y)=\delta_n(x)+\delta_n(y)+xy.\]
\item The $\delta$ operations satisfy the following Cartan formula:
\[\delta_i(xy)=\begin{cases}
x^2\delta_i(y),&\textup{if }|x|=0\textup{ and }|y|\geq2;\\
y^2\delta_i(x),&\textup{if }|y|=0\textup{ and }|x|\geq2;\\
0,&\textup{otherwise}.
%\\,&\textup{if }
\end{cases}
\]
\end{enumerate}
\subsubsection{$\Q$ operations on the homotopy of restricted Lie algebras}
%The homotopy groups $\pi_*(\frakg)$ of $\frakg$ support the following structure:
\begin{enumerate}\squishlist
\setlength{\parindent}{.25in}
\item $\pi_*(\frakg)$ is a restricted Lie algebra under a bracket $\pi_i(\frakg)\otimes \pi_j(\frakg)\to \pi_{i+j}(\frakg)$;
\item There are operations $\Q^i:\pi_n(\frakg)\to \pi_{n+i}(\frakg)$, defined only for $0\leq i\leq n$. As the operations $\Q^i$ are only partially defined, $\pi_*(\frakg)$ does \emph{not} become a left module over the Dyer-Lashov algebra. However, when these operations are written on the left, they satisfy the $\Q$-Adem relations whenever one can be applied.
\item The non-top operations, $\Q^i:\pi_n(\frakg)\to \pi_{n+i}(\frakg)$ for $i<n$, are linear. The top operation, $\Q^n:\pi_n(\frakg)\to \pi_{2n}(\frakg)$ equals the restriction, and is thus quadratic:
\[\textup{if $|x|=|y|=n$, then }\Q^n(x)=\restn{(x)}\textup{ and }\Q^n(x+y)=\Q^n(x)+\Q^n(y)+[x,y].\]
\item The $\Q$ operations satisfy the following Cartan formula. For $|x|\leq |y|$:
\[[\Q^i(x),y]=\begin{cases}
[x,[x,y]],&\textup{if }i=|x|;\\
0,&\textup{otherwise}.
%\\,&\textup{if }
\end{cases}
\]
\end{enumerate}

Note that it is not immediately obvious that the $\delta$-Adem ($\Q$-Adem) relations are well defined when viewed as relations that hold between operations on the homotopy of commuatative algebras (Lie algebras). That is, we are supposed to have $\Q^i\Q^jx=\cdots $ whenever $i>2j$ and $\Q^i\Q^jx$ is defined, but we haven't checked that every term in the right hand side is defined. This does indeed happen, by lemma \ref{lemOnAdemChangeInM} (to follow). 

\subsection{The combinatorics of admissible sequences}

Now each of these algebras, being homogeneous, is bigraded%footnote:
\footnote{I'm not sure what the point is in bigrading things, really, but there's some cool stuff happening. For example, if you make $P$ act on something that's only singly graded, it makes sense to demand that degrees change via the associtated singly graded algebra, in which $|P^i|=i+1$, and viewing $P$ as bigraded makes the unstable condition seem more natural. Moreover, whenever you see a $\delta_j$ in this context, it does change internal degree by $i$, and homological degree by $1$. What it does to deeper degrees is just spooky. I guess the same thing holds for $\SqShift^i$, which is intense, given that it equals $\Sq^{i+1}$! I don't know.}, setting (\textbf{I guess}) 
\[\|P^i\|=\|\delta_i\|=\|\SqShift^i\|=\|\Q^i\|=(1,i).\]
Now let $G$ be any of the symbols $P$, $\delta$, $\SqShift$, or $\Q$. For a sequence $I=(i_{\ell},\ldots,i_1)$ of positive integers (with $i_j\geq2$ if $G$ is $P$ or $\delta$), write $G^I$ for the product $G^{i_\ell}\cdots G^{i_1}$. Say that $I$ is $G$-admissible if no $G$-Adem relations can be applied directly to $G^I$, so that one of the four evident inequalities hold between adjacent indices in $I$. We'll write $\produces{I}{J}{G}$ if $I\neq J$, and when $G^I$ is written as a sum of $G$-admissible expressions, $G^J$ appears therein (an odd number of times). Trivially, if $\produces{I}{J}{G}$, then $J$ is $G$-admissible, and $I$ is $G$-inadmissible.
\begin{lem}
All of the Adem relations listed above express the inadmissible length 2 sequence on the left hand side as a sum of admissible length 2 sequences. All four of the algebras are homogeneous Koszul algebras. The algebras $\CommSteen$ and $\Delta$ are Koszul dual, under $P^i\longleftrightarrow\delta_i$. The algebras $\LieSteen$ and $R$ are Koszul dual, under $\SqShift^i\longleftrightarrow\Q^i$. In particular, for length 2 sequences $K$ and $L$:
\[\produces{K}{L}{P}\iff\produces{L}{K}{\delta}\textup{ \ and \ }\produces{K}{L}{\SqShift}\iff\produces{L}{K}{\Q}\]

\end{lem}
\begin{proof}
The first statement is checked manually from the Adem relations in each case.

Priddy's PBW stuff, along with the fact that none of the above Adem relations do any more than expected, shows that everything is Koszul. To check that the algebras are indeed Koszul dual to each other, one just looks closely at the Adem relations and checks the above `$\iff$' statements
\end{proof}
Thus, for all four of these algebras, the $G$-Adem relations can be used iteratively to write any expression as a sum of terms $G^I$ with $I$ $G$-admissible. 

It will be important to have a criterion for when $\delta_Ix$ is defined for an element $x\in\pi_*(A)$, or when $P^Ix$ can be nonzero, for $x\in H^*_Q(A)$, etc.
To answer such questions, we make the following two definitions. For $I=(i_\ell,\ldots,i_1)$ a sequence of nonnegative integers, we define
\[\minDim(I):=\begin{cases}
-\infty,&\textup{if }I=\emptyset;\\
\max\{(i_1),\,(i_2-i_1),\,\ldots,\,(i_{\ell}-i_{\ell-1}-\cdots -i_1)\},&\textup{otherwise};
%\\,&\textup{if }
\end{cases}
\]
\[\minDimP(I):=\begin{cases}
-\infty,&\textup{if }I=\emptyset;\\
\max\{(i_1),\,(i_2-i_1-1),\,(i_3-i_2-i_1-2),\,\ldots,\,(i_{\ell}-\cdots-i_1-\ell+1)\},&\textup{otherwise}.
%\\,&\textup{if }
\end{cases}
\]
If $I=(i_{\ell},\ldots,i_1)$ with all $i_j\geq2$, then \textbf{rewrite?}:
\begin{itemize}
\setlength{\parindent}{.25in}
\squishlist
\item $\delta_Ix$ is defined for $x\in \pi_n(A)$ iff $\minDim(I)\leq n$;
\item if $I\in\admis{P}$, then $P^Ix$ can be nonzero for $x\in H^n_Q(A)$ iff $\minDimP(I)\leq n$.
\end{itemize}
If $I=(i_{\ell},\ldots,i_1)$ with all $i_j\geq0$, then:
\begin{itemize}
\setlength{\parindent}{.25in}
\squishlist
\item $\delta_Ix$ is defined for $x\in \pi_n(\frakg)$ iff $\minDim(I)\leq n$;
\item if $I\in\admis{\SqShift}$, then $\SqShift^Ix$ can be nonzero for $x\in H^n_Q(\frakg)$ iff $\minDimP(I)\leq n$. (\textbf{?})
\end{itemize}

%With this notation in hand, we may finally discuss whether or not the Adem relations that the $\delta$ operations and and $\Delta$ are well defined.
\begin{lem}\label{lemOnAdemChangeInM}
In the following statements, $K$ and $L$ are each length 2 sequences:%, with entries nonnegative in cases (i) and (ii), and entries at least 2 in cases (iii) and (iv).
%
%In the following statements, $K=(i,j)$ and $L=(i',j')$ are to be two length 2 sequences, with entries nonnegative in cases (i) and (ii), and entries at least 2 in cases (iii) and (iv).
\begin{enumerate}[i)]
\setlength{\parindent}{.25in}
\item Suppose that $\produces{K}{L}{\Q}$. Then $\minDim(L)<\minDim(K)$, and $L$ has positive entries.
%\item Suppose that $\produces{K}{L}{\SqShift}$. Then $\minDim(L)>\minDim(K)$, and $K$ has positive entries. (shrug)
\item Suppose that $\produces{K}{L}{P}$. Then $\minDimP(L) > \minDimP(K)$.
%\item Suppose that $\produces{K}{L}{\delta}$. Then $\minDimP(L) < \minDimP(K)$. (shrug)
\end{enumerate}
In (ii), the entries of $K$ and $L$ should be at least 2.
\end{lem}
\begin{proof}
In both of the proofs, let $K=(i,j)$ and $L=(i',j')$.

For (i), we wish to show that $\max\{j',i'-j'\}<\max\{j,i-j\}$. As $i'\leq 2j'$ and $i>2j$, this is to show that $j'<i-j$ which is immediate from the indexing of the $\Q$-Adem relation. That $L$ has positive entries is also immediate.

For (ii), we wish to show that $\max\{j',i'-j'-1\}>\max\{j,i-j-1\}$. 
As $i\geq2j$, $\max\{j,i-j-1\}\leq i-j$, and $i-j<j'$ by inspection of the $P$-Adem relation.
%As $i'< 2j'$ and $i\geq2j$, it happens that $\max\{j,i-j-1\}$ is at most $i-j$ (one distinguishes the cases $i=2j$ and $i>2j$). Thus, we need only show that $j'>i-j$, which is again immediate from the Adem relation.
\end{proof}
%\begin{lem}
%Suppose that %$i>2j$, and 
%$\produces{(i,j)}{(i',j')}{\Q}$. Then $\minDim(i,j)>
%\minDim(i',j')$, and $i',j'>0$.
%
%In particular, if $\Q^i\Q^jx$ is defined, and $i>2j$, it can be rewritten as a sum of terms $\Q^{i'}\Q^{j'}x$ in which neither of the $\Q^{i'}$ and $\Q^{j'}$ that appear are top operations.
%\end{lem}
%\begin{proof}
%It must be that $(i,j)$ is $\Q$-inadmissible, so that $i>2j$. Then,
%by inspection, $i>j+i/2\geq i'\geq2j+1$, and $i-j-1\geq j'\geq i/2>j$. These inequalities demonstrate (i). For (ii), note that (i) shows that we needn't worry about $\Q^0$ failing to be defined, as it won't appear. Our only concern is that either $i'$ or $j'$ will be too large. Suppose then that $\Q^i\Q^j$ is defined on an $n$-dimensional class, so that $j\leq n$ and $i\leq n+j$. Then one notes that $j'\leq i-j-1\leq n-1$, and $i'-j'\leq (j+i/2)-(i/2)<i/2\leq(n+j)/2\leq n$. The first chain gives $j'\leq n-1$, and the second $i'\leq n-1+j'$, as the quantities involved are integers. These inequalities demonstrate (ii) and the final comment.
%\end{proof}
%
%\begin{lem}
%Suppose that $\produces{(i,j)}{(i',j')}{P}$. Then $\minDimP(i',j') > \minDimP(i,j)$.
%\end{lem}
%\begin{proof}
%Since $i\geq2j$, 
%$j'\geq i-j+1=
%j+(i-2j+1)=
%i-j-1+(2)
%$, so that $j'>\max\{j,i-j-1\}$.
%\end{proof}


%
%\subsection{The Steenrod algebra for simplicial Lie algebras and its Koszul dual}
%Let  $\LieSteen$ be the Steenrod algebra for simplicial restricted Lie algebras \cite[\S7]{PriddySimplicialLie.pdf}. This is the quotient of the standard Steenrod algebra by the two-sided ideal generated by $\Sq^0$. We'll introduce the notation $\SqShift^i:=\Sq^{i+1}$, so that $\LieSteen$ is generated by $\SqShift^i$ for $i\geq-1$, subject to relations:
%\[\SqShift^{-1}=0\textup{ and }\SqShift^i\SqShift^j=\sum_{k=0}^{(i-1)/2}{j-k-1\choose i-2k-1}\SqShift^{i+j-k}\SqShift^k\textup{\ for $i\leq2j$.}\]
%This is a homogeneous Koszul algebra in the sense of \cite{PriddyKoszul.pdf}. Its cohomology algebra $H^*\LieSteen:=\Ext_{\LieSteen}^{**}(\F_2,\F_2)$ is also a homogeneous Koszul algebra, generated by the $\Q^i$ for $i\geq0$, where $\Q^i$ is dual to $\SqShift^i$ \cite[\S8]{PriddySimplicialLie.pdf}. These classes satisfy their own Adem relations%footnote:
%\footnote{Compare with \cite[p.143]{Miller-infDeloopSS.pdf}, and May-Cohen-Lada page 9.}:
%\[\Q^i\Q^j=\sum_{k=0}^{(i-2j)/2-1}{i-2j-2-k\choose k}\Q^{2j+1+k}\Q^{i-j-1-k}\textup{\ for $i>2j$.}\]
%Importantly, $H^*\LieSteen$ is isomorphic to the opposite of the $\Lambda$-algebra, and $\Q^i\longleftrightarrow\lambda_i$ under this anti-isomorphism. While it is not precisely true that the $\Lambda$-algebra has an unstable right action on the homotopy of any simplicial Lie algebra, there are various $\lambda$ operators acting thereupon, satisfying the Adem relations in the $\Lambda$-algebra when written on the right \cite[Prop 8.6]{CurtisSimplicialHtpy.pdf}. We prefer to think of this as a left action of various $\Q$ operators, in order to simplify notation. Details to follow.
%
%\subsection{The Steenrod algebra for simplicial commutative algebras and its Koszul dual}
%It follows from Goerss' determination \cite[p.14]{MR1089001} of the cohomology of Eilenberg-Mac Lane objects in simplicial augmented commutative $\F_2$-algebras that the corresponding Steenrod algebra, $\CommSteen$ may be described thus.
%Let $\CommSteen$ be the algebra generated by $P^i$ for $i\geq0$ subject to relations:
%\[P^0=0,\ P^1=0\textup{ and }P^iP^jx=\sum_{s=i-j+1}^{i+j-2}{2s-i-1\choose s-j}P^{i+j-s}P^sx\textup{ for $i\geq2j$}.\]
%The Quillen cohomology of a simplicial augmented $\F_2$-algebra has an unstable left action of this algebra. This is also a homogeneous Koszul algebra, as is its cohomology algebra $H^*\CommSteen=\deltaAlgebra$. The algebra $\Delta$ is generated by classes $\delta_i$ dual to $P^i$ for $i\geq 2$. These classes satisfy the following $\delta$-Adem relation:
%\[\delta_i\delta_j:=\sum_{s=(i+1)/2}^{(i+j)/3}{j+s-i-1\choose j-s}\delta_{i+j-s}\delta_s\textup{ for $i<2j$.}\]
%It is again not precisely true that $\deltaAlgebra$ acts on the homotopy of simplicial commutative algebras. There are, however, various $\delta$ operations acting on the homotopy of a simplicial commutative algebra, and they satisfy the $\delta$-Adem relation wherever it makes sense.
\end{SteenrodAlgebrasAndTheirKoszulDuals}

\begin{CategoriesOfInterest}
\section{Categories of interest}\label{ChapterCategoriesOfInterest}
In the following subsections we detail a number of algebraic categories relevant to our study.
\subsection{Categories of graded sets and vector spaces}
\begin{itemize}
\setlength{\parindent}{.25in}
\item For $n\geq1$, let $\GpS{n}$ be the category of $n$-times non-negatively graded pointed sets%footnote:
\footnote{The question `what should be the domain category for our free/forget adjunctions' is probably only relevant when using Dold to prove something about PR Lie homology. Once you've done PBW you're fine, and can show that U preserves weak equivalences.} $X$ whose graded part $X_{a_n,\ldots,a_1}$ is a single point whenever $a_1=0$. That is, an object $X\in\GpS{n}$ is a collection $X_{a_n,\ldots,a_1}$ of pointed sets, one for each $n$-tuple $(a_n,\ldots,a_1)$ of non-negative integers, such that $X_{a_n,\ldots,a_2,0}=\{\star\}$ for all $a_n,\ldots,a_2$. For $x\in X_{a_n,\ldots,a_1}$, we write $|x|:=a_n$ and $\|x\|:=(a_n,\ldots,a_1)$.
\item For $n\geq1$, let $\GS{n}$ be the category of $n$-times non-negatively graded sets $X$ whose graded part $X_{a_n,\ldots,a_1}$ is empty whenever $a_1=0$. For $x\in X_{a_n,\ldots,a_1}$, we write $|x|:=a_n$ and $\|x\|:=(a_n,\ldots,a_1)$

\item For $n\geq1$, let $\GR{n}$ be the category of $n$-times non-negatively graded $\F_2$-vector spaces $V$ such that $V_{a_n,\ldots,a_1}=0$ whenever $a_1=0$.  For a nonzero homogeneous element $v\in V_{a_n,\ldots,a_1}$, we write $|v|:=a_n$ and $\|v\|:=(a_n,\ldots,a_1)$.
\end{itemize}
All the categories $\mathsf{C}=\mathsf{C}^n$ to be defined below will consist of objects $X$ of $\GR{n}$ enriched with certain extra structure. For $x\in X_{a_n,\ldots,a_1}$ a non-zero homogeneous element of $X$, we will continue to write $|x|:=a_n$ and $\|x\|:=(a_n,\ldots,a_1)$ in each of these contexts.
Moreover, for any of the following categories there are free/forget adjunctions $\GS{n} \rightleftarrows \mathsf{C}$ and $\GpS{n} \rightleftarrows \mathsf{C}$. We'll denote either of these free functors simply by $X\mapsto\Fr{\mathsf{C}}(X)$.


\subsection{Categories of graded Lie algebras}

\begin{itemize}
\setlength{\parindent}{.25in}
\item For $n\geq1$, let $\BadLie{n}$ be the following category of $n$-graded bad Lie algebras over $\F_2$. An object of $\BadLie{n}$ is to be an object $L$ of $\GR{n}$ with skew-symmetric%footnote:
\footnote{As the ground field has characteristic 2, the concepts of symmetry and skew-symmetry coincide.} maps
\[[\DASH,\DASH]:L_{a_n,\ldots,a_1}\otimes L_{b_n,\ldots,b_1}\to L_{a_n+b_n,\ldots,a_2+b_2,a_1+b_1+1}\]
satisfying the Jacobi identity. Note that this Lie bracket has degree $(0,\ldots,0,1)$. Importantly, skew-symmetry does not imply the `alternating' relation, so that there may be $x\in L$ with $[x,x]$ nonzero. This explains the $\mathsf{b}$ in the notation --- we refer to such Lie algebras as \emph{bad} Lie algebras.
Note that the `self-bracket' operation $x\mapsto [x,x]$ is linear in any bad Lie algebra.
\item For $n\geq1$, let $\GoodLie{n}$ be the full subcategory of $\BadLie{n}$ consisting of those Lie algebras which are \emph{good}. That is, those $L\in\BadLie{n}$ for which $[x,x]=0$ for all $x\in L$.
\item For $n\geq1$, let $\RestLie{n}$ be the category of restricted Lie algebras in $\GoodLie{n}$. That is, an object of $\RestLie{n}$ is an object $L\in\GoodLie{n}$ along with `restriction' operations
\[\restn{}:L_{a_n,\ldots,a_1}\to L_{2a_n,2a_{n-1},\ldots,2a_2,2a_1+1}\]
which satisfy, for $x,y\in L_{a_n,\ldots,a_1}$ and $z\in L$: \[\restn{(x+y)}=\restn{(x)}+\restn{(y)}+[x,y]\textup{ and }[\restn{x},z]=[x,[x,z]].\]
\item For $n\geq1$, let $\PRLie{n}$ be the following category of partially restricted $n$-graded \emph{(good)} Lie algebras. An object of $\PRLie{n}$ is an object $L$ of $\GoodLie{n}$
%For $n\geq1$, let $\PRLie{n}$ be the following category of partially restricted $n$-graded Lie algebras. An object of $\PRLie{n}$ is a good Lie algebra $L=\bigoplus L_{a_n,\ldots,a_1}$, under a $(0,\ldots,0,1)$-shifted bracket as above
admitting restriction operations
\[\restn{}:L_{a_n,\ldots,a_1}\to L_{2a_n,2a_{n-1},\ldots,2a_2,2a_1+1}\]
defined whenever not all of $a_n,\ldots,a_{2}$ are zero. %The graded part $L_{a_n,\ldots,a_1}$ is assumed to be zero whenever $a_1=0$, and we write $|x|$ as above. 
For $x,y\in L_{a_n,\ldots,a_1}$ and $z\in L$ the restriction operations satisfy \[\restn{(x+y)}=\restn{(x)}+\restn{(y)}+[x,y]\textup{ and }[\restn{(x)},z]=[x,[x,z]].\]
Note that when $n=1$ the restriction is not defined \emph{anywhere}, so that the categories $\PRLie{1}$ and $\GoodLie{1}$ are isomorphic.
\end{itemize}
We have defined the bracket and restrictions on homogeneous elements only, but there is no problem extending these operations to non-homogeneous elements using the bilinearity of the bracket and the above formula for the restriction of a sum.
%We can define $\restn{}$ on the direct sum of those $L_{a_n,\ldots,a_1}$ for which not all of $a_n,\ldots,a_2$ are zero, using the above formula for $\restn{(x+y)}$. Then both of the above two formulae will still hold for non-homogeneous $x,y,z$. Thus, we could equally have defined $\PRLie{n}$ by giving a restriction on this entire ideal.
%\begin{prop}
%If $X\in\GS{n}$ a graded set, the `Hall basis' of $\Fr{\GoodLie{n}}(X)$ \cite{MR0038336} consists of certain `standard monomials' $b$ in the elements of $X$. Then $\Fr{\BadLie{n}}(X)$ has a basis $\{b\}\cup\{[b,b]\}$.
%\end{prop}

If $X\in\GS{n}$ a graded set, one may choose a `Hall basis derived from $X$' \cite{MR0038336} for the free good Lie algebra $\Fr{\GoodLie{n}}(X)$ consisting of certain (homogeneous) `standard monomials' $b$ in the elements of $X$.

\begin{prop}\label{PropBasesOfFreeLieAlgs}
For $n\geq1$, suppose that $X\in\GS{n}$ is a graded set. Let $B$ be a Hall basis derived from $X$ for the free good Lie algebra $\Fr{\GoodLie{n}}(X)$. Note that $B\subset \Fr{\GoodLie{n}}(X)$ is in particular a graded set. Then
\begin{alignat*}{2}
%\Fr{\BadLie{n}}(X)
%&=
%\left\langle B\right\rangle\oplus \left\langle[b,b]\,:\,b\in B\right\rangle,%
%\\
\Fr{\RestLie{n}}(X)
&=
\left\langle \iteratedrestn{b}{r}\,:\,b\in
B,\,r\geq0\right\rangle\textup{, \ and}%
\\
\Fr{\PRLie{n}}(X)
&=
\left\langle \iteratedrestn{b}{r}\,:\,b\in
B,\,r\geq0,\textup{ where if }a_{n}\!=\!\cdots\!=\!a_2\!=\!0\textup{ then $r=0$}\right\rangle,%
\end{alignat*}
where by $a_n,\ldots,a_2$ we mean all but the rightmost index in $\|b\|=(a_{n},\ldots,a_1)$.
\end{prop}
%\begin{prop}
%Let $X\in\GS{n}$ be a graded set, and let $B$ be a Hall basis derived from $X$ for $\Fr{\GoodLie{n}}(X)$. Note that $B\subset \Fr{\GoodLie{n}}(X)$ is actually a graded set. Then:
%\begin{enumerate}[i)]\squishlist
%\setlength{\parindent}{.25in}
%\item $\Fr{\BadLie{n}}(X)$ has a basis $B\cup\{[b,b]\,:\,b\in B\}$;
%\item $\Fr{\RestLie{n}}(X)$ has a basis $\{\iteratedrestn{b}{r}\,:\,b\in B,\,r\geq0\}$; and
%\item $\Fr{\PRLie{n}}(X)$ has a basis $B\cup \{\iteratedrestn{b}{r}\,:\,b\in B_{a_{n},\ldots,a_1},\,r\geq1,\textup{ not all of $a_n,\ldots,a_2$ are zero}\}$.
%\end{enumerate}
%\end{prop}
\begin{proof}
%To calculate $\Fr{\BadLie{n}}(X)$,  note that the Jacobi identity implies that any self-bracket $[y,y]$ in a bad Lie algebra is central. Thus, the proposed basis spans $\Fr{\BadLie{n}}(X)$.  On the other hand, as one easily defines a Lie algebra structure on the vector space $\langle B\rangle\oplus \langle [b,b]\,:\,b\in B\rangle$ by declaring the $[b,b]$ to be central, all of the elements of $B$ and all of the $[b,b]$ are linearly independent in $\Fr{\BadLie{n}}(X)$.
The calculation of $\Fr{\RestLie{n}}(X)$ is standard \cite[Proposition 14, p.66]{MR886063}.
To calculate $\Fr{\PRLie{n}}(X)$, note that the proposed basis must span $\Fr{\PRLie{n}}(X)$, and can be seen to be linearly independent by considering the map $\Fr{\PRLie{n}}(X)\to \forget{\PRLie{}}(\Fr{\RestLie{n}}(X))$ of partially restricted Lie algebras.
\end{proof}


\subsection{Homotopy operations for simplicial Lie algebras}
The goal of this section is to determine the natural operations on the homotopy of a simplicial object in the category $\mathsf{C}$, for $\mathsf{C}=\mathsf{C}^n$ any of the above categories of graded Lie algebras. This is equivalent to calculating the homotopy of `wedges of spheres' in $s\mathsf{C}$.
That is, suppose that $X\in\GS{n+1}$ is a graded set. Then one can form a simplicial pointed graded set $S(X)\in s\GpS{n}$ by the formula
\[(S(X))_{a_n,\ldots,a_1}=\bigvee_{x\in X_{d,a_n,\ldots,a_1}}S^d_{a_n,\ldots,a_1},\]
where $S^d_{a_n,\ldots,a_1}$ is the standard simplicial $d$-sphere $\Delta^d/\partial\Delta^d$ lying in the $a_n,\ldots,a_1$ grading.
Then $\Fr{\mathsf{C}}S(X)\in s\mathsf{C}$ represents the functor $s\mathsf{C}\to\AbGp$ defined by
\[C\mapsto \prod_{x\in X_{d,a_n,\ldots,a_1}}\pi_d(C)_{a_n,\ldots,a_1},\]
where $\pi_d(C)_{a_n,\ldots,a_1}$ is the grading $a_n,\ldots,a_1$ part of $\pi_d(C)$. In particular, by Yoneda's lemma $+$ \textbf{something about taking homotopy classes}:
\[\left\{\left(\prod_{x\in X_{d,a_n,\ldots,a_1}}\pi_d(\DASH)_{a_n,\ldots,a_1}\right)\overset{\substack{\textup{nat}\\\textup{trans}}}{\to}\left(\pi_D(\DASH)_{A_n,\ldots,A_1}
\rule{0mm}{\heightof{\ensuremath{ \displaystyle \prod_{x\in X_{d,a_n,\ldots,a_1}}}}}
\right)\right\}\cong\pi_D\left(\Fr{\mathsf{C}}S(X)\right)_{A_n,\ldots,A_1}\]
Thus we're interested in the homotopy groups of $\mathsf{C}S(X)$. These are well known when $\mathsf{C}$ is $\GoodLie{n}$ or $\RestLie{n}$, and from these calculations we will deduce the result when $\mathsf{C}=\PRLie{n}$. Before stating the results, we recall certain well known operations on the homotopy of good and restricted Lie algebras, as discussed in \cite[Thm 7.12, Prop 8.8]{CurtisSimplicialHtpy.pdf}:
\begin{prop}
For $X$ a simplicial object either in $\GoodLie{n}$ or $\RestLie{n}$, $\pi_{*}(X)$ naturally becomes a good Lie algebra (in $\GoodLie{n+1}$) under a bracket induced by the shuffle map. Moreover, $\pi_{*}(X)$ has natural operations
\[\Q^i:\pi_{a_{n+1}}(X)_{a_n,\ldots,a_1}\to \pi_{a_{n+1}+i}(X)_{2a_n,\ldots,2a_2,2a_1+1}
\]
defined whenever $0<i\leq a_{n+1}$ and when $i=0$ if $X\in\RestLie{n}$. The following $\Q$-Adem relations hold whenever the left hand side is defined:
\[\Q^i\Q^jx=\sum_{k=0}^{(i-2j)/2-1}{i-2j-2-k\choose k}\Q^{2j+1+k}\Q^{i-j-1-k}x\textup{ for }i>2j.\]
The non-top operations are linear and are killed by the bracket, which is to say that for $x,y\in \pi_{a_{n+1}}(X)_{a_n,\ldots,a_1}$, $z\in \pi_*(X)$ and $i<a_{n+1}$: 
\[\Q^i{(x+y)}=\Q^i{(x)}+\Q^i{(y)}\textup{ and }[\Q^i{(x)},z]=0.\]
The top $\Q$ operation acts as a restriction for the Lie algebra structure whenever it is defined, so that, for $x,y\in \pi_i(X)_{ a_{n},\ldots,a_1}$ and $z\in \pi_*(X)$,
\[\Q^i(x+y)=\Q^i(x)+\Q^i(y)+[x,y]\textup{ and }[\Q^{i}(x),z]=[x,[x,z]].\]
\end{prop}
%%%%%
%%%%%
%%%%%\begin{prop}
%%%%%For $X$ a simplicial object either in $\GoodLie{n}$ or $\RestLie{n}$, $\pi_{*}(X)$ naturally becomes a good Lie algebra (in $\GoodLie{n+1}$) under a bracket induced by the shuffle map. Moreover, $\pi_{*}(X)$ has natural operations
%%%%%\[\Q^i:\pi_{a_{n+1}}(X)_{a_n,\ldots,a_1}\to \pi_{a_{n+1}+i}(X)_{2a_n,\ldots,2a_2,2a_1+1}
%%%%%\]
%%%%%defined whenever $0<i\leq a_{n+1}$ and when $i=0$ if $X\in\RestLie{n}$. The following relations hold between the various operations (whenever the left hand side of each relation is defined).
%%%%%%\[\textup{for $i>2j,$\ \ }\Q^i\Q^jx=\sum_{k=0}^{(i-2j)/2-1}{i-2j-2-k\choose k}\Q^{2j+1+k}\Q^{i-j-1-k}x.\]
%%%%%%\[\Q^i{(x+y)}=\Q^i{x}+\Q^i{y}\textup{ and }[\Q^i{x},z]=0\textup{ for $i<|x|=|y|$ and any $z$}\]
%%%%%%\[\Q^i(x+y)=\Q^ix+\Q^iy+[x,y]\textup{ and }[\Q^{i}x,z]=[x,[x,z]]\textup{ for $i=|x|=|y|$ and any $z$}.\]
%%%%%The $\Q$ operations satisfy the $\Q$-Adem relations:%, so that if $i>2j$, and $\Q^i\Q^jx$ is defined, then:
%%%%%\[\Q^i\Q^jx=\sum_{k=0}^{(i-2j)/2-1}{i-2j-2-k\choose k}\Q^{2j+1+k}\Q^{i-j-1-k}x\textup{ for }i>2j.\]
%%%%%The non-top operations are linear and are killed by the bracket.
%%%%%That is, for $x,y\in \pi_{a_{n+1}}(X)_{a_n,\ldots,a_1}$, $z\in \pi_*(X)$ and $i<a_{n+1}$: \[\Q^i{(x+y)}=\Q^i{x}+\Q^i{y}\textup{ and }[\Q^i{x},z]=0.\]
%%%%%On the other hand, the top $\Q$ operation, 
%%%%%acts as a (partially defined) restriction, so that, for $x,y\in \pi_i(X)_{ a_{n},\ldots,a_1}$ and $z\in \pi_*(X)$,
%%%%%\[\Q^i(x+y)=\Q^ix+\Q^iy+[x,y]\textup{ and }[\Q^{i}x,z]=[x,[x,z]].\]
%%%%%
%%%%%%
%%%%%%
%%%%%%
%%%%%%
%%%%%% The top $\Q^i$ acts as a restriction operation, so that $\Q^{|x|}(x)=\restn(x)$ whenever both sides are defined. The non-top operations are linear and are killed by the bracket.
%%%%%%That is, for $x,y\in L_{a_n,\ldots,a_1}$, $i<a_n$ (with not all of $i,a_{n-1},\ldots,a_{2}$ being zero) and $z\in L$: \[\Q^i{(x+y)}=\Q^i{x}+\Q^i{y}\textup{ and }[\Q^i{x},z]=0.\]
%%%%%%Finally, the $\Q$ operations are assumed to satisfy the following $\Q$-Adem relations wherever they make sense. If $i>2j$, and $\Q^i\Q^jx$ is defined (i.e.\ $j\leq|x|$, not all of $j,a_{n-1},\ldots,a_2$ are zero, and $i\leq|x|+j$):
%%%%%%\[\Q^i\Q^jx=\sum_{k=0}^{(i-2j)/2-1}{i-2j-2-k\choose k}\Q^{2j+1+k}\Q^{i-j-1-k}x.\]
%%%%%\end{prop}

\noindent Again, it is not immediately obvious that $\Q$-Adem relations are well defined, but this follows from lemma \ref{lemOnAdemChangeInM}, even with the restrictions on $\Q^0$. 
%\begin{lem}
%Suppose that %$i>2j$, and 
%$\produces{(i,j)}{(i',j')}{\Q}$. Then
%\begin{enumerate}[i)]\squishlist
%\setlength{\parindent}{.25in}
%\item Both $i'$ and $j'$ are nonzero.
%\item If $\Q^i\Q^j$ is defined on a class in $\pi_n (L)$ for $L$ a simplicial Lie algebra, then $\Q^{i'}\Q^{j'}$ is defined on all of $\pi_{n-1}(L)$, which is to say that $\minDim(i',j')<
%\minDim(i,j)$.
%\end{enumerate}
%In particular, if $\Q^i\Q^jx$ is defined, and $i>2j$, it can be rewritten as a sum of terms $\Q^{i'}\Q^{j'}x$ in which neither of the $\Q^{i'}$ and $\Q^{j'}$ that appear are top operations.
%\end{lem}
With these operations in hand, we can express the homotopy groups of spheres as follows:





\begin{prop}
For $n\geq1$, suppose that $X\in\GS{n+1}$ is a graded set. Let $B$ be a Hall basis derived from $X$ for the free good Lie algebra $\Fr{\GoodLie{n+1}}(X)$. Note that $B\subset \Fr{\GoodLie{n+1}}(X)$ is in particular a graded set. \textbf{(roman numerals?)}
\begin{alignat*}{2}
\pi_*(\makebox[\widthof{$\Fr{\GoodLie{n}}S(X)$}][c]{$\Fr{\GoodLie{n}}S(X)$})
&=
\left\langle \Q^Ib\,:\,\genfrac{}{}{0pt}{}{b\in B,\,I\in\admis{\Q},\ i_1\leq|b|,}{\textup{$I$ doesn't contain 0}}\right\rangle,%
\\
\pi_*(\makebox[\widthof{$\Fr{\GoodLie{n}}S(X)$}][c]{$\Fr{\RestLie{n}}S(X)$})
&=
\left\langle \Q^Ib\,:\,b\in
B,\,I\in\admis{\Q},\ i_1\leq|b|\right\rangle\textup{, \ and}%
\\
\pi_*(\makebox[\widthof{$\Fr{\GoodLie{n}}S(X)$}][c]{$\Fr{\PRLie{n}}S(X)$})
&=
\left\langle \Q^Ib\,:\,\genfrac{}{}{0pt}{}{b\in B,\,I\in\admis{\Q},\ i_1\leq|b|,}{\textup{if }a_{n}\!=\!\cdots\!=\!a_2\!=\!0\textup{ then $I$ doesn't contain 0}}\right\rangle,%
\end{alignat*}
where by $a_n,\ldots,a_2$ we mean all but the outermost indices in $\|b\|=(a_{n+1},\ldots,a_1)$, by $i_1$ we mean the rightmost entry of $I=(i_\ell,\ldots,i_1)$, and we ignore the condition on $i_1$ when $I$ is empty.
%\begin{enumerate}[i)]\squishlist
%\setlength{\parindent}{.25in}
%\item $\pi_*(\Fr{\GoodLie{n}}S(X))$ has basis $\{\Q^Ib\,:\,b\in
%B,\,I\in\admis{\Q},\ i_1\leq|b|,\ I\textup{ doesn't contain zero}\}$;
%\item $\pi_*(\Fr{\RestLie{n}}S(X))$ has basis $\{\Q^Ib\,:\,b\in
%B,\,I\in\admis{\Q},\ i_1\leq|b|\}$; and
%\item $\pi_*(\Fr{\PRLie{n}}S(X))$ has basis
%\[\left\{\Q^Ib\,:\,\genfrac{}{}{0pt}{}{b\in B_{a_{n+1},\ldots,a_1},\,I\in\admis{\Q},\ i_1\leq|b|=a_{n+1},}{\textup{if }a_{n}=\cdots=a_2=0\textup{ then $I$ doesn't contain 0}}\right\}\]
%\end{enumerate}
%We always allow $I$ to be the empty sequence in the above bases, in which case the condition on $i_{1}$ is to be ignored. If $I$ is nonempty, we mean that $i_{1}$ is the rightmost entry of $I=(i_\ell,\ldots,i_1)$.
\end{prop}
\noindent The first two parts assert that the natural homotopy operations in the
homotopy of a simplicial good or restricted Lie algebra are generated by
the operations of the previous proposition, and are subject only to those
relations stated. Part (iii) asserts that the natural homotopy operations
for partially restricted Lie algebras interpolate between those for good
and those for restricted Lie algebras in the obvious way. This demonstrates
that $\PiAlg{\PRLie{n}}$, the natural target category for the functor
$\pi_*$ on $s\PRLie{n}$, is in fact the category $\LL{n+1}$ that we'll
define in the next section.
\begin{proof}
See
\cite[Thm 8.8 and proof]{CurtisSimplicialHtpy.pdf} for the proof of parts
(i) and (ii).

For any set $Y\in\GpS{n}$, write $Y^\textup{z}\in\GpS{n}$ for the object
defined by
\[(Y^\textup{z})_{a_n,\ldots,a_1}=\begin{cases}
Y_{0,\ldots,0,a_1},&\textup{if }a_{n}\!=\!\cdots\!=\!a_2\!=\!0;\\
\{\star\},&\textup{otherwise}.
%\\,&\textup{if }
\end{cases}
\]
This assignment is functorial in $Y$, and there is a natural surjection
of restricted Lie
algebras $\gamma:\Fr{\RestLie{n}}(Y)\to\Fr{\RestLie{n}}(Y^\textup{z})$. 
Moreover, there is a natural surjection $\rho:\Fr{\RestLie{n}}(Y^\textup{z})\to \Fr{\RestLie{n}}(Y^\textup{z})/\Fr{\GoodLie{n}}(Y^\textup{z})$.
The calculation \ref{PropBasesOfFreeLieAlgs} of bases for the free constructions in
$\RestLie{n}$ and $\PRLie{n}$ shows that the kernel of the
composite
$\rho\gamma$
is precisely $\Fr{\PRLie{n}}(Y)$. Applying this calculation levelwise to $S(X)\in s\GpS{n}$, there are two short exact sequences of simplicial vector spaces, and a commuting triangle:
\[\xymatrix@R=-2mm{
0
\ar[r]
&%r1c1
\Fr{\PRLie{n}}S(X)
\ar[r]^{\alpha}
&%r1c2
\Fr{\RestLie{n}}S(X)
\ar[dr]^-(.4){\rho\gamma}\ar[dd]^-{\gamma}
&%r1c3
%\Fr{\RestLie{n}}S(X)/\Fr{\PRLie{n}}S(X)
%\ar[r]
%\ar[d]^-{\cong}
%&%r1c4
\\%r1c5
&&&\frac{\displaystyle \Fr{\RestLie{n}}S(X^\textup{z})}{\displaystyle \Fr{\GoodLie{n}}S(X^\textup{z})}
\ar[r]
&
0
\\
0
\ar[r]
&%r1c1
\Fr{\GoodLie{n}}S(X^\textup{z})
\ar[r]^{\beta}
&%r1c2
\Fr{\RestLie{n}}S(X^\textup{z})
\ar[ru]^-(.4){\rho}&%r1c3
%\Fr{\RestLie{n}}S(X^\textup{z})/\Fr{\GoodLie{n}}S(X^\textup{z})
%\ar[r]
%&%r1c4
%0
}\]
Now using parts (i) and (ii) we can describe the maps $\beta_*$ and $\gamma_*$ as the evident inclusion and projection between
\begin{alignat*}{2}
\pi_*(\makebox[\widthof{$\Fr{\GoodLie{n}}S(X^\textup{z})$}][c]{$\Fr{\GoodLie{n}}S(X^\textup{z})$})
&=
\langle \Q^Ib\,:\,a_{n}\!=\!\cdots\!=\!a_2\!=\!0,\, i_1\leq|b|,\, 0\notin I\rangle.%
\\
\pi_*(\makebox[\widthof{$\Fr{\GoodLie{n}}S(X^\textup{z})$}][c]{$\Fr{\RestLie{n}}S(X^\textup{z})$})
&=
\langle \Q^Ib\,:\,a_{n}\!=\!\cdots\!=\!a_2\!=\!0,\, i_1\leq|b|\rangle\textup{, \ and}%
\\
\pi_*(\makebox[\widthof{$\Fr{\GoodLie{n}}S(X^\textup{z})$}][c]{$\Fr{\RestLie{n}}S(X)$})
&=
\langle \Q^Ib\,:\,i_1\leq|b|\rangle,%
\end{alignat*}
(where in each of the above specifications we mean that $b\in B_{a_{n+1},\ldots,a_1}$ and $I\in\admis{\Q}$).
As $\beta_*$ is injective, $\rho_*$ is surjective, thus $(\rho\gamma)_*$ is surjective, and $\alpha_*$ is injective. The resulting identification of $\pi_*(\Fr{\PRLie{n}}S(X))$ with $(\gamma_*)^{-1}(\im(\beta_*))$ gives (iii).
\end{proof}
\subsection{Categories of graded Lie algebras with unstable operations}
\begin{itemize}
\setlength{\parindent}{.25in}
%\item Let $\LL{1}$ be the full subcategory of connected objects in $\calW$. We already have the definition for these in Goerss' book, but I'll spell this out here, to emphasise certain interesting aspects of the character of $\LL{1}$.
%An object of $\LL{1}$ is a (potentially bad) graded Lie algebra $L$ in $\BadLie{1}$ with an unstable (linear) action of $\CommSteen$.
%That is, $L$ admits $P$ operations
%\[P^i:L_{a_1}\to L_{a_1+i+1}\]
%which are \emph{always} defined for $i\geq0$, but \emph{equal zero} unless $2\leq i\leq a_n$.
%%By `bad' Lie algebra, we mean that the bracket satisfies the Jacobi identity, and is skew symmetric (``$[x,y]=[y,x]$ for all $x,y$'').
%%The graded part $L_{0}$ is assumed to be zero, and for $x\in L_{a_1}$ we write $|x|:=a_1$. 
%The $P$ operations are assumed to satisfy the following $P$-Adem relations. If $i\geq 2j$:
%\[P^iP^jx=\sum_{s=i-j+1}^{i+j-2}{2s-i-1\choose s-j}P^{i+j-s}P^sx.\]
%Every $P$ operation is assumed linear, and $[x,P^iy]=0$ for all $i,x,y$. Additionally, the top $P$ operation is the `self-square': for $x\in L_{i}$, $P^{i}x=[x,x]$.
%%%%%\item Let $\LL{1}$ be the following category of graded bad Lie algebras which are also unstable left $\Q$-modules.
%%%%%An object of $\LL{1}$ is a (potentially) bad graded Lie algebra $L$ in $\BadLie{1}$ %with an unstable (linear) action of $\CommSteen$.
%%%%%admitting linear operations
%%%%%\[P^i:L_{a_1}\to L_{a_1+i+1}\]
%%%%%which are \emph{always} defined for $i\geq0$, but \emph{equal zero} unless $2\leq i\leq a_n$. They satisfy the $P$-Adem operations when written on the left. Finally, $[x,P^iy]=0$ for all $i,x,y$, and the top $P$ operation is the `self-square': for $x\in L_{i}$, $P^{i}x=[x,x]$.
%%%%%
%%%%%Note that this $\LL{1}$ is just the full subcategory of connected objects in Goerss' category. See \cite[p.14]{MR1089001}.
\item Let $\LL{1}$ be the following category of graded bad Lie algebras which are also unstable left $P$-modules.
An object of $\LL{1}$ is a (potentially) bad graded Lie algebra $L$ in $\BadLie{1}$ %with an unstable (linear) action of $\CommSteen$.
admitting linear operations
\[P^i:L_{a_1}\to L_{a_1+i+1}\]
which are \emph{always} defined for $i\geq2$, but \emph{equal zero} unless $i\leq a_n$. They satisfy the $P$-Adem operations when acting from the left. We also require that $[x,P^iy]=0$ for all $i,x,y$, and the top $P$ operation is the `self-square':
\[P^{i}x=[x,x]\ \textup{for any $x\in L_{i}$ with $i\geq2$}.\]
Finally, we require that $L$ is `good in dimension $1$', that is, for $x\in L_1$, $[x,x]=0$.

Importantly, the category $\LL{1}$ can be identified with the full subcategory of connected objects in Goerss' category%footnote:
\footnote{See \cite[p.14]{MR1089001}. Note that Goerss defines the $P^i$ for all $i\geq0$, but notes that $P^0$ and $P^1$ are identically zero. It suits our exposition to exclude the symbols $P^0$ and $P^1$.} $\calW$.  

\item Let $\nontop{1}$ be the following category of graded unstable $P$-modules. An object of $\nontop{1}$ is a graded vector space $M$ in $\GR{1}$,
admitting linear operations
\[P^i:M_{a_1}\to M_{a_1+i+1}\]
which are always defined for $i\geq2$, but are zero unless $2\leq i\leq a_n$.
%The graded part $M_{0}$ is assumed to be zero, and for $x\in M_{a_n,\ldots,a_1}$ we write $|x|:=a_n$. 
The $P$ operations are assumed to satisfy the same $P$-Adem relations as in $\LL{1}$. %Every operation $P^i$ is assumed linear.
\item For $n\geq2$, let $\LL{n}$ be the following category of $n$-graded partially restricted Lie algebras with unstable $\Q$ operators acting from the left. An object of $\LL{n}$ consists of an object $L$ of $\PRLie{n}$ 
admitting $\Q$ operations
\[\Q^i:L_{a_n,\ldots,a_1}\to L_{a_n+i,2a_{n-1},\ldots,2a_2,2a_1+1}\]
defined whenever $0\leq i\leq a_n$ and not all of $i,a_{n-1},\ldots,a_{2}$ are zero. The top $\Q^i$ \emph{equals} the restriction operation, so that $\Q^{i}(x)=\restn(x)$ whenever $|x|=i$ and both sides are defined%footnote:
\footnote{The two sides of $\Q^{i}(x)=\restn(x)$, where $\|x\|=(a_n,\ldots,a_1)$ and $i=a_n$, are either both defined or both not defined: they  are undefined \emph{iff} $a_n\!=\!\cdots \!=\!a_2\!=\!0$.}. The non-top operations are linear and are killed by the bracket.
That is, for $x,y\in L_{a_n,\ldots,a_1}$, $i<a_n$ (with not all of $i,a_{n-1},\ldots,a_{2}$ being zero) and $z\in L$: \[\Q^i{(x+y)}=\Q^i{x}+\Q^i{y}\textup{ and }[\Q^i{x},z]=0.\]
Finally, the following $\Q$-Adem relations hold whenever $i>2j$ and $\Q^i\Q^jx$ is defined (i.e.\ $j\leq|x|$, not all of $j,a_{n-1},\ldots,a_2$ are zero, and $i\leq|x|+j$):
\[\Q^i\Q^jx=\sum_{k=0}^{(i-2j)/2-1}{i-2j-2-k\choose k}\Q^{2j+1+k}\Q^{i-j-1-k}x.\]
%\item For $n\geq2$, let $\LL{n}$ be the following category of $n$-graded \emph{good} Lie algebras with unstable $\Q$ operators acting from the left. An object of $\LL{n}$ consists of an object $L$ of $\GoodLie{n}$
%%, under a $(0,\ldots,0,1)$-shifted bracket:
%%\[[\DASH,\DASH]:L_{a_n,\ldots,a_1}\otimes L_{b_n,\ldots,b_1}\to L_{a_n+b_n,\ldots,a_2+b_2,a_1+b_1+1}\]
%admitting $\Q$ operations
%\[\Q^i:L_{a_n,\ldots,a_1}\to L_{a_n+i,2a_{n-1},\ldots,2a_2,2a_1+1}\]
%defined whenever $i\leq a_n$ and not all of $i,a_{n-1},\ldots,a_{2}$ are zero. %By `good' Lie algebra, we mean that the bracket satisfies the Jacobi identity, and is alternating (``$[x,x]=0$ for all $x$'').
%%The graded part $L_{a_n,\ldots,a_1}$ is assumed to be zero whenever $a_1=0$, and for $x\in L_{a_n,\ldots,a_1}$ we write $|x|:=a_n$.
%The $\Q$ operations are assumed to satisfy the following $\Q$-Adem relations wherever they make sense. If $i>2j$, and $\Q^i\Q^jx$ is defined (i.e.\ $j\leq|x|$, not all of $j,a_{n-1},\ldots,a_2$ are zero, and $i\leq|x|+j$):
%\[\Q^i\Q^jx=\sum_{k=0}^{(i-2j)/2-1}{i-2j-2-k\choose k}\Q^{2j+1+k}\Q^{i-j-1-k}x.\]
%The non-top operations are linear and are killed by the bracket.
%That is, for $x,y\in L_{a_n,\ldots,a_1}$ and $z\in L$: \[\Q^i{(x+y)}=\Q^i{x}+\Q^i{y}\textup{ and }[\Q^i{x},z]=0.\]
%On the other hand, the top operation, 
%\[\Q^i:L_{i, a_{n-1},\ldots,a_1}\to L_{2i, 2a_{n-1},\ldots,2a_2,2a_1+1}\]
%acts as a partially defined restriction for $L$, satsfying, for $x,y\in L_{i, a_{n-1},\ldots,a_1}$ and $z\in L$,
%\[\Q^i(x+y)=\Q^ix+\Q^iy+[x,y]\textup{ and }[\Q^{i}x,z]=[x,[x,z]].\]

\item For $n\geq2$, let $\nontop{n}$ be the following category of $n$-graded vector spaces with unstable $\Q$ operators acting from the left. An object of $\nontop{n}$ is an $n$-graded vector space $M$ in $\GR{n}$,
admitting linear operations
\[\Q^i:M_{a_n,\ldots,a_1}\to M_{a_n+i,2a_{n-1},\ldots,2a_2,2a_1+1}\]
defined whenever $i< a_n$ (note the strict inequality) and not all of $i,a_{n-1},\ldots,a_{2}$ are zero.
%The graded part $M_{a_n,\ldots,a_1}$ is assumed to be zero whenever $a_1=0$, and for $x\in M_{a_n,\ldots,a_1}$ we write $|x|:=a_n$. 
The $\Q$ operations are assumed to satisfy the same $\Q$-Adem relations as in $\LL{n}$, whenever the left hand side makes sense in $\nontop{n}$. %The same discussion shows that this too is a meaningful definition. %Every operation $\Q^i$ is assumed linear.
\end{itemize}

\begin{prop}\label{PropFreeKandUconstructions}
For $n\geq1$, suppose that $X\in\GS{n}$ is a graded set. Let $B$ be a Hall basis derived from $X$ for the free good Lie algebra $\Fr{\GoodLie{n}}(X)$. Note that $B\subset \Fr{\GoodLie{n}}(X)$ is in particular a graded set. Then when $n=1$:
\begin{alignat*}{2}
{\Fr{\LL{1}}(X)}
&=
\left\langle P^Ib\,:\,b\in B,\,I\in\admis{P}\textup{ with }i_1\leq|b|  \right\rangle,\textup{ and}%
\\
{\Fr{\nontop{1}}(X)}
&=
\left\langle P^Ix\,:\,x\in X,\,I\in\admis{P}\textup{ with }i_1\leq|x|\right\rangle,%
\end{alignat*}
and when $n\geq2$:
\begin{alignat*}{2}
{\Fr{\LL{n}}(X)}
&=
\left\langle \Q^Ib\,:\,\genfrac{}{}{0pt}{}{b\in B,\,I\in\admis{\Q},\ i_1\leq|b|,}{\textup{if }a_{n-1}\!=\!\cdots\!=\!a_2\!=\!0\textup{ then $I$ doesn't contain 0}}\right\rangle,\textup{ and}%
\\
{\Fr{\nontop{n}}(X)}
&=
\left\langle \Q^Ix\,:\,\genfrac{}{}{0pt}{}{x\in X,\,I\in\admis{\Q},\ i_1<|x|,}{\textup{if }a_{n-1}\!=\!\cdots\!=\!a_2\!=\!0\textup{ then $I$ doesn't contain 0}}\right\rangle,%
\end{alignat*}
where by $a_{n-1},\ldots,a_2$ we mean all but the outermost indices in $\|b\|=(a_{n},\ldots,a_1)$ or $\|x\|=(a_{n},\ldots,a_1)$ respectively, by $i_1$ we mean the rightmost entry of $I=(i_\ell,\ldots,i_1)$, and we ignore the condition on $i_1$ when $I$ is empty. Note that when describing  $\nontop{n}$, the strictness of the inequality on $i_1$ differs according to whether $n=1$ or $n\geq2$.
\end{prop}
%\begin{prop}
%For $n\geq1$, let $X\in\GS{n}$ be a graded set, and let $B$ be a Hall basis derived from $X$ for $\Fr{\GoodLie{n}}(X)$. Note that $B\subset \Fr{\GoodLie{n}}(X)$ is in particular a graded set. Then when $n=1$:
%\begin{enumerate}[i)]\squishlist
%\setlength{\parindent}{.25in}
%\item $\Fr{\LL{1}}(X)$ has basis $\{P^Ib\,:\,b\in B,\,I\in\admis{P}\textup{ with }i_1\leq|b| \}$;
%\item $\Fr{\nontop{1}}(X)$ has basis $\{P^Ix\,:\,x\in X,\,I\in\admis{P}\textup{ with }i_1\leq|x| \}$;
%\end{enumerate}
%and when $n\geq2$:
%\begin{enumerate}[i)]\squishlist
%\setlength{\parindent}{.25in}
%\item[iii)] $\Fr{\LL{n}}(X)$ has basis $\{\Q^Ib\,:\,b\in B_{a_{n},\ldots,a_1},\,I\in\admis{\Q},\ i_1\leq|b|,\ i_1,a_{n-1},\ldots,a_2\textup{ not all zero}\}$;
%\item[iv)] $\Fr{\nontop{n}}(X)$ has basis $\{\Q^Ix\,:\,x\in X_{a_{n},\ldots,a_1},\,I\in\admis{\Q},\ i_1<|x|,\ i_1,a_{n-1},\ldots,a_2\textup{ not all zero}\}$.
%\end{enumerate}
%We always allow $I$ to be the empty sequence in the above bases, in which case the condition on $i_{1}$ is to be ignored. If $I$ is nonempty, we mean that $I=(i_\ell,\ldots,i_1)$, so that $i_{1}$ is the rightmost entry of $I$.
%Note that the inequalities involved differ according to whether $n=1$ or $n\geq2$.
%\end{prop}
%\begin{prop}
%If I ever use this, I'll swap the `i_ell's
%For $n\geq2$, for $X\in\GS{n}$ a graded set, $\Fr{\LL{n}}(X)$ has basis the compositions $\Q^{I}b$, where $b$ runs over a homogeneous Hall basis for the free (unrestricted, good) Lie algebra $\Fr{\GoodLie{n}}(X)$, and $I=(i_1,\ldots,i_\ell)$ is a $\Q$-admissible sequence with $i_\ell\leq|b|$ and not all of $i_{\ell},a_{n-1},\ldots,a_2$ are zero if $b$ has grading $(a_n,\ldots,a_1)$. 
%
%$\Fr{\nontop{n}}(X)$ has basis the compositions $\Q^{I}x$ where $x\in X$ and $I=(i_1,\ldots,i_\ell)$ is a $\Q$-admissible sequence with $i_\ell<|x|$ and not all of $i_{\ell},a_{n-1},\ldots,a_2$ are zero if $x$ has grading $(a_n,\ldots,a_1)$.
%%Here, $L^uX$ is an object of $\GoodLie{n}$ if we stipulate that the bracket shifts degrees as appropriate.
%\end{prop}
\begin{proof}
The identification of $\Fr{\LL{1}}(X)$ is \cite[Thm F, p.15]{MR1089001}.
For each of the remaining three calculations, it is clear that the proposed bases span the free constructions. It remains to show that the elements are linearly independent.
For any $n\geq1$, the proposed basis of $\Fr{\nontop{n}}(X)$ can be seen to be linearly independent using the map $\Fr{\nontop{n}}(X)\to \forget{\nontop{}}\Fr{\LL{n}}(X)$, as long as $\Fr{\LL{n}}(X)$ is known. To calculate $\Fr{\LL{n}}(X)$ when $n\geq2$, note that we have calculated the object $\pi_*(\Fr{\PRLie{n-1}}S(X))\in\PiAlg{\PRLie{n-1}}\cong\LL{n}$, and that the elements of our proposed basis of $\Fr{\LL{n}}(X)$ map to a basis of $\pi_*(\Fr{\PRLie{n-1}}S(X))$ under the canonical map $\Fr{\LL{n}}(X)\to \pi_*(\Fr{\PRLie{n-1}}S(X))$. %This proves the linear independence statement we require.
%sequences I here are back to front again
%For part (iii), choose a simplicial object of $\GR{n-1}$, say $V\in s(\GR{n-1})$, such that $\pi_{*}V=\Fr{\GR{n}}(X)$. Then \cite[Thm 8.6, Thm 8.8]{CurtisSimplicialHtpy.pdf} shows that $\pi_*(\Fr{\RestLie{n}}(V))$ is an element of $\LL{n}$, and has a basis consisting of tensors $\Q^{I}\otimes b$, for $b\in B$, and $I\in\admis{\Q}$ with $i_\ell\leq|b|$. The `not all zero' condition is dropped here, so that $\pi_*(\Fr{\RestLie{n}}(V))$ is larger than our proposed $\LL{n}(X)$.
%
%Consider then the morphism $\Fr{\LL{n}}(X)\to \pi_*(\Fr{\RestLie{n}}(V))$ induced by the universal property of the free construction. The images of the proposed basis for $\Fr{\LL{n}}(X)$ map to linearly independent elements in $\pi_*(\Fr{\RestLie{n}}(V))$, showing that $\Fr{\LL{n}}(X)$ is as described. The same argument will prove (iv), this time viewing $\pi_*(\Fr{\RestLie{n}}(V))$ as an element of $\nontop{n}$.
\end{proof}
\begin{cor}
For all $n\geq 1$, the forgetful functor $\forget{\nontop{}}:\LL{n}\to\nontop{n}$ preserves free objects.
\end{cor}
\begin{proof}
When $n=1$, $\forget{\nontop{}}(\Fr{\LL{1}}(X))\in\nontop{1}$ is free on $B$. For $n\geq2$, $\forget{\nontop{}}(\Fr{\LL{n}}(X))\in\nontop{n}$  is free on %$B\cup \{\iteratedrestn{b}{n}\,:\,b\in B_{a_{n},\ldots,a_1},\,n\geq1,\textup{ not all of $a_n,\ldots,a_2$ are zero}\}$,
$\left\langle \iteratedrestn{b}{r}\,:\,b\in
B_{a_{n},\ldots,a_1},\,r\geq0,\textup{ where if }a_{n}\!=\!\cdots\!=\!a_2\!=\!0\textup{ then $r=0$}\right\rangle,$
 since
\[\iteratedrestn{b}{n}=\Q^{(2^{n-1}|b|)}\cdots \Q^{(2|b|)}\Q^{|b|}b.\qedhere\]
\end{proof}
\noindent Due to the choices involved in taking a Hall basis, I see no reason that $\forget{\nontop{}}$ should preserve free maps of free objects.
\end{CategoriesOfInterest}

\begin{HomotopicalAlgebra}
\section{Grothendieck spectral sequences}
\subsection{Homotopical algebra}


Every category $\mathsf{C}=\mathsf{C}^n$ that we defined in \S\ref{ChapterCategoriesOfInterest} is an example of an (underlying abelian) category of universal graded algebras, in the sense of \cite{Blanc_Stover-Groth_SS.pdf}. In particular, $s\mathsf{C}$ is a model category, in which the weak equivalences are the weak equivalences of underlying sets, and the fibrations are those maps which are surjections on the basepoint components of the underlying simplicial groups. The cofibrant objects are exactly the retracts of the almost free objects%footnote:
\footnote{Note that for cofibrant objects, Quillen, page 4.11, states that the cofibrant objects are retracts of the free objects. Now as we have constructed $\Delta bX$, it is free up from vector spaces, but maybe not up from sets. Hrm. Miller does up from vector spaces. Presumably this is no big deal --- Miller gets around it without mention. That the adjunction doesn't go as far down should be countered by the fact that we've got $\F_2$-vector spaces to map to!}. See \cite{Blanc_Stover-Groth_SS.pdf}, \cite[II\S4]{QuillenHomAlg.pdf} and \cite[\S3]{MillerSullivanConjecture.pdf}.
%If $\mathsf{C}$ is any category of universal graded algebras (or CUGA, see \cite{Blanc_Stover-Groth_SS.pdf}), $s\mathsf{C}$ is a model category, in which the weak equivalences are the weak equivalences of underlying sets, and the fibrations are those maps which are surjections on the basepoint components of the underlying simplicial groups. See \cite{Blanc_Stover-Groth_SS.pdf} and \cite[II\S4]{QuillenHomAlg.pdf}.

For any $X$ in $\mathsf{C}$, we define simplicial objects $\BarConst{\mathsf{C}}X$ and $\Constant X$ by the formulae $(\BarConst{\mathsf{C}}X)_s=\Fr{\mathsf{C}}^{s+1}X$ and $(\Constant{\mathsf{C}}X)_s=X$. There is a natural augmentation map $a:\BarConst{\mathsf{C}}X\to\Constant X$ which is a weak equivalence, as the bar construction has a natural contraction as a simplicial vector space. Moreover, $a$ is surjective by construction, and being a weak equivalence, is a surjection on basepoint components, and thus an acyclic fibration. Finally, $\BarConst{\mathsf{C}}X$ is almost free by construction, and is thus cofibrant, so that $a$ is a functorial cofibrant replacement for $\Constant X$.

One can extend this definition to simplicial objects $X\in s\mathsf{C}$, producing bisimplicial objects defined by
$(\BarConst{\mathsf{C}}X)_{s,t}=\Fr{\mathsf{C}}^{t+1}X_s$ and $(\Constant{\mathsf{C}}X)_{s,t}=X_s$. Again, there is a natural augmentation map $a:\BarConst{\mathsf{C}}X\to\Constant X$. The resulting map on diagonal simplicial objects, $\Delta a:\Delta\BarConst{\mathsf{C}}X\to\Delta \Constant X= X$, is a weak equivalence, as confirmed by a comparison of spectral sequences. For the same reasons, then, $\Delta a$ is a functorial cofibrant replacement for $X$.
\end{HomotopicalAlgebra}
\begin{DiagramOfFunctors}
\subsection{Indecomposables}
For each of the categories of \S\ref{ChapterCategoriesOfInterest}, there is a functor $\Ind{\mathsf{C}}:\mathsf{C}^n\to\GR{n}$ which sends $X\in\mathsf{C}^n$ to the quotient of the underlying vector space of $X$ by the subspace generated by the images of all the operations defining $X$.

The functor $\Ind{\mathsf{C}}$ is always left adjoint to a functor $K:\GR{n}\to\mathsf{C}^n$ which sends a graded vector space $V$ to the corresponding trivial object in $\mathsf{C}^n$. That is, $KV$ has $V$ as underlying vector space, and all operations in $KV$ are zero. Note then that the right adjoint, $K$, preserves weak equivalences and fibrations, as membership of these classes is determined on the level of underlying simplicial groups. Thus, $\Ind{\mathsf{C}}$ is a left Quillen functor, so that we can readily define the left derived functors $\derived_*\Ind{\mathsf{C}}$.


This analysis extends to the case where $\mathsf{C}$ is the category of non-unital commutative $\F_2$-algebras, which is evidently equivalent to $\Comm$. Under this equivalence, $\derived_*\Ind{\Comm}$ recovers the definition of the Quillen homology $H_*^Q$.










%\subsection{A diagram of functors}
%%%%%Now for each category $\mathsf{C}$ amongst $\LL{n}$, $\PRLie{n}$ and $\nontop{n}$, there is a functor $\Ind{\mathsf{C}}:\mathsf{C\overset{}{\to}\GR{n}}$, which takes the quotient of the underlying vector space by the subspace spanned by the image of all of the structure maps. These provide the solid horizontal arrows in the following diagram, while the vertical maps are the evident forgetful functors: %There are also free/forget adjunctions $\GS{n} \rightleftarrows \mathsf{C}$. %Finally, there's a forgetful functor $\forgetSymbol:\LL{n}\to\nontop{n}$, which, importantly, preserves  free objects%footnote:
%%%%%%\footnote{I think that the adjunction does need to come from graded pointed sets, but that that's no problem for Blanc-Stover. The only thing that matters is that $\forgetSymbol$ preserves frees, but it does --- frees in $\nontop{n}$ are easy to describe.}. These functors give the solid maps in the following diagram:

The indecomposables functors described above provide the solid horizontal arrows in the following diagram, while the vertical maps are the evident forgetful functors:

\[\xymatrix@R=8mm@C=15mm@!C{
\LL{n}
\ar@/^1.5em/[rr]^-{\Ind{\LL{}}}
\ar[d]_-{\forget{\nontop{}}}
\ar@{-->}[r]_-{\Psi}%\Ind{\nontop{}}}
&%r1c1
\PRLie{n}
\ar[d]^-{\forget{\GR{}}}
\ar[r]_-{\Ind{\PRLie{}}}
&%r1c2
\GR{n}
\\%r1c3
\nontop{n}
\ar[r]_-{\Ind{\nontop{}}}
&%r2c1
\GR{n}
&%r2c2
%r2c3
}\]
\begin{prop}\label{PropOnTheCommutingDiagramOfIndFunctors}
For all $n\geq1$ there exists a functor $\Psi:\LL{n}\to\PRLie{n}$ which completes the diagram above to a commuting square and triangle. This functor %, which we also name $\Ind{\nontop{}}:\LL{n}\to\PRLie{n}$,
preserves free objects.
\end{prop}
\begin{proof}
If we are to choose $\Psi$ such that the square commutes, the vector space underlying $\Psi(X)$ must be the quotient of $X$ by the image of the $\nontop{n}$-operations in $X$. We'll show that the bracket and restriction operations defined on $X$ descend to the quotient to give an $\PRLie{n}$ structure. %that are required to define an object of $\nontop{n}$ are well defined after quotienting out the image of the operations in $\Ind{\nontop{}}$.

When $n=1$ we are to quotient by the image of all of the $P$ operations. There are no restriction operations to be defined on an object of $\PRLie{1}$, and the bracket is well defined due to the axiom ``$[x,P^iy]=0$ for \emph{all} $i$''. Although we started with a bad Lie algebra, the axiom ``$P^{|x|}x=[x,x]$'' shows that the quotient is a good lie algebra, as required.

When $n\geq2$  we are to quotient by the image of all of the \emph{non-top} $\Q$ operations. Again, the bracket is well defined on the quotient as brackets kill \emph{non-top} $\Q$ operations. To see that the restriction $\restn{}$ (which in an object of $\LL{n}$ is precisely the top $\Q$ operation) is well defined on the quotient, %we must check the following fact: for homogeneous $x,y\in L$, and $0\leq i<|y|$ such that $\Q^iy$ is defined and $i+|y|=|x|$, $\restn{(x)}$ and $\restn(x+\Q^iy)$ differ by a sum of images of nontop $\Q$ operations. This follows from the fact that $\Q$-Adem relations do not produce top $\Q$ operations, since \[\restn{(x+\Q^iy)}=\restn{(x)}+\restn{(\Q^iy)}+[x,\Q^iy]=\restn{(x)}+\Q^{|y|+i}\Q^iy+0,\] and $|x|=i+|y|>2i$, so that there's an Adem relation to apply to $\Q^{|x|}\Q^iy$.
suppose that $\Q^iy$ is the image of a well-defined non-top operation, so that $i<|y|$. Then
\[\restn{(x+\Q^iy)}=\restn{(x)}+\restn{(\Q^iy)}+[x,\Q^iy]=\restn{(x)}+\Q^{|y|+i}\Q^iy+0.\]
The term $\Q^{|y|+i}\Q^iy$ admits the application of an Adem relation, as $|y|+i>2i$, but lemma \ref{lemOnAdemChangeInM} implies that $\Q$-Adem relations never return top $\Q$ operations, so that $\Q^{|y|+i}\Q^iy$ vanishes in the quotient. Thus we've shown that $\Psi$ is a functor, and that the square commutes.

To check that the triangle commutes is the note that we didn't fail to kill any of the $\LL{n}$-operations over the course of applying the two functors $\Psi$ and $\Ind{\PRLie{}}$. This is easy both in case $n=1$ and in case $n\geq2$. Finally, we'll check that there's a (non-natural) isomorphism $\Psi(\Fr{\LL{n}}(X))\cong\Fr{\PRLie{n}}(X)$ in either case.

Suppose $n\geq2$. If $Q^Ib\in\Fr{\LL{n}}(X)$ is one of the basis elements of proposition \ref{PropFreeKandUconstructions}, then $\Q^Ib$ will be the image of a non-top operation except possibly when $I=(2^{\ell-1}|b|,\ldots,|b|)$, in which case $\Q^Ib=\iteratedrestn{b}{\ell}$. In fact, the $\iteratedrestn{b}{\ell}$ are certainly not involved in the image of a nontop operation, as no $\Q$-Adem relation produces a top $\Q$ operation. Thus, $\Psi(\Fr{\LL{n}}(X))$ has basis those $\iteratedrestn{b}{\ell}$ appearing in the basis given in proposition \ref{PropFreeKandUconstructions}, i.e.\ such that either $\ell=0$, or $\ell>0$ and not all of $a_n,\ldots,a_2$ are zero, after writing $\|b\|=(a_n,\ldots,a_1)$. This is exactly the basis of $\Fr{\PRLie{n}}(X)$ given in proposition \ref{PropBasesOfFreeLieAlgs}.

The $n=1$ case is simpler: $\Psi$ kills \emph{all} the $P$ operations in $\Fr{\LL{1}}(X)$, leaving only the elements of a Hall basis for $\Fr{\GoodLie{1}}(X)\cong \Fr{\PRLie{1}}(X)$.
\end{proof}

%The point is that there is a functor $\LL{n}\to\PRLie{n}$ which completes this diagram to a commuting square and triangle, which we also name $\Ind{\nontop{}}$. It will be important to note that $\Ind{\nontop{}}$ sends free objects in $\LL{n}$ to free objects in $\PRLie{n}$.
%
%When $n=1$, $\Ind{\nontop{}}$ quotients the underlying vector space by the images of all of the $P$ operations. The result inherits the structure of an element of $\PRLie{n}$ from the bracket in $\LL{1}$. This is well defined due to the axioms ``$[x,P^iy]=0$'' and ``$P^{|x|}x=[x,x]$''. 
%
%When $n\geq2$, $\Ind{\nontop{}}$ quotients the underlying vector space by the images of all of the non-top $\Q$ operations. That is, by the span of the $\Q^ix$ for $i<|x|$. The result inherits the structure of an element of $\PRLie{n}$ as follows. The Lie bracket is induced by that in $\LL{n}$; well defined by the axiom ``$[\Q^i{x},z]=0$ for $i<|x|$''. The restriction is induced by the top $\Q$ operations, being defined by the formula $\restn{x}:=\Q^{|x|}x$. It is a little less obvious that this is in fact well defined. One must check, for homogeneous $x,y\in L$, and $0\leq i<|y|$ such that $\Q^iy$ is defined and $i+|y|=|x|$, that $\Q^{|x|}x$ and $\Q^{|x|}(x+\Q^iy)$ differ by a sum of images of non-top $\Q$ operations. But\[\Q^{|x|}(x+\Q^iy)-\Q^{|x|}x=\Q^{|x|}(\Q^iy)+[x,\Q^iy]=\Q^{|x|}\Q^iy,\]and the above calcuation regarding indices in the $\Q$-Adem relations shows that $\Q^{|x|}\Q^iy$ can be rewritten without the appearance of a top $\Q$.
\end{DiagramOfFunctors}


\begin{GrothendieckSpectralSequences}
\subsection{The derived functors of $\Ind{\nontop{}}$}
\begin{lem}\label{LemLevelwiseFreeWillSuffice}
Let $\mathsf{C}$ be any of $\PRLie{n}$, $\LL{n}$ and $\nontop{n}$, and suppose that $W$ is a levelwise free object in $s\mathsf{C}$. Then the map $\Ind{\mathsf{C}}\Delta a:\Ind{\mathsf{C}}\Delta {\BarConst{\mathsf{C}}W}\to \Ind{\mathsf{C}}W$ is a weak equivalence in $s\GR{n}$, which is to say that $W$ is already close enough to being cofibrant for calculation of $\derived\Ind{\mathsf{C}}W$.
\end{lem}
\begin{proof}
The map $\Ind{\mathsf{C}}\Delta a$ is the map on diagonal simplicial objects induced by the map $\Ind{\mathsf{C}}a:\Ind{\mathsf{C}}\BarConst{\mathsf{C}}W\to \Ind{\mathsf{C}}\Constant W$ of bisimplicial objects, which is the prolongation of the augmentation maps $\Ind{\mathsf{C}}a:\Ind{\mathsf{C}}\BarConst{\mathsf{C}}W_s\to \Ind{\mathsf{C}}\Constant W_s$ of singly simplicial objects for each $s$. For each $s$, the map $a:\BarConst{\mathsf{C}}W_s\to \Constant W_s$ is a weak equivalence of cofibrant objects in $s\nontop{n}$, so as $\Ind{\mathsf{C}}$ is left Quillen, $\Ind{\mathsf{C}}a:\Ind{\mathsf{C}}\BarConst{\mathsf{C}}W_s\to \Ind{\mathsf{C}}\Constant W_s$ is a weak equivalence in $s\GR{n}$. Thus, the induced map of spectral sequences (those obtained by filtering by $s$) is an isomorphism from $E^1$.
\end{proof}




%\begin{lem}\label{LemLevelwiseFreeWillSuffice}
%Suppose that $W$ is a levelwise free object in $s\nontop{n}$. Then the map $\Ind{\nontop{}}\Delta a:\Ind{\nontop{}}\Delta {\BarConst{\nontop{}}W}\to \Ind{\nontop{}}W$ is a weak equivalence in $s\GR{n}$, which is to say that $W$ is already close enough to being cofibrant for calculation of $\derived\Ind{\nontop{}}W$.
%\end{lem}
%\begin{proof}
%The map $\Ind{\nontop{}}\Delta a$ is the map on diagonal simplicial objects induced by the map $\Ind{\nontop{}}a:\Ind{\nontop{}}\BarConst{\nontop{}}W\to \Ind{\nontop{}}\Constant W$ of bisimplicial objects, which is the prolongation of the augmentation maps $\Ind{\nontop{}}a:\Ind{\nontop{}}\BarConst{\nontop{}}W_s\to \Ind{\nontop{}}\Constant W_s$ of singly simplicial objects for each $s$. For each $s$, the map $a:\BarConst{\nontop{}}W_s\to \Constant W_s$ is a weak equivalence of cofibrant objects in $s\nontop{n}$, so as $\Ind{\nontop{}}$ is left Quillen, $\Ind{\nontop{}}a:\Ind{\nontop{}}\BarConst{\nontop{}}W_s\to \Ind{\nontop{}}\Constant W_s$ is a weak equivalence in $s\GR{n}$. Thus, the induced map of spectral sequences (those obtained by filtering by $s$) is an isomorphism from $E^1$.
%\end{proof}

\begin{Omitted}
\textbf{This was a clumsy proof which is valid nevertheless.}
\begin{lem}\label{LemLevelwiseFreeWillSuffice}
Suppose that $R$ is a levelwise free object in $s\nontop{n}$. Let ${bR}$ be the bisimplicial object which is the levelwise bar construction on $R$, defined by ${bR}_{s,t}=U^{t+1}R_s$. Then %the natural map $\Delta {bR}\to R$ is a cofibrant replacement \textbf{really? what's needed?}, and 
$\pi_*(I\Delta {bR})\to\pi_*(IR)$ is an isomorphism. %That is, the complex $IR$ already calculates $(\derived I)R$.
\end{lem}
\begin{proof}
The proof will proceed by factoring the map $\pi_*(I\Delta {bR})\to\pi_*(IR)$ as the composite of three isomorphisms:
\[\xymatrix@R=4mm{
\pi_*(I\Delta {bR})
\ar[r]_-{}
\ar[d]_-{\textup{AW}}
&%r1c1
\pi_*(IR)
\\%r1c2
H_*(\Tot CI{bR})\ar[r]^-{\textup{edge}}
&%r2c1
E^2_{*,0}\ar[u]_-{\tau}
%r2c2
}\]
The first will be homology isomorphism induced by the Alexander-Whitney map $C\Delta I{bR}\to\Tot CI{bR}$. The second map will be an edge homomorphism in the spectral sequence obtained by filtering $\Tot CI{bR}$ by `$s$':
\begin{alignat*}{2}
E^0_{st}
&=
IU^{t+1}R_s%
\\
E^1_{st}
&=
(\derived_t I)(R_s)=\begin{cases}
IR_s,&\textup{if }t=0;\\
0,&\textup{otherwise}.
\end{cases}
&\qquad&\text{(as $R_s$ is free)}\\
E^2_{st}
&=
\begin{cases}
\pi_s(IR),&\textup{if }t=0;\\
0,&\textup{otherwise}.
\end{cases}
\end{alignat*}
The isomorphism $\tau$ is induced, for each $s$, by the augmentation in the $t$ direction, $(I(\textup{id},d_0)):IUR_s\to IR_s$ %$I\rho:IUR_s\to IR_s$, where $\rho$ is the structure morphism $UR_s\to R_s$. 
The edge homomorphism sends the class of an $s$-cycle in $\Tot CI{bR}$, say,
\[\sum_{i=0}^s z_{i,s-i},\textup{ where }z_{i,s-i}\in I{bR}_{i,s-i}\]
to the class $z_{s,0}$ in $E^2_{s,0}$. Suppose then that $w\in I{bR}_{ss}$ is a cycle in $CI\Delta {bR}$. Then $\textup{edge}(\textup{AW}(w))$ is represented simply by $(I(\textup{id},(d_0)^s))(w)$. Thus, the effect of the composite of the three isomorphisms on a cycle is $w\mapsto (I(\textup{id},(d_0)^{s+1}))(w)$, but $(\textup{id},(d_0)^{s+1})$ is exactly the natural map $\Delta {bR}\to R$, demonstrating that the above diagram commutes.
\end{proof}
\end{Omitted}
From the above preparations, for any $X\in\LL{n}$, we can construct the solid arrows in the following diagram in the model category $s\nontop{n}$:
\[{
%\def\objectstyle{\scriptstyle}
\xymatrix{
{\ _{\hphantom{n}}0_{\hphantom{q}}}
\ar@{^{(}->}[r]
\ar@{^{(}->}[d]
&%r1c1
\BarConst{\nontop{}}(\forget{\nontop{}}X)
\ar@{->>}[r]^-{\sim}_-{a_1}
\ar@{-->}[d]_-{i}
\ar@{-->}[dl]^-{\exists}_-{j}
&%r1c2
\Constant(\forget{\nontop{}}X)
\ar@{=}[d]
\\%r1c3
\Delta \BarConst{\nontop{}}(\forget{\nontop{}} \BarConst{\LL{}}(X))
\ar@{->>}[r]^-{\sim}_-{a}
&%r2c1
\forget{\nontop{}} \BarConst{\LL{}}(X)
%B_\bullet(\Fr{\LL{n}},\Fr{\LL{n}},X)
\ar@{->>}[r]^-{\sim}_-{a_2}
&%r2c2
\forget{\nontop{}}\Constant(X)
%r2c3
}}\]
%%%%Old version of this diagram without updated bar constn notation
%%%\[{
%%%%\def\objectstyle{\scriptstyle}
%%%\xymatrix{
%%%{\ _{\phantom{n}}0_{\phantom{q}}}
%%%\ar@{^{(}->}[r]
%%%\ar@{^{(}->}[d]
%%%&%r1c1
%%%B_\bullet(\Fr{\nontop{n}},\Fr{\nontop{n}},\forget{\nontop{}}X)
%%%\ar@{->>}[r]^-{\sim}_-{a_1}
%%%\ar@{-->}[d]_-{i}
%%%\ar@{-->}[dl]^-{\exists}_-{\widetilde{i}}
%%%&%r1c2
%%%\textup{const}(\forget{\nontop{}}X)
%%%\ar@{=}[d]
%%%\\%r1c3
%%%\Delta b(\forget{\nontop{}} B_\bullet(\Fr{\LL{n}},\Fr{\LL{n}},X))
%%%\ar@{->>}[r]^-{\sim}_-{a}
%%%&%r2c1
%%%\forget{\nontop{}} B_\bullet(\Fr{\LL{n}},\Fr{\LL{n}},X)
%%%\ar@{->>}[r]^-{\sim}_-{a_2}
%%%&%r2c2
%%%\forget{\nontop{}}\textup{const}(X)
%%%%r2c3
%%%}}\]
Here, $a_2$ is an acyclic fibration because $\forget{\nontop{}}$ preserves acyclic fibrations, and all other decorations (on solid arrows) come directly from the above discussion. 
%The ``$\Delta b$'' used in the cofibrant replacement $a$ is taken in $s\nontop{n}$. 
Lemma \ref{LemLevelwiseFreeWillSuffice}, with $\mathsf{C}=\nontop{n}$ and $W=\forget{\nontop{}} \BarConst{\LL{}}(X)$, shows that $a$ remains a weak equivalence after applying $\Ind{\nontop{}}$. Supposing then that we can find a map $i$ such that $a_2i=a_1$, $i$ will lift to a map $j$. By 2-of-3, $j$ will be a weak equivalence of cofibrant objects, so that it too will remain an equivalence after applying the left Quillen functor $\Ind{\nontop{}}$. Thus $i$ too will remain an equivalence after applying $\Ind{\nontop{}}$.

There is a simple choice for the map $i$ such that $a_2i=a_1$. One just takes an expression in $(\Fr{\nontop{n})}^{\bullet+1}(\forget{\nontop{}}X)$ and views it as an expression in $(\Fr{\LL{n})}^{\bullet+1}(X)$ via the natural map $\Fr{\nontop{n}}W\to \Fr{\LL{n}}W$ adjoint to the inclusion $W\to \forget{\nontop{}}\Fr{\LL{n}}W$ for any $W\in\GR{n}$.
Recalling that $\Ind{\nontop{}}\forget{\nontop{}}=\forget{\GR{}}\Psi$, we have shown 
\begin{prop}
For any $X\in\LL{n}$, the map $\Ind{\nontop{}}(i)$ induces an isomorphism
\[(\derived\Ind{\nontop{}})(\forget{\nontop{}}X)\to \forget{\GR{}}(\derived\Psi)(X)\]
between  $(\derived\Ind{\nontop{}})(\forget{\nontop{}}X)$ and the vector space underlying $(\derived\Psi)X$.
\end{prop}

In light of this result, it seems reasonable to begin writing $\Ind{\nontop{}}$ for the functor $\Psi$. Then we have two commuting squares:
\[\vcenter{\xymatrix@R=12mm@C=24mm@!0{
\LL{n}
\ar[d]_-{\forget{\nontop{}}}
\ar@{->}[r]_-{\Ind{\nontop{}}}
&%r1c1
\PRLie{n}
\ar[d]^-{\forget{\GR{}}}
\\%r1c2
\nontop{n}
\ar[r]^-{\Ind{\nontop{}}}
&%r2c1
\GR{n}
}}
\textup{ \ and \ }
\vcenter{\xymatrix@R=12mm@C=24mm@!0{
\LL{n}
\ar[d]_-{\forget{\nontop{}}}
\ar@{->}[r]_-{\derived\Ind{\nontop{}}}
&%r1c1
\LL{n+1}\makebox[0pt][l]{\,\small$=\PiAlg{\PRLie{n}}$}
\ar[d]^-{\forget{\GR{}}}
\\%r1c2
\nontop{n}
\ar[r]^-{\derived\Ind{\nontop{}}}
&%r2c1
\GR{n+1}\makebox[0pt][l]{\,\small$=\PiAlg{\GR{n}}$}
}}
\hspace{1.5cm}
\]
It will be important in what follows, for various $X\in\LL{n}$, to understand $(\derived\Ind{\nontop{}})X$ as an object of $\LL{n+1}$. On the other hand, the underlying graded vector space $\forget{\GR{}}(\derived\Ind{\nontop{}})X\in\GR{n+1}$ can be calculated by a Koszul complex. The plan is to map representing cycles in the Koszul complex into the large resolution, perform the homotopy operations of interest at the chain level, and compress back into the Koszul complex.

%%%%%%\noindent It should cause no confusion when we denote the functor of this proposition $\Ind{\nontop{}}:\LL{n}\to \PRLie{n}$ as well, giving a commuting diagram
%%%%%%\[\xymatrix@R=8mm@C=15mm@!C{
%%%%%%\LL{n}
%%%%%%\ar@/^1.5em/[rr]^-{\Ind{\LL{n}}}
%%%%%%\ar[d]_-{\forget{\nontop{}}}
%%%%%%\ar@{->}[r]_-{\Ind{\nontop{}}}
%%%%%%&%r1c1
%%%%%%\PRLie{n}
%%%%%%\ar[d]^-{\forget{\GR{}}}
%%%%%%\ar[r]_-{\Ind{\PRLie{n}}}
%%%%%%&%r1c2
%%%%%%\GR{n}
%%%%%%\\%r1c3
%%%%%%\nontop{n}
%%%%%%\ar[r]^-{\Ind{\nontop{}}}
%%%%%%&%r2c1
%%%%%%\GR{n}
%%%%%%&%r2c2
%%%%%%%r2c3
%%%%%%}\]
%%%%%%in which the functors $\forgetSymbol$ and $\Ind{\nontop{}}$ with domain $\LL{n}$ both preserve free objects.



\subsection{Grothendieck spectral sequences}
We are interested in the derived functors $(\derived\Ind{\LL{}})X\in\GR{n+1}$ for $X\in\LL{n}$, most of all when $n=1$. It is convenient for us to omit the customary asterisk in $\derived_*$, and to think of the derived functors as a single object with one more grading than $X$ has. We always write the new grading to the left of the previous gradings, as is customary.

Proposition \ref{PropOnTheCommutingDiagramOfIndFunctors} included the fact that ${\Ind{\nontop{}}}:\LL{n}\to\PRLie{n}$ preserves free objects. Thus, for $X\in\LL{n}$, we may study $(\derived\Ind{\LL{}})(X)$ using the Grothendieck spectral sequence derived in \cite{Blanc_Stover-Groth_SS.pdf}, which takes the form:
\[E^2_{st}=(\derived_s(\overline{\Ind{\PRLie{}}}_t))(\derived{\Ind{\nontop{}}})X\implies(\derived_{s+t}\Ind{\LL{}})X\textup{ with }d^{r}:E_{st}\to E_{s-r,t+r-1}.\]
Here, $(\derived{\Ind{\nontop{}}})X\in\PiAlg{\PRLie{n}}$, and $\overline{\Ind{\PRLie{}}}$ is the functor $\PiAlg{\PRLie{n}}\to\GR{n+1}$ induced by $\Ind{\PRLie{}}$. The $t$ subscript indicates the value of the second grading from the left, which arose as the homological degree in $(\derived{\Ind{\nontop{}}})X$. It is simpler to write:
\[E^2=(\derived\overline{\Ind{\PRLie{}}})(\derived{\Ind{\nontop{}}})X\implies(\derived\Ind{\LL{}})X\textup{ with }\deg(d^r)=(-r,r-1,0,\ldots,0).\]
Here, each $E^r$ is $(n+2)$-graded, while $(\derived\Ind{\LL{}})X$ is $(n+1)$-graded. Each page $E^r$ has two homological degrees in addition to the $n$ internal degrees of $X$, and the leftmost two degrees on $E^\infty$ sum to the leftmost degree in $(\derived\Ind{\LL{}})X$. It isn't difficult to identify $\overline{\Ind{\PRLie{}}}$:
\begin{prop}
Under the identification of $\PiAlg{\PRLie{}}$ with $\LL{n+1}$, the functor $\overline{\Ind{\PRLie{}}}:\PiAlg{\PRLie{}}\to\GR{n+1}$ coincides with ${\Ind{\LL{}}}:\LL{n+1}\to\GR{n+1}$.
%%%%%There's a commuting diagram of functors:
%%%%%\[\xymatrix@R=0mm@C=14mm{
%%%%%\PiAlg{\PRLie{n}}
%%%%%\ar[dr]^-{\overline{\Ind{\PRLie{n}}}}
%%%%%\ar[dd]_-{\cong}
%%%%%&%r1c1
%%%%%\\%r1c2
%%%%%&%r2c1
%%%%%\GR{n+1}\\%r2c2
%%%%%\LL{n+1}
%%%%%\ar[ur]_-{\Ind{\LL{n+1}}}
%%%%%&%r3c1
%%%%%%r3c2
%%%%%}\]
\end{prop}
\begin{proof}
As $\overline{\Ind{\PRLie{}}}$ is a $0^\textup{th}$ left derived functor, and $\Ind{\LL{}}$ preserves colimits, it is enough to check this claim on free objects in $\PiAlg{\PRLie{n}}$. All free objects are isomorphic to $\pi_*(\Fr{\PRLie{n}}SX)$ for some $X\in\GS{n+1}$, and
\[\overline{\Ind{\PRLie{}}}(\pi_*(\Fr{\PRLie{n}}SX)):=\pi_*(\Ind{\PRLie{}}(\Fr{\PRLie{n}}SX))=X=\Ind{\LL{}}(\pi_*(\Fr{\PRLie{n}}SX)).\qedhere\]
%The functor $\overline{\Ind{\PRLie{n}}}$ is the $0^\textup{th}$ left derived functor of the functor $\textup{free}(\PiAlg{\PRLie{n}})\to\GR{n+1}$ defined by $\pi_*(\Fr{\PRLie{n}}SX)\mapsto \pi_*(\Ind{\PRLie{n}}(\Fr{\PRLie{n}}SX))$ for $X\in\GS{n+1}$. One calculates $\pi_*(\Ind{\PRLie{n}}(\Fr{\PRLie{n}}SX))=\pi_*(SX)=X$
\end{proof}
\noindent We can now describe the Grothendieck spectral sequence relatively tidily:
\[E^2=(\derived{\Ind{\LL{}}})(\derived{\Ind{\nontop{}}})X\implies(\derived\Ind{\LL{}})X.\]
For emphasis, each page $E^r$, for $r\leq\infty$ is (at least) an element of  $\GR{n+2}$, with $d^r$ of degree $(-r,r-1,0,\ldots,0)$. The target, $(\derived\Ind{\LL{}})X$, is (at least) an object of $\GR{n+1}$. The spectral sequence is convergent, which is to say that there is a positive, increasing, exhaustive filtration $F$ on $(\derived\Ind{\LL{}})X$ such that
\[E^\infty_{a_{n+2},\ldots,a_1}=(F_{a_{n+2}}/F_{a_{n+2}-1})((\derived\Ind{\LL{}})X)_{a_{n+2}+a_{n+1},a_n,\ldots,a_1}.\]
%%%%%\begin{cor}
%%%%%For $n\geq1$, and any $X\in\LL{n}$, there is a spectral sequence:
%%%%%\[E^2=(\derived{\Ind{\LL{n+1}}})(\derived{\Ind{\nontop{}}})X\implies(\derived\Ind{\LL{n}})X.\]
%%%%%\begin{itemise}
%%%%%\setlength{\parindent}{.25in}
%%%%%\item For $r\leq\infty$, the $E^r$ page is an object of $\GR{n+2}$.
%%%%%\item For $r<\infty$, $d^r:E^r\to E^r$ has degree $(-r,r-1,0,\ldots,0)$.
%%%%%\item The target, $(\derived\Ind{\LL{n}})X$, is an object of $\GR{n+1}$.
%%%%%\item There is an increasing filtration $F$ on $(\derived\Ind{\LL{n}})X$ such that
%%%%%\[E^\infty_{a_{n+2},\ldots,a_1}=(F_{a_{n+2}}/F_{a_{n+2}-1})((\derived\Ind{\LL{n}})X)_{a_{n+2}+a_{n+1},a_n,\ldots,a_1}.\]
%%%%%\end{itemise}
%%%%%%The pages $E^r$ ($r\leq\infty$) are $(n+2)$-graded, with $\deg(d^r)=(-r,r-1,0,\ldots,0)$. The term $E^\infty_{a_{n+2},\ldots,a_1}$ is a subquotient of the degree $(a_{n+2}+a_{n+1},a_n,\ldots,a_1)$ part of the target.
%%%%%\end{cor}

\end{GrothendieckSpectralSequences}


\begin{ConversationWithHaynes_InclusionOfBarConstructions}
The stuff in here is what I wrote after a brief chat with Haynes about the inclusion of bar constructions that was causing me some confusion. There's a newer version in play.
%\begin{prop}
%For $n\geq1$, and any $X\in\LL{n}$, there's an isomorphism of vector spaces
%\[(\derived\Psi)(\forget{\nontop{}}X)\cong\forget{\GR{}}(\derived{{\Ind{\nontop{}}}})X,\]
%induced by the evident inclusion 
%\[\Ind{\nontop{}}B_\bullet(\Fr{\nontop{n}},\Fr{\nontop{n}},\forget{\nontop{}}X)\to \forget{\GR{}}\Psi B_\bullet(\Fr{\LL{n}},\Fr{\LL{n}},X).\]
%You might write this as $B_\bullet(\forgetSymbol,\Fr{\nontop{n}},\forget{\nontop{}}X)\to B_\bullet(\PRLie{n},\Fr{\LL{n}},X)$.
%\end{prop}
%\noindent In light of this result, we write $(\derived{\Ind{\nontop{}}})X$ for either $(\derived{\Ind{\nontop{}}})X$ or $(\derived{\Ind{\nontop{}}})\forgetSymbol(X)$, without ambiguity. \textbf{Discuss} the meaning of the fg and Rn in the bar construction.
%\begin{proof}
%As $\forgetSymbol:\LL{n}\to\nontop{n}$ preserves free objects and weak equivalences, there's a one-line Grothendieck spectral sequence producing the isomorphism, but this argument does not show that the isomorphism is induced by the inclusion of bar constructions. If $\forgetSymbol(B_\bullet(\LL{n},\LL{n},X))$ were almost free in $s\nontop{n}$, the map ${B}_\bullet(\nontop{n},\nontop{n},\forgetSymbol(X))\to{B}_\bullet(\LL{n},\LL{n},X)$ would be a map of cofibrant replacements of $\forgetSymbol(X)$ in $s\nontop{n}$, and thus 
%
%Or should I say: ``the obvious inclusion map $\Boverline_*(\nontop{n},\nontop{n},X)\to\Boverline_*(\LL{n},\LL{n},X)$ is a map of almost free objects realising the identity on $X$''? Is the latter actually almost free in $s\nontop{n}$?
%\end{proof}
%\begin{lem}
%For any $X$ in $s\LL{n}$, $(\derived{\Ind{\nontop{}}})(\forgetSymbol(X))$ may be calculated as the homotopy of the object 
%\[\forgetSymbol(\Ind{\nontop{}}(B_\bullet(\Fr{\LL{n}},\Fr{\LL{n}},X)))=\forgetSymbol(B_\bullet(\Fr{\PRLie{n}},\Fr{\LL{n}},X))\in s\GR{n}.\]
%\end{lem}
%\begin{proof}
%Although $B:=B_\bullet(\Fr{\LL{n}},\Fr{\LL{n}},X)$ is an acyclic levelwise free object in $s\nontop{n}$, with $\pi_0(B)=X$, it may fail to be `almost free'. The point here is that $B$ can be used to calculate the derived functors of $\Ind{\nontop{}}$ anyway. To see this, consider the bisimplicial object $A:=\Ind{\nontop{}}(B_\bullet(\Fr{\nontop{n}},\Fr{\nontop{n}},B))$ formed by applying $\Ind{\nontop{}}$ to the levelwise bar construction on $B$, so that
%\[A_{s,t}=(\Ind{\nontop{}})((\Fr{\nontop{n}})^{s+1})B_t\cong (\Fr{\nontop{n}})^sB_t.\]
%Now there are two first-quadrant spectral sequences converging to $\pi_*(\Delta A)$, one with ${E}^{2,I}=\pi_s\pi^e_tA$ and one with $E^{2,II}=\pi_t\pi^i_sA$. We'll show that they are both one-line spectral sequences by $E^2$, with ${E}^{2,I}_{s,0}=(\derived_s\Ind{\nontop{}})\forgetSymbol(X)$, and ${E}^{2,II}_{0,t}=\pi_t(\Ind{\nontop{}}B)$, proving the result.
%
%
%To identify ${E}^{2,I}_{s,t}$, write $A_{s,\bullet}$ for the singly simplicial object obtained by fixing the inner coordinate $s$. Then by lemma \ref{lemIteratedFreeUobjectPreservesAcyclicity}, $A_{s,\bullet}$ is acyclic, with $\pi_0 A_{s,\bullet}=(\Fr{\nontop{n}})^s\pi_0(B_\bullet)= (\Fr{\nontop{n}})^sX$. Now $\pi_0 A_{s,\bullet}$ is simplicial in $s$, and can be identified with the bar construction computing $(\derived_s{\Ind{\nontop{}}})\forgetSymbol(X)$.
%
%On the other hand, for fixed $t$, $A_{\bullet,t}$ is simply the bar construction calculating $(\derived\Ind{\nontop{}})B_t$. As $B_t$ is free for each $t$, $(\derived_0\Ind{\nontop{}})B_t=\Ind{\nontop{}}(B_t)$, and the higher derived functors vanish, identifying $E^{2,II}$.
%\end{proof}
%\begin{lem}\label{lemIteratedFreeUobjectPreservesAcyclicity}
%Suppose that $W\in s\nontop{n}$ is acyclic. Then for each $s\geq0$,  $(\Fr{\nontop{n}})^sW\in s\nontop{n}$ is acyclic, with $\pi_0((\Fr{\nontop{n}})^sW)=(\Fr{\nontop{n}})^s(\pi_0(W))$.
%\end{lem}
%\begin{proof}
%By induction on $s$ it is enough to prove the statement when $s=1$. The calculation of $\pi_0$ is immediate from the fact that $\Fr{\nontop{n}}$, being a left adjoint, preserves colimits.
%
%To approach the acyclicity question, we'll use unreduced chain complexes, writing $\partial:W_r\to W_{r-1}$ for the sum of the face maps. Now suppose that $a\in(\Fr{\nontop{n}}W)_r$ is a cycle. Without loss of generality, we may write
%$a:=\sum_j Q^{I_j}x_j$
%where all of the $I_j$ are distinct, and the $x_j$ are (potentially inhomogeneous) elements of $W_r$ (such that $Q^{I_j}x_j$ is defined). Then $\partial (a)=\sum Q^{I_j}\partial(x_j)=0$, which implies that each $\partial(x_j)$ is zero, so that $x_j=\partial(y_j)$ for some $y_j\in W_{r+1}$. Note that the internal dimensions of the homogeneous summands of $y_j$ may be taken to be the same as those of $x_j$, in which case $\sum Q^{I_j}y_j$ is defined in $\Fr{\nontop{n}}W_{r+1}$, and has boundary $a$.
%\end{proof}
\end{ConversationWithHaynes_InclusionOfBarConstructions}

\begin{Grothendieck Multiplicativity}
\section{Multiplicativity of the Grothendieck spectral sequence}
\subsection{Smash products}
Suppose that $\mathsf{C}$ is one of the above categories of Lie algebras, or something. Then for $X,Y\in\mathsf{C}$ define $X\wedge Y$ to be the kernel of the natural map $X\sqcup Y\to X\times Y$. 
\begin{prop}
There is a unique map $j:Q(X\wedge Y)\to QX\otimes QY$ such that
\[\lambda_{i_1}\cdots \lambda_{i_r}[z_1,\cdots ,z_a]\overset{j}{\mapsto}\begin{cases}
%0,&\textup{if }r>0\textup{ or }s>2;\\
z_1\otimes z_2,&\textup{if }r=0,\,a=2,\,z_1\in X\textup{ and } z_2\in Y,
\\0,&\textup{otherwise, }
\end{cases}
\]
where each of $z_1,\ldots,z_a$ is an element of either $X$ or $Y$, and $[z_1,\ldots,z_s]$ represents some choice of bracketing. It is natural in $X$ and $Y$.
\end{prop}
%, and by
%\[[x,y]\mapsto x\otimes y\textup{ where }x\in X,\ y\in Y.\]
\begin{proof}
There's a commuting diagram in $\mathsf{C}$
\[\xymatrix@R=4mm{
\mathsf{C}\mathsf{C}X\wedge \mathsf{C}\mathsf{C}Y
\ar[d]^-{\textup{in}_{\mathsf{C}\mathsf{C}}}
\ar@<.3ex>[r]^-{\alpha_{\wedge }}
\ar@<-.3ex>[r]_-{\beta_{\wedge }}
&%r1c1
\mathsf{C}X\wedge \mathsf{C}Y
\ar[d]^-{\textup{in}_{\mathsf{C}}}
\ar[r]^-{\rho\wedge \rho}
&%r1c2
X\wedge Y
\ar[d]%^-{\textup{in}}
\ar[r]
%\ar@{..>}[dr]_-{j'}
&
Q(X\wedge Y)
\ar@{..>}[d]^-{j}
\\
\mathsf{C}(\mathsf{C}X\oplus\mathsf{C}Y)
\ar[d]
\ar@<.3ex>[r]^-{\alpha_{\sqcup}}
\ar@<-.3ex>[r]_-{\beta_{\sqcup}}
&%r1c1
\mathsf{C}(X\oplus Y)
\ar@{..>}@/_1em/[rr]_-(.85){j''}
\ar[d]
\ar[r]^-{\rho\sqcup \rho}
&%r1c2
X\sqcup Y
\ar[d]
\ar@{..>}[r]^-{j'}
&
QX\otimes QY
\\
\mathsf{C}\mathsf{C}X\times \mathsf{C}\mathsf{C}Y
\ar@<.3ex>[r]^-{\alpha_{\times}}
\ar@<-.3ex>[r]_-{\beta_{\times}}
\ar@{..>}@/^.75em/@<0ex>[u]
&%r1c1
\mathsf{C}X\times \mathsf{C}Y
\ar[r]^-{\rho\times \rho}
\ar@{..>}@/^.75em/@<0ex>[u]
&%r1c2
X\times Y
\ar@{..>}@/^.75em/@<0ex>[u]
}\]
Here, $\alpha_{\sqcup}$ is induced by the inclusion $\mathsf{C}(X)\oplus\mathsf{C}(Y)\to\mathsf{C}(X\oplus Y)$ and $\alpha_{\times}=\mu\times\mu$ (writing $\mu:\mathsf{C}\mathsf{C}\to\mathsf{C}$), while $\beta_{\sqcup}$ is induced by $\rho\oplus\rho:\mathsf{C}(X)\oplus\mathsf{C}(Y)\to X\oplus Y$ and $\beta_{\times}=\mathsf{C}(\rho)\times\mathsf{C}(\rho)$ (writing $\rho:\mathsf{C}X\to X$ for the action map). Finally, $\alpha_{\wedge }$ and $\beta_{\wedge }$ are just the restrictions of $\alpha_{\sqcup }$ and $\beta_{\sqcup }$.

The three-by-three square above has columns short exact sequences which commute with the horizontal maps. The columns have an evident splitting (of vectorspaces), which also commutes with the horizontal maps. The middle two rows are evidently coequalizers (of vectorspaces), and in light of the splittings, the tow row must be a coequalizer.

The specifications in the proposition's statement define a map $j''$.
As the top row is a coequaliser diagram of vectorspaces, $j''$ induces%footnote:
\footnote{I once thought that actually only the composite with $\textup{in}_{\mathsf{C}\mathsf{C}}$ was zero. Was I right all along?} a map $j'$ as long as $j''\circ(\alpha_{\sqcup}-\beta_{\sqcup})=0$. Then $j'$ will induce a map $j$ as long as $j'$ vanishes on nontrivial operations in $X\wedge Y$.

The vectorspace $\mathsf{C}(X\oplus Y)$ is spanned by formal expressions $e^0(x_i,y_k)$. The vectorspace $\mathsf{C}(\mathsf{C}(X)\oplus\mathsf{C}(Y))$ is spanned%footnote:
\footnote{Now irrelevant: its subspace $\mathsf{C}\mathsf{C}X\wedge \mathsf{C}\mathsf{C}Y$ is spanned by those in which $e^0$ only contains terms which involve some element of both $X$ and $Y$.} by doubly-formal expressions $E=e^0(e^1_{i_1}(x_{i_1i_2}),e^1_{k_1}(y_{k_1k_2}))$. The image of this expression under the map $(\alpha_{\sqcup}-\beta_{\sqcup})$ is
\[e^0(e^0_{i_1}(x_{i_1i_2}),e^0_{k_1}(y_{k_1k_2}))-e^0(e_{i_1}(x_{i_1i_2}),e_{k_1}(y_{k_1k_2})).\]
Consider then a monomial summand of $e^0$, when $E\in \mathsf{C}(\mathsf{C}X\oplus \mathsf{C}Y)$. If this summand is not simply of the form $[e_1^1(x_i),e_2^1(y_k)]^0$, then it has no contribution to the value of $(j''\circ(\alpha_{\sqcup}-\beta_{\sqcup}))(E)$. Thus we may consider only summands of this form, which under $j''\circ (\alpha_{\sqcup}-\beta_{\sqcup})\circ $ map to
\[[e^0_{i}(x_{i}),e^0_{k}(y_{k})]^0-[e_{i}(x_{i}),e_{k}(y_{k})]^0.\]
Unless both of $e_{i}$ and $e_{k}$ are just the identity, both of these terms are sent to zero by $j''$. If both are the identity, so that the element we're looking at is $[x,y]^0$, it happens that $(\alpha_{\sqcup}-\beta_{\sqcup})E=0$.
This demonstrates that the extension $j'$ exists.
That $j'$ extends to $j$ follows from the fact that $\rho\wedge \rho$ is surjective and the observation that any nontrivial operation on elements of $\mathsf{C}X\wedge \mathsf{C}Y$ will create more complicated%footnote:
\footnote{Complexity should be measured by quadratic grading.} expressions than those of the form $[x,y]^0$.
\end{proof}

\subsection{Coproducts in the simplicial bar construction}
Suppose that $X\in\LL{}$. We may construct a coproduct (although I'm not sure if it's homotopy coassociative, etc.) on the simplicial bar construction $\Q_{\LL{}}[\Fr{\LL{}}]^{\bullet+1}X$, using Goerss' method. First we'll construct a map $\overline{\xi}_{\LL{}}$ with a factoring $\xi_{\LL{}}$ as follows:
\[\xymatrix@R=4mm{
[\Fr{\LL{}}]^{n+1}X\ar[rr]^-{\overline{\xi}_{\LL{}}}
\ar@{-->}[rd]_-{\xi_{\LL{}}}
&&%r1c1
([\Fr{\LL{}}]^{n}X)^{\sqcup2}\\%r1c3
&%r2c1
([\Fr{\LL{}}]^{n}X)^{\wedge 2}
\ar@{ >->}[ur]
}\]
Then the chain level coproduct will be given by 
\[\Q_{\LL{}}[\Fr{\LL{}}]^{n+1}X\overset{Q(\xi_{\LL{}})}{\to} \Q_{\LL{}}([\Fr{\LL{}}]^{n}X)^{\wedge 2}\overset{j}{\to} (\Q_{\LL{}}[\Fr{\LL{}}]^{n}X)^{\otimes 2}.\]
One constructs the map $\overline{\xi}_{\LL{}}$ using a diagonal on the free construction, the map $\psi_{\LL{}}$ induced, for $Y\in \GR{}$, by the diagonal $Y\to Y^{\oplus2}$:
\[\psi_{\LL{}}:\Fr{\LL{}}Y\to (\Fr{\LL{}}Y)^{\sqcup2}\]
\begin{lem}
For $Y\in\GR{}$, $\Fr{\LL{}}Y$ is naturally a (strict) commutative cogroup object, having comultiplication map $\psi_{\LL{}}$, counit map $0:\Fr{\LL{}}Y\to 0$, and inverse map $\textup{id}:\Fr{\LL{}}Y\to \Fr{\LL{}}Y$. In particular, $\hom(\Fr{\LL{}}Y,\DASH)$ takes values in $\ensuremath{\F_2}$-vectorspaces. %The map $\psi$, along with the zero map $WX\to0$.
\end{lem}
We then may define $\overline{\xi}_{\LL{}}$ to be the sum of the maps $((d_0)^{\wedge 2})\psi_{\LL{}}$ and $\psi_{\LL{}}d_0$ in the group $\Hom({[\Fr{\LL{}}]}^{n+1}X,({[\Fr{\LL{}}]}^{n}X)^{\sqcup2})$. Then, following Goerss, one shows that the image of $\overline{\xi}_{\LL{}}$ is contained in the smash power, giving a factorization $\xi_{\LL{}}$.
\begin{prop}
The image of $f^0(g^1_{i_1}(h^2_{i_1i_2}))\in \Q_{\LL{}}[\Fr{\LL{}}]^{n+1}X$ under the map $j\circ Q(\xi_{\LL{}})$ is given by $q(uf(g_{i_1}))(h^1_{i_1i_2})$.
\end{prop}
To understand the notation here, we mean that the $h^2_{i_1i_2}$ are elements of $[\Fr{\LL{}}]^{n-1}X$, so that they are two free constructions into the bar construction, that the $g^1_{i_1}$ are $\LL{}$ operators applied to the $h^2_{i_1i_2}$ to obtain elements of $[\Fr{\LL{}}]^{n}X$, and that $f$ is a $\LL{}$ operator. In the image of the morphism being described, $h$ has superscript 1, to indicate that it is shallower in the image than it was in the source.
\begin{proof}
The element $f^0(g^1_{i_1}(h^2_{i_1i_2}))\in \Q_{\LL{}}[\Fr{\LL{}}]^{n+1}X$ \emph{equals} the element $uf(g^1_{i_1}(h^2_{i_1i_2}))$, since we are calculating in the indecomposables. Thus, we may assume that $f$ is just the identity, so that there is only one index $i_1$, and show that
\[\textup{elt}:=g^1_{i_1}(h^2_{i_1i_2})\overset{j\circ Q(\xi_{\LL{}})}{\mapsto} q(g^1_{i_1})(h^1_{i_1i_2}).\]
Now that we've assumed that $\textup{elt}$ is in the image of $s_{-1}$, $\overline{\xi}_{\LL{}}(\textup{elt})$ may be calculated as the sum of $(d_0)^{\sqcup 2}(\psi_{\LL{}}(\textup{elt}))$ and $\psi_{\LL{}}(d_0(\textup{elt}))$. In order to denote elements of the self-coproduct, we'll write expressions like $\overline{x}+\overline{\overline{x}}$. Then
\begin{gather*}
\textup{elt}\overset{\psi_{\LL{}}}{\mapsto}\overline{\textup{elt}}+\overline{\overline{\textup{elt}}} \overset{(d_0)^{\sqcup 2}}{\to}\overline{g^0_{i_1}(h^1_{i_1i_2})}+ \overline{\overline{g^0_{i_1}(h^1_{i_1i_2})}},\textup{ and}\\
\textup{elt}\overset{d_0}{\mapsto} g^0_{i_1}(h^1_{i_1i_2})\overset{\psi_{\LL{}}}{\to} g^0_{i_1}\left(\overline{h^1_{i_1i_2}}+ \overline{\overline{h^1_{i_1i_2}}}\right).
\end{gather*}
The image of $\textup{elt}$ is then $j(\textup{cr}(g^0_{i_1})(h^1_{i_1i_2}))$, where
\[\textup{cr}(g^0_{i_1})(h^1_{i_1i_2}):=
g^0_{i_1}\left(\overline{h^1_{i_1i_2}} +\overline{\overline{h^1_{i_1i_2}}}\right)
-\overline{g^0_{i_1}(h^1_{i_1i_2})}- \overline{\overline{g^0_{i_1}(h^1_{i_1i_2})}}.\]
Now $j:Q(A\wedge B)\to QA\otimes QB$ picks out the simple brackets $[a,b]$ for $a\in A$ and $b\in B$, and in the expression $\textup{cr}(g^0_{i_1})(h^1_{i_1i_2})$, there are only two sources for such brackets. Firstly, if $g^0_{i_1}$ contains a bracket $[h^1_{i_1i_2},h^1_{i_1i'_2}]$ then $\textup{cr}(g^0_{i_1})(h^1_{i_1i_2})$ contains terms
\[[\overline{h^1_{i_1i_2}},\overline{\overline{h^1_{i_1i'_2}}}]+ [\overline{\overline{h^1_{i_1i_2}}},\overline{h^1_{i_1i'_2}}]\overset{j}{\mapsto}h^1_{i_1i_2}\otimes h^1_{i_1i'_2}+h^1_{i_1i'_2}\otimes h^1_{i_1i_2}.\]
Secondly, if $g^0_{i_1}$ contains a restriction $(h^1_{i_1i_2})^{[2]}$, then $\textup{cr}(g^0_{i_1})(h^1_{i_1i_2})$ contains
\[(\overline{h^1_{i_1i_2}}+\overline{\overline{h^1_{i_1i_2}}})^{[2]}-
(\overline{h^1_{i_1i_2}})^{[2]}
-(\overline{\overline{h^1_{i_1i_2}}})^{[2]}
=[\overline{h^1_{i_1i_2}},\overline{\overline{h^1_{i_1i_2}}}]\overset{j}{\mapsto}h^1_{i_1i_2}\otimes h^1_{i_1i_2}.\qedhere\]
\end{proof}

\subsection{Homotopy of a smash}
Suppose that $X$ and $Y$ are models in simplicial Lie algebras, which is to say that $X$ and $Y$ are each (almost free, and) weakly equivalent to a wedge of spheres. 

The natural map $i:(\pi X)\sqcup (\pi Y)\to \pi(X\sqcup Y)$ is an isomorphism, since both source and target represent the free Lie-$\Pi$-algebra on generators corresponding to the wedge summands of $X$ and $Y$ taken together. Moreover, the natural map $\pi(X\times Y)\to (\pi X)\times (\pi Y)$ is an isomorphism (as products of Lie algebras are taken in vector spaces). Thus we have a commuting square
\[\xymatrix@R=4mm{
(\pi X)\sqcup (\pi Y)
\ar[r]
\ar[d]^-{i}_-{\cong}
&%r1c3
(\pi X)\times (\pi Y)
\ar[d]_-{\cong}\\
\pi (X\sqcup Y)
\ar[r]
&%r1c3
\pi (X\times Y)
}\]
This shows that the long exact sequence coming from $X\wedge Y\to X\sqcup Y\to X\times Y$ is just a short exact sequence, and that there is a commuting diagram
\[\xymatrix@R=4mm{
0\ar[r]&%r1c1
(\pi X)\wedge (\pi Y)
\ar[r]
\ar[d]^-{i}_-{\cong}
&%r1c2
(\pi X)\sqcup (\pi Y)
\ar[r]
\ar[d]^-{i}_-{\cong}
&%r1c3
(\pi X)\times (\pi Y)
\ar[r]
\ar[d]_-{\cong}
&%r1c4
0\\%r1c5
0\ar[r]&%r1c1
\pi (X\wedge  Y)
\ar[r]
&%r1c2
\pi (X\sqcup Y)
\ar[r]
&%r1c3
\pi (X\times Y)
\ar[r]
&%r1c4
0
}\]
In particular, for $X$ and $Y$ models, there is an isomorphism $i:(\pi X)\wedge (\pi Y)\to\pi(X\wedge Y)$ (natural for maps of pairs of models).

For any simplicial object $A\in s\PRLie{}$, there's a natural map $\gamma:Q_{\LL{}}\pi A\to \pi Q_{\PRLie{}}A$.
\begin{lem}
If $A$ is a model, then $\gamma:Q\pi A\to \pi QA$ is an isomorphism.
\end{lem}
\begin{proof}
$A$ is homotopic to a wedge of spheres, and $\pi A$ is free on generators in correspondence with the wedge summands.
\end{proof}

\subsection{A commuting diagram}



Supposing still that $X$ and $Y$ are models in simplicial Lie algebras, we now have a commuting diagram:
\[
\xymatrix@C-2em{
& \pi Q(X\wedge Y) \ar[rr]^{\pi(j)} && \pi(QX\otimes QY) \ar[dr]^{\textup{AW}}_{\cong} \\
Q\pi(X\wedge Y) \ar[ur]^{\gamma} &&&& (\pi QX)\otimes (\pi QY) \ar[dl];[]^{\gamma\otimes\gamma}_{\cong} \\
& Q((\pi X)\wedge (\pi Y)) \ar[ul]^{\cong}_{Q(i)} && (Q\pi X)\otimes (Q\pi Y) \ar[ll];[]^{j}
}
\]


For notational convenience, write $\textup{br}:\pi X\otimes \pi Y\to \pi X\wedge \pi Y$ for the bracket $a\otimes b\mapsto [a,b]$ on homotopy, and similarly write $\textup{br}:X\otimes Y\to X\wedge Y$ for the chain level bracket.

\begin{proof}
Any element of $Q((\pi X)\wedge (\pi Y))$ is the class of some $E\in (\pi X)\wedge (\pi Y)$. We may write $E=\textup{br}(\overline{x}\otimes\overline{y})+E'$ where $\overline{x}\in\pi X$ and $\overline{y}\in\pi Y$ are represented by $x\in N_*X$ and $y\in N_*Y$, and $E'\in(\pi X)\wedge (\pi Y)$ is a sum of terms which are $\lambda$-operations applied to 3-fold brackets of elements of $\pi X$ and $\pi Y$. The map $Q(i)$ is induced by the Eilenberg-Zilber formula, and
\[\textup{br}(\overline{x}\otimes\overline{y})\overset{\gamma\circ Q(i)}{\mapsto}\overline{\textup{br}(\textup{EZ}(x\otimes y))}\overset{\pi(j)}{\mapsto}\overline{(\textup{EZ}(x\otimes y))}\overset{\textup{AW}}{\mapsto}\overline{x}\otimes \overline{y}.\]
The last mapping follows from the fact that $\textup{AW}\circ\textup{EZ}=\textup{id}$.
Similar considerations show that $E'$ is annihilated by $\pi(j)\circ\gamma\circ Q(i)$. As $E'$ is also annihilated by $j$, and $j(\textup{br}(\overline{x}\otimes\overline{y}))=\overline{x}\otimes\overline{y}$, the diagram commutes.
\end{proof}
\subsection{The $W$ construction on simplicial Lie algebras}
Blanc and Stover \cite{Blanc_Stover-Groth_SS.pdf} define a comonad $W$ on $s\PRLie{}$ by defining $WZ$ to be the pushout
\[\xymatrix@R=4mm{
\bigvee_{h}S^n
\ar[r]\ar[d]
&%r1c1
\bigvee_{f}S^n_f\ar[d]\\%r1c2
\bigvee_{h}CS^n_h
\ar[r]&%r2c1
WX%r2c2
}\]
Then for each map $f:S^n\to X$ that extends to a map $CS^n\to X$, choose an extension $h_0(f)$, so that $h_0(f)\imath=f$, writing $\imath:S^n\to CS^n$ for the standard inclusion. Then $WX$ contains the contractible subobject $C_0:=\bigvee_f CS^n_{h_0(f)}$. Moreover, $WX/C_0$ is a wedge of spheres:
\[WX/C_0\cong \left(\bigvee_{h\textup{ not chosen}}CS^n_h/\partial(CS^n_h)\right)\vee\left(\bigvee_{f\textup{ not null}}S^n_f\right)\]
In particular, $\pi(WX)$ is a free object in $\LL{}$ with generators in bijection with the above wedge summands. Now there is a bisimplicial object $W^{n+1}X$ using the comonad structure, and we may choose the contractible subojects of each $W^{n+1}X$ in such a way that the degeneracies $W^{n+1}X\to W^{n+2}X$ are homotopic to wedge inclusions of spheres, c.f.\ \cite{StoverVanKampen.pdf}. In particular, the degeneracies of $\pi(W^{n+1}X)\in s\LL{}$ preserve generators.

Now suppose that $a:X\to Y$ is a map in $s\PRLie{}$. Then the natural map $WX\to WY$ sends $S^n_f$ and $CS^n_h$ identically onto $S^n_{af}$ and $CS^n_{ah}$ respectively. Representatives for the generators of $\pi(WX)$ are 
\[\{CS^n_h-CS^n_{h_0(h\imath)}\ |\ \textup{$h$ was not chosen}\}\sqcup\{S^n_f\ |\ \textup{$f$ not null}\}.\]
Evidently $Wa$ sends each $S^n_f$ either to a generator of $\pi(WY)$ or zero (depending on whether $af$ is null). Moreover,
\begin{alignat*}{2}
Wa(CS^n_h-CS^n_{h_0(h\imath)})
&=
CS^n_{ah}-CS^n_{ah_0(h\imath)}%
\\
&=
\left(CS^n_{ah}-CS^n_{h_0(ah\imath)}\right)-\left(CS^n_{ah_0(h\imath)}-CS^n_{h_0(ah\imath)}\right)%
\\
&=
\left(CS^n_{ah}-CS^n_{h_0(ah\imath)}\right)-\left(CS^n_{ah_0(h\imath)}-CS^n_{h_0(ah_0(h\imath)\imath)}\right)%
\end{alignat*}
is a difference of two generators of $\pi(WY)$ (the last equality here was simply the substitution $h\imath=h_0(h\imath)\imath$). So, if by almost free we mean that there are vector spaces of generators which are preserved by everything except $d_0$, we have shown that $\pi(W^{\bullet+1}X)$ is an almost free object in $s\LL{}$.
\subsection{A chain level cogroup structure}
We showed that $\pi(WX)\in\LL{}$ is a free object, say $\pi(WX)\cong\Fr{\LL{}}(V)$, where $V$ is the vector space spanned by the classes of the representatives above. We'll now define a natural map $\psi_{\textup{ch}}:WX\to WX\sqcup WX$ which on homotopy induces the same map $\Fr{\LL{}}(V)\to \Fr{\LL{}}(V)\sqcup \Fr{\LL{}}(V)\cong \Fr{\LL{}}(V\oplus V)$ as is induced by the diagonal map $V\mapsto V\oplus V$. It is enough to construct a commuting diagram
\[\xymatrix@R=4mm{
S^n\ar[r]^-{\psi_1}
\ar[d]^-{\imath}
&%r1c1
S^n\sqcup S^n\ar[d]^-{\imath\sqcup\imath}
\\%r1c2
CS^n\ar[r]^-{\psi_2}
&%r2c1
CS^n\sqcup CS^n%r2c2
}\]
In fact, this diagram can be formed in $s\GR{}$, before application of $\Fr{\PRLie{}}:\GR{}\to\PRLie{}$, and $\psi_1$ and $\psi_2$ are just the levelwise application of the diagonal map on a simplicial vector space. To understand the effect of $\psi_{\textup{ch}}$ on homotopy, it is enough to identify where the generators of $\pi(WX)$ are sent in $\pi(WX)\sqcup\pi(WX)$, which is easy.
\begin{lem}
$WX$ is naturally a (strict) commutative cogroup object, having comultiplication map $\psi_{\textup{ch}}$, counit map $0:WX\to 0$, and inverse map $\textup{id}:WX\to WX$. In particular, $\hom(WX,\DASH)$ takes values in $\ensuremath{\F_2}$-vectorspaces. %The map $\psi$, along with the zero map $WX\to0$.
\end{lem}
\begin{proof}
There are a few axioms to check, for example, left counitality: that the composite 
\[\xymatrix@R=4mm@1{
WX\ar[r]^-{\psi_{\textup{ch}}}
&%r1c1
WX\sqcup WX\ar[r]^-{\textup{id}\sqcup0}
&%r1c2
WX%r1c3
}\] is the identity. This follows since $(\textup{id}\sqcup0)\psi_1$ is the identity of $S^n$ and $(\textup{id}\sqcup0)\psi_2$ is the identity of $CS^n$. The other axioms follow similarly.
\end{proof}
Using the notation $*$ for the group operation on $\hom_{s\PRLie{}}(WX,Y)$, we have the following:
\begin{lem}
For maps $f,g:WX\to Y$ we have $Q(f*g)=Qf+Qg$.
\end{lem}
\begin{proof}
It's enough to check that $Q(\psi_\textup{ch}):QWX\to Q(WX\sqcup WX)$ equals the diagonal map $QWX\to QWX\oplus QWX$. For this, $Q$ converts all the colimits involved in the construction of $WX$ to colimits in $s\GR{}$, and $\psi_1$ and $\psi_2$ are both precisely the diagonal map.
\end{proof}

Now let $\overline{\xi}_{\textup{ch}}$ denote the following composite:
%\[W^2X\overset{\psi_{\textup{ch}}}{\to}(W^2X)^{\sqcup2}\overset{\psi_{\textup{ch}}\sqcup \epsilon}{\to}(W^2X)^{\sqcup2}\sqcup(WX)\overset{\epsilon^{\sqcup2}\sqcup\psi_{ch}}{\to}((WX)^{\sqcup2})^{\sqcup2}\overset{\textup{fold}}{\to}(WX)^{\sqcup2}\]
\[\makebox[0mm][r]{$\overline{\xi}_{\textup{ch}}:\ $}W^2X\overset{\psi_{\textup{ch}}}{\to}(W^2X)^{\sqcup2}\overset{a\sqcup b}{\to}(WX)^{\sqcup2}\]
where $a,b:W^2X\to(WX)^{\sqcup2}$ are the composites
\[\xymatrix@R=2mm{
\makebox[0mm][r]{$a:\ $}W^2X\ar[r]^-{\psi_{\textup{ch}}}
&%r1c1
(W^2X)^{\sqcup2}\ar[r]^-{\epsilon^{\sqcup2}}
&%r1c2
(WX)^{\sqcup2}\\
\makebox[0mm][r]{$b:\ $}W^2X\ar[r]^-{\epsilon}
&%r1c1
(WX)\ar[r]^-{\psi_{\textup{ch}}}
&%r1c2
(WX)^{\sqcup2}
}\]
\begin{lem}
The composite 
$W^2X\overset{\overline{\xi}_{\textup{ch}}}{\to}(WX)^{\sqcup2}\to(WX)^{\times2}$ is zero.
\end{lem}
\begin{proof}
This follows from the easy observation that both composites $(\textup{id}\sqcup0)\overline{\xi}_{\textup{ch}}$ and $(0\sqcup\textup{id})\overline{\xi}_{\textup{ch}}$ equal $\epsilon:W^2X\to WX$.
\end{proof}
In particular, $\overline{\xi}_{\textup{ch}}$ factors through the smash product, defining a natural map $\xi_{\textup{ch}}:W^2X\to (WX)^{\wedge 2}$.
\subsection{The resolution $\scrW_\bullet X$ and the chain map $j\circ Q\xi_\textup{ch}$}
For $X\in s\PRLie{}$, we have a bisimplicial object $\scrW X$ defined by $\scrW_\bullet X=W^{\bullet+1} X$, the comonadic bar construction. Then the maps $\xi_\textup{ch}:\scrW_\bullet X\to (\scrW_{\bullet-1} X)^{\wedge 2}$ are maps in $s\PRLie{}$, but evidently do not assemble to a map of bisimplicial objects -- they don't respect the filtration degree. To control this, we'll need the following lemma, importing \cite[lemma 5.5]{MR1089001}:
\begin{lem}
$(d_i)^{\wedge 2}\xi_\textup{ch}=\xi_\textup{ch}d_{i+1}$ for $i\geq1$, and $(d_0)^{\wedge 2}\xi_\textup{ch}=(\xi_\textup{ch}d_{0})*(\xi_\textup{ch}d_{1})$. Thus $\Q(\xi_\textup{ch})$ induces a degree $(-1,0)$ bicomplex map 
\[N_*^\textup{fil}N_*^\textup{int}(\Q\scrW X)\to N_*^\textup{fil}N_*^\textup{int}(\Q((\scrW X)^{\wedge 2})).\]
\end{lem}
\begin{proof}
%Comm diagrams, for $i\geq1$:
%\[\xymatrix@R=4mm{
%\calW_{\bullet}X
%\ar[r]^-{a}
%\ar[d]^-{d_{i+1}}
%&%r1c1
%(\calW_{\bullet-1}X)^{\sqcup2}
%\ar[d]^-{(d_i)^{\sqcup2}}
%&%r1c2
%\calW_{\bullet}X
%\ar[r]^-{b}
%\ar[d]^-{d_{i+1}}
%&%r1c1
%(\calW_{\bullet-1}X)^{\sqcup2}
%\ar[d]^-{(d_i)^{\sqcup2}}\\%r1c5
%\calW_{\bullet-1}X
%\ar[r]^-{a}
%&%r2c1
%(\calW_{\bullet-2}X)^{\sqcup2}&%r2c2
%\calW_{\bullet-1}X
%\ar[r]^-{b}
%&%r2c1
%(\calW_{\bullet-2}X)^{\sqcup2}
%}\]
%That does the first part. For the second, 
Entirely analogous to the proof of \cite[lemma 5.5]{MR1089001}.
\end{proof}
We're interested in the composite $\Delta_\textup{ch}=j\circ Q(\xi_\textup{ch})$:
\[N_*^\textup{fil}N_*^\textup{int}(\Q\scrW X)\overset{\Q(\xi_\textup{ch})}{\to} N_*^\textup{fil}N_*^\textup{int}(\Q((\scrW X)^{\wedge 2}))\overset{j}{\to} N_*^\textup{fil}N_*^\textup{int}((\Q\scrW X)^{\otimes 2})\]
which will induce a coproduct on the pages of the spectral sequence
\begin{prop}
The coproduct $\Delta_\textup{ch}:N_*^\textup{int}(QW^{s+1}X)\to N_*^\textup{int}((QW^{s}X)^{\otimes2})$ is given by the formula
\[k_{e(s_{\alpha_1}k_{a_1},\ldots,s_{\alpha_m}k_{a_m})}\overset{\Delta_\textup{ch}}{\mapsto}q(e)(s_{\alpha_1}k_{a_1},\ldots,s_{\alpha_m}k_{a_m}).\]
\end{prop}
To explain this notation, the $a_j$ are elements of $N_*(W^{s-1}X)$, and each $k_{a_j}\in W^sX$ are each either a sphere or cone class corresponding to $a_j$. The $s_{\alpha_j}$ are degeneracies (so that all of the $s_{\alpha_j}k_{a_j}$ are in the same simplicial degree). Finally, $e$ is some Lie operator applied to these classes, so that $e(\cdots )\in N_*(W^sX)$, and we can write $k_{e(\cdots )}$ to mean either a sphere or cone class in $W^{s+1}X$.
\begin{proof}
This is formally similar to the corresponding fact for $\Delta_{\LL{}}$.
\end{proof}
\subsection{Identification of the coproduct at $E_2$}
\begin{thm}
For $X\in s\PRLie{}$, write $Y\in ss\PRLie{}$ for the bisimplicial object obtained from $X$ by repeated application of $W$. There is a commuting diagram
\[\xymatrix@C=13mm@R=4mm{
((N\pi^{\textup{in}}QY)^{\otimes2})_n&%r1c1
((NQ\pi^{\textup{in}}Y)^{\otimes2})_{n}
\ar[l]_-{N(\gamma)\otimes N(\gamma)}
\\%r1c2
N_n((\pi^{\textup{in}}QY)^{\otimes2})
\ar[u]^-{\textup{AW}^\textup{out}}
&%r2c1
N_n((Q\pi^{\textup{in}}Y)^{\otimes2})
\ar[u]_-{\textup{AW}}
\ar[l]_-{\gamma\otimes\gamma}
\\%r2c2
N_n(\pi^\textup{in}((QY)^{\otimes2}))
\ar[u]^-{\pi(\textup{AW}^\textup{in})}
&%r3c1
\\%r3c2
N_{n+1}(\pi^\textup{in} QY)
\ar[u]^-{\pi(j\circ Q(\xi_\textup{ch}))}
&%r4c1
N_{n+1}(Q\pi^\textup{in} Y)
\ar[uu]_-{j\circ Q(\xi_{\LL{}})}
\ar[l]_-{\gamma}
%r4c2
}\]
with all the horizontal maps isomorphisms, and with the two vertical composites chain maps. This demonstrates that the products on $E_2$ coincide with those defined on the whole spectral sequence using $j\circ Q(\xi_\textup{ch})$.
\end{thm}
\begin{proof}
The top square is easy. The bottom square may be expanded into the following larger but simpler commuting diagram:
\[
\def\objectstyle{\scriptstyle}
\xymatrix@R=8mm@C=13mm@!0{
%&%r1c1
%(\pi (QW^{n+1}X))^{\otimes2}\ar[rrrrrr];[]_-{=}
%&%r1c2
%&%r1c3
%&%r1c4
%&%r1c5
%&%r1c6
%&%r1c7
%(\pi (QW^{n+1}X))^{\otimes2}\\\\%r1c8
&%r2c1
\pi((QW^{n+1}X)^{\otimes2})\ar[rrr]^-{\textup{AW}_{\textup{in}}}
&%r2c2
&%r2c3
&%r2c4
(\pi (QW^{n+1}X))^{\otimes2}&%r2c5
&%r2c6
&%r2c7
(Q\pi(W^{n+1}X))^{\otimes2}\ar[lll]_-{\gamma^{\otimes2}}\\\\%r2c8
&%r3c1
\pi Q((W^{n+1}X)^{\wedge 2})\ar[uu]^-{\pi(j)}
\ar[dl]
&%r3c2
&%r3c3
&%r3c4
Q\pi((W^{n+1}X)^{\wedge 2})\ar[lll]_-{\gamma}\ar[dl]
&%r3c5
&%r3c6
&%r3c7
Q((\pi( W^{n+1}X))^{\wedge 2})\ar[dl]
\ar[uu]_-{j}
\ar[lll]_-{Q(i)}
\\
%r3c1
\pi Q((W^{n+1}X)^{\sqcup 2})
&%r3c2
&%r3c3
&%r3c4
Q\pi((W^{n+1}X)^{\sqcup 2})\ar[lll]_-(.3){\gamma}
&%r3c5
&%r3c6
&%r3c7
Q((\pi( W^{n+1}X))^{\sqcup 2})
\ar[lll]_-(.3){Q(i)}
%r3c8
\\\\
&%r3c1
\pi Q(W^{n+2}X)
\ar@/_.5em/[uuu]_-(.25){\pi Q(\xi_{\textup{ch}})}
\ar[uul]^-(.65){\pi Q(\overline{\xi}_{\textup{ch}})}
&%r3c2
&%r3c3
&%r3c4
Q\pi(W^{n+2}X)\ar[lll]_-{\gamma}
\ar@/_.5em/[uuu]_-(.25){Q\pi(\xi_{\textup{ch}})}
\ar[uul]^-(.65){Q\pi(\overline{\xi}_{\textup{ch}})}
&%r3c5
&%r3c6
&%r3c7
Q(\pi( W^{n+2}X))
\ar@/_.5em/[uuu]_-(.25){Q(\xi_{\LL{}})}
\ar[lll]_-{=}
\ar[uul]^-(.65){Q(\overline{\xi}_{\LL{}})}
}\]
The top hexagon is from above. The left triangular prism commutes simply by naturality of $\gamma$. The right prism's commutativity reduces to checking the fact that the composite $\pi(WZ)\overset{\psi_{\LL{}}}{\to}\pi(WZ)^{\sqcup2}\overset{i}{\to}\pi((WZ)^{\sqcup2})$ equals $\pi(\psi_{\textup{ch}})$. One may check this on generators, where it is obvious.
\end{proof}
\subsection{Identification of the coproduct at $E_\infty$}
In order to identify the coproduct at $E_\infty$, we'll focus on a diagram
\[\xymatrix@R=4mm{
T_{n+1}(Q\scrW X)
\ar[r]^-{\Delta_\textup{ch}}
\ar[d]^-{\epsilon}
&%r1c1
T_n((Q\scrW X)^{\otimes 2})\ar[r]^-{DAW}
\ar[d]^-{\epsilon^{\otimes 2}}
&%r1c2
(T_*(Q\scrW X))^{\otimes 2}_n
\ar[d]^-{\epsilon^{\otimes 2}}
\\%r1c3
N_{n+1}(QX)
\ar[r]^-{\Delta_{\LL{}}}
&%r1c1
N_n((Q X)^{\otimes 2})\ar[r]^-{AW}
&%r1c2
(N_*(Q X))^{\otimes 2}_n
}\]
Now the right hand square commutes, and $DAW$ preserves filtration if the tensor product of complexes is given the tensor filtration \cite{SingerBook}. The maps $\Delta_\textup{ch}$ and $\Delta_\LL{}$ each have degree $-1$, and $\Delta_\textup{ch}$ decreases filtration by $1$.

If we can prove that the left hand square commutes up to homotopy, then we'll have shown that if a class $a\in E_{\infty}$ detects a class $\widetilde{a}$ in $\pi(QX)$, then $\Delta(a)$ detects $\Delta(\widetilde{a})$.
\begin{prop}
The left hand square in the above diagram commutes up to homotopy whenever $X$ is the $\Lambda$-indecomposables on the simplicial bar construction, $\Q_{\Lambda}[\Fr{\LL{}}]^{\bullet+1}M$, for $M\in\LL{}$.
\end{prop}
\begin{proof}
The chain homotopy $\Phi$ is constructed as follows. Let $\Phi$ be zero except on $N_0^\textup{fil}N_*^\textup{int}(Q\scrW X)$, where it is given by the map
\[k_{f^0(g^1_{i_1}(h^2_{i_1i_2}))}\mapsto q(f)(g^1_{i_1}(h^2_{i_1i_2})).\]
This definition makes sense (and yields a nontrivial map) because $f_0$ is an operator in $\Q_{\Lambda}\Fr{\LL{}}$.
The chain map $d\Phi+\Phi d$ is a sum of three terms:
\begin{enumerate}\squishlist
\setlength{\parindent}{.25in}
\item[(a)] $d\circ\Phi:N_0^\textup{fil}N_*^\textup{int}(Q\scrW X)\overset{\Phi}{\to} N_*((QX)^{\otimes2})\overset{d}{\to} N_{*-1}((QX)^{\otimes2})$
\item[(b)] $\Phi\circ d^\textup{int}:N_0^\textup{fil}N_{*}^\textup{int}(Q\scrW X)\overset{d^\textup{int}}{\to} N_0^\textup{fil}N_{*-1}^\textup{int}(Q\scrW X)\overset{\Phi}{\to} N_{*-1}((QX)^{\otimes2})$
\item[(c)] $\Phi\circ d^\textup{fil}:N_1^\textup{fil}N_{*-1}^\textup{int}(Q\scrW X)\overset{d^\textup{fil}}{\to} N_0^\textup{fil}N_{*-1}^\textup{int}(Q\scrW X)\overset{\Phi}{\to} N_{*-1}((QX)^{\otimes2})$
\end{enumerate}
Let's start by identifying (a) and (b) by considering a term of the form $K:=k_{f^0(g^1_{i_1}(h^2_{i_1i_2}))}$. Then
\[(d\circ\Phi)(K)=d( q(f)(g^1_{i_1}(h^2_{i_1i_2})))=q(f)(g^0_{i_1}(h^1_{i_1i_2})),\textup{ and}\]
\[(\Phi\circ d^\textup{int})(K)=\Phi( k_{f^0(g^0_{i_1})(h^1_{i_1i_2}))})=q(f(g_{i_1}))(h^1_{i_1i_2}).\]
However, $g^0_{i_1}(h^1_{i_1i_2})=ug_{i_1}(h^1_{i_1i_2})$, as we are calculating in $(QX)^{\otimes2}$, so
\[(d\circ\Phi)(K)=q(f)(ug_{i_1}(h^1_{i_1i_2}))=q(f(ug_{i_1})(h^1_{i_1i_2})).\]
Thus, the sum of these two terms is
\[\left(q(f(g_{i_1}))+q(f(ug_{i_1}))\right)(h^1_{i_1i_2})=q(uf(g_{i_1}))(h^1_{i_1i_2}).\]
This is exactly the formula for $(\Delta_{\LL{}}\circ\epsilon)(K)$.

It remains to show that $\Phi\circ d^\textup{fil}$ coincides with $\epsilon^{\otimes 2}\circ\Delta_{\LL{}}$. A class in $N_1^\textup{fil}N_{*-1}^\textup{int}(Q\scrW X)$ is a linear combination
\[K':=\sum_j k_{e_j\left(s_{\alpha_{ji_0}} k_{f^0_{ji_0}g^1_{ji_0i_1}h^2_{ji_0i_1i_2}}\right)}\]
which satisfies the equation
%\[d_1^\textup{fil}(K')=\sum_j k_{e_j\left(s_{\alpha_{ji}}\left( f^0_{ji_0}g^1_{ji_1}h^2_{ji_1i_2}\right)\right)}=0,\]
\[d_1^\textup{fil}(K')=\sum_j k_{e_j\left(s_{\alpha_{ji_0}} {f^0_{ji_0}g^1_{ji_0i_1}h^2_{ji_0i_1i_2}}\right)}=0\]
which may be rewritten as
\[\sum_j k_{e_j(f_{ji_0})\left(s_{\alpha_{ji_0}} {g^1_{ji_0i_1}h^2_{ji_0i_1i_2}}\right)}=0.\]
Since the symbol $k$ is used in this condition, it is \emph{very} strong. It can only hold when, the subscripts appearing therein, as $j$ varies, are each repeated an even number of times! As the subscripts are elements of $Q_{\Lambda}[\Fr{\LL{}}]^{*}M$, these coincidences also occur if we take quadratic parts of the $e_j(f_{ji_0})$, so that there holds the following equation in $N_{*-1}((QX)^{\otimes2})$:
%\[\sum_j {q(e^0_j(f^0_{ji_0}))\left(s_{\alpha_{ji}} g^1_{ji_1}h^2_{ji_1i_2}\right)}=0\in N_{*-1}((QX)^{\otimes2}).\]
\begin{gather*}
q(e_j(f_{ji_0}))\left(s_{\alpha_{ji_0}} {g^1_{ji_0i_1}h^2_{ji_0i_1i_2}}\right)=0,\textup{ or equivalently}\\
q(e_j(uf_{ji_0}))\left(s_{\alpha_{ji_0}} {g^1_{ji_0i_1}h^2_{ji_0i_1i_2}}\right)
=
q(ue_j(f_{ji_0}))\left(s_{\alpha_{ji_0}} {g^1_{ji_0i_1}h^2_{ji_0i_1i_2}}\right).
\end{gather*}
The proof is completed upon noting that the LHS equals $(\epsilon^{\otimes 2}\circ \Delta_\textup{ch})(K')$, and that the RHS equals $(\Phi\circ d^\textup{fil})(K')$. For the LHS:
\begin{alignat*}{2}
(\epsilon^{\otimes 2}\circ \Delta_\textup{ch})(K')
&=
\epsilon^{\otimes 2}\left(\sum q(e_j)\left(s_{\alpha_{ji_0}} k_{f^0_{ji_0}g^1_{ji_0i_1}h^2_{ji_0i_1i_2}}\right)\right)%
\\
&=
\sum q(e_j)\left(s_{\alpha_{ji_0}} f^0_{ji_0}g^1_{ji_0i_1}h^2_{ji_0i_1i_2}\right)%
\\
&=
\sum q(e_j(uf_{ji_0}))\left(s_{\alpha_{ji_0}} g^1_{ji_0i_1}h^2_{ji_0i_1i_2}\right)
\end{alignat*}
where we may simplify $f_{ji_0}$ to $uf_{ji_0}$ since we take values in $(QX)^{\otimes2}$. Next,
\begin{alignat*}{2}
d^\textup{fil}(K')
&=
\sum e_j\left(s_{\alpha_{ji_0}} k_{f^0_{ji_0}g^1_{ji_0i_1}h^2_{ji_0i_1i_2}}\right)%
\\
&=
\sum ue_j\left(s_{\alpha_{ji_0}} k_{f^0_{ji_0}g^1_{ji_0i_1}h^2_{ji_0i_1i_2}}\right)%
&\qquad&\text{(in $N_{*-1}^\textup{int}(QWX)$)}\\
&=
\sum ue_j\left( k_{f^0_{ji_0}g^1_{ji_0i_1}h^2_{ji_0i_1i_2}}\right)%
&\qquad&\text{(relevant $s_{\alpha_{ji_0}}$ are $\textup{id}$)}\\
\Phi(d^\textup{fil}(K'))&=
\sum ue_j\left( q(f_{ji_0})(g^1_{ji_0i_1}h^2_{ji_0i_1i_2})\right)%
\\
&=
\sum ue_j\left( q(f_{ji_0})(s_{\alpha_{ji_0}}g^1_{ji_0i_1}h^2_{ji_0i_1i_2})\right)%
&\qquad&\text{(relevant $s_{\alpha_{ji_0}}$ are $\textup{id}$)}\\
&=\textup{RHS}.
\end{alignat*}
To explain further the removal of the $s_{\alpha_{ji_0}}$, note that the quantity $\sum ue_j(s_{\cdots }k_{\cdots })$ is an element of the normalized chain complex of $QWX$, and as such, no nontrivial degeneracies may appear therein. That is, it must happen that after restricting to the identity part $ue_j$ of $e_j$, no nontrivial degeneracies remain, and it makes sense to remove the symbol. It similarly makes sense to replace the symbol after applying $\Phi$.
\end{proof}
\end{Grothendieck Multiplicativity}

\begin{Thoughts on Adams Multiplicativity}
\section{Adams multiplicativity}
Let $\widetilde{A}^n:=c(K\overline{Q})^{n+1}A$, for $A$ a simplicial algebra. Then $\widetilde{A}^\bullet$ is a levelwise cofibrant cosimplicial object, with a Hurewicz morphism to $Q\widetilde{A}^\bullet$. However, $Q\widetilde{A}^\bullet$ is contractible, and the cosimplicial homology coalgebra $\pi_*(Q\widetilde{A}^\bullet)$ is thus a cofree resolution of $H_*(A)$. The Hurewicz morphism is in each level an isomorphism onto the primitives:
\[\pi_*(\widetilde{A}^\bullet)\overset{\simeq}{\to}P\pi_*(Q\widetilde{A}^\bullet)\overset{}{\to}\pi_*(Q\widetilde{A}^\bullet)\]
so that the $E^2$ term is derived primitives on $H_*(A)$. As such, it should admit a product. In fact, the product should shift both gradings by 1, the internal grading $*$ and the filtration grading $\bullet$.
This product on $E^2$ may be defined using a map $\overline{\xi}:(\pi_*(Q\widetilde{A}^{\bullet}))^{\times2}\overset{}{\to}\pi_*(Q\widetilde{A}^{\bullet-1})$ and its factoring $\xi$:
%
%
%
%
%First we'll construct a map $\overline{\xi}_{\LL{}}$ with a factoring $\xi_{\LL{}}$ as follows:
%\[\xymatrix@R=4mm{
%\pi_*(\widetilde{A}^{\bullet-1})
%\ar[r]^-{h}
%\ar@{-->}[dr];[]^-{\pi_*(\xi_\textup{res})}
%&
%\pi_*(Q\widetilde{A}^{\bullet-1})
%\ar[rr];[]_-{\overline{\xi}}
%\ar@{-->}[rd];[]_-{\xi}
%&&%r1c1
%\pi_*(Q\widetilde{A}^{\bullet})^{\times2}\\%r1c3
%&
%\pi_*(\widetilde{A}^{\bullet-1})
%\ar[r]^-{h}&%r2c1
%\pi_*(Q\widetilde{A}^{\bullet})^{\wedge 2}
%\ar@{->>}[ur];[]○
%}\]
\[\xymatrix@R=4mm{
&&&\pi((\widetilde{A}^\bullet)^{\sqcup2})
\ar[dl]
\\
\pi(\widetilde{A}^{\bullet-1})
\ar[rr];[]_-{\pi(\overline{\xi}_\textup{res})}
\ar@{->}[rd];[]_-{\pi(\xi_\textup{res})}
\ar[d]^-{h}_\simeq
&&%r1c1
\pi(c((\widetilde{A}^\bullet)^{\times2}))
\ar@{=}[r]
\ar[dd]^-{h}
&
\pi(\widetilde{A}^{\bullet})^{\oplus2}\\%r1c3
P\pi(Q\widetilde{A}^{\bullet-1})
\ar[d]
&%r2c1
\pi(\textup{cof})
\ar@{->}[ur];[]○
\ar[d]^{h}
&
&
\pi(Q((\widetilde{A}^\bullet)^{\sqcup2}))
\ar@{=}[r]
\ar@{ >->}[ld]
&
\pi(Q\widetilde{A}^{\bullet})^{\sqcup2}
\ar@{ >->}[ld]
\\
\pi(Q\widetilde{A}^{\bullet-1})
\ar@/^1em/[rr];[]_-(.9){\overline{\xi}}
\ar@{->}[rd];[]^-{\xi}
&
P\pi(Q\textup{cof})
\ar[ul]^-(.8){P(\xi)}
\ar[d]
&%r1c1
\pi(Qc((\widetilde{A}^\bullet)^{\times2}))
\ar@{=}[r]
&
\pi(Q\widetilde{A}^{\bullet})^{\times2}
\ar@{->>}[ld]\\%r1c3
&%r2c1
\pi(Q\textup{cof})
\ar@{->>}[ur];[]○
\ar@{=}[r]
&
\pi(Q\widetilde{A}^{\bullet})^{\wedge 2}
}\]
We'd like to use the map $P(\xi):P\pi(Q\textup{cof})\to P\pi(Q\widetilde{A}^{\bullet-1})$, and it would be awesome if $(P(\xi)\circ h)$ could actually be defined on the simplicial level. It'll be enough to define a map $\overline{\xi}_{\textup{res}}$ that corresponds to $\overline{\xi}$ under the Hurewicz, and then to note that it factors through the cofibre.
\[\xymatrix@R=10mm@C=23mm@!0{
\pi(A\otimes A)
\ar@{=}[r]
\ar@{-->}[d]_-{?}
&%r1c1
(P\pi(QA))^{\otimes 2}
\ar[drrrr]^-(.8){j'}
\ar[dr]_-{j}
&%r1c2
&%r1c3
&%r1c4
\makebox[0cm][c]{$(A:=A^\bullet,\,A_+:=A^{\bullet+1})$}&%r1c5
\\%r1c6
\pi(\textup{cof})
\ar@/^1em/[dddr]^-(.625){\pi(\xi_\textup{res})}
\ar[rr]
&%r2c1
&%r2c2
P\pi(Q\textup{cof})
\ar[rr]
\ar@/^1em/[dddr]^-(.625){P\pi(\xi_\textup{res})}
&%r2c3
&%r2c4
\pi(Q\textup{cof})\ar@{=}[r]
&%r2c5
\pi(QA)^{\wedge 2}
\ar@/^1em/[dddr]^-(.625){\xi}
\\
\pi(c(A^{\times2}))
\ar@/^1em/[ddr]_-(.8){\pi(\overline{\xi}_\textup{res})}
\ar[rr]
\ar[u]
&%r2c1
&%r2c2
P\pi(Qc(A^{\times2}))
\ar@/^1em/[ddr]_-(.8){P\pi(\overline{\xi}_\textup{res})}
\ar[rr]
\ar[u]
&%r2c3
&%r2c4
\ar[u]
\pi(Qc(A^{\times2}))\ar@{=}[r]
&%r2c5
\pi(QA)^{\times 2}
\ar[u]
\ar@/^1em/[ddr]_-(.8){\overline{\xi}}
\\
\pi(c(A^{\sqcup2}))
\ar[rr]
\ar[u]
&%r2c1
&%r2c2
P\pi(Qc(A^{\sqcup2}))
\ar[rr]
\ar[u]
&%r2c3
&%r2c4
\ar[u]
\pi(Qc(A^{\sqcup2}))\ar@{=}[r]
&%r2c5
\pi(QA)^{\sqcup 2}
\ar[u]
\\
&
\pi(A_+)
\ar[rr]
&%r2c1
&%r2c2
P\pi(QA_+)
\ar[rrr]
&%r2c3
&%r2c4
&
\pi(QA_+)
}\]
I guess the map $?$ will need to shift degrees up by 1, to keep up with $j'$, and to keep products on $E^2$ having shift $(+1,+1)$. It could have the properties that Goerss likes. It's enough for it simply to commute with $j'$, and not $j$, although that's trivial.
\subsection{A general construction}
Let $f:X\to Y$ be a map of simplicial algebras, and write `$\textup{ker}$' for the (strict) kernel of $f$, and `$\textup{cof}$' for the homotopy cofibre of $f$, built using the small object argument (i.e.\ $\textup{cof}:=m(sf)\sqcup_{sA}0$). Then there's a commuting diagram:
\[\xymatrix@C=16mm@R=6mm@!0{
\overline{W}(c(\textup{ker}))\ar@{-->}[rr]
&%r1c1
&%r1c2
\Sigma(\textup{ker})\ar[rr]
&%r1c3
&%r1c4
\textup{cof}\\\\%r1c5
W(c(\textup{ker}))\ar@{-->}[rr]\ar[uu]
&%r2c1
&%r2c2
m''\ar[rr]\ar@{->>}[dr]_-{\sim}
\ar[uu]
&%r2c3
&%r2c4
m'\ar@{->>}[dr]^-{\sim}
\ar[uu]
\\%r2c5
&%r3c1
&%r3c2
&%r3c3
0\ar[rr]
&%r3c4
&c(Y)
\\%r3c5
c(\textup{ker})\ar@{ >->}[uu]
\ar@{=}[rr]
&%r4c1
&%r4c2
c(\textup{ker})\ar@{ >->}[uu]
\ar[rr]\ar[ru]
&%r4c3
&%r4c4
c(X)\ar@{ >->}[uu]
\ar[ur]
}\]
Note that the lower dotted map is that filling the commuting square
\[\xymatrix@R=4mm{
c(\textup{ker})\ar@{ >->}[d]\ar@{ >->}[r]
&%r1c1
m''\ar@{->>}[d]^-{\sim}
\\%r1c2
W(c(\textup{ker}))\ar[r]\ar@{-->}[ru]
&%r2c1
0%r2c2
}\]
This furnishes us with a map $N_*(c(\textup{ker}))\to N_*(\textup{cof})$. Now if we use a back-to-front version of the $W$-construction, there is a natural map $N_*Z\to N_{*+1}\overline{W}Z$ for any simplicial NUCA $Z$, constructed by mapping $Z_n$ isomorphically onto the final summand of $\overline{W}^{n+1}Z=Z_0\sqcup\cdots \sqcup Z_n$. This is trivially a chain map.

Using $X=A^{\sqcup2}$ and $Y=A^{\times2}$, and $Z=\textup{ker(f)}=A^{\otimes2}$, we get a zig-zag on normalized complexes
\[N_*(A^{\otimes2})\overset{\sim}{\from}N_*(c(A^{\otimes2}))\to N_{*+1}(\overline{W}(c(A^{\otimes2})))\to N_{*+1}(\textup{cof}).\]
\subsection{Finishing the proof}
Consider the following diagram:
\[\xymatrix@R=4mm{
(T_*(\widetilde{A}))^{\otimes 2}\ar[r]^-{\textup{AWEZ}}
&%r1c1
T_*(\widetilde{A}^{\otimes 2})&%r1c2
T_*(c(\widetilde{A}^{\otimes 2}))\ar[l]_-{\epsilon}
\ar[r]^-{\textup{monster}}
&%r1c3
T_*(\widetilde{A})\\%r1c4
(N_*(A))^{\otimes 2}\ar[r]^-{\textup{EZ}}\ar[u]
&%r1c1
N_*({A}^{\otimes 2})\ar[u]&%r1c2
T_*(c({A}^{\otimes 2}))\ar[l]_-{\epsilon}\ar[u]
\ar[r]^-{\textup{mult}\circ\epsilon}
&%r1c3
N_*(A)\ar[u]
}\]
\subsubsection{An $E^r$-level product for $r\geq1$}
The three top horizontal maps are perfectly decent maps of filtered complexes, with the `monster' map increasing filtration by one.
\subsubsection{the rest}
The maps $\textup{AWEZ}$ and $\epsilon$ are both induced by maps of double complexes.

We just need to see that the right hand square commutes up to homotopy. That's because the middle backwards maps (trivially preserve filtration and) are levelwise weak equivalences. Thus, whatever cycle pops up in $N(A^{\otimes2})$, you can hit it (up to a boundary therein) with a cycle in $N(A^{\otimes2})$. But after mapping out to the right hand copy of $N(A^{\otimes2})$, your error is exactly the same cycle!

The $E_r$ level product will be given by ``apply AWEZ, pull back to something in TcA2, and then apply the monster map.'' \textbf{Actually I'm not sure whether or not that's well defined. There might be something to check.}




\subsection{The construction of $\xi_\textup{res}$}
\[\xymatrix@R=0mm{
(HA)^{\wedge 2}
\ar@{-->}@/^3em/[rrrrdd]^-(.9){\xi}
\\
&%r1c1
(HA)^{\times2}
\ar[r]^-{d_0\times d_0}
\ar@{}[dd]|{\times}="timesOne"
&%r1c2
(HA_+)^{\times2}
\ar[r]^-{\textup{hopf}}
\ar@{}[dd]|{\times}="timesTwo"&%r1c3
\ar@{}[dd]|{\times}="timesThree"
HA_+&%r1c4
\\%r1c5
(HA)^{\times2}\ar;"timesOne"_-{\textup{diag}}
\ar@{-->}@/^3.9em/[rrrr]^-(.1){\overline{\xi}}
\ar@{->>}[uu]
&%r2c1
&%r2c2
&%r2c3
&%r2c4
HA_+\ar"timesThree";[]^-{\textup{hopf}}
\\%r2c5
&%r3c1
(HA)^{\times2}\ar[r]_-{\textup{hopf}}
&%r3c2
HA\ar[r]_-{d_0}
&%r3c3
HA_+&%r3c4
\\
(HA)^{\sqcup2}
\ar[uu]
\\\\
&&\makebox[0cm][c]{\,\textup{is $\pi(Q\DASH)$ applied to}}
\\
\textup{cof}
\ar@{-->}@/^1em/[rrrrdd]^-(.9){\xi_\textup{res}}
\\\\%r3c5
\ar[uu]
\ar@{-->}@/^1.3em/[rrrr]^-(.1){\overline{\xi}_\textup{res}}
c(A^2)\ar[r]_-{c(\textup{diag})}
&%r4c1
c(A^2\times A^2)\ar[dd]^-{\beta}
&%r4c2
c(A_+^2\times A)
\ar[dd]^-{\beta}
&%r4c3
c(A_+\times A_+)\ar[r]_-{h}
%\ar[dd]^-{\beta}
&%r4c4
A_+\\\\%r4c5
c(A^{\sqcup2})\ar[uu]
&%r5c1
c(c(A^2\times A^2))\ar[dd]
&%r5c2
c(c(A_+^2\times A))\ar[dd]
&%r5c3
%c(c(A_+\times A_+))\ar[uur]_-{c(h)}
&%r5c4
%r5c5
\\\\%r4c5
%%%%%&%r5c1
%%%%%\ar[dd]
%%%%%\makebox[0cm][c]{$c(c(A^2)\times c(A^2))$}&%r5c2
%%%%%&%r5c3
%%%%%&%r5c4
%%%%%%r5c5
%%%%%\\\\%r4c5
&%r5c1
c(A^2\times c(A^2))
\ar@/_0em/[uuuur]^-(.7){c(d_0^2\times h)}
&%r5c2
c(c(A^2_+)\times A)
\ar@/_0em/[uuuur]^-(.7){c(h\times d_0)}&%r5c3
&%r5c4
%r5c5
}\]
\[\xymatrix@R=0mm{
\textup{cof}
\ar@{-->}@/^1em/[rrrrdd]^-(.9){\xi_\textup{res}}
\\\\%r3c5
\ar[uu]
\ar@{-->}@/^1.3em/[rrrr]^-(.1){\overline{\xi}_\textup{res}}
c(A^2)\ar[r]_-{\beta^2\circ c(\textup{diag})}
&%r4c1
ccc(A^2\times A^2)\ar[ddd]
&%r4c2
cc(A_+^2\times A)
\ar[ddd]
&%r4c3
c(A_+\times A_+)\ar[r]_-{h}
%\ar[dd]^-{\beta}
&%r4c4
A_+\\\\%r4c5
c(A^{\sqcup2})\ar[uu]
\\
&%r5c1
%r5c1
cc(A^2\times c(A^2))
\ar@/_0em/[uuur]^-(.45){cc(d_0^2\times h)}
&%r5c2
c(c(A^2_+)\times A)
\ar@/_0em/[uuur]^-(.45){c(h\times d_0)}&%r5c3
&%r5c4
%r5c5
}\]
\[\xymatrix@R=0mm{
\textup{cof}
\ar@{-->}@/^1em/[rrrrdd]^-(.9){\xi_\textup{res}}
\\\\%r3c5
\ar[uu]
\ar@{-->}@/^1.3em/[rrrr]^-(.1){\overline{\xi}_\textup{res}}
c(A^2)\ar[r]_-{c(\textup{diag})\circ\beta}
&%r4c1
c(c(A^2)\times c(A^2))\ar[r]_-{c(c(d_0^2)\times h)}
%\ar[ddd]
&%r4c2
c(c(A_+^2)\times A)
\ar[r]_-{c(h\times d_0)}
%\ar[ddd]
&%r4c3
c(A_+\times A_+)\ar[r]_-{h}
%\ar[dd]^-{\beta}
&%r4c4
A_+\\\\%r4c5
c(A^{\sqcup2})\ar[uu]
}\]
\[\xymatrix@R=0mm{
\textup{cof}
\ar@{-->}@/^1em/[rrrrdd]^-(.9){\xi_\textup{res}}
\\\\%r3c5
\ar[uu]
\ar@{-->}@/^1.3em/[rrrr]^-(.1){\overline{\xi}_\textup{res}}
c(A^2)\ar[r]_-{c(\textup{diag})\circ\beta}
&%r4c1
c(A^2\times c(A^2))\ar[r]_-{c((d_0^2)\times h)}
%\ar[ddd]
&%r4c2
c(A_+^2\times A)
\ar[r]_-{c(h\times (\epsilon\circ d_0))}
%\ar[ddd]
&%r4c3
c(K_+\times K_+)\ar[r]_-{h}
%\ar[dd]^-{\beta}
&%r4c4
A_+\\\\%r4c5
c(A^{\sqcup2})\ar[uu]
}\]
Here, $A$ was being used as a *terrible* shorthand for $\widetilde{A}^n:=c(K\overline{Q})^{n+1}A$, and unadorned superscripts were used to indicate iterated products. The cofibre `$\textup{cof}$' is that of the map $A^{\sqcup2}\to A^2$. If we just write $k$ for the zero-square algebra $(K\overline{Q})^{n+1}A$, then the map $h:c(A^2)\to A$ is defined to be the composite:
\[c(A^{2})=c((ck)^{2})\overset{c(\epsilon^{2})}{\to}c(k^2)\overset{c(\textup{add})}{\to}ck=A,\]
the point being that this is an algebra map, where $\textup{add}:A^2\to A$ is not.
\begin{lem}
The map $\overline{\xi}_\textup{res}$ factors through the cofibre.
\end{lem}
\begin{proof}
It's enough to check that the composite $c(A^{\sqcup2})\to c(A^{\times2})\to c(A_+)$ is null. However, the codomain $c(A_+)$ is a GEM, so that a map to $c(A_+)$ is null if it is zero on cohomology. Fortunately, we have already described the maps on homology induced by these two maps --- they are the inclusion $(HA)^{\sqcup2}\to(HA)^{\times}2$ and $\overline{\xi}:(HA)^{\times}2\to HA_+$ respectively, and the composite of these two maps is zero.
\end{proof}
\subsection{The diagonal map of the small object argument}
For practice, here's the derivation of the diagonal. We'll fill the dotted verticals in the following diagram, one at a time:
\[\xymatrix@R=4mm{
{*}\ar@{=}[d]
\ar@{ >->}[r]
&%r1c1
s_1B
\ar@{ >->}[r]\ar@{..>}[d]^-{\delta_1}
&%r1c2
s_2B
\ar@{ >->}[r]\ar@{..>}[d]^-{\delta_2}
&%r1c3
\cdots \ar@{ >->}[r]&%r1c4
sB\ar@{=}[rd]
\ar@{..>}[d]_-{\delta}
&%r1c5
\\%r1c6
{*}\ar@{ >->}[r]&%r2c1
s_1s_1B\ar@{ >->}[r]&%r2c2
s_2s_2B\ar@{ >->}[r]&%r2c3
\cdots \ar@{ >->}[r]&%r2c4
ssB\ar@{->>}[r]_-{\sim}&%r2c5
sB%r2c6
}\]
Assume that the $\delta_{\beta'}$ for $\beta'<\beta$ have been defined. Then let's define $\delta_\beta$. Let $s_{\beta}^-B$ be the colimit of the $s_{\beta'} B$ for $\beta'<\beta$, and let $\delta^-_\beta$ be the induced map $s_{\beta}^-B\to s_{\beta}^-s_\beta^-B$. The object $s_\beta B$ is a pushout of $s_{\beta}^-B$ (and we know where to send this stuff) with a copy of $T$ for every square as at top left of the following diagram:
\[\xymatrix@R=4mm{
S\ar[r]
\ar@{ >->}[d]
&%r1c1
s^-_\beta B\ar@{..>}[r]^-{\delta^-_{\beta}}
\ar[d]
&%r1c2
s_\beta^-s_\beta^-B\ar@{..>}[r]
&s_\beta^-s_\beta B\ar@{..>}[d]
\\%r1c3
T\ar@{..>}@/_1em/[rrr]
\ar[r]
%\ar@{>..>}[dr]
&%r2c1
B&%r2c2
&s_\beta B
}\]
Now we should check that the outer square (involving $S,T,s_\beta^-s_\beta B,s_\beta B$) commutes. The right-down composite, using inductively the commutativity of the diagram sought, simply equals the map $S\to s_\beta B$ corresponding to the solid square. The map $T\to s_\beta B$ also corresponds to the solid square, and the down-right composite is its restriction to $S$, and thus the outer square commutes. This provides a candidate copy of $T$ in $s_\beta s_\beta B$ onto which we may map the given copy of $T$.
\subsubsection{On cofibres}
For $f:A\to B$, let's write $m(f)$ for the middle functor of the SOA factorisation: $A\to m(f)\to B$. We can upgrade the above argument to produce a section $\delta$ of the map 
$\xymatrix@R=4mm@1{
m(sf)\ar@{->>}[r]^-{\sim}
&%r1c1
sB%r1c2
}$. The changes are minor, and reflected in the following alteration of the diagram above:
\[\xymatrix@R=4mm{
{*}\ar[d]
\ar@{ >->}[r]
&%r1c1
s_1B
\ar@{ >->}[r]\ar@{..>}[d]^-{\delta_1}
&%r1c2
s_2B
\ar@{ >->}[r]\ar@{..>}[d]^-{\delta_2}
&%r1c3
\cdots \ar@{ >->}[r]&%r1c4
sB\ar@{=}[rd]
\ar@{..>}[d]_-{\delta}
&%r1c5
\\%r1c6
sA\ar@{ >->}[r]&%r2c1
m_1(sf)\ar@{ >->}[r]&%r2c2
m_2(sf)\ar@{ >->}[r]&%r2c3
\cdots \ar@{ >->}[r]&%r2c4
m(sf)\ar@{->>}[r]_-{\sim}&%r2c5
sB%r2c6
}\]
We can use this section to construct a natural map $sB\to\textup{cof(f)}$. 
\[\xymatrix@R=4mm{
A\ar[dd]_-{f}
%r1c1
&%r1c2
sA\ar[dd]_-{sf}
\ar@{ >->}[dr]
&%r1c3
&%r1c4
\\%r1c5
%r2c1
&%r2c2
&%r2c3
m(sf)\ar@{->>}@<.25ex>[dl]^-{\sim}\ar[dr]
&%r2c4
\\%r2c5
B
&
sB\ar@{->}@/^.41em/@<.25ex>[ur]^\delta
&%r3c1
&%r3c2
\textup{cof}(f)\makebox[0cm][l]{\,$:=m(sf)\sqcup_{sA}0$}&%r3c3
&%r3c4
%r3c5
}\]
\subsubsection{A map from the bar construction to the SOA}
There is one! In fact, there's a map from the $n$-skeleton of the bar construction to the $n^\textup{th}$ stage of the SOA.
\subsubsection{A cofibration}
Write $S^n$ for (the free construction on) $\Delta^n/\partial\Delta^n$, and $CS^n$ for (the free construction on) the reduced cone thereupon. Now there's a map $S^{n+m}\to S^n\sqcup S^m$ sending the fundamental class to the Eilenberg-Zilber formula. That is, if $\mu:(S^n\sqcup S^m)^{\otimes 2}\to S^n\sqcup S^m$ is multiplication, the map we are describing sends $\imath_{n+m}\mapsto \mu(\textup{EZ}(\imath_n,\imath_m))$. Now consider the complex $C_{n,m}$ formed by coning off the corresponding homotopy class:
\[\xymatrix@R=4mm{
S^{n+m}\ar[r]
\ar@{ >->}[d]
&%r1c1
S^n\sqcup S^m\ar@{ >->}[d]
\\%r1c2
CS^{n+m}\ar[r]
&%r2c1
C{n,m}%r2c2
}\]
Now $C_{n,m}$ has an $n+m+1$-dimensional generator $g$, the image of the cone on the fundamental class of $S^{n+m}$. The homology LES shows that $H_*(C_{n,m})$ is three-dimensional, containing classes $\imath_n$, $\imath_m$ and $g$.
\begin{prop}
The diagonal of $g$ is $\imath_n\otimes\imath_m+\imath_m\otimes\imath_n$.
\end{prop}
\begin{proof}
The representative $g$ has the property that $d_0(g)=\mu(\textup{EZ}(\imath_n,\imath_m))$ and $d_i(g)=0$ for $i>0$. We calculate 
\begin{alignat*}{2}
\psi(d_0(g))
&=
\mu(\textup{EZ}(\overline{\imath_n}+\overline{\overline{\imath_n}},\overline{\imath_m}+\overline{\overline{\imath_m}}))%
\\
\xi(g)
&=
\mu(\textup{EZ}(\overline{\imath_n},\overline{\overline{\imath_m}})+\textup{EZ}(\overline{\overline{\imath_n}},\overline{\imath_m}))\in Q((C_{n,m})_{n+m}^{\wedge 2})%
\\
% Left hand side
j(\xi(g))
% Relation
&=
% Right hand side
\textup{EZ}^\otimes ({\imath_n},{{\imath_m}})+\textup{EZ}^\otimes ({\imath_m},{{\imath_n}})\in (Q((C_{n,m})_{n+m}))^{\otimes 2}%
% Comment
\end{alignat*}
This class represents $\imath_n\otimes\imath_m+\imath_m\otimes\imath_n$.
\end{proof}
Now suppose that $k$ is a (not zero-square) algebra, and that $\alpha,\beta$ are homotopy classes in $\pi_n (k)$ and $\pi_m (k)$ respectively. Then we'll be able to use the complex $C_{n,m}$ to construct a homology class $g_{\alpha,\beta}\in H_{n+m+1}(k)$ whose diagonal is $\alpha\otimes\beta+\beta\otimes\alpha$. Firstly, note that there is a commuting diagram
\[\xymatrix@R=4mm{
0\ar[r]
\ar@{ >->}[d]
&%r1c1
0\ar[d]
\\%r1c2
S^n\sqcup S^m\ar[r]^-{\alpha\sqcup\beta}
&%r2c1
k^{\times2}%r2c2
}\]
which furnishes $c_1(k^{\times2})$ with a map from of $S^n\sqcup S^m$. Next, there's a diagram %\newdir{ >}{{}*!/-7pt/\dir{>}}
\[\xymatrix@R=4mm{
S^{n+m}\ar@{->}[r]_-{\textup{EZ}^\mu(\imath_n,\imath_m)}
\ar@{ >->}[d]
&%r1c1
S^n\sqcup S^m\ar@{ >->}[d]
\ar[r]
&c_1(k^{\times2})\ar[d]
\\%r1c2
CS^{n+m}\ar[r]\ar@/_1em/[rr]_-{0}
&%r2c1
C{n,m}\ar@{..>}[r]
%r2c2
&k^{\times2}}\]
where the outer rectangle commutes since in $k^{\times2}$, $(a,0)\cdot(0,b)=(0,0)$. This gives a map $C_{n,m}\to c_2(k^{\times2})$ whose restriction to the subcomplex $S^n\sqcup S^m$ gives the homotopy classes $\alpha$ and $\beta$. The image $g_{\alpha,\beta}$ of the class $g\in(C_{n,m})_{n+m+1}$ is then an explicit representative for the homology class sought.
\end{Thoughts on Adams Multiplicativity}
\begin{Thoughts on Adams Multiplicativity II}
\section{Reimagining multiplicativity --- mostly obsolete}
To use the construction of Bousfield-Kan \cite{BK_pairings_products.pdf,BK_pairings.pdf}, we'll need to understand the construction of $D^1X$, for cosimplicial objects $X$:
\[\xymatrix@R=4mm{
D^1X\ar[r]
\ar[d]^-{\delta}
&%r1c1
\Lambda VX\ar[d]^-{\lambda}
\\%r1c2
X\ar[r]^-{\nu}
&%r2c1
VX%r2c2
}\]
Now $VX$ is the cosimplicial object given by shifting $X$ down and forgetting the 0th face and degeneracy. $d_0$ induces a map $\nu:X\to VX$. Amazing. Next the endofunctor $\Lambda:s\Comm\to s\Comm$ is applied in each cosimplicial level. This is the ``standard simplicial path fibration'' (c.f.\ \cite[p.82]{BousKanSSeq.pdf}), which acts on $Y\in s\Comm$ by shifting down and restricting to a kernel: $\Lambda Y_s=\ker(d_{s+1}\cdots d_1:Y_{s+1}\to Y_0)$; again we forget the 0th face and degeneracy. This time, $d_0:\Lambda Y\to Y$ is a fibration from an acyclic object.

The game will be to somehow create a factorization
\[\xymatrix@R=4mm{
\widetilde{A}\otimes \widetilde{A}\ar[r]\ar@{~>}[d]
&%r1c1
\widetilde{A}\\%r1c2
D^1\widetilde{A}\ar[ur]_-{\delta}
&%r2c1
%r2c2
}\textup{\ \ \ or\ \ \ }
\xymatrix@R=4mm{
c(\widetilde{A}\otimes \widetilde{A})\ar[r]\ar@{~>}[d]
&%r1c1
\widetilde{A}\\%r1c2
D^1\widetilde{A}\ar[ur]_-{\delta}
&%r2c1
%r2c2
}\]
where the wavy map is a (zigzag of) cosimplicial maps.

We'll seek to use a few insights into the functors $\Lambda$ and $W$. Indeed they're adjoint, I hope, giving a correspondence between nullhomotopies given by factorizations through $W(\textup{domain})$ and factorizations through $\Lambda(\textup{codomain})$. This'll give a $\Sigma$-$\Omega$ adjunction which we'll be able to use. It's also time that we reassess how we define the cofibre and the map out of it. I'm thinking that the correct method is to use a pushout
\[\xymatrix@R=4mm{
W(c(\widetilde{A}^{\sqcup2}))\ar[r]
\ar@{-->}[rrd]&%r1c1
\textup{cof}\ar@{-->}[rd]
&%r1c2
\\%r1c3
c(\widetilde{A}^{\sqcup2})\ar[r]
\ar@{ >->}[u]
&%r2c1
c(\widetilde{A}^{\times2})\ar@{ >->}[u]
\ar[r]_-{\xi_\textup{res}}
&%r2c2
V\widetilde{A}%r2c3
}\]
Indeed \textbf{we'll need} to have $\xi_\textup{res}$ a map to $V\widetilde{A}$, so that it'll turn the cosimplicial $d^i$ into $d^{i+1}$ (I think it does), and we'll \textbf{also need} to choose a nullhomotopy of the composite $c(\widetilde{A}^{\sqcup2})\to V\widetilde{A}$ consistently in the cosimplicial direction. Once this is achieved, there's a map to this pushout diagram from the diagram
\[\xymatrix@R=4mm{
W(c(\widetilde{A}^{\otimes2}))\ar[r]
&%r1c1
\overline{W}(c(\widetilde{A}^{\otimes2}))
\\%r1c3
c(\widetilde{A}^{\otimes2})\ar[r]
\ar@{ >->}[u]
&%r2c1
0\ar@{ >->}[u]
}\]
giving a composite
\[\overline{W}(c(\widetilde{A}^{\otimes2}))\to\textup{cof}\to V\widetilde{A}\]
or by adjunction, the top map in a diagram
\[\xymatrix@R=4mm{
c(\widetilde{A}^{\otimes2})\ar[r]
\ar[d]^-{\mu}
&%r1c1
\Lambda V\widetilde{A}\ar[d]^-{\lambda}
\\%r1c2
\widetilde{A}\ar[r]^-{\nu}
&%r2c1
V\widetilde{A}%r2c2
}\]
which (\textbf{if this diagram commutes}) will give the desired factorisation.
\subsection{On $E_1$}
Why do I want to involve the map $\xi_\textup{res}$? Well the $E_1$-level pairing is given by the top composite in
\[\xymatrix@!0@R=10mm@C=31mm{
\pi'_t\widetilde{A}^s\otimes \pi'_{t}\widetilde{A}^{s'}\ar[r]^-{\textup{EZAW}}_-{\textup{zig}}
&%r1c1
\pi'_{t+t'}(c(\widetilde{A}^{s+s'})^{\otimes 2})\ar[r]
&%r1c2
\pi'_{t+t'}(D^1\widetilde{A}^{s+s'})\ar[rd]
&%r1c3
\ar@{..>}[l]_-{\partial}
\pi'_{t+t'+1}(\widetilde{A}^{s+s'+1})\\%r1c4
&%r2c1
&%r2c2
&%r2c3
\pi'_{*}(\widetilde{A}^{s+s'}\times\Lambda (\widetilde{A}^{s+s'+1}))\ar[u]_-{\nu\times\lambda}
%r2c4
}\]
I can't quite figure out the formal stuff with the loops and connecting homomorphisms, but surely it helps to have the component $c(A^{\otimes 2})\to \Lambda \widetilde{A}^{s+s'+1}\to \widetilde{A}^{s+s'+1}$ equaling the map which gives the pairing on primitive elements that you want to see on $E^1$. This is especially compelling since we're looking at a situation in which the connecting homomorphism is dominating. Perhaps one can make this make sense by drawing in the loops LES for $cA^{\otimes 2}$ and mapping between the two LESs.

Actually, here goes. If $X$ is a simplicial object, then let's look at the fibre sequence $\Omega X\to\Lambda X\to X$. The connecting homomorphism $\pi_n X\to \pi_{n-1}\Omega X$ is an isomorphism, which we'll now examine. It is the connecting homomorphism from the SES of chain complexes
\[0\to N_*(\Omega X)\to N_*(\Lambda X)\to N_* (X)\to 0.\]
Suppose that $x\in N_n(X)$ is a cycle (so that $d_ix=0$ for $0\leq i \leq n$). Then as the homomorphism $N_*(\Lambda X)\to N_*(X)$ is just given by $d_0$, 
\[\xymatrix@!0@R=8mm@C=21mm{
&%r1c1
N_n(\Lambda X)\ar[r]^-{\lambda}
\ar[d]^-{\partial}
&%r1c2
N_n(X)
&
&%r1c1
{\exists}\ar@{|->}[r]^-{d_0}
\ar@{|->}[d]^-{d_0}
&%r1c2
x
\\%r1c6
N_{n-1}(\Omega X)\ar[r]
&%r2c1
N_{n-1}(\Lambda X)\ar[r]&%r2c2
N_{n-1}(X)
&
x\ar@{|->}[r]
&%r2c1
x\ar@{|->}@{|->}[r]^-{d_0}&%r2c2
0
}\]
This is to say that the connecting homomorphism is induced by the map $Z_*NX\to Z_{*-1}\Omega X$ given by $x\mapsto x$. Now we relate $D^1\widetilde{A}$ to $\Omega V\widetilde{A}$ using the following diagram:
\[\xymatrix@R=4mm{
c(\widetilde{A}^{\otimes 2})\ar[dr]
&%r1c1
&%r1c2
\Omega V\widetilde{A}\ar[dl]
\ar[dd]
\ar[dr]
&%r1c3
\\%r1c4
&%r2c1
D^1\widetilde{A}\ar[dd]
\ar[rr]
&%r2c2
&%r2c3
\Lambda V\widetilde{A}\ar[dd]^-{\lambda}
\\%r2c4
&%r3c1
&%r3c2
0\ar[dl]
\ar[rd]
&%r3c3
\\%r3c4
&%r4c1
\widetilde{A}\ar[rr]_-{\nu}
&%r4c2
&%r4c3
V\widetilde{A}%r4c4
}\]



\subsection{At $E_\infty$}
I guess one uses what's known about smash pairings in the sseq of a cosimplicial \emph{space} to get a pairing of sseqs
\[\pi'_{t}\widetilde{A}^{s}\wedge \pi'_{t'}\widetilde{A}^{s'}\overset{\textup{AW}}{\to}\pi'_{t}\widetilde{A}^{s+s'}\wedge \pi'_{t'}\widetilde{A}^{s+s'}\overset{\wedge =\textup{`EZ'}}{\to}\pi'_{t+t'}((\widetilde{A}^{s+s'})^{\otimes 2}),\]
and then notes that this descends to the version in which you stick tensor products everywhere. You can also zig-zag this with the collapse $c\widetilde{A}^{\otimes 2}\to\widetilde{A}^{\otimes 2}$.
\[\xymatrix@R=4mm{
E(\widetilde{A})\otimes E(\widetilde{A})\ar[r]
\ar@{=>}@<2.5ex>[d]
\ar@{=>}@<-2.5ex>[d]
&%r1c1
E(\widetilde{A}\otimes \widetilde{A})
\ar@{=>}[d]
&
E(c(\widetilde{A}\otimes \widetilde{A}))
\ar[l]^-{\cong}\ar@{=>}[d]
\\%r1c2
\pi(\widetilde{A})\otimes \pi(\widetilde{A})\ar[r]&%r2c1
\pi(\widetilde{A}\otimes \widetilde{A})%r2c2
&
\pi(c(\widetilde{A}\otimes \widetilde{A}))
\ar[l]_-{\cong}
}\]
Then you follow the map of spectral sequences with the maps induced by $\delta$ and the found factorisation of $\mu$ through $\delta$.

Moreover, I bet you can do more using ``second quadrant tricks'', not only to create an external product of spectral sequences, but also an external `cup-$i$ product(?)' of spectral sequences, in order to construct some nice operations. To be done after you get products to work!





\subsection{try again}
The pullback $D^1\widetilde{A}$ is the kernel of the simplicial vectorspace map $\widetilde{A}\times \Lambda V\widetilde{A}\to V\widetilde{A}$.  The connecting homomorphism $\pi_n V\widetilde{A}\to \pi_{n-1}D^1\widetilde{A}$ is an isomorphism, which we'll now examine. It is the connecting homomorphism from the SES of chain complexes
\[0\to N_*(D^1\widetilde{A})\to N_*(\widetilde{A}\times \Lambda V\widetilde{A})\to N_* (V\widetilde{A})\to 0.\]
Suppose that $x\in N_n(V\widetilde{A})$ is a cycle (so that $d_ix=0$ for $0\leq i \leq n$). Then as the homomorphism $\lambda$ is just given by $d_0$, 
\[\xymatrix@R=4mm{
&%r1c1
N_n(\widetilde{A}\times \Lambda V\widetilde{A})\ar[r]^-{\lambda}
\ar[d]^-{\partial}
&%r1c2
N_n(V\widetilde{A})
&
&%r1c1
{(0,y)}\ar@{|->}[r]^-{d_0}
\ar@{|->}[d]^-{d_0}
&%r1c2
x
\\%r1c6
N_{n-1}(D^1\widetilde{A})\ar[r]
&%r2c1
N_{n-1}(\widetilde{A}\times \Lambda V\widetilde{A})\ar[r]&%r2c2
N_{n-1}(V\widetilde{A})
&
(0,x)\ar@{|->}[r]
&%r2c1
(0,x)\ar@{|->}@{|->}[r]^-{d_0}&%r2c2
0
}\]
This is to say that the connecting homomorphism is induced by the map $Z_*NX\to Z_{*-1}\Omega X$ given by $x\mapsto x$. Now we relate $D^1\widetilde{A}$ to $\Omega V\widetilde{A}$ using the following diagram:
\[\xymatrix@R=4mm{
c(\widetilde{A}^{\otimes 2})\ar[dr]
&%r1c1
&%r1c2
\Omega V\widetilde{A}\ar[dl]
\ar[dd]
\ar[dr]
&%r1c3
\\%r1c4
&%r2c1
D^1\widetilde{A}\ar[dd]
\ar[rr]
&%r2c2
&%r2c3
\Lambda V\widetilde{A}\ar[dd]^-{\lambda}
\\%r2c4
&%r3c1
&%r3c2
0\ar[dl]
\ar[rd]
&%r3c3
\\%r3c4
&%r4c1
\widetilde{A}\ar[rr]_-{\nu}
&%r4c2
&%r4c3
V\widetilde{A}%r4c4
}\]
\end{Thoughts on Adams Multiplicativity II}
\begin{prereqs for Thoughts III}
\section{Multiplicativity of the Adams sseq}
%\section{Adams multiplicativity}
Let $\widetilde{A}^n:=c(K\overline{Q})^{n+1}A$, for $A$ a simplicial algebra. Then $\widetilde{A}^\bullet$ is a levelwise cofibrant cosimplicial object, with a Hurewicz morphism to $Q\widetilde{A}^\bullet$. However, $Q\widetilde{A}^\bullet$ is contractible, and the cosimplicial homology coalgebra $\pi_*(Q\widetilde{A}^\bullet)$ is thus a cofree resolution of $H_*(A)$. The Hurewicz morphism is in each level an isomorphism onto the primitives:
\[\pi_*(\widetilde{A}^\bullet)\overset{\simeq}{\to}P\pi_*(Q\widetilde{A}^\bullet)\overset{}{\to}\pi_*(Q\widetilde{A}^\bullet)\]
so that the $E^2$ term is derived primitives on $H_*(A)$. As such, it should admit a product. In fact, the product should shift both gradings by 1, the internal grading $*$ and the filtration grading $\bullet$.

\subsection{Multiplicativity via the $D^1$-construction}
We'll use the construction of Bousfield-Kan \cite{BK_pairings_products.pdf,BK_pairings.pdf}, applied to cosimplicial simplicial algebras $X$:
\[\xymatrix@R=4mm{
D^1X\ar[r]
\ar[d]^-{\delta}
&%r1c1
\Lambda VX\ar[d]^-{\lambda}
\\%r1c2
X\ar[r]^-{v}
&%r2c1
VX%r2c2
}\]
Now $VX$ is the cosimplicial object given by shifting $X$ down and forgetting the 0th face and degeneracy. $d^0$ induces a map $v:X\to VX$. Amazing. Next the endofunctor $\Lambda:s\Comm\to s\Comm$ is applied in each cosimplicial level. This is the ``standard simplicial path fibration'' (c.f.\ \cite[p.82]{BousKanSSeq.pdf}), which acts on $Y\in s\Comm$ by shifting down and restricting to a kernel: $\Lambda Y_s=\ker(d_{s+1}\cdots d_1:Y_{s+1}\to Y_0)$; again we forget the 0th face and degeneracy. This time, $d_0:\Lambda Y\to Y$ is a fibration from an acyclic object.

The game will be to create a factorization of the multiplication map $\mu:\widetilde{A}\otimes \widetilde{A}\to\widetilde{A}$ through $\delta:D^1\widetilde{A}\to A$. This is possible (using a zig-zag) in the case of interest to us --- when $X=\widetilde{A}$ is the Radelescu-Banu resolution of $A$. In this case, not only is $V\widetilde{A}^n=c(K\overline{Q})^{n+2}A$ a cosimplicial object, but $V_\textup{fl}\widetilde{A}^n:=(K\overline{Q})^{n+2}A$ is cosimplicial: we don't need the outermost cofibrant replacement, as we've discarded $d_0$. Of course there's a cosimplicial map $\epsilon:V\widetilde{A}\to V_\textup{fl}\widetilde{A}$ which is a levelwise weak equivalence. Finally, one notes that the composite $\epsilon\circ v:\widetilde{A}\to V_\textup{fl}\widetilde{A}$ is a levelwise fibration of simplicial algebras. 

This discussion leads us to define $\overline{D}^1X$ to be the pullback of the first column in the following diagram:
\[\xymatrix@R=4mm{
&%r1c1
0\ar[d]
\ar[r]
&%r1c2
\Lambda X^{\fraks+1}_\textup{fl}\ar[d]
&%r1c3
\Lambda X^{\fraks+1}\ar[l]^-{\Lambda(\epsilon)}
\ar[d]
\\%r1c4
X^\fraks\otimes X^\fraks\ar[ur]^-{0}
\ar[dr]_-{\mu}
&%r2c1
X^{\fraks+1}_\textup{fl}\ar@{=}[r]
&%r2c2
X^{\fraks+1}_\textup{fl}&%r2c3
X^{\fraks+1}\ar[l]^-{\epsilon}
\\%r2c4
&%r3c1
X\ar@{->>}[u]_-{\epsilon\circ d^0}
\ar@{=}[r]
&%r3c2
X\ar@{->>}[u]_-{\epsilon\circ d^0}&%r3c3
X\ar[u]_-{d^0}\ar@{=}[l]%r3c4
}\]
This diagram constructs both a map $\overline{\mu}:X^\fraks\otimes X^\fraks\to(\overline{D}^1X)^\fraks$ and a zig-zag of weak equivalences from $\overline{D}^1X$ to $D^1X$. Of course, the resulting connecting homomorphisms commute:
\[\mathclap{\xymatrix@R=4mm{
\pi_{\frakt}((\overline{D}^1X)^\fraks)&%r1c3
\pi_{\frakt+1}(X_\textup{fl}^{\fraks+1})\ar[l]^-{\cong\textup{-ish}}_-{\partial_\textup{conn}}
\\%r1c4
\pi_{\frakt}((D^1X)^\fraks)\ar@{-}[u]^-{\textup{zig-zag}}_-{\cong}
&%r2c3
\pi_{\frakt+1}(X^{\fraks+1})\ar[u]_-{\cong}
\ar[l]^-{\cong\textup{-ish}}_-{\partial_\textup{conn}}
}}\]


\subsection{A chain-level construction $\xi_\textup{res}$ of $\xi_\textup{alg}$}
\[\mathclap{\xymatrix@R=0mm{
(HA)^{\wedge 2}
\ar@{-->}@/^3em/[rrrrdd]^-(.9){\xi_\textup{alg}}
\\
&%r1c1
(HA)^{\times2}
\ar[r]^-{d^0\times d^0}
\ar@{}[dd]|{\times}="timesOne"
&%r1c2
(HA_+)^{\times2}
\ar[r]^-{\textup{hopf}}
\ar@{}[dd]|{\times}="timesTwo"&%r1c3
\ar@{}[dd]|{\times}="timesThree"
HA_+&%r1c4
\\%r1c5
(HA)^{\times2}\ar;"timesOne"_-{\textup{diag}}
\ar@{-->}@/^3.9em/[rrrr]^-(.1){\overline{\xi}_\textup{alg}}
\ar@{->>}[uu]
&%r2c1
&%r2c2
&%r2c3
&%r2c4
HA_+\ar"timesThree";[]^-{\textup{hopf}}
\\%r2c5
&%r3c1
(HA)^{\times2}\ar[r]_-{\textup{hopf}}
&%r3c2
HA\ar[r]_-{d^0}
&%r3c3
HA_+&%r3c4
\\
(HA)^{\sqcup2}
\ar[uu]
\\\\
&&\raisebox{.5em}[0mm][0mm]{\makebox[0cm][c]{\,\textup{is $\pi(Q\DASH)$ applied to}}}
\\
\textup{cof}
\ar@{-->}@/^1em/[rrrrdd]^-(.9){\xi_\textup{res}}
\\\\%r3c5
\ar[uu]
\ar@{-->}@/^1.3em/[rrrr]^-(.1){\overline{\xi}_\textup{res}}
c(A^2)\ar[r]_-{c(\textup{diag})\circ\beta}
&%r4c1
c(A^2\times c(A^2))\ar[r]_-{c(((d^0)^2)\times h)}
%\ar[ddd]
&%r4c2
c(A_+^2\times A)
\ar[r]_-{c(h\times (\epsilon\circ d^0))}
%\ar[ddd]
&%r4c3
c(K_+\times K_+)\ar[r]_-{h}
%\ar[dd]^-{\beta}
&%r4c4
A_+\\\\%r4c5
c(A^{\sqcup2})\ar[uu]
}}\]
Here, $A$ was being used as a *terrible* shorthand for $\widetilde{A}^n:=c(K\overline{Q})^{n+1}A$, and unadorned superscripts were used to indicate iterated products. The cofibre `$\textup{cof}$' is that of the map $A^{\sqcup2}\to A^2$. If we just write $k$ for the zero-square algebra $(K\overline{Q})^{n+1}A$, then the map $h:c(A^2)\to A$ is defined to be the composite:
\[c(A^{2})=c((ck)^{2})\overset{c(\epsilon^{2})}{\to}c(k^2)\overset{c(\textup{add})}{\to}ck=A,\]
the point being that this is an algebra map, where $\textup{add}:A^2\to A$ is not.
\begin{lem}
The map $\overline{\xi}_\textup{res}$ factors through the cofibre.
\end{lem}
\begin{proof}
It's enough to check that the composite $c(A^{\sqcup2})\to c(A^{\times2})\to c(A_+)$ is null. However, the codomain $c(A_+)$ is a GEM, so that a map to $c(A_+)$ is null if it is zero on cohomology. Fortunately, we have already described the maps on homology induced by these two maps --- they are the inclusion $(HA)^{\sqcup2}\to(HA)^{\times2}$ and $\overline{\xi}:(HA)^{\times2}\to HA_+$ respectively, and the composite of these two maps is zero.
\end{proof}


\subsection{A chain level construction of $j$}
Write $S^n$ for (the free construction on) $\Delta^n/\partial\Delta^n$, and $CS^n$ for (the free construction on) the reduced cone thereupon. Now there's a map $S^{n+m}\to S^n\sqcup S^m$ sending the fundamental class to the Eilenberg-Zilber formula. That is, if $\mu:(S^n\sqcup S^m)^{\otimes 2}\to S^n\sqcup S^m$ is multiplication, the map we are describing sends $\imath_{n+m}\mapsto \mu(\textup{EZ}(\imath_n,\imath_m))$. Now consider the complex $C_{n,m}$ formed by coning off the corresponding homotopy class:
\[\xymatrix@R=4mm{
S^{n+m}\ar[r]
\ar@{ >->}[d]
&%r1c1
S^n\sqcup S^m\ar@{ >->}[d]
\\%r1c2
CS^{n+m}\ar[r]
&%r2c1
C{n,m}%r2c2
}\]
Now $C_{n,m}$ has an $n+m+1$-dimensional generator $g$, the image of the cone on the fundamental class of $S^{n+m}$. The homology LES shows that $H_*(C_{n,m})$ is three-dimensional, containing classes $\imath_n$, $\imath_m$ and $g$.
\begin{prop}
The diagonal of $g$ is $\imath_n\otimes\imath_m+\imath_m\otimes\imath_n$.
\end{prop}
\begin{proof}
The representative $g$ has the property that $d_0(g)=\mu(\textup{EZ}(\imath_n,\imath_m))$ and $d_i(g)=0$ for $i>0$. We calculate 
\begin{alignat*}{2}
\psi(d_0(g))
&=
\mu(\textup{EZ}(\overline{\imath_n}+\overline{\overline{\imath_n}},\overline{\imath_m}+\overline{\overline{\imath_m}}))%
\\
\xi(g)
&=
\mu(\textup{EZ}(\overline{\imath_n},\overline{\overline{\imath_m}})+\textup{EZ}(\overline{\overline{\imath_n}},\overline{\imath_m}))\in Q((C_{n,m})_{n+m}^{\wedge 2})%
\\
% Left hand side
j(\xi(g))
% Relation
&=
% Right hand side
\textup{EZ}^\otimes ({\imath_n},{{\imath_m}})+\textup{EZ}^\otimes ({\imath_m},{{\imath_n}})\in (Q((C_{n,m})_{n+m}))^{\otimes 2}%
% Comment
\end{alignat*}
This class represents $\imath_n\otimes\imath_m+\imath_m\otimes\imath_n$.
\end{proof}
%Now suppose that $L$ is a (not necessarily zero-square) algebra, and that $\alpha,\beta$ are homotopy classes in $\pi_n (L)$ and $\pi_m (L)$ respectively. Then we'll be able to use the complex $C_{n,m}$ to construct a homology class $g_{\alpha,\beta}\in H_{n+m+1}(L)$ whose diagonal is $\alpha\otimes\beta+\beta\otimes\alpha$ [probably just when $L$ is zero square, using the map $L\times L\to L$. Else, get a class in $H(L^2)$ with diagonal $\overline{\alpha}\otimes\overline{\overline{\beta}}+ \overline{\overline{\beta}}\otimes\overline{\alpha}$]. Firstly, note that there is a commuting diagram
%\[\xymatrix@R=4mm{
%0\ar[r]
%\ar@{ >->}[d]
%&%r1c1
%0\ar[d]
%\\%r1c2
%S^n\sqcup S^m\ar[r]^-{\alpha\sqcup\beta}
%&%r2c1
%L^{\times2}%r2c2
%}\]
%which furnishes $c_1(L^{\times2})$ with a map from $S^n\sqcup S^m$. Next, there's a diagram %\newdir{ >}{{}*!/-7pt/\dir{>}}
%\[\xymatrix@R=4mm{
%S^{n+m}\ar@{->}[r]_-{\textup{EZ}^\mu(\imath_n,\imath_m)}
%\ar@{ >->}[d]
%&%r1c1
%S^n\sqcup S^m\ar@{ >->}[d]
%\ar[r]
%&c_1(L^{\times2})\ar[d]
%\\%r1c2
%CS^{n+m}\ar[r]\ar@/_1em/[rr]_-{0}
%&%r2c1
%C{n,m}\ar@{..>}[r]
%%r2c2
%&L^{\times2}}\]
%where the outer rectangle commutes since in $L^{\times2}$, $(a,0)\cdot(0,b)=(0,0)$. This gives a map $C_{n,m}\to c_2(L^{\times2})$ whose restriction to the subcomplex $S^n\sqcup S^m$ gives the homotopy classes $\alpha$ and $\beta$. The image $g_{\alpha,\beta}$ of the class $g\in(C_{n,m})_{n+m+1}$ is then an explicit representative for the homology class sought.
We can use this cofibration to build a chain level construction of the map $j:\textup{Pr}_t(HX)\otimes \textup{Pr}_{t'}(HY)\to \textup{Pr}_{t+t'+1}(HX\wedge HY)$ on spherical homology classes:
\begin{prop}
There is a natural function (i.e.\ map of sets)
\[\overline{F}:\hom_{s\Comm}(S^t,X)\times\hom_{s\Comm}(S^{t'},Y)\to \hom_{s\Comm}(C_{t,t'},c(X\times Y))\]
and the function
\[F:\hom_{s\Comm}(S^t,X)\times\hom_{s\Comm}(S^{t'},Y)\to \pi_{\frakt+1}(Qc(X\times Y)):=H_{\frakt+1}(X\times Y)\]
defined by $F(\alpha,\beta):=H_*(\overline{F}(\alpha,\beta))(g)$ makes the following diagram commute:
\[\mathclap{\xymatrix@R=4mm{
%(H(X\times Y)^{\otimes2})_\frakt\ar[d]^-{\textup{id}+\tau}
%&
\ar@/_2em/@<-3ex>[dd]_-{(\textup{id}+\tau)\circ \textup{hur}^{\otimes2}}
%\ar[l]_-{\textup{hur}^{\otimes2}}
\hom_{s\Comm}(S^t,X)\times\hom_{s\Comm}(S^{t'},Y)\ar[d]^-{F}
\ar[r]^-{\textup{hur}^{\otimes2}}
&%r1c1
\textup{Pr}_t(HX)\otimes\textup{Pr}_{t'}(HY)\ar[r]^-{j}
&%r1c2
\textup{Pr}_{\frakt+1}(HX\wedge HY)\ar@{ >->}[d]
\\%r1c5
%(H(X\times Y)^{\otimes2})_\frakt
%&
H_{\frakt+1}(X\times Y)\ar[r]%\ar[l]_-{\Delta}
\ar[d]^-{\Delta}
&%r2c1
(HX\times HY)_{\frakt+1}\ar@{->>}[r]
\ar[d]^-{\Delta}
&%r2c2
(HX\wedge  HY)_{\frakt+1}
\\
%&
(H(X\times Y)^{\otimes2})_\frakt\ar[r]
&((HX\times HY)^{\otimes2})_\frakt
}}\]
\end{prop}
\begin{proof}
The value of $\overline{F}$ on $(\alpha,\beta)$ is defined as follows. Corresponding to the commuting diagrams
\[\xymatrix@R=4mm{
0\ar[r]
\ar@{ >->}[d]
&%r1c1
0\ar[d]
\\%r1c2
S^t\ar[r]^-{(\alpha,0)}
&%r2c1
X\times Y%r2c2
}
\textup{\quad and\quad }
\xymatrix@R=4mm{
0\ar[r]
\ar@{ >->}[d]
&%r1c1
0\ar[d]
\\%r1c2
S^{t'}\ar[r]^-{(0,\beta)}
&%r2c1
X\times Y%r2c2
}\]
are maps $\widetilde{(\alpha,0)}:S^t\to c_1(X\times Y)$ and $\widetilde{(0,\beta)}:S^{t'}\to c_1(X\times Y)$, and there is a commuting diagram (since in $X\times Y$, products of the form $(a,0)(0,b)$ vanish):
\[\xymatrix@R=4mm@C=15mm{
S^{\frakt}\ar@{->}[r]_-{\textup{EZ}^\mu(\imath_t,\imath_{t'})}
\ar@{ >->}[d]
&%r1c1
S^t\sqcup S^{t'}\ar@{ >->}[d]
\ar[r]^{\widetilde{(\alpha,0)}\sqcup\widetilde{(0,\beta)}}
&c_1(X\times Y)\ar[d]
\\%r1c2
CS^{\frakt}\ar[r]\ar@/_1em/[rr]_-{0}
&%r2c1
C{t,t'}\ar@{..>}[r]
%r2c2
&X\times Y}\]
Corresponding to the right square is a map $C_{t,t'}\to c_2(X\times Y)$, and the composite with the cofibration $c_2(X\times Y)\to c(X\times Y)$ is $\overline{F}(\alpha,\beta)$. This function $\overline{F}$ is evidently natural in $X$ and $Y$, and so then is $F$.

The square at the bottom of the diagram commutes as the horizontals are maps of homology Lie coalgebras, and we can see that the triangle commutes because we understand the Lie coalgebra structure of $H_*(C_{t,t'})$. As all of the maps in the six-arrow commuting diagram  are natural, we may check it on the universal example: $(\imath_t,\imath_{t'})\in\hom_{s\Comm}(S^t,S^t)\times\hom_{s\Comm}(S^{t'},S^{t'})$. That is, it's enough to check that the following diagram commutes:
\[\mathclap{\xymatrix@R=4mm{
\{(\imath_t,\imath_{t'})\}\ar[d]^-{F}
\ar[r]^-{\textup{hur}^{\otimes2}}
&%r1c1
\textup{Pr}_t(HS^t)\otimes\textup{Pr}_{t'}(HS^{t'})\ar[r]^-{j}
&%r1c2
\textup{Pr}_{\frakt+1}(HS^t\wedge HS^{t'})\ar@{ >->}[d]^-{\textup{inc}}
\\%r1c5
H_{\frakt+1}(S^t\times S^{t'})\ar[r]^-{r}
&%r2c1
(HS^t\times HS^{t'})_{\frakt+1}\ar@{->>}[r]^-{\textup{proj}}
&%r2c2
(HS^t\wedge  HS^{t'})_{\frakt+1}
}}\]
The rightmost four vector spaces in this diagram are 1-dimensional, and the maps $j$, $\textup{inc}$ and $\textup{proj}$ connecting them are isomorphisms, so it's enough to check that $r(F(\imath_t,\imath_{t'}))$ is nonzero. However, using the commuting square and triangle already established, we may calculate that $\Delta(r(F(\imath_t,\imath_{t'})))=\imath_t\otimes\imath_{t'}+\imath_{t'}\otimes\imath_{t}$, which is nonzero.
\end{proof}

\subsection{A left inverse for the Hurewicz}
Suppose that $k$ is a zero-square simplicial algebra. Then $Qk=k$, giving a commuting diagram
\[\xymatrix@R=4mm{
k\ar@{=}[r]
&%r1c1
Qk&%r1c2
&%r1c3
\pi_*(k)\ar@{=}[r]\ar[ddr]_-(.75){\ \ \textup{hur}}
&%r1c1
\pi_*(Qk)\\%r1c5
ck\ar[u]^-{\epsilon}
\ar[r]
&%r2c1
Qck\ar[u]^-{Q\epsilon}
&%r2c2
\textup{yielding}&%r2c3
\pi_*(ck)\ar[u]^-{\pi_*(\epsilon)}_-{\cong}
\ar[r]
\ar[d]^-{\cong}
&%r2c1
\pi_*(Qck)\ar[u]_-{\pi_*(Q\epsilon)}\ar@{=}[d]
\\%r2c5
&%r3c1
&%r3c2
&%r3c3
\textup{Pr}(H_*k)\ar@{ >->}[r]
&%r3c4
H_*k%r3c5
}\]
which is to say that $\pi_*(Q\epsilon)$ provides a left inverse for the Hurewicz.
\end{prereqs for Thoughts III}
\begin{Thoughts on Adams Multiplicativity III}
\subsection{$E^1$-level identification of the product}
Throughout, I'll write $\fraks=s+s'$ and $\frakt=t+t'$ in order to save a bit of space. We're going to need the following commutative diagram, where $X$ is the standard cosimplicial resolution of a simplicial algebra, and $\epsilon:X\to X_\textup{fl}$ is collapsing off the outermost cofibrant replacement:
\[\mathclap{\xymatrix@R=4mm{
\pi_t(X^\fraks)\otimes \pi_{t'}(X^\fraks)\ar[dd]^-{h\otimes h}
\ar[r]^-{\textup{sEZ}}
&%r1c1
\pi_{\frakt}(X^\fraks\otimes X^\fraks)\ar[r]
&%r1c2
\pi_{\frakt}((\overline{D}^1X)^\fraks)&%r1c3
\pi_{\frakt+1}(X_\textup{fl}^{\fraks+1})
\ar[l]^-{\cong\textup{-ish}}_-{\partial_\textup{conn}}
\\%r1c4
&%r2c1
&%r2c2
\pi_{\frakt}((D^1X)^\fraks)\ar@{-}[u]^-{\textup{zig-zag}}_-{\cong}
&%r2c3
\pi_{\frakt+1}(X^{\fraks+1})\ar[u]_-{\cong}
\ar[l]^-{\cong\textup{-ish}}_-{\partial_\textup{conn}}
\ar[d]^-{\cong}
\\%r2c4
\textup{Pr}_t(HX^{\fraks})\otimes \textup{Pr}_{t'}(HX^{\fraks})\ar[r]^-{j}
&%r3c1
\textup{Pr}_{\frakt+1}(HX^\fraks\wedge HX^\fraks)\ar[rr]^-{\xi_\textup{alg}}
&%r3c2
&%r3c3
\textup{Pr}_{\frakt+1}(HX^{\fraks+1})%r3c4
}}\]
It helps to modify this diagram a little. Indeed, for each $(\alpha,\beta)\in \hom_{s\Comm}(S^t,X^\fraks)\times\hom_{s\Comm}(S^{t'},X^\fraks)$, there is a diagram starting with the singleton set $\{(\alpha,\beta)\}$:% such that $\alpha=\widetilde{\epsilon\alpha}$ and $\beta=\widetilde{\epsilon\beta}$
%\[\xymatrix@R=4mm{
%\pi_t(X^\fraks)\otimes \pi_{t'}(X^\fraks)\ar@{~>}[drd]^-{j\circ(h\otimes h)}
%\ar[r]^-{\textup{sEZ}}
%\ar@{..>}[ddddr]_-(.3){A}
%&%r1c1
%\pi_{\frakt}(X^\fraks\otimes X^\fraks)\ar[r]
%&%r1c2
%\pi_{\frakt}((\overline{D}^1X)^\fraks)&%r1c3
%\pi_{\frakt+1}(X_\textup{fl}^{\fraks+1})\ar[l]^-{\cong}_-{\partial}
%\ar@{~}[dd]^-{\textup{zig-zag}}_-{\cong}
%\\%r1c4
%&%r2c1
%&%r2c2
%&%r2c3
%%\pi_{\frakt+1}(X^{\fraks+1})\ar[u]_-{\cong}
%%\ar[d]^-{\cong}
%\\%r2c4
%%\textup{Pr}_t(HX^{\fraks})\otimes \textup{Pr}_{t'}(HX^{\fraks})\ar[r]^-{j}
%&%r3c1
%\textup{Pr}_{\frakt+1}(HX^\fraks\wedge HX^\fraks)
%\ar@{~>}[rr]_-(.82){\xi_\textup{alg}}
%\ar@{ >->}[dl]
%\ar@{ >->}[d]
%&%r3c2
%&%r3c3
%\textup{Pr}_{\frakt+1}(HX^{\fraks+1})\ar@{ >->}[d]
%\\%r3c4
%(HX^\fraks\wedge HX^\fraks)_{\frakt+1}\ar@{=}[r]
%&\pi_{\frakt+1}Q(\textup{cof})
%\ar[rr]^-(.8){(\xi_\textup{res})_*}
%&
%&
%H_{\frakt+1}X^{\fraks+1}
%\\
%(HX^\fraks\times HX^\fraks)_{\frakt+1}\ar@{=}[r]\ar@{->>}[u]
%&\pi_{\frakt+1}Qc(X^\fraks\times X^\fraks)\ar@{->>}[u]
%\ar[urr]_-(.76){(\overline{\xi}_\textup{res})_*}
%\ar@{..>}[uuuurr]^-(.75){B}
%\\
%(HX^\fraks\oplus HX^\fraks)_{\frakt+1}\ar@{=}[r]\ar@{ >->}[u]
%&\pi_{\frakt+1}Q(X^\fraks\sqcup X^\fraks)\ar@{ >->}[u]
%}\]
%The method will be to introduce the dotted maps $A$ and $B$ and show that the composite $B\circ A$ equals the composite of the wavy maps. Then we will be able to calculate the composite $\partial\circ B\circ A$, and compare it with the composite along the top row.
\[\mathclap{\xymatrix@R=4mm{
\{(\alpha,\beta)\}
%\pi_t(X^\fraks)\otimes \pi_{t'}(X^\fraks)
\ar@{~>}[dd]_-{j\circ(h\otimes h)}
\ar[r]^-{\textup{sEZ}}
\ar@{..>}[ddddr]^-(.2){F}
&%r1c1
\pi_{\frakt}(X^\fraks\otimes X^\fraks)\ar[r]^-{\overline{\mu}_*}
&%r1c2
\pi_{\frakt}((\overline{D}^1X)^\fraks)&%r1c3
\pi_{\frakt+1}(X_\textup{fl}^{\fraks+1})\ar[l]_-{\partial_\textup{conn}}
\ar@{~}[dd]^-{\textup{zig-zag}}_-{\cong}
\\%r1c4
&%r2c1
&%r2c2
\pi_{\frakt+1}(QX_\textup{fl}^{\fraks+1})\ar@{..>}[ur]_-{=}&%r2c3
%\pi_{\frakt+1}(X^{\fraks+1})\ar[u]_-{\cong}
%\ar[d]^-{\cong}
\\%r2c4
%\textup{Pr}_t(HX^{\fraks})\otimes \textup{Pr}_{t'}(HX^{\fraks})\ar[r]^-{j}
%r3c1
\textup{Pr}_{\frakt+1}(HX^\fraks\wedge HX^\fraks)
\ar@{~>}[rrr]^-{\xi_\textup{alg}}
\ar@{ >->}[d]
%\ar@{ >->}[dr]
&
&%r3c2
&%r3c3
\textup{Pr}_{\frakt+1}(HX^{\fraks+1})\ar@{ >->}[d]
\\%r3c4
(HX^\fraks\wedge HX^\fraks)_{\frakt+1}\ar@{=}[r]
&\pi_{\frakt+1}Q(\textup{cof})
\ar[r]^-{\pi_*(Q\xi_\textup{res})}
&
\pi_{\frakt+1}QX^{\fraks+1}\ar@{=}[r]
\ar@{..>}[uu]_-(.75){\pi_*(Q\epsilon)}
&
H_{\frakt+1}X^{\fraks+1}
\\
(HX^\fraks\times HX^\fraks)_{\frakt+1}\ar@{=}[r]\ar@{->>}[u]
&\pi_{\frakt+1}Qc(X^\fraks\times X^\fraks)\ar@{->>}[u]
\ar@{..>}[ur]_-(.6){\pi_*(Q\overline{\xi}_\textup{res})}
\\
(HX^\fraks\oplus HX^\fraks)_{\frakt+1}\ar@{=}[r]\ar@{ >->}[u]
&\pi_{\frakt+1}Q(X^\fraks\sqcup X^\fraks)\ar@{ >->}[u]
}}\]
Although all of the arrows in this modified diagram have already been defined, we've decorated some of them for emphasis. It will be enough to check that it commutes whenever $\alpha=\widetilde{\epsilon\alpha}$ and $\beta=\widetilde{\epsilon\beta}$, since the collection of such $(\alpha,\beta)$ will exhaust all of the pure tensors in $\pi_t(X^\fraks)\otimes \pi_{t'}(X^\fraks)$.

The composite of the dotted maps equals the composite of the wavy maps, so it serves to investigate the composite
\[Qc(X^\fraks\times X^\fraks)\overset{Q\overline{\xi}_\textup{res}}{\to}QX^{\fraks+1}\overset{Q\epsilon}{\to}QX^{\fraks+1}_\textup{fl}=X^{\fraks+1}_\textup{fl}.\]
Fortunately, postcomposition with $Q\epsilon$ is pretty destructive:
\[\newcommand{\Times}{{\cdot}}
\def\labelstyle{\scriptstyle}
\mathclap{\xymatrix@R=4mm@C=10mm{
\ar@{-->}@/^1.3em/[rrrr]^-(.1){Q\overline{\xi}_\textup{res}}
Qc(A^2)\ar[r]%_-{Q(\cdots)}
%\ar[dr]_-{Q(\epsilon\pi_1)\Times Q(\epsilon\pi_2)\Times \textup{id\ \ \ }}
\ar[d]^-{(Qc\pi_1)\Times (Qc\pi_2)\Times \textup{id}}
&%r4c1
Qc(A^2\Times cA^2)\ar[r]_-{Qc((d^0)^2\Times h)}
\ar[d]^-{Q\epsilon}
&%r4c2
Qc(A_+^2\Times A)
\ar[r]_-{Qc(h'\Times (\epsilon\circ d^0))}
\ar[d]^-{Q\epsilon}
&%r4c3
Qc(K_+\Times K_+)\ar[r]_-{Qh'}
\ar[d]^-{Q\epsilon}
&%r4c4
QA_+\ar[d]^-{Q\epsilon}\\
(QcA)^2\Times Qc(A^2)
\ar[r]_-{(Q\epsilon)^2\Times\textup{id}}
&
(QA)^2\Times QcA^2
\ar[r]^-{(Qd^0)^2\Times Qh}
%\ar[dr]_-{(Q(\epsilon\circ d^0))^2\Times Q(d^0\circ h)}
\ar@/_1em/[dr]_-(.3){(Q(\epsilon\circ d^0))^2\Times Q(d^0\circ h)\ \ \ }
&
(QA_+)^2\Times QA
%\ar[r]^-{Qh'\Times Q(\epsilon\circ d^0)}
\ar[d]_-{(Q\epsilon)^2\Times Qd^0}
&
QK_+\Times QK_+\ar[r]^-{\textup{add}}
%\ar@{<-}@/^1em/[dl]^-(.3){\textup{add}\times Q\epsilon}
&
QK_+
\\
&
&
(QK_+)^2\Times QA_+%\ar[ur]_-{\textup{add}\times Q\epsilon}
\ar[ur]^-{\textup{add}\Times Q\epsilon}\ar[r]^-{(\textup{id})^2\cdot Q\epsilon}
&
(QK_+)^3
\ar@/_.5em/[ur]_-{\textup{add}}
}}\]
Finally, the counitality of the comonad $c$ implies that $\epsilon\circ d^0:A\to K_+$ coincides with the unit $\eta:\textup{id}\Rightarrow KQ$ of the adjunction between $K$ and $Q$. Thus we may conclude that the composite in question is the sum of the three composites
\[\xymatrix@R=4mm{
&%r1c1
QcA\ar@/^.55em/[dr]^-{Q\epsilon}
&%r1c2
&%r1c3
&%r1c4
\\%r1c5
Qc(A^2)\ar[rr]^-{Qh}
\ar@/^.5em/[ur]^-{Qc(\pi_1)}
\ar@/_.5em/[dr]_-{Qc(\pi_2)}
&%r2c1
&%r2c2
QA\ar[r]^-{Q\eta}
&%r2c3
QK_+\makebox[0cm][l]{$\,=K_+$}%\ar@{=}[r]
\\%r2c5
&%r3c1
QcA\ar@/_.55em/[ur]_-{Q\epsilon}&%r3c2
&%r3c3
}\]
We must be getting close. Indeed, write `$\textup{dot}$' for the composite of the dotted maps, and write $g\in QC_{t,t'}$ for the a representative of $g\in\pi_{\frakt+1}(QC_{t,t'})$. A representative for $\textup{dot}(\alpha,\beta)$ is given by the image of $g$ under the composite
\[QC_{t,t'}\overset{Q\overline{F}(\alpha,\beta)}{\to}Qc(X^\fraks\times X^\fraks)\overset{Q\overline{\xi}_\textup{res}}{\to}QX^{\fraks+1}\overset{Q\epsilon}{\to}QX^{\fraks+1}_\textup{fl}=X^{\fraks+1}_\textup{fl},\]
which is the sum of the images of $g$ under the three composites
\[\xymatrix@R=4mm{
&
&%r1c1
QcA\ar@/^.55em/[dr]^-{Q\epsilon}
&%r1c2
&%r1c3
&%r1c4
\\%r1c5
QC_{t,t'}
\ar[r]^-{Q\overline{F}(\alpha,\beta)}
&Qc(A^2)\ar[rr]^-{Qh}
\ar@/^.5em/[ur]^-{Qc(\pi_1)}
\ar@/_.5em/[dr]_-{Qc(\pi_2)}
&%r2c1
&%r2c2
QA\ar[r]^-{Q\eta}
&%r2c3
QK_+\makebox[0cm][l]{$\,=K_+$}%\ar@{=}[r]
\\%r2c5
&
&%r3c1
QcA\ar@/_.55em/[ur]_-{Q\epsilon}&%r3c2
&%r3c3
}\]
Let's do the top composite first (the bottom composite is similar). Firstly, by naturality of $\overline{F}$, we have that $c(\pi_1)\circ\overline{F}(\alpha,\beta)=\overline{F}(\alpha,0):C_{t,t'}\to c(A\times 0)$. As such, $\epsilon\circ c(\pi_1)\circ\overline{F}(\alpha,\beta):C_{t,t'}\to A$ is the map out of the pushout in the diagram:
\[\xymatrix@R=4mm@C=15mm{
S^{\frakt}\ar@{->}[r]_-{\textup{EZ}^\mu(\imath_t,\imath_{t'})}
\ar@{ >->}[d]
&%r1c1
S^t\sqcup S^{t'}\ar@{ >->}[d]
\ar[dr]^{\alpha\sqcup0}
%&c_1(X\times Y)\ar[d]
\\%r1c2
CS^{\frakt}\ar[r]\ar@/_1em/[rr]_-{0}
&%r2c1
C{t,t'}\ar@{..>}[r]
%r2c2
&A}\]
The point then is that $g\in CS^\frakt$ is a (normalized) chain with the property that $d_0(g)$ is the image of the fundamental class in $S^\frakt$. In $C_{t,t'}$, $d_0(g)$ is identified with $EZ^\mu(\imath_t,\imath_{t'})$, which is decomposable, so that $g$ becomes a cycle in $QC_{t,t'}$, representing the $(t+t'+1)$-dimensional homology class. As this $g$ is in the image of the map $CS^\frakt\to C_{t,t'}$, it maps to $0$ in $A$, and on homology, $g$ is annihilated by the top (and bottom) composite.
\begin{lem}
The composite 
\[S^t\sqcup S^{t'}\to C_{t,t'}\overset{\overline{F}(\alpha,\beta)}{\to}c(X^\fraks\times X^\fraks)\overset{h}{\to}X^\fraks\]
equals $\widetilde{(\epsilon\alpha)}\sqcup \widetilde{(\epsilon\beta)}$.
\end{lem}
\begin{proof}
Using naturality of $\overline{F}$ and the fact that $h=c(\textup{add}\circ(\epsilon\times\epsilon))$:
\[\xymatrix@R=4mm{
S^t\sqcup S^{t'}\ar@{ >->}[d]
%\ar[rrd]_-{\widetilde{(\epsilon\alpha,0)}\sqcup \widetilde{(0,\epsilon\beta)}}
\ar@/^1em/[rrd]_-{\widetilde{(\epsilon\alpha,0)}\sqcup \widetilde{(0,\epsilon\beta)}}
\ar@/^1em/[rrrd]^-(.7){\widetilde{(\epsilon\alpha)} \sqcup \widetilde{(\epsilon\beta)}}
&%r1c1
&%r1c2
&%r1c3
\\%r1c4
C_{t,t'}\ar[r]^-{\overline{F}(\alpha,\beta)}
\ar@/_1em/[rr]_-{\overline{F}(\epsilon\alpha,\epsilon\beta)}
&%r2c1
c(X^\fraks\times X^\fraks)\ar[r]^-{c(\epsilon\times\epsilon)}
&%r2c2
c(X_\textup{fr}^\fraks\times X_\textup{fr}^\fraks)\ar[r]_-{c(\textup{add})}
&%r2c3
cX_\textup{fl}^\fraks%r2c4
}\qedhere\]
\end{proof}

\begin{thm}
The modified diagram commutes (when $\alpha=\widetilde{(\epsilon \alpha)}$ and $\beta=\widetilde{(\epsilon \beta)}$).
\end{thm}
\begin{proof}
We must show that $\partial_\textup{conn}(\textup{dot}(\alpha,\beta))=\overline{\mu}_*(\textup{sEZ}(\alpha,\beta))$.
All of the previous discussion together has shown $\textup{dot}(\alpha,\beta)$ is the image in $QX^{\fraks+1}_\textup{fl}=X^{\fraks+1}_\textup{fl}$ of the cycle $g\in Qc_{t,t'}$ under the top row of the following commuting diagram:
\[\xymatrix@R=4mm{
\makebox[0cm][r]{$g\in\,$}QC_{t,t'}\ar[r]^-{Q\overline{F}(\alpha,\beta)}
&%r1c2
Qc(X^\fraks)^2\ar[r]^-{Qh}
&%r1c3
QX^\fraks\ar[r]^-{Q(\epsilon d^0)}
&%r1c4
QX^{\fraks+1}_\textup{fl}\\%r1c5
\makebox[0cm][r]{$g\in\,$}
C_{t,t'}\ar[r]^-{\overline{F}(\alpha,\beta)}
\ar@{->>}[u]
\ar[d]_-{d_0}
&%r1c2
c(X^\fraks)^2\ar[r]^-{h}
\ar@{->>}[u]
\ar[d]_-{d_0}
&%r1c3
X^\fraks\ar@{->>}[r]^-{\epsilon d^0}
\ar@{->>}[u]
\ar[d]_-{d_0}
&%r1c4
X^{\fraks+1}_\textup{fl}\makebox[0cm][l]{$\,\ni\textup{dot}(\alpha,\beta)$}\ar@{=}[u]
\\%r2c5
\makebox[0cm][r]{$\textup{EZ}^\mu(\imath_t,\imath_t')\in\,$}C_{t,t'}\ar[r]^-{\overline{F}(\alpha,\beta)}
\ar@{ >->}[d]
&%r1c2
c(X^\fraks)^2\ar[r]^-{h}
&%r1c3
X^\fraks\makebox[0cm][l]{$\,\ni\partial_\textup{conn}(\textup{dot}(\alpha,\beta))$}
&%r1c4
\\%r3c5
\makebox[0cm][r]{$\textup{EZ}^\mu(\imath_t,\imath_t')\in\,$}S^t\sqcup S^{t'}\ar[urr]_-{\widetilde{(\epsilon\alpha)} \sqcup \widetilde{(\epsilon\beta)}=\alpha\sqcup\beta}
&%r4c2
&%r4c3
&%r4c4
%r4c5
}\]
Moreover, at the right of this diagram we see exactly the maps needed in order to construct the connecting homomorphism for the SES 
\[0\to\overline{D}^1X^\fraks\to X^\fraks\overset{\epsilon d^0}{\to} X^{\fraks+1}_\textup{fl}\to 0,\]
 confirming that $\partial_\textup{conn}(\textup{dot}(\alpha,\beta))$ is the image under $\alpha\sqcup\beta$ of $\textup{EZ}^\mu(\imath_t,\imath_t')$, completing the proof.
\end{proof}
\end{Thoughts on Adams Multiplicativity III}

\begin{Adams Muliplicativity}
\section{Multiplicativity of the Adams sseq}
%\section{Adams multiplicativity}
Let $X^n:=c(K\overline{Q})^{n+1}A$, for $A$ a simplicial algebra. Then $X$ is a levelwise cofibrant cosimplicial object, with a Hurewicz morphism to $QX$. However, $QX$ is contractible, and the cosimplicial homology coalgebra $\pi_*(QX)$ is thus a cofree resolution of $H_*(A)$. The Hurewicz morphism is in each level an isomorphism onto the primitives:
\[\pi_*(X)\overset{\simeq}{\to}P\pi_*(QX)\overset{}{\to}\pi_*(QX)\]
so that the $E^2$ term is derived primitives on $H_*(A)$. As such, it should admit a product. In fact, the product should shift both gradings by 1, the internal grading $*$ and the filtration grading $\bullet$.

\subsection{Multiplicativity via the $D^1$-construction}
We'll use the construction of Bousfield-Kan \cite{BK_pairings_products.pdf,BK_pairings.pdf}, applied to cosimplicial simplicial algebras $Z$:
\[\xymatrix@R=4mm{
D^1Z\ar[r]
\ar[d]^-{\delta}
&%r1c1
\Lambda VZ\ar[d]^-{\lambda}
\\%r1c2
Z\ar[r]^-{v}
&%r2c1
VZ%r2c2
}\]
Now $VZ$ is the cosimplicial object given by shifting $Z$ down and forgetting the 0th face and degeneracy. $d^0$ induces a map $v:Z\to VZ$. Amazing. Next the endofunctor $\Lambda:s\Comm\to s\Comm$ is applied in each cosimplicial level. This is the standard simplicial path fibration (c.f.\ \cite[p.82]{BousKanSSeq.pdf}), which acts on $Y\in s\Comm$ by shifting down and restricting to a kernel: $\Lambda Y_s=\ker(d_{s+1}\cdots d_1:Y_{s+1}\to Y_0)$; again we forget the 0th face and degeneracy. This time, $d_0:\Lambda Y\to Y$ is a fibration from an acyclic object.

The goal will be to create a factorization of the multiplication map $\mu:X\wedge  X\to X$ through $\delta:D^1X\to X$. This is possible (using a zig-zag) when $X=c(K\overline{Q})^{\bullet+1}A$ is the Radelescu-Banu resolution of a simplicial algebra $A$. In this case, not only is $VX=c(K\overline{Q})^{\bullet+2}A$ a cosimplicial object, but $\overline{V}X:=(K\overline{Q})^{\bullet+2}A$ is cosimplicial: we don't need the outermost cofibrant replacement, as we've discarded $d^0$. Of course there's a cosimplicial map $\epsilon:VX\to \overline{V}X$ which is a levelwise weak equivalence. Finally, one notes that the composite $\overline{v}:=\epsilon\circ v:X\to \overline{V}X$ is a levelwise fibration of simplicial algebras, as it is given by the formula $\overline{v}=\eta:c(K\overline{Q})^{\bullet+2}A\to KQc(K\overline{Q})^{\bullet+2}A$. 

This discussion leads us to define $\overline{D}^1X$ to be the pullback of the first column in the following diagram in $cs\Comm$:
\[\xymatrix@R=4mm{
&%r1c1
0\ar[d]
\ar[r]
&%r1c2
\Lambda \overline{V}X\ar[d]^-{\lambda}
&%r1c3
\Lambda \overline{V}X\ar[l]^-{\Lambda(\epsilon)}
\ar[d]^-{\lambda}
\\%r1c4
X\wedge X\ar[ur]^-{0}
\ar[dr]_-{\mu}
&%r2c1
\overline{V}X\ar@{=}[r]
&%r2c2
\overline{V}X&%r2c3
VX\ar[l]^-{\epsilon}
\\%r2c4
&%r3c1
X\ar@{->>}[u]_-{\overline{v}}
\ar@{=}[r]
&%r3c2
X\ar@{->>}[u]_-{\overline{v}}&%r3c3
X\ar[u]_-{v}\ar@{=}[l]%r3c4
}\]
This diagram constructs both a factorisation $\overline{\mu}:X\wedge  X \to \overline{D}^1X$ (and even $X\wedge_{\Sigma_2} X\to \overline{D}^1X$) and a zig-zag of levelwise (write $E^2$?) weak equivalences from $\overline{D}^1X$ to $D^1X$. To understand why $\overline{v}\circ\mu=0$ in this diagram, note that $\mu:X\wedge X\to X$ factors through $P_2X$, so that $\overline{v}\circ\mu$ factors through $P_2(\overline{V}X)$, which is zero, as $\overline{V}X$ is abelian. [Recall (not yet written) that $Y\wedge Z$ is just $Y\otimes Z$ as a vector space, being the kernel of the natural surjection $Y\sqcup Z\to Y\times Z$]. Of course, the resulting connecting homomorphisms commute:
\[\mathclap{\xymatrix@R=4mm{
\pi_{\frakt}(\overline{D}^1X)&%r1c3
\pi_{\frakt+1}(\overline{V}X)\ar[l]^-{\cong\textup{-ish}}_-{\partial_\textup{conn}}
\\%r1c4
\pi_{\frakt}(D^1X)\ar@{-}[u]^-{\textup{zig-zag}}_-{\cong}
&%r2c3
\pi_{\frakt+1}(VX)\ar[u]_-{\cong}
\ar[l]^-{\cong\textup{-ish}}_-{\partial_\textup{conn}}
}}\]
I \textbf{guess} that what is meant by `$\cong$-ish' is actually that the composite
\[N^\fraks\pi_\frakt(D^1X)\overset{\partial_\textup{conn}}{\from}N^\fraks\pi_{\frakt+1}(VX)\supset N^{\fraks+1}\pi_{\frakt+1}(X)\]
is an isomorphism, where we note that the condition to lie in $N^{s+1}(O)\subset O^{s+1}$ is stricter than the condition to lie in $N^s(VO)\subset O^{s+1}$, for $O$ a cosimplicial module. Note that this composite is a cochain map, the reason being that the extra ``$d^0$'' in the coboundary out of $N^{\fraks+1}\pi_{\frakt+1}(X)$ that does not appear in the coboundary out of $N^{\fraks}\pi_{\frakt+1}(VX)$ becomes null after application of $\partial_\textup{conn}$, which is to say that the composite
\[\pi_{\frakt+1}(X^{\fraks+1})\overset{d^0}{\to}\pi_{\frakt+1}(X^{\fraks+2})\overset{\partial_\textup{conn}}{\to}\pi_\frakt((D^1X)^{\fraks+1})\]
is zero, which is clear, as we're just looking at adjacent maps in the homotopy LES.


\subsection{A chain-level construction $\xi_\textup{res}$ of $\xi_\textup{alg}$}
We'll here construct the algebraically defined map $\xi_\textup{alg}$ at the chain level. Note that this map is \emph{not} a cosimplicial map $(HX^\bullet)^{\times2}\to HVX^\bullet$, however it \emph{does} induce a chain map on the normalised complexes of primitives.
%\[\newcommand{\Times}{{\cdot}}
%\mathclap{\xymatrix@R=0mm{
%(HX^\fraks)^{\wedge 2}
%\ar@{-->}@/^3em/[rrrrdd]^-(.9){\xi_\textup{alg}}
%\\
%&%r1c1
%(HX^\fraks)^{\times2}
%\ar[r]^-{Hv\times Hv}
%\ar@{}[dd]|{\times}="timesOne"
%&%r1c2
%(HVX^\fraks)^{\times2}
%\ar[r]^-{\textup{hopf}}
%\ar@{}[dd]|{\times}="timesTwo"&%r1c3
%\ar@{}[dd]|{\times}="timesThree"
%HVX^\fraks&%r1c4
%\\%r1c5
%(HX^\fraks)^{\times2}\ar;"timesOne"_-{\textup{diag}}
%\ar@{-->}@/^3.9em/[rrrr]^-(.1){\overline{\xi}_\textup{alg}}
%\ar@{->>}[uu]
%&%r2c1
%&%r2c2
%&%r2c3
%&%r2c4
%HVX^\fraks\ar"timesThree";[]^-{\textup{hopf}}
%\\%r2c5
%&%r3c1
%(HX^\fraks)^{\times2}\ar[r]_-{\textup{hopf}}
%&%r3c2
%HX^\fraks\ar[r]_-{Hv}
%&%r3c3
%HVX^\fraks&%r3c4
%\\
%(HX^\fraks)^{\sqcup2}
%\ar[uu]
%\\\\
%&&\raisebox{.5em}[0mm][0mm]{\makebox[0cm][c]{\,\textup{is $\pi(Q\DASH)$ applied to}}}
%\\
%\textup{cof}
%\ar@{-->}@/^1em/[rrrrdd]^-(.9){\xi_\textup{res}}
%\\\\%r3c5
%\ar[uu]
%\ar@{-->}@/^1.3em/[rrrr]^-(.1){\overline{\xi}_\textup{res}}
%c((X^\fraks)^2)\ar[r]_-{c(\textup{diag})\circ\beta}
%&%r4c1
%c((X^\fraks)^2\Times c((X^\fraks)^2))\ar[r]_-{c(v^2\Times h)}
%%\ar[ddd]
%&%r4c2
%c((VX^\fraks)^2\Times X^\fraks)
%\ar[r]_-{c(h'\Times \epsilon v)}
%%\ar[ddd]
%&%r4c3
%c(\overline{V}X^\fraks\Times \overline{V}X^\fraks)\ar[r]_-{h}
%%\ar[dd]^-{\beta}
%&%r4c4
%VX^\fraks\\\\%r4c5
%c((X^\fraks)^{\sqcup2})\ar[uu]
%}}\]
Let's abbreviate a little for the sake of compactness. Write $X$ for $X^\fraks$, $V$ for $VX^\fraks$, $\overline{V}$ for $\overline{V}X^\fraks$, dots for categorical products, and superscripts for categorical self-products.
%\[\newcommand{\Times}{{\cdot}}
%\mathclap{\xymatrix@R=0.5mm{
%(HX)^{\wedge 2}
%\ar@{-->}@/^1em/[rrrrdd]^-(.9){\xi_\textup{alg}}
%\\\\%r3c5
%\ar@{->>}[uu]
%\ar@{-->}@/^1.3em/[rrrr]^-(.1){\overline{\xi}_\textup{alg}}
%(HX)^2\ar[r]_-{(\textup{id},\textup{id})}
%&%r4c1
%(HX)^2\Times(HX)^2\ar[r]_-{v\Times v\Times \textup{hpf}}
%%\ar[ddd]
%&%r4c2
%(HV)^2\Times HX
%\ar[r]_-{\textup{hpf}\Times v}
%%\ar[ddd]
%&%r4c3
%HV\Times HV\ar[r]_-{\textup{hpf}}
%%\ar[dd]^-{\beta}
%&%r4c4
%HV\\\\%r4c5
%HX\sqcup HX\ar@{ >->}[uu]
%\\\\
%&&\raisebox{.5em}[0mm][0mm]{\makebox[0cm][c]{\,\textup{is $\pi(Q\DASH)$ applied to}}}
%\\
%\textup{cof}
%\ar@{-->}@/^1em/[rrrrdd]^-(.9){\xi_\textup{res}}
%\\\\%r3c5
%\ar[uu]
%\ar@{-->}@/^1.3em/[rrrr]^-(.1){\overline{\xi}_\textup{res}}
%c(X^2)\ar[r]_-{c(\epsilon,\textup{id})\circ\beta}
%&%r4c1
%c(X^2\Times c(X^2))\ar[r]_-{c(\overline{v}^2\Times c(\textup{add}\circ\epsilon^2))}
%%\ar[ddd]
%&%r4c2
%c(\overline{V}^2\Times X)
%\ar[r]_-{c(\textup{add}\Times \overline{v})}
%%\ar[ddd]
%&%r4c3
%c(\overline{V}\Times \overline{V})\ar[r]_-{c(\textup{add})}
%%\ar[dd]^-{\beta}
%&%r4c4
%V\\\\%r4c5
%c(X\sqcup X)\ar[uu]
%}}\]
%
\[\newcommand{\Times}{{\cdot}}
\mathclap{\xymatrix@R=4mm{
(HX)^{\wedge 2}
\ar@{-->}@/^1em/[rrrrd]^-(.9){\xi_\textup{alg}}
\\%r3c5
\ar@{->>}[u]
\ar@/^1.3em/[rrrr]^-(.1){\overline{\xi}_\textup{alg}}
(HX)^2\ar[r]_-{(\textup{id},\textup{id})}
&%r4c1
(HX)^2\Times(HX)^2\ar[r]_-{v\Times v\Times \textup{hpf}}
%\ar[ddd]
&%r4c2
(HV)^2\Times HX
\ar[r]_-{\textup{hpf}\Times v}
%\ar[ddd]
&%r4c3
HV\Times HV\ar[r]_-{\textup{hpf}}
%\ar[dd]^-{\beta}
&%r4c4
HV\\%r4c5
HX\sqcup HX\ar@{ >->}[u]
\\
\textup{cof}
\ar@{-->}@/^1em/[rrrrd]^-(.9){\xi_\textup{res}}
&&\raisebox{1.5em}[0mm][0mm]{\makebox[0cm][c]{\,\textup{is $\pi(Q\DASH)$ applied to\qquad }}}
\\%r3c5
\ar[u]
\ar@/^1.3em/[rrrr]^-(.1){\overline{\xi}_\textup{res}}
c(X^2)\ar[r]_-(.7){c(\epsilon,\textup{id})\circ\beta}
\ar@{..>}[dr]_-(.8){(\epsilon,\textup{id})}
&%r4c1
c(X^2\Times c(X^2))\ar[r]_-{c(\overline{v}^2\Times c(\textup{add}\circ\epsilon^2))}
\ar@{..>}[d]^-{\epsilon}
%\ar[ddd]
&%r4c2
c(\overline{V}^2\Times X)
\ar[r]_-{c(\textup{add}\Times \overline{v})}
\ar@{..>}[d]^-{\epsilon}
%\ar[ddd]
&%r4c3
c(\overline{V}\Times \overline{V})\ar[r]_-{c(\textup{add})}
\ar@{..>}[d]^-{\epsilon}
%\ar[dd]^-{\beta}
&%r4c4
V\ar@{..>}[d]^-{\epsilon}
\\%r4c5
c(X\sqcup X)\ar[u]
&X^2\Times c(X^2)
\ar@{..>}[r]^-{\overline{v}^2\Times c(\textup{add}\circ\epsilon^2)}
&
\overline{V}^2\Times X
\ar@{..>}[r]^-{\textup{add}\Times \overline{v}}
&
\overline{V}\Times \overline{V}\ar@{..>}[r]^-{\textup{add}}
&
\overline{V}
}}\]
This map $\overline{\xi}_\textup{res}$ is very rich, but it will be important to note that postcomposition with $\epsilon$ destroys much of that richness. To understand that, the commuting diagram written with dotted maps below the definition of $\overline{\xi}_\textup{res}$ shows that $\epsilon\circ \overline{\xi}_\textup{res}$
% Indeed, there is a commuting diagram:
%\[\newcommand{\Times}{{\cdot}}
%\def\labelstyle{\scriptstyle}
%\mathclap{\xymatrix@R=4mm@C=10mm{
%\ar@{-->}@/^1.3em/[rrrr]^-(.1){\overline{\xi}_\textup{res}}
%c(X^2)\ar[r]%_-{Q(\cdots)}
%%\ar[dr]_-{Q(\epsilon\pi_1)\Times Q(\epsilon\pi_2)\Times \textup{id\ \ \ }}
%%\ar[d]^-{(Qc\pi_1,Qc\pi_2,\textup{id})}
%\ar@/^1em/@<-.2ex>[dr]_-(.68){(\epsilon,\textup{id})\quad }
%&%r4c1
%c(X^2\Times cX^2)\ar[r]_-{c(\overline{v}^2\Times c(\textup{add}\circ\epsilon^2))}
%\ar[d]^-{\epsilon}
%&%r4c2
%c(\overline{V}^2\Times X)
%\ar[r]_-{c(\textup{add}\Times\overline{v})}
%\ar[d]^-{\epsilon}
%&%r4c3
%c(\overline{V}\Times \overline{V})\ar[r]_-{c(\textup{add})}
%\ar[d]^-{\epsilon}
%&%r4c4
%V\ar[d]^-{\epsilon}\\
%%(cX)^2\Times c(X^2)
%%\ar[r]_-{(\epsilon)^2\Times\textup{id}}
%&
%X^2\Times cX^2
%\ar[r]^-{\overline{v}^2\Times c(\textbf{+}\circ\epsilon^2)}
%%\ar[dr]_-{(Q(\epsilon\circ d^0))^2\Times Q(d^0\circ h)}
%%%%%%%%%%%%%%%%%%%%%\ar@/_1em/[dr]_-(.3){(Q(\epsilon\circ d^0))^2\Times Q(d^0\circ h)\ \ \ }
%&
%\overline{V}^2\Times X
%\ar[r]^-{\textup{add}\Times \overline{v}}
%%%%%%%%%%%%%%%%%%%%%\ar[d]_-{(Q\epsilon)^2\Times Qd^0}
%&
%\overline{V}\Times \overline{V}\ar[r]^-{\textup{add}}
%%\ar@{<-}@/^1em/[dl]^-(.3){\textup{add}\times Q\epsilon}
%&
%\overline{V}
%%\\
%%&
%%&
%%(QK_+)^2\Times QA_+%\ar[ur]_-{\textup{add}\times Q\epsilon}
%%\ar[ur]^-{\textup{add}\Times Q\epsilon}\ar[r]^-{(\textup{id})^2\cdot Q\epsilon}
%%&
%%(QK_+)^3
%%\ar@/_.5em/[ur]_-{\textup{add}}
%}}\]
%
%
%
%
%\[\newcommand{\Times}{{\cdot}}
%\def\labelstyle{\scriptstyle}
%\mathclap{\xymatrix@R=4mm@C=10mm{
%\ar@{-->}@/^1.3em/[rrrr]^-(.1){Q\overline{\xi}_\textup{res}}
%Qc(X^2)\ar[r]%_-{Q(\cdots)}
%%\ar[dr]_-{Q(\epsilon\pi_1)\Times Q(\epsilon\pi_2)\Times \textup{id\ \ \ }}
%%\ar[d]^-{(Qc\pi_1,Qc\pi_2,\textup{id})}
%\ar@/^1em/@<-.2ex>[dr]_-(.68){(Q(\pi_1\circ\epsilon),Q(\pi_2\circ\epsilon),\textup{id})\quad }
%&%r4c1
%Qc(X^2\Times cX^2)\ar[r]_-{Qc(\overline{v}^2\Times c(\textup{add}\circ\epsilon^2))}
%\ar[d]^-{Q\epsilon}
%&%r4c2
%Qc(\overline{V}^2\Times X)
%\ar[r]_-{Qc(\textup{add}\Times\overline{v})}
%\ar[d]^-{Q\epsilon}
%&%r4c3
%Qc(\overline{V}\Times \overline{V})\ar[r]_-{Qc(\textup{add})}
%\ar[d]^-{Q\epsilon}
%&%r4c4
%QV\ar[d]^-{Q\epsilon}\\
%%(QcX)^2\Times Qc(X^2)
%%\ar[r]_-{(Q\epsilon)^2\Times\textup{id}}
%&
%(QX)^2\Times QcX^2
%\ar[r]^-{(Q\overline{v})^2\Times Qc(\textbf{+}\circ\epsilon^2)}
%%\ar[dr]_-{(Q(\epsilon\circ d^0))^2\Times Q(d^0\circ h)}
%%%%%%%%%%%%%%%%%%%%%\ar@/_1em/[dr]_-(.3){(Q(\epsilon\circ d^0))^2\Times Q(d^0\circ h)\ \ \ }
%&
%(Q\overline{V})^2\Times QX
%\ar[r]^-{\textup{add}\Times Q\overline{v}}
%%%%%%%%%%%%%%%%%%%%%\ar[d]_-{(Q\epsilon)^2\Times Qd^0}
%&
%Q\overline{V}\Times Q\overline{V}\ar[r]^-{\textup{add}}
%%\ar@{<-}@/^1em/[dl]^-(.3){\textup{add}\times Q\epsilon}
%&
%Q\overline{V}
%%\\
%%&
%%&
%%(QK_+)^2\Times QA_+%\ar[ur]_-{\textup{add}\times Q\epsilon}
%%\ar[ur]^-{\textup{add}\Times Q\epsilon}\ar[r]^-{(\textup{id})^2\cdot Q\epsilon}
%%&
%%(QK_+)^3
%%\ar@/_.5em/[ur]_-{\textup{add}}
%}}\]
is the sum of the three composite maps $c(X^2)\to \overline{V}$, being $\overline{v}\circ c(\textup{add})\circ c(\epsilon^2)$ and $\overline{v}\circ\pi_i\circ\epsilon$ for $i=1$ and $2$.
%\[\xymatrix@R=4mm{
%&%r1c1
%QX^2\ar@/^.55em/[dr]^-{Q\pi_1}
%&%r1c2
%&%r1c3
%&%r1c4
%\\%r1c5
%Qc(X^2)\ar[r]^-{Qc(\epsilon^2)}
%\ar@/^.5em/[ur]^-{Q\epsilon}
%\ar@/_.5em/[dr]_-{Q\epsilon}
%&%r2c1
%Qc(\overline{X}^2)\ar[r]^-{Qc(\textup{add})}
%&%r2c2
%QA\ar[r]^-{Q\overline{v}}
%&%r2c3
%QK_+\makebox[0cm][l]{$\,=K_+$}%\ar@{=}[r]
%\\%r2c5
%&%r3c1
%QX^2\ar@/_.55em/[ur]_-{Q\pi_2}&%r3c2
%&%r3c3
%}\]
%
%Here dots and unadorned superscripts were used to indicate products. The cofibre `$\textup{cof}$' is that of the map $A^{\sqcup2}\to A^2$. %If we just write $k$ for the zero-square algebra $(K\overline{Q})^{n+1}A$, then the map $h:c(A^2)\to A$ is defined to be the composite:
%%\[c(A^{2})=c((ck)^{2})\overset{c(\epsilon^{2})}{\to}c(k^2)\overset{c(\textup{add})}{\to}ck=A,\]
%%the point being that this is an algebra map, where $\textup{add}:A^2\to A$ is not.
%The maps $h$ and $h'$ are the composites [\textbf{notation clash with Hurewicz}]:
%\[h:\left[c((X^\fraks)^2)\overset{c(\epsilon^2)}{\to}c((\overline{V}X^{\fraks-1})^2)\overset{c(\textup{add})}{\to}c(\overline{V}X^{\fraks-1})=X^\fraks\right]\]
%\[h':\left[(VX^\fraks)^2\overset{\epsilon^2}{\to}(\overline{V}X^\fraks)^2\overset{\textup{add}}{\to}\overline{V}X^\fraks\right]\]
\begin{lem}
The map $\overline{\xi}_\textup{res}$ factors through the cofibre as shown.
\end{lem}
\begin{proof}
It's enough to check that the composite $c(X\sqcup X)\to  V$ is null. However, a map into the codomain is null iff it is zero on homology, as the codomain is a GEM. Fortunately, we have already described the relevant maps on homology algebraically, the point being that their composite is zero.
\end{proof}


\subsection{A chain level construction of $j$}
Write $S^n$ for (the free construction on) $\Delta^n/\partial\Delta^n$, and $CS^n$ for (the free construction on) the reduced cone thereupon. Now there's a map $S^{n+m}\to S^n\sqcup S^m$ sending the fundamental class to the Eilenberg-Zilber formula. That is, if $\mu:(S^n\sqcup S^m)^{\otimes 2}\to S^n\sqcup S^m$ is multiplication, the map we are describing sends $\imath_{n+m}\mapsto \mu(\textup{EZ}(\imath_n,\imath_m))$. Now consider the complex $C_{n,m}$ formed by coning off the corresponding homotopy class:
\[\xymatrix@R=4mm{
S^{n+m}\ar[r]
\ar@{ >->}[d]
&%r1c1
S^n\sqcup S^m\ar@{ >->}[d]
\\%r1c2
CS^{n+m}\ar[r]
&%r2c1
C{n,m}%r2c2
}\]
Now $C_{n,m}$ has an $n+m+1$-dimensional generator $g$, the image of the cone on the fundamental class of $S^{n+m}$. The homology LES shows that $H_*(C_{n,m})$ is three-dimensional, containing classes $\imath_n$, $\imath_m$ and $g$.
\begin{prop}
The diagonal of $g$ is $\imath_n\otimes\imath_m+\imath_m\otimes\imath_n$.
\end{prop}
\begin{proof}
The representative $g$ has the property that $d_0(g)=\mu(\textup{EZ}(\imath_n,\imath_m))$ and $d_i(g)=0$ for $i>0$. We calculate 
\begin{alignat*}{2}
\psi(d_0(g))
&=
\mu(\textup{EZ}(\overline{\imath_n}+\overline{\overline{\imath_n}},\overline{\imath_m}+\overline{\overline{\imath_m}}))%
\\
\xi(g)
&=
\mu(\textup{EZ}(\overline{\imath_n},\overline{\overline{\imath_m}})+\textup{EZ}(\overline{\overline{\imath_n}},\overline{\imath_m}))\in Q((C_{n,m})_{n+m}^{\wedge 2})%
\\
% Left hand side
j(\xi(g))
% Relation
&=
% Right hand side
\textup{EZ}^\otimes ({\imath_n},{{\imath_m}})+\textup{EZ}^\otimes ({\imath_m},{{\imath_n}})\in (Q((C_{n,m})_{n+m}))^{\otimes 2}%
% Comment
\end{alignat*}
This class represents $\imath_n\otimes\imath_m+\imath_m\otimes\imath_n$.
\end{proof}
%Now suppose that $L$ is a (not necessarily zero-square) algebra, and that $\alpha,\beta$ are homotopy classes in $\pi_n (L)$ and $\pi_m (L)$ respectively. Then we'll be able to use the complex $C_{n,m}$ to construct a homology class $g_{\alpha,\beta}\in H_{n+m+1}(L)$ whose diagonal is $\alpha\otimes\beta+\beta\otimes\alpha$ [probably just when $L$ is zero square, using the map $L\times L\to L$. Else, get a class in $H(L^2)$ with diagonal $\overline{\alpha}\otimes\overline{\overline{\beta}}+ \overline{\overline{\beta}}\otimes\overline{\alpha}$]. Firstly, note that there is a commuting diagram
%\[\xymatrix@R=4mm{
%0\ar[r]
%\ar@{ >->}[d]
%&%r1c1
%0\ar[d]
%\\%r1c2
%S^n\sqcup S^m\ar[r]^-{\alpha\sqcup\beta}
%&%r2c1
%L^{\times2}%r2c2
%}\]
%which furnishes $c_1(L^{\times2})$ with a map from $S^n\sqcup S^m$. Next, there's a diagram %\newdir{ >}{{}*!/-7pt/\dir{>}}
%\[\xymatrix@R=4mm{
%S^{n+m}\ar@{->}[r]_-{\textup{EZ}^\mu(\imath_n,\imath_m)}
%\ar@{ >->}[d]
%&%r1c1
%S^n\sqcup S^m\ar@{ >->}[d]
%\ar[r]
%&c_1(L^{\times2})\ar[d]
%\\%r1c2
%CS^{n+m}\ar[r]\ar@/_1em/[rr]_-{0}
%&%r2c1
%C{n,m}\ar@{..>}[r]
%%r2c2
%&L^{\times2}}\]
%where the outer rectangle commutes since in $L^{\times2}$, $(a,0)\cdot(0,b)=(0,0)$. This gives a map $C_{n,m}\to c_2(L^{\times2})$ whose restriction to the subcomplex $S^n\sqcup S^m$ gives the homotopy classes $\alpha$ and $\beta$. The image $g_{\alpha,\beta}$ of the class $g\in(C_{n,m})_{n+m+1}$ is then an explicit representative for the homology class sought.
We can use this cofibration to build a chain level construction of the map $j:\textup{Pr}_t(HY)\otimes \textup{Pr}_{t'}(HZ)\to \textup{Pr}_{t+t'+1}(HY\wedge HZ)$ on spherical homology classes:
\begin{prop}
There is a function, natural in $X,Y\in s\Comm$
\[\overline{F}:\hom_{s\Comm}(S^t,Y)\times\hom_{s\Comm}(S^{t'},Z)\to \hom_{s\Comm}(C_{t,t'},c(Y\times Z))\]
such that the function
\[F:\hom_{s\Comm}(S^t,Y)\times\hom_{s\Comm}(S^{t'},Z)\to \pi_{\frakt+1}(Qc(Y\times Z)):=H_{\frakt+1}(Y\times Z)\]
defined by $F(\alpha,\beta):=H_*(\overline{F}(\alpha,\beta))(g)$ makes the following diagram commute:
\[\mathclap{\xymatrix@R=4mm{
%(H(Y\times Z)^{\otimes2})_\frakt\ar[d]^-{\textup{id}+\tau}
%&
\ar@/_2em/@<-3ex>[dd]_-{(\textup{id}+\tau)\circ \textup{hur}^{\otimes2}}
%\ar[l]_-{\textup{hur}^{\otimes2}}
\hom_{s\Comm}(S^t,Y)\times\hom_{s\Comm}(S^{t'},Z)\ar[d]^-{F}
\ar[r]^-{\textup{hur}^{\otimes2}}
&%r1c1
\textup{Pr}_t(HY)\otimes\textup{Pr}_{t'}(HZ)\ar[r]^-{j}
&%r1c2
\textup{Pr}_{\frakt+1}(HY\wedge HZ)\ar@{ >->}[d]
\\%r1c5
%(H(Y\times Z)^{\otimes2})_\frakt
%&
H_{\frakt+1}(Y\times Z)\ar[r]%\ar[l]_-{\Delta}
\ar[d]^-{\Delta}
&%r2c1
(HY\times HZ)_{\frakt+1}\ar@{->>}[r]
\ar[d]^-{\Delta}
&%r2c2
(HY\wedge  HZ)_{\frakt+1}
\\
%&
(H(Y\times Z)^{\otimes2})_\frakt\ar[r]
&((HY\times HZ)^{\otimes2})_\frakt
}}\]
\end{prop}
\begin{proof}
The value of $\overline{F}$ on $(\alpha,\beta)$ is defined as follows. Corresponding to the commuting diagrams
\[\xymatrix@R=4mm{
0\ar[r]
\ar@{ >->}[d]
&%r1c1
0\ar[d]
\\%r1c2
S^t\ar[r]^-{(\alpha,0)}
&%r2c1
Y\times Z%r2c2
}
\textup{\quad and\quad }
\xymatrix@R=4mm{
0\ar[r]
\ar@{ >->}[d]
&%r1c1
0\ar[d]
\\%r1c2
S^{t'}\ar[r]^-{(0,\beta)}
&%r2c1
Y\times Z%r2c2
}\]
are maps $\widetilde{(\alpha,0)}:S^t\to c_1(Y\times Z)$ and $\widetilde{(0,\beta)}:S^{t'}\to c_1(Y\times Z)$, and there is a commuting diagram (since in $Y\times Z$, products of the form $(a,0)(0,b)$ vanish):
\[\xymatrix@R=4mm@C=15mm{
S^{\frakt}\ar@{->}[r]_-{\textup{EZ}^\mu(\imath_t,\imath_{t'})}
\ar@{ >->}[d]
&%r1c1
S^t\sqcup S^{t'}\ar@{ >->}[d]
\ar[r]^{\widetilde{(\alpha,0)}\sqcup\widetilde{(0,\beta)}}
&c_1(Y\times Z)\ar[d]
\\%r1c2
CS^{\frakt}\ar[r]\ar@/_1em/[rr]_-{0}
&%r2c1
C{t,t'}\ar@{..>}[r]
%r2c2
&Y\times Z}\]
Corresponding to the right square is a map $C_{t,t'}\to c_2(Y\times Z)$, and the composite with the cofibration $c_2(Y\times Z)\to c(Y\times Z)$ is $\overline{F}(\alpha,\beta)$. This function $\overline{F}$ is evidently natural in $Y$ and $Z$, and so then is $F$.

The square at the bottom of the diagram commutes as the horizontals are maps of homology Lie coalgebras, and we can see that the triangle commutes because we understand the Lie coalgebra structure of $H_*(C_{t,t'})$. As all of the maps in the six-arrow commuting diagram  are natural, we may check it on the universal example: $(\imath_t,\imath_{t'})\in\hom_{s\Comm}(S^t,S^t)\times\hom_{s\Comm}(S^{t'},S^{t'})$. That is, it's enough to check that the following diagram commutes:
\[\mathclap{\xymatrix@R=4mm{
\{(\imath_t,\imath_{t'})\}\ar[d]^-{F}
\ar[r]^-{\textup{hur}^{\otimes2}}
&%r1c1
\textup{Pr}_t(HS^t)\otimes\textup{Pr}_{t'}(HS^{t'})\ar[r]^-{j}
&%r1c2
\textup{Pr}_{\frakt+1}(HS^t\wedge HS^{t'})\ar@{ >->}[d]^-{\textup{inc}}
\\%r1c5
H_{\frakt+1}(S^t\times S^{t'})\ar[r]^-{r}
&%r2c1
(HS^t\times HS^{t'})_{\frakt+1}\ar@{->>}[r]^-{\textup{proj}}
&%r2c2
(HS^t\wedge  HS^{t'})_{\frakt+1}
}}\]
The rightmost four vector spaces in this diagram are 1-dimensional, and the maps $j$, $\textup{inc}$ and $\textup{proj}$ connecting them are isomorphisms, so it's enough to check that $r(F(\imath_t,\imath_{t'}))$ is nonzero. However, using the commuting square and triangle already established, we may calculate that $\Delta(r(F(\imath_t,\imath_{t'})))=\imath_t\otimes\imath_{t'}+\imath_{t'}\otimes\imath_{t}$, which is nonzero.
\end{proof}
We record here a useful calculation:
\begin{lem}
The composite 
\[S^t\sqcup S^{t'}\to C_{t,t'}\overset{\overline{F}(\alpha,\beta)}{\to}c(X\times X)\overset{c(\textup{add}\circ(\epsilon^2))}{\to}X\]
equals $\widetilde{(\epsilon\alpha)}\sqcup \widetilde{(\epsilon\beta)}$.
\end{lem}
\begin{proof}Indeed, the following commutes:
%Using naturality of $\overline{F}$ and the fact that $h=c(\textup{add}\circ(\epsilon\times\epsilon))$:
\[\newcommand{\Times}{{\cdot}}
\xymatrix@R=4mm{
S^t\sqcup S^{t'}\ar@{ >->}[d]
%\ar[rrd]_-{\widetilde{(\epsilon\alpha,0)}\sqcup \widetilde{(0,\epsilon\beta)}}
\ar@/^1em/[rrd]_-{\widetilde{(\epsilon\alpha,0)}\sqcup \widetilde{(0,\epsilon\beta)}}
\ar@/^1em/[rrrd]^-(.7){\widetilde{(\epsilon\alpha)} \sqcup \widetilde{(\epsilon\beta)}}
&%r1c1
&%r1c2
&%r1c3
\\%r1c4
C_{t,t'}\ar[r]^-{\overline{F}(\alpha,\beta)}
\ar@/_1em/[rr]_-{\overline{F}(\epsilon\alpha,\epsilon\beta)}
&%r2c1
c(X^\fraks\Times X^\fraks)\ar[r]^-{c(\epsilon\times\epsilon)}
&%r2c2
c(\overline{V}X^{\fraks-1}\Times \overline{V}X^{\fraks-1})\ar[r]_-{c(\textup{add})}
&%r2c3
c(\overline{V}X^{\fraks-1})\makebox[0cm][l]{\,$=X^\fraks$}\phantom{=}%r2c4
}\]
%Starting with the straight arrows, we need to justify the rightmost curved arrow. 
The naturality of $\overline{F}$ allows us to see the bottom curved arrow, then the definition of $\overline{F}$ gives the next curved arrow, and naturality of the standard lift gives the third curved arrow.
\end{proof}
\subsection{A left inverse for the Hurewicz}
Suppose that $Y$ is an abelian (i.e.\ zero-square)  simplicial algebra. Then $QY=Y$, giving a commuting diagram
\[\xymatrix@R=4mm{
Y\ar@{=}[r]
&%r1c1
QY&%r1c2
&%r1c3
\pi_*(Y)\ar@{=}[r]\ar[ddr]_-(.75){\ \ \textup{hur}}
&%r1c1
\pi_*(QY)\\%r1c5
cY\ar[u]^-{\epsilon}
\ar[r]
&%r2c1
QcY\ar[u]^-{Q\epsilon}
&%r2c2
\textup{yielding}&%r2c3
\pi_*(cY)\ar[u]^-{\pi_*(\epsilon)}_-{\cong}
\ar[r]
\ar[d]^-{\cong}
&%r2c1
\pi_*(QcY)\ar[u]_-{\pi_*(Q\epsilon)}\ar@{=}[d]
\\%r2c5
&%r3c1
&%r3c2
&%r3c3
\textup{Pr}(H_*Y)\ar@{ >->}[r]
&%r3c4
H_*Y%r3c5
}\]
which is to say that $\pi_*(Q\epsilon)$ provides a left inverse for the Hurewicz.

\subsection{$E^1$-level identification of the product}
Throughout, I'll continue with the above abbreviations, and write $\frakt=t+t'$. We're going to need the following commutative diagram of vector spaces:
\[\mathclap{\xymatrix@R=4mm{
\pi_t(X)\otimes \pi_{t'}(X)\ar[dd]^-{h\otimes h}
\ar[r]^-{\textup{sEZ}}
&%r1c1
\pi_{\frakt}(X\wedge  X)\ar[r]
&%r1c2
\pi_{\frakt}(\overline{D}^1X)&%r1c3
\pi_{\frakt+1}(\overline{V}X)
\ar[l]^-{\cong\textup{-ish}}_-{\partial_\textup{conn}}
\\%r1c4
&%r2c1
&%r2c2
\pi_{\frakt}(D^1X)\ar@{-}[u]^-{\textup{zig-zag}}_-{\cong}
&%r2c3
\pi_{\frakt+1}(VX)\ar[u]_-{\cong}
\ar[l]^-{\cong\textup{-ish}}_-{\partial_\textup{conn}}
\ar[d]^-{\cong}
\\%r2c4
\textup{Pr}_t(HX)\otimes \textup{Pr}_{t'}(HX)\ar[r]^-{j}
&%r3c1
\textup{Pr}_{\frakt+1}(HX\wedge HX)\ar[rr]^-{\xi_\textup{alg}}
&%r3c2
&%r3c3
\textup{Pr}_{\frakt+1}(HVX)%r3c4
}}\]
It helps to modify this diagram a little. Indeed, for each $(\alpha,\beta)\in \hom_{s\Comm}(S^t,X)\times\hom_{s\Comm}(S^{t'},X)$, there is a diagram starting with the singleton set $\{(\alpha,\beta)\}$:% such that $\alpha=\widetilde{\epsilon\alpha}$ and $\alpha=\widetilde{\epsilon\alpha}$
%\[\xymatrix@R=4mm{
%\pi_t(X^\fraks)\otimes \pi_{t'}(X^\fraks)\ar@{~>}[drd]^-{j\circ(h\otimes h)}
%\ar[r]^-{\textup{sEZ}}
%\ar@{..>}[ddddr]_-(.3){A}
%&%r1c1
%\pi_{\frakt}(X^\fraks\otimes X^\fraks)\ar[r]
%&%r1c2
%\pi_{\frakt}((\overline{D}^1X)^\fraks)&%r1c3
%\pi_{\frakt+1}(X_\textup{fl}^{\fraks+1})\ar[l]^-{\cong}_-{\partial}
%\ar@{~}[dd]^-{\textup{zig-zag}}_-{\cong}
%\\%r1c4
%&%r2c1
%&%r2c2
%&%r2c3
%%\pi_{\frakt+1}(X^{\fraks+1})\ar[u]_-{\cong}
%%\ar[d]^-{\cong}
%\\%r2c4
%%\textup{Pr}_t(HX^{\fraks})\otimes \textup{Pr}_{t'}(HX^{\fraks})\ar[r]^-{j}
%&%r3c1
%\textup{Pr}_{\frakt+1}(HX^\fraks\wedge HX^\fraks)
%\ar@{~>}[rr]_-(.82){\xi_\textup{alg}}
%\ar@{ >->}[dl]
%\ar@{ >->}[d]
%&%r3c2
%&%r3c3
%\textup{Pr}_{\frakt+1}(HX^{\fraks+1})\ar@{ >->}[d]
%\\%r3c4
%(HX^\fraks\wedge HX^\fraks)_{\frakt+1}\ar@{=}[r]
%&\pi_{\frakt+1}Q(\textup{cof})
%\ar[rr]^-(.8){(\xi_\textup{res})_*}
%&
%&
%H_{\frakt+1}X^{\fraks+1}
%\\
%(HX^\fraks\times HX^\fraks)_{\frakt+1}\ar@{=}[r]\ar@{->>}[u]
%&\pi_{\frakt+1}Qc(X^\fraks\times X^\fraks)\ar@{->>}[u]
%\ar[urr]_-(.76){(\overline{\xi}_\textup{res})_*}
%\ar@{..>}[uuuurr]^-(.75){B}
%\\
%(HX^\fraks\oplus HX^\fraks)_{\frakt+1}\ar@{=}[r]\ar@{ >->}[u]
%&\pi_{\frakt+1}Q(X^\fraks\sqcup X^\fraks)\ar@{ >->}[u]
%}\]
%The method will be to introduce the dotted maps $A$ and $B$ and show that the composite $B\circ A$ equals the composite of the wavy maps. Then we will be able to calculate the composite $\partial\circ B\circ A$, and compare it with the composite along the top row.
\[\mathclap{\xymatrix@R=4mm{
\{(\alpha,\beta)\}
%\pi_t(X)\otimes \pi_{t'}(X)
\ar@{~>}[dd]_-{j\circ(h\otimes h)}
\ar[r]^-{\textup{sEZ}}
\ar@{..>}[ddddr]^-(.2){F}
&%r1c1
\pi_{\frakt}(X\wedge  X)\ar[r]^-{\overline{\mu}_*}
&%r1c2
\pi_{\frakt}(\overline{D}^1X)&%r1c3
\pi_{\frakt+1}(\overline{V}X)\ar[l]_-{\partial_\textup{conn}}
\ar@{~}[dd]^-{\textup{zig-zag}}_-{\cong}
\\%r1c4
&%r2c1
&%r2c2
\pi_{\frakt+1}(Q\overline{V}X)\ar@{..>}[ur]_-{=}&%r2c3
%\pi_{\frakt+1}(X^{\fraks+1})\ar[u]_-{\cong}
%\ar[d]^-{\cong}
\\%r2c4
%\textup{Pr}_t(HX^{\fraks})\otimes \textup{Pr}_{t'}(HX^{\fraks})\ar[r]^-{j}
%r3c1
\textup{Pr}_{\frakt+1}(HX\wedge HX)
\ar@{~>}[rrr]^-{\xi_\textup{alg}}
\ar@{ >->}[d]
%\ar@{ >->}[dr]
&
&%r3c2
&%r3c3
\textup{Pr}_{\frakt+1}(HVX)\ar@{ >->}[d]
\\%r3c4
(HX\wedge HX)_{\frakt+1}\ar@{=}[r]
&\pi_{\frakt+1}Q(\textup{cof})
\ar[r]^-{\pi_*(Q\xi_\textup{res})}
&
\pi_{\frakt+1}QVX\ar@{=}[r]
\ar@{..>}[uu]_-(.75){\pi_*(Q\epsilon)}
&
H_{\frakt+1}VX
\\
(HX\times HX)_{\frakt+1}\ar@{=}[r]\ar@{->>}[u]
&\pi_{\frakt+1}Qc(X\times X)\ar@{->>}[u]
\ar@{..>}[ur]_-(.6){\pi_*(Q\overline{\xi}_\textup{res})}
}}\]
Although all of the arrows in this modified diagram have already been defined, we've decorated some of them for emphasis. It will be enough to check that all of these diagrams commute, since the collection of such $(\alpha,\beta)$ will exhaust all of the pure tensors in $\pi_t(X)\otimes \pi_{t'}(X)$.

The composite of the dotted maps equals the composite of the wavy maps, by results above. Thus the image of $(\alpha,\beta)$ under either the wavy or dotted maps equals the image of $g\in\pi (QC_{t,t'})$ under the composite
\[QC_{t,t'}\overset{Q\overline{F}(\alpha,\beta)}{\to}Qc(X\times X)\overset{Q\overline{\xi}_\textup{res}}{\to}QVX\overset{Q\epsilon}{\to}Q\overline{V}X=\overline{V}X,\]
which we've seen decomposes as the sum of the three maps $\overline{v}\circ c(\textup{add})\circ c(\epsilon^2)\circ\overline{F}(\alpha,\beta)$ and $\overline{v}\circ\pi_i\circ\epsilon\circ\overline{F}(\alpha,\beta)$ for $i=1$ and $2$.
%Indeed, write `$\textup{dot}$' for the composite of the dotted maps, and write $g\in QC_{t,t'}$ for the a representative of $g\in\pi_{\frakt+1}(QC_{t,t'})$. A representative for $\textup{dot}(\alpha,\beta)$ is given by the image of $g$ under the composite
%\[QC_{t,t'}\overset{Q\overline{F}(\alpha,\beta)}{\to}Qc(X^\fraks\times X^\fraks)\overset{Q\overline{\xi}_\textup{res}}{\to}QX^{\fraks+1}\overset{Q\epsilon}{\to}QX^{\fraks+1}_\textup{fl}=X^{\fraks+1}_\textup{fl},\]
%which is the sum of the images of $g$ under the three composites
%\[\xymatrix@R=4mm{
%&
%&%r1c1
%QcA\ar@/^.55em/[dr]^-{Q\epsilon}
%&%r1c2
%&%r1c3
%&%r1c4
%\\%r1c5
%QC_{t,t'}
%\ar[r]^-{Q\overline{F}(\alpha,\beta)}
%&Qc(A^2)\ar[rr]^-{Qh}
%\ar@/^.5em/[ur]^-{Qc(\pi_1)}
%\ar@/_.5em/[dr]_-{Qc(\pi_2)}
%&%r2c1
%&%r2c2
%QA\ar[r]^-{Q\eta}
%&%r2c3
%QK_+\makebox[0cm][l]{$\,=K_+$}%\ar@{=}[r]
%\\%r2c5
%&
%&%r3c1
%QcA\ar@/_.55em/[ur]_-{Q\epsilon}&%r3c2
%&%r3c3
%}\]
The composite $\pi_1\circ \epsilon\circ\overline{F}(\alpha,\beta):C_{t,t'}\to X$ is the map out of the pushout in the diagram:
\[\xymatrix@R=4mm@C=15mm{
S^{\frakt}\ar@{->}[r]_-{\textup{EZ}^\mu(\imath_t,\imath_{t'})}
\ar@{ >->}[d]
&%r1c1
S^t\sqcup S^{t'}\ar@{ >->}[d]
\ar[dr]^{\alpha\sqcup0}
\\%r1c2
CS^{\frakt}\ar[r]\ar@/_1em/[rr]_-{0}
&%r2c1
C{t,t'}\ar@{..>}[r]
%r2c2
&X}\]
The point then is that $g\in CS^\frakt$ is a (normalized) chain with the property that $d_0(g)$ is the image of the fundamental class in $S^\frakt$. In $C_{t,t'}$, $d_0(g)$ is identified with $EZ^\mu(\imath_t,\imath_{t'})$, which is decomposable, so that $g$ becomes a cycle in $QC_{t,t'}$, representing the $(t+t'+1)$-dimensional homology class. As this $g$ is in the image of the map $CS^\frakt\to C_{t,t'}$, it maps to $0$ in $A$, and on homology, $g$ is annihilated by the composite $\overline{v}\circ\pi_1\circ\epsilon\circ\overline{F}(\alpha,\beta)$. The same holds for $\overline{v}\circ\pi_2\circ\epsilon\circ\overline{F}(\alpha,\beta)$, so we may restrict our attention to the image of $g$ under $\overline{v}\circ c(\textup{add}\circ \epsilon^2)\circ\overline{F}(\alpha,\beta)$, and prove:
\begin{thm}
The modified diagram commutes.
\end{thm}
\begin{proof}
We must show that $\partial_\textup{conn}(\textup{dot}(\alpha,\beta))=\overline{\mu}_*(\textup{sEZ}(\alpha,\beta))$.
All of the previous discussion together has shown $\textup{dot}(\alpha,\beta)$ is the image in $Q\overline{V}X=\overline{V}X$ of the cycle $g\in Qc_{t,t'}$ under the top row of the following commuting diagram:
\[\xymatrix@R=4mm@C=10mm{
\makebox[0cm][r]{$g\in\,$}QC_{t,t'}\ar[r]^-{Q\overline{F}(\alpha,\beta)}
&%r1c2
Qc(X^2)\ar[r]^-{Qc(\textup{add}\circ\epsilon^2)}
&%r1c3
QX\ar[r]^-{Q\overline{v}}
&%r1c4
Q\overline{V}X\\%r1c5
\makebox[0cm][r]{$g\in\,$}
C_{t,t'}\ar[r]^-{\overline{F}(\alpha,\beta)}
\ar@{->>}[u]
\ar[d]_-{d_0}
&%r1c2
c(X^2)\ar[r]^-{c(\textup{add}\circ\epsilon^2)}
\ar@{->>}[u]
\ar[d]_-{d_0}
&%r1c3
X\ar@{->>}[r]^-{\overline{v}}
\ar@{->>}[u]
\ar[d]^-{d_0}
&%r1c4
\overline{V}X\makebox[0cm][l]{$\,\ni\textup{dot}(\alpha,\beta)$}\ar@{=}[u]
\\%r2c5
\makebox[0cm][r]{$\textup{EZ}^\mu(\imath_t,\imath_t')\in\,$}C_{t,t'}\ar[r]^-{\overline{F}(\alpha,\beta)}
&%r1c2
c(X^2)\ar[r]^-{c(\textup{add}\circ\epsilon^2)}
&%r1c3
X\makebox[0cm][l]{$\,\ni\partial_\textup{conn}(\textup{dot}(\alpha,\beta))$}
&%r1c4
\\%r3c5
\makebox[0cm][r]{$\textup{EZ}^\mu(\imath_t,\imath_t')\in\,$}S^t\sqcup S^{t'}\ar[urr]_-{\widetilde{(\epsilon\alpha)} \sqcup \widetilde{(\epsilon\beta)}}
\ar@{ >->}[u]
&%r4c2
&%r4c3
&%r4c4
%r4c5
}\]
Moreover, at the right of this diagram we see exactly the maps needed in order to construct the connecting homomorphism for the SES defining $\overline{D}^1X^\fraks$.
%\[0\to\overline{D}^1X^\fraks\to X^\fraks\overset{\overline{v}}{\to} \overline{V}X^\fraks\to 0,\]
This confirms that $\partial_\textup{conn}(\textup{dot}(\alpha,\beta))$ is the image under $\widetilde{(\epsilon\alpha)} \sqcup \widetilde{(\epsilon\beta)}$ of $\textup{EZ}^\mu(\imath_t,\imath_t')$. As $\alpha\sim \widetilde{(\epsilon\alpha)}$ and $\beta\sim \widetilde{(\epsilon\beta)}$, this completes the proof.
\end{proof}


\end{Adams Muliplicativity}
\begin{Adams sseq operations}
\section{Operations in the Adams spectral sequence}
\subsubsection*{Higher cosimplicial Alexander-Whitney maps}
For cosimplicial vectorspaces $R,S$, let $\{D^k\}$ be a collection of higher cosimplicial Alexander-Whitney maps, i.e.\ maps
\[D^k:(CR\otimes CS)^{i+k}\to N(R\otimes S)^i\textup{\ \ for $i,k\geq0$}\]
such that $dD^k+D^kd=D^{k-1}+TD^{k-1}T$ for $k>1$, and $D^0$ is a chain homotopy equivalence inducing the identity in dimension zero. It is a useful convention to define $D^i=0$ for $i<0$.
\subsubsection*{Higher simplicial Eilenberg-Zilber maps}
For simplicial vectorspaces $U,V$, let $\{\Delta_k\}$ be a collection of higher simplicial Eilenberg-Zilber maps, i.e.\ maps
\[\Delta_k:(CU\otimes CV)_{i+k}\to N(U\otimes V)_i\textup{\ \ for $0\leq k\leq i$}\]
such that
\[\Delta_k+T\Delta_kT=\phi_k+\begin{cases}
\Delta_{k-1}\partial+\partial\Delta_{k-1},&\textup{if }k\geq1;\\
\Delta,&\textup{if }k=0.
%\\,&\textup{if }
\end{cases}
\]
where $\Delta:CU\otimes CV\to C(U\times V)$ is a chain homotopy equivalence inducing the identity in dimension zero, and $\phi_k$ is the map $(CU\otimes CV)_{i+k}\to C(U\otimes V)_i$ which vanishes except on $U_k\otimes V_k$, where its value is just the projection $U_k\otimes V_k\to N(U\times V)_k$. It is not wise to define $\Delta_i$ for $i<0$.



For mixed simplicial vector spaces $X$ and $Y$, write $C(X\otimes_\textup{v}Y)$ for the double complex which in degree $(\fraks,\frakt)$ equals the direct sum $\bigoplus_{t+t'=\frakt}X_t^\fraks\otimes X_{t'}^\fraks$. The following bicomplex maps are given by prolonging $D^k$, $\Delta$, $\Delta_k$ and $\psi_k$:
\begin{align*}
\overline{D}^k:(CX\otimes CY)_{\fraks+k,\frakt}&\to C(X\otimes_\textup{v}Y)_{\fraks,\frakt}\\
\overline{\Delta}:C(X\otimes_\textup{v} Y)_{\fraks,\frakt}&\to C(X\otimes Y)_{\fraks,\frakt}\\
\overline{\Delta}_k:C(X\otimes_\textup{v} Y)_{\fraks,\frakt+k}&\to C(X\otimes Y)_{\fraks,\frakt}\textup{\quad for $0\leq k\leq \frakt$}\\
\overline{\phi}_k:C(X\otimes_\textup{v} Y)_{\fraks,\frakt+k}&\to C(X\otimes Y)_{\fraks,\frakt}\textup{\quad for $0\leq k\leq \frakt$}
\end{align*}
%COMMENTED OUT INTERMEDIATE STEPS
%\[\overline{\Delta}_k:C(X\otimes_\textup{v} Y)_{\fraks,\frakt}\to C(X\otimes Y)_{\fraks,\frakt-k}\textup{\quad for $0\leq k\leq \frakt-k$}\]
%\[\overline{\Delta}_k:C(X\otimes_\textup{v} Y)_{\fraks,n+\fraks}\to C(X\otimes Y)_{\fraks,n+\fraks-k}\textup{\quad for $0\leq k\leq n+\fraks-k$}\]
We may rewrite them with a focus on filtration, along with two further functions in the case $Y=X$:
\begin{align*}
\overline{D}^k:F_{p}(CX\otimes CY)_{n}&\to F_{p-k}C(X\otimes_\textup{v}Y)_{n+k}\\
\overline{\Delta}:F_{p}C(X\otimes_\textup{v} Y)_{n}&\to F_{p}C(X\otimes Y)_{n}\\
\overline{\Delta}_k:F_{p}C(X\otimes_\textup{v} Y)_{n}&\to F_{p}C(X\otimes Y)_{n-k}\textup{\quad \xcancel{for $p\geq 2k-n$},\,$k\geq0$}\\
\overline{\phi}_k:F_{p}C(X\otimes_\textup{v} Y)_{n}&\to F_{p}C(X\otimes Y)_{n-k}\textup{\quad \xcancel{for $p\geq 2k-n$},\,$k\geq0$}\\
\rho:F_{p}C(X\otimes X)_{n}&\to F_{p}C(X\otimes_{\Sigma_2} X)_{n}\\
\tau:F_{p}CX_n&\to F_{2p}(CX\otimes CX)_{2n}
\end{align*}
where $\rho$ the projection onto $\Sigma_2$ coinvariants, a chain map, and $\tau:x\mapsto x\otimes x$ the self-tensor-square, which is of course not even linear. \textbf{Actually}, I think it's better to define these maps everywhere, and just have them zero whenever the non-underlined map is not defined. In fact, in this case we have the equation
\[\textup{put in the equations}\]


Note that $\overline{\psi}_k$ vanishes on $F_{2k-n+1}C(X\otimes_\textup{v} Y)_n$.



\[\mathclap{\xymatrix@R=4mm{
\pi_t(X)\otimes \pi_{t'}(X)\ar[dd]^-{h\otimes h}
\ar[r]^-{\textup{sEZ}}
&%r1c1
\pi_{\frakt}(X\wedge  X)\ar[r]
&%r1c2
\pi_{\frakt}(\overline{D}^1X)&%r1c3
\pi_{\frakt+1}(\overline{V}X)
\ar[l]^-{\cong\textup{-ish}}_-{\partial_\textup{conn}}
\\%r1c4
&%r2c1
&%r2c2
\pi_{\frakt}(D^1X)\ar@{-}[u]^-{\textup{zig-zag}}_-{\cong}
&%r2c3
\pi_{\frakt+1}(VX)\ar[u]_-{\cong}
\ar[l]^-{\cong\textup{-ish}}_-{\partial_\textup{conn}}
\ar[d]^-{\cong}
\\%r2c4
\textup{Pr}_t(HX)\otimes \textup{Pr}_{t'}(HX)\ar[r]^-{j}
&%r3c1
\textup{Pr}_{\frakt+1}(HX\wedge HX)\ar[rr]^-{\xi_\textup{alg}}
&%r3c2
&%r3c3
\textup{Pr}_{\frakt+1}(HVX)%r3c4
}}\]

\[\mathclap{\xymatrix@R=4mm{
N^s(\pi_t(X)\otimes \pi_{t'}(X))
\ar[r]^-{\textup{sEZ}}
&%r1c1
N^s\pi_{\frakt}(X\wedge  X)\ar[r]
&%r1c2
N^s\pi_{\frakt}(\overline{D}^1X)&%r1c3
N^s\pi_{\frakt+1}(\overline{V}X)
\ar[l]_-{\partial_\textup{conn}}
&
N^{s+1}\pi_{t+1}X
\ar@{ >->}[l]
\ar@/_2em/[ll]_-{\cong}
}}\]

\subsection{Divided squares algebraically defined on $E^2$}
The formula for the $E^2$ page is $E^2_{st}=\pi^s\pi_t(X)$. We'll now define divided square operations $\delta_i^\textup{alg}:E^2_{s,t}\to E^2_{s+1,t+i+1}$ defined when $2\leq i \leq t$. Indeed, we may define these by the formula
\[N^s\pi_t(X)\overset{N^s(\gamma_i)}{\to}N^s\pi_{t+i}(X\wedge_{\Sigma_2} X)\overset{N^s(\overline{\mu})}{\to}N^s\pi_{t+i}(\overline{D}^1X)\cong N^{s+1}\pi_{t+i+1}(X)\]
where $\gamma_i:\pi_t(X^s)\to\pi_{t+i}(X^s\wedge_{\Sigma_2} X^s)$ is induced by the function on cycles $ZN_t(X^s)\to ZN_{t+i}(X^s\wedge_{\Sigma_2} X^s)$ defined by
\[x\mapsto\Delta_{t-i}(x\otimes x).\]
The map $\gamma_i$ is well defined, is linear when $i<t$, and is a quadratic refinement of the product pairing when $i=n$ (c.f.\ \cite[4.1,4.2]{DwyerHtpyOpsSimpComAlg.pdf}).
Now $\gamma_i$ and $\overline{\mu}$ commute with the cosimplicial structure maps, and the final isomorphism is a chain map, so that the whole composite is a chain map, and so defines operations on $E^2$ as hoped, when $i<t$. Note that there's no operation when $i=t$, since the top operation is not linear. Maybe that is to say that it supports a $d_1$.

In order to shoehorn this into a more usable format, choose the $D^k$ to be a \textbf{special} cosimplicial Eilenberg-Zilber map (using the terminology that Turner uses, and that Dwyer  and Singer use), as in \cite[5.2]{turner_opns_and_sseqs_I.pdf}. Then there is a commuting diagram
\[\mathclap{\xymatrix@R=4mm{
(CX)_{\fraks,\frakt}\ar[r]^-{\tau}
\ar[dr]_-{\tau}
&%r1c1
(CX\otimes CX)_{2\fraks,2\frakt}\ar[r]^-{\overline{D}^\fraks}
&%r1c2
C(X\otimes_\textup{v}X)_{s,2t}
\ar[r]^-{\rho\overline{\Delta}_{\frakt-i}}
&
C(X\otimes_{\Sigma_2} X)_{\fraks,\frakt+i}\\%r1c3
&%r2c1
(CX)_{s,t}\otimes (CX)_{s,t}\ar@{ >->}[u]
\ar@{=}[r]
&%r2c2
C(X^s)_{t}\otimes C(X^s)_{t}\ar@{ >->}[u]\ar[r]^-{\rho\Delta_{t-i}}
&%r2c3
C(X^\fraks\otimes_{\Sigma_2} X^\fraks)_{\frakt+i}
\ar@{ >->}[u]
}}\]
The composite to $C(X^\fraks\otimes_{\Sigma_2}X^\fraks)_{\frakt+i}$ is the definition of $\delta$ operations on the simplicial algebras which form each cosimplicial level. The diagram shows that we may replace it with the top composite, $\rho\overline{\Delta}_{t-i}\overline{D^s}\tau$, which will be more suitable for our purposes.


Actually, let's upgrade this a little bit. We can define the $\delta$ operations on a simplicial algebra by the formula $a\mapsto \Delta^{n-i}(a\otimes a)+\Delta^{n-i-1}(a\otimes \partial a)$, where we use the convention (which I once advised against) that $\Delta_\textup{negative}=0$. This is a chain map as long as we're away from the top operation.

Now we propose the map
\[x\mapsto \rho\overline{\Delta}_{\frakt-i}\overline{D}^\fraks(x\otimes x)+\rho\overline{\Delta}_{\frakt-i+1}\overline{D}^\fraks(x\otimes dx)\]
as a version of the $E^2$ $\delta$ structure. Indeed, this is supposed to preserve filtration (when mapping into the symmetric square spectral sequence), but the part of $dx$ coming from the cosimplicial differential has filtration one higher, and can thus be ignored at $E^2$. Then the remaining two summands, $\rho\overline{\Delta}_{\frakt-i}\overline{D}^\fraks(x\otimes x)+\rho\overline{\Delta}_{\frakt-i+1}\overline{D}^\fraks(x\otimes d_\textup{sim}x)$ pass happily through a commuting diagram like the one above, using the properties of the \emph{special} EZ map.

{\tiny[It's strange to wonder whether the top $\delta^\textup{alg}$ operation is the square, since it's not even \emph{defined}, and since the square has different filtration. It's a bit confusing whether or not they satisfy the Adem relations, due to the lift, but hopefully the proof of the Adem relations can be externalised to a fact about iterated $\gamma_i$, and there is some diagram involving iterated $\overline{\mu}$. Moreover, the product derived in this way will coincide with the spectral sequence product that we defined earlier (which agrees with the Lie coalgebra homology product). We haven't yet extended the $\delta$ operations to higher pages.]}


\subsection{Steenrod operations algebraically defined on $E^2$}
Define Steenrod operations $\Sq^\textup{alg}_i:E^2_{s,t}\to E^2_{s+i+1,2t+1}$ for all $i\geq 0$ using the composite:
\[N^s\pi_tX\overset{\textup{``}\overline{D}^{s-i}\circ\tau\textup{''}}{\to}N^{s+i}((\pi_tX)^{\otimes2})\overset{N^{s+i}(\Delta)}{\to}N^{s+i}\pi_{2t}(X\wedge X)\to N^{s+i+1}\pi_{2t+1}(X)\]
where (for want of a naming convention) $\Gamma^i$ is induced by the cochain map $N^s\pi_t(X)\to N^{s+i}(\pi_t(X)^{\otimes 2})$ defined by the following expression involving higher cosimplicial AW maps:
\[x\mapsto D^{s-i}(x\otimes x)+D^{s-i+1}(x\otimes dx)\]
Using the monster diagram, these are the same operations that one might define on Lie coalgebra homology. The same holds for the pairing, and the pairing here will coincide with the pairing defined in the previous subsection, which is to say that any two pairings you can think of coincide.
[{See that $\Sq^0=0$ as we're looking at Lie coalgebra homology.} Maybe $\Sq^1=0$, reflecting that it's hit by a $d_1$, but maybe not, expecially since there are $\SqShift^0$ terms in the Koszul sequences.]
Again with the shoehorn:
\[CX_{s,t}\overset{\textup{``}\overline{D}^{s-i}\circ\tau\textup{''}}{\to} (CX\otimes_\textup{v}CX)_{s+i,2t} \overset{\overline{\Delta}}{\to}C(X\otimes X)_{s+i,2t}\overset{\rho}{\to}C(X\otimes_{\Sigma_2} X)_{s+i,2t}\]
%\subsection{What remains}
%We need to deform the obvious definition of $\delta$ operations on the total complex, which is filtration doubling. The deformation will end up being the Steenrod operations. Then the unstablising differentials will presumably follow formally, and we'll presumably know how to strip off the extra $\delta$ sequences. I should try to emulate 2.7, and figure out what it implies.

\subsection{Divided squares doubling filtration}
% Now write $K_k$ for the composite
%\[(CX\otimes CY)_{\fraks,\frakt}\overset{D^0}{\to}C(X\otimes_\textup{v}Y)_{\fraks,\frakt}\overset{\Delta_k}{\to}C(X\otimes Y)_{\fraks,\frakt-k}\]
%defined when $2k\leq \frakt$. Similarly, write $P_k$ for the composite
%\[(CX\otimes CY)_{\fraks,\frakt}\overset{D^0}{\to}C(X\otimes_\textup{v}Y)_{\fraks,\frakt}\overset{\phi_k}{\to}C(X\otimes Y)_{\fraks,\frakt-k}\]
%defined when $2k\leq \frakt$. One readily observes that
%\[K_k+TK_kT=K_{k-1}d+dK_{k-1}+P_k.\]
%Now specialising to the case $X=Y$, write $\rho:X\otimes X\to X\otimes_{\Sigma_2}X$ for the natural surjection onto $\Sigma_2$-coinvariants.
{\tiny\begin{prop}
For $2\leq i\leq n$, the composite function $\rho\overline{\Delta}_{n-i}\overline{D}^0\tau$:
\[\mathclap{F_p CX_{n}\overset{\tau}{\to} F_{2p}(CX\otimes CX)_{2n}\overset{\overline{D}^0}{\to}F_{2p}C(X\otimes_\textup{v} X)_{2n}\overset{\overline{\Delta}_{n-i}}{\to}F_{2p}C(X\otimes X)_{n+i}\overset{\rho}{\to}F_{2p}C(X\otimes_{\Sigma_2} X)_{n+i}}\]
is well defined and preserves cycles and homologies between cycles. It is linear up to homology whenever $i<n$, and a quadratic refinement of the product when $i=n$ \textbf{if this makes sense}.
\end{prop}
\begin{proof}
Firstly, to check that this function is defined is to check that $2p\geq2(n-i)-2n$, which is true. In order to check that cycles are preserved, we calculate $d\rho(K_{\frakt-\fraks-i}(x\otimes x))$ to be
\[\rho(K_{\frakt-\fraks-i+1}((1+T)(x\otimes x)))+\rho K_{\frakt-\fraks-i}(d(x\otimes x))+\rho P_{\frakt-\fraks-i+1}(x\otimes x).\]
The first two of the three terms obviously vanish. As the simplicial degree of $x$ is uniformly at least $\frakt$, which exceeds $\frakt-\fraks-i+1$, the third term also vanishes.
\textbf{More to check, I guess.}
\end{proof}}
I think we'd better use $x\mapsto\rho\overline{\Delta}_{n-i}\overline{D}^0(x\otimes x)+\rho\overline{\Delta}_{n-i+1}\overline{D}^0(x\otimes dx)$.
\subsection{Attempted homotopy of $\Delta_{n-i} D^{0}\cdots$ ($E_r$ $\delta$-ops)}

Write $\twist F$ as shorthand for the function $TFT$. Then we have equations:
\[d_\textup{co}\Delta_{k-1}+\Delta_{k-1}d_\textup{co}= \phi_{k}+(1+\twist)\Delta_{k}\]
\[d_\textup{si}D^{j}+D^{j}d_\textup{si}=(1+\twist)D^{j-1}\]
showing that the chain homotopy $d_\textup{bi}(\Delta_{k-1}D^{j})+(\Delta_{k-1}D^{j}) d_\textup{bi}$ equals
\[\phi_{k}D^{j}+(1+\twist)\Delta_{k}D^{j}+\Delta_{k-1}(1+\twist)D^{j-1}.\]
Now write $J_r$ for the sum
\[J_r=\sum_{\beta-\alpha=r}\Delta_\beta\twist^\beta D^\alpha.\]
Let's calculate $dJ_{r-1}+J_{r-1}d$ (when $r\geq0\textbf{?}1$). Write
\[J_{r-1}=\sum_{j\geq0}\Delta_{j+r-1}\twist^{j+r-1}D^{j}\]
so that the homotopy equals
\[\sum_{j\geq0}\phi_{j+r}D^{j}+ (1+\twist)\Delta_{j+r}\twist^{j+r-1}D^{j}+\Delta_{j+r-1}(1+\twist)\twist^{j+r-1}D^{j-1}\]
\begin{alignat*}{2}
\textup{htpy}-\sum_{j\geq0}\phi_{j+r}D^{j}
&=
\sum_{j\geq0}(1+\twist)\Delta_{j+r}\twist^{j+r-1}D^{j}+\Delta_{j+r-1}(1+\twist)\twist^{j+r-1}D^{j-1}%
\\
&=
\Delta_{r-1}(1+\twist)D^{-1}+\Delta\sum_{j\geq0}(1+\twist)\Delta_{j+r}\twist^{j+r-1}D^{j}+\Delta_{j+r}(1+\twist)\twist^{j+r-1}D^{j-1}%
\end{alignat*}
\textbf{This must be wrong in detail, but also, we have the real problem that this technique will always return paired-up output (i.e.\ with a $(1+\twist)$), and there's NO equation $D=(1+\twist)D^0+?$.} We need to halve the homotopy, but how? HAVE I GOT THE DIRECTION OF THE DIFFERENTIALS BACKWARDS? Don't the $\delta$'s need to hit the squares, and so must themselves be the homotopy?

Now the following equations hold if you swap out each entry `$\textup{exp}$' for $\rho(\textup{exp})\tau y$:
\begin{alignat*}{2}
\Delta D^{s-i}
&=
(\Delta_0(1+T)+\phi_0)D^{s-i}%
\\
&\sim
\Delta_1(1+T)D^{s-i+1}+\phi_1D^{s-i+1}+\phi_0D^{s-i}%
\\
% Left hand side
% Relation
&\sim
% Right hand side
\Delta_r(1+T)D^{s-i+r}+\phi_rD^{s-i+r}+\cdots+\phi_0D^{s-i}%
% Comment
\end{alignat*}
where the homotopy used was
\[\Delta_0D^{s-i+1}+\cdots +\Delta_{r-1}D^{s-i+r}\]
I guess that if we push this all the way to $r=s-i$, you've used a homotopy with leading term $\Delta_{s-i-1}D^0$, which recalls the formula for $\delta_{i+1}$ on an element of $E^2_{t,s}$, so that $\delta_{i+i}$ has a differential to whatever Steenrod operation we were thinking about. Who knows?


\end{Adams sseq operations}

\begin{letter to Dwyer}

\noindent Dear Bill,

I'm sorry to have taken a while to put this stuff into writing. I figured out a few details since our last correspondence, and I thought that it made sense for me to make my calculations precise.

Let me explain the situation, as I've understood it. If $X$ is a mixed simplicial vectorspace, filtered by cosimplicial degree, there are three different types of operations $E^r_{-s,t}(X)\to E^?_{-?,?}(X\otimes_{\Sigma_2}X)$:
\begin{itemise}
\setlength{\parindent}{.25in}
\item \textbf{Steenrod operations:} $\Sq^i_\textup{alg}:E^?_{-s,t}(X)\to E^?_{-s-i,2t}(X\otimes_{\Sigma_2}X)$ (with certain indeterminacy and survival properties), represented on the $s$-line by a function $\textup{SQ}^{i,s}$, defined for all $i\geq0$ but zero unless $i\leq s$ (I think), and given by the correct formula on $E^2$;
\item \textbf{higher divided square operations doubling filtration:} $\delta_i^\textup{dub}:E^?_{-s,t}(X)\to E^?_{-2s,t+s+i}(X\otimes_{\Sigma_2}X)$, defined for $i\leq t-s$, which I presume are zero for $s>0$, but agree with the `target' homotopy operations $\pi_*(X^{-1})\to \pi_*(X^{-1}\otimes_{\Sigma_2}X^{-1})$ when $X$ is coaugmented. The operation $\delta_i^\textup{dub}$ is represented by a function `$\textup{DUB}_i$'
\item \textbf{higher divided square operations preserving filtration:}
$\delta_i^\textup{alg}:E^?_{-s,t}(X)\to E^?_{-s,t+i}(X\otimes_{\Sigma_2}X)$, defined for $i<t$ (with indeterminacy and survival properties unstudied), represented by a function $\textup{DEL}_{i}$.
\end{itemise}
The chain-level calculation which tells you about differentials on $\delta_i^\textup{alg}x$ is:
\begin{prop*}
For $2\leq i<t$ and $x\in Z^1_{-s,t}$:
\[d(\textup{DEL}_i(x))+ \textup{DEL}_i(dx)=\begin{cases}
\textup{SQ}^{t-i+1,s}(x),&\textup{if }n+1\leq i \leq t-1;\\
d(\textup{DUB}_i(x)),&\textup{if }i=n;\\
d(\textup{DUB}_i(x))+\textup{DUB}_i(dx),&\textup{if }2\leq i\leq n-1.
\end{cases}\]
\end{prop*}
This implies, for example, that if $x$ is a permanent cycle of dimension $n=t-s$, and $n<i\leq t-1$ (so that $x$ should support $\delta_i$ on $E^2$ but not on the target), then $d_{t-i+1}(\delta_i^\textup{alg}x)=\Sq^{t-i+1}_\textup{alg}x$. This gives a span of differentials from `excess $\delta^\textup{alg}$'-operations to steenrod operations like one might have hoped.

If $dx$ is in filtration too low, it appears that $\delta_i^\textup{alg}x$ will support an earlier differential to $\delta_i^\textup{alg}(d_rx)$, and I'm guessing that in that case the Steenrod operation will end up supporting a differential.

This has all been rather fun, but here's my current problem: I haven't been able to find a homotopy between $\textup{DEL}_i$ and $\textup{DUB}_i$. It would only be important to prove that $\textup{DEL}_i(x)$ is homologous to $\textup{DUB}_i(x)$ for permanent cycles $x$ (satisfying $i\leq n$), but this evades me. It seems like \ref{boundaryVsBBD} below is providing ``twice'' the required null-homology of $\textup{DEL}_i(x)-\textup{DUB}_i(x)$, when $dx=0$. To be explicit, when $dx=0$, $\textup{DEL}_i(x)=\mathbb{D}_{n-i}(x\otimes x)$, and $\textup{DUB}_i(x)=\Delta_{n-i}D^0(x\otimes x)$, and one has an equation
\[d\mathbb{D}_{n-i-1}+\mathbb{D}_{n-i-1}d=\mathbb{D}_{n-i}+T\mathbb{D}_{n-i}T +\Delta_{n-i}D^0+\Delta_{n-i}TD^0T,\]
where $T$ is always the available symmetry isomorphism. Applying this equation to $x\otimes x$, and passing to $\Sigma_2$-coinvariants via the map $\rho:C(X\otimes X)\to C(X\otimes_{\Sigma_2} X)$, we have
\[(d\mathbb{D}_{n-i-1}+\mathbb{D}_{n-i-1}d)(x\otimes x)=(\mathbb{D}_{n-i}+\mathbb{D}_{n-i}T)(x\otimes x)+\Delta_{n-i}(D^0+T D^0 T)(x\otimes x).\]
Almost, but not quite! Do you have any words of wisdom? Do you think I have the right operations?

Cheers,

 Michael.

\section*{Operations in the Adams spectral sequence}
\subsubsection*{Higher cosimplicial Eilenberg-Zilber map}
Let $\{D^k\}$ be a special cosimplicial Eilenberg-Zilber map, i.e.\  maps
\[D^k:(CR\otimes CS)_{-i-k}\to N(R\otimes S)_{-i}\textup{\ \ for $i,k\geq0$},\]
natural in cosimplicial vectorspaces $R,S$,
with the properties:
\begin{itemise}
\setlength{\parindent}{.25in}
\item $dD^k+D^kd=D^{k-1}+TD^{k-1}T$ for $k>1$;
\item $D^0$ is a chain homotopy equivalence inducing the identity in dimension zero;
\item the restriction of $D^k$ to $C_{-i}R\otimes C_{-j}S$ is zero unless $i\geq k$ and $j\geq k$; and
\item $D^k$ maps $C_{-k}R\otimes C_{-k}S$ identically onto $C_{-k}(R\otimes S)$.
\end{itemise}
It is a natural convention to define $D^k=0$ for $k<0$.
\subsubsection*{Higher simplicial Eilenberg-Mac Lane map}
Let $\{\Delta_k\}$ be a higher simplicial Eilenberg-Mac Lane map, i.e.\ maps
\[\Delta_k:(CU\otimes CV)_{i+k}\to N(U\otimes V)_i\textup{\ \ for $0\leq k\leq i$}\]
such that the identities
\[\Delta_k+T\Delta_kT=\phi_k+\begin{cases}
\Delta_{k-1}\partial+\partial\Delta_{k-1},&\textup{if }k\geq1;\\
\Delta,&\textup{if }k=0.
%\\,&\textup{if }
\end{cases}
\]
hold on classes of dimension at least $2k$, and:
\begin{itemise}
\setlength{\parindent}{.25in}
\item $\Delta:CU\otimes CV\to C(U\times V)$ is a chain homotopy equivalence inducing the identity in dimension zero; and
\item $\phi_k$ is the map $(CU\otimes CV)_{i+k}\to C(U\otimes V)_i$ which vanishes except on $U_k\otimes V_k$, where its value is just the projection $U_k\otimes V_k\to N(U\times V)_k$.
\end{itemise}
Note that $\phi$ commutes with symmetry isomorphisms, and thus so must $\Delta$.

It is not a natural convention to regard $\Delta_k$ as zero where it is not defined, however, we will perform this abuse shortly for notational convenience. In this case, we will be able to use the equation
\[(d\Delta_k+\Delta_kd)z=(\phi_{k+1}+(1+\twist)\Delta_{k+1})z\]
whenever $k\geq0$ and the simplicial dimension of $z$ does not equal either of $2k$ and $2k+1$.

\subsubsection*{Mixed simplicial vectorspaces over $\F_2$}
For mixed simplicial vector spaces $X$ and $Y$, write $C(X\otimes_\textup{v}Y)$ for the double complex which in degree $(\fraks,\frakt)$ equals the direct sum $\bigoplus_{t+t'=\frakt}X_t^\fraks\otimes X_{t'}^\fraks$. The following vectorspace maps are given by prolonging $D^k$, $\Delta$, $\Delta_k$ and $\psi_k$:
\begin{align*}
{D}^k:(CX\otimes CY)_{-\fraks-k,\frakt}&\to C(X\otimes_\textup{v}Y)_{-\fraks,\frakt}&\quad&\textup{(zero unless $k\leq \fraks$)}\\
{\Delta}:C(X\otimes_\textup{v} Y)_{-\fraks,\frakt}&\to C(X\otimes Y)_{-\fraks,\frakt}\\
{\Delta}_k:C(X\otimes_\textup{v} Y)_{-\fraks,\frakt+k}&\to C(X\otimes Y)_{-\fraks,\frakt}&\quad&\textup{for $0\leq k\leq \frakt$}\\
{\phi}_k:C(X\otimes_\textup{v} Y)_{-\fraks,\frakt+k}&\to C(X\otimes Y)_{-\fraks,\frakt}&\quad&\textup{for $0\leq k\leq \frakt$}\\
{\rho}:C(X\otimes X)_{-\fraks,\frakt}&\to C(X\otimes_{\Sigma_2} X)_{-\fraks,\frakt}&\quad&\textup{(proj onto coinvariants)}\\
\end{align*}
Actually, let's define these on the entire double complex by just defining them to be zero in dimensions they aren't supposed to be defined. Also, we'll always write $T$ for symmetry isomorphisms, and write ``$\twist F$'' as shorthand for the function $TFT$. Although this notation is potentially ambiguous, whenever we write $\sigma FG$, for functions $F$ and $G$, we mean $(\sigma F)G$, not $\sigma(FG)$.

\subsubsection*{Instead of mixed simplicial algebras over $\F_2$}
Instead of letting $X$ be a mixed simplicial commutative non-unital $\F_2$-algebra, I'd prefer to constructing spectral sequence operations as maps $E^r_{-s,t}(X)\to E^?_{?,?}(X\otimes_{\Sigma_2}X)$, for $X$ simply a mixed simplicial vectorspace.

\subsection{Steenrod operations}
In this case, the `surprising' operations are easy to define. That is, we can define Steenrod operations $\Sq^i_\textup{alg}:E^r_{-s,t}(X)\to E^r_{-s-i,2t}(X\otimes_{\Sigma_2}X)$ for all $i\geq 0$ and $r\geq2$, which survive to $E^{2r-1}$ and have indeterminacy vanishing by $E^{2r-2}$, using the chain-level formula:
\[\textup{SQ}^{i,s}:x\mapsto \rho\Delta (D^{s-i}(x\otimes x)+D^{s-i+1}(x\otimes dx)).\]
The usual trick, along with the fact that $\Delta$ is a symmetric chain map, shows that $d(\textup{SQ}^{i,s}(x))=\rho\Delta D^{s-i+1}(dx\otimes dx)$, so that if $x\in E^r_{-s,t}$, $\textup{SQ}^{i,s}(x)\in E^{2r-1}_{-s-i,2t}$. To examine the indeterminacy, $x\in E^r_{-s,t}$ is determined up to boundaries of $y\in E^{r-1}_{-(s-r+1),t+r+2}$, and the following equation%footnote:
\footnote{Here's how to prove said equation, using analogues of \cite[(1.111),(1.112)]{MR2245560}:
\begin{alignat*}{2}
\textup{RHS}
&=
d\rho\Delta [D^{s-i-1}(y\otimes y)+D^{s-i}(y\otimes dy)+D^{s-i+1}(dy\otimes x)]%
\\
&=
\rho\Delta [D^{s-i-1}(dy\otimes y+y\otimes dy)+D^{s-i}(dy\otimes dy)+D^{s-i+1}(dy\otimes dx)]%
\\
&\ \ \ +
\rho\Delta [0+D^{s-i-1}(y\otimes dy+dy\otimes y)+D^{s-i}(dy\otimes x+x\otimes dy)]%
\\
&=
\rho\Delta [D^{s-i}(dy\otimes x+x\otimes dy+dy\otimes dy)
+D^{s-i+1}(dy\otimes dx)]=\textup{LHS}%
\end{alignat*}} shows how this affects $\Sq^{i}_\textup{alg}$:
\[\textup{SQ}^{i,s}(x)-\textup{SQ}^{i,s}(x+dy)=d\rho\Delta [D^{s-i-1}(y\otimes y)+D^{s-i}(y\otimes dy)+D^{s-i+1}(dy\otimes x)].\]
The filtration of this bounding cycle is at least $2(s-r+1)-(s-i-1)=(s+i)-(2r-3)$, so that $\Sq_\textup{alg}^i$ has indeterminacy $2r-2$ (phew).



\subsection{$\delta$ operations doubling filtration}
For $x\in CX_{n}$, define
\[\textup{DUB}_i(x):=\rho(\Delta_{n-i}D^0(x\otimes x)+\Delta_{n-i-1}D^0(x\otimes dx))\textup{\quad for $i<n$}\]
and (setting $i=n$) define 
\[\textup{DUB}_n(x):=\rho\Delta_{0}D^0(x\otimes x).\]
This gives operations on the total complex which are compatible with the $\delta$-operations on the target. Unfortunately, these double filtration, and are \textbf{presumably} zero on $E^2_{-s,t}$ when $s>0$.
\begin{lem}\label{htpyInducedByBUB}
For $2\leq i<n$:
\[d(\textup{DUB}_i(x))+\textup{DUB}_i(dx)=\rho(\Delta_{n-i+1}(1+\twist)D^0(x\otimes x)+\Delta_{n-i}(1+\twist)D^0(dx\otimes x))\]
and when $i=n$:
\[d(\textup{DUB}_i(x))+\rho\Delta D^0(x\otimes dx)=\rho(\Delta_{1}(1+\twist)D^0(x\otimes x)+\Delta_{0}(1+\twist)D^0(dx\otimes x))\]
\end{lem}
\begin{proof}
Suppose that $i<n$. Before calculating $(d\Delta_{n-i}+\Delta_{n-i} d)(D^0(x\otimes x))$, we note that the simplicial dimension of $D^0(x\otimes x)$ is $2t\geq2n>2(n-i+1)$. Similarly, the simplicial dimension of $D^0(x\otimes dx)$ is at least $2t-1>2((n-i-1)+1)$. Thus we may use the formula commuting $d$ and $\Delta_k$. In light of the fact that $D^0$ is a chain map,
\begin{alignat*}{2}
d(\textup{DUB}_i(x))
&=
\rho\Delta_{n-i}D^0d(x\otimes x)+\rho (\phi_{n-i+1}+(1+\twist)\Delta_{n-i+1})D^0(x\otimes x)+{}%
\\
&\ \ +\rho\Delta_{n-i-1}D^0d(x\otimes dx)+\rho (\phi_{n-i}+(1+\twist)\Delta_{n-i})D^0(x\otimes dx)
\end{alignat*}
As ever the $\phi$ terms dissappear, and $d(\textup{DUB}_i(x))+\textup{DUB}_i(dx)$ equals
\begin{alignat*}{2}
\rho&(\Delta_{n-i}D^0d(x\otimes x)+(1+\twist)\Delta_{n-i+1}D^0(x\otimes x)+(1+\twist)\Delta_{n-i}D^0(x\otimes dx))\\
&=\rho(\Delta_{n-i}D^0(1+T)(x\otimes dx)+\Delta_{n-i+1}(1+\twist)D^0(x\otimes x)+\Delta_{n-i}(1+T)D^0(x\otimes dx))
\end{alignat*}
which equals the desired result, as $(1+T)F+F(1+T)=(1+\twist)FT$.

In the case $i=n$, the second line in the expression for $d(\textup{DUB}_i(x))$ does not appear. The first term therein is the vestigial ``$\textup{DUB}_n(dx)$'', while the second term
$\rho (\phi_{n-i}+(1+\twist)\Delta_{n-i})D^0(x\otimes dx)$ equals $\rho\Delta D^0(x\otimes dx)$.
\end{proof}

\subsection{$\delta$ operations}

Now write $\mathbb{D}_r$ for the sum (for any $r$, positive or otherwise):
\[\mathbb{D}_r=\sum_{\alpha-\beta=r}\Delta_\alpha\twist^\alpha D^\beta=\begin{cases}
\sum_{j\geq0}\Delta_{j}\twist^{j}D^{j-r},&\textup{if }r\leq0;\\
\sum_{j\geq0}\Delta_{j+r}\twist^{j+r}D^{j},&\textup{if }r\geq0.
%\\,&\textup{if }
\end{cases}
\]
Now suppose that $x\in E^r_{-s,t}$. Let's try to define $\delta_i^\textup{alg}(x)$ for $n<i<t$, where $n=t-s=|x|$. I believe that the correct formula may be
\[x\mapsto \textup{DEL}_{i}(x):=\rho(\mathbb{D}_{n-i}(x\otimes x)+\mathbb{D}_{n-i-1}(dx\otimes x)).\]
\subsubsection*{Correctness at $E^2$}
Let's calculate the leading term of $\textup{DEL}_{i}(x)$ whenever $2\leq i<n$. Write $x=x^s_t+x^{>s}_{>t}$, where $x^s_t\in X^s_t$, and $x^{>s}_{>t}$ lies in filtration $s+1$. Then the leading term will be $\rho\Delta_{t-i}\twist^{t-i}D^s (x^s_t\otimes x^s_t)$. Moreover, as $D$ is \emph{special}, this equals $\rho\overline{\Delta}_{t-i}(x^s_t\otimes x^s_t)\in C(X\otimes_{\Sigma_2} X)_{-s,t+i}$ (where we think of $(x^s_t\otimes x^s_t)$ as an element of $C(X\otimes_\textup{v}X)_{-s,2t}$), so that `$x^s_t\mapsto D^s(x^s_t\otimes x^s_t)$' implements the assignment `$z\mapsto z\otimes z$' at the $E^1$-level, and we're getting a representative for the obvious $E^2$-level $\delta_i$ operation.

\newcommand{\twolinesum}[2]{\mathop{\sum_{\mathclap{#1}}}_{\mathclap{#2}}}
\newcommand{\onelinesum}[1]{\sum_{\mathclap{#1}}}
\subsubsection*{Formula for the boundary}
\begin{lem}\label{boundaryVsBBD}
For all $r$, positive or otherwise, the equation:
\[d\mathbb{D}_r+\mathbb{D}_rd=(1+\twist)\mathbb{D}_{r+1}+\Delta\sigma D^{-r-1}+\Delta_{r+1}(1+\twist)D^0\]
holds when applied to an element $z$ of $(CX\otimes CX)_{n}$ with $n> 2(r+1)$.
\end{lem}
\begin{proof}We'll need to use our formula for $d\Delta_\alpha+\Delta_\alpha d$ applied to $\twist^\alpha D^\beta(z)$. %, where $z\in (CX\otimes CX)_{-s,t}$ is the element to which we're applying the above equation.
The formula holds unless $t=2\alpha+e$, where $e\in\{0,1\}$, in which case $\beta=(t-e)/2-r$. However, $D^\beta(z)$ is zero unless $\beta\leq s/2$, so the formula can only fail if $(t-e)/2-r\leq s/2$. As we've assumed that $n=t-s>2(r+1)$, this does not occur.
%
% Now $D^\beta(z)$ is zero unless $\beta\leq 2s$, so that the term can be ignored unless $\alpha=\beta+r\leq 2s+r$. As $t\geq 2(2s+r+1)\geq 2(\alpha+1)$, we can apply the rule.
\begin{alignat*}{2}
\textup{LHS}
&=
\onelinesum{\alpha-\beta=r}\left(d\Delta_\alpha\twist^\alpha D^\beta+\Delta_\alpha\twist^\alpha D^\beta d\right)%
\\
&=
\twolinesum{\alpha-\beta=r}{\alpha\geq0,\,\beta\geq0}\left(((1+\twist)\Delta_{\alpha+1}+\phi_{\alpha+1})\twist^\alpha D^\beta+\Delta_\alpha(1+\twist)\twist^\alpha D^{\beta-1}\right)%
\\
&=
\onelinesum{\alpha-\beta=r+1}\left(((1+\twist)\Delta_{\alpha}+ \phi_{\alpha})\twist^{\alpha-1} D^\beta+\Delta_\alpha(1+\twist)\twist^\alpha D^{\beta}\right) + {}%
\\
&\ \ \ \ \ \ \ \ \ \ \ +
((1+\twist)\Delta_{0}+\phi_{0})\twist D^{-r-1}+\Delta_{r+1}(1+\twist)\twist^\alpha D^{0}%
\end{alignat*}
Using the identity $(1+\twist)\Delta_{0}+\phi_{0}=\Delta$, and the observations that $(1+\twist)\twist^\alpha F=(1+\twist)F$ and $(1+\twist)F\twist G+F(1+\twist)G=(1+\twist)(FG)$, one sees that this differs from the desired value only by the term
\[\sum_{\mathclap{\alpha-\beta=r+1}} \phi_{\alpha}\sigma^{\alpha-1}D^{\beta}(z)=\sum_{\mathclap{\alpha-\beta=r+1}} \phi_{\alpha}\sigma^{\alpha-1}D^{\alpha-r-1}(z_{2\alpha}^{2\alpha-n})\]
but the cosimplicial degree $2\alpha-n$ is less than $2(\alpha-r-1)$, and $\{D^k\}$ is special, so that these terms vanish.
\end{proof}
\begin{cor}
For $2\leq i<t$ and $x\in Z^1_{-s,t}$:
\[d(\textup{DEL}_i(x))+ \textup{DEL}_i(dx)=\begin{cases}
\textup{SQ}^{t-i+1,s}(x),&\textup{if }n+1\leq i \leq t-1;\\
d(\textup{DUB}_i(x)),&\textup{if }i=n;\\
d(\textup{DUB}_i(x))+\textup{DUB}_i(dx),&\textup{if }2\leq i\leq n-1.
\end{cases}\]
\end{cor}
\begin{proof}
We may apply \ref{boundaryVsBBD} to calculate $d\mathbb{D}_{n-i}(x\otimes x)$ and $d\mathbb{D}_{n-i-1}(dx\otimes x)$, since $|x\otimes x|=2n> 2(n-i+1)$ and $|dx\otimes x|=2n-1> 2(n-i-1+1)$. Thus:
\begin{alignat*}{2}
\textup{LHS}
&=
\rho d(\mathbb{D}_{n-i}(x\otimes x)+\mathbb{D}_{n-i-1}(dx\otimes x))+\rho\mathbb{D}_{n-i-1}(dx\otimes dx)%
\\
&=
\rho\Bigl\{d\mathbb{D}_{n-i}(x\otimes x)\Bigr\}+\rho\Bigl\{d\mathbb{D}_{n-i-1}(dx\otimes x))+\mathbb{D}_{n-i-1}(d(dx\otimes x))\Bigr\}%
\\
&=
\rho\Bigl\{\mathbb{D}_{n-i}d(x\otimes x)+(1+\twist)\mathbb{D}_{n-i+1}(x\otimes x)+{}
\\
&\ \ \ \ \ \ \ \ \ +\bigl(\Delta\sigma D^{i-n-1}+\Delta_{n-i+1}(1+\twist)D^0\bigr)(x\otimes x)\Bigr\}+{}%
\\
&\ \ +\rho\Bigl\{(1+\twist)\mathbb{D}_{n-i}(dx\otimes x)+{}
\\
&\ \ \ \ \ \ \ \ \ +\bigl(\Delta\sigma D^{i-n}+\Delta_{n-i}(1+\twist)D^0\bigr)(dx\otimes x)\Bigr\}
\end{alignat*}
%The term $\rho(1+\twist)\mathbb{D}_{n-i+1}(x\otimes x)$ is zero, and the first terms of each of the first and third line cancel. What remains is the following sum:
The contents of the first and third lines cancel, leaving the sum of:
%\begin{alignat*}{2}
%\textup{term A }
%&=\ 
%\sum_{\mathclap{\alpha-\beta=n-i+1}} \rho\phi_{\alpha}\sigma^{\alpha-1}D^{\beta}(x\otimes x)+\sum_{\mathclap{\alpha-\beta=n-i}} \rho\phi_{\alpha}\sigma^{\alpha-1}D^{\beta}(dx\otimes x)%
%\\
%&=\ 
%\sum_{\mathclap{\alpha-\beta=n-i+1}} \rho\phi_{\alpha}\sigma^{\alpha-1}D^{\alpha-n+(i-1)}(x_\alpha^{\alpha-n}\otimes x_\alpha^{\alpha-n})+\textit{similar}%
%\end{alignat*}
%and everything here vanishes, as the $D^k$ are special and $i\geq2$.
\begin{alignat*}{2}
\textup{summand A}
&=
\rho\Delta(D^{i-n-1}(x\otimes x)+D^{i-n}(x\otimes dx))%
\\
&=
\begin{cases}
\textup{SQ}^{t-i+1,s}(x),&\textup{if }n+1\leq i \leq t-1;\\
\rho\Delta D^0(x\otimes dx),&\textup{if }i=n;\\
0,&\textup{if }2\leq i\leq n-1,
\end{cases}\\
%\end{alignat*}
%And finally, using \ref{htpyInducedByBUB}, we have
%\begin{alignat*}{2}
\textup{summand B}
&=
\rho\Delta_{n-i+1}(1+\twist)D^0(x\otimes x)+\rho\Delta_{n-i}(1+\twist)D^0(dx\otimes x)%
\\
&=
\begin{cases}
0,&\textup{if }n+2\leq i \leq t-1;\\
\rho\Delta_{0}(1+\twist)D^0(x\otimes x),&\textup{if }i=n+1;\\
d(\textup{DUB}_i(x))+\rho\Delta D^0(x\otimes dx),&\textup{if }i=n;\\
d(\textup{DUB}_i(x))+\textup{DUB}_i(dx),&\textup{if }2\leq i\leq n-1.
\end{cases}
\end{alignat*}
Now the term appearing in summand B when $i=n+1$ is actually zero, as it equals $\rho\phi_0D^0(x\otimes x)$, which vanishes, as $t>i\geq2$.
%\begin{alignat*}{2}
%&
%\rho\Delta(D^{i-n-1}(x\otimes x)+D^{i-n}(x\otimes dx))+\\
%&\ \ +\rho\Delta_{n-i+1}(1+\twist)D^0(x\otimes x)+\rho\Delta_{n-i}(1+\twist)D^0(dx\otimes x)+\\
%&\ \ +\sum_{\mathclap{\alpha-\beta=n-i+1}} \rho\phi_{\alpha}\sigma^{\alpha-1}D^{\beta}(x\otimes x)+\sum_{\mathclap{\alpha-\beta=n-i}} \rho\phi_{\alpha}\sigma^{\alpha-1}D^{\beta}(dx\otimes x).
%\end{alignat*}
%Now the bottom line of this expression is actually zero, since it can be written as a sum of linear functions of the expressions
%\[\sigma^{\alpha-1}D^{\alpha-n+(i-1)}(x_\alpha^{\alpha-n}\otimes x_\alpha^{\alpha-n})\textup{ and }\sigma^{\alpha-1}D^{\alpha-n+i}((dx)_\alpha^{\alpha-n+1}\otimes x_\alpha^{\alpha-n})\]
%which all vanish, since the $D^k$ are special and $i\geq2$. 
%%As for the middle line, supposing that $i\leq n$ we manipulate:
%%\begin{alignat*}{2}
%%\rho\Delta_{n-i+1}(1+\twist)D^0(x\otimes x)
%%&=
%%\rho(1+\twist)\Delta_{n-i+1}D^0(x\otimes x)%
%%\\
%%&=
%%\rho(\phi_{n-i+1}+d\Delta_{n-i}+\Delta_{n-i}d)D^0(x\otimes x)%
%%\\
%%% Left hand side
%%% Relation
%%&=
%%% Right hand side
%%d\rho\Delta_{n-i}(x\otimes x)+\rho\Delta_{n-i}(1+\twist)D^0(dx\otimes x)%
%%% Comment
%%\end{alignat*}
%%where the term corresponding to $\phi_{n-i+1}$ vanished for dimension reasons, and we used the fact that $D^0$ is a chain map.
%%\[\rho\Delta_{n-i+1}(1+\twist)D^0(x\otimes x)=\rho(1+\twist)\Delta_{n-i+1}D^0(x\otimes x)\]
%The middle line is our formula derived for $d(\textup{DUB}_i(x))+\textup{DUB}_i(dx)$ (\textbf{at least when $i<<n$}).
\end{proof}
%%%AN EARLY, SIMPLISTIC CALCULATION
%It's not too hard to check%footnote: 
%\footnote{
%We'll need the equations:
%\[d\Delta_{k-1}+\Delta_{k-1}d= \phi_{k}+(1+\twist)\Delta_{k}\]
%\[dD^{j}+D^{j}d=(1+\twist)D^{j-1}\]
%showing that the chain homotopy $d(\Delta_{k}\sigma^kD^{j})+(\Delta_{k}\sigma^kD^{j}) d$ equals
%\[\phi_{k+1}\sigma^kD^{j}+(1+\twist)\Delta_{k+1}\sigma^kD^{j}+ \Delta_{k}(1+\twist)\sigma^kD^{j-1}.\]
%Let's calculate $d\mathbb{D}_{r}+\mathbb{D}_{r}d$ (when $r\leq-1$).
%\begin{alignat*}{2}
%d\mathbb{D}_r
%&+
%\mathbb{D}_rd-\sum_{j\geq0} \phi_{j+1}\sigma^jD^{j-r}-(1+\twist)\Delta_0\sigma D^{-r-1}%
%\\
%&=
%\sum_{j\geq-1} 
%(1+\twist)\Delta_{j+1}\sigma^jD^{j-r}+ \sum_{j\geq0}\Delta_{j}(1+\twist)\sigma^jD^{j-r-1}
%\\
%&=
%\sum_{j\geq0} 
%\left[(1+\twist)\Delta_{j}\sigma\sigma^jD^{j-r-1}+ \Delta_{j}(1+\twist)\sigma^jD^{j-r-1}\right]
%\\
%% Left hand side
%% Relation
%&=
%% Right hand side
%(1+\twist)\mathbb{D}_{r+1}
%% Comment
%\end{alignat*}
%using the observation that $(1+\twist)F\twist G+F(1+\twist)G=(1+\twist)(FG)$. Then use the fact that $(1+\twist)\Delta_0=\Delta+\phi_0$.} that for $r\leq-1$:
%\[d\mathbb{D}_r+\mathbb{D}_rd=(1+\twist)\mathbb{D}_{r+1}+\Delta\sigma D^{-r-1}+ \sum_{j\geq0} \phi_{j}\sigma^{j-1}D^{j-r-1},\]
%from which it's a formality to derive that for $i>|x|$
%\[d(\textup{DEL}^\textup{alg}_i(x))= \textup{DEL}^\textup{alg}_i(dx)+S_\textup{alg}^{t-i+1,s}(x)+\cancel{\textup{error}}\]
%where the crossed out error term,
%\[\sum_{j\geq0}\phi_j\sigma^{j-1}D^{j+i-n-1}(x\otimes x)+\sum_{j\geq0}\phi_j\sigma^{j-1}D^{j+i-n}(dx\otimes x))\]
%is in fact zero%footnote:
%\footnote{Write $x=\sum x_{j}^{j-n}$ as the sum of its various homogeneous parts. Then 
%\[\sum\phi_j\sigma^{j-1}D^{j+i-n-1}(x\otimes x)=\sum\phi_j\sigma^{j-1}D^{j+i-n-1}(x_j^{j-n}\otimes x_j^{j-n})=0,\]
%as $D^{j-n+(i-1)}(x_j^{j-n}\otimes x_j^{j-n})=0$ (since $i\geq2$, and the $D^k$ are special). This shows that the first summand is zero. For the second, we end up considering terms $D^{j-n+i}(T^{j-1}((dx)_{j}^{j-n+1}\otimes x_j^{j-n}))$, which vanish as before.}.




\subsection{$\textup{DEL}+\textup{DUB}$}
Suppose that $x$ is a permanent cycle of dimension $n$, and $y$ has dimension $n+1$.
\begin{prop}
$\textup{DpD}_i(x)\sim \textup{DpD}_i(x+dy)$ for $i<n$.
\end{prop}
\begin{proof}
{\small
\begin{alignat*}{4}
d\mathbb{D}_{n-i-1}(x\cdot dy)&=
\cancel{\mathbb{D}_{n-i-1}(dx\cdot dy)}&&+
(1{+}\twist)\mathbb{D}_{n-i}(x\cdot dy)&&+
\Delta_{n-i}(1{+}\sigma)D^0(x\cdot dy)\\
d\mathbb{D}_{n-i}(dy\cdot y)&=
\mathbb{D}_{n-i}(dy\cdot dy)&&+
(1{+}\twist)\mathbb{D}_{n-i+1}(dy\cdot y)&&+
\Delta_{n-i+1}(1{+}\sigma)D^0(dy\cdot y)\\
d\mathbb{D}_{n-i+1}(y\cdot y)&=
\mathbb{D}_{n-i+1}d(y\cdot y)&&+
\cancel{(1{+}\twist)\mathbb{D}_{n-i+2}(y\cdot y)}&&+
\Delta_{n-i+2}(1{+}\sigma)D^0(y\cdot y)\\
d\Delta_{n-i-1}\twist D^0(x\cdot dy)&=
\cancel{\Delta_{n-i-1}\twist D^0(dx\cdot dy)}&&+
(1{+}\twist)\Delta_{n-i}\twist D^0(x\cdot dy)&&+
\cancel{\phi_{n-i}\twist D^0(x\cdot dy)}\\
d\Delta_{n-i}D^0(dy\cdot y)&=
\Delta_{n-i}D^0(dy\cdot dy)&&+
(1{+}\twist)\Delta_{n-i+1}D^0(dy\cdot y)&&+
\cancel{\phi_{n-i+1}D^0(dy\cdot y)}\\
d\Delta_{n-i}\twist D^0(y\cdot y)&=
\Delta_{n-i}\twist D^0d(y\cdot y)&&+
(1{+}\twist)\Delta_{n-i+2}\twist D^0(y\cdot y)&&+
\cancel{\phi_{n-i+2}\twist D^0(y\cdot y)}
\end{alignat*}}
\[\textup{A2}+\textup{B1}=\mathbb{D}_{n-i}((x+dy)\cdot(x+dy)-x\cdot x)\]
\[\textup{A5}+\textup{B4}+\textup{C1}=\Delta_{n-i}D^0((x+dy)\cdot(x+dy)-x\cdot x)\]
\[\textup{A3}+\textup{B2}=\textup{A6}+\textup{B5}+\textup{C2}= \textup{B6}+\textup{C3}=0\]
\end{proof}

\pagebreak

\subsection{Alternating $\delta$ operations doubling filtration}
For $x\in CX_{n}$, define
\[\textup{DUB}'_i(x):=\rho(\Delta_{n-i}\twist^{n-i}D^0(x\otimes x)+\Delta_{n-i-1}\twist^{n-i-1}D^0(x\otimes dx))\textup{\quad for $i<n$}\]
and (setting $i=n$) define 
\[\textup{DUB}'_n(x):=\rho\Delta_{0}D^0(x\otimes x).\]
This gives operations on the total complex which are compatible with the $\delta$-operations on the target. Unfortunately, these double filtration, and are \textbf{presumably} zero on $E^2_{-s,t}$ when $s>0$.
\begin{lem}\label{htpyInducedByBUB}
For $2\leq i<n$:
\[d(\textup{DUB}'_i(x))+\textup{DUB}'_i(dx)=\rho(\Delta_{n-i+1}(1+\twist)D^0(x\otimes x)+\Delta_{n-i}(1+\twist)D^0(dx\otimes x))\]
and when $i=n$:
\[d(\textup{DUB}'_i(x))+\rho\Delta D^0(dx\otimes x)=\rho(\Delta_{1}(1+\twist)D^0(x\otimes x)+\Delta_{0}(1+\twist)D^0(dx\otimes x))\]
\end{lem}
\begin{proof}
Suppose that $i<n$. Before calculating $(d\Delta_{n-i}+\Delta_{n-i} d)(\twist^{n-i}D^0(x\otimes x))$, we note that the simplicial dimension of $D^0(x\otimes x)$ is $2t\geq2n>2(n-i+1)$. Similarly, the simplicial dimension of $D^0(x\otimes dx)$ is at least $2t-1>2((n-i-1)+1)$. Thus we may use the formula commuting $d$ and $\Delta_{n-i}$. In light of the fact that $D^0$ is a chain map, $d(\textup{DUB}'_i(x))$ equals
\begin{alignat*}{2}
&
\rho\Delta_{n-i}\twist^{n-i}D^0d(x\otimes x)+\rho (\phi_{n-i+1}+(1+\twist)\Delta_{n-i+1})\twist^{n-i}D^0(x\otimes x)+{}%
\\
&\ \ +\rho\Delta_{n-i-1}\twist^{n-i-1}D^0d(x\otimes dx)+\rho (\phi_{n-i}+(1+\twist)\Delta_{n-i})\twist^{n-i-1}D^0(x\otimes dx)
\end{alignat*}
As ever the $\phi$ terms dissappear, and $d(\textup{DUB}'_i(x))+\textup{DUB}'_i(dx)$ equals
\begin{alignat*}{2}
\rho&((1+\twist)\Delta_{n-i+1}\twist^{n-i}D^0(x\otimes x)+
\Delta_{n-i}\twist^{n-i}D^0d(x\otimes x)+
(1+\twist)\Delta_{n-i}\twist^{n-i-1}D^0(x\otimes dx))\\
&=\rho(\Delta_{n-i+1}(1+\twist)D^0(x\otimes x)+\Delta_{n-i} \twist^{n-i}D^0(1+T)(x\otimes dx)+\Delta_{n-i}(1+T)\twist^{n-i-1}D^0(x\otimes dx))
\end{alignat*}
which equals the desired result, as $(1+T)F+F(1+T)=(1+\twist)FT$.

In the case $i=n$, the second line in the expression for $d(\textup{DUB}'_i(x))$ does not appear. The first term therein is the vestigial ``$\textup{DUB}'_n(dx)$'', while the second term
$\rho (\phi_{n-i}+(1+\twist)\Delta_{n-i})\twist^{n-i-1}D^0(x\otimes dx)$ equals $\rho\Delta D^0(dx\otimes x)$.
\end{proof}





\end{letter to Dwyer}

\begin{KoszulComplexes_n>1}
\section{Unstable Koszul complexes for $\derived\Ind{\nontop{}}$} \label{sectionOnUnstableKoszulComplexes}
\subsection{Koszul complexes for $\derived\Ind{\nontop{}}$ for $n\geq2$}
We will define Koszul complexes calculating $\derived\Ind{\nontop{}}$ for all $n$, using the techniques of \cite{PriddyKoszul.pdf}. It will be useful to record here a customized version of Priddy's arguments, for three reasons: his results do not apply to our situation without minor modification to accomodate unstable conditions; we will require a technique from his proof of \cite[Thm 5.3]{PriddyKoszul.pdf} in what follows; we prefer to work with the Koszul complex, not the coKoszul complex.
%We Due to the fact that we are working with categories of unstable modules, there are straightforward changes to be made to Priddy's exposition. We make these changes here. The other benefit to covering this material is that we will
%Moreover, for certain calculations to follow we will need to refer to certain techniques found in the proof of \cite[Thm 5.3]{PriddyKoszul.pdf}. In order that our reader may understand these techniques, we 

%\textbf{remove }For $I=(i_\ell,\ldots,i_1)$ a sequence of nonnegative integers, define
%\[\minDim(I):=\begin{cases}
%-\infty,&\textup{if }I=\emptyset;\\
%\max\{(i_1),\,(i_2-i_1),\,\ldots,\,(i_{\ell}-i_{\ell-1}-\cdots i_1)\},&\textup{otherwise}.
%%\\,&\textup{if }
%\end{cases}
%\]
%Thus, for $n\geq2$:
%\begin{itemise}
%\setlength{\parindent}{.25in}
%\item  for $I\neq\emptyset$, $\minDim(I)$ is the lowest dimension $d\geq0$ such that $\Q^I$ may act on a $d$-dimensional element of an object in $\LL{n}$, if we ignore the extra restrictions on $\Q^0$; and
%\item $\minDim(I)$ is the unique element of $\{-\infty,0,1,2,\ldots\}$ such that for any $X\in\nontop{n}$ and any $x\in X$, $\Q^Ix$ is defined if and only if $\minDim(I)<|x|$, if we ignore the extra restrictions on $\Q^0$.
%\end{itemise}

%\subsubsection{The unstable bar construction and Koszul complex for $\nontop{n}$, $n\geq2$}
%%%%\begin{shaded}
%%%%For any object $X\in\nontop{n}$, the bar construction ${B}_\bullet(\Fr{\nontop{n}},\Fr{\nontop{n}},X)=(\Fr{\nontop{n}})^{\bullet+1}X$ is a free simplicial resolution of $X$. Now there is an isomorphism of graded vector spaces $\Ind{\nontop{}}(\Fr{\nontop{n}}(X))\cong\forgetSymbol(X)$. Thus $(\derived\Ind{\nontop{}})X$ may be calculated as the homotopy of the bar construction $B_\bullet(\forgetSymbol,\Fr{\nontop{n}},X)=(\Fr{\nontop{n}})^\bullet X$. Normalizing this complex by taking the quotient by degenerate simplices, one obtains the reduced Bar construction $\Boverline_*(\forgetSymbol,\Fr{\nontop{n}},X)$, a chain complex calculating $(\derived\Ind{\nontop{}})X$, which we now describe.
%%%%\end{shaded}
For any object $X\in\nontop{n}$, write $\Boverline_*:= \ModDegeneracies{\Ind{\nontop{}}(\BarConst{\nontop{}}X)}_*$. This is the normalized complex, obtained by taking the quotient by degenerate simplices, of the simplicial bar construction $\Ind{\nontop{}}(\BarConst{\nontop{}}X)$ calculating $(\derived\Ind{\nontop{}})X$.
\begin{prop}\label{PropDescnOfReducedBarConstruction}
As a vector space each $\Boverline_r$ is graded, with $\Boverline_{r,\ell}$
%For $r\geq0$, $\Boverline_r(\forgetSymbol,\Fr{\nontop{n}},X)$ is 
spanned by monomials
\[\BarMonomial{I}{\textbf{k}}{x}:=\left[\Q^{i_{k_r+\cdots +k_1}}\cdots \Q^{i_{k_{r-1}+\cdots +k_1+1}}
\middle|\cdots 
\middle|\Q^{i_{k_2+k_1}}\cdots \Q^{i_{k_1+1}}
\middle|\Q^{i_{k_1}}\cdots \Q^{i_1}\right]
x\]
where:
\begin{itemize}
\setlength{\parindent}{.25in}
\item $x\in X_{a_n,\ldots,a_1}$ is a homogeneous element of $X$;
\item $\textbf{k}=(k_r,k_{r-1},\ldots,k_{1})$ is a sequence of positive integers with sum $\ell$;
%\item $0<k_1<k_2<\cdots <k_{r-1}<\ell$ is increasing, so that each bar is nonempty;
\item $I=(i_\ell,\ldots,i_1)$ is a sequence of nonnegative integers with $\minDim(I)<a_n$; and
\item if all of $a_{n-1},\ldots,a_2$ equal zero, $0$ does not appear in $I$.
\end{itemize}
%As Adem relations may be applied between each pair of bars, 
A basis of $\Boverline_r$ consists of all the above monomials in which each of the $r$ distinguished subsequences of $I$ is $\Q$-admissible, as $x$ runs through a homogeneous basis for $X$. The formula for the differential is the standard one, making %under the $\ell$ grading on each $\Boverline_r$ makes
 $\Boverline_*$ a chain complex with increasing filtration, $F_p\Boverline_r=\bigoplus_{\ell\leq p}\Boverline_{r,\ell}$. 
\end{prop}
\noindent We'll say that a monomial $\BarMonomial{I}{\textbf{k}}{x}\in\Boverline_r$ is bar-admissible if each of the $r$ distinguished subsequences of $I$ are $\Q$-admissible. Thus a basis for $\Boverline_{r,\ell}$ consists of the bar-admissible monomials on a basis of $X$. Note that $\Boverline_{r,\ell}=0$ for $r>\ell$, and that for any fixed element $x$ and any fixed $\ell$, the condition on $\minDim(I)$ implies that there are only finitely many monomials of the form $\BarMonomial{I}{\textbf{k}}{x}$ in $\Boverline_{r,\ell}$.
\begin{prop}\label{PriddyAlgProof}
Any cycle in $\Boverline_{r,\ell}$, for $\ell>r$ is homologous to a cycle in $\Boverline_{r,r}$.
\end{prop}
\begin{proof}
First, a construction. For any \emph{bar-admissible} monomial $\BarMonomial{I}{\textbf{k}}{x}\in\Boverline_{r,\ell}$ with $\ell>r$, choose the unique integer $e$ such that
the leftmost $e$ bars of $\BarMonomial{I}{\textbf{k}}{x}$ have length 1, but not the leftmost $e+1$ bars, 
% $k_r=k_{r-1}=\cdots =k_{r-e+1}=1<k_{r-e}$, 
so that
\[\BarMonomial{I}{\textbf{k}}{x}=\left[\Q^{i_\ell}\middle|\Q^{i_{\ell-1}} \middle|\Q^{i_{\ell-2}}\middle|\cdots \middle|\Q^{i_{\ell-e+1}}\middle|\Q^{i_{\ell-e}}\Q^{i_{\ell-e-1}}\cdots \middle|\cdots \right]x.\]
%That is, we've identified the length, $e$, of the longest initial segment of length one bars. 
Then define
\[\Gamma(\BarMonomial{I}{\textbf{k}}{x})=\begin{cases}
0,\textup{ if }(i_{\ell},\ldots,i_{\ell-e})\textup{ is not $\SqShift$-admissible};\\
\left[\Q^{i_\ell}\middle|\Q^{i_{\ell-1}} \middle|\Q^{i_{\ell-2}}\middle|\cdots \middle|\Q^{i_{\ell-e+1}}\middle|\Q^{i_{\ell-e}}\middle|\Q^{i_{\ell-e-1}}\cdots \middle|\cdots \right]x,\textup{ o.w.}
%\\,&\textup{if }
\end{cases}
\]
Thus $\Gamma$ either returns zero or adds a single bar to produce a new bar-admissible monomial in $\Boverline_{r+1,\ell}$. One may extend $\Gamma$ by linearity to produce linear operators $\Gamma:\Boverline_{r,\ell}\to\Boverline_{r+1,\ell}$ for $\ell>r$, and extend $\Gamma$ by zero on $\Boverline_{r,r}$ to produce $\Gamma:\Boverline_{r}\to\Boverline_{r+1}$.


Define a total preorder%footnote:
\footnote{A total preorder on a set $M$ is simply a transitive relation $\gtrsim$ on $M$ such that for any $m,m'\in M$, either $m\gtrsim m'$ or $m'\gtrsim m$.  One writes $m\sim m'$ whenever $m\gtrsim m'$ and $m'\gtrsim m$. This defines an equivalence relation on $M$, and $M/\sim$ is totally ordered. One writes $m\gnsim m'$ whenever $m\gtrsim m'$ but $m'\nsim m$, saying that $m$ is \emph{strictly greater} than $m'$.
} on the bar-admissible monomials by declaring%footnote:
\footnote{When comparing distinct words $I$ and $I'$ of equal length lexicographically, one writes $I\geq I'$ if $I$ has the larger entry in the leftmost position of disagreement.}
\[\BarMonomial{I}{\textbf{k}}{x}\gtrsim \BarMonomial{I'}{\textbf{k}'}{x'}\textup{ if either } \ell>\ell'\textup{ or }\ell=\ell'\textup{ and }I\geq I'\textup{ lexicographically}.\]
Then, as remarked in the proof of \cite[Thm 5.3]{PriddyKoszul.pdf}, a straightforward verification shows that for any bar-admissible monomial $m\in\Boverline_{r,\ell}$ with $\ell>r$, $\partial\Gamma(m)+\Gamma\partial(m)\equiv m$ modulo monomials strictly less than $m$.
%Now any element of $\Boverline_r$ can be written in the form $z=\sum_{j=1}^N \BarMonomial{I^{(j)}}{\textbf{k}^{(j)}}{x^{(j)}}$ for some nonincreasing sequence $\left\{\BarMonomial{I^{(j)}}{\textbf{k}^{(j)}}{x^{(j)}}\right\}$ of monomials. Choose the largest $M$ such that $$

Now suppose that $z\in Z\Boverline_r$ is a cycle which is not already in $\Boverline_{r,r}$. Then write 
$z=a+b$, where $a=\sum_i a_i$ and $b=\sum_j b_j$ are sums of bar-admissible monomials, such the $a_i$ are equivalent under $\gtrsim$, and $a_i\gnsim b_j$ for all $i$ and $j$. Then
\[z+\partial\Gamma(a)\equiv z+\partial\Gamma(z)\equiv\Gamma(\partial z)=\Gamma(0)=0\] where the congruences are modulo monomials strictly less than the $a_i$.
Thus, $z+\partial\Gamma(a)$ is a homologous cycle in which every monomial appearing (when written as a sum of bar-admissibles) is strictly less than the $a_i$. We may iterate this process, reducing the largest equivalence class of monomials appearing in $z$ at each step. As there are only finitely many possible sequences $I$ appearing, eventually a cycle in $\Boverline_{r,r}$ is reached.
\end{proof}

\begin{prop}\label{propDerivedIndTrivialUobject}
Suppose that $X\in\nontop{n}$ is trivial, so that all of the $\Q$ operations act as zero. Then %the differential $\partial $ of $\Boverline_r$ respects the $\ell$ grading, so that 
$(\derived_r\Ind{\nontop{}})X$ has a basis $\{\Q(I)x\}$, where
\begin{itemize}
\setlength{\parindent}{.25in}
\item $x\in X_{a_n,\ldots,a_1}$ runs through the elements $x$ of a chosen basis of $X$;
\item $I=(i_r,\ldots,i_1)$ is a $\SqShift$-admissible sequence with $\minDim(I)<a_n$; and
\item if all of $a_{n-1},\ldots,a_2$ equal zero, $0$ does not appear in $I$. %\textbf{ie i1 isn't zero.}
\end{itemize}
That is, the sequences involved are those $\SqShift$-admissible $I$ such that $Q^Ix$ is defined in $X\in\nontop{n}$.
These classes are defined by the following sum taken over certain sequences $J=(j_{r},\ldots,j_1)$ of non-negative integers:
\[\Q(I)x:=\left[\Q^{i_r}\middle|\cdots\middle|\Q^{i_1} \right]x+\sum_{\produces{J}{I}{\SqShift}}\left[\Q^{j_r} \middle|\cdots\middle|\Q^{j_1} \right]x.\]%\textup{ (sum over nonnegative sequences $J=(j_{r},\ldots,j_1)$)}\]
\end{prop}
\noindent Lemma \ref{lemOnAdemChangeInM} demonstrates that if $I$ and $J$ are sequences of non-negative integers (of any length), such that $\produces{J}{I}{\SqShift}$, then $\minDim(J)\leq\minDim(I)$, and if $0$ appears in $J$, it must also appear in $I$. Thus, all the terms appearing in the above definition of $\Q(I)x$ are indeed valid in the bar construction.
\begin{proof}
As $X$ is trivial, the differential of $\Boverline_r$ respects the $\ell$ grading, so that even the derived functors $(\derived_r\Ind{\nontop{}})X$ have the extra $\ell$ grading. By the previous proposition, they are concentrated along the diagonal $r=\ell$.

It appears to be easier at this stage to work with the duals $\overline{C}^{r,\ell}$ of the $\Boverline_{r,\ell}$, and determine the cokernel of $\delta:\overline{C}^{r-1,r}\to\overline{C}^{r,r}$. There is no danger in dualizing twice: as $X$ is trivial, it is the union of its finite dimensional subobjects, so that we may assume that $X$ is finite dimensional. Fix a basis $\{x\}$ of $X$. %, inducing a dual basis $\{x^*\}$ of $X^*$. 
Then, we have a basis $\{\BarMonomial{I}{\textbf{k}}{x}^*\}$ for $\overline{C}^{*,*}$ dual to the above basis for $\Boverline_{*,*}$.

Now $\delta:\overline{C}^{r-1,r}\to\overline{C}^{r,r}$ may be described as follows. A basis for $\overline{C}^{r-1,r}$ is given by duals of bar-$\Q$-admissible monomials of the form
\[\left(\left[\Q^{i_r}\middle|\cdots\middle|\Q^{i_{m+2}}  \middle|\Q^{i_{m+1}}\Q^{i_m}\middle|\Q^{i_{m-1}}\middle|\cdots \middle|\Q^{i_1} \right]x\right)^*\]
Under $\delta$, such a basis element maps to
\[
\left(\left[\Q^{i_r}\middle|\cdots \middle|\Q^{i_1} \right]x\right)^*
+
\sum_{\produces{(\alpha,\beta)}{(i_{m+1},i_m)}{\Q}}\left(\left[\Q^{i_r}\middle|\cdots\middle|\Q^{i_{m+2}}  \middle|\Q^{\alpha}\middle|\Q^{\beta}\middle|\Q^{i_{m-1}}\middle|\cdots \middle|\Q^{i_1} \right]x\right)^*\]
We write this formula in the cobar construction, which is naturally the \emph{quotient} of the ``stable'' cobar construction $C(\F_2,\DyerLashov,X)$ by those cobars not satisfying the unstable condition. Thus, in the above expression, any summand with $\minDim(I)\geq|x|$, or which fails the condition on $\Q^0$, will naturally vanish.
%
%Where, if $x\in X_{a_n,\ldots,a_1}$:
%\begin{itemise}
%\setlength{\parindent}{.25in}
%\item we omit those summands in which $\minDim(i_{r},\ldots,i_1)\geq a_n$; and
%\item if all of $a_{n-1},\ldots,a_2$ are zero, we omit any summands with $i_m=0$ or $i_{m+1}=0$.
%\end{itemise}
The term with $(i_{m+1},i_m)=(\alpha,\beta)$ does not vanish in this way. Moreover, for length 2 sequences $K$ and $L$, as $\Q$ and $\SqShift$ are Koszul dual, $\produces{K}{L}{\Q}$ if and only if $\produces{L}{K}{\SqShift}$. Thus, we obtain a relation on $H^{r,r}\overline{C}$:
\[\left(\left[\Q^{i_r}\middle|\cdots \middle|\Q^{i_1} \right]x\right)^*
=
\sum_{\produces{(i_{m+1},i_m)}{(\alpha,\beta)}{\SqShift}}\left(\left[\Q^{i_r}\middle|\cdots\middle|\Q^{i_{m+2}}  \middle|\Q^{\alpha}\middle|\Q^{\beta}\middle|\Q^{i_{m-1}}\middle|\cdots \middle|\Q^{i_1} \right]x\right)^*\]
where we may again choose to omit certain terms on the right hand side. Thus, the image of $\delta$ encodes precisely the $\SqShift$-Adem relations applied to the various cobars in $\overline{C}^{r,r}$. %the relation here is simply the $\SqShift$-Adem relation.
A basis of $H^{r,r}\overline{C}$ is then represented by the cocycles $\BarMonomial{I}{\textbf{k}}{x}^*$, where $I$ and $x$ are as in the statement of this proposition, and $\textbf{k}=(1,\ldots,1)$.

As long as each of the proposed basis elements $\Q(I)x$ are cycles, they will form the dual basis for $H_{rr}\Boverline$. %, as in the sum defining $\Q(I)x$, the only $\SqShift$-admissible $J$ appearing is $J=I$. 
In order to check that $\Q(I)x$ is a cycle, one may check that it pairs to zero with the image of $\delta$, i.e.\ with the $\SqShift$-Adem relations displayed above. But that is not hard to see. By definition of $\Q(I)x$, the functional `pair with $\Q(I)x$' on $\overline{C}^{r,r}$ \emph{equals} the functional `reduce to $\SqShift$-admissible form using $\SqShift$-Adem relations, and pair with $\left[\Q^{i_r}\middle|\cdots\middle|\Q^{i_1} \right]x$'. This second functional obviously returns zero on the $\SqShift$-Adem relations.
\end{proof}
Now we are in a position to describe a complex calculating $(\derived\Ind{\nontop{}})X$. For any $X\in\nontop{n}$, define the \emph{Koszul complex} $\Koverline_*X$, for $X\in\nontop{n}$, to be the subspace of $\Boverline_*X$ spanned by the classes $\Q(I)x$, for $I$ and $x$ as in proposition \ref{propDerivedIndTrivialUobject}.
\begin{prop}\label{KoszulComplexN>2}
%Suppose instead that $X$ is any object in $\nontop{n}$, 
The Koszul complex $\Koverline_*X$ is a subcomplex of $\Boverline_*X$, whose inclusion is a homology equivalence. The differential of $\Koverline_*X$ is given by the following formula:
\[\partial(\Q(I)x)=\Q((i_{\ell},\ldots,i_2))(\Q^{i_1}x)+ \!\!\!\!\!\!\!\!\!\!\!\!\!\!\!\!\!\sum_{\substack{\produces{J}{I}{\SqShift}\\(j_{\ell},\ldots,j_2)\in\admis{\SqShift}}}\!\!\!\!\!\!\!\!\!\!\!\! \Q((j_{\ell},\ldots,j_2))(\Q^{j_1}x),\]
where the sum is taken over those $J=(j_{\ell},\ldots,j_1)$ such that $(j_{\ell},\ldots,j_2)$ is $\SqShift$-admissible, and yet $\produces{J}{I}{\SqShift}$.
\end{prop}
\begin{proof}

Consider the spectral sequence associated to the filtered complex $\Boverline_*X$, indexed so that $E^0_{r,\ell}=\Boverline_{r,\ell}(X)$ (as vector space), and $d^q:E^q_{r,\ell}\to E^q_{r-1,\ell-q}$. In particular, each $E^q_{r,\ell}$ is concentrated where $\ell\geq r$. Now the chain complex $E^0$ can be identified with $\Boverline_{r,\ell}(KX)$, where $KX$ is a trivial object of $\nontop{n}$ on the same underlying vector space as $X$. Thus, $E^1$ is concentrated along the line $r=\ell$, and there are natural inclusions of vector spaces $E^1_{rr}\to E^0_{rr}\to \Boverline_r$, whose composite is a map of complexes which is a homology equivalence. One is left to note that this inclusion is modeled by the inclusion of the subspace of $\Boverline_*$ spanned by the $\Q(I)x$.




%The inclusion $\Koverline_*X\to \Boverline_*X$ is exactly

%The spectral sequence associated to the filtration on $\Boverline_*X$ has $E^0$ the bar construction $\Boverline_{r,\ell}(KX)$, with $d^0$ its differential, where $KX$ is a trivial object of $\nontop{n}$ on the same underlying vector space as $X$. Precisely, $E^0_{r,\ell}=\Boverline_{r,\ell}(KX)$. %The differential $d^0$ implements the differential on this reduced bar construction, so that
%Thus, $E^1_{r,\ell}=(\derived_{r,\ell}\Ind{\nontop{}})(KX)$ is  zero except when $r=\ell$, so that $d^1$ is the last nonzero differential, after which the spectral sequence collapses.

The formula for $d^1:E^1_{r,r}\to E^1_{r-1,r-1}$ is simply:
\[\sum_j \left[\Q^{i_r^{(j)}}\middle|\cdots \middle|\Q^{i_1^{(j)}}\right]x^{(j)}
\mapsto
\sum_j \left[\Q^{i_r^{(j)}}\middle|\cdots \middle|\Q^{i_2^{(j)}}\right]\left(\Q^{i_1^{(j)}}x^{(j)}\right)\]
Now as $E^1_{r,r}\subseteq E^0_{r,r}$, in order to write this in terms of the $\Q(I)x$ we need only focus on the terms involving $\SqShift$-admissible sequences, and the formula for $\partial$ follows.
\end{proof}


\end{KoszulComplexes_n>1}

\begin{KoszulComplexes1}
\subsection{Koszul complexes for $\derived\Ind{\nontop{1}}$}
We'll record here the changes to the above required when $n=1$. As before, for any object $X\in\nontop{1}$, write $\Boverline_*:= \ModDegeneracies{\Ind{\nontop{1}}(\BarConst{\nontop{1}}X)}_*$.

%\textbf{remove} For $I=(i_\ell,\ldots,i_1)$ a sequence of integers with each $i_j\geq2$, define
%\[\minDimP(I):=\begin{cases}
%-\infty,&\textup{if }I=\emptyset;\\
%\max\{(i_1),\,(i_2-i_1-1)\,(i_3-i_2-i_1-2),\,\ldots,\,(i_{\ell}-i_{\ell-1}-\cdots i_1-\ell+1)\},&\textup{otherwise}.
%%\\,&\textup{if }
%\end{cases}
%\]
%Thus, for $I\neq\emptyset$, $\minDimP(I)$ is the lowest dimension $d\geq0$ such that $P^Ix$ may be nonzero, for $x$ a $d$-dimensional element of an object in either $\LL{1}$ or $\nontop{1}$.




\begin{prop}
As a vector space each $\Boverline_r$ is graded, with $\Boverline_{r,\ell}$
%For $r\geq0$, $\Boverline_r(\forgetSymbol,\Fr{\nontop{n}},X)$ is 
spanned by monomials
\[\BarMonomial{I}{\textbf{k}}{x}:=\left[P^{i_{k_r+\cdots +k_1}}\cdots P^{i_{k_{r-1}+\cdots +k_1+1}}
\middle|\cdots 
\middle|P^{i_{k_2+k_1}}\cdots P^{i_{k_1+1}}
\middle|P^{i_{k_1}}\cdots P^{i_1}\right]
x\]
where:
\begin{itemize}
\setlength{\parindent}{.25in}
\item $x\in X_{a_1}$ is a homogeneous element of $X$;
\item $\textbf{k}=(k_r,k_{r-1},\ldots,k_{1})$ is a sequence of positive integers with sum $\ell$; and
\item $I=(i_\ell,\ldots,i_1)$ is a sequence of integers with $i_j\geq2$ and $\minDimP(I)\leq a_1$;
\end{itemize}
A basis of $\Boverline_r$ consists of all the above monomials in which each of the $r$ distinguished subsequences of $I$ is $P$-admissible, as $x$ runs through a homogeneous basis for $X$. The formula for the differential is the standard one, making %under the $\ell$ grading on each $\Boverline_r$ makes
 $\Boverline_*$ a chain complex with increasing filtration, $F_p\Boverline_r=\bigoplus_{\ell\leq p}\Boverline_{r,\ell}$. 
\end{prop}
Note that in $\Boverline_*$, any $I$, $\textbf{k}$ and $x$ will produce a well-defined monomial; those violating the inequality $\minDimP(I)\leq a_1$ are simply zero, rather than undefined.

\begin{prop}
Suppose that $X$ is any object in $\nontop{1}$, the Koszul complex $\Koverline_*X$ is the subcomplex of $\Boverline_*X$ with basis the classes $P(I)x$, where 
\begin{itemize}
\setlength{\parindent}{.25in}
\item $x\in X_{a_1}$  runs through the elements $x$ of a chosen basis of $X$; and
\item $I=(i_r,\ldots,i_1)$ a $\delta$-admissible sequence with $i_j\geq2$ and $\minDimP(I)\leq a_1$.
\end{itemize}
and where 
\[P(I)x:=\left[P^{i_r} \middle|\cdots\middle|P^{i_1} \right]x+\sum_{\substack{\produces{J}{I}{\delta} \\ \minDimP(J)\leq|x|}}\left[P^{j_r} \middle|\cdots\middle|P^{j_1} \right]x.\]
The inclusion of $\Koverline_*X$ into $\Boverline_*X$ is a homology equivalence, the restriction of the differential to $\Koverline_*X$ is given by the following formula:
\[\partial(P(I)x)=P((i_{\ell},\ldots,i_2))(P^{i_1}x)+\sum_{\substack{\produces{J}{I}{\delta} \\ \minDimP(J)\leq|x|}} P((j_{\ell},\ldots,j_2))(P^{j_1}x),\]
where the sum is taken over those $J=(j_{\ell},\ldots,j_1)$ such that $(j_{\ell},\ldots,j_2)$ is $\delta$-admissible, and $\minDimP(J)\leq|x|$.
\end{prop}
\begin{proof}
The proof of this proposition is essentially the same as the proof of propositions \ref{propDerivedIndTrivialUobject} and \ref{KoszulComplexN>2}: one uses the same algorithm to compress to the diagonal, and the same spectral sequence. There are two main differences between the two cases:
\begin{enumerate}\squishlist
\setlength{\parindent}{.25in}
\item In the $n\geq2$ case, an element of $\nontop{n}$ supports only \emph{non-top} $\Q$ operations, so that the various constraints on $\minDim(I)$ were strict. On the other hand, objects of $\nontop{1}$ support \emph{all} the $P$ operations, so that the inequalities on $\minDimP(I)$ are not strict.
\item In the $n\geq2$ case, the bar construction was the subobject of a stable bar construction, and the cobar was a quotient. Thus, the formula for the differential on $\Koverline_*$ didn't contain any extraneous terms, but our interim calculation with the cobar construction did. When $n=1$, this situation is reversed, so that certain summands in the natural description of the differential on $\Koverline_*$ are zero. This explains why we may add the condition ``$\minDimP(J)\leq |x|$'' on the sums of this proposition, while no such condition was appropriate in proposition \ref{KoszulComplexN>2}.\qedhere
\end{enumerate}
%
%Note that in the $n=1$ case, there are no summands to omit in the formula for $\delta:\overline{C}^{r-1,r}\to\overline{C}^{r,r}$. [Really, in the $n\geq2$ case, the formula for $\delta$ lands in the quotient by inadmissible terms, while in the $n=1$ case, $\delta$ lands in the subcomplex of admissible terms. This is, of course, dual to the situation in the Bar construction or in the free $\nontop{n}$ construction.] For example, in the $n=1$ case, if $\produces{J}{I}{\delta}$, then $\minDimP(J)\geq\minDimP(I)$, which is the opposite to the inequality in the $n\geq2$ case. However, all of the above summands are defined, even those with $\minDimP(J)>|x|$ --- it's just that those summands are zero, due to the unstable condition for $P$. Thus, when in the above, we include a second subscript on the sums, it's just that the sum would contain extra terms, all automatically zero, if we didn't.
\end{proof}
\end{KoszulComplexes1}


\begin{DerivedFunctorsLowDimension}
\section{Lie algebra homology and low dimensional calculations}
\subsection{Calculating $\derived\Ind{\LL{}}$ in internal dimension zero}
Suppose that $n\geq2$. Then the dimension zero part of an object $X$ of $\LL{n}$ is a partially restricted Lie algebra, in $\PRLie{n-1}$. That is: 
\begin{lem}
For $n\geq2$, there is a `zeroth part' functor $\LL{n}\to\PRLie{n-1}$, written $X\mapsto X_0$, where $X_0\in\PRLie{n-1}$ is defined by
\[(X_0)_{a_{n-1},\ldots,a_1}:=X_{0,a_{n-1},\ldots,a_1}\]
with bracket and restrictions on $X_0$ inherited from those on $X$. Equivalently, the restriction on $X_0$ is given by the operator $\Q^0$ on $X$. Moreover, taking zeroth parts commutes with the two free constructions, which is to say that for $S\in\GS{n}$, we have $(\Fr{\LL{n}}S)_0\cong\Fr{\PRLie{n-1}}(S_0)$, writing $S_0\in\GS{n-1}$ for the zeroth part of $S$.
\end{lem}
\begin{proof}
As objects of $\LL{n}$ are themselves objects of $\PRLie{n}$, with restriction given by $\Q^0$ in dimension zero, it's enough just to check that $\Q^0$ is defined on $X_{0,a_{n-1},\ldots,a_1}$ whenever $\restn{}$ is supposed to be defined on a dimension $(a_{n-1},\ldots,a_1)$ element of an object of $\PRLie{n-1}$. This follows from the definition.

To check that the free constructions agree, recall (\ref{PropFreeKandUconstructions}) that $\Fr{\LL{n}}S$ has basis
\[{\Fr{\LL{n}}S}
=
\left\langle \Q^Ib\,:\,\genfrac{}{}{0pt}{}{b\in B_{a_n,\ldots,a_1},\,I\in\admis{\Q},\ i_1\leq|b|,}{\textup{if }a_{n-1}\!=\!\cdots\!=\!a_2\!=\!0\textup{ then $I$ doesn't contain 0}}\right\rangle\]
%\[\{\Q^Ib\,:\,b\in B_{a_{n},\ldots,a_1},\,I\in\admis{\Q},\ i_1\leq|b|,\ i_1,a_{n-1},\ldots,a_2\textup{ not all zero}\}\] this was wrong
where $b$ runs over a homogeneous Hall basis, $B$. The only basis elements $\Q^Ib$ in the zeroth part $(\Fr{\LL{n}}S)_0$ are those where $|b|=0$ and in which $I$ only contains zeros. These classes are exactly the basis for $\Fr{\PRLie{n-1}}(S_0)$ in \ref{PropBasesOfFreeLieAlgs}.
\end{proof}
\begin{prop}
There is a natural isomorphism
\[((\derived_{p}\Ind{\LL{}})X)_{0,a_{n-1},\ldots,a_1}\cong ((\derived_p\Ind{\PRLie{n-1}})X_0)_{a_{n-1},\ldots,a_1}\]
\begin{proof}
Consider the bar construction used to define $((\derived_{p}\Ind{\LL{}})X)_{0,a_{n-1},\ldots,a_1}$, constructed by repeated application of the free ${\LL{n}}$ functor. In dimension zero, this amounts to repeated application of the free $\PRLie{n-1}$ functor, and we obtain the bar construction for $X_0$ as an object of $\PRLie{n-1}$.
\end{proof}
\end{prop}
%\subsubsection{$\derived\Ind{\LL{1}}$}
\begin{shaded}
Choose $n\geq 1$, and define 
\[d=\begin{cases}
4,&\textup{if }n=1;\\
1,&\textup{if }n=2;
\\0,&\textup{if }n\geq3.
\end{cases}
\]
We will here present a method to calculate the object $(\derived\Ind{\LL{}})X=((\derived\Ind{\LL{}})X)_{a_{n+1},\ldots,a_1}$ in dimensions with $a_n\leq d$. Consider the diagram:
\[\xymatrix@R=2mm@C=13mm{
&%r1c1
\ar[dd]^-{\Ind{\nontop{}}}
\ar@<.5ex>[dl]%^-{\forgetSymbol}
\LL{n}\\%r1c2
\ar@<.5ex>[ur]^-{\Fr{\LL{n}}}
\ar@<.5ex>[dr]^-{\Fr{\PRLie{n}}}
\GR{n}&%r2c1
\\%r2c2
&%r3c1
\PRLie{n}
\ar@<.5ex>[ul]%^-{\forgetSymbol}
%r3c2
}\]
In this diagram, $\Ind{\nontop{}}$ does not change the underlying vector space in dimensions $(a_n,\ldots,a_1)$ with $a_n\leq d$. Moreover, the free constructions $\Fr{\LL{n}}$ and $\Fr{\PRLie{n}}$ agree in these dimensions, as do the structure maps of the free/forget adjunctions. Thus, there is an isomorphism between $(\derived\Ind{\LL{}})(X)$ and $(\derived\Ind{\PRLie{}})(\Ind{\nontop{}}(X))$ in dimensions with $a_n\leq d$. [Note that these are objects of $\GR{n+1}$, and so have gradings $a_{n+1},\ldots,a_1$ --- our inequality involves $a_n$.] %, the outermost grading of an object of $\LL{n}$.]
\end{shaded}

There is a complex that computes $\derived\Ind{\PRLie{}}$ --- it interpolates between the Cartan-Eilenberg resolution for the homology of Lie algebras, and May's $\overline{X}$ complex for the homology of restricted Lie algebraa.
\end{DerivedFunctorsLowDimension}

\begin{PRlieKoszulComplexCalculationOriginalWithSSeq}
\subsection{Partially restricted Lie algebra homology}
We're interested in derived abelianization of partially restricted Lie algebras. For a PR Lie algebra $L$, write $L^r$ for the ideal of restrictable elements and $L^u$ for the linear subspace of unrestrictable elements (which is well defined as long as restrictability is defined by a degree condition). Write $UL$ for the PR universal enveloping algebra of $L$ --- this is the initial associative algebra with a PRL map from $L$. To make sense of partial restrictions, recall that restrictability is always defined by nonvanishing of certain gradings.

\begin{lem}
The prolonged functor $U:s\mathsf{PRL}\to s\mathsf{Ass}$ preserves weak equivalences.
\end{lem}
\begin{proof}
Suppose that $X_\bullet\to Y_\bullet$ is a weak equivalence of simplicial PRLs. The Lie filtration makes $UX_\bullet\to UY_\bullet$ a map of filtered simplicial algebras, and so there is an induced map of the resulting spectral sequences. The $E^0$ page of the spectral sequence for $UX_\bullet$ is the algebra $\F_2[X_\bullet^u]\otimes E[X_\bullet^r]$, and $d_0$ is induced by the simplicial structure maps. By Dold's theorem, the $E^1$ page is then a functor of $\pi_*(X_\bullet^u)$ and $\pi_*(X_\bullet^r)$. As the induced maps $\pi_*(X_\bullet^u)\to\pi_*(Y_\bullet^u)$ and $\pi_*(X_\bullet^r)\to\pi_*(Y_\bullet^r)$ are isomorphisms, the map of spectral sequences is an isomorphism from $E^1$.
\end{proof}
%\begin{lem}
%The functor $\overline{W}:s\mathsf{Ass}\to$
%\end{lem}

\begin{lem}
If $X=\Fr{PRL}(S)$ is a free PRL, then $UX$ is the tensor algebra $T(S)$ on $S$.
\end{lem}
\begin{proof}
Firstly, there is a unique PRL map from $X\to T(S)$ which is the identity on $S$. Now suppose that $\psi:X\to A$ is a PRL map, for some associative algebra $A$. There is a unique algebra map $\overline{\psi}:T(S)\to A$ sending $s\in S$ to $\psi(s)$. This map makes the triangle commute, as $X$ is free on $S$.
\end{proof}

\begin{prop}
There is a natural isomorphism $\derived_q QX\cong \Tor^{UX}_{q+1}(\F_2,\F_2)$, for $X$ a partially restricted Lie algebra.
\end{prop}
\begin{proof}
The proof, analogous to the proof of \cite[3.5]{PriddySimplicialLie.pdf}, depends on spectral sequences (a) and (b) from \cite[II.6.6]{QuillenHomAlg.pdf}. Let $L_\bullet$ be an almost free simplicial replacement for $X$. Then spectral sequence (a) is of the form
\[E^2_{pq}=\pi_p(\Tor^{UL_p}_q(\F_2,\F_2))\implies \pi_{p+q}(K(\F_2,0)\otimes_{UL}^{L}K(\F_2,0))\]
Now the map $K(\F_2,0)\otimes_{UL}^{L}K(\F_2,0) \to K(\F_2,0)\otimes_{UX}^{L}K(\F_2,0)$ induced by the weak equivalence $UL_\bullet\to UX$ is a weak equivalence, by a comparison argument using spectral sequence (b). Thus, the target is isomorphic $\Tor^{UX}_{p+q}(\F_2,\F_2)$.

On the other hand, $E^2_{pq}$ is easy to calculate. As $L_p$ is a free PRL for each $p$, $UL_p$ is the tensor algebra on $Q(L_p)$. Thus
\[\Tor^{UL_p}_q(\F_2,\F_2)=\begin{cases}
\F_2,&\textup{if }q=0;\\
Q(L_p),&\textup{if }q=1;\\
0,&\textup{o.w. }
\end{cases}
\]
Thus, the spectral sequence degenerates to the claimed isomorphism.
\end{proof}

\subsection{The Cartan-Eilenberg-May complex for partially restricted Lie algebra homology}
The PBW theorem states that $E^0VL=A(L^u)\otimes E(L^r)$, and the homology of this algebra (see \cite[\S7]{PriddyKoszul.pdf}) is thus $E(L^u)\otimes \Gamma(L^r)$. We'll think of this as the sub-coalgebra of $\Gamma(L)$ in which there are no $\gamma_i$ for $i>1$ applied to unrestrictable elements. It's the shuffle product that produces representatives in $\Boverline(E^0V(L))$ for classes in $E(L^u)\otimes \Gamma(L^r)$, and in order to calculate with these representatives, we first introduce some notation.
\begin{itemize}
\setlength{\parindent}{.25in}
\item $\Z^n:=\{\textbf{z}=(z_1,\ldots,z_n);\ z_i\in\Z\}$
\item $r_i:\Z^n\to\Z^n$ given by $(z_1,\ldots,z_n)\mapsto(z_1,\ldots,z_i-1,\ldots,z_n)$
\item $r_{ij}:=r_i\circ r_j:\Z^n\to\Z^n$
\item $\sigma:\Z^n\to\Z$ given by summing entries
\item for $\textbf{z}\in\Z^n$ write $\calP(\textbf{z})$ for the set of permutations of the multiset $\{i^{z_i};i=1,\ldots,n\}$, written as sequences $\textbf{p}=(p_1,\ldots,p_{\sigma \textbf{z}})$. This set is empty if $\textbf{z}$ has a negative entry. For such a $\textbf{p}$, write $p\sigma_i$ for $(p_1,\ldots,p_{i-1},p_{i+1},p_{i},p_{i+2},\ldots,p_{\sigma \textbf{z}})$.
\item $d_k:\calP(\textbf{z})\to\calP((r_{p_kp_{k+1}}\textbf{z})*(1))$ (the star denotes concatenation), in which $(p_1,\ldots,p_{\sigma \textbf{z}})$ is sent to $(p_1,\ldots,p_{k-1},\ell(\textbf{z})+1,p_{k+2},\ldots,p_{\sigma \textbf{z}})$. That is, two entries are removed, and in their position we place a single entry, $\ell(\textbf{z})+1$.
\end{itemize}
\begin{prop}
Choose $\textbf{z}\in\Z^n$, and $i$ satisfying $1\leq i\leq n$. Then as $k$ ranges from $1$ to $\sigma \textbf{z}-1$, and $\textbf{p}$ ranges through those elements of $\calP(\textbf{z})$ satisfying $(p_k,p_{k+1})=(i,i)$, then $d_kp$ assumes each value in $\calP((r_{ii}\textbf{z})*(1))$ exactly once.
\end{prop}
\begin{prop}
Choose $\textbf{z}\in\Z^n$, and $i\neq j$ satisfying $1\leq i,j\leq n$. Then as $k$ ranges from $1$ to $\sigma \textbf{z}-1$, and $\textbf{p}$ ranges through those elements of $\calP(\textbf{z})$ satisfying $(p_k,p_{k+1})=(i,j)$, then $d_k\textbf{p}$ assumes each value in $\calP((r_{ij}\textbf{z})*(1))$ exactly once. This occurs symmetrically in $i$ and $j$, which is to say that for such a pair $(k,\textbf{p})$, the pair $(k,\textbf{p}\sigma_k)$ works with $i$ and $j$ swapped, and we have $d_k(\textbf{p}\sigma_k)=d_k(\textbf{p})$.
\end{prop}


We can respresent a product $\prod_{i=1}^{n}\gamma_{z_i}(x_i)$ in $E(L^u)\otimes \Gamma(L^r)$ by a class $\left\langle \textbf{z},\textbf{x}\right\rangle$ in $\Boverline(E^0V(L))$ which we now define:

\begin{itemize}
\setlength{\parindent}{.25in}
\item $(hL)^n:=\{\textbf{x}=(x_1,\ldots,x_n);\ x_i\textup{ homogeneous in }V(L)\}$
%\item $d_{ij}:(hL)^n\to(hL)^{n+1}$ given by $\textbf{x}\mapsto\textbf{x}*(x_ix_j)$
%\item $b_{ij}:(hL)^n\to(hL)^{n+1}$ given by $\textbf{x}\mapsto\textbf{x}*([x_i,x_j])$
\item Choose $\textbf{x}\in(hL)^n$, $\textbf{z}\in\Z^n$ and $\textbf{p}\in\calP(\textbf{z}).$ Then define $\left\langle \textbf{p},\textbf{x}\right\rangle\in\Boverline_{\sigma\textbf{z},*}(E^0V(L))$ to be the bar $[x_{p_1},\ldots,x_{p_{\sigma\textbf{z}}}]$. Then define $\left\langle \textbf{z},\textbf{x}\right\rangle:= \sum_{\textbf{p}\in\calP(\textbf{z})}\left\langle \textbf{p},\textbf{x}\right\rangle$. 
\end{itemize}
Now $\Koverline_*(V(L)):=H_{*=*}(E^0V(L))$ is spanned by the bars $\left\langle \textbf{z},\textbf{x}\right\rangle$ where each $x_i$ is a homogeneous element of $L$, which must be restrictable if $z_i>1$ (here, $\sigma\textbf{z}$ is the homological degree). In fact, there's a basis consisting of the $\left\langle \textbf{z},\textbf{x}\right\rangle$ where $\textbf{x}$ runs over (strictly) increasing sequences of elements of an ordered basis chosen for $L$, and the same restrictability condition holds.

Finally we may calculate the differential on $\Koverline_*(V(L))$. Here goes:
\begin{alignat*}{2}
d\left\langle \textbf{z},\textbf{x}\right\rangle
&=
\sum_{k=1}^{\sigma\textbf{z}-1} \sum_{\textbf{p}\in\calP(\textbf{z})}d_k\left\langle \textbf{p},\textbf{x}\right\rangle%
\\
%&=
%\sum_{k}\sum_{\textbf{p}}\left\langle d_k\textbf{p},d_{p_kp_{k+1}}\textbf{x}\right\rangle%
%\\
&=
\sum_{i=1}^{n}\sum_{j=1}^{n}\sum_{k}\sum_{\textbf{p}:{p_k=i\atop p_{k+1}=j}}\left\langle d_k\textbf{p},\textbf{x}*(x_{p_k}x_{p_{k+1}})\right\rangle%
\\
&=
\sum_{i}\sum_{k}\sum_{\textbf{p}:{p_k=i\atop p_{k+1}=i}}\left\langle d_k\textbf{p},\textbf{x}*(x_ix_i)\right\rangle+\sum_{i<j} \sum_{k}\sum_{\textbf{p}:{p_k=i\atop p_{k+1}=j}}\left\langle d_k\textbf{p},\textbf{x}*(x_ix_j+x_jx_i)\right\rangle%
\\
&=
\sum_{i}\sum_{k}\sum_{\textbf{p}:{p_k=i\atop p_{k+1}=i}}\left\langle d_k\textbf{p},\textbf{x}*\left(\restn{(x_i)}\right)\right\rangle+ \sum_{i<j}\sum_{k}\sum_{\textbf{p}:{p_k=i\atop p_{k+1}=j}}\left\langle d_k\textbf{p},\textbf{x}*\left([x_i,x_j]\right)\right\rangle%
\\
&=
\sum_{i}\sum_{\textbf{q}\in\calP((r_{ii}\textbf{z})*(1))}\left\langle \textbf{q},\textbf{x}*\left(\restn{(x_i)}\right)\right\rangle+\sum_{i<j} \sum_{\textbf{q}\in\calP((r_{ij}\textbf{z})*(1))}\left\langle \textbf{q},\textbf{x}*\left([x_i,x_j]\right)\right\rangle%
\\
% Left hand side
% Relation
&=
% Right hand side
\sum_{i}\left\langle (r_{ii}\textbf{z})*(1),\textbf{x}*\left(\restn{(x_i)}\right)\right\rangle+\sum_{i<j} \left\langle (r_{ij}\textbf{z})*(1),\textbf{x}*\left([x_i,x_j]\right)\right\rangle%
% Comment
\end{alignat*}
This is the obvious restriction of May's $\overline{X}$, see page 141 of ``The cohomology of restricted Lie algebras and of Hopf algebras'' (J.\ Alg.\ 3). We use the subspace with fewer $\gamma_i$, and the differential he gives makes sense on and preserves this subspace, and coincides with that here.

He writes the differential on $f=\prod_{i=1}^{n}\gamma_{z_i}(y_i)$ as
\[d(f)=\left(\sum_{i=1}^{n}f_iy_i\right)+ \sum_{i=1}^{n}f_{i,i}\gamma_1(\restn{(y_i)})+\sum_{1\leq i<j\leq n}f_{i,j}\gamma_1([y_i,y_j]).\]
Here, the parenthetical term vanishes in $\Koverline$ --- it's there in case we're using nontrivial coefficients.

\end{PRlieKoszulComplexCalculationOriginalWithSSeq}

\begin{Corestricted Lie coalgebras Executive Summary}
\subsection{Executive Summary}
I'm interested in chain-level structure on the derived primitives of a `corestricted Lie coalgebra' (in characteristic 2). By `corestricted Lie coalgebra' I mean a left comodule over the dual of the Lie operad ($\dualLieOperad$) which is concentrated in arity zero. More generally, let $R$ be any left $\dualLieOperad$-comodule. The structure which is described in your paper includes maps:
\[K(I,\dualLieOperad,R) \from B(I,\dualLieOperad,R) \from N(I,\dualLieOperad,R)\]
where $N(I,\dualLieOperad,R)$ is the normalized complex of $C(I,\dualLieOperad,R)$.
From now on I'll write $K$, $B$ and $N$ for the three complexes, and $C$ for the cosimplicial object $C(I,\dualLieOperad,R)$.

There are degree $-1$ maps $K\otimes K\to K$ and $B\otimes B\to B$ defined on treewise tensor products by joining two trees at a single extra vertex at the bottom, and labeling this vertex with the unique nonzero class in $\dualLieOperad(2)$. This defines an action of the commutative operad on K, but only an action of $\Omega$ on B, where by $\Omega$ we mean the free operad on a binary operation (in grading $-1$). The two operations are compatible under the projection $\Omega\to \CommOperad$.

We'll also be able to define an action of $\Omega$ on N, but it's not quite as easy. We have a map 
\[\psi: C^p\otimes C^p \to C^{p+1}\]
which is defined on levelwise tree modules by the same process as the maps on $K$ and $B$. One needs to add a new level for the new vertex. If $T$ swaps the factors of $C^p\otimes C^p$, then we have $\psi=\psi T$. Given $\psi$, one can construct a degree $-1$ map $N\otimes N \to N$ to be the composite
\[(N\otimes N)_{-p} \overset{\textup{AW}}{\to} C^p\otimes C^p \overset{\psi}{\to} C^{p+1}\]
of $\psi$ with the standard cosimplicial Alexander-Whitney map,
\[\textup{AW}(t_1\otimes t_2)=(\delta^{p}\cdots \delta^{s+1}t_1)\otimes (\delta^{s-1}\cdots \delta^{0}t_2)\textup{ if $t_1\in N_{-s}$ and $t_2\in N_{-(p-s)}$.}\]
It's easy to check that this map lands in $N_{-p-1}$, and satisfies a Liebnitz rule (as the class in $\dualLieOperad(2)$ is primitive). Moreover, the square
\[\xymatrix@R=4mm{
B\otimes B
\ar[d]
&%r1c1
\ar[d]
\ar[l]
N\otimes N\\%r1c2
B
&%r2c1
\ar[l]
N%r2c2
}\]
commutes, so that the three multiplications are all compatible. (The square commutes since the dual levelisation map is so destructive -- the $\delta^p\cdots \delta^{s+1}$ in the definition of $\textup{AW}$ introduces many summands, but the counitality of the left coaction means that all but the most trivial of these terms is sent to zero by dual levelization.)

Now specialise to the case when $R$ is a finite-dimensional corestricted Lie coalgebra $G$ (i.e.\ $R=G$ is concentrated in arity 0). I've checked in this case that the not only is $K(I,\dualLieOperad,G)$ equal as a vector space to May's complex for the cohomology of the restricted Lie algebra $G\dual$, but that the two are equal as DGAs. This shows that the composite
\[K(I,\dualLieOperad,G) \from B(I,\dualLieOperad,G) \from N(I,\dualLieOperad,G)\]
is a weak equivalence, even though $G$ is evidently not connected.

Something I'm unsure about is as follows. The cohomology of a restricted Lie algebra should posess Steenrod operations. I can write down candidates for these, using what seems to be a standard method.
For each $i\geq0$, we can form a chain map $\Sq^{i}:N(I,\dualLieOperad,R)\to N(I,\dualLieOperad,R)$ of grading $i$, as the sum of the following two composites:
\[\xymatrix@R=4mm{
&%r1c1
(N\otimes N)_{-2n}
\ar[dr]^-{\Delta_{n+1-i}}
&%r1c2
&%r1c3
\\%r1c4
N_{-n}
\ar[ur]^-{\textup{`$x\mapsto x\otimes x$'}}
\ar[dr]_-{\textup{`$x\mapsto x\otimes dx$'}}&%r2c1
+&%r2c2
N_{-n-i+1}(C\otimes C)
\ar[r]^-{\psi}
&%r2c3
N_{-n-i}\\%r2c4
&%r3c1
(N\otimes N)_{-2n-1}
\ar[ur]_-{\Delta_{n+2-i}}
&%r3c2
&%r3c3
%r3c4
}\]
Where by $\Delta_k$ we mean higher cosimplicial Alexander-Whitney maps, satisfying $d\Delta_k+\Delta_k d=\Delta_{k-1}+T\Delta_{k-1}T$, in the spirit of  \cite{DwyerHigherDividedSquares.pdf}.
It's a formal consequence of the defining property of the $\{\Delta_k\}$, and the fact that $\psi=\psi T$, that the map $\Sq^i$ is a chain map, and that $\Sq^{n+1}:N_{-n}\to N_{-2n-1}$ is equal to the squaring map.

The other properties are less clear to me at present. In particular, I haven't yet seen why this definition should give the same operations on homotopy as the definition one might obtain by lifting the $\Omega$-action to an $E_{\infty}$-action, and forming the standard construction. It's also not clear to me whether the standard Adem relations imply Adem relations between these operations.
\end{Corestricted Lie coalgebras Executive Summary}

\begin{PRlieKoszulComplexCalculation}

\subsection{Corestricted Lie coalgebras }
We'll define a corestricted Lie coalgebra to be a left comodule $G$ over the cooperad $\dualLieOperad$, which as a symmetric sequence is concentrated in degree zero. That is, a coalgebra for the comonad $S(\dualLieOperad)$. Instead of working simply with a corestricted Lie coalgebra, we'll work with a left $\dualLieOperad$-comodule $R$.
%The purpose of this section will be to identify the Koszul homology of such with the May complex for the cohomology of $G^$

Our goal will be to construct a commuting diagram
\[\xymatrix@R=4mm{
\CommOperad\circ \overline{K}^c(\dualLieOperad)\circ R
\ar[d]^-{\mu_{\CommOperad}\circ R}
&%r1c1
\Omega\circ \overline{B}(\dualLieOperad)\circ R
\ar@{->>}[l]_-{\rho\circ\pi\circ R}
\ar[d]^-{\sigma\circ R}
&%r1c2
\Omega\circ \overline{N}(\dualLieOperad)\circ R
\ar[d]^-{f}
\ar@{->>}[l]_-{\Omega\circ\oldphi\circ R}
\\%r1c3
\overline{K}^c(\dualLieOperad)\circ R
&%r1c1
\overline{B}(\dualLieOperad)\circ R
\ar@{->>}[l]_-{\pi\circ R}
&%r1c2
\overline{N}(\dualLieOperad)\circ R
\ar@{->>}[l]_-{\oldphi\circ R}
}\]
of dg symmetric modules, in which the three vertical maps are all structure maps for left modules over either the commutative operad $\CommOperad$ or the free operad $\Omega$ on a binary generator.

The objects in this diagram are defined as follows. The operad $\Omega$ is free on the dg symmetric sequence $(0,0,\F_2[\Sigma_2]\langle x\rangle,0,\ldots)$, where the generating binary operation $x$ has grading $-1$. The operad $\CommOperad$ is the commutative operator, with binary generator in degree $-1$.  This is isomorphic to the Koszul construction $\overline{K}^c(\dualLieOperad)$. Moreover, the projection $\pi:\overline{K}^c(\dualLieOperad)\to \overline{K}^c(\dualLieOperad)$ sends all non-binary trees to zero, and all binary trees to the generator of the relevant level of $\overline{K}^c(\dualLieOperad)$.

The cosimplicial cobar construction $\overline{N}(\dualLieOperad)\circ R=N(I,\dualLieOperad,R)$ is the  normalization (formed by taking the kernel of the codegeneracies) of the cosimplicial symmetric sequence $C^n(I,\dualLieOperad,R):=I\circ (\dualLieOperad)^{\circ n}\circ R$ formed using the left comodule structure of $R$ and the right comodule structure of $I$ given by the unique cooperad map $I\to \dualLieOperad$. We think of the normalization as living in negative degrees: $N_{-n}(I,\dualLieOperad,R)\subset C^n(I,\dualLieOperad,R)$.

Now $\phi:\overline{N}(\dualLieOperad)\to \overline{B}(\dualLieOperad)$ is the delevelization map. It sends all trees to zero that do not have exactly one non-unit vertex in each level (there are $n$ levels in a tree in $\overline{N}_{-n}(\dualLieOperad)$). For trees which are not sent to zero for this reason, $\phi$ strips away the nonunit vertices (and adds in the necessary desuspension operators) to obtain an element of $\overline{B}(\dualLieOperad)$.

All that remains is to define the maps $\sigma$ and $f$. For a symmetric sequence $S$, a left module structure map $\Omega\circ S\to S$ is equivalent to maps $S(n)\otimes S(m)\to S(n+m)$ for each $n,m\geq0$. If $S$ is a dg symmetric sequence, then $S$ becomes a dg module iff these maps are dg. Anyway, we let $\sigma$ correspond to the maps $\overline{B}(\dualLieOperad)(n)\otimes \overline{B}(\dualLieOperad)(m)\to \overline{B}(\dualLieOperad)(n+m)$ which take a pair of trees and attaches them to the entries of the `Y' shaped tree with $l\dual$ at the vertex (where $l\dual$ is the generator of $\dualLieOperad(2)$). As $l\dual$ is appropriately primitive, this pairing satisfies the Liebnitz rule.

It's a little trickier to define $f$. 
%: one uses the cosimplicial Alexander-Whitney map. Moreover,  That is, writing $N_{-n}=N_{-n}(I,\dualLieOperad,R)$, etc.:
There is a pairing
\[\psi:(C^p(I,\dualLieOperad,R))^{\otimes 2}\to C^{p+1}(I,\dualLieOperad,R)\]
which takes a pair of trees and attaches them to the entries of the `Y' shaped tree with $l\dual$ at the vertex. One must add a level for the new vertex, which causes the shift. Moreover, the map $\psi$ restricts to Moore complexes, as it preserves the property of `having at least one non-unit vertex in each level'. Thus we obtain a map
\[\psi:(N_{-p}(I,\dualLieOperad,R))^{\otimes 2}\to N_{-p-1}(I,\dualLieOperad,R)\]
 The left action of $\Omega$ on $N(I,\dualLieOperad,R)$ is then given by the composite
\[(N(I,\dualLieOperad,R)^{\otimes 2})_{-p}\overset{\textup{AW}}{\to} (N_{-p}(I,\dualLieOperad,R))^{\otimes 2}\overset{\psi}{\to}N_{-p-1}(I,\dualLieOperad,R).\]
By formal properties of the Alexander-Whitney map, and again as $l\dual$ is appropriately primitive, this pairing satisfies the Liebnitz rule.

All that remains is to show that the right hand square commutes, for which it is enough to show that the map $\oldphi\circ R$ commutes with the products corresponding to $x\in\Omega(2)$ induced by the maps $f$ and $\sigma\circ R$. Thus, we consider a pair of elements $t_1\in \overline{N}_{-r}(\dualLieOperad)\circ R$ and $t_2\in \overline{N}_{-s}(\dualLieOperad)\circ R$, where $r+s=p$. Then $\textup{AW}(t_1\otimes t_2)=(\delta^{p}\cdots \delta^{s+1}t_1)\otimes (\delta^{s-1}\cdots \delta^{0}t_2)$, which is to say that on $t_1$ one applies the coaction map $\rho:R\to\dualLieOperad\circ R$, $r$ times, and on $t_2$ one applies the coaction map $I\to I\circ \dualLieOperad$, $s$ times. The second of these processes in uninteresting, while repeated application of $\rho$ is a complicated process. However, after composing with $\psi$ (which is to finish applying $f$), and applying $\oldphi\circ R$, we will drop all summands in which application of $\rho^r$ introduces a non-unit vertex. The axiom $(\epsilon\circ R) \rho=1$ for a left coaction demonstrates that in the only summand in $\delta^{p}\cdots \delta^{s+1}t_1$ in which the created rows contain only unit vertices is the obvious summand. Thus, after applying the delevelization function $\oldphi$, all the extra material coming from application of $\textup{AW}$ is sent to zero.

\subsection{Products and Steenrod operations}
Assume that we have a collection of higher cosimplicial Alexander-Whitney maps $\{\Delta_k\}$. Then we have defined a product on $N(I,\dualLieOperad,R)$ by the composite
\[(N(I,\dualLieOperad,R)^{\otimes 2})_{-p}\overset{\textup{AW}}{\to} (N_{-p}(I,\dualLieOperad,R))^{\otimes 2}\overset{\psi}{\to}N_{-p-1}(I,\dualLieOperad,R).\]
Next, for each $i\geq0$, we can form a chain map $\Sq^{i}:N(I,\dualLieOperad,R)\to N(I,\dualLieOperad,R)$ of grading $i$, as the sum of the following two composites (minus signs went missing starting here):
\[\xymatrix@R=4mm{
&%r1c1
(N\otimes N)_{2n}
\ar[dr]^-{\Delta_{n+1-i}}
&%r1c2
&%r1c3
\\%r1c4
N_n
\ar[ur]^-{\textup{`$x\mapsto x\otimes x$'}}
\ar[dr]_-{\textup{`$x\mapsto x\otimes dx$'}}&%r2c1
+&%r2c2
N_{n+i-1}(C\otimes C)
\ar[r]^-{\psi}
&%r2c3
N_{n+i}\\%r2c4
&%r3c1
(N\otimes N)_{2n+1}
\ar[ur]_-{\Delta_{n+2-i}}
&%r3c2
&%r3c3
%r3c4
}\]
I guess we mean that $\Delta_k=0$ for $k<0$. It's a formal consequence of the defining property of the $\{\Delta_k\}$, and the fact that $\psi=\psi T$, that the map $\Sq^i$ is a chain map, and that $\Sq^{n+1}:N_{n}\to N_{2n+1}$ is equal to the squaring map.

It's not obvious that these (natural) operations are the Steenrod operations on the cohomology of a Lie algebra, so that $\Sq^0=0$, the Adem relations are satisfied, etc. I would be disappointed if this were not so.

\subsection{The Koszul homology complex $\overline{K}^c(\dualLieOperad)\circ G$}
Suppose one has a corestricted co-Lie coalgebra $G$, that is, a left comodule $G$ over the cooperad $\dualLieOperad$, which as a symmetric sequence is concentrated in degree zero. We'd like to see that this complex is dual to May's complex for the homology of the restricted Lie algebra $G\dual$. We'll assume that $G$ is finite dimensional (or graded and locally finite).

The reduced Koszul construction $\overline{K}^c(\dualLieOperad)$ itself has no differential (as it is concentrated in degrees $\overline{K}^c_{-a+1}(\dualLieOperad)(a)$). Thus, the differential of $\overline{K}^c_{-a+1}(I,\dualLieOperad,G)$ is given solely by application of the coaction $G\to\dualLieOperad\circ G$. However, understanding that only reduced \emph{binary} trees are nonzero in $\overline{K}^c(\dualLieOperad)$, identifying $G$ with $G(0)$, only the summand $G\to(\dualLieOperad(2)\otimes G\otimes G)_{\Sigma_2}$ of the coaction is not sent to zero. This is equivalent to the map $\Delta:G\to(G\otimes G)_{\Sigma_2}$ which defines the corestricted co-Lie coalgebra $G$.

Thus, the Koszul complex is given by the symmetric algebra: $\overline{K}^c_{-n+1}(I,\dualLieOperad,G)=(G^{\otimes n})_{\Sigma_n}$, and the differential $(G^{\otimes n})_{\Sigma_n}\to (G^{\otimes n+1})_{\Sigma_{n+1}}$ is induced by the sum of the maps $(G^{\otimes i-1}\otimes\Delta\otimes G^{\otimes n-i})$ for $1\leq i\leq n$. This map evidently satisfies the Liebnitz rule for the multiplication on the symmetric algebra.

For notational simplicity, let $L=G\dual$. Then the dual complex is $(L^{\otimes n})^{\Sigma_n}$, with differential induced $(L^{\otimes n+1})^{\Sigma_{n+1}}\to (L^{\otimes n})^{\Sigma_n}$ is induced by the sum of the maps $L^{\otimes i-1}\otimes\lambda\otimes L^{\otimes n-i}$ for $1\leq i\leq n$, where $\lambda:(L\otimes L)^{\Sigma_2}\to L$ is the structure map for the restricted Lie algebra $L$. This map is tricky to understand, but at very least, it's a coderivation for the coproduct on this divided power algebra.

Now May's complex is a differential on the same underlying vector space, the divided power algebra, after a shift in homological dimension. Our goal is simply to show that the differentials coincide. The challenge is mainly in taming the notation. We'll calculate the Koszul differential on $e:=\prod_{i=1}^k\gamma_{r_i}(x_1)$, where $\sum r_i=n+1$. This element is the class
\[e=\sum_{p\in{n+1\choose r_1,\ldots,r_k}}[x_{p_1}|\cdots| x_{p_{n+1}}],\]
where the sum is taken over sequences $p=(p_1,\ldots,p_{n+1})$ in which $i$ appears $r_i$ times (for $1\leq i\leq k$), and we use bars instead of tensor symbols for compactness. We can rewrite $e$ as a sum of two terms:
\[\sum_{a=1}^n\sum_{i=1}^k\sum_{p\in{n-1\choose \ldots,r_i-2,\ldots}}[x_{p_1}|\cdots | x_{p_{a-1}}| x_i| x_i| x_{p_{a}}|  \cdots| x_{p_{n-1}}], \textup{ and}\]
\[\sum_{a=1}^n\sum_{i<j}\sum_{p\in{n-1\choose \ldots,r_i-1,\ldots,r_j-1,\ldots}}[x_{p_1}|\cdots | x_{p_{a-1}}| (x_i| x_j+x_j| x_i)| x_{p_{a}}|  \cdots| x_{p_{n-1}}].\]
As $\lambda(x_i\otimes x_i)=\restn{(x_i)}$ and $\lambda(x_i\otimes x_j+x_j\otimes x_i)=[x_i,x_j]$, the images of these two summands under the differential are
\[\sum_{a=1}^n\sum_{i=1}^k\sum_{p\in{n-1\choose \ldots,r_i-2,\ldots}}[x_{p_1}|\cdots | x_{p_{a-1}}| \restn{(x_i)}| x_{p_{a}}|  \cdots| x_{p_{n-1}}], \textup{ and}\]
\[\sum_{a=1}^n\sum_{i<j}\sum_{p\in{n-1\choose \ldots,r_i-1,\ldots,r_j-1,\ldots}}[x_{p_1}|\cdots | x_{p_{a-1}}| [x_i,x_j]| x_{p_{a}}|  \cdots| x_{p_{n-1}}].\]
These are the two components of the formula for May's differential (with trivial coefficients).
\end{PRlieKoszulComplexCalculation}

\begin{LieLambdaStructureOnKoszul}
\section{Structure on the derived functors $(\derived\Ind{\nontop{}})X$} 
\subsection{Chain level operations and the Koszul complex} \label{SectionHtpyOpnsOnLieAlgs}
For any simplicial abelian group $A$, %in this case $A=\Ind{\nontop{}}(\BarConst{\nontop{}}X)$, 
there are three associated chain complexes calculating $\pi_*A$, connected by natural homology equivalences
\[NA_*\to CA_*\to \ModDegeneracies{A}_*\]
of which the first is an inclusion, the second is a surjection, and the composite is an isomorphism. The complex $CA_*$ has $CA_r=A_r$, with the sum of the face maps as boundary. The subcomplex $NA_*$ is defined by $NA_r=\cap_{i=1}^r\ker(d_i)$, and the quotient is by the subcomplex of $CA_*$ spanned by degenerate simplices.

If $A$ is a simplicial object in $\PRLie{n}$, then there are readily defined (see \cite[\S8]{CurtisSimplicialHtpy.pdf}) operations in $NA_*$ which when applied to cycles give $\pi_*A$ its $\PiAlgebra{\PRLie{n}}$ structure. %Giving the inverse isomorphism could cause difficulty: for any $z\in A_n$, there is a unique degenerate element $b$ such that $z+b\in NA_n$, but one must find this `$b$'.
To calculate these operations on homotopy, our approach will be to use the constructions in $NA_*$, but to produce homologies in the larger complex $CA_*$. We can do this freely, as $NA_*\to CA_*$ is a homology equivalence.

Now for $X\in\nontop{n}$, for any $n\geq1$, the Koszul complex $\Koverline_*(X)$ was defined as a subcomplex of $\Boverline_*(X):=\ModDegeneracies{\Ind{\nontop{}}\BarConst{\nontop{}}(X)}_*$. We'll explain this a little further for clarity, in the $n\geq2$ case%footnote:
\footnote{This analysis is unchanged in the $n=1$ case, except that one swaps `$\Q$' for `$P$' in each bar construction.}. The $r^\textup{th}$ simplicial level $(\Ind{\nontop{}}\BarConst{\nontop{}}(X))_r$
is spanned by the monomials of proposition \ref{PropDescnOfReducedBarConstruction}, except that here we \emph{allow} empty bars (the restriction that no bar should be empty being the result of taking the quotient by degenerate simplices to form $\Boverline_*(X)$). The face maps are
\begin{alignat*}{2}
d_0([e_r|\cdots |e_1]x)
&=
[e_r|\cdots |e_2]e_1x%
\\
d_i([e_r|\cdots |e_1]x)
&=
[e_r|\cdots|e_{r+1}e_r|\cdots  |e_1]x
&\qquad&\text{\hspace{-1cm}($0<i<r$)}\\
% Left hand side
d_r([e_r|\cdots |e_1]x)
% Relation
&=
% Right hand side
\begin{cases}
[e_{r-1}|\cdots |e_1]x,&\textup{if $e_r$ is an empty bar};\\
0,&\textup{otherwise.}
%\\,&\textup{if }
\end{cases}
% Comment
\end{alignat*}
Now an element of the Koszul complex is an element of $\Boverline_{rr}(X)$ which survives to $E^1$ in the length filtration spectral sequence for $\Boverline_*(X)$. However, the differential on $E^0$ implements the sum of all the face maps but $d_0$, so that any element of $\Koverline_r$ can be written as the image in $\Boverline_rX:=\ModDegeneracies{\Ind{\nontop{}}(\BarConst{\nontop{}}X)}_r$ of a unique expression of the form
\[z=\sum_{j}\left[\Q^{i_{j,r}} \middle|\cdots\middle|\Q^{i_{j,1}} \right]x_j\in \Ind{\nontop{}}(\BarConst{\nontop{}}X)_r\]
and this element will satisfy
\[\sum_{k=1}^{r}d_kz:=\sum_{k=1}^{r-1}\sum_{j}\left[\Q^{i_{j,r}} \middle|\cdots\middle|\Q^{i_{j,k+1}}\Q^{i_{j,k}}\middle|\cdots\middle|\Q^{i_{j,1}} \right]x_j=0\textup{ in }\Boverline_{r-1,r}(X).\]
%However, the fact that we quotient by degenerate elements in forming $\Boverline_*(X)$ is irrelevant to this condition, so 
This condition implies that $d_kz$ is zero in ${\Ind{\nontop{}}(\BarConst{\nontop{}}X)}_{r-1}$ for each $k>0$, not just zero in the quotient. Thus $z$ lies in the subcomplex $N({\Ind{\nontop{}}(\BarConst{\nontop{}}X)})_{*}$.

Thus we've constructed homology isomorphisms, for $X\in\LL{n}$:
%\[\Koverline_*X\to N(\Ind{\nontop{}}(\BarConst{\nontop{}}(\forget{\nontop{}}X)))\overset{N(\Ind{\nontop{}}(i))}{\to} N(\Ind{\nontop{}}(\forget{\nontop{}}\BarConst{\LL{n}}(X)))\]
\begin{alignat*}{2}
\Koverline_*(\forget{\nontop{}}X)
&\xrightarrow{\hphantom{N(\Ind{\nontop{}}(i))}}
N(\Ind{\nontop{}}(\BarConst{\nontop{}}(\forget{\nontop{}}X)))_*%\overset{\cong}{\to}\Boverline_*(\forget{\nontop{}}X)%
\\
&\xrightarrow{\smash{N(\Ind{\nontop{}}(i))}}
N(\Ind{\nontop{}}(\forget{\nontop{}}\BarConst{\LL{}}(X)))_*\\
% Left hand side
% Relation
&\hphantom{\xrightarrow{{N(\Ind{\nontop{}}(i))}}}\ =
% Right hand side
\forget{\GR{}}N(\Ind{\nontop{}}(\BarConst{\LL{}}(X)))_*=:\forget{\GR{}}\Boverline{}_*^\textup{L}(X)
\end{alignat*}
Here we've introduced the abbreviation $\Boverline_*^\textup{L}(X)$ for $N(\Ind{\nontop{}}(\BarConst{\LL{}}(X)))_*$ --- the $\textup{L}$ denotes that this is the larger normalised bar construction, as compared to $\Boverline_*X:=\ModDegeneracies{\Ind{\nontop{}}(\BarConst{\nontop{}}X)}_*$. It's on $\Boverline_*^\textup{L}(X)$ that we'll define operations inducing the $\PiAlgebra{\PRLie{n}}$ structure on $(\derived\Ind{\nontop{}})X$.
These two maps are inclusions, so that $\Koverline_*(\forget{\nontop{}}X)$ is a homology-equivalent subcomplex of $\Boverline_*^\textup{L}(X)$. Suppressing the forgetful functors, let 
\[\jmath:\Koverline_*(X)\to \Boverline_*^\textup{L}(X)\]
be this composite inclusion. In order to calculate with this larger complex, we'll need to be explicit about the stucture maps in the larger simplicial bar construction $\Ind{\nontop{}}\BarConst{\LL{}}(X)$. Using the natural isomorphism $\Ind{\nontop{}}(\Fr{\LL{n}}\DASH) \cong\Fr{\PRLie{n}}\DASH$ of functors $\GR{n}\to\PRLie{n}$, we can write:
\[(\Ind{\nontop{}}\BarConst{\LL{}}(X))_r=\Fr[out]{\PRLie{n}}\Fr[r]{\LL{n}}\cdots\Fr[1]{\LL{n}} X.\]
Here, we've %introduced the notation ``$\Boverline^\textup{L}_\bullet$'' for this \emph{large} bar construction, and 
added subscripts to the various free constructions involved, in order to disambiguate them in what follows. The subcript in $\Fr[out]{\PRLie{n}}$ denotes simply that this is the outermost free constuction.
Bar notation is suitable to describe elements of $\Koverline_*(\forget{\nontop{}}X)$ and $\Boverline_*(\forget{\nontop{}}X)$, but is not rich enough to describe elements of $\Boverline^\textup{L}_*(X)$. Instead, we add subscripts corresponding to those added to the free constructions above. For example, we exhibit two elements of $(\Ind{\nontop{}}\BarConst{\LL{}}(X))_3$ (defined only if $x,y\in X$ are of sufficiently large degrees):
\[[[x,\Q^3_2y]_3,\Q^5_1x]_{out}\textup{\quad and\quad }\restn[out]{([\Q^4_2\Q^3_1x,y]_3)}.\]
 The simplicial structure maps are defined by the following natural transformations:
\begin{itemize}\squishlist
\setlength{\parindent}{.25in}
\item $d_0:(\Ind{\nontop{}}\BarConst{\LL{}}(X))_r\to (\Ind{\nontop{}}\BarConst{\LL{}}(X))_{r-1}$ is induced by $\Fr[1]{\LL{n}}X\to X$;
\item $d_i:(\Ind{\nontop{}}\BarConst{\LL{}}(X))_r\to (\Ind{\nontop{}}\BarConst{\LL{}}(X))_{r-1}$ is induced by the natural transformation $\Fr[i+1]{\LL{n}}\Fr[i]{\LL{n}}\to \Fr[i]{\LL{n}}$ for $0<i<r$;
\item $d_r:(\Ind{\nontop{}}\BarConst{\LL{}}(X))_r\to (\Ind{\nontop{}}\BarConst{\LL{}}(X))_{r-1}$ is induced by the natural transformation $\Fr[out]{\PRLie{n}}\Fr[r]{\LL{n}}\to \Fr[out]{\PRLie{n}}$ which kills nontop $\Q^i_r$ terms and identifies the top $\Q^i_r$ with $\restn[out]{}$ (and when $n=1$, kills all $P^i_r$ operations);
\item $s_i:(\Ind{\nontop{}}\BarConst{\LL{}}(X))_r\to (\Ind{\nontop{}}\BarConst{\LL{}}(X))_{r+1}$ is induced by the unit map of an instance of $\Fr{\LL{n}}$ inserted between $\Fr[i+1]{\LL{n}}$ and $\Fr[i]{\LL{n}}$, for $0\leq i\leq r$.
\end{itemize}
We may describe $\jmath$ simply as the linear extension of the assignment
\[\left[\Q^{i_{r}} \middle|\cdots\middle|\Q^{i_{1}} \right]x\mapsto\Q^{i_{r}}_r\cdots \Q^{i_{1}}_1x.\]
Note that $\jmath$ (when viewed as a map into the full simplicial bar construction) factors through the unit map of $\Fr[out]{\PRLie{n}}$:
\[\xymatrix@R=0mm{
&%r1c1
\Boverline^\textup{L}_rX
\ar[rd]^{\textup{\tiny inc}}
&%r1c2
\\%r1c3
\Koverline_r(X)
\ar[ur]^-{\jmath}
\ar[dr]_-{\jmathbar}
&%r2c1
&%r2c2
(\Ind{\nontop{}}\BarConst{\LL{}}(X))_r\\%r2c3
&%r3c1
\Fr[r]{\LL{n}}\cdots\Fr[1]{\LL{n}} X
\ar[ur]_-{{\eta_{\PRLie{n}}}}
&%r3c2
%r3c3
}\]


We are now in a position to calculate the $\PiAlgebra{\PRLie{n}}$ structure on $(\derived\Ind{\nontop{}})X$ in certain cases, using the formulae of \cite[\S8]{CurtisSimplicialHtpy.pdf}. For $z\in Z\Koverline_p(X)$ and $w\in Z\Koverline_q(X)$ cycles representing classes $\overline{z},\overline{w}\in(\derived\Ind{\nontop{}})X$:
\begin{itemize}
\setlength{\parindent}{.25in}
\squishlist
\item $[\overline{z},\overline{w}]$ is represented by $\sum_{(\alpha,\beta)\in\Shuffles{p}{q}}[s_\beta(\jmath z), s_\alpha(\jmath w)]_{out}\in \Boverline^\textup{L}_{p+q}X$;
\item $Q^i\overline{z}$ is represented by $\sum_{(\alpha,\beta)\in\HalfShuffles{i}{i}}[s_\beta(\jmath z), s_\alpha(\jmath z)]_{out}\in \Boverline^\textup{L}_{p+i}X$ for $i$ satisfying $0<i\leq p$; and
\item $Q^0\overline{z}$ is represented by $\restn[out]{(z)}\in \Boverline^\textup{L}_{p}X$ as long as $Q^0\overline{z}$ is defined, i.e.\ when not all of $a_n,\ldots,a_2$ are zero, where $\overline{z}\in((\derived_{p}\Ind{\nontop{}})X)_{a_n,\ldots,a_1}$.
\end{itemize}
\noindent Here, $\Shuffles{p}{q}$ is the set of $(p,q)$-shuffles, pairs $(\alpha,\beta)$ with $\alpha=(\alpha_p,\ldots,\alpha_1)$ and $\beta=(\beta_q,\ldots,\beta_1)$ monotonically decreasing sequences that together partition the set $\{0,\ldots,p+q-1\}$, and $s_{\alpha}$ denotes the repeated degeneracy $s_{\alpha_p}\cdots s_{\alpha_1}$.
Finally, $\HalfShuffles{i}{i}$ consists of those shuffles $(\alpha,\beta)\in\Shuffles{i}{i}$ such that $\alpha_1=0$.




\begin{Omitted}
Let $\frakg$ be a simplicial Lie algebra over $\F_2$. Let $N\frakg_*$ be the Moore complex of $\frakg$, defined as
\[N\frakg_n:=\bigcap_{i=0}^{n-1}\ker(d_i:\frakg_n\to\frakg_{n-1})\textup{ with differential }\partial:=d_n|_{{N\frakg}_n}.\]
Then $N\frakg_*$ is a differential graded Lie algebra, under the map
\[N\frakg_*\otimes N\frakg_*\overset{\nabla}{\to}N(\frakg_*\otimes \frakg_*)\overset{N([\,,\,])}{\to} N\frakg_*.\]
Here, $\nabla$ is the Eilenberg-Zilber map, or `shuffle map'. Thus, when $|z|=p$ and $|w|=q$:
\[z\otimes w\overset{\nabla}{\mapsto}\sum_{(\alpha,\beta)\in\Shuffles{p}{q}}s_\beta z\otimes s_\alpha w\mapsto \sum_{(\alpha,\beta)}[s_\beta z, s_\alpha w].\]
Here, $\Shuffles{p}{q}$ is the set of $(p,q)$-shuffles, pairs $(\alpha,\beta)$ with $\alpha=(\alpha_p,\ldots,\alpha_1)$ and $\beta=(\beta_q,\ldots,\beta_1)$ monotonically decreasing sequences that together partition the set $\{0,\ldots,p+q-1\}$, and $s_{\alpha}$ denotes the repeated degeneracy $s_{\alpha_p}\cdots s_{\alpha_1}$.

Next, we have operations $\Q^i:N\frakg_n\to N\frakg_{n+i}$, defined for $0<i\leq n$, by the following formula:
\[z\mapsto \sum_{{(\alpha,\beta)\in\HalfShuffles{i}{i}}}[s_\beta z, s_\alpha z].\]
Here, $\HalfShuffles{i}{i}$ consists of those elements $(\alpha,\beta)$ of $\Shuffles{i}{i}$ such that $\alpha_1=0$.
One notes that as $i\leq n$, the expressions $s_\alpha z$ and $s_\beta z$ both still make sense, and that when $i=n$, this formula is one half of the formula for ``$[z,z]$'' (in $N\frakg_*$).

Finally, if $\frakg$ supports a partial restriction $\restn{}$ on a subspace $\frakg^\textup{r}$, we have an operation $\Q^0:N\frakg^\textup{r}_n\to N\frakg_n$, defined simply by $z\mapsto\restn(z)$.
\end{Omitted}
Before we work with these definitions, we recall some useful simplicial relations:
\begin{lem}\label{LemmaOnSimplicialRelations}
Choose $i\geq0$ and $\alpha=(\alpha_p,\ldots,\alpha_1)$ with $\alpha_p>\cdots >\alpha_1\geq0$.
\begin{enumerate}[i)]\squishlist
\setlength{\parindent}{.25in}
\item If neither $i$ nor $i-1$ appear in $\alpha$, then  $d_is_\alpha=s_{\alpha_p-1}\cdots s_{\alpha_{c+1}-1}s_{\alpha_c}\cdots s_{\alpha_1}d_{i-c}$.
\item If one of $i$ and $i-1$ appears in $\alpha$, then  $d_is_\alpha=s_{\alpha_p-1}\cdots s_{\alpha_{c+1}-1}s_{\alpha_{c-1}}\cdots s_{\alpha_1}$.
\end{enumerate}
Here, $c$ is the index such that $\alpha_{c+1}>i\geq\alpha_{c}$, where, for notational convenience, we set $\alpha_{p+1}=\infty$ and $\alpha_0=-\infty$. More crudely:
\begin{enumerate}[i)]\squishlist
\setlength{\parindent}{.25in}
\item[iii)] If neither $i$ nor $i-1$ appear in $\alpha$, then  $d_is_\alpha=s_{\alpha'}d_{i'}$ for some $\alpha'$ and $i'$.
\item[iv)] If exactly one of $i$ and $i-1$ appears in $\alpha$, then  $d_is_\alpha$ does not depend on which of $i$ and $i-1$ appeared.
\end{enumerate}
\end{lem}
%%%%%\subsection{Chain level operations and the Koszul complex}
%%%%%
%%%%%For this discussion, fix an object $X\in\LL{n}$.
%%%%%The Koszul complex $\Koverline_*(X)$, as constructed in \S\ref{sectionOnUnstableKoszulComplexes}, is a subcomplex of the reduced bar construction $\Boverline_*:= \ModDegeneracies{\Ind{\nontop{}}(\BarConst{\nontop{}}X)}_*$, thought of as the quotient of the unreduced bar construction by the subcomplex of degenerate simplices. The operations defined in \S\ref{SectionHtpyOpnsOnLieAlgs} are defined, at chain level, in the Moore complex of a simplicial Lie algebra, considered as a subcomplex. These two definitions of the Moore complex are isomorphic: for any simplicial abelian group $A$, in this case $A=\Ind{\nontop{}}(\BarConst{\nontop{}}X)$, there are natural homology equivalences
%%%%%\[NA_*\to CA_*\to \ModDegeneracies{A}_*\]
%%%%%of which the first is an inclusion, the second is a surjection, and the composite is an isomorphism. Giving the inverse isomorphism could cause difficulty: for any $z\in A_n$, there is a unique degenerate element $b$ such that $z+b\in NA_n$, but one must find this `$b$'.
%%%%%
%%%%%Fortunately, it happens that the inclusion of the Koszul complex factors through this isomorphism \emph{explicitly}. To understand this, one thinks that an element of the Koszul complex is an element of $\Boverline_{rr}$ which survives to $E^1$ in the length filtration spectral sequence for $\Boverline_*$. However, $d^0$ implements the sum of all the face maps but one, so that any element $z$ of $\Koverline_r$ can be written as the image in $\ModDegeneracies{\Ind{\nontop{}}(\BarConst{\nontop{}}X)}_r$ of a unique expression in $\Ind{\nontop{}}(\BarConst{\nontop{}}X)_r$ of the form
%%%%%\[z=\sum_{j}\left[\Q^{i_{j,r}} \middle|\cdots\middle|\Q^{i_{j,1}} \right]x_j\]
%%%%%such that 
%%%%%\[\sum_{k=1}^{r}d_kz:=\sum_{k=1}^{r-1}\sum_{j}\left[\Q^{i_{j,r}} \middle|\cdots\middle|\Q^{i_{j,k+1}}\Q^{i_{j,k}}\middle|\cdots\middle|\Q^{i_{j,1}} \right]x_j=0\textup{ in }\Boverline_{r-1,r}(X).\]
%%%%%However, the fact that we quotient by degenerate elements in forming $\Boverline_*(X)$ is irrelevant to this condition, so that for $k\neq 0$, $d_kz$ is zero in ${\Ind{\nontop{}}(\BarConst{\nontop{}}X)}_{r-1}$, not only in the quotient $\Boverline_{r-1}(X)$, so that $z$ lies in the subcomplex $NB_r$, with no correction `$b$' required. One swaps `$\Q$' for `$P$' when $n=1$.
%%%%%
%%%%%Thus we've constructed homology isomorphisms, for $X\in\LL{n}$:
%%%%%%\[\Koverline_*X\to N(\Ind{\nontop{}}(\BarConst{\nontop{}}(\forget{\nontop{}}X)))\overset{N(\Ind{\nontop{}}(i))}{\to} N(\Ind{\nontop{}}(\forget{\nontop{}}\BarConst{\LL{n}}(X)))\]
%%%%%\begin{alignat*}{2}
%%%%%\Koverline_*(\forget{\nontop{}}X)
%%%%%&\xrightarrow{\hphantom{N(\Ind{\nontop{}}(i))}}
%%%%%N(\Ind{\nontop{}}(\BarConst{\nontop{}}(\forget{\nontop{}}X)))_*=:\Boverline_*(\forget{\nontop{}}X)%
%%%%%\\
%%%%%&\xrightarrow{N(\Ind{\nontop{}}(i))}
%%%%%N(\Ind{\nontop{}}(\forget{\nontop{}}\BarConst{\LL{}}(X)))_*\\
%%%%%% Left hand side
%%%%%% Relation
%%%%%&\hphantom{\xrightarrow{{N(\Ind{\nontop{}}(i))}}}\ =
%%%%%% Right hand side
%%%%%\forget{\GR{}}N(\Ind{\nontop{}}(\BarConst{\LL{}}(X)))_*=:\forget{\GR{}}\Boverline{}_*^\textup{L}(X)
%%%%%\end{alignat*}
%%%%%Here we've introduced the notation $\Boverline_*^\textup{L}(X)$ for $N(\Ind{\nontop{}}(\BarConst{\LL{}}(X)))_*$, the differential graded partially restricted Lie algebra which calculates the $\PiAlgebra{\PRLie{n}}$ $(\derived\Ind{\nontop{}})X$.
%%%%%These two maps are inclusions, so that $\Koverline_*(\forget{\nontop{}}X)$ is a homology-equivalent subcomplex of $\Boverline_*^\textup{L}(X)$. Suppressing the forgetful functors, let 
%%%%%\[\jmath:\Koverline_*(X)\to \Boverline_*^\textup{L}(X)\]
%%%%%be this composite inclusion. In order to calculate with this larger complex, we'll need to be explicit about the stucture maps in the larger simplicial bar construction $\Ind{\nontop{}}\BarConst{\LL{}}(X)$. Using the natural isomorphism $\Ind{\nontop{}}(\Fr{\LL{n}})W\cong\Fr{\PRLie{n}}W$ for $W\in\GR{n}$, we can write:
%%%%%\[(\Ind{\nontop{}}\BarConst{\LL{}}(X))_r=\Fr[out]{\PRLie{n}}\Fr[r]{\LL{n}}\cdots\Fr[1]{\LL{n}} X.\]
%%%%%Here, we've %introduced the notation ``$\Boverline^\textup{L}_\bullet$'' for this \emph{large} bar construction, and 
%%%%%added subscripts to the various free constructions involved, in order to disambiguate them in what follows. The simplicial structure maps are defined by the following natural transformations:
%%%%%\begin{itemize}\squishlist
%%%%%\setlength{\parindent}{.25in}
%%%%%\item $d_0:(\Ind{\nontop{}}\BarConst{\LL{}}(X))_r\to (\Ind{\nontop{}}\BarConst{\LL{}}(X))_{r-1}$ is induced by $\Fr[1]{\LL{n}}X\to X$;
%%%%%\item $d_i:(\Ind{\nontop{}}\BarConst{\LL{}}(X))_r\to (\Ind{\nontop{}}\BarConst{\LL{}}(X))_{r-1}$ is induced by $\Fr[i+1]{\LL{n}}\Fr[i]{\LL{n}}\to \Fr[i]{\LL{n}}$ for $0<i<r$;
%%%%%\item $d_r:(\Ind{\nontop{}}\BarConst{\LL{}}(X))_r\to (\Ind{\nontop{}}\BarConst{\LL{}}(X))_{r-1}$ is induced by $\Fr[out]{\PRLie{n}}\Fr[r]{\LL{n}}\to \Fr[out]{\PRLie{n}}$ (which, for $n\geq2$, kills nontop $\Q^i$ terms and identifies the top $\Q$ with $\restn{}$, and when $n=1$, kills all $P$ operations);
%%%%%\item $s_i:(\Ind{\nontop{}}\BarConst{\LL{}}(X))_r\to (\Ind{\nontop{}}\BarConst{\LL{}}(X))_{r+1}$ is induced by the unit map of an instance of $\Fr{\LL{n}}$ inserted between $\Fr[i+1]{\LL{n}}$ and $\Fr[i]{\LL{n}}$, for $0\leq i\leq r$.
%%%%%\end{itemize}
%%%%%The bar notation used in \S\ref{sectionOnUnstableKoszulComplexes} is suitable to describe elements of $\Koverline_*(\forget{\nontop{}}X)$ and $\Boverline_*(\forget{\nontop{}}X)$, but is not rich enough to describe elements of $\Boverline^\textup{L}_*(X)$. Instead, we add subscripts corresponding to those added to the free constructions above. For example, here are two elements of $(\Ind{\nontop{}}\BarConst{\LL{}}(X))_3$ (when $n\geq2$), if $x,y\in X$ are of sufficiently large degrees:
%%%%%\[[[x,\Q^3_2y]_3,\Q^5_1x]_{out}\textup{\quad and\quad }\restn[out]{([\Q^4_2\Q^3_1x,y]_3)}.\]
%%%%%Then we may describe $\jmath$ simply as the linear extension of the assignment
%%%%%\[\left[\Q^{i_{r}} \middle|\cdots\middle|\Q^{i_{1}} \right]x\mapsto\Q^{i_{r}}_r\cdots \Q^{i_{1}}_1x.\]
%%%%%
%%%%%
%%%%%We are now in a position to calculate the $\LL{n+1}$ structure on $(\derived\Ind{\nontop{}})X$ in certain cases, since we can write down chain-level formulae. For $z\in Z\Koverline_p(X)$ and $w\in Z\Koverline_q(X)$ cycles representing classes $\overline{z},\overline{w}\in(\derived\Ind{\nontop{}})X$:
%%%%%\begin{itemise}
%%%%%\setlength{\parindent}{.25in}
%%%%%\item $[\overline{z},\overline{w}]$ is represented by $\sum_{(\alpha,\beta)\in\Shuffles{p}{q}}[s_\beta(\jmath z), s_\alpha(\jmath w)]_{out}\in \Boverline^\textup{L}_{p+q}X$;
%%%%%\item $Q^i\overline{z}$ is represented by $\sum_{(\alpha,\beta)\in\HalfShuffles{i}{i}}[s_\beta(\jmath z), s_\alpha(\jmath z)]_{out}\in \Boverline^\textup{L}_{p+i}X$ for $0<i\leq p$; and
%%%%%\item $Q^0\overline{z}$ is represented by $\restn[out]{(z)}\in \Boverline^\textup{L}_{p}X$ as long as $Q^0\overline{z}$ is defined, (i.e.\ when not all of $a_n,\ldots,a_2$ are zero, where $\overline{z}\in((\derived_{p}\Ind{\nontop{}})X)_{a_n,\ldots,a_1}$).
%%%%%\end{itemise}
%%%%%%%%%%%%
%%%%%%%%%%%%At this point we need to be explicit about the maps inside the bar construction. Moreover, we'll need notation for elements of $B(\PRLie{n},\LL{n},X)$, and the bar notation we've used so far will not suffice. The bar construction is given by the formula $B_r:=\Fr{\PRLie{n}}\left(\Fr{\LL{n}}\right)^rX$, and we'll add subscripts to the various free constructions here in order to disambiguate them in what follows, writing
%%%%%%%%%%%%\[B_r:=\Fr[out]{\PRLie{n}}\Fr[r]{\LL{n}}\cdots\Fr[1]{\LL{n}} X.\]
%%%%%%%%%%%%Here, $out$ stands for `outer'. We'll index the faces and degeneracies in the opposite order to the usual one, and inexpensive way to keep notation simple. Thus:
%%%%%%%%%%%%\begin{itemize}\squishlist
%%%%%%%%%%%%\setlength{\parindent}{.25in}
%%%%%%%%%%%%\item $d_0:B_r\to B_{r-1}$ is induced by $\Fr[1]{\LL{n}}X\to X$;
%%%%%%%%%%%%\item $d_i:B_r\to B_{r-1}$ is induced by $\Fr[i+1]{\LL{n}}\Fr[i]{\LL{n}}\to \Fr[i]{\LL{n}}$ for $0<i<r$;
%%%%%%%%%%%%\item $d_r:B_r\to B_{r-1}$ is induced by $\Fr[out]{\PRLie{n}}\Fr[r]{\LL{n}}\to \Fr[out]{\PRLie{n}}$ (which, for $n\geq2$, kills nontop $\Q^i$ terms and identifies the top $\Q$ with $\restn{}$, and when $n=1$, kills all $P$ operations);
%%%%%%%%%%%%\item $s_i:B_r\to B_{r+1}$ is induced by the unit map, after inserting another $\Fr{\LL{n}}$ between $\Fr[i+1]{\LL{n}}$ and $\Fr[i]{\LL{n}}$ , for $0\leq i\leq r$.
%%%%%%%%%%%%\end{itemize}
%%%%%%%%%%%%
%%%%%%%%%%%%
%%%%%%%%%%%%We'll also include the subscripts in notation for the elements of $B_r$, for example, here are two elements of $B_3$ (when $n\geq2$), if $x,y\in X$ are of sufficiently large degrees:
%%%%%%%%%%%%\[[[x,\Q^3_2y]_3,\Q^5_1x]_{out}\textup{\quad and\quad }\restn[out]{([\Q^4_2\Q^3_1x,y]_3)}.\]
%%%%%%%%%%%%%The second expression means, for $x,y\in X$, form $\Q^3_1x\in\Fr{\LL{n}}X$.....
%%%%%%%%%%%%Then we may describe $\jmath$ simply as
%%%%%%%%%%%%\[\left[\Q^{i_{r}} \middle|\cdots\middle|\Q^{i_{1}} \right]x\mapsto\Q^{i_{r}}_r\cdots \Q^{i_{1}}_1x.\]
%%%%%%%%%%%%
%%%%%%%%%%%%
%%%%%%%%%%%%We are then in a position to calculate the $\LL{n+1}$ structure on $(\derived\Ind{\nontop{}})X$ in certain cases, since we can write down chain-level formulae. For $z\in Z\Koverline_p(X)$ and $w\in Z\Koverline_q(X)$, representing classes $\overline{z},\overline{w}\in(\derived\Ind{\nontop{}})X$:
%%%%%%%%%%%%\begin{itemise}
%%%%%%%%%%%%\setlength{\parindent}{.25in}
%%%%%%%%%%%%\item $[\overline{z},\overline{w}]$ is represented by $\sum_{(\alpha,\beta)\in\Shuffles{p}{q}}[s_\beta(\jmath z), s_\alpha(\jmath w)]_{out}$;
%%%%%%%%%%%%\item $Q^i\overline{z}$ is represented by $\sum_{(\alpha,\beta)\in\HalfShuffles{i}{i}}[s_\beta(\jmath z), s_\alpha(\jmath z)]_{out}$ for $0<i\leq p$; and
%%%%%%%%%%%%\item $Q^0\overline{z}$ is represented by $\restn[out]{(z)}$ if not all of $a_n,\ldots,a_2$ are zero, where $\overline{z}\in((\derived_{p}\Ind{\nontop{}})X)_{a_n,\ldots,a_1}$.
%%%%%%%%%%%%\end{itemise}
\subsection{Lie brackets on $(\derived\Ind{\nontop{}})X$ are trivial}
\begin{prop}\label{LieBracketsTrivial}
Suppose that $z\in Z\Koverline_p(X)$ and $w\in Z\Koverline_q(X)$ are cycles. Then on homotopy classes:% representing classes $\overline{x}$ and $\overline{y}$. Then
\[[\overline{z},\overline{w}]=\begin{cases}
\overline{[z,w]},&\textup{if }p=q=0;\\
0,&\textup{otherwise}.
%\\,&\textup{if }
\end{cases}
\]
Here, $Z\Koverline_0(X)=X$ has the Lie bracket supplied by the $\LL{n}$ structure of $X$.
%Suppose that $p>0$ and $q>0$. Then $[\overline{x},\overline{y}]=0$ for cycles $x\in Z\Koverline_p(X)$ and $y\in Z\Koverline_q(X)$.
\end{prop}
\begin{proof}
We perform the proof for $n\geq2$, but it works the same way for $n=1$, when one substitutes $\Q$ operations for $P$ operations and ignores all discussion of top and non-top operations.

Consider the following element of $(\Ind{\nontop{}}\BarConst{\LL{}}(X))_{p+q+1}$:
\[a:=\sum_{(\alpha,\beta)\in\Shuffles{p}{q}}[s_{p+q}s_\beta(\jmathbar z), s_{p+q}s_\alpha(\jmathbar w)]_{p+q+1}\]
To unpack this definition, on should think that $\jmathbar z\in \Fr[p]{\LL{n}}\cdots\Fr[1]{\LL{n}} X$, and $s_{p+q}s_\beta(\jmathbar z)$ is the element in $\jmathbar z\in \Fr[p+q+1]{\LL{n}}\cdots\Fr[1]{\LL{n}} X$ obtained by interspersing identities appropriately. One performs the Lie bracket using the structure given by $\Fr[p+q+1]{\LL{n}}$, and maps into $\Fr[out]{\PRLie{n}}\Fr[p+q+1]{\LL{n}}\cdots\Fr[1]{\LL{n}} X$ by the unit map of $\Fr[out]{\PRLie{n}}$.

We'll calculate $d_ia$ for $i=0,\ldots,p+q+1$, and find that $d_ia=0$ for $i<p+q+1$ (except if $p=q=0$), and that $d_{p+q+1}a$ is our formula for $[\overline{z},\overline{w}]$. Although $a$ does not live in $N(\Ind{\nontop{}}\BarConst{\LL{}}(X))_*$, it does provide a nullhomology for the image of $[z,w]$ in $C(\Ind{\nontop{}}\BarConst{\LL{}}(X))$, and thus shows that $[\overline{z},\overline{w}]=0$. Firstly, as $d_{p+q+1}$ applies $\Fr[out]{\PRLie{n}}\Fr[p+q+1]{\LL{n}}\to\Fr[out]{\PRLie{n}}$, $d_{p+q+1}a$ represents $[\overline{z},\overline{w}]$:
\[d_{p+q+1}a=\sum [d_{p+q+1}s_{p+q}s_\beta(\jmath z), d_{p+q+1}s_{p+q}s_\alpha(\jmath w)]_{out}=\sum [s_\beta(\jmath z), s_\alpha(\jmath w)]_{out}.\]
\begin{alignat*}{2}
d_{p+q+1}a
&=\sum [d_{p+q+1}s_{p+q}s_\beta(\jmathbar z), d_{p+q+1}s_{p+q}s_\alpha(\jmathbar w)]_{out}
%
\\
&=\sum [s_\beta(\jmathbar z), s_\alpha(\jmathbar w)]_{out}\\
&=\sum [s_\beta(\jmath z), s_\alpha(\jmath w)]_{out}
\end{alignat*}
where the final equality holds as the Lie brackets were being taken in the $\Fr[out]{\PRLie{n}}$ structure. Next, as $d_{p+q}$ applies $\Fr[p+q+1]{\LL{n}}\Fr[p+q]{\LL{n}}\to\Fr[p+q]{\LL{n}}$:
\begin{alignat*}{2}
d_{p+q}a
&=\sum [d_{p+q}s_{p+q}s_\beta(\jmathbar z), d_{p+q}s_{p+q}s_\alpha(\jmathbar w)]_{p+q}
%
\\
&=\sum [s_\beta(\jmathbar z), s_\alpha(\jmathbar w)]_{p+q}
\end{alignat*}
%\[d_{p+q}a=\sum [d_{p+q}s_{p+q}s_\beta(\jmath z), d_{p+q}s_{p+q}s_\alpha(\jmath w)]_{p+q}=\sum [s_\beta(\jmath z), s_\alpha(\jmath w)]_{p+q}.\]
Now in each term of this sum, one of $s_\beta(\jmathbar z)$ and $s_\alpha(\jmathbar w)$ will have a $\Q_{p+q}^i$ term in every one of its summands (as long as $p+q>0$). Each of these $\Q$ operations will be non-top, due to the strict inequality on $\minDim(I)$ imposed on the basis elements, and the fact that $\SqShift$-Adem relations do not decrease $\minDim$. Thus, the axiom ``$[Q^ix,y]=0$ for $Q^i$ non-top'' in $\Fr[p+q]{\LL{n}}$ implies that each summand $[s_\beta(\jmathbar z), s_\alpha(\jmathbar w)]_{p+q}$ vanishes, so that $d_{p+q}a=0$. If $p=q=0$, there is only a single term, which is $[z,w]$ as calculated inside $X$.

Next, we calculate that $d_{i}a=0$ for $0\leq i< p+q$:
\[d_{i}a=\sum [s_{p+q-1}d_{i}s_\beta(\jmathbar z), s_{p+q-1}d_{i}s_\alpha(\jmathbar w)]_{p+q}.\]
%Now consider a given shuffle $(\alpha,\beta)$. If $i$ and $i-1$ both lie in $\beta$, say, then $d_is_\alpha(\jmath y)=s_{\alpha'}d_{i'}(\jmath y)=0$, as $\jmath y$ is a cycle in $NB_*$. Thus we can ignore all summands in which $i$ and $i-1$ lie in the same half of the shuffle.
%If $i$ and $i-1$ do not lie in the same half of the shuffle, then let $(\gamma,\epsilon)$ be the shuffle obtained from $(\alpha,\beta)$ by swapping the positions of $i$ and $i-1$. But then, $d_is_\beta=d_is_{\epsilon}$, and $d_is_\alpha=d_is_{\gamma}$, so that this pair of summands cancel with each other. \{\{\}\}
For this, we'll define an involution $\rho_i$ of the set $\Shuffles{p}{q}$ indexing the sum.
If $\alpha$ and $\beta$ do not each contain exactly one of $0$ and $1$, then $\rho_i$ fixes $(\alpha,\beta)$. Otherwise, $\rho_i$ interchanges the positions of $i$ and $i-1$ in $(\alpha,\beta)$. 

If $(\alpha,\beta)$ is a fixed point of $\rho_i$, then one of $\alpha$ and $\beta$, say $\alpha$, contains neither of $i$ and $i-1$. Then by lemma \ref{LemmaOnSimplicialRelations}(iii), $d_is_\alpha(\jmathbar w)=s_{\alpha'}d_{i'}(\jmathbar w)=0$, as $\jmath w$ is a cycle in $N(\Ind{\nontop{}}(\BarConst{\LL{}}(X)))_*$, and thus lies in the kernel of all of the face maps.

Given a shuffle $(\alpha,\beta)$ which is not fixed by $\rho_i$, lemma \ref{LemmaOnSimplicialRelations}(iv) shows that the summand corresponding to $(\alpha,\beta)$ equals the summand corresponding to $\rho_i(\alpha,\beta)$, so these two summands cancel with each other.
%
%Now consider a given shuffle $(\alpha,\beta)$. If, say, $\alpha$ contains neither of $i$ and $i-1$, then by lemma \ref{LemmaOnSimplicialRelations}(iii), $d_is_\alpha(\jmath w)=s_{\alpha'}d_{i'}(\jmath w)$, which is zero, as $\jmath w$ is a cycle in $N(\Ind{\nontop{}}(\BarConst{\LL{}}(X)))_*$. Thus we can ignore all summands in which $\alpha$ and $\beta$ do not contain one each of $i$ and $i-1$.
%If $\alpha$ and $\beta$ contain one each of $i$ and $i-1$, then let $(\gamma,\epsilon)$ be the shuffle obtained from $(\alpha,\beta)$ by swapping the positions of $i$ and $i-1$. Then lemma \ref{LemmaOnSimplicialRelations}(iv) explains that $d_is_\beta=d_is_{\epsilon}$ and $d_is_\alpha=d_is_{\gamma}$, so that the summands corresponding to $(\alpha,\beta)$ and to $(\gamma,\epsilon)$ cancel with each other.
\end{proof}
\subsection{$\Q^k$ operations on $(\derived\Ind{\nontop{}})X$ for $k\geq1$}
\begin{Omitted}
In the previous section, we produced an element $a\in B_{p+q+1}(\Fr{\PRLie{n}},\Fr{\LL{n}},X)$, and showed that it was actually an element of $NB_{p+q+1}$, whose boundary was represented the homotopy class we were trying to nullhomotope. In the next two sections, we will choose such an $a$ again, but in some cases, the top \emph{two} face maps will be nonzero on $a$. Thus we will move outside of the normalised complex to construct a homotopy between two nonzero elements. This makes no difference to our calculations, especially since we in the end will always map down to $\Koverline_*$, which doesn't see the distinction anyway.
\end{Omitted}
\begin{prop}\label{QkTrivial}
Suppose that $z\in Z\Koverline_p(X)_{a_n,\ldots,a_1}$.
\begin{itemize}\squishlist
\setlength{\parindent}{.25in}
\item  If $n\geq2$, $\Q^k\overline{z}=0$ for $1\leq k\leq p$.
\item  If $n=1$, $\Q^k\overline{z}=0$ for $2\leq k\leq p$.
\item Suppose that $n=1$, and that $X$ is a trivial object in $\LL{1}$. Write $z$ as %\in Z\Koverline_p(X)_{a_1}=\Koverline_p(X)_{a_1}$ as a sum 
\[z=\sum_{j}P(I^{(j)})x^{(j)}\in Z\Koverline_p(X)_{a_1}=\Koverline_p(X)_{a_1}\]
where each $I^{(j)}=(i_p^{(j)},\ldots,i_1^{(j)})$ is $\delta$-admissible with $\minDimP(I^{(j)})\leq |x^{(j)}|$. Then
\[\Q^1\overline{z}\textup{ is represented by }\sum_{j}P((a_{1},i_p^{(j)},\ldots,i_1^{(j)}))x^{(j)}.\]
%%\item  if $n=1$, and $P(I)y$ is a basis element (which is also a cycle - fix up this statement) for $\Koverline_*X$ (so that $I=(i_\ell,\ldots,i_1)$ is $\delta$-admissible, and $\minDimP(I)\leq |y|$), with $\ell\geq1$, we have:
%%\[\LambdaOp^1(P(I)y)=P((|P^{i_\ell}\cdots P^{i_1}y|,i_{\ell},\ldots,i_1))y.\]
That is, $\Q^1$ adjoins a top $P$ operation to the left of each sequence $I^{(j)}$.
\end{itemize}
\end{prop}
\begin{proof}
The class $\Q^k\overline{z}$ is represented by $\sum_{(\alpha,\beta)\in\HalfShuffles{k}{k}}[s_\beta(\jmath z), s_\alpha(\jmath z)]_{out}$. As in the previous proof, we use the element
\[a:=\sum_{(\alpha,\beta)\in\HalfShuffles{k}{k}}[s_{p+k}s_\beta(\jmathbar z), s_{p+k}s_\alpha(\jmathbar z)]_{p+k+1}.\]
The same argument shows that $d_{p+k+1}a$ is the above representative for $\Q^k\overline{z}$, and that $d_{p+k}a=0$. We are left to calculate $d_ia$ for $0\leq i<p+k$, that is
\[d_ia=\sum_{(\alpha,\beta)\in\HalfShuffles{k}{k}}[s_{p+k-1}d_is_\beta(\jmathbar z), s_{p+k-1}d_is_\alpha(\jmathbar z)]_{p+k}\]
Our strategy here will be the same as in the previous proof, except when $i=1$. That is, for $i\in\{0,2,3,\ldots,p+k-1\}$, the involution $\rho_i$ of $\Shuffles{k}{k}$ preserves the subset $\HalfShuffles{k}{k}$, so that the argument from the proof of \ref{LieBracketsTrivial} carries over to show that this sum vanishes. To avoid confusion, note that $\rho_i$ is the identity on $\Shuffles{k}{k}$ unless $0<i<2k$.

When $i=1$, $\rho_1$ does not preserve $\HalfShuffles{k}{k}$. Instead, we use the following involution, $\widetilde{\rho}_1$, of $\HalfShuffles{k}{k}$. If $\alpha$ and $\beta$ do not each contain exactly one of $0$ and $1$ each, then $\widetilde{\rho}_1$ fixes $(\alpha,\beta)$. Otherwise, $\widetilde{\rho}_1$ returns $\rho_1(\beta,\alpha)$, which is to say that $\widetilde{\rho}_1$ swaps \emph{everything but} $0$ and $1$.

Now the summands in this formula exhibit an extra symmetry, given that $z$ appears twice. This symmetry, along with lemma \ref{LemmaOnSimplicialRelations}(iv), shows that all the summands corresponding to shuffles not fixed by $\widetilde{\rho}_1$ cancel out. When $k>1$, the fixed points of $\widetilde{\rho}_1$ are only those shuffles in which one of $\alpha$ and $\beta$ contains neither $0$ nor $1$, and these summands vanish, by \ref{LemmaOnSimplicialRelations}(iii). When $k=1$, however, $\widetilde{\rho}_1$ has an \emph{extra} fixed point, the shuffle $((0),(1))$, which fails to differ from its image under $\widetilde{\rho}_1$.

We have shown that $d_ia=0$ whenever $0\leq i<p+k$, except when $i=k=1$. In this special case:
\begin{alignat*}{2}
d_1a
&=
[s_{p}d_1s_0(\jmathbar z), s_{p}d_1s_1(\jmathbar z)]_{p+1}%
\\
&=
[s_{p}(\jmathbar z), s_{p}(\jmathbar z)]_{p+1}%
\\
% Left hand side
% Relation
&=
% Right hand side
\begin{cases}
0,&\textup{if }n\geq2,\\
P^{a_1}_{p+1}s_{p}(\jmathbar z),&\textup{if }n=1,
%\\,&\textup{if }
\end{cases}
\end{alignat*}
%
%
%Now if $i\geq 2k$ or $i=0$, at least one of $\alpha$ and $\beta$, say $\alpha$, will not include either $i$ or $i-1$, so that $d_is_\alpha(\jmath z)=s_{\alpha'}d_{i'}(\jmath z)=0$, as $\jmath z$ is a cycle in $ZN\Ind{\nontop{}}\BarConst{\LL{}}X$. Thus, we are left to consider the cases $0< i< 2k$.
%
%In the previous proof, the method in this range was to construct a fixed-point free involution (`swap $i$ and $i-1$') of the set indexing the sum, under which the summands were invariant. The same involution works here, as long as it preserves the subset $\HalfShuffles{i}{i}$ of $\Shuffles{i}{i}$, which it does when $2\leq i<2k$.
%
%%Again the only nonzero summands correspond to pairs $(\alpha,\beta)$ in which each of $\alpha$ and $\beta$ contain one each of $i$ and $i-1$. If $i=0$ there are obviously no such pairs. If $i\geq2$, the same involution of the shuffles preserves the subset $\HalfShuffles{i}{i}$ of $\Shuffles{i}{i}$, and so gives the cancellation desired.
%
%The final case is $d_1$, in which case, the involution of $\Shuffles{i}{i}$ used above does not preserve the subset $\HalfShuffles{i}{i}$. Fortunately, since our summand uses $z$ twice, not both $z$ and $w$, the summands are invariant under a new involution, `swap $\alpha$ and $\beta$'. The composite of these two involutions preserves $\HalfShuffles{i}{i}$, and is fixed-point free unless $k=1$, proving, for all $n$, that the operations $\Q^k$ are zero on homotopy unless $k=1$.
%
%In case $k=1$, $\HalfShuffles{i}{i}$ contains only $((1),(2))$, and we are left with a single term:
%\[d_1a=[s_{p}d_1s_0(\jmath z), s_{p}d_1s_1(\jmath z)]_{p+2}=[s_{p}(\jmath z), s_{p}(\jmath z)]_{p+1}=\begin{cases}
%0,&\textup{if }n\geq2;\\
%P^{a_1}_{p+1}s_{p}(\jmath z),&\textup{if }n=1,
%%\\,&\textup{if }
%\end{cases}
%\]
as self-brackets in $\LL{n}$ are zero when $n\geq2$, and equal the top $P$ operation when $n=1$. As the top $P$ operation is linear, in the $n=1$ case $d_1a$ is then the result of adding a $P^{a_1}$ to the front of every bar involved in the sum defining $z$, so that
\[d_1a:=\sum_j\left(\BarMonomial{(a_1)\star I^{(j)}}{\textbf{1}}{x^{(j)}}+\sum_{\substack{\produces{K}{I^{(j)}}{\delta}\\\minDimP(K)\leq|x^{(j)}|}}\BarMonomial{(a_1)\star K}{\textbf{1}}{x^{(j)}}\right)\]
Where we use the notation from proposition \ref{PropDescnOfReducedBarConstruction}, and write $\textbf{1}:=(1,\ldots,1)$.

Thus we've seen that $\Q^1\overline{z}$ is represented in $C(\Ind{\nontop{}}(\BarConst{\LL{}}(X)))_*$ by the cycle $d_1a$, but $d_1a$ actually lives in the subcomplex $C(\Ind{\nontop{}}(\BarConst{\nontop{}}(X)))_*$. Its image in the quotient $\Boverline_{p+1,p+1}(X)$ is a cycle, which as $X$ is trivial, must%footnote:
\footnote{As $X$ is a trivial object, $\Boverline_{p+1,p+1}$ contains no nonzero boundaries, so that any cycle therein must be the image of a cycle in $\Koverline_{p+1}$, seeing as $\Koverline_*\to \Boverline_*$ is a homology equivalence.} be in the image of $\Koverline_{p+1}(X)$. To determine its unique preimage in $\Koverline_*(X)_{p+1}$, one only needs to consider those monomials, in this formula for $d_1a$, in which the underlying sequence is $\delta$-admissible. There are only the $(a_1)\star I^{(j)}$.
%%%%Recall that this sum is not just indexed over $j$, but rather,
%%%%\[z:=\sum_j\Bigl(\BarMonomial{I^{(j)}}{\textbf{1}}{x^{(j)}}+\sum_{\substack{\produces{K}{I^{(j)}}{\delta}\\\minDimP(K)\leq|x^{(j)}|}}\BarMonomial{K}{\textbf{1}}{x^{(j)}}\Bigr)\]
%%%%using the notation from the description of \textbf{Priddy's algorithm}, and writing $\textbf{1}:=(1,\ldots,1)$. 
%%%%
%%%%
%%%%Thus
%%%%\[d_1a:=\sum_j\left(\BarMonomial{(a_1)\star I^{(j)}}{\textbf{1}}{x^{(j)}}+\sum_{\produces{K}{I^{(j)}}{\delta}}\BarMonomial{(a_1)\star K}{\textbf{1}}{x^{(j)}}\right)\]
%%%%where $\star$ denotes concatenation. None of the $(a_1)\star K$ could hope to be $\delta$-admissible, but all of the $(a_1)\star I^{(j)}$ \emph{are} $\delta$-admissible, simply as $P^{a_1}$ is a top operation. Thus, the image of $d_1a$ in $\Boverline_*$ can be none other than that which we propose.
%
% also over all the $J$ such that $\produces{J}{I^{(j)}}{\delta}$. After making these concatenations, the bars which are $\delta$-admissible are exactly those which are of the form 
%\[d_1a=P^{a_1}_{p+1}s_{p}(\jmath z)=\sum_{j}\sum_{\produces{K}{I^{(j)}}{\delta}}P^{a_1}_{p+1}s_{p}(\jmath z)\]
% %However, objects of $\LL{1}$ may have non-zero self-squares, and the self-square is given by the top $P$ operation. Thus a top $P$ operation is applied to $s_p(\jmath z)$ in $\Fr[p+1]{\LL{n}}$. As all $P$ operations are linear, a cycle $P(I)y$, defined by
%\[P(I)y:=\left[P^{i_r} \middle|\cdots\middle|P^{i_1} \right]y+\sum_{\substack{\produces{J}{I}{\delta} \\ \minDimP(J)\leq|y|}}\left[P^{j_r} \middle|\cdots\middle|P^{j_1} \right]y\]
%receives the same top $P$ operation as a new bar in each of the bars in the above definition. There is only the one $\delta$-admissible sequence appearing in the new sum, completing the proof. \textbf{Sometime, think about what happens here in a little more detail.}
\end{proof}

\subsection{$\Q^0$ operations on $(\derived\Ind{\nontop{}})X$}
Assume for this section that $n\geq2$, so that objects of $\LL{n+1}$ support a $\Q^0$ operation, defined when $a_n,\ldots,a_2$ are not all zero. We'll now go some way towards calculating these operations.
\begin{lem}\label{PrelimQ0Calc}
Suppose that $z\in Z\Koverline_p(X)_{a_n,\ldots,a_1}$, where $a_n,\ldots,a_2$ are not all zero. 
\begin{itemise}
\setlength{\parindent}{.25in}
\item If $p>0$, $Q^0\overline{z}$ is represented by the cycle $\Q^{a_n}_p\jmathbar z$ in $C(\Ind{\nontop{}}\BarConst{\LL{}}X)_*$. Note that this cycle is in general not in the image of the Koszul complex. %the bar construction $NB_p(\Fr{\PRLie{n}},\Fr{\LL{n}},X)$.
\item If $p=0$, $Q^0\overline{z}$ is represented by the cycle $\Q^{a_n}z$, as calculated in $X$.%=Z\Koverline_0(X)$.% in the bar construction $NB_p(\Fr{\PRLie{n}},\Fr{\LL{n}},X)$
\end{itemise}
\end{lem}
\begin{proof}
We use the same homotopy $a$ as in the previous cases. That is, write
$a:=\restn[p+1]{(s_{p}(\jmathbar z))}$. Then $d_{p+1}a=\restn[out]{(s_{p}(\jmathbar z))}$ represents $Q^0\overline{z}$, $d_pa=\restn[p]{(\jmathbar z)}$, and for $0\leq i<p$, $d_ia=0$. The result follows by recalling that the restriction in $\LL{n}$ is the top $\Q$ operation.
\end{proof}
\begin{lem}\label{Q0ZeroByPriddyAlg}
Suppose that $z\in Z\Koverline_p(X)_{a_n,\ldots,a_1}$, for $p>0$, is written as a sum
\[z=\sum_{j}\left[\Q^{i^{{(j)}}_p}\middle|\cdots \middle|\Q^{i^{{(j)}}_1}\right]x^{(j)}\]
such that for all $j$, 
%such that all of the $i^{(j)}_k$ satisfy $i^{(j)}_k>d$, and 
$\Q^ix^{(j)}=0$ whenever $i\geq i^{{(j)}}_1$. Then $\Q^0\overline{z}=0$.

In particular, suppose that a cycle in $\Koverline_p(X)$, for $p>0$, is written as \[z=\sum_{j}Q(I^{(j)})x^{(j)},\]
for some collection $I^{(j)}=(i^{(j)}_{p},\ldots,i^{(j)}_{1})$ of  $\SqShift$-Admissible sequences, and corresponding homogeneous elements $x^{(j)}$ of $X$. Suppose further that, for all $j$, $\Q^ix^{(j)}=0$ whenever $i\geq i^{{(j)}}_1$. Then $\Q^0\overline{z}=0$.
%In particular, suppose that $I^{(j)}=(i^{(j)}_{p},\ldots,i^{(j)}_{1})$ is a collection of $\SqShift$-Admissible sequences, and $x^{(j)}$ is a sequence of homogeneous elements of $X$, such that
%\[z=\sum_{j}Q(I^{(j)})x^{(j)}\]
%is a valid cycle in $\Koverline_p(X)$, and that for all $j$, $\Q^ix^{(j)}=0$ whenever $i\geq i^{{(j)}}_1$. Then $\Q^0\overline{z}=0$.
\end{lem}
\begin{proof}
The second statement follows from the first, as whenever $\produces{J}{I}{\SqShift}$, the final entry of $J$ is at least as large that of $I$.

We've seen that $\Q^0\overline{z}$ is represented by the cycle
\[\sum_{j}\Q^{a_n}_p\Q^{i^{{(j)}}_p}_p\cdots \Q^{i^{{(j)}}_1}_1x^{(j)}\]
in the bar construction $C(\Ind{\nontop{}}\BarConst{\LL{}}X)_*$. This cycle is the image of a cycle in $C(\Ind{\nontop{}}\BarConst{\nontop{}}X)_*$ which, after mapping down to the quotient by degenerate simplices, we dare write as:
\[\sum_{j}\left[\Q^{a_n}\Q^{i^{{(j)}}_p}\middle|\cdots \middle|\Q^{i^{{(j)}}_1}\right]x^{(j)}\in\Boverline_{p,p+1}.\]
Now in order to compress this into the Koszul complex, we must apply the algorithm from the proof of \ref{PriddyAlgProof}. But one observes readily that if $\produces{(i,j)}{(i',j')}{\Q}$, then $j'>j$, so that as we perform the algorithm, we never decrease the index of the final $\Q$ appearing in each bar. Then the hypothesis on the $x^{{j}}$ guarantees that as we run the algorithm, every term that ends up in $\Boverline_{p,p}$ will vanish. Thus the algorithm terminates at zero.
\end{proof}

\end{LieLambdaStructureOnKoszul}



\begin{SequenceOfSequencesIntro}

\section{Calculations on positive spheres}
Choose a positive integer $d$, and write $H:=H^*_Q(S^d)$ for the Andr\'e-Quillen cohomology of the $d$-dimensional sphere. It is our goal to compute $(\derived\Ind{\LL{1}})H$, the $E_2$-page for the Adams spectral sequence. For this, it might be useful to calculate, for each $N\geq1$, the derived functors
\[H^{(N)}:=(\derived\Ind{\nontop{N}})(\derived\Ind{\nontop{N-1}})\cdots (\derived\Ind{\nontop{1}})H\in\LL{N+1}.\]
This we'll do, one step at a time, but before we start, we'll take care to accurately set out the combinatorics involved in describing a basis of $H^{(n)}$.

\subsection{Sequences of admissible sequences}
To parameterize a basis of $H^{(N)}\in\LL{N+1}$ we will use sequences of $N$ admissible sequences, which we shall always write as $(I^N,\ldots,I^1)$. To further simplify notation, we'll employ the following shorthand:
\begin{itemize}
\squishlist
\setlength{\parindent}{.25in}
\item $\ell^i$ is the number of entries of $I^i$;
\item $n^i$ is the sum of the entries of $I^i$;
\item $\minDimP^1$ is $\minDimP(I^1)$, and $\minDim^i$ is $\minDim(I^i)$ for $i\geq2$.
\end{itemize}
Moreover, whenever we index a set of generators over $S=(I^N,\ldots,I^1)$, we will always impose the following conditions on $S$:
\begin{enumerate}[A)]
\squishlist
\setlength{\parindent}{.25in}
\item[\textup{(A)}] $I^1$ is $\delta$-admissible (with entries at least 2) and satisfies $\minDimP^1\leq d$;
\item[\textup{(B)}] for $i\geq2$, $I^i$ is $\SqShift$-admissible (with entries nonnegative);
\item[\textup{(C)}] $I^2$ does not contain zero, and $\minDim^i<\ell^{i-1}$ for all $i\geq2$.
\end{enumerate}
%\begin{itemise}
%\setlength{\parindent}{.25in}
%\item for $i>1$, $I^i$ does not contain zero unless one of $\ell_{i-2},\ldots,\ell_1$ is nonzero (and in particular, $I^2$ does not contain zero). Note that as $I^i$ is $\SqShift$-admissible, it can have at most one zero, which must be its rightmost entry.
%\end{itemise}
%All other restrictions on the $I^i$ will be made explicit.
A basis element corresponding to $S=(I^N,\ldots,I^1)$ will be written $\Q(I^N)\cdots \Q(I^2)P(I^1)\imath$, or $b(S)$, and will have grading%. The element $b(S)$ will have gradings $(a_{N+1},\ldots,a_1)$ equal to
%\[\left(\ell^N,\ell^{N-1}+n^N,2^{\ell^N}\!\left(\ell^{N-2}+n^{N-1}\right), 2^{\ell^N+\ell^{N-1}}\!\left(\ell^{N-3}+n^{N-2}\right),\ldots,2^{\ell^N+\cdots+\ell^{3}}(\ell^1+n^2),\ a_1\ \right)\]
\begin{align*}
 (a_{N+1},\ldots,a_1) &= \left( \ell^N,\ell^{N-1}+n^N,2^{\ell^N}\!\left(\ell^{N-2}+n^{N-1}\right), 2^{\ell^N+\ell^{N-1}}\!\left(\ell^{N-3}+n^{N-2}\right),\ldots \right.\\
 &\qquad \left.{}\ldots,2^{\ell^N+\cdots+\ell^{4}}(\ell^2+n^3),2^{\ell^N+\cdots+\ell^{3}}(\ell^1+n^2),2^{\ell^N+\cdots+\ell^{2}}(d+n^1+\ell^1+1)-1 \right).
\end{align*}
The formula for $a_1$ is a little different to those for the other $a_i$ due to the $+1$-shift.
%where this pattern continues up until the final grading $a_1$, at which point the $+1$-shift muddies the water. 
We note some immediate consequences of the conditions just imposed on the $I^i$, after defining a further condition that $S$ may satisfy:
%\begin{itemise}
%\setlength{\parindent}{.25in}
%\item Suppose that $\ell^i=0$, then $0=\ell^{i+1}=\ell^{i+2}=\ell^{i+3}=\cdots $;
%\item if $\ell^N>0$ for $N\geq3$, then both $a_N=\ell^{N-1}+n^N\neq0$ and $a_{N-1}=2^{\ell^N}\!\left(\ell^{N-2}+n^{N-1}\right)\neq0$.
%\end{itemise}
%\begin{lem}
%The conditions we have put on the $I^i$ in $\Q(I^N)\cdots \Q(I^2)P(I^1)\imath$ are equivalent to the following requirements:
%\begin{itemise}
%\setlength{\parindent}{.25in}
%\item[i)] $I^1$ is $\delta$-admissible (with entries at least 2) and satisfies $\minDimP^1\leq d$;
%%\item for $i>1$, $I^i$ is $\SqShift$-admissible (with entries nonnegative) and satisfies $\minDim^i<\ell^{i-1}$; and
%%\item $I^1$ is $\delta$-admissible (with entries at least 2) and for $i>1$, $I^i$ is $\SqShift$-admissible (with entries nonnegative);
%%\item there exists an object $X$ of $\LL{1}$, and an element $x\in X$ with $|x|=d$, such that $P^{I^1}x$ is nonzero;
%\item[ii)] for each $i$ satisfying $2\leq i\leq N$, $I^i$ is $\SqShift$-admissible (with entries nonnegative), and the operation $\Q^{I^i}$ is defined on classes of dimension $\|\Q(I^{i-1})\cdots \Q(I^2)P(I^1)\imath\|$ in objects of $\nontop{i}$.
%%
%% there exists an object $X$ of $\LL{i}$, and an element $x\in X$ with $\|x\|=\|\Q(I^{i-1})\cdots \Q(I^2)P(I^1)\imath\|$, such that $\Q^{I^i}x$ is defined, and none of the $\Q$ in this expression act as a top operation.
%\end{itemise}
%\end{lem}
\begin{enumerate}[A)]
\squishlist
%%%%%%%%%%%%% A2 is a silly misconception! Don't put it in here. \item[(A2)] \textbf{WRONG}for some $X\in\nontop{1}$ (or equivalently $\LL{1}$) and some element $x\in X_d$, $P^{I^1}x$ is nonzero.
\item[(C2)] for each $i$ satisfying $2\leq i\leq N$, and each object $X\in\nontop{i}$ the operation $\Q^{I^i}$ is defined on classes of dimension $\|\Q(I^{i-1})\cdots \Q(I^2)P(I^1)\imath\|$ in $X$.
\end{enumerate}
\begin{lem}
Suppose that $N\geq1$ and that $S=(I^N,\ldots,I^1)$. Write $\left\|b(S)\right\|=(a_{N+1},\ldots,a_1)$. Then
\begin{itemise}
%\setlength{\parindent}{.25in}
\item[i)]  If $S$ satisfies \textup{(A)}-\textup{(C)}, and if $\ell^i=0$, then $\ell^{i+1}=\ell^{i+2}=\cdots=\ell^{N}=0$;
\item[ii)]  If $S$ satisfies \textup{(A)}-\textup{(C)}, then $\ell^i\neq0$ iff $a_{i+1}\neq0$ iff none of $a_{i+1},\ldots,a_1$ are zero; %the final $i+1$ gradings of $\Q(I^{N})\cdots \Q(I^2)P(I^1)\imath$ are all nonzero;%both $a_N=\ell^{N-1}+n^N\neq0$ and $a_{N-1}=2^{\ell^N}\!\left(\ell^{N-2}+n^{N-1}\right)\neq0$.
\item[iii)] If $S$ satisfies \textup{(A)} and \textup{(B)}, then $S$ satisfies \textup{(C)} iff $S$ satisfies \textup{(C2)}.
\end{itemise}
\end{lem}
\noindent Statement (iii) would be obvious if \textup{(C)} contained appropriate conditions on the appearance of zeros in $I^i$ for $i\geq3$. The point then is that \textup{(A)}-\textup{(C)} together make these conditions redundant.
%
%\noindent The point of the final statement is that we do not need an extra condition on when the $I^i$, for $i\geq3$, can contain zero, in order that the second statement of the lemma holds --- this is already wrapped up in the condition $m^i<\ell^{i-1}$ and the condition that $I^2$ doesn't contain zero.
%
%\begin{proof}
%The first condition is common to the two sets of conditions we are comparing, so can be ignored.
%The case $i=2$ is easy, as no $\Q^0$ operations are ever defined in $\nontop{2}$, leaving us those cases with $i\geq3$. Now in order that $\Q^{I^i}$ is defined on classes of dimension $\|\Q(I^{i-1})\cdots \Q(I^2)P(I^1)\imath\|$ in objects of $\nontop{i}$, we must have $\minDim^i<|\Q(I^{i-1})\cdots \Q(I^2)P(I^1)|=\ell^{i-1}$. Finally we must check that the conditions $\minDim^j<\ell^{j-1}$ for all $j$ imply that there is no zero appearing in $\Q^{I^i}$, for $i\geq3$, is erroneous. But if $I^i$ contains a zero, then $\ell^i>0$, so that $\ell^{i-1},\ell^{i-2}>0$, and writing $\|\Q(I^{i-1})\cdots \Q(I^2)P(I^1)\imath\|=(a_i,\ldots,a_1)$, we have that $a_{i}=\ell^{i-1}>0$ and $a_{i-1}=\ell^{i-2}+n^{i-1}>0$, so that $\Q^0$ is indeed defined on such classes.
%\end{proof}

\begin{proof}
To prove (i), note that if $\ell^i=0$, then $\minDim^{i+1}<\ell^i=0$, which can only occur if $I^{i+1}=\emptyset$. (ii) follows immediately from (i). Thus we are to show that, when \textup{(A)} and \textup{(B)} hold, \textup{(C)} and (iii) are equivalent. In what follows, write $(a'_i,\ldots,a'_1)=\|\Q(I^{i-1})\cdots \Q(I^2)P(I^1)\imath\|$.

%The $i=2$ part is easy, as no $\Q^0$ operations are ever defined in $\nontop{2}$, leaving us those cases with $i\geq3$. 
Evidently, \textup{(C2)} implies \textup{(C)}, as in  order that $\Q^{I^i}$ is defined on classes of dimension $(a'_i,\ldots,a'_1)$ in objects of $\nontop{i}$, we must have $\minDim^i<a'_i=\ell^{i-1}$, and no $\Q^0$ operations are ever defined on objects in $\nontop{2}$.

To check that \textup{(A)}-\textup{(C)} imply \textup{(C2)}, recall that there are two conditions to be checked in order that $\Q^{I^i}$ can act. Firstly, there is the condition on $\minDim(I^i)$, which is present in \textup{(C)}. The other condition required is that $I^i$ contains no zero unless one of $a'_{i-1},\ldots,a'_{2}$ is nonzero.
%concerns the appearance of zeros in $I^i$. 
As \textup{(C)} explicitly forbids zeros in $I^2$, we may restrict attention to when $i\geq3$. %Now as $I^i$ is $\SqShift$-admissible, all of the zeros therein appear at the right, so the question comes down to whether a \emph{single} $\Q^0$ can appear.
If $I^i$ contains a zero, then $\ell^i>0$, and part (ii) implies that $\ell^{i-1}>0$. Part (ii), applied to the shorter sequence $(I^{i-1},\ldots,I^1)$, shows that all $i$ entries of $(a'_i,\ldots,a'_1)$ are nonzero. As $i\geq3$, this means that $\Q^0$ is indeed defined where required in $\nontop{i}$.
%
% Finally we must check that the conditions $\minDim^j<\ell^{j-1}$ for all $j$ imply that there is no zero appearing in $\Q^{I^i}$, for $i\geq3$, is erroneous. But if $I^i$ contains a zero, then $\ell^i>0$, so that $\ell^{i-1},\ell^{i-2}>0$, and writing $\|\Q(I^{i-1})\cdots \Q(I^2)P(I^1)\imath\|=(a_i,\ldots,a_1)$, we have that $a_{i}=\ell^{i-1}>0$ and $a_{i-1}=\ell^{i-2}+n^{i-1}>0$, so that $\Q^0$ is indeed defined on such classes.
\end{proof}

Before we go on, it is convenient to introduce two more conditions that $S=(I^N,\ldots,I^1)$ may satisfy, for $N\geq2$. These will appear unmotivated until later:
\begin{itemise}
\setlength{\parindent}{.25in}
\item[(D)] $I^2$ doesn't contain 1.
\item[(E)] $I^1\in\im(t_d)$, $\ell^1>2$, and $I^2$ doesn't contain 1 or 2.
\end{itemise}
Here, $t_d$ is the function which adds a top $P$ operation to a \emph{nonempty} $\delta$-admissible sequence%footnote:
\footnote{Note that we can characterise the image of $t_d$ as the set of $I^1$ of length at least 2 satisfying the excess condition $e^1=d+\ell^1-1$.}:
\begin{lem}\label{tdPreservesA}
Consider the operation $t_d$ which sends a nonempty sequence  $I^1$ to $(d+n^1+\ell^1)\star I^1$. Then $t_dI^1$ satisfies \textup{(A)} if and only if $I^1$ satisfies \textup{(A)}.
\end{lem}
\begin{proof}
The \emph{only if} part is immediate, as removing the leftmost entry of a sequence only makes it more likely to be admissible, and cannot increase $\minDimP$. Assume then that $I^1=(i_{\ell^1},\ldots,i_1)$ satisfies \textup{(A)}, with $\ell^1\geq1$. Then:%Now by definition of $\minDimP$,
\[\minDimP(t_dI)=\max\{\minDimP(I),(d+n^1+\ell^1-i_{\ell^1}-\cdots -i_1-\ell^1)\}=\max\{\minDimP(I),d\}=d.\]
We should check that $t_dI^1$ is $\delta$-admissible. We only need to check that $d+n^1+\ell^1\geq2i_{\ell^1}$, or $d+1\geq2i_{\ell^1}-n^1-\ell^1+1$, but the right hand side of this inequality is a term in the max defining $\minDimP(I)$, so this last inequality holds (even strictly).
\end{proof}
\end{SequenceOfSequencesIntro}


\begin{KoszulSequenceCombinatorics}
\subsection{Combinatorics of  the sequences $(I^N,\ldots,I^1)$}
\begin{Omitted}
Suppose that $k\geq2$, and that we've already chosen $(I^{k-1},\ldots,I^1)$ satisfying \textup{(A)}-\textup{(C)}. One may ask the question ``which sequences $I^k$ can we choose so that \textup{(A)}-\textup{(C)} hold for $(I^{k},\ldots,I^1)$?''
\begin{prop}
Suppose that $S=(I^N,\ldots,I^1)$ satisfies \textup{(A)}-\textup{(C)}. Then the sequence $(\ell^N,\ell^{N-1},\ldots,\ell^1)$ is nondecreasing, and $I^k=(\ldots,7,3,1,0)$ whenever $\ell^k=\ell^{k-1}$.%, it must be that $I^k=(\ldots,7,3,1,0)$.
\end{prop}
\begin{proof}
To investigate this question, for a given choice $I^k$ with $\ell^{k}>0$, we define the following two quantities:
%We define the following two quantities, for $k\geq2$, which relate to the question ``which sequences $I^k$ are permitted given that $I^{k-1}$ has already been chosen?'' under constraints \textup{(A)}-\textup{(C)} as always. Define the following two quantities when $\ell^k>0$:
\begin{itemise}
\setlength{\parindent}{.25in}
\item Let $\widetilde{e}^k=\minDim^k-\ell^{k}+1$. This is a nonnegative integer, as if we write $I^k=(i_{\ell^k},\ldots,i_1)$, then, as $I^k$ is $\SqShift$-admissible, $\minDim^k=i_{\ell^k}-i_{\ell^k-1}-\cdots -i_1$, and so:
\[\widetilde{e}^k=(i_{\ell^k}-2i_{\ell^k-1}-1)+ (i_{\ell^k-1}-2i_{\ell^k-2}-1)+\cdots + (i_{2}-2i_{1}-1)+(i_1)\geq0.\]
\item Let $f^k=\ell^{k-1}-1-\minDim^k$. This too is nonnegative, as $\minDim^k<\ell^{k-1}$.
\end{itemise}
These two quantities measure how wasteful we have been in choosing $I^k$, if our objective is to make $\ell^k$ as large as possible:
\begin{itemise}
\setlength{\parindent}{.25in}
\item $\widetilde{e}^k$ is the total amount of waste caused, first by choosing $i_1$ unnecessarily high, then by choosing $i_2$ unnecessarily high (subject to the admissibility constraint imposed by $i_1$), etc. [Note that choosing the entries of $I^k$ higher than they need to be cramps us for space, leading to shorter sequences being allowed.]
\item $f^k$ is the waste caused by choosing $I^k$ an unnecessarily short sequence. [Note that given a $\SqShift$-admissible sequence $I$, one can always create another $\SqShift$-admissible sequence $I'$ by adding a single entry onto the left which is as small as possible. This increases both $\ell$ and $\minDim$ by one. If we are trying to maximise $\ell$, we should really do this to $I^k$ whenever $\minDim^k$ is any less than $\ell^{k-1}-1$.]
\end{itemise}
The point, then, is that whenever $\ell^k>0$ for $k\geq2$, $\ell^k=\ell^{k-1}-\widetilde{e}^k-f^k$.
\end{proof}
\end{Omitted}

\begin{lem}
Suppose that $I\in\admis{\SqShift}$ with $\ell(I)>0$. Then $\ell(I)\leq \minDim(I)+1$, with equality iff $I=(\ldots,7,3,1,0)$. Moreover, $\minDim(I)=e(I)$, the Serre excess of $I$. The following inequalities hold:
\begin{equation}
2^{\ell(I)}-1-\ell(I)\leq n(I)\leq2^{\minDim(I)+1}-1-(\minDim(I)+1).\tag{i.1) \& (i.2}
\end{equation}
If $I$ does not contain 0, then \textup{(i.1)} can be strengthened:
\[2(2^{\ell(I)}-1)-\ell(I)\leq n(I),\]
and if $I$ does not contain either 0 or 1, both \textup{(i.1)} and \textup{(i.2)} can be strengthened:
\[3(2^{\ell(I)}-1)-\ell(I)\leq n(I)\leq 3\cdot2^{\minDim(I)-1}-\minDim(I)-2.\]
%the first inequality can be strengthened to $n(I)\geq 3(2^{\ell(I)}-1)-\ell(I)$.
\end{lem}
\begin{proof}
Note first that there is a bijection, for $\ell\geq1$:
\begin{alignat*}{2}
\gamma:\left\{(m_{\ell},\ldots,m_1)\ \middle|\ m_\ell,\ldots,m_2\geq1,\ m_1\geq0\right\}&\overset{\cong}{\to}\left\{\textup{$\SqShift$-admissible sequences of length $\ell$}\right\}
\\
(m_\ell,\ldots,m_1)
&\overset{\gamma}{\mapsto}
(\ldots,2^2m_1+2m_2+m_3,2m_1+m_2,m_1)\\
(i_\ell-2i_{\ell-1},\ldots,i_2-2i_1,i_1)
&\overset{\gamma^{-1}}{\mapsfrom}
(i_\ell,\ldots,i_1)%
\end{alignat*}
%The expression for $m(\gamma(\textbf{m}))$ is
Formulae for $m(\gamma(\textbf{m}))$ and $n(\gamma(\textbf{m}))$ are easy to write down:
\[m(\gamma(\textbf{m}))=\max\left\{m_j+\cdots +m_1\right\}_{j=1}^\ell=m_\ell+\cdots +m_1=:e(\gamma(\textbf{m}));\]
\[%\minDim(\gamma(\textbf{m})):=\max\{m_{\ell}+\cdots +m_1,\ldots,m_2+m_1,m_1\}=\sum_{i=1}^{\ell}m_i;\quad
n(\gamma(\textbf{m}))=\sum_{i=0}^{\ell-1}\left(2^{i+1}-1\right)m_{\ell-i}=2i_\ell-\sum_{i=0}^\ell m_i.\]
In particular, as all of $m_\ell,\ldots,m_2$ are positive, $\minDim(\gamma(\textbf{m}))\geq\ell-1$, with equality iff $\textbf{m}=(1,\ldots,1,0)$, proving the first claim.
The condition that $\gamma(\textbf{m})$ contains no 0 (resp.\ no 0 or 1) translates simply to $m_1\geq1$ (resp.\ $m_1\geq2$), so that \textup{(i.1)} and its two strengthenings are easy:
\[n(\gamma(\textbf{m}))=\sum_{i=0}^{\ell-1}\left(2^{i+1}-1\right)m_{\ell-i}\geq(2^{\ell}-1)m_1+ \sum_{i=0}^{\ell-2}\left(2^{i+1}-1\right)
=(m_1+1)(2^{\ell}-1)-\ell.\]
For \textup{(i.2)}, fix $M$, and consider those $\textbf{m}$ for which $\minDim(\gamma(\textbf{m}))=M$, so that $M=\sum_im_i$. Then $n(\gamma(\textbf{m}))$ cannot exceed $n(\gamma((1,\ldots,1)))$, where the $1$ is repeated $M$ times. To see this, note that $n(\gamma(\textbf{m}))$ increases as $\textbf{m}$ becomes more weighted towards $m_i$ for $i$ low, so that the maximum must occur when $\textbf{m}$ is of the form $(1,\ldots,1,a)$. But $n(\gamma((1,\ldots,1,a)))\leq n(\gamma((1,\ldots,1,1,a-1)))$ whenever $a\geq1$, with equality iff $a=1$, so $\minDim(\gamma(\textbf{m}))$ is maximised on $\textbf{m}=(1,\ldots,1)$, with $\ell=M$, and:
\[n(\gamma(\textbf{m}))=\sum_{i=0}^{M-1}(2^{i+1}-1)=2^{M+1}-M-2.\]
To prove the strengthening of \textup{(i.2)}, a similar argument, after restricting to those $\textbf{m}$ with $m_1\geq2$, shows that $n(\gamma(\textbf{m}))$ is maximised when $\textbf{m}=(1,\ldots,1,2)$, with $\ell=M-1$. Then
\[n(\gamma(\textbf{m}))=(2^{M-1}-1)\cdot2+\sum_{i=0}^{(M-1)-2}(2^{i+1}-1)=3\cdot2^{M-1}-M-2.\qedhere\]
\end{proof}
%\begin{cor}Suppose that $N\geq2$ and $(I^N,\ldots,I^1)$ satisfies \textup{(A)}-\textup{(D)}.
%Then the sequence $(n^N,\ldots,n^1)$ is nondecreasing.%, and whenever $n^k=n^{k-1}$, it must be that $\ell^k=\ell^{k-1}$, and $I^k=(\ldots,7,3,1,0)$.
%\end{cor}
%\begin{cor}Suppose that $N\geq2$ and that $(I^N,\ldots,I^1)$ satisfies \textup{(A)}-\textup{(D)} with $\ell^N>0$.
%Then the sequence $(n^N,\ldots,n^2,n^1/2)$ is nondecreasing, and there's a chain of inequalities, for $N\geq3$:
%\[\ell^N\leq\ell^{N-1}+n^N\leq\ell^{N-2}+n^{N-1}\leq\cdots \leq \ell^2+n^3\leq(\ell^1+n^2)/3.\]
%%\[\ell^1+n^2\geq3(\ell^2+n^3),\]
%If $N=2$ we just obtain $\ell^2\leq(\ell^1+n^2)/3$.
%\end{cor}
%
%\begin{proof}
%Write $g$ for the integer sequence $i\mapsto 2^i-i-1$. Then $g(0)=g(1)=0$, and $g(i-1)<g(i)$ for $i\geq2$.
%%, and also, $g(i)\leq 2^{i+1}-2$ for $i\geq0$. 
%We've seen that $g(\ell^k)\leq n^k\leq g(\minDim^k+1)$ for all $k\geq2$. That $g$ is nondecreasing, taken with the inequality from constraint \textup{(C)}, gives a chain
%\[\cdots \leq n^k\leq g(\minDim^k+1)\leq g(\ell^{k-1})\leq n^{k-1}\leq\cdots\leq n^2\leq g(\minDim^2+1)\leq g(\ell^{1}). \]
%Although $I^1$ doesn't fit into the same pattern, $n^1$ can be no less than $\sum_{i=1}^{\ell^1}2^i=2^{\ell^1+1}-2$, as a $\delta$-admissible sequence only involves indices of size \emph{at least $2$}. We can then complete the above chain with further inequalities $g(\ell^1)\leq2^{\ell^1}-1\leq n^1/2$. This proves that $(n^N,\ldots,n^2,n^1/2)$ is nondecreasing. As we've shown that $(\ell^N,\ldots,\ell^1)$ is nondecreasing, we've actually shown all  we set out to prove, except that $\ell^1+n^2\geq3(\ell^2+n^3)$.
%
%$I^2$ can contain no 0 or 1, so that $3(2^{\ell^2}-1)-\ell^2\leq n^2$. Moreover, we've just shown that $n^3\leq g(\ell^2)=2^{\ell^2}-\ell^2-1$.
%%Combine these two inequalities with $\ell^1\geq \ell^2$ to get the last result.
%Then $\ell^1+n^2\geq n^2+\ell^2\geq3(2^{\ell^2}-1)\geq3(\ell^2+n^3)$.
%\end{proof}
%\begin{prop}
%Suppose that $N\geq2$ and that $(I^N,\ldots,I^1)$ satisfies \textup{(A)}-\textup{(D)} with $\ell^N>0$. Then $a_2\geq 3(a_3+\cdots +a_{N+1})$.
%\end{prop}
%\begin{proof}
%When $N=2$, this is just the final inequality in the previous corollary, so assume $N\geq3$. In this case, the same corollary's chain of inequalities may be restated as:
%\[a_N\geq a_{N+1},\  a_{N-1}\geq 2^{\ell^{N}}a_N,\ \ldots,\ a_3\geq 2^{\ell^{4}}a_4,\ a_2/3\geq 2^{\ell^{3}}a_3,\]
%and all of the $\ell^i$ are at least 1. Thus
%\[a_2/3\geq 2a_3\geq a_3+2a_4\geq a_3+a_4+2a_5\geq\cdots \geq a_3+\cdots +a_{N+1}.\qedhere\]
%\end{proof}
%\begin{prop}
%Writing $\left\|\Q(I^N)\cdots \Q(I^2)P(I^1)\imath\right\|=(a_{N+1},\ldots,a_{1})$, suppose that $\ell^N\neq0$. Then $a_i\geq 2a_{i+1}$ for all $i$ with $2\leq i\leq N-1$. Moreover, $a_{N}\geq2a_{N+1}$ unless $I^N=(0)$ and $\ell^{N-1}=1$, when $a_{N+1}=a_N=1$.
%\end{prop}
%\begin{proof}
%When $i=N$, it's enough to note that $n^N\geq\ell^N$ unless $I^N=(0)$. For the cases $2\leq i\leq N-1$, one uses the fact that the sequences $(n^N,\ldots,n^1)$ and $(\ell^N,\ldots,\ell^1)$ are nondecreasing.
%\end{proof}

%\begin{cor}Suppose that $N\geq1$ and that $(I^N,\ldots,I^1)$ satisfies \textup{(A)}-\textup{(C)} with $\ell^N>0$.
%Then if $N\geq2$ there is a chain of $N-1$ inequalities:
%\[n^N\leq n^{N-1}\leq\cdots \leq n^2\leq \tfrac{1}{2}n^1,\]
%and a chain of $N$ inequalities:
%\[(\ell^N)\leq(\ell^{N-1}+n^N)\leq(\ell^{N-2}+n^{N-1})\leq\cdots \leq (\ell^2+n^3)\leq\tfrac{1}{2}(\ell^1+n^2)\leq \tfrac{1}{3}(n^1+\ell^1).\]
%When $S$ also satisfies \textup{(D)}, we can strengthen the second last inequality:
%\[(\ell^N)\leq(\ell^{N-1}+n^N)\leq(\ell^{N-2}+n^{N-1})\leq\cdots \leq (\ell^2+n^3)\leq\tfrac{1}{3}(\ell^1+n^2)\leq \tfrac{2}{9}(n^1+\ell^1).\]
%%$(n^N,\ldots,n^2,n^1/2)$ is nondecreasing, and there's a chain of inequalities, for $N\geq3$:
%%\[\ell^N\leq\ell^{N-1}+n^N\leq\ell^{N-2}+n^{N-1}\leq\cdots \leq \ell^2+n^3\leq(\ell^1+n^2)/3.\]
%%\[\ell^1+n^2\geq3(\ell^2+n^3),\]
%If $N=1$, the stronger inequality $\ell^1\leq\tfrac{1}{3}(n^1+\ell^1)$ is satisfied.
%\end{cor}


\begin{cor}Suppose that $N\geq1$ and that $(I^N,\ldots,I^1)$ satisfies \textup{(A)}-\textup{(C)}. % with $\ell^N>0$.
Then if $N\geq2$ there are two chains of $N-1$ inequalities and two chains of $N-2$ inequalities:
\begin{equation}
\ell^N\leq \ell^{N-1}\leq\cdots \leq \ell^2\leq \ell^1;\tag{c.0}
\end{equation}
\begin{equation}
n^N\leq n^{N-1}\leq\cdots \leq n^2\leq \tfrac{1}{2}n^1;\tag{c.1}
\end{equation}
\begin{equation}
\minDim^N\leq \minDim^{N-1}\leq\cdots \leq \minDim^2;\tag{c.2}
\end{equation}
\begin{equation}
(\ell^N)\leq(\ell^{N-1}+n^N)\leq(\ell^{N-2}+n^{N-1})\leq\cdots \leq (\ell^2+n^3).\tag{c.3}
\end{equation}
Moreover (reading $n^3$ as zero when $N=2$):
\begin{equation}
2(\ell^2+n^3)\leq(\ell^1+n^2)\textup{\ and\ }3(\ell^1+n^2)\leq 2(n^1+\ell^1).\tag{i.3) \& (i.4}
\end{equation}
When $N=1$, we obtain only one inequality, a stronger version $3\ell^1\leq(n^1+\ell^1)$ of \textup{(i.4)}.
Finally, when $N\geq2$ and $S$ also satisfies \textup{(D)}, we have a stronger version $3(\ell^2+n^3)\leq(\ell^1+n^2)$ of \textup{(i.3)}.
\end{cor}

\begin{proof}
Write $g(i):=2^i-i-1$ for $i\geq0$, and $g(-\infty)=0$. Then
\[0=g(-\infty)=g(0)=g(1)<g(2)<g(3)<\cdots,\quad \textup{and}\quad g(i)\leq 2^i-1 \textup{ for $i\geq0$}.\]
Inequalities \textup{(i.1)} and \textup{(i.2)} state that $g(\ell^k)\leq n^k\leq g(\minDim^k+1)$ for all $k\geq2$. %Evidently, $g(0)=g(1)=0$, and $g(i)>g(i-1)$ for $i\geq2$. For convenience, write $g(-\infty)$
%, and also, $g(i)\leq 2^{i+1}-2$ for $i\geq0$.
Moreover, $n^1$ can be no less than $\sum_{i=1}^{\ell^1}2^i=2^{\ell^1+1}-2$, as a $\delta$-admissible sequence only involves indices of size \emph{at least $2$}, implying that $2^{\ell^1}-1\leq n^1/2$. This fact, along with the obeservation that $g$ is nondecreasing, and the inequality $m^i<\ell^{i-1}$ from \textup{(C)}, gives a chain:
\begin{multline*}
g(\ell^N)\leq n^N\leq g(\minDim^N+1)\leq\cdots\\\cdots \leq g(\ell^k)\leq n^k\leq g(\minDim^k+1)\leq g(\ell^{k-1})\leq n^{k-1}\leq g(\minDim^{k-1}+1)\leq\cdots\\
\cdots\leq n^2\leq g(\minDim^2+1)\leq g(\ell^{1})\leq 2^{\ell^1}-1\leq \tfrac{1}{2}n^1
\end{multline*}
This proves \textup{(c.1)}, which is a subchain. Although $g$ is not strictly increasing near zero, \textup{(c.0)} follows anyway, since $\ell^k=0$ implies that $\ell^{k+1}=0$. Equivalently, $m^k=-\infty$ implies $m^{k+1}=-\infty$, and \textup{(c.2)} follows. \textup{(c.3)} is immediate from \textup{(c.0)} and \textup{(c.1)}.
%Although $I^1$ doesn't fit into the same pattern, $n^1$ can be no less than $\sum_{i=1}^{\ell^1}2^i=2^{\ell^1+1}-2$, as a $\delta$-admissible sequence only involves indices of size \emph{at least $2$}. We can then augment the above chain with further inequalities $g(\ell^1)\leq2^{\ell^1}-1\leq n^1/2$, verifying \textup{(c.1)}. As $(\ell^N,\ldots,\ell^1)$ is nondecreasing, \textup{(c.1)} implies \textup{(c.3)}.%, and also the $N=1$ case of \textup{(i.4)}.

The inequality $3\ell^1\leq(n^1+\ell^1)$ in fact holds for all $N$, and follows from the observation that $n^1\geq 2^{\ell^1+1}-2\geq 2\ell^1$. This proves the version of \textup{(i.4)} relevant when $N=1$. When $N\geq2$, \textup{(i.4)} follows from established inequalities:
\[2(n^1+\ell^1)-3(\ell^1+n^2)=\tfrac{3}{2}(n^1-2n^2)+\tfrac{1}{2}(n^1-2\ell^1)\geq0+0=0.\]
This leaves the two versions of \textup{(i.3)}. Write $\rho=3$ when \textup{(D)} holds, and $\rho=2$ otherwise. Then
%Let's demonstrate that $(\ell^1+n^2)\geq3(\ell^2+n^3)$ when $N\geq2$ (where if $N=2$, we read $n^3$ as zero). As $I^2$ can contain no 0 or 1, 
the relevant version of \textup{(i.1)} is $\rho(2^{\ell^2}-1)-\ell^2\leq n^2$. Moreover, we've just shown that $n^3\leq g(\ell^2)=2^{\ell^2}-\ell^2-1$.
Then, using \textup{(c.0)}, $(\ell^1+n^2)\geq (n^2+\ell^2)\geq\rho(2^{\ell^2}-1)\geq\rho(\ell^2+n^3)$.
\end{proof}
\begin{prop}\label{InequalitiesONai}
Suppose that $N\geq1$ and that $S=(I^N,\ldots,I^1)$ satisfies \textup{(A)}-\textup{(C)} with $\ell^N>0$. Let $(a_{N+1},\ldots,a_1)$ be the degree of $b(S)$.
\begin{enumerate}[i)]
\setlength{\parindent}{.25in}
\squishlist
\item Suppose that $N=1$. Then
\[a_1\geq3a_2+d.\]
\item Suppose that $N\geq2$. Then 
\[a_1\geq2(a_2+\cdots +a_{N+1})+2^{N-1}(d+1)-1\textup{ \ and \ }a_2\geq2(a_3+\cdots +a_{N+1}).\]
and when $S$ also satisfies \textup{(D)}, these inequalities may be strengthened to
\[a_1\geq\tfrac{9}{4}(a_2+\cdots +a_{N+1})+2^{N-1}(d+1)-1\textup{ \ and \ }a_2\geq3(a_3+\cdots +a_{N+1}).\]


\item Suppose that $N\geq2$. Then 
\[a_1\geq2(a_2+\cdots +a_{N+1})+2^{N-1}(d+1)-1\textup{ \ and \ }a_2\geq2(a_3+\cdots +a_{N+1}).\]
and when $S$ also satisfies \textup{(D)}, these inequalities may be strengthened to
\[a_1\geq\tfrac{9}{4}(a_2+\cdots +a_{N+1})+2^{N-1}(d+1)-1\textup{ \ and \ }a_2\geq3(a_3+\cdots +a_{N+1}).\]


\item Suppose that $N\geq3$. Then for $i$ satisfying $3\leq i\leq N$,
\[a_i\geq a_{i+1}+\cdots +a_{N+1}.\]
\end{enumerate}
In particular, if $d\geq2$ and $S$ satisfies $\textup{(A)}-\textup{(D)}$, then $a_1\geq\tfrac{9}{4}(a_2+\cdots +a_{N+1}+1)$.
\end{prop}
\begin{proof}
Suppose that $N=1$, so that $(a_2,a_1)=(\ell^1,d+n^1+\ell^1)$. Then the inequality required is equivalent to $n^1\geq2\ell^1$, which we established in the previous proof, demonstrating (i).

Now suppose that $N\geq2$, and as before, write $\rho=3$ when \textup{(D)} holds, and $\rho=2$ otherwise. As all of the $\ell^i$ are at least 1, \textup{(c.3)} and \textup{(i.3)} imply:
\begin{itemise}
%\setlength{\parindent}{.25in}
\item $\rho a_3\leq a_2$ when $N=2$;
\item $a_4\leq a_3$ and $2\rho a_3\leq a_2$ when $N=3$;
\item $a_5\leq a_4$, $2a_4\leq a_3$ and $2\rho a_3\leq a_2$ when $N=4$;
\item $a_6\leq a_5$, $2a_5\leq a_4$, $2a_4\leq a_3$ and $2\rho a_3\leq a_2$ when $N=5$; etc.
\end{itemise}
These imply (iii), as when $3\leq i\leq N$:
\[a_i\geq 2a_{i+1}\geq a_{i+1}+2a_{i+2}\geq\cdots \geq a_{i+1}+\cdots +a_{N+1}.\]
This also proves the second inequality in each part of (ii), which is simply $\rho a_3\leq a_2$ when $N=2$, and follows from (iii) when $N\geq3$:
\[a_2\geq 2\rho a_3\geq \rho a_3+\rho (a_4+\cdots +a_{N+1})=\rho (a_3+\cdots +a_{N+1}).\]
%Note that when $N=2$, the second inequality in each case of (ii) is simply $\rho a_3\leq a_2$, which is already present in the above list.

All that remains is the first inequality in each part of (ii), which relates $\overline{a}_1$ with $(a_2+\cdots a_{N+1})$, where, for brevity, we write $\overline{a}_1:=a_1-(2^{\ell^N+\cdots +\ell^2}(d+1)-1)$.
%\[\overline{a}_1:=a_1-(2^{\ell^N+\cdots +\ell^2}(d+1)-1)=2^{\ell^N+\cdots +\ell^2}(n^1+\ell^1).\]
By \textup{(i.4)}, and as $\ell^2>0$:%, and the relevant version of \textup{(i.3)}:
\[\overline{a}_1:=2^{\ell^N+\cdots \ell^2}(n^1+\ell^1)\geq 3\cdot2^{\ell^N+\cdots +\ell^2-1}(\ell^1+n^2)\geq 3\cdot2^{\ell^N+\cdots +\ell^3}(\ell^1+n^2) =3a_2.\]
We've also shown that $a_2\geq\rho (a_3+\cdots +a_{N+1})$. Thus:
\[(\rho+1)\overline{a}_1\geq 3\rho a_2+3a_2 \geq 3\rho a_2+3\rho (a_3+\cdots a_{N+1})=3\rho(a_2+\cdots a_{N+1}),\]
proving that $\overline{a}_1\geq\frac{3\rho}{\rho+1}(a_2+\cdots a_{N+1})$,  the first inequality in (ii).
\end{proof}

\begin{cor}\label{boundOnA1}Suppose that $N\geq1$ and that $S=(I^N,\ldots,I^1)$ satisfies \textup{(A)}-\textup{(C)}. Suppose further that $S\neq(\emptyset,\ldots,\emptyset)$. Then (without assuming that $\ell^N>0$):
\[a_1\geq2(a_2+\cdots +a_{N+1}+1).\]
If $S$ also satisfies \textup{(D)}, then the factor $2$ can be improved to $\tfrac{9}{4}$.
\end{cor}
\begin{proof}
We can assume, without loss of generality, that $\ell^N>0$. Now when $d=1$, \textup{(A)} cannot be satisfied unless $I^1=\emptyset$, but this implies that $S=(\emptyset,\ldots,\emptyset)$, so the statement is vacuous unless $d\geq2$. If $N=1$, since $a_2=\ell^1>0$, whether or not $S$ satisfies \textup{(D)}, one finds
\[a_1\geq 3a_2+d \geq \tfrac{9}{4}a_2+ \tfrac{3}{4}+d\geq\tfrac{9}{4}a_2+\tfrac{11}{4}.\]
Finally, in case $N\geq2$ and $d\geq2$, one finds $2^{N-1}(d+1)-1\geq5>\tfrac{9}{4}$.
\end{proof}
\end{KoszulSequenceCombinatorics}



\begin{CalculatingRepeatedKoszul}
\subsection{Calculation of $H^{(N)}:=(\derived\Ind{\nontop{N}})\cdots (\derived\Ind{\nontop{1}})H$}

What remains to be done then is to calculate the $H^{(N)}$, by calculating the Koszul complex on $H^{(N-1)}$ along with its differential, and taking homology. If there were no differentials at any stage, $H^{(N)}$ would have basis the $\Q(I^N)\cdots \Q(I^2)P(I^1)\imath$ for which $(I^N,\ldots,I^1)$ satisfies \textup{(A)}-\textup{(C)}. This is almost true: except for the Koszul complex on $H^{(1)}$ which calculates $H^{(2)}$, all of the Koszul complexes have zero differential. Thus, the correct subquotient of the above vector space takes a rather simple form.



%\begin{prop}
%As a graded vector space,
%\[H^{(0)}=\left\langle \imath\right\rangle\]
%where $\|\imath\|=d$. As an object of $\LL{1}$, all the structure maps are trivial.
%\end{prop}

\begin{prop}
As a graded vector space,
\[H^{(1)}=\left\langle P(I^1)\imath\ \middle|\ (I^1)\textup{ satisfies \textup{(A)}}\right\rangle\]
%where $\|P(I^1)\imath\|=(\ell^1,d+n^1+\ell^1)$. 
As an object of $\LL{2}$, all the structure maps are trivial except for $\Q^1$, which is defined whenever $\ell^1>0$, and given by the formula
\[\Q^1:P(I^1)\imath\overset{\Q^1}{\mapsto} P(t_dI^1)\imath\textup{ when $\ell^1>0$}.\]
\end{prop}
\begin{proof}
As $H^{(0)}=\left\langle \imath\right\rangle$ has all structure trivial, the Koszul complex has zero differential, and so \emph{equals} $H^{(1)}$. The calculation of $\LL{2}$ structure was made in \ref{LieBracketsTrivial} and \ref{QkTrivial}.
\end{proof}

\begin{prop}
As a graded vector space,
%\[H^{(2)}=\frac{\left\langle \Q(I^2)P(I^1)\imath\ \middle|\ 1\notin I^2\right\rangle}{\left\langle \Q(I^2)P(I^1)\imath\ \middle|\ 1,2\notin I^2,\ I^1\in\im(t_d)\textup{, and $\ell^1>2$}\right\rangle}.\]
%\[H^{(2)}=\frac{\left\langle \Q(I^2)P(I^1)\imath\ \middle|\  (I^2,I^1)\textup{ satisfies \textup{(A)}-\textup{(D)}}\right\rangle}{\left\langle \Q(I^2)P(I^1)\imath\ \middle|\  (I^2,I^1)\textup{ satisfies \textup{(A)}-\textup{(E)}}\right\rangle}.\]
\begin{alignat*}{2}
H^{(2)}
&=
\frac{\left\langle \Q(I^2)P(I^1)\imath\ \middle|\  (I^2,I^1)\textup{ satisfies \textup{(A)}-\textup{(D)}}\right\rangle}{\left\langle \Q(I^2)P(I^1)\imath\ \middle|\  (I^2,I^1)\textup{ satisfies \textup{(A)}-\textup{(E)}}\right\rangle}%
\\
% Left hand side
% Relation
&\cong
% Right hand side
{\left\langle \Q(I^2)P(I^1)\imath\ \middle|\  (I^2,I^1)\textup{ satisfies \textup{(A)}-\textup{(D)} but not \textup{(E)}}\right\rangle}%
% Comment
\end{alignat*}
As an object of $\LL{3}$, all the structure maps are trivial except for $\Q^0$, which is defined whenever $\ell^1>0$, and is zero unless $\ell^2=0$ and $\ell^1=1$, in which case it is given by the formula \[\Q(\emptyset)P(I^1)\imath \overset{\Q^0}{\mapsto} \Q(\emptyset)P(t_dI^1)\imath\textup{ when $(\ell^2,\ell^1)=(0,1)$}.\]
\end{prop}
\begin{proof}
The Koszul complex, as a vector space, is \[\left\langle \Q(I^2)P(I^1)\imath\ \middle|\ (I^2,I^1)\textup{ satisfies \textup{(A)}-\textup{(C)}}\right\rangle.\] % --- as $H^{(1)}$ is an object of $\LL{2}$, it supports no $\Q^0$ operations, explaining why $I^2$ mustn't contain zero. 
Now to calculate the differential, we use the formula from \ref{KoszulComplexN>2}, setting $x=P(I^1)\imath$ and $I^2=(i_{\ell^2},\ldots,i_2,i_1)$
\[\partial(\Q(I^2)x)=\Q((i_{\ell^2},\ldots,i_2))(\Q^{i_1}x)+ \!\!\!\!\!\!\!\!\!\!\!\!\!\!\!\!\!\sum_{\substack{\produces{J}{I^2}{\SqShift}\\(j_{\ell^2},\ldots,j_2)\in\admis{\SqShift}}}\!\!\!\!\!\!\!\!\!\!\!\! \Q((j_{\ell^2},\ldots,j_2))(\Q^{j_1}x).\]
Now the format of the $\SqShift$-Adem relations shows that in each summand, $j_1$ exceeds $i_1$. As $i_1$ cannot be zero, $j_1$ must then be at least two, so that $Q^{j_1}x=0$ by our calculation of the $\LL{2}$ structure of $H^{(1)}$. Thus, only the first term can be nonzero, and even this term vanishes if $i_1>1$. If $i_1=1$, then $\Q^{i_1}x=\Q^1x=P(t_dI^1)\imath$, so that the differential is
\[\Q((i^2_{\ell^2},i^2_{\ell^2-1},\ldots,i^2_{2},1))P(I^1)\imath\mapsto \Q((i^2_{\ell^2},i^2_{\ell^2-1},\ldots,i^2_{2}))P(t_dI^1)\imath,\]
where any $\Q(I^2)P(I^1)\imath$ in which $I^2$ does not end in 1 is sent to zero. As all of these images are linearly independent (when nonzero), the kernel of $\partial$ is spanned by those basis elements in which $I^2$ contains no 1, explaining the presence of condition \textup{(D)} in the numerator.

Evidently, the denominator is to be the image of $\partial$. In order that $\Q(I^2)P(I^1)\imath$ is in the image, it must be that $I^1=t_dJ$, with the following conditions satisfied:
\begin{enumerate}[(a)]
\squishlist
\setlength{\parindent}{.25in}
\item $J$ is $\delta$-admissible (with entries at least 2) and satisfies $\minDimP(J)\leq d$;
\item $I^2\star(1)$ is $\SqShift$-admissible;
\item $I^2\star(1)$ does not contain zero, and $\minDim(I^2\star(1))<\ell(J)$.
\end{enumerate}
%\begin{itemise}
%\setlength{\parindent}{.25in}
%\item[a)] $J$ is $\delta$-admissible and $\minDimP(J)\leq d$;
%\item[b)] $I^2\star(1)$ is $\SqShift$-admissible, with $\minDim(I^2\star(1))<\ell(J)$.
%\end{itemise}
We saw in \ref{tdPreservesA} that (a) is equivalent to the standard condition \textup{(A)} for $I^1$. Condition (b) and the first half of condition (c) together require simply that $I^2$ is $\SqShift$-admissible and does not contain any of 0, 1 or 2.

The condition on $\ell(J)$ is more interesting, and we consider two cases. Firstly, if $I^2=\emptyset$, it becomes $1<\ell(J)$, i.e.\ $\ell^1>2$. Secondly, if $I^2$ is nonempty, the admissibility condition in (b) demands that $I^2$ contains no 1 or 2, in which case $\minDim(I^2\star(1))=\minDim(I^2)-1$, so that $\minDim(I^2\star(1))<\ell(J)$ is equivalent to the standard condition $\minDim^2<\ell^1$. Note that when these conditions are satisfied, it happens that $\ell^1>m^2>2$, so that the condition $\ell^1>2$ is satisfied automatically. %Note finally that we don't have to explicitly remove the condition $\ell^1>2$ in this case, as it holds automatically when $I^2$ is nonempty but contains no 1 or 2. 
This completes our identification of the image of $\partial$, as those sequences satisfying \textup{(A)}-\textup{(E)}, and thus our identification of $H^{(2)}$ as a vector space.


%$J$ is $\delta$-admissible and $\minDimP(J)\leq d$, and It is in the image iff $I^1$ has a top operation at the left, say $t$, and $\Q(I^2\star(1))P(I^1\setminus \{t\})\imath$ is a valid element of $H^{(1)}$. In order for it to be valid, the $\SqShift$-admissibility condition implies that $I^2$ is either empty, or has rightmost entry at least 3. If $I^2$ is empty, in order that $\Q(I^2\star(1))P(I^1\setminus \{t\})\imath$ is allowable we need  $\minDim((1))=1<\ell^1-1$, explaining the condition on $\ell^1$. If $I^2$ is nonempty, one finds that the two inequalities $m(I^2\star(1))<\ell^1-1$ and $m(I^2)<\ell^1$ are equivalent, and as $I^2$ has a 3 or greater at right, $3\leq\minDim(I^2)<\ell^1$ implies the extra condition ``$\ell^1>2$'' anyway, so that we can leave the condition $\ell^1>2$ on the denominator without adding the caveat ``when $I^2$ is empty''. This completes the identification of the underlying vector space.

Next, by \ref{LieBracketsTrivial} and \ref{QkTrivial}, all operations on $H^{(2)}\in\LL{3}$ are zero, except perhaps $\Q^0$. Consider the operation $Q^0$ on a \emph{cycle} $z:=\Q(I^2)P(I^1)\imath\in \Koverline_*H^{(1)}$ with $\ell^2>0$. As $z$ is a cycle, $I^2$ does not contain a 1. As $\Q^1$ is the only operation which is ever nonzero on $H^{(1)}$, lemma \ref{Q0ZeroByPriddyAlg} shows that $\Q^0$ vanishes on $z$. Finally, if $\ell^2=0$, \ref{PrelimQ0Calc} indicates that we must apply $\Q^{\ell^1}$ to $P(I^1)\imath\in H^{(1)}$, and this is only nonzero when $\ell^1=1$, completing the calculation of $\LL{3}$ structure.
\end{proof}

\begin{prop}
For $N\geq2$, as a graded vector space,
%\[H^{(N)}=\frac{\left\langle \Q(I^N)\cdots \Q(I^2)P(I^1)\imath\ \middle|\ 1\notin I^2\right\rangle}{\left\langle \Q(I^N)\cdots \Q(I^2)P(I^1)\imath\ \middle|\ 1,2\notin I^2,\ I^1\in\im(t_d)\textup{, and $\ell^1>2$}\right\rangle}.\]
\[H^{(N)}={\left\langle \Q(I^N)\cdots \Q(I^2)P(I^1)\imath\ \middle|\  (I^N,\ldots,I^1)\textup{ satisfies \textup{(A)}-\textup{(D)} but not \textup{(E)}}\right\rangle}.\]
As an object of $\LL{N+1}$, all the structure maps are trivial except for $\Q^0$, which is zero unless $\ell^N=\cdots =\ell^2=0$ and $\ell^1=1$, in which case it is given by the formula
\[\Q(\emptyset)\cdots \Q(\emptyset)P(I^1)\imath \overset{\Q^0}{\mapsto} \Q(\emptyset)\cdots \Q(\emptyset)P(t_dI^1)\imath\textup{ when $(\ell^N,\ldots,\ell^1)=(0,\ldots,0,1)$}.\]
%$\Q(\emptyset)\cdots \Q(\emptyset)P(I^1)\imath\mapsto \Q(\emptyset)\cdots \Q(\emptyset)P(t_dI^1)\imath$.
\end{prop}
\begin{proof}
In order to obtain this description of the underlying vector space, we must show that for all $N\geq3$, the Koszul complex $\Koverline_*H^{(N-1)}$ calculating $H^{(N)}$ has zero differential. This is true when $N=3$, as the only nonzero operation on $H^{(2)}\in\LL{3}$ is a \emph{top} operation. This pattern will continue, so there'll never be any differentials.

Of course we must understand $H^{(2)}$ as an element of $\LL{3}$. As before, the only operations with a chance of being nonzero are the $Q^0$. First, consider the operation $Q^0$ on a cycle $z:=\Q(I^3)\Q(I^2)P(I^1)\imath$ with $\ell^3>0$. For this class, $\ell^2>0$, so that $\Q(I^2)P(I^1)\imath$ is in the kernel of \emph{every} $\Q$ operation defined thereupon. Lemma \ref{Q0ZeroByPriddyAlg} then shows that $Q^0z=0$. Next, consider $\Q^0$ on cycles $\Q(\emptyset)\Q(I^2)P(I^1)\imath$. This takes value $\Q(\emptyset)\Q^{\ell^2}\Q(I^2)P(I^1)\imath$, which is only nonzero if $\ell^2=0$ and $\ell^1=1$. This confirms all our claims when $N=3$, and these arguments can be repeated inductively for larger $N$.
\end{proof}
\subsection{Concise descriptions of the $H^{(N)}$}
In order to simplify notation, we'll organise the relevant sequences $(I^{N},\ldots,I^1)$ of admissible sequences parameterising a basis of $H^{{(N)}}$ as follows.
\begin{itemize}
\squishlist
%\item Let $\calS^{(0)}=\{\emptyset\}$ be the set which contains only the empty sequence.
%\item Let $\calS^{(1)}$ be the set of length 1 sequences $(I^1)$, where $I^1$ is a sequence satisfying \textup{(A)}.
%\item For $N\geq2$, let $\calS^{(N)}$ be the set of (length $N$) sequences of sequences $(I^{N},\ldots,I^1)$ satisfying conditions \textup{(A)}-\textup{(C)}, such that $1\notin I^2$, and for which it is not true that ``$1,2\notin I^2,\ I^1\in\im(t_d)\textup{, and $\ell^1>2$}$''.
\item For $N\geq0$, let $\calS^{(N)}$ be the set of length $N$ sequences $(I^{N},\ldots,I^1)$ of sequences satisfying conditions \textup{(A)}-\textup{(D)} but not \textup{(E)}.
%\item Write $b(S):=\Q(I^{N})\cdots \Q(I^2)P(I^1)\imath\in H^{(N)}$, for $S=(I^{N},\ldots,I^1)\in \calS^{(N)}$.
\end{itemize}
Then we've proven that for all $N\neq1$, $H^{(N)}=\langle b(S)\ |\ S\in\calS^{(N)}\rangle$, while
\[H^{(1)}=
\langle b(S)\ |\ S\in\calS^{(1)}\rangle\oplus\langle b(S)\ |\ S=(I^1)\textup{ satisfies (A) and (E)}\rangle
\]
%Note also that for $N\geq1$ we have $H^{(N)}_0=\langle b(S)\ |\ S\in\calS^{(N-1)}\rangle$. Next we form three subsets of $\calS^{(N)}$ (what range on $N$?).
In order to help simplify the continuing exposition, define, for $N\geq 1$:
\begin{alignat*}{2}
\calS^{(N)}_0
&:=
\{(\emptyset,\ldots,\emptyset)\}%
\\
\calS^{(N)}_1
%&:=\{S\in \calS^{(N)}\ |\ \ell^{N}=\cdots =\ell^2=0,\ \ell^1=1\}\\
&:=%\hphantom{:}=
%\begin{cases}
%\{(\emptyset,\ldots,\emptyset,(i))\ |\ 2\leq i\leq d\},&\textup{if }N>0;\\
%\emptyset,&\textup{if }N=0.
%\end{cases}
\{(\emptyset,\ldots,\emptyset,(i))\ |\ 2\leq i\leq d\}\\
\calS^{(N)}_2&:=
%\begin{cases}
%\{(\emptyset,\ldots,\emptyset,(d+i+1,i))\ |\ 2\leq i\leq d\},&\textup{if }N>0;\\
%\emptyset,&\textup{if }N=0.
%\end{cases}
\{(\emptyset,\ldots,\emptyset,(d+i+1,i))\ |\ 2\leq i\leq d\}\\
%\{(\emptyset,\ldots,\emptyset,(i))\ |\ 2\leq i\leq d\}&\qquad&\text{(if $N>0$)}
\calS^{(N)}_3&:=\calS^{(N)}\setminus(\calS^{(N)}_0 \cup\calS^{(N)}_1\cup \calS^{(N)}_2)
\end{alignat*}
%and define
%\[\calS^{(0)}_0=\{\emptyset\},\ \calS^{(0)}_1=\calS^{(0)}_2=\calS^{(0)}_3=\emptyset,\]
so that $\calS^{(N)}$ is the disjoint union $\calS^{(N)}_0\cup\calS^{(N)}_1\cup\calS^{(N)}_2\cup\calS^{(N)}_3$ of four nonempty subsets. Write $\calS^{(N)}_{ab}$ for the (disjoint) union $\calS^{(N)}_{a}\cup\calS^{(N)}_{b}$, and so on.

We'll use the convention that each of $\calS^{(0)}_1$, $\calS^{(0)}_2$ and $\calS^{(0)}_3$ are the empty set, while $\calS^{(0)}_0=\{\emptyset\}$, the one element set containing the empty sequence.
With this convenction, these graded sets have the property that for any $N\geq1$ and $i\in\{0,1,2,3\}$ there is a bijection $(\calS^{(N)}_i)_0\cong\calS^{(N-1)}_i$ given by $(\emptyset,I^{N-1},\ldots,I^1)\mapsto(I^{N-1},\ldots,I^1)$.
%Next, define
%%\[\calS^{(N)}_2=\im(\overline{t}_d)=\{(\emptyset,\ldots,\emptyset,(d+i+1,i))\ |\ 2\leq i\leq d\},\]
%\begin{alignat*}{2}
%\calS^{(N)}_2
%&:=
%\im(\overline{t}_d)%
%\\
%% Left hand side
%% Relation
%&\hphantom{:}=
%% Right hand side
%\{(\emptyset,\ldots,\emptyset,(d+i+1,i))\ |\ 2\leq i\leq d\}% Comment
%&\qquad&\text{(if $N>0$)}
%\end{alignat*}
%and finally:
%\[\calS^{(N)}_3:=\calS^{(N)}\setminus(\calS^{(N)}_1\cup \calS^{(N)}_2\cup\{(\emptyset,\ldots,\emptyset)\}).\]
%For $N\geq1$, all four of the subsets $\calS^{(N)}_i$ is nonempty.


\begin{thm}\label{longDescriptionOfHN}
If we decompose $H^{(1)}\in\LL{2}$ as
%\[H^{(1)}= \left\langle b(S)\ \middle|\ S\textup{ satisfies \textup{(A)}, $S\notin\im(t_d)$}\right\rangle\oplus \left\langle b(S)\ \middle|\ S\textup{ satisfies \textup{(A)}, $S\in\im(t_d)$}\right\rangle\]
%\[H^{(1)}= \left\langle \Q(I^1)\imath\ \middle|\ I^1\textup{ satisfies \textup{(A)}, $I^1\notin\im(t_d)$}\right\rangle\oplus \left\langle \Q(I^1)\imath\ \middle|\ I^1\textup{ satisfies \textup{(A)}, $I^1\in\im(t_d)$}\right\rangle\]
\begin{alignat*}{2}
H^{(1)}
&=
\left\langle b(S)\ \middle|\ S\in\calS_{013}^{(N)}\right\rangle\oplus\left\langle b(S)\ \middle|\ S\in\calS_{2}^{(N)}\right\rangle
\\
&
\quad\quad  \oplus\langle b(S)\ |\ S=(I^1)\textup{ satisfies (A) and (E)}\rangle% Comment
\end{alignat*}
%%%\begin{alignat*}{2}
%%%H^{(1)}
%%%&=
%%%\left\langle \Q(I^1)\imath\ \middle|\ I^1\textup{ satisfies \textup{(A)}, $I^1\notin\im(t_d)$}\right\rangle%
%%%\\
%%%&
%%%\quad\quad  \oplus\left\langle \Q(I^1)\imath\ \middle|\ I^1\textup{ satisfies \textup{(A)}, $I^1\in\im(t_d)$}\right\rangle% Comment
%%%\end{alignat*}
then $\Q^1$, the only operation which is ever nonzero, has image the second and third summands.%footnote:
\footnote{The following became gramatically incorrect: [Recall that $t_d$ is only defined when $\ell^1\geq1$, just as $Q^1$ is only defined on $Q(I^1)\imath$ when $\ell^1\geq1$.]}

For $N\neq1$, if we decompose $H^{(N)}\in\LL{N+1}$ as
%Then we can further decompose $H^{(N)}$ as follows:
%\[H^{(n)}=\langle \imath\rangle\oplus \left\langle b(S)\ \middle|\ S\in\calS_3^{(N)}\cup \calS_1^{(N)}\right\rangle \oplus \left\langle b(S)\ \middle|\ S\in\calS_2^{(N)}\right\rangle\]
\[H^{(N)}=
\left\langle b(S)\ \middle|\ S\in\calS_{03}^{(N)}\right\rangle\oplus
%\left\langle \imath\rule{0mm}{\heightof{$\calS_1^{(N)}$}}\right\rangle\oplus 
\left\langle b(S)\ \middle|\ S\in\calS_1^{(N)}\right\rangle \oplus \left\langle b(S)\ \middle|\ S\in\calS_2^{(N)}\right\rangle\]
then the first and second summands consist of indecomposable elements, while the third summand is the isomorphic image of the second under the operation $\Q^0$.

$H^{(1)}_0\in\PRLie{1}\cong \GoodLie{1}$ is the one dimensional abelian Lie algebra  $\langle \imath\rangle$.
More broadly, for $N\geq1$, if we decompose $H^{(N)}_0\in\PRLie{N}$ as 
\[H^{(N)}_0=\left\langle \imath\rule{0mm}{\heightof{$\calS_1^{(N)}$}}\right\rangle\oplus \left\langle b(S)\ \middle|\ S\in\calS_1^{(N-1)}\right\rangle \oplus \left\langle b(S)\ \middle|\ S\in\calS_2^{(N-1)}\right\rangle \oplus\left\langle b(S)\ \middle|\ S\in\calS_3^{(N-1)}\right\rangle,\]
the restriction, defined on the last three summands, sends the second isomorphically onto the third, and vanishes on the third and fourth summands. %In particular, $H^{(1)}_0$ is the trivial element $\langle \imath\rangle$ of 
\end{thm}
%Then for $N\geq2$, if we decompose $H^{(N)}$ as
%%Then we can further decompose $H^{(N)}$ as follows:
%%\[H^{(n)}=\langle \imath\rangle\oplus \left\langle b(S)\ \middle|\ S\in\calS_3^{(N)}\cup \calS_1^{(N)}\right\rangle \oplus \left\langle b(S)\ \middle|\ S\in\calS_2^{(N)}\right\rangle\]
%\[H^{(n)}=\langle \imath\rangle\oplus \left\langle b(S)\ \middle|\ S\in\calS_1^{(N)}\right\rangle \oplus \left\langle b(S)\ \middle|\ S\in\calS_2^{(N)}\right\rangle \oplus\left\langle b(S)\ \middle|\ S\in\calS_3^{(N)}\right\rangle\]
%then the first, second and fourth summands consist of indecomposable elements, while the third summand is the isomorphic image of the second under the operation $\Q^0$.
%
%Next, for $N=1$, if we decompose $H^{(1)}$ as
%\[H^{(1)}= \left\langle b(S)\ \middle|\ S\textup{ satisfies \textup{(A)}, $S\notin\im(t_d)$}\right\rangle\oplus \left\langle b(S)\ \middle|\ S\textup{ satisfies \textup{(A)}, $S\in\im(t_d)$}\right\rangle\]
%then $\Q^1$, the only operation which is ever nonzero, has image the second summand. Recall that $t_d$ is only defined on $I^1$ such that $\ell^1\geq1$.
%




%\begin{itemize}
%\squishlist
%\setlength{\parindent}{.25in}
%%\item Define $\calS^{(N)}_1:=\calS^{(N)}\cap\{(I^{N},\ldots,I^1)\ |\ \ell^{N}=\cdots =\ell^2=0,\ \ell^1=1\}$. This is the set of sequences $S$ such that $b(S)\in H^{(N+1)}_0$ supports a nonzero restriction.
%%\item Define $\calS^{(N)}_1:=\calS^{(N)}\cap\{(I^{N},\ldots,tI^1)\ |\ \ell^{N}=\cdots =\ell^2=0,\ \ell^1=1\}$. This is the set of sequences $S$ such that $b(S)\in H^{(N+1)}_0$ is the image of a restriction.
%%%\item Define $\calS^{(N-1)}_2:=S^{(N-1)}\cap\{(I^{N-1},\ldots,t(I^1))\ |\ \ell^{N-1}=\cdots =\ell^2=0,\ \ell^1=1\}$, the set of sequences $S$ such that $b(S)\in H^{(N)}_0$ is the image of a restriction.
%%\item Let $t:\calS_1^{(N)}\to\calS_2^{(N)}$ be the function $(I^{N},\ldots,I^1)\mapsto (I^{N},\ldots,tI^1)$.
%%\item Define $\calS^{(N)}_3:=\calS^{(N)}\setminus(\calS^{(N)}_1\cup \calS^{(N)}_2\cup\{(\emptyset,\ldots,\emptyset)\})$, the set of sequences $S$, not in $\calS^{(N)}_1$ or $\calS^{(N)}_2$, such that the restriction is defined on $b(S)$.
%\item %$\calS^{(N)}_1:=\{S\in \calS^{(N)}\ |\ \ell^{N}=\cdots =\ell^2=0,\ \ell^1=1\}=\calS^{(N)}\cap\{(\emptyset,\ldots,\emptyset,I^1)\ |\ \ell^1=1\}$.
%\begin{alignat*}{2}
%\calS^{(N)}_1
%&:=
%\{S\in \calS^{(N)}\ |\ \ell^{N}=\cdots =\ell^2=0,\ \ell^1=1\}%
%\\
%&=
%\{(\emptyset,\ldots,\emptyset,(i))\ |\ 2\leq i\leq d\}
%\end{alignat*}
%
%
%%\item Define $\calS^{(N)}_2:=\calS^{(N)}\cap\{(\emptyset,\ldots,\emptyset,t_dI^1)\ |\ \ell^1=1\}$.
%\item Let $t:\calS_1^{(N)}\to\calS^{(N)}$ be the function $(\emptyset,\ldots,\emptyset,I^1)\mapsto (\emptyset,\ldots,\emptyset,t_dI^1)$.
%\item Let $\calS^{(N)}_2$ be the image of $t_d$.
%\item Define $\calS^{(N)}_3:=\calS^{(N)}\setminus(\calS^{(N)}_1\cup \calS^{(N)}_2\cup\{(\emptyset,\ldots,\emptyset)\})$.
%\end{itemize}
\end{CalculatingRepeatedKoszul}

\begin{DimZeroPart}
\subsection{Calculation of $E^{2,N}=(\derived\Ind{\LL{N+1}})H^{(N)}$ along the first two axes}
Suppose that $N\geq0$, so that $E^{2,N}:=(\derived\Ind{\LL{N+1}})H^{(N)}\in\GR{N+2}$ is the $E^2$ page of the $N^\textup{th}$ Grothendieck spectral sequence%footnote:
\footnote{Recall that the $0^\textup{th}$ Grothendieck spectral sequence is the Adams spectral sequence itself.}. The results of the previous section allow a complete calculation of the groups
\[E^{2,N}_{0,a_{N+1},a_N,\ldots,a_1}=\Ind{\LL{N+1}}H^{(N)}.\]
\begin{prop}
For $N\geq0$, the graded vector space $E^{2,N}_{0,a_{N+1},a_N,\ldots,a_1}\in\GR{N+1}$ is
\[\left\langle b(S)\ \middle|\ S\in\calS_{013}^{(N)}\right\rangle.\]
\end{prop}
\begin{proof}
This follows from \ref{longDescriptionOfHN}.
\end{proof}
%\subsection{Calculation of $E^{2,N}=(\derived\Ind{\LL{N+1}})H^{(N)}$ in zero internal dimension}

We can also completely calculate the groups
\[E^{2,N}_{a_{N+2},0,a_N,\ldots,a_1}\]
using partially restricted Lie algebra homology. From the section on low dimensional calculations:
\[E^{2,N}_{a_{N+2},0,a_N,\ldots,a_1}:=((\derived_{a_{N+2}}\Ind{\LL{N+1}})H^{(N)})_{0,a_{N},\ldots,a_1}\cong ((\derived_{a_{N+2}}\Ind{\PRLie{N}})H^{(N)}_0)_{a_{N},\ldots,a_1}\]
%The point now is that $H^{(N)}_0\in\PRLie{N}$ is easy to describe, for all $N\geq1$. As a vector space it is $\langle b(S)\ |\ S\in\calS^{(N-1)}\rangle$. It has Lie bracket zero. Its restriction, defined whenever $\ell_{N-1},\ldots,\ell_1$ are not all zero, is only nonzero when %in case $a_N=\cdots =a_3=0$ and $a_2=1$, that is, when 
%$\ell^N=\cdots =\ell^2=0$ and $\ell^1=1$, in which case:
%\[\Q(\emptyset)\cdots \Q(\emptyset)P(I^1)\imath\overset{\restn{}}{\mapsto}\Q(\emptyset)\cdots \Q(\emptyset)P(t_dI^1)\imath.\]
%Note that all of the classes $\Q(\emptyset)\cdots \Q(\emptyset)P(t_dI^1)\imath$, where $\ell^1=1$, are linearly independant in $H^{(N)}_0$. Moreover, there is only one basis element in non-restrictable dimensions, and that is $\Q(\emptyset)\cdots \Q(\emptyset)P(\emptyset)\imath$.
\begin{prop}
For $N\geq1$, the graded vector space $E^{2,N}_{a_{N+2},0,a_N,\ldots,a_1}\in\GR{N+1}$ is isomorphic (as a graded vector space only) to the kernel of the counit of the coalgebra
\[E\left\langle b(S)\ \middle|\ S\in\calS_{01}^{(N-1)}\right\rangle\otimes \Gamma_\textup{ev}\left\langle b(S)\ \middle|\ S\in\calS_2^{(N-1)}\right\rangle\otimes \Gamma\left\langle b(S)\ \middle|\ S\in\calS_3^{(N-1)}\right\rangle \]
%where by $\Gamma_{\textup{ev}}$ we mean the even divided powers.
The degree $\|a_{N+2},0,a_{N}\ldots,a_1\|$ of an expression $\prod_j\gamma_{i_j}(x_j)$ is determined by:
\[a_{{N+2}}=-1+\sum_{\smash j}i_j,\quad (a_N,\ldots,a_1)=(0,\ldots,0,a_{N+2})+\sum_{\smash j}i_j\|x_j\|,\]
which is to say that $E^{2,N}_{a_{N+2},0,*,*,\ldots}$ is calculated by monomials of total exponent $a_{N+2}+1$, that $a_{N},\ldots,a_2$ are all additive, and that $a_1$ is additive with a $+1$-shift.
%one takes $a_{N+2}$ to be $\sum_ji_j-1$, $a_{N+1}$ to be 0, and $(a_N,\ldots,a_1)$ to be $\sum_ji_j\|x_j\|$. That is, to calculate the degree $a_{N+2}$ part, we need to look at monomials of total exponent $a_{N+2}+1$.
\end{prop}
\begin{proof} %For the course of this proof, we will supress the $(N-1)$ superscripts that normally appear on the $\calS^{(N-1)}$, and also the parentheses that usually appear in $b(S)$.
In this proof, write $\overline{t}_d:\calS_1^{(N)}\to\calS^{(N)}_2$ for the bijection $(\emptyset,\ldots,\emptyset,I^1)\mapsto (\emptyset,\ldots,\emptyset,t_dI^1)$. 
As a vector space, the Eilenberg-Cartan-May complex is \[E\langle \imath\rangle\otimes \Gamma\left\langle bS\ \middle|\ S\in \calS_1^{(N-1)}\cup\calS_2^{(N-1)}\cup\calS_3^{(N-1)}\right\rangle.\] As so few operations are nonzero, it splits into a tensor product of complexes
\[E\langle \imath\rule{0mm}{\heightof{$\calS_1^{(N)}$}}\rangle\otimes \!\!\!\!\!\bigotimes_{S\in \calS_1^{(N-1)}}\!\!\!\!\!\Gamma\langle\rule{0mm}{\heightof{$\calS_1^{(N)}$}} bS,b\overline{t}_dS\rangle\otimes \Gamma\left\langle bS\ \middle|\ S\in \calS_3^{(N-1)}\right\rangle\]
where only the factors $\Gamma\langle bS,b\overline{t}_dS\rangle$ have nonzero differential. The differential on $\Gamma\langle bS,b\overline{t}_dS\rangle$ is
\[\gamma_i( bS)\cdot\gamma_j(b\overline{t}_dS)\mapsto\gamma_{i-2}( bS)\cdot b\overline{t}_dS\cdot\gamma_j(b\overline{t}_dS)=\begin{cases}
\gamma_{i-2}(bS)\cdot\gamma_{j+1}(b\overline{t}_dS),&\textup{if }j\textup{ is even};\\
0,&\textup{if }j\textup{ is odd},
%\\,&\textup{if }
\end{cases}
\]
so that $H_*(\Gamma\langle bS,b\overline{t}_dS\rangle)$ has basis the expressions $\gamma_i(bS)\cdot\gamma_j(b\overline{t}_dS)$ for $i\leq1$ and $j$ even.
Finally, note that the C-E-M complex calculates partially restricted Lie algebra homology, which is Quillen homology up to a shift. %, explaining why $a_{N+2}$ is $(\sum i_j)-1$.
\end{proof}
Note that this differential is in fact dual to the differential on the symmetric algebra which sets every generator to be a cycle except for the duals of $b\overline{t}_dS$, which are sent to the squares of the duals of $bS$.
\end{DimZeroPart}
\begin{AxisComputationSummary}

\begin{shaded}
\begin{enumerate}[A)]
\squishlist
\setlength{\parindent}{.25in}
\item[\textup{(A)}] $I^1$ is $\delta$-admissible (with entries at least 2) and satisfies $\minDimP^1\leq d$;
\item[\textup{(B)}] for $i\geq2$, $I^i$ is $\SqShift$-admissible (with entries nonnegative);
\item[\textup{(C)}] $I^2$ does not contain zero, and $\minDim^i<\ell^{i-1}$ for all $i\geq2$.
\item[(D)] $I^2$ doesn't contain 1.
\item[(E)] $I^1\in\im(t_d)$, $\ell^1>2$, and $I^2$ doesn't contain 1 or 2.
\end{enumerate}
For $N\geq0$, let $\calS^{(N)}$ be the set of length $N$ sequences $(I^{N},\ldots,I^1)$ of sequences satisfying conditions \textup{(A)}-\textup{(D)} but not \textup{(E)}.

Then we've proven that for all $N\neq1$, $H^{(N)}=\langle b(S)\ |\ S\in\calS^{(N)}\rangle$.

Let $\calS^{(0)}_1=\{\emptyset\}$ and $\calS^{(0)}_1=\calS^{(0)}_2=\calS^{(0)}_3=\emptyset$. For $N\geq1$ define
\begin{alignat*}{2}
\calS^{(N)}_0
&:=
\{(\emptyset,\ldots,\emptyset)\}%
\\
\calS^{(N)}_1
%&:=\{S\in \calS^{(N)}\ |\ \ell^{N}=\cdots =\ell^2=0,\ \ell^1=1\}\\
&:=%\hphantom{:}=
%\begin{cases}
%\{(\emptyset,\ldots,\emptyset,(i))\ |\ 2\leq i\leq d\},&\textup{if }N>0;\\
%\emptyset,&\textup{if }N=0.
%\end{cases}
\{(\emptyset,\ldots,\emptyset,(i))\ |\ 2\leq i\leq d\}\\
\calS^{(N)}_2&:=
%\begin{cases}
%\{(\emptyset,\ldots,\emptyset,(d+i+1,i))\ |\ 2\leq i\leq d\},&\textup{if }N>0;\\
%\emptyset,&\textup{if }N=0.
%\end{cases}
\{(\emptyset,\ldots,\emptyset,(d+i+1,i))\ |\ 2\leq i\leq d\}\\
%\{(\emptyset,\ldots,\emptyset,(i))\ |\ 2\leq i\leq d\}&\qquad&\text{(if $N>0$)}
\calS^{(N)}_3&:=\calS^{(N)}\setminus(\calS^{(N)}_0 \cup\calS^{(N)}_1\cup \calS^{(N)}_2)
\end{alignat*}
The degree $ (a_{N+1},\ldots,a_1) $ of a sequence $(I^N,\ldots,I^1)\in\calS^{(N)}$ is:
\begin{align*}
& \left( \ell^N,\ell^{N-1}+n^N,2^{\ell^N}\!\left(\ell^{N-2}+n^{N-1}\right), 2^{\ell^N+\ell^{N-1}}\!\left(\ell^{N-3}+n^{N-2}\right),\ldots \right.\\
 &\qquad \left.{}\ldots,2^{\ell^N+\cdots+\ell^{3}}(\ell^1+n^2),2^{\ell^N+\cdots+\ell^{2}}(d+n^1+\ell^1+1)-1 \right).
\end{align*}


%Here we include a summary of the $E^2$ pages of the Grothendieck spectral sequences along the first two axes.
\noindent \textbf{Homological degree zero} on $N^\textup{th}$ gss, for $N\geq0$:
\begin{alignat*}{2}
E^{2,N}_{0,a_{N+1},a_N,\ldots,a_1}
&=
(\Ind{\LL{N+1}}H^{(N)})_{a_{N+1},\ldots,a_1}%
\\
% Left hand side
% Relation
E^{2,N}_{0,a_{N+1},a_N,\ldots,a_1}&=
% Right hand side
%%%%\begin{cases}
%%%%\left\langle \Q(I^1)\imath\rule{0mm}{\heightof{$\calS_1^{(N)}$}}\ \middle|\ (I^1)\in\calS^{(1)},\ I^1\notin\im(t_d)\right\rangle,&\textup{if }N=1;\\
%%%%\left\langle \imath\rule{0mm}{\heightof{$\calS_1^{(N)}$}}\right\rangle\oplus 
%%%%\left\langle b(S)\ \middle|\ S\in\calS_1^{(N)}\cup\calS_3^{(N)}\right\rangle,&\textup{if }N\geq2.
%%%%%\\,&\textup{if }
%%%%\end{cases}
\left\langle b(S)\ \middle|\ S\in\calS_{013}^{(N)}\right\rangle\\
\intertext{\textbf{Internal degree zero} on $N^\textup{th}$ gss, for $N\geq1$:}
E^{2,N}_{a_{N+2},0,a_N,\ldots,a_1}
&=
H_{a_{N+2}+1}\left(H^{(N)}_0\right)\phantom{H\!\!\!\!\!\!\!}_{a_N,\ldots,a_1}\\
H_{*}\left(H^{(N)}_0\right)&=%E\left\langle \imath\rule{0mm}{\heightof{$\calS_1^{(N)}$}}\right\rangle \otimes E\left\langle b(S)\ \middle|\ S\in\calS_1^{(N-1)}\right\rangle\\
%&\qquad \otimes\Gamma_\textup{ev}\left\langle b(S)\ \middle|\ S\in\calS_2^{(N-1)}\right\rangle\otimes \Gamma\left\langle b(S)\ \middle|\ S\in\calS_3^{(N-1)}\right\rangle\\
\left(\genfrac{}{}{0pt}{}{E\left\langle \imath\rule{0mm}{\heightof{$\calS_1^{(N)}$}}\right\rangle \otimes E\left\langle b(S)\ \middle|\ S\in\calS_1^{(N-1)}\right\rangle\otimes}{\Gamma_\textup{ev}\left\langle b(S)\ \middle|\ S\in\calS_2^{(N-1)}\right\rangle\otimes \Gamma\left\langle b(S)\ \middle|\ S\in\calS_3^{(N-1)}\right\rangle}\right)
%%
%
%
%&\left(\genfrac{}{}{0pt}{}{E\left\langle \imath\rule{0mm}{\heightof{$\calS_1^{(N)}$}}\right\rangle \otimes E\left\langle b(S)\ \middle|\ S\in\calS_1^{(N-1)}\right\rangle\otimes}{\Gamma_\textup{ev}\left\langle b(S)\ \middle|\ S\in\calS_2^{(N-1)}\right\rangle\otimes \Gamma\left\langle b(S)\ \middle|\ S\in\calS_3^{(N-1)}\right\rangle}\right)_{a_{N+2}+1,0,a_N,\ldots,a_1}
\end{alignat*}
The degree $\|a_{N+2},0,a_{N}\ldots,a_1\|$ of an expression $\prod_j\gamma_{i_j}(x_j)$ is determined by:
\[a_{{N+2}}=-1+\sum_{\smash j}i_j,\quad (a_N,\ldots,a_1)=(0,\ldots,0,a_{N+2})+\sum_{\smash j}i_j\|x_j\|,\]
which is to say that $E^{2,N}_{a_{N+2},0,*,*,\ldots}$ is calculated by monomials of total exponent $a_{N+2}+1$, that $a_{N},\ldots,a_2$ are all additive, and that $a_1$ is additive with a $+1$-shift.

In particular, to calculate the dimension $t$ of a monomial $(b(S))^{2^k}$, one calculates $2^{k+\ell^N+\cdots+\ell^{2}}(d+n^1+\ell^1+1)-1$. As the Adams differential $d_r$ has $r=t_{\textup{tar}}-t_{\textup{src}}+1$, a differential $(b(S_\textup{src}))^{2^{k_\textup{src}}}\mapsto b(S_\textup{tar})$ has
\[r=2^{\ell_\textup{tar}^N+\cdots+\ell_\textup{tar}^{2}}(d+n_\textup{tar}^1+\ell_\textup{tar}^1+1)-
    2^{k_\textup{src}+\ell_\textup{src}^N+\cdots+\ell_\textup{src}^{2}}(d+n_\textup{src}^1+\ell_\textup{src}^1+1)+1\]
\end{shaded}
\end{AxisComputationSummary}

\begin{VanishingLines}
\subsection{A vanishing line}
A class in $E^{2,N}_{a_{N+2},\ldots,a_1}$ which survives all the way to the $E_2$ page of the Adams spectral sequence represents a class in $E_2^{s,t}$ with dimensions $s=a_{N+2}+\cdots +a_2$ and $t=a_1$. Now we've seen above that $E^{2,N}_{0,a_{N+1},a_N,\ldots,a_1}$ is a subquotient of
\[\left\langle b(S)\ \middle|\ S\in\calS^{(N)}\right\rangle_{a_{N+1},\ldots,a_1}\]
so that any homogeneous class therein satisfies $t\geq\tfrac{9}{4}s$.
Next, $E^{2,N}_{a_{N+2},0,a_N,\ldots,a_1}$ is a subquotient of
\[\left(\Sigma^{-1}\left(E\left\langle s\imath\rule{0mm}{\heightof{$\calS_1^{(N)}$}}\right\rangle\otimes \Gamma\left\langle sb(S)\ \middle|\ S\in\calS^{(N-1)}\right\rangle\right)\right)_{a_{N+2},0,a_N,\ldots,a_1}\]
\begin{thm}
Suppose that $X$ is a connected simplicial commutative algebra of finite type. % which is connected, and is of finite type in dimension 1. That is, $H^0_Q(X)=0$ and $H^1_Q(X)$ is a finite dimensional $\F_2$-vector space. 
Then the Adams spectral sequence for $X$ has a vanishing line of slope $T-1$:
\[E_2^{s,t}=0\textup{\ unless $t\geq Ts+T-1$, i.e.\ $s\leq\tfrac{1}{T-1}(t-s)+1$,}\]
where $T$ is the constant $2$, or $9/4$ if $X$ is the sphere $S^d$ for $d\geq2$.
\end{thm}
\begin{proof}
We have, from \ref{boundOnA1}, that except on $\imath$,
\[a_1\geq T(a_2+\cdots +a_{N+1}+1).\]
This'll give plenty of information about classes originating in $E^{2,N}_{0,a_{N+1},a_N,\ldots,a_1}$, especially since $\imath$ lives in $0,\ldots,0,d$ here!

Consider an expression 
\[\prod_{k=1}^ny_k\,\cdot\,\prod_j\gamma_{i_j}(b(S_j)x_j)\in E^{2,N}_{a_{N+2},0,a_N,\ldots,a_1}\]
where the $y_k$ and $x_j$ are elements of $H^*_QX$. For notational convenience, write $\|b(S_j)x_j\|=(a_{Nj},\ldots,a_{1j})$. We have
\[a_1=-1+\sum(|y_k|+1)+\sum i_j\left[a_1(x_j)+1\right]\textup{ and }a_l=\sum i_j\left[a_{lj}\right]\textup{ for $2\leq l\leq N$}\]
while $a_{N+1}=0$ and $a_{N+2}=-1+n+\sum i_j$. Starting with \ref{boundOnA1}:
\begin{alignat*}{2}
0
&\leq
\sum i_j\left[a_{1j}-T(a_{Nj}+\cdots +a_{2j}+1)\right]%
\\
&=
\sum i_j\left[[a_{1j}+1]-[T+1]-T[a_{Nj}+\cdots +a_{2j}]\right]%
\\
% Left hand side
% Relation
&=
% Right hand side
[a_1+1-\sum(|y_k|+1)]-(T+1)[a_{N+2}+1-n]-T[a_N+\cdots +a_2]\\%
%&=t-Ts-\sum(|y_k|+1)-(T+1)[a_{N+2}+1-n]+Ta_{N+2}+1\\%
&=t-Ts-\left[\sum(|y_k|+1)+(T+1)[a_{N+2}+1-n]-Ta_{N+2}-1\right]\\%
&=t-Ts-T+1-\left[\sum(|y_k|+1-T)+(a_{N+2}+1-n)\right]
\end{alignat*}
and one notes that each of the parenthesized quatities in the final expression are positive.
\end{proof}

In particular, on a sphere $S^d$ with $d\geq2$, the indices of nonzero groups must satisfy both $t-s\geq(T-1)(s-1)$ and $t-s\geq d-s$, where $T=9/4$. These two lower bounds on $t-s$ are equal when $s=(d+T-1)/T$, with value $(d-1)(T-1)/T$, so that
\[E_2^{s,t}=0\textup{\ unless $t-s\geq \tfrac{5}{9}(d-1)>0$.}\]

\end{VanishingLines}


\begin{convergenceOLD}
\subsection{Convergence of the spectral sequence}
Following \cite{BousKanSSeq.pdf}, for any functor $F:s\Comm\to s\Comm$, we'll define the $r^\textup{th}$ derived functor $(D^sF)$ of $F$ with respect to Andr\'e-Quillen homology. The definition is inductive: one sets $(D^0F)=F$, and
\[(D^sF)(X):=\ker((D^{s-1}F)(cX)\to (D^{s-1}F)(QcX)).\]
Here, $Q$ is the indecomposables functor, and $c:s\Comm\to s\Comm$ is (the diagonal of) the standard bar construction, $(cX)_n:=T^{n+1}X_n$, where $T$ is the monad from the free/forget adjunction between vector spaces and nucas.

There is a natural transformation $\delta:(D^sF)\to (D^{s-1}F)$ given by the following composite:
\[(D^sF)(X)\to (D^{s-1}F)(cX)\to (D^{s-1}F)(X).\]
Our purpose in this section will be to prove the following, where we write $I$ for the identity functor $s\Comm\to s\Comm$:
\begin{prop}\label{convergenceProp}
The functors $(D^sI):s\Comm\to s\Comm$ preserve weak equivalences.
For $X\in s\Comm$ a connected simplicial algebra, consider the tower:
\[\xymatrix@R=4mm{
\cdots 
\ar[r]
&%r1c1
(D^{t}I)X
\ar[r]
&%r1c3
\cdots \ar[r]
&%r1c4
(D^1I)X
\ar[r]&%r1c5
(D^0I)X=X.
}\]
For integers $t\geq2$ and $q\geq0$, the map $\pi_q(D^{2t+q-1}I)X\to\pi_q(D^{t}I)X$ is zero.
\end{prop}
\begin{proof}%[Proof of \ref{convergenceProp}]
The first claim is \ref{towerIsHomotopical}. Apply $(D^t\DASH)X$ to the diagram of functors constructed in \ref{towerWithPowers}, to obtain the following diagram in $s\Comm$:
\[\xymatrix@R=4mm{
\cdots 
\ar[r]
&%r1c1
(D^{2t+q-1}I)X
\ar[r]
\ar[d]
&
\cdots \ar[r]
&%r1c4
(D^{t+1}I)X
\ar[r]
\ar[d]&%r1c5
(D^tI)X
\ar@{=}[d]
\\%r1c6
\cdots
\ar[r]
&%r2c1
(D^tP^{t+q})X
\ar[r]
&%r2c3
\cdots 
\ar[r]&%r2c4
(D^tP^2)X
\ar[r]
&%r2c5
(D^tI)X%r2c6
}\]
By \ref{connectivityOfDerivedPowers}, $(D^tP^{t+q})(A)$ is $q$-connected, and the result follows.
\end{proof}


Before proving this fact, we'll perform a little analysis of the derived functors of the identity functor $I$, and of the ``$s^\textup{th}$ power'' functor $P^sX:=X^s$.
\begin{enumerate}\squishlist
\setlength{\parindent}{.25in}
\item For any functor $F$, there is a natural isomorphism $(D^s(D^tF))\cong (D^{s+t}F)$ so that the evident diagram commutes:
\[\xymatrix@R=4mm{
(D^s(D^tF))
\ar[r]^-{\cong}
\ar[d]^-{(D^s\delta)}
&%r1c1
(D^{s+t}F)
\ar[d]^-{\delta}
\\%r1c2
(D^s(D^{t-1}F))
\ar[r]^-{\cong}&%r1c1
(D^{s+t-1}F)
}\]
\item If $F$ is the prolongation of a functor $\Comm\to\Comm$, one notes that $(D^sF)$ can be defined one level at a time. That is,  we may define functors $(D^s_nF):\Comm\to\Comm$ such that $(D^sF)(X)_n=(D^s_nF)(X_n)$, by defining $(D^0_nF):=F$, and
\[(D^s_nF)(Y):=\ker((D^{s-1}_nF)(T^{n+1}Y)\to (D^{s-1}_nF)(QT^{n+1}Y)).\]
\end{enumerate}
\begin{lem}
For $Y\in\Comm$, and integers $r,n\geq0$ and $s>0$:
\begin{itemise}
\setlength{\parindent}{.25in}
\item $(D^r_nI)Y\subset (\calT_1T^n_1)\cdots (\calT_rT^n_r)Y$ is the subset consiting of those mega\-monomials which have a doubling in each of the $\calT_1,\ldots,\calT_r$ functors.
\item $(D^r_nP^{s})Y\subset (\calT_1T^n_1)\cdots (\calT_rT^n_r)Y$ is the subset consiting of those mega\-monomials which have a doubling in each of the $\calT_2,\ldots,\calT_r$ functors, and a power of $s$ in the $\calT_1$ functor.
\end{itemise}

\end{lem}
\begin{proof}
%We now have enough notation to prove this claim, which is really a claim about $D^r_nY$, by induction on $r$. 
This is trivial when $r=0$. Supposing an inductive hypothesis:
\begin{alignat*}{2}
(D^r_nI)(Y)
&:=
\ker((D^{r-1}_nI)(\calT_r T_r^n Y)\to (D^{r-1}_nI) (Q\calT_rT_r^n X_n))%
\\
&=
(D^{r-1}_nI)(\calT_r T_r^n Y)\cap\langle \textup{meganomials with a doubling in $\calT_r$}\rangle\\
&=
\langle \textup{meganomials with a doubling in $\calT_1,\ldots,\calT_{r-1}$}\rangle\cap\langle \textup{doubling in $\calT_r$}\rangle\\
&=
\langle \textup{meganomials with a doubling in $\calT_1,\ldots,\calT_{r-1},\calT_{r}$}\rangle
\end{alignat*}
The proof for derived functors of $P^{s}$ is identical.
\end{proof}
\begin{cor}\label{towerWithPowers}
The functors $(D^rI)$ and $(D^rP^{s})$ preserve surjective maps, and there is a commuting diagram of functors:
\[\xymatrix@R=4mm{
\cdots 
\ar[r]
&%r1c1
(D^rI)
\ar[r]
\ar[d]
&%r1c2
\cdots \ar[r]
&%r1c4
(D^2I)
\ar[r]
\ar[d]
&%r1c4
(D^1I)
\ar[r]
\ar[d]&%r1c5
(D^0I)
\ar@{=}[d]
\\%r1c6
\cdots
\ar[r]
&%r2c1
P^{r+1}
\ar[r]
&%r2c3
\cdots 
\ar[r]&%r2c4
P^3
\ar[r]
&P^2
\ar[r]
&%r2c5
I%r2c6
}\]
\end{cor}
\begin{cor}\label{towerIsHomotopical}
The functors $(D^rI):s\Comm\to s\Comm$ preserve all weak equivalences.
\end{cor}
\begin{proof}
This is trivial when $r=0$. Suppose, by induction on $r$, that $(D^{r-1}I)$ preserves weak equivalences, and that $f:X\to X'$ is a weak equivalence in $s\Comm$. Then we have a commutative diagram in $s\Comm$:
\[\xymatrix@R=4mm{
0\ar[r]&
(D^rI)X\ar[d]^{(D^rI)(f)}\ar[r]&
(D^{r-1}I)(cX)\ar[d]^{(D^{r-1}I)(cf)}\ar[r]&
(D^{r-1}I)(QcX)\ar[d]^{(D^{r-1}I)(Qcf)}\ar[r]&
0\\0\ar[r]&
(D^rI)X'\ar[r]&
(D^{r-1}I)(cX')\ar[r]&
(D^{r-1}I)(QcX')\ar[r]&
0
}\]
Now the rows are short exact sequences, as $(D^{r-1}I)$ preserves surjective maps. The maps $cf$ and $Qcf$ are weak equivalences, as $c$ is a cofibrant replacement, and $Q$ is left Quillen. Thus, the induction hypothesis implies that the $(D^{r-1}I)(cf)$ and $(D^{r-1}I)(Qcf)$ are weak equivalences. The homotopy long exact sequence and the 5-lemma together imply that ${(D^rI)(f)}$ is a weak equivalence.
\end{proof}

\begin{prop}\label{connectivityOfPowerQuillen}
For $A\in s\Comm$ almost free and connected, and any $s\geq0$, $P^{s}A$ is $(s-1)$-connected.
\end{prop}
\begin{proof}
Truncate Quillen's fundamental spectral sequence, as presented in \cite[thm 6.2]{MR1089001}.
\end{proof}
\begin{cor}\label{connectivityOfDerivedPowers}
For $A\in s\Comm$ connected, and any $t>0$ and $s\geq2$, $(D^tP^{s})(A)$ is $(s-t)$-connected.
\end{cor}
\begin{proof}
We'll prove this by induction on $t$. When $t=1$:
\[(D^1P^{s})A:=\ker(P^{s}(cA)\to P^{s}(QcA))=P^{s}(cA)\]
Thus $(D^1P^{s})A$ is $(s-1)$-connected, by \ref{connectivityOfPowerQuillen}.

Now suppose, by induction, that $(D^{t-1}P^{s})(B)$ is $(s-(t-1))$-connected for any connected $B$. Then by \ref{towerWithPowers}, there's a short exact sequence:
\[\xymatrix{
0\ar[r]&
(D^{t}P^{s})(A)\ar[r]&
(D^{t-1}P^{s})(cA)\ar[r]&
(D^{t-1}P^{s})(QcA)\ar[r]&
0
}\]
in which the rightmost two objects are each $(s-t+1)$-connected. The associated long exact sequence shows that $(D^{t}P^{s})(A)$ is $(s-t)$-connected.
\end{proof}

%\begin{itemise}
%\setlength{\parindent}{.25in}
%\item need connectivity of $P^s$ on cofibrant dudes i.e.\ a curtis theorem
%\item need to make an easy argument on the connectivity of $D^{t}P^{s+1}$ as s grows - it's easy because we've shown that the maps involved are all surjections, so that there are long exact sequences. The connectivity loss invoked by $t$ is at most $t$, and we can push $s+1$ way up.
%\item we'd like the $D^s$ to be homotopical - I think that's easy enough.
%\end{itemise}


%%%%%
%%%%%\subsection*{The whole point}
%%%%%
%%%%%
%%%%%\begin{proof}[Proof of \ref{convergenceProp}]
%%%%%Apply $(D^t\DASH)X$ to the ladder from \ref{towerWithPowers}
%%%%%to obtain a ladder in $s\Comm$:
%%%%%\[\xymatrix@R=4mm{
%%%%%\cdots 
%%%%%\ar[r]
%%%%%&%r1c1
%%%%%(D^{s+t}I)X
%%%%%\ar[r]
%%%%%\ar[d]
%%%%%&
%%%%%\cdots \ar[r]
%%%%%&%r1c4
%%%%%(D^{t+1}I)X
%%%%%\ar[r]
%%%%%\ar[d]&%r1c5
%%%%%(D^tI)X
%%%%%\ar@{=}[d]
%%%%%\\%r1c6
%%%%%\cdots
%%%%%\ar[r]
%%%%%&%r2c1
%%%%%(D^tP^{s+1})X
%%%%%\ar[r]
%%%%%&%r2c3
%%%%%\cdots 
%%%%%\ar[r]&%r2c4
%%%%%(D^tP^2)X
%%%%%\ar[r]
%%%%%&%r2c5
%%%%%(D^tI)X%r2c6
%%%%%}\]
%%%%%The result follows by noting that, for any fixed $t$, the connectivity of $(D^tP^{s+1})X$ grows without bound as $s$ increases (quantify and include).
%%%%%\end{proof}
%%%%%
%%%%%
%%%%%\subsection*{old stuff}
%%%%%
%%%%%
%%%%%
%%%%%
%%%%%Define, inductively, functors $D^s:s\Comm\to s\Comm$, by setting $D^0X:=X$, and writing
%%%%%\[D^s(X):=\ker(D^{s-1}(cX)\to D^{s-1}(QcX)).\]
%%%%%One notes that $D^s(X)$ is a subalgebra of $c^sX$. We'll be able to identify exactly which subalgebra it is. Write $(cX)_n:=T^{n+1}X_n$, where $T$ is the non-unital tensor algebra functor from vector spaces to nucas. The construction is performed levelwise, so we may fix $n$. That is to say that we may define functors $D^s_n:\Comm\to\Comm$ such that $D^s(X)_n=D^s_n(X_n)$, by defining 
%%%%%\[D^0_n(Y):=Y\textup{ and }D^s_n(Y):=\ker(D^{s-1}_n(T^{n+1}Y)\to D^{s-1}_n(QT^{n+1}Y)).\]
%%%%%\begin{lem}
%%%%%$D^s_n(Y)\subset (\calT_1T^n_1)\cdots (\calT_sT^n_s)Y$ is the subset consiting of those mega\-monomials which have a doubling in each of the $\calT_1,\ldots,\calT_s$ functors.
%%%%%\end{lem}
%%%%%\begin{proof}
%%%%%%We now have enough notation to prove this claim, which is really a claim about $D^s_nY$, by induction on $s$. 
%%%%%This is trivial when $s=0$. Supposing an inductive hypothesis:
%%%%%\begin{alignat*}{2}
%%%%%D^s_n(Y)
%%%%%&:=
%%%%%\ker(D^{s-1}_n(\calT_s T_s^n Y)\to D^{s-1}_n (Q\calT_sT_s^n X_n))%
%%%%%\\
%%%%%&=
%%%%%D^{s-1}_n(\calT_s T_s^n Y)\cap\langle \textup{meganomials with a doubling in $\scrT_s$}\rangle\\
%%%%%&=
%%%%%\langle \textup{meganomials with a doubling in $\scrT_1,\ldots,\scrT_{s-1}$}\rangle\cap\langle \textup{doubling in $\scrT_s$}\rangle\\
%%%%%&=
%%%%%\langle \textup{meganomials with a doubling in $\scrT_1,\ldots,\scrT_{s-1},\scrT_{s}$}\rangle\qedhere
%%%%%\end{alignat*}
%%%%%\end{proof}
%%%%%It is no more difficult than this to prove the following:
%%%%%\begin{prop}
%%%%%For $s,t\geq0$, $D^tP^{s+1}(X)_n\subset (\calT_1T^n_1)\cdots (\calT_tT^n_t)X_n$ is the subset consiting of those mega\-monomials which have a 2-fold product in each of the $\calT_2,\ldots,\calT_t$ functors, and which have a $\min\{s,2\}$-fold product in $\calT^1$.
%%%%%\end{prop}
%%%%%\begin{cor}
%%%%%The natural map $D^tD^{s}(X)_n$
%%%%%\end{cor}
%%%%%
%%%%%As $cX\to QcX$ is always surjective, this kernel is also the homotopy fiber. Now there are natural maps $D^s(X)\to c^sX\to X$, forming a tower above $X$, which factors through the tower of inclusions of powers of $X$:
%%%%%\[\xymatrix@R=4mm{
%%%%%\cdots 
%%%%%\ar[r]
%%%%%&%r1c1
%%%%%D^s(X)
%%%%%\ar[r]
%%%%%\ar[d]
%%%%%&%r1c2
%%%%%D^{s-1}(X)
%%%%%\ar[r]
%%%%%\ar[d]&%r1c3
%%%%%\cdots \ar[r]
%%%%%&%r1c4
%%%%%D^1(X)
%%%%%\ar[r]
%%%%%\ar[d]&%r1c5
%%%%%D^0(X)
%%%%%\ar@{=}[d]
%%%%%\\%r1c6
%%%%%\cdots
%%%%%\ar[r]
%%%%%&%r2c1
%%%%%X^{s+1}
%%%%%\ar[r]
%%%%%&%r2c2
%%%%%X^{s}
%%%%%\ar[r]
%%%%%&%r2c3
%%%%%\cdots 
%%%%%\ar[r]&%r2c4
%%%%%X^2
%%%%%\ar[r]
%%%%%&%r2c5
%%%%%X%r2c6
%%%%%}\]
%%%%%% which we'll show factor through $X^{s+1}$, the $(s+1)^{\textup{st}}$ power of $X$. 
%%%%%In particular, for any given $n$, there exists $s=s(n)$ such that $\pi_nD^s(X)\to\pi_n(X)$ is zero.

\end{convergenceOLD}
\begin{conjectured differentials}
\section{Conjectured differentials}
\subsection{First round}
Let's consider differentials of the form $\delta_{i_\ell,\ldots,i_1}\imath\mapsto \SqShift^a\delta_{i_{\ell-1},\ldots,i_1}\imath$. The degrees of these two elements are $(\ell,d+n^1+\ell)$ and $(1,a+\ell-1,2(d+n^1-i_\ell+\ell)-1)$. Thus, in order for them to be connected by $d_r$, which has degree $(r,r-1)$, we'll need $r=a$, and $a-1=d+n^1+\ell-2i_\ell-1$, or
\[a=d+(i_1+1)+\cdots +(i_{\ell-1}+1)-i_\ell+1\textup{, i.e.\ }d+i_1+\cdots +i_{\ell-1}-i_\ell+\ell.\]
\begin{alignat*}{2}
a
&=
d+(i_1+1)+\cdots +(i_{\ell-1}+1)-i_\ell+1%
\\
&=
d+i_1+\cdots +i_{\ell-1}-i_\ell+\ell\\
% Left hand side
% Relation
&=
% Right hand side
d+\ell-\textstyle\sum_{i=1}^\ell m_i% Comment
\end{alignat*}

Now $i_\ell\geq i_{\ell-1}+\cdots +i_2+2i_1$ with equality iff the sequence only admits doublings. When $d=2$, $i_1=d$, so we can catelogue the possible values of $a$ as follows:
\begin{itemise}
\setlength{\parindent}{.25in}
\item $a=1$: $i_\ell$ is a top operation. That (E) cannot hold shows that the only example of this (with $I^2$ empty) is $\delta_{5,2}\imath$ ($\delta_{2}\imath$ is a fringe case).
\item $2\leq a\leq\ell-2$: $i_\ell$ is sub-top (by a distance of $a-1$), with excess profile not 0102 or 0002.
\item $a=\ell-1$: occurs iff $e^1=d+1$. These are the classes that have excess one too high.
\item $a\geq\ell$:  occurs iff $e^1\leq d$. These classes are supposed to be exterior in the abutment, so you want them to have their squares hit (and never to fire).

One wonders: what will hit the nonsensical class $\SqShift^{\ell-1}\delta_{i_{\ell-1},\ldots,i_1}\imath$? It should be $\delta_{i_{\ell},\ldots,i_1}\imath$ so that $d-\sum_{i=1}^\ell m_i=-1$. Thus, all of the abutment classes can have their squares hit by an obvious candidate.
\end{itemise}
Now, of course, $\SqShift^{\ell-1}\delta_{i_{\ell-1},\ldots,i_1}\imath$ isn't unstable, and $\SqShift^{\ell}\delta_{i_{\ell-1},\ldots,i_1}\imath$ is catastrophically non-unstable. Whenever we see $\SqShift^{\ell-1}\delta_{i_{\ell-1},\ldots,i_1}\imath$, replace it with $(\delta_{i_{\ell-1},\ldots,i_1}\imath)^2$, and hope never to see the even worse version. Here's a collection of differentials we conjecture:
\[\cdots, \delta_{20,10,5,2}\overset{d_{3}}{\to}(\delta_{10,5,2})^2,\ 
\delta_{10,5,2}\overset{d_{2}}{\to}(\delta_{5,2})^2,\ 
\delta_{5,2}\overset{d_{1}}{\to}(\delta_{2})^2. \]
\[\cdots, \delta_{36,18,9,4,2}\overset{d_{4}}{\to}(\delta_{18,9,4,2})^2,\ 
\delta_{18,9,4,2}\overset{d_{3}}{\to}(\delta_{9,4,2})^2,\ 
\delta_{9,4,2}\overset{d_{2}}{\to}(\delta_{42})^2. \]
The one $d_1$ differential seen here really lives in the koszul complex for PRLie cohomology, but it's nice to bring them all together. This is how the classes $\delta_{\ldots ,8,4,2}$ become exterior, and how the classes with excess format 0000100002 dissappear.

On the other hand, for $\ell>2$, since (E) cannot hold, $i_\ell$ must be non-top, and so $i_\ell< d+(i_1+1)+\cdots +(i_{\ell-1}+1)$, and $a\geq2$. We get a sequence of differentials, for $n\geq0$:
\[(\delta_{i_\ell,\ldots,i_1}\imath)^{2^n}\overset{d_{(2^n(a-1)+1)}}{\mapsto} (\SqShift^0)^{n}\SqShift^a\delta_{i_{\ell-1},\ldots,i_1}\imath\]
To understand the page number of this differential, note that the equation for calculating the $r$ in which $d_r$ is required is ``$t_\textup{src}+r-1=t_\textup{tar}$''. Now the effect on $t$ of both `squaring' and `precomposing with $\SqShift^0$' is $t\mapsto 2t+1$. This operation doubles $t_\textup{tar}-t_\textup{src}=r-1$.

\subsection*{Effect so far:}
\begin{itemise}
\setlength{\parindent}{.25in}
\item Not all $(\SqShift^0)^{n}\SqShift^a\delta_{i_{\ell^1},\ldots,i_1}\imath$ with  $a\leq\ell^1-2$ are gone: extras are those with $\ell>a\geq\ell-[i_\ell-(d+i_1+\cdots +i_{\ell-1}+2)]$. This looks better with excess notation:
\[\ell>a\geq 2+d+\ell^1-e^1\]
\item All $\delta_{i_\ell,\ldots,i_1}$ are gone, except the exterior generators which are permanent cycles.
\end{itemise}

\subsection{Second round}
Whenever $\ell>j\geq2+d+\ell-e$, there's a differential
\[d_{2(d+\ell^1-e^1-1)+1}:\SqShift^j\delta_{m_\ell,\ldots,m_1}\imath\mapsto\SqShift^{j-1-(d+\ell-e),d+\ell-e}\delta_{m_{\ell-1},\ldots,m_1}\imath\]
To explain the index of the differential:
\begin{alignat*}{2}
r
&=
2^2(d+n_\textup{tar}^1+\ell^1)-2^1(d+n^1+\ell^1+1)+1%
\\
&=
(2^2-2^1)(d+\ell^1)-(2^1-1)-2^1(n^1-2n^1_\textup{tar})%
\\
&=
(2^2-2^1)(d+\ell^1)-(2^1-1)-2^1(i^{\ell^1}-i^{\ell^1-1}-\cdots -i^{1})%
\\
&=
(2^2-2^1)(d+\ell^1)-(2^1-1)-2^1e^1%
\end{alignat*}
We've written this is this way to demonstrate that if we modify this differential by altering both source and target, without changing final sequences, and adding k `doublings and extra operators' to both sides, we get a differential of length
\[r=(2^{1+k})(d+\ell^1-e^1-1)+1.\]
It's $r-1$ that does the doubling, as $r-1=\Delta t$.

%We have postulated the following further differentials:
%\begin{itemise}
%\setlength{\parindent}{.25in}
%\item For $k\geq1$, with $r=2^{1+k}(d+\ell^1-e^1-1)+1$:
%\[\left(\SqShift^j\delta_{m_\ell,\ldots,m_1}\imath\right)^{2^k}\mapsto\SqShift^0\cdots \SqShift^0\SqShift^1\SqShift^{j-1-(d+\ell-e),d+\ell-e}\delta_{m_{\ell-1},\ldots,m_1}\imath\]
%where there are $k$ symbols in `$\SqShift^0\cdots \SqShift^0\SqShift^1$'.
%\item With $r=2^{2}(d+\ell^1-e^1-1)+1$:
%\[\SqShift^0\SqShift^j\delta_{m_\ell,\ldots,m_1}\imath\mapsto\SqShift^0\SqShift^{j-1-(d+\ell-e),d+\ell-e}\delta_{m_{\ell-1},\ldots,m_1}\imath\]
%\item With $r=2^{3}(d+\ell^1-e^1-1)+1$:
%\[\left(\SqShift^0\SqShift^j\delta_{m_\ell,\ldots,m_1}\imath\right)^2 \mapsto\SqShift^0\SqShift^0\SqShift^{j-1-(d+\ell-e),d+\ell-e}\delta_{m_{\ell-1},\ldots,m_1}\imath\]
%\item With $r=2^{3}(d+\ell^1-e^1-1)+1$ (all in excess notation):
%\[\SqShift^0\SqShift^0\SqShift^j\delta_{m_\ell,\ldots,m_1}\imath\mapsto\SqShift^{1,0}\SqShift^{j-1-(d+\ell-e),d+\ell-e}\delta_{m_{\ell-1},\ldots,m_1}\imath\]
%\item When $j=4$ and $\ell^1>5$ you seem to get ($r=2^{2}(d+\ell^1-e^1-1)+1$)
%\[\SqShift^{1,j}\delta_{m_\ell,\ldots,m_1}\imath\mapsto\SqShift^{1,j-1-(d+\ell-e),d+\ell-e}\delta_{m_{\ell-1},\ldots,m_1}\imath\]
%\end{itemise}


\subsection{A discussion of $t-s$}
The degree $ (a_{N+1},\ldots,a_1) $ of a sequence $(I^N,\ldots,I^1)\in\calS^{(N)}$ is:
\begin{align*}
& \left( \ell^N,\ell^{N-1}+n^N,2^{\ell^N}\!\left(\ell^{N-2}+n^{N-1}\right), 2^{\ell^N+\ell^{N-1}}\!\left(\ell^{N-3}+n^{N-2}\right),\ldots \right.\\
 &\qquad \left.{}\ldots,2^{\ell^N+\cdots+\ell^{3}}(\ell^1+n^2),2^{\ell^N+\cdots+\ell^{2}}(d+n^1+\ell^1+1)-1 \right).
\end{align*}
Let's consider the effect on $t-s$ of adding the sequence $I^N$ to $(I^{N-1},\ldots,I^1)$, for $N\geq2$. Let $t_{N-1}$ and $s_{N-1}$ be the dimensions of the sequence $(I^{N-1},\ldots,I^1)$, and $t_N$ and $s_N$ those of $(I^{N},\ldots,I^1)$. Then
\begin{alignat*}{2}
t_N
&=
2^{\ell^N}(t_{N-1}+1)-1%
\\
s_N&=
2^{\ell^N}(s_{N-1}-\ell^{N-1})+\ell^{N}+\ell^{N-1}+n^N%
\\
t_N-s_N&=
%2^{\ell^N}(t_{N-1}+1)-1-[2^{\ell^N}(s_{N-1}-\ell^{N-1})+\ell^{N}+\ell^{N-1}+n^N]%
%\\
%&=
%2^{\ell^N}(t_{N-1}-s_{N-1})+(2^{\ell^N}-1)-[2^{\ell^N}(-\ell^{N-1})+\ell^{N}+\ell^{N-1}+n^N]%
%\\
%&=
2^{\ell^N}(t_{N-1}-s_{N-1})+(2^{\ell^N}-1)(\ell^{N-1}+1)-[\ell^{N}+n^N]%
\\
&\geq
2^{\ell^N}(t_{N-1}-s_{N-1})+(2^{\ell^N}-1)(\ell^{N-1}+1-[\ell^{N-1}-\ell^N+1])%
\\
&=
2^{\ell^N}(t_{N-1}-s_{N-1})+(2^{\ell^N}-1)\,\ell^{N}%
\end{alignat*}
For the inequality here, we have used the fact that $\ell^N+n^N\leq(2^{\ell^N}-1)(\ell^{N-1}-\ell^N+1)$. To understand this, note that the excess of $I^N$ can be no more than $\ell^{N-1}-1$. Out of those sequences of length $\ell^N$ with excess at most $\ell^{N-1}-1$, the one with largest value of $n^N$, when written in excess format, is $(1,\ldots,1,\ell^{N-1}-\ell^N)$ (where the sequence of ones has length $\ell^N-1$). For this sequence:
\begin{alignat*}{2}
\ell^N+n^N
&=
\ell^N+(2^{\ell^N}-1)\cdot(\ell^{N-1}-\ell^N)+ \textstyle\sum_{i=0}^{\ell^N-2}\left(2^{i+1}-1\right)\cdot 1%
\\
&=
\ell^N+(2^{\ell^N}-1)(\ell^{N-1}-\ell^N)+2^{\ell^N}-\ell^N-1%
\\
&=
(2^{\ell^N}-1)(\ell^{N-1}-\ell^N+1).%
\end{alignat*}

In conclusion, we have:
\begin{alignat*}{2}
t_0-s_0
&=
d;%
\\
t_1-s_1
&=
d+n^1;\textup{\ and}\\
t_r-s_r
&\geq
2^{\ell^r}(t_{r-1}-s_{r-1})+(2^{\ell^r}-1)\,\ell^r
\text{\ for $2\leq r\leq N$.}
\end{alignat*}

\subsection{Calculating $\derived\Ind{\LL{k}}$ in small internal dimension}
I guess that in topological degrees less than (say) $2^N$ (?), $\derived\Ind{\LL{k}H^{(N-1)}}$ can be calculated just by Lie algebra homology. Why? There are no $\Q$ operations that can operate in such low dimensions on a sequence of this length, except for $\Q^0$, the restriction, so that the bar construction making the calculation coincides with that in which all the structure has been forgotten but the PRL structure.

\section{Propagation of differentials}
I believe that I've found all of the differentials in the spectral sequence. This appears to be being verified by \verb|investigate_tree(1)|,..,\verb|investigate_tree(3)|. I think that one does all the differentials one can, subject to the below constraints, deciding whether the target should add or app according to a preference for adding. The algorithm is coded in \verb|possible_upgrades(...)|.

What's the method? First, see what we can add to the source --- try to add $\SqShift^{k_0}$ for each $k_0>0$. One calculates $k_1$ under an (add,add), and if it's nonnegative, one adds it in. If $k_1<0$, swap to (add,app) mode, and do that.

Next, unless $\ell^{N_1}+\ell^{N_1-1}-e^{N_1}-(\ell^{N_0}+\ell^{N_0-1}-e^{N_0})-1=0$ we just stop. If this equality holds, we simply see what we can app to the source --- we'll need $j_0$ to be positive and to not increase $e^{N_0}$ as high as $\ell^{N_0-1}$. Other than this constraint, we go for the (app,add), or if this fails, the (app,app).

\subsection{The effect of adding a new bar containing `$\SqShift^k$'}
\[t_{N_0+1}-s_{N_0+1}=
2(t_{N_0}-s_{N_0})+\ell^{N_0}-k\]
Thus, the bonus is $B_0=\ell^{N_0}-k$.

\subsection{The effect of appending `$\SqShift^j$' to the current bar}
\begin{alignat*}{2}
t_{N_1}'
&=
2(t_{N_1}+1)-1%
\\
s_{N_1}'&=
2(s_{N_1}-(\ell^{N_1}+\ell^{N_1-1}+n^{N_1}))+ (\ell^{N_1}+1)+\ell^{N_1-1}+{n'}^{N_1}\\
t_{N_1}'-s_{N_1}'&=2(t_{N_1}-s_{N_1})+2(\ell^{N_1}+\ell^{N_1-1}+n^{N_1}) -(\ell^{N_1}+\ell^{N_1-1}+{n'}^{N_1})\\
&=2(t_{N_1}-s_{N_1})+\ell^{N_1}+\ell^{N_1-1}-({n'}^{N_1}-2n^{N_1})\\
&=2(t_{N_1}-s_{N_1})+\ell^{N_1}+\ell^{N_1-1}-{e'}^{N_1}\\
&=2(t_{N_1}-s_{N_1})+\ell^{N_1}+\ell^{N_1-1}-{e}^{N_1}-j\\
\end{alignat*}
Thus, the bonus is $B_1=\ell^{N_1}+\ell^{N_1-1}-{e}^{N_1}-j$.
Here, we've used the important fact that the excess satisfies $e=n-2n(1\textup{ removed})$.
\subsection{add-propagation}
Here it seems that it don't need to be target-app-gainless, so you get many more of these. We choose (add,add) over (add,app) whenever we can, which is to say whenever what is added to the target is still a non-negative operation.

(add,app) is easier since it exhibits greater gain.

\subsubsection{$(\textup{add},\textup{app})$-propagation}
Suppose that there's a differential from 0 to 1 so that the source dimension is one larger than the target dimension:
\[d_1-d_0+1=0\]
Then we'll need that the bonuses $B_0$ and $B_1$ satisfy
\[1=d'_1-d'_0+2=2(d_1-d_0+1)+B_1-B_0=B_1-B_0.\]
That is,
\[\ell^{N_1}+\ell^{N_1-1}-{e}^{N_1}-j-\ell^{N_0}+k=1\]
In particular:
\[j_1=k_0+\ell^{N_1}+\ell^{N_1-1}-{e}^{N_1}-\ell^{N_0}-1\]

\subsubsection{$(\textup{add},\textup{add})$-propagation}
The relation $B_1-B_0=1$ becomes
\[\ell^{N_1}-k_1-(\ell^{N_0}-k_0)=1\]
\[k_1=k_0+\ell^{N_1}-\ell^{N_0}-1\]

\subsection{app-propagation}
I'm guessing that we only propagate from an `app' when
\[\ell^{N_1}+\ell^{N_1-1}-{e}^{N_1} -(\ell^{N_0}+\ell^{N_0-1}-{e}^{N_0})-1=0.\]
We can choose to (app,add) with a gain of $e^{N_1}-\ell^{N_1-1}$ (a negative quantity), or to (app,app) with zero gain. We choose (app,add) whenever we can, which is to say whenever what is added to the target is still a non-negative operation.

\subsubsection{$(\textup{app},\textup{add})$-propagation}
The relation $B_1-B_0=1$ becomes
\[\ell^{N_1}-k_1-(\ell^{N_0}+\ell^{N_0-1}-{e}^{N_0}-j_0)=1\]
\[k_1=j_0+\ell^{N_1}-(\ell^{N_0}+\ell^{N_0-1}-{e}^{N_0})-1\]
I've only witnessed these when $j_1=j_0$ when attempting app-app. In this case, one finds a negative gain: $e^{N_1}-\ell^{N_1-1}$.
\subsubsection{$(\textup{app},\textup{app})$-propagation}
The relation $B_1-B_0=1$ becomes
\[\ell^{N_1}+\ell^{N_1-1}-{e}^{N_1}-j_1-(\ell^{N_0}+\ell^{N_0-1}-{e}^{N_0}-j_0)=1\]
\[j_1=j_0+\ell^{N_1}+\ell^{N_1-1}-{e}^{N_1} -(\ell^{N_0}+\ell^{N_0-1}-{e}^{N_0})-1\]
That said, I've only ever witnessed these when $j_1=j_0$, so maybe these are `rare'.

\subsection{Can one strip off squares?}
Does (app,add)-propagation ever add a top $\SqShift^{k_1}=\ell^{N_1}-1$ to the target? Well, for this we would need
\[0=j_0-(\ell^{N_0}+\ell^{N_0-1}-{e}^{N_0}) =(j_0+e^{N_0}-\ell^{N_0-1})-\ell^{N_0}\]
which is a negative number minus a positive one. This cannot occur.

Does (add,add)-propagation ever add a top $\SqShift^{k_1}$ to the target? Well, for this we would need
\[0=k_0-\ell^{N_0}\]
which is negative as well, impossible.

What about add-propagation adding a top $\SqShift^{k_1}$ to the source? That is $k_0=\ell^{N_0}-1$. If so, then
\[k_1=\ell^{N_1}-2\]
which works just fine.

I guess this last one isn't such a big deal, since we're interested mainly in seeing whether things are fundamentally sources or targets --- the answer to that question is invariant under stripping off a lonesome top $\SqShift^{k}$.
%\subsection{iterating the relation}
%\begin{alignat*}{2}
%t_N-s_N
%&=
%2^{\ell^N}(t_{N-1}-s_{N-1})+(2^{\ell^N}-1)(\ell^{N-1}+1)-[\ell^{N}+n^N]%
%\\
%&=
%2^{\ell^N}\left(2^{\ell^{N-1}}(t_{N-2}-s_{N-2})+(2^{\ell^{N-1}}-1)(\ell^{N-2}+1)-[\ell^{N-1}+n^{N-1}]\right)+(2^{\ell^N}-1)(\ell^{N-1}+1)-[\ell^{N}+n^N]%
%%
%\\
%&=
%2^{\ell^N}\left((2^{\ell^{N-1}}-1)(\ell^{N-2}+1)-[\ell^{N-1}+n^{N-1}]\right)+(2^{\ell^N}-1)(\ell^{N-1}+1)-[\ell^{N}+n^N]%
%%
%\\
%% Left hand side
%% Relation
%&=
%% Right hand side
%-(2^{\ell^N}n^{N-1}+n^N)+(2^{\ell^N}(2^{\ell^{N-1}}-1)\ell^{N-2}+(2^{\ell^{N}}-1)\ell^{N-1}+\ell^N)
%+(2^{\ell^N}+2^{\ell^{N-1}}-2)% Comment
%
%\end{alignat*}


\end{conjectured differentials}
\begin{grothendieck collapse}
\section{Pre-proof: a collapsing theorem for the GSS}
Assume a multiplicative structure on the Grothendieck sseqs, so that we only have to look at differentials on indecomposables. Assume we've used a clever induction to make sure that the later GSSs collapse.

Model a basis of $H^{(N-1)}\in\LL{n}$ by $\{a_i\}\cup \{b_{j}\}\cup \{c_{j}\}$ where $a_i$ is niether the source nor target of any nontrivial operation, while $b_{j}$ is the source of a single $\Q$ or $P$ operation with image $c_j$. Note that any of the $H^{(N-1)}$ can be written in this form.

By use of quadratic gradings, we see that no differential can leave the evident sub-algebra, so that we can imagine simply that $H^{(N-1)}$ is either $\F_2\{a\}$ or $\F_2\{b,c\}$.

For future reference, recall that in the re-dualised sseq, $d_r$ has degree $(r,-(r-1),0,\ldots)$, and when multiplying in the polynomial algebra, both the first and last degrees are $+1$-shifted.
\subsection{When $H^{(N-1)}=\F_2\{a\}$}
Let $(D_N,\ldots,D_1)$ be the dimension vector of the element $a$.
The spectral sequence in question starts with an $E_2$ term generated as a polynomial algebra by classes $\SqShift^Ja$, of dimension
\[\|\SqShift^Ja\|=(0,\ell(J),D_N+n(J),2^{\ell(J)}D_{N-1},\ldots,2^{\ell(J)}D_{2},2^{\ell(J)}(D_{1}+1)-1)\]
Now we only need to examine the plausibility of differentials
\[d_r:\SqShift^Ja\mapsto \prod_{i=0}^r\SqShift^{I_i}a.\]
The source has dimension as above, while the target has dimension
\[(r,\textstyle\sum\ell(I_i),?,\ldots,?,(\textstyle\sum2^{\ell(I_i)})(D_{1}+1)-1)\]
From the second coordinate, $\sum\ell(I_i)=\ell(J)-(r-1)<\ell(J)$. From the final coordinate, $\sum2^{\ell(I_i)}=2^{\ell(J)}>2^{\sum\ell(I_i)}$, which is impossible, by Jensen's inequality.
\subsection{When $H^{(N-1)}=\F_2\{b,c\}$}
As $H^{(0)}$ is one-dimensional, we must have $N\geq2$. For the record, when $N=2$, we've got $Q^1$ adding a top operation, and when $N\geq3$ we have $Q^0$ sending $\delta_{i}\imath\mapsto \delta_{i+d+1,i}\imath$. Importantly, the dimension of any of these classes is of the form $(0,\ldots,0,?,?)$. If there are any zeros in this dimension, then the only corresponding classes in $H^{(N)}$ are $\SqShift^{\emptyset}b$ and $\SqShift^{\emptyset}c$, which have dimensions of the form $(0,0,\ldots,0,?,?)$ in $H^{(N)}$, and in the $\LL{N+1}$ derived functors, have dimension $(0,0,0,\ldots,0,?,?)$. The second zero here shows that there can be no differentials nonzero on these classes, and so both are permanent cycles.

Thus, we can restrict to the case $N=2$, in which $b=\Q(K)\imath$ and $c=\Q(t_dK)\imath$, in dimensions $(\ell(K),d+n(K)+\ell(K))$ and $(\ell(K)+1,2(d+n(K)+\ell(K)+1)-1)$ as elements of $\LL{2}$. To abbreviate, $\|b\|=(D_2,D_1)$, and $\|c\|=(D_2+1,2(D_1+1)-1)$. The dimensions of classes in $H^2$, viewed as algebra generators, are:
\[\|\SqShift^Ib\|=(0,\ell(I),D_2+n(I),2^{\ell(I)}(D_1+1)-1)\]
\[\|\SqShift^Jc\|=(0,\ell(J),D_2+1+n(J),2^{\ell(J)+1}(D_1+1)-1)\]
The most general target for a $d_{r-1}$ is the $r$-fold product
\[(\prod_{i=1}^{r_1}\Sq^{I_i}b)(\prod_{j=1}^{r_2}\Sq^{J_j}c)\]
which has dimension
\[(r-1,\textstyle\sum\ell(I_i)+\textstyle\sum\ell(J_j),rD_2+r_2+ \textstyle\sum n(I_i)+\textstyle\sum n(J_j), \left(\textstyle\sum2^{\ell(I_i)}+\textstyle\sum2^{\ell(J_j)+1}\right)(D_1+1)-1).\]
The most general source is $\SqShift^L?$, to be found in
\[(0,\ell(L),D_2+n(L)+1?0,2^{\ell(L)+1?0}(D_1+1)-1).\]
One concludes that (with $r\geq3$), $\ell(L)=\textstyle\sum\ell(I_i)+\textstyle\sum\ell(J_j)+(r-2)$, while
\[\left(\textstyle\sum2^{\ell(I_i)}+\textstyle\sum2^{\ell(J_j)+1}\right) =2^{\ell(L)+1?0}.\]
Combining these equations,
\[\left(\textstyle\sum2^{\ell(I_i)}+\textstyle\sum2^{\ell(J_j)+1}\right) =2^{\left(\textstyle\sum\ell(I_i)+ \textstyle\sum\ell(J_j)+(r-2)+1?0\right)}.\]
The occurrences of length combinations satisfying these conditions can be checked using the maple code to follow. It will rule out almost everything, and then one needs to think about the remaining grading...

\textbf{}

\noindent
\verb|for srcadd from 0 to 1 do|\\
\verb| for iadd from 0 to 1 do|\\
\verb|  for jadd from 0 to 1 do|\\
\verb|   for kadd from 0 to 1 do|\\
\verb|    for i from 0 to 20 do|\\
\verb|     for j from i to 20 do|\\
\verb|      for k from j to 20 do|\\
\verb|       if 2^(i+iadd)+2^(j+jadd)+2^(k+kadd)=2^(i+j+k+(3-2)+srcadd) then|\\
\verb|        print(i,j,k,iadd,jadd,kadd,srcadd);|\\
\verb|       fi;|\\
\verb|      od;|\\
\verb|     od;|\\
\verb|    od;|\\
\verb|   od;|\\
\verb|  od;|\\
\verb| od;|\\
\verb|od;|





%Now suppose that $\SqShift^Jc$ is the source of a differential. T






%\textup{}
%
%Now the bonus grading of $c$ is twice that of $b$, so that $b$ alone cannot be the source of a differential. Moreover, $c$ alone can only be the source of a differential 

\end{grothendieck collapse}
\begin{Endmatter}
\printbibliography
\end{Endmatter}
\end{document}










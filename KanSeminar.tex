% !TEX root = z_output/_KanSeminar.tex
%%%%%%%%%%%%%%%%%%%%%%%%%%%%%%%%%%%%%%%%%%%%%%%%%%%%%%%%%%%%%%%%%%%%%%%%%%%%%%%%
%%%%%%%%%%%%%%%%%%%%%%%%%%% 80 characters %%%%%%%%%%%%%%%%%%%%%%%%%%%%%%%%%%%%%%
%%%%%%%%%%%%%%%%%%%%%%%%%%%%%%%%%%%%%%%%%%%%%%%%%%%%%%%%%%%%%%%%%%%%%%%%%%%%%%%%
\documentclass[11pt]{article}
\usepackage{fullpage}
\usepackage{amsmath,amsthm,amssymb}
\usepackage{mathrsfs,nicefrac}
\usepackage{amssymb}
\usepackage{epsfig}
\usepackage[all,2cell]{xy}
\usepackage{sseq}
\usepackage{tocloft}
\usepackage{cancel}
\usepackage[strict]{changepage}
\usepackage{color}
\usepackage{tikz}
\usepackage{extpfeil}
\usepackage{version}
\usepackage{framed}
\definecolor{shadecolor}{rgb}{.925,0.925,0.925}

%\usepackage{ifthen}
%Used for disabling hyperref
\ifx\dontloadhyperref\undefined
%\usepackage[pdftex,pdfborder={0 0 0 [1 1]}]{hyperref}
\usepackage[pdftex,pdfborder={0 0 .5 [1 1]}]{hyperref}
\else
\providecommand{\texorpdfstring}[2]{#1}
\fi
%>>>>>>>>>>>>>>>>>>>>>>>>>>>>>>
%<<<        Versions        <<<
%>>>>>>>>>>>>>>>>>>>>>>>>>>>>>>
%Add in the following line to include all the versions.
%\def\excludeversion#1{\includeversion{#1}}

%>>>>>>>>>>>>>>>>>>>>>>>>>>>>>>
%<<<       Better ToC       <<<
%>>>>>>>>>>>>>>>>>>>>>>>>>>>>>>
\setlength{\cftbeforesecskip}{0.5ex}

%>>>>>>>>>>>>>>>>>>>>>>>>>>>>>>
%<<<      Hyperref mod      <<<
%>>>>>>>>>>>>>>>>>>>>>>>>>>>>>>

%needs more testing
\newcounter{dummyforrefstepcounter}
\newcommand{\labelRIGHTHERE}[1]
{\refstepcounter{dummyforrefstepcounter}\label{#1}}


%>>>>>>>>>>>>>>>>>>>>>>>>>>>>>>
%<<<  Theorem Environments  <<<
%>>>>>>>>>>>>>>>>>>>>>>>>>>>>>>
\ifx\dontloaddefinitionsoftheoremenvironments\undefined
\theoremstyle{plain}
\newtheorem{thm}{Theorem}[section]
\newtheorem*{thm*}{Theorem}
\newtheorem{lem}[thm]{Lemma}
\newtheorem*{lem*}{Lemma}
\newtheorem{prop}[thm]{Proposition}
\newtheorem*{prop*}{Proposition}
\newtheorem{cor}[thm]{Corollary}
\newtheorem*{cor*}{Corollary}
\newtheorem{defprop}[thm]{Definition-Proposition}
\newtheorem*{punchline}{Punchline}
\newtheorem*{conjecture}{Conjecture}
\newtheorem*{claim}{Claim}

\theoremstyle{definition}
\newtheorem{defn}{Definition}[section]
\newtheorem*{defn*}{Definition}
\newtheorem{exmp}{Example}[section]
\newtheorem*{exmp*}{Example}
\newtheorem*{exmps*}{Examples}
\newtheorem*{nonexmp*}{Non-example}
\newtheorem{asspt}{Assumption}[section]
\newtheorem{notation}{Notation}[section]
\newtheorem{exercise}{Exercise}[section]
\newtheorem*{fact*}{Fact}
\newtheorem*{rmk*}{Remark}
\newtheorem{fact}{Fact}
\newtheorem*{aside}{Aside}
\newtheorem*{question}{Question}
\newtheorem*{answer}{Answer}

\else\relax\fi

%>>>>>>>>>>>>>>>>>>>>>>>>>>>>>>
%<<<      Fields, etc.      <<<
%>>>>>>>>>>>>>>>>>>>>>>>>>>>>>>
\DeclareSymbolFont{AMSb}{U}{msb}{m}{n}
\DeclareMathSymbol{\N}{\mathbin}{AMSb}{"4E}
\DeclareMathSymbol{\Octonions}{\mathbin}{AMSb}{"4F}
\DeclareMathSymbol{\Z}{\mathbin}{AMSb}{"5A}
\DeclareMathSymbol{\R}{\mathbin}{AMSb}{"52}
\DeclareMathSymbol{\Q}{\mathbin}{AMSb}{"51}
\DeclareMathSymbol{\PP}{\mathbin}{AMSb}{"50}
\DeclareMathSymbol{\I}{\mathbin}{AMSb}{"49}
\DeclareMathSymbol{\C}{\mathbin}{AMSb}{"43}
\DeclareMathSymbol{\A}{\mathbin}{AMSb}{"41}
\DeclareMathSymbol{\F}{\mathbin}{AMSb}{"46}
\DeclareMathSymbol{\G}{\mathbin}{AMSb}{"47}
\DeclareMathSymbol{\Quaternions}{\mathbin}{AMSb}{"48}


%>>>>>>>>>>>>>>>>>>>>>>>>>>>>>>
%<<<       Operators        <<<
%>>>>>>>>>>>>>>>>>>>>>>>>>>>>>>
\DeclareMathOperator{\ad}{\textbf{ad}}
\DeclareMathOperator{\coker}{coker}
\renewcommand{\ker}{\textup{ker}\,}
\DeclareMathOperator{\End}{End}
\DeclareMathOperator{\Aut}{Aut}
\DeclareMathOperator{\Hom}{Hom}
\DeclareMathOperator{\Maps}{Maps}
\DeclareMathOperator{\Mor}{Mor}
\DeclareMathOperator{\Gal}{Gal}
\DeclareMathOperator{\Ext}{Ext}
\DeclareMathOperator{\Tor}{Tor}
\DeclareMathOperator{\Map}{Map}
\DeclareMathOperator{\Der}{Der}
\DeclareMathOperator{\Rad}{Rad}
\DeclareMathOperator{\rank}{rank}
\DeclareMathOperator{\ArfInvariant}{Arf}
\DeclareMathOperator{\KervaireInvariant}{Ker}
\DeclareMathOperator{\im}{im}
\DeclareMathOperator{\coim}{coim}
\DeclareMathOperator{\trace}{tr}
\DeclareMathOperator{\supp}{supp}
\DeclareMathOperator{\ann}{ann}
\DeclareMathOperator{\spec}{Spec}
\DeclareMathOperator{\SPEC}{\textbf{Spec}}
\DeclareMathOperator{\proj}{Proj}
\DeclareMathOperator{\PROJ}{\textbf{Proj}}
\DeclareMathOperator{\fiber}{F}
\DeclareMathOperator{\cofiber}{C}
\DeclareMathOperator{\cone}{cone}
\DeclareMathOperator{\skel}{sk}
\DeclareMathOperator{\coskel}{cosk}
\DeclareMathOperator{\conn}{conn}
\DeclareMathOperator{\colim}{colim}
\DeclareMathOperator{\limit}{lim}
\DeclareMathOperator{\ch}{ch}
\DeclareMathOperator{\Vect}{Vect}
\DeclareMathOperator{\GrthGrp}{GrthGp}
\DeclareMathOperator{\Sym}{Sym}
\DeclareMathOperator{\Prob}{\mathbb{P}}
\DeclareMathOperator{\Exp}{\mathbb{E}}
\DeclareMathOperator{\GeomMean}{\mathbb{G}}
\DeclareMathOperator{\Var}{Var}
\DeclareMathOperator{\Cov}{Cov}
\DeclareMathOperator{\Sp}{Sp}
\DeclareMathOperator{\Seq}{Seq}
\DeclareMathOperator{\Cyl}{Cyl}
\DeclareMathOperator{\Ev}{Ev}
\DeclareMathOperator{\sh}{sh}
\DeclareMathOperator{\intHom}{\underline{Hom}}
\DeclareMathOperator{\Frac}{frac}



%>>>>>>>>>>>>>>>>>>>>>>>>>>>>>>
%<<<   Cohomology Theories  <<<
%>>>>>>>>>>>>>>>>>>>>>>>>>>>>>>
\DeclareMathOperator{\KR}{{K\R}}
\DeclareMathOperator{\KO}{{KO}}
\DeclareMathOperator{\K}{{K}}
\DeclareMathOperator{\OmegaO}{{\Omega_{\Octonions}}}

%>>>>>>>>>>>>>>>>>>>>>>>>>>>>>>
%<<<   Algebraic Geometry   <<<
%>>>>>>>>>>>>>>>>>>>>>>>>>>>>>>
\DeclareMathOperator{\Spec}{Spec}
\DeclareMathOperator{\Proj}{Proj}
\DeclareMathOperator{\Sing}{Sing}
\DeclareMathOperator{\shfHom}{\mathscr{H}\textit{\!\!om}}
\DeclareMathOperator{\WeilDivisors}{{Div}}
\DeclareMathOperator{\CartierDivisors}{{CaDiv}}
\DeclareMathOperator{\PrincipalWeilDivisors}{{PrDiv}}
\DeclareMathOperator{\LocallyPrincipalWeilDivisors}{{LPDiv}}
\DeclareMathOperator{\PrincipalCartierDivisors}{{PrCaDiv}}
\DeclareMathOperator{\DivisorClass}{{Cl}}
\DeclareMathOperator{\CartierClass}{{CaCl}}
\DeclareMathOperator{\Picard}{{Pic}}
\DeclareMathOperator{\Frob}{Frob}


%>>>>>>>>>>>>>>>>>>>>>>>>>>>>>>
%<<<  Mathematical Objects  <<<
%>>>>>>>>>>>>>>>>>>>>>>>>>>>>>>
\newcommand{\sll}{\mathfrak{sl}}
\newcommand{\gl}{\mathfrak{gl}}
\newcommand{\GL}{\mbox{GL}}
\newcommand{\PGL}{\mbox{PGL}}
\newcommand{\SL}{\mbox{SL}}
\newcommand{\Mat}{\mbox{Mat}}
\newcommand{\Gr}{\textup{Gr}}
\newcommand{\Squ}{\textup{Sq}}
\newcommand{\catSet}{\textit{Sets}}
\newcommand{\RP}{{\R\PP}}
\newcommand{\CP}{{\C\PP}}
\newcommand{\Steen}{\mathscr{A}}
\newcommand{\Orth}{\textup{\textbf{O}}}

%>>>>>>>>>>>>>>>>>>>>>>>>>>>>>>
%<<<  Mathematical Symbols  <<<
%>>>>>>>>>>>>>>>>>>>>>>>>>>>>>>
\newcommand{\DASH}{\textup{---}}
\newcommand{\op}{\textup{op}}
\newcommand{\CW}{\textup{CW}}
\newcommand{\ob}{\textup{ob}\,}
\newcommand{\ho}{\textup{ho}}
\newcommand{\st}{\textup{st}}
\newcommand{\id}{\textup{id}}
\newcommand{\Bullet}{\ensuremath{\bullet} }
\newcommand{\sprod}{\wedge}

%>>>>>>>>>>>>>>>>>>>>>>>>>>>>>>
%<<<      Some Arrows       <<<
%>>>>>>>>>>>>>>>>>>>>>>>>>>>>>>
\newcommand{\nt}{\Longrightarrow}
\let\shortmapsto\mapsto
\let\mapsto\longmapsto
\newcommand{\mapsfrom}{\,\reflectbox{$\mapsto$}\ }
\newcommand{\bigrightsquig}{\scalebox{2}{\ensuremath{\rightsquigarrow}}}
\newcommand{\bigleftsquig}{\reflectbox{\scalebox{2}{\ensuremath{\rightsquigarrow}}}}

%\newcommand{\cofibration}{\xhookrightarrow{\phantom{\ \,{\sim\!}\ \ }}}
%\newcommand{\fibration}{\xtwoheadrightarrow{\phantom{\sim\!}}}
%\newcommand{\acycliccofibration}{\xhookrightarrow{\ \,{\sim\!}\ \ }}
%\newcommand{\acyclicfibration}{\xtwoheadrightarrow{\sim\!}}
%\newcommand{\leftcofibration}{\xhookleftarrow{\phantom{\ \,{\sim\!}\ \ }}}
%\newcommand{\leftfibration}{\xtwoheadleftarrow{\phantom{\sim\!}}}
%\newcommand{\leftacycliccofibration}{\xhookleftarrow{\ \ {\sim\!}\,\ }}
%\newcommand{\leftacyclicfibration}{\xtwoheadleftarrow{\sim\!}}
%\newcommand{\weakequiv}{\xrightarrow{\ \,\sim\,\ }}
%\newcommand{\leftweakequiv}{\xleftarrow{\ \,\sim\,\ }}

\newcommand{\cofibration}
{\xhookrightarrow{\phantom{\ \,{\raisebox{-.3ex}[0ex][0ex]{\scriptsize$\sim$}\!}\ \ }}}
\newcommand{\fibration}
{\xtwoheadrightarrow{\phantom{\raisebox{-.3ex}[0ex][0ex]{\scriptsize$\sim$}\!}}}
\newcommand{\acycliccofibration}
{\xhookrightarrow{\ \,{\raisebox{-.55ex}[0ex][0ex]{\scriptsize$\sim$}\!}\ \ }}
\newcommand{\acyclicfibration}
{\xtwoheadrightarrow{\raisebox{-.6ex}[0ex][0ex]{\scriptsize$\sim$}\!}}
\newcommand{\leftcofibration}
{\xhookleftarrow{\phantom{\ \,{\raisebox{-.3ex}[0ex][0ex]{\scriptsize$\sim$}\!}\ \ }}}
\newcommand{\leftfibration}
{\xtwoheadleftarrow{\phantom{\raisebox{-.3ex}[0ex][0ex]{\scriptsize$\sim$}\!}}}
\newcommand{\leftacycliccofibration}
{\xhookleftarrow{\ \ {\raisebox{-.55ex}[0ex][0ex]{\scriptsize$\sim$}\!}\,\ }}
\newcommand{\leftacyclicfibration}
{\xtwoheadleftarrow{\raisebox{-.6ex}[0ex][0ex]{\scriptsize$\sim$}\!}}
\newcommand{\weakequiv}
{\xrightarrow{\ \,\raisebox{-.3ex}[0ex][0ex]{\scriptsize$\sim$}\,\ }}
\newcommand{\leftweakequiv}
{\xleftarrow{\ \,\raisebox{-.3ex}[0ex][0ex]{\scriptsize$\sim$}\,\ }}

%>>>>>>>>>>>>>>>>>>>>>>>>>>>>>>
%<<<    xymatrix Arrows     <<<
%>>>>>>>>>>>>>>>>>>>>>>>>>>>>>>
\newdir{ >}{{}*!/-5pt/@{>}}
\newcommand{\xycof}{\ar@{ >->}}
\newcommand{\xycofib}{\ar@{^{(}->}}
\newcommand{\xycofibdown}{\ar@{_{(}->}}
\newcommand{\xyfib}{\ar@{->>}}
\newcommand{\xymapsto}{\ar@{|->}}

%>>>>>>>>>>>>>>>>>>>>>>>>>>>>>>
%<<<     Greek Letters      <<<
%>>>>>>>>>>>>>>>>>>>>>>>>>>>>>>
%\newcommand{\oldphi}{\phi}
%\renewcommand{\phi}{\varphi}
\let\oldphi\phi
\let\phi\varphi
\renewcommand{\to}{\longrightarrow}
\newcommand{\from}{\longleftarrow}
\newcommand{\eps}{\varepsilon}

%>>>>>>>>>>>>>>>>>>>>>>>>>>>>>>
%<<<  1st-4th & parentheses <<<
%>>>>>>>>>>>>>>>>>>>>>>>>>>>>>>
\newcommand{\first}{^\text{st}}
\newcommand{\second}{^\text{nd}}
\newcommand{\third}{^\text{rd}}
\newcommand{\fourth}{^\text{th}}
\newcommand{\ZEROTH}{$0^\text{th}$ }
\newcommand{\FIRST}{$1^\text{st}$ }
\newcommand{\SECOND}{$2^\text{nd}$ }
\newcommand{\THIRD}{$3^\text{rd}$ }
\newcommand{\FOURTH}{$4^\text{th}$ }
\newcommand{\iTH}{$i^\text{th}$ }
\newcommand{\jTH}{$j^\text{th}$ }
\newcommand{\nTH}{$n^\text{th}$ }

%>>>>>>>>>>>>>>>>>>>>>>>>>>>>>>
%<<<    upright commands    <<<
%>>>>>>>>>>>>>>>>>>>>>>>>>>>>>>
\newcommand{\upcol}{\textup{:}}
\newcommand{\upsemi}{\textup{;}}
\providecommand{\lparen}{\textup{(}}
\providecommand{\rparen}{\textup{)}}
\renewcommand{\lparen}{\textup{(}}
\renewcommand{\rparen}{\textup{)}}
\newcommand{\Iff}{\emph{iff} }

%>>>>>>>>>>>>>>>>>>>>>>>>>>>>>>
%<<<     Environments       <<<
%>>>>>>>>>>>>>>>>>>>>>>>>>>>>>>
\newcommand{\squishlist}
{ %\setlength{\topsep}{100pt} doesn't seem to do anything.
  \setlength{\itemsep}{.5pt}
  \setlength{\parskip}{0pt}
  \setlength{\parsep}{0pt}}
\newenvironment{itemise}{
\begin{list}{\textup{$\rightsquigarrow$}}
   {  \setlength{\topsep}{1mm}
      \setlength{\itemsep}{1pt}
      \setlength{\parskip}{0pt}
      \setlength{\parsep}{0pt}
   }
}{\end{list}\vspace{-.1cm}}
\newcommand{\INDENT}{\textbf{}\phantom{space}}
\renewcommand{\INDENT}{\rule{.7cm}{0cm}}

\newcommand{\itm}[1][$\rightsquigarrow$]{\item[{\makebox[.5cm][c]{\textup{#1}}}]}


%\newcommand{\rednote}[1]{{\color{red}#1}\makebox[0cm][l]{\scalebox{.1}{rednote}}}
%\newcommand{\bluenote}[1]{{\color{blue}#1}\makebox[0cm][l]{\scalebox{.1}{rednote}}}

\newcommand{\rednote}[1]
{{\color{red}#1}\makebox[0cm][l]{\scalebox{.1}{\rotatebox{90}{?????}}}}
\newcommand{\bluenote}[1]
{{\color{blue}#1}\makebox[0cm][l]{\scalebox{.1}{\rotatebox{90}{?????}}}}


\newcommand{\funcdef}[4]{\begin{align*}
#1&\to #2\\
#3&\mapsto#4
\end{align*}}

%\newcommand{\comment}[1]{}

%>>>>>>>>>>>>>>>>>>>>>>>>>>>>>>
%<<<       Categories       <<<
%>>>>>>>>>>>>>>>>>>>>>>>>>>>>>>
\newcommand{\Ens}{{\mathscr{E}ns}}
\DeclareMathOperator{\Sheaves}{{\mathsf{Shf}}}
\DeclareMathOperator{\Presheaves}{{\mathsf{PreShf}}}
\DeclareMathOperator{\Psh}{{\mathsf{Psh}}}
\DeclareMathOperator{\Shf}{{\mathsf{Shf}}}
\DeclareMathOperator{\Varieties}{{\mathsf{Var}}}
\DeclareMathOperator{\Schemes}{{\mathsf{Sch}}}
\DeclareMathOperator{\Rings}{{\mathsf{Rings}}}
\DeclareMathOperator{\AbGp}{{\mathsf{AbGp}}}
\DeclareMathOperator{\Modules}{{\mathsf{\!-Mod}}}
\DeclareMathOperator{\fgModules}{{\mathsf{\!-Mod}^{\textup{fg}}}}
\DeclareMathOperator{\QuasiCoherent}{{\mathsf{QCoh}}}
\DeclareMathOperator{\Coherent}{{\mathsf{Coh}}}
\DeclareMathOperator{\GSW}{{\mathcal{SW}^G}}
\DeclareMathOperator{\Burnside}{{\mathsf{Burn}}}
\DeclareMathOperator{\GSet}{{G\mathsf{Set}}}
\DeclareMathOperator{\FinGSet}{{G\mathsf{Set}^\textup{fin}}}
\DeclareMathOperator{\HSet}{{H\mathsf{Set}}}
\DeclareMathOperator{\Cat}{{\mathsf{Cat}}}
\DeclareMathOperator{\Fun}{{\mathsf{Fun}}}
\DeclareMathOperator{\Orb}{{\mathsf{Orb}}}
\DeclareMathOperator{\Set}{{\mathsf{Set}}}
\DeclareMathOperator{\sSet}{{\mathsf{sSet}}}
\DeclareMathOperator{\Top}{{\mathsf{Top}}}
\DeclareMathOperator{\GSpectra}{{G-\mathsf{Spectra}}}
\DeclareMathOperator{\Lan}{Lan}
\DeclareMathOperator{\Ran}{Ran}

%>>>>>>>>>>>>>>>>>>>>>>>>>>>>>>
%<<<     Script Letters     <<<
%>>>>>>>>>>>>>>>>>>>>>>>>>>>>>>
\newcommand{\scrQ}{\mathscr{Q}}
\newcommand{\scrW}{\mathscr{W}}
\newcommand{\scrE}{\mathscr{E}}
\newcommand{\scrR}{\mathscr{R}}
\newcommand{\scrT}{\mathscr{T}}
\newcommand{\scrY}{\mathscr{Y}}
\newcommand{\scrU}{\mathscr{U}}
\newcommand{\scrI}{\mathscr{I}}
\newcommand{\scrO}{\mathscr{O}}
\newcommand{\scrP}{\mathscr{P}}
\newcommand{\scrA}{\mathscr{A}}
\newcommand{\scrS}{\mathscr{S}}
\newcommand{\scrD}{\mathscr{D}}
\newcommand{\scrF}{\mathscr{F}}
\newcommand{\scrG}{\mathscr{G}}
\newcommand{\scrH}{\mathscr{H}}
\newcommand{\scrJ}{\mathscr{J}}
\newcommand{\scrK}{\mathscr{K}}
\newcommand{\scrL}{\mathscr{L}}
\newcommand{\scrZ}{\mathscr{Z}}
\newcommand{\scrX}{\mathscr{X}}
\newcommand{\scrC}{\mathscr{C}}
\newcommand{\scrV}{\mathscr{V}}
\newcommand{\scrB}{\mathscr{B}}
\newcommand{\scrN}{\mathscr{N}}
\newcommand{\scrM}{\mathscr{M}}

%>>>>>>>>>>>>>>>>>>>>>>>>>>>>>>
%<<<     Fractur Letters    <<<
%>>>>>>>>>>>>>>>>>>>>>>>>>>>>>>
\newcommand{\frakQ}{\mathfrak{Q}}
\newcommand{\frakW}{\mathfrak{W}}
\newcommand{\frakE}{\mathfrak{E}}
\newcommand{\frakR}{\mathfrak{R}}
\newcommand{\frakT}{\mathfrak{T}}
\newcommand{\frakY}{\mathfrak{Y}}
\newcommand{\frakU}{\mathfrak{U}}
\newcommand{\frakI}{\mathfrak{I}}
\newcommand{\frakO}{\mathfrak{O}}
\newcommand{\frakP}{\mathfrak{P}}
\newcommand{\frakA}{\mathfrak{A}}
\newcommand{\frakS}{\mathfrak{S}}
\newcommand{\frakD}{\mathfrak{D}}
\newcommand{\frakF}{\mathfrak{F}}
\newcommand{\frakG}{\mathfrak{G}}
\newcommand{\frakH}{\mathfrak{H}}
\newcommand{\frakJ}{\mathfrak{J}}
\newcommand{\frakK}{\mathfrak{K}}
\newcommand{\frakL}{\mathfrak{L}}
\newcommand{\frakZ}{\mathfrak{Z}}
\newcommand{\frakX}{\mathfrak{X}}
\newcommand{\frakC}{\mathfrak{C}}
\newcommand{\frakV}{\mathfrak{V}}
\newcommand{\frakB}{\mathfrak{B}}
\newcommand{\frakN}{\mathfrak{N}}
\newcommand{\frakM}{\mathfrak{M}}

\newcommand{\frakq}{\mathfrak{q}}
\newcommand{\frakw}{\mathfrak{w}}
\newcommand{\frake}{\mathfrak{e}}
\newcommand{\frakr}{\mathfrak{r}}
\newcommand{\frakt}{\mathfrak{t}}
\newcommand{\fraky}{\mathfrak{y}}
\newcommand{\fraku}{\mathfrak{u}}
\newcommand{\fraki}{\mathfrak{i}}
\newcommand{\frako}{\mathfrak{o}}
\newcommand{\frakp}{\mathfrak{p}}
\newcommand{\fraka}{\mathfrak{a}}
\newcommand{\fraks}{\mathfrak{s}}
\newcommand{\frakd}{\mathfrak{d}}
\newcommand{\frakf}{\mathfrak{f}}
\newcommand{\frakg}{\mathfrak{g}}
\newcommand{\frakh}{\mathfrak{h}}
\newcommand{\frakj}{\mathfrak{j}}
\newcommand{\frakk}{\mathfrak{k}}
\newcommand{\frakl}{\mathfrak{l}}
\newcommand{\frakz}{\mathfrak{z}}
\newcommand{\frakx}{\mathfrak{x}}
\newcommand{\frakc}{\mathfrak{c}}
\newcommand{\frakv}{\mathfrak{v}}
\newcommand{\frakb}{\mathfrak{b}}
\newcommand{\frakn}{\mathfrak{n}}
\newcommand{\frakm}{\mathfrak{m}}

%>>>>>>>>>>>>>>>>>>>>>>>>>>>>>>
%<<<  Caligraphic Letters   <<<
%>>>>>>>>>>>>>>>>>>>>>>>>>>>>>>
\newcommand{\calQ}{\mathcal{Q}}
\newcommand{\calW}{\mathcal{W}}
\newcommand{\calE}{\mathcal{E}}
\newcommand{\calR}{\mathcal{R}}
\newcommand{\calT}{\mathcal{T}}
\newcommand{\calY}{\mathcal{Y}}
\newcommand{\calU}{\mathcal{U}}
\newcommand{\calI}{\mathcal{I}}
\newcommand{\calO}{\mathcal{O}}
\newcommand{\calP}{\mathcal{P}}
\newcommand{\calA}{\mathcal{A}}
\newcommand{\calS}{\mathcal{S}}
\newcommand{\calD}{\mathcal{D}}
\newcommand{\calF}{\mathcal{F}}
\newcommand{\calG}{\mathcal{G}}
\newcommand{\calH}{\mathcal{H}}
\newcommand{\calJ}{\mathcal{J}}
\newcommand{\calK}{\mathcal{K}}
\newcommand{\calL}{\mathcal{L}}
\newcommand{\calZ}{\mathcal{Z}}
\newcommand{\calX}{\mathcal{X}}
\newcommand{\calC}{\mathcal{C}}
\newcommand{\calV}{\mathcal{V}}
\newcommand{\calB}{\mathcal{B}}
\newcommand{\calN}{\mathcal{N}}
\newcommand{\calM}{\mathcal{M}}

%>>>>>>>>>>>>>>>>>>>>>>>>>>>>>>
%<<<<<<<<<DEPRECIATED<<<<<<<<<<
%>>>>>>>>>>>>>>>>>>>>>>>>>>>>>>

%%% From Kac's template
% 1-inch margins, from fullpage.sty by H.Partl, Version 2, Dec. 15, 1988.
%\topmargin 0pt
%\advance \topmargin by -\headheight
%\advance \topmargin by -\headsep
%\textheight 9.1in
%\oddsidemargin 0pt
%\evensidemargin \oddsidemargin
%\marginparwidth 0.5in
%\textwidth 6.5in
%
%\parindent 0in
%\parskip 1.5ex
%%\renewcommand{\baselinestretch}{1.25}

%%% From the net
%\newcommand{\pullbackcorner}[1][dr]{\save*!/#1+1.2pc/#1:(1,-1)@^{|-}\restore}
%\newcommand{\pushoutcorner}[1][dr]{\save*!/#1-1.2pc/#1:(-1,1)@^{|-}\restore}









\usepackage{fancyhdr}


% Others' practice talks
\excludeversion{GuozhenEqKthyPractice}
\excludeversion{MarkusKtheoryPractice}
\excludeversion{MarkusHopfInvOnePractice}
\excludeversion{KrishanuAdamsStableHtpyPractice}
\excludeversion{JeremyQuillenPractice}
\excludeversion{JandrGammaPractice}
\excludeversion{MarkusLocalisationPractice}
\excludeversion{JeremyModelPractice}
\includeversion{JeremyRationalHomotopyPractice}

% Reading responses
\excludeversion{SerreHomSingFibSpace}
\excludeversion{CharacteristicClasses}
\excludeversion{MilnorGeometricRealization}
\excludeversion{ThomGlobalPropertiesPartOne}
\excludeversion{ThomGlobalPropertiesPartTwo}
\excludeversion{HopfInvOne}
\excludeversion{KervaireNoDifferentiableStructure}
\excludeversion{AtiyahSegalEqKthy}
\excludeversion{AdamsStableHtpy}
\excludeversion{QuillenSpectrumOfEqCohomology}
\excludeversion{SegalCategoriesAndCohomologyTheories}
\excludeversion{BousfieldLoc}
\excludeversion{MoorePostnikovSystems}
\excludeversion{KTheoryQuillen}
\excludeversion{RationalHomotopyQuillen}

%\def\excludeversion#1{\includeversion{#1}}

% Others' talks
\excludeversion{JeremyThomTalk}
\excludeversion{JandrKervaire}
\excludeversion{MarkusKtheory}
\excludeversion{GuozhenEqKthy}
\excludeversion{MarkusHopfInvOne}
\excludeversion{KrishanuAdamsStableHtpy}
\excludeversion{JeremyQuillen}
\excludeversion{JandrGammaSpaces}
\excludeversion{MarkusLocalisation}
\excludeversion{KrishanuKanComplexes}
\excludeversion{JeremyModelCategories}
\excludeversion{JeremyRationalHomotopy}

% My talks
\excludeversion{SteenrodTalk}
\excludeversion{SerreCohModTwoEilMacLane}
\excludeversion{SignatureThmTalk}
\excludeversion{Operads}



\rfoot{\footnotesize Michael Donovan}
\newcommand{\KanSemResponse}[1]
{
\thispagestyle{fancy}
\subsection*{#1}
}

\begin{document}
\begin{SerreHomSingFibSpace}
\KanSemResponse
{``Homologie singuli\`ere des espaces fibr\'es'' --- J.-P.\ Serre}
The first part of this paper consists of a clear exposition of the methods used thereafter to produce spectral sequences. All the technology that is to be used is displayed with such clarity that one feels almost equipped to finish the writing oneself. (Strangely,) I hadn't really been familiar with the derivation of a spectral sequence from a filtered differential graded algebra, as I had seen it before I had started to become comfortable with spectral sequences. Thus, reading this derivation was a pleasure.

I had been happy with the `births and deaths' interpretation of the Serre spectral sequence which we discussed, but I hadn't noticed that the differentials therein could be so closely connected with the actual differentials in a singular (actually cubical) chain complex for the total space. This really does seem reasonable, on reflection.

As I have used the Serre spectral sequence before, it was a pleasure to see how Serre sets up the machinery. For example, the discussion in chapter 1 of the transgression greatly clarifies the standard description of the transgression in the Serre spectral sequence. To me, this description had been a black box, and the derivation given has relieved me of the notion that something complicated was going on.

In the second chapter, Serre defines cubical homology, which seems perfectly suited to his goals. Given a Serre fibration, he filters the cubical chains $u$ on the total space according to how deep a projection $I^n\to I^p$ the projection of $u$ to the base factors through. This has the virtue that a $p+q$-cube which lies in the $p^\text{th}$ filtration induces a $p$-cube on the base and a $q$-cube on the fibre (after assuming the base and fibre connected and adjusting the cubical complexes of total and base spaces to only include those with all corners at the basepoints). The induced chain homotopy between the $E_0$ page and the tensor product of chains on the base with those on the fibre make it much easier to properly understand the early pages of the spectral sequence.

The local coefficients are built seamlessly into the picture --- I had until now been afraid of them. In particular, after reading this, I am happier with the proof I may present on Wednesday regarding the homology of the $\pi$-adic construction, which uses the Serre spectral sequence with base space $B\pi$.

From this point the paper turns to applications, many familiar, some unfamiliar. On particularly charming idea is to take a space and repeatedly take the universal cover and then apply $\Omega$, to drag down the homotopy groups. It is this type of operation which Serre seems to apply often with admirable confidence. Anyhow, this process is used to show that the rational homotopy of odd spheres is zero, which is really a fantastic fact. This too relies on Serre's mod $\calC$ theory.

Along the same lines, it is shown where the first $p$-torsion arises in the homotopy groups of a sphere. Both of these proofs have evaded my complete understanding, as they are in French, and when I have a little more time, I will try to decode them.

Although I'm looking forward to reading about these applications in more detail, along with the other applications that I have not mentioned, I think that the main thing I will get out of this paper is clarity regarding the workings of the Serre spectral sequence. I felt recently that I had for the first time understood the differentials, using the births/deaths analogy which we discussed, coming from an exact triangle. Although this perspective is great when the coefficients are trivial, the $E_1$ page is not really canonical, and the multiplicative structure is a mystery. Now, however, this paper has opened up a route to understanding which is perhaps more direct, ``natural from $E_0$'' (as Michael Andrews will emphasise), and in which the multiplicative structure is finally transparent.

\pagebreak
\end{SerreHomSingFibSpace}
\begin{SteenrodTalk}
\KanSemResponse
{A derivation of the Steenrod Squares}
Choose $p$ prime, $n>0$, $\pi\subseteq \Sigma_n$. Fix spaces $E\pi\overset{r}{\to} B\pi$, and basepoints $e\mapsto b$.
\[\xymatrix@R=.3cm{
X^{(n)}\ar[r]^i\ar@{=}[d]&D_\pi X\ar@{=}[d]&\ar[l]_jB\pi_+\wedge X\ar@{=}[d]\\
X^n/FW\ar[r]^{\text{incl}}&\frac{E\pi\times_\pi X^n}{E\pi\times_\pi FW}&
\ar[l]_\Delta\frac{B\pi\times X}{B\pi\times\{*\}}\\
y\ar@{|->}[r]&(e,y),\ (f,\Delta x)&\ar@{|->}[l](r(f),x)
}
\xymatrix@R=.3cm{
&&\widetilde H^q(X)\ar[lld]_{u\mapsto u^{\wedge n}}\ar[ld]^{P_\pi}
\ar[dd]_{\text{SQ}}\\
\widetilde H^{nq}(X^{(n)})&\ar[l]^{i^*}\widetilde H^{nq}(D_\pi X)
\ar[rd]^{j^*}\\
&&\widetilde H^{nq}(B\pi_+\wedge X)
}
\]
If $n=p=2$ and $\pi=\Z_2$, we have $\widetilde H^{nq}(B\pi_+\wedge X)=\Z_2[x]\otimes \widetilde H^*(X)$. Thus we can write $\text{SQ}(u)=\sum x^{q-i}\otimes\Squ^iu$, where $\Squ^iu\in \widetilde H^{q+i}(X)$.

We are working over a field, so that homology and cohomology are dual. Thus to understand a cohomology class, it is enough to understand how it pairs with homology classes. So, we should take $s\in H_{q+i}(X)$ and calculate $(\Squ^iu)(s)$. To do so, one takes $t_{q-i}\in H_{q-i}(\RP^\infty)$ dual to $x^{q-i}$,\footnote{$(x^{q-i}\otimes \Squ^i(u))(t_{q-i}\otimes s)=\Squ^i(u)(s)$.}
and calculates the pairing of $P_\pi(u)$ with $j^*(t_{q-i}\otimes s)$.

\subsection*{Thom class vs.\ Steenrod power}
If $\xi$ is an ($R$-oriented) $\R^n$-bundle, get an $n$-disk bundle $D(\xi)$ and an $(n-1)$-sphere sub-bundle $S(\xi)$.
\[\xymatrix@!0@R=12mm@C=2.5cm{
\iota\\
\tau(\xi)\ar@{|->}[u]
}
\raisebox{-.6cm}{\qquad in $H^n(\DASH;R)$ of\qquad}
\xymatrix@!0@R=12mm@C=2.5cm{
(D^n,S^{n-1})\ar[d]\\
(D(\xi),S(\xi))
}\]
Very similarly:
\[\xymatrix@!0@R=12mm@C=2.5cm{
u^{(n)}\\
P_\pi u\ar@{|->}[u]\ar@{|->}[d]\\
\text{SQ}u
}
\raisebox{-.6cm}{\qquad in $H^n(\DASH;R)$ of\qquad}
\xymatrix@!0@R=12mm@C=2.5cm{
(X^n,FW)\ar[d]^i\\
(E\pi\times_\pi X^n,E\pi\times_\pi FW)\\
(B\pi\times X,B\pi\times\{*\})\ar[u]_j
}\]

\pagebreak
\subsection*{Construction of Steenrod Power $P_\pi$}
\begin{lem*}\label{lemaboutpiadic}
Suppose that $\widetilde H^i (X) = 0$ for $i < q$ and that $\widetilde H^q (X)$ is finite dimensional.  Then,
\[\widetilde H^i(D_\pi X)=
\begin{cases}0;&\textup{if }i<nq;\\(\widetilde H^q(X)^{\otimes n})^\pi;&\textup{if }i=nq,%;\\???;&\textup{if }i>nq
\end{cases}
\]
and $i_X^*:\widetilde H^{nq}(D_\pi X)\to \widetilde H^{np}(X^{(n)})$ is the inclusion of the $\pi$-invariants $(\widetilde H^q(X)^{\otimes n})^\pi\subset\widetilde H^q(X)^{\otimes n}$.
\end{lem*}
\begin{proof}
%One could produce a CW-complex for $D_\pi X$ having no cells of dimension less than $nq$. No, one cannot! What if $X$ is a $K(X/3,d)$ and the field is $X/2$? Then $H^*(X;Z/2)=0$, but there's plenty of cells.
We have a map (drawn with dotted arrows) of bundle-subbundle pairs:
\[\xymatrix@!0@C=1.7cm@R=.8cm{
F_{n-1}X^n\ar[dd]\ar[rd]\ar@{.>}[rr]&&F_{n-1}X^n\ar'[d][dd]\ar[rd]\\
&X^n\ar[dd]\ar@{.>}[rr]&&\ar[dd]X^n\\
F_{n-1}X^n\ar[dd]\ar[rd]\ar@{.>}'[r][rr]&&E\pi\times_\pi F_{n-1}X^n\ar'[d][dd]\ar[rd]\\
&X^n\ar[dd]\ar@{.>}[rr]&&\ar[dd]E\pi\times_\pi X^n\\
\{b\}\ar@{.>}'[r][rr]\ar@{=}[rd]&&B\pi\ar@{=}[rd]\\
&\{b\}\ar@{.>}[rr]&&B\pi
}\]
\begin{itemize}\squishlist
\item $H^i(X^n,FW)=\widetilde H^i(X^{(n)})$ is zero for $i<nq$, $H^*(X)^{\otimes n}$ when $i=nq$, and the identification is equivariant w.r.t.\ permutations.
\item $E_2^{s,t} = H^s(B\pi; \{H^t(X^n, F_{n-1} X^n)\}) \Rightarrow \widetilde H^{s+t}(D_\pi X)$ is zero below row $nq$.
\item Have edge homomorphism: 
\[\xymatrix{
H^{nq}(E\pi\times_\pi X^n,E\pi\times_\pi FW)\ar@{->>}[r]^{\text{\qquad\qquad\qquad \ \ (iso)}} &E^{0,nq}_\infty\ar@{^{(}->}[r]^{\text{(iso)}}&E^{0,nq}_2\ar@{^{(}->}[r]^{\text{(inv)}}&E_1^{0,nq}\ar@{=}[r]&H^q(X^n,FW)
}\]
To get from $E_1^{0,nq}$ to $E_2^{0,nq}$ (after which the group is stable) one takes the $\pi$-invariants (recall that in Serre's model, one can take $E_1^{0,nq}$ to be $C^0(B)\otimes H^{nq}(X^n,F_{n-1}X^n)$, and only use cubes with corners at the basepoint).\qedhere
\end{itemize}
\end{proof}
\begin{itemise}
\item If $X=K_q=K(\F_p,q)$, this applies. Moreover, $K^q(K(\F_p,q);\F_p)$ is the field $\F_p\langle\imath_q\rangle$.
\item Thus $i^*:\widetilde H^{nq}(D_\pi K_q)\to \widetilde H^{np}(K_q^{(n)})$ is an isomorphism. There is then a unique $P_\pi\imath_q$ mapping to $\imath_q^{\wedge n}$ under $i^*$.
\item Now suppose that $u:X\to K_q$ represents $u\in \widetilde H^q(X)$. Need
\[\xymatrix{
\widetilde H^q(X)\ar[r]^{P_\pi}&\widetilde H^q(D_\pi X)\\
\widetilde H^q(K_q)\ar[r]^{P_\pi}\ar[u]^{u^*}&\widetilde H^q(D_\pi K_q)\ar[u]_{(D_\pi u)^*}
}\]
Thus must have $P_\pi u=(D_\pi u)^*P_\pi\imath_q$. This map is natural, is the only way to extend $P_\mu$ from what it needs to be on $K_q$, and satisfies the formula, since it is satisfied in the universal case.
\end{itemise}

\pagebreak

\subsection*{Properties}
\begin{itemize}
\item $\Squ^i u=0$ for $i<0$ (universality) and for $i>q$ (no negative powers of $x$).
\item $\Squ^q u=u^2$.
\[\xymatrix@R=.6cm{
B\pi_+ \wedge X \ar[r]^j & D_\pi X \\
S^0 \wedge X  \ar[r]^\Delta\ar[u]^{k\wedge 1} & X^{(2)}\ar[u]^i
}
\raisebox{-.5cm}{\textup{\qquad inducing:\qquad}}
\xymatrix@R=.6cm{
\sum_{i=0}^q x^{q-i} \otimes \Squ^i u\ar@{|->}[d]&
\ar@{|->}[l]  Pu \ar@{|->}[d]\\
u\smile u & u^{\wedge2}\ar@{|->}[l]
}\raisebox{-.5cm}{\textup{\qquad on $\widetilde H^{2q}(\DASH)$.\qquad}}
\]
\item Cartan formula. Have map $\delta:D_\pi(X\wedge Y)\to D_\pi(X)\wedge D_\pi(Y)$ which is $\Delta$ on $E_\pi$ components.
\[\xymatrix@!0@C=4.4cm@R=1.8cm{
%\xymatrix{
(X\wedge Y)^{(2)}\ar[r]^i\ar[d]_T&D_\pi(X\wedge Y)\ar[d]^{\delta}_{(\Delta_{E\pi_+})}\\
X^{(2)}\wedge Y^{(2)}\ar[r]^{i\wedge i}&D_\pi X\wedge D_\pi Y
&B\pi_+\wedge X\wedge Y\ar[ul]_j\ar[d]^{\Delta_{B\pi_+}}\\
&&B\pi_+\wedge X\wedge B\pi_+\wedge Y\ar[ul]_{j\wedge j}
}\]
\[\qquad\qquad\xymatrix@!0@C=4.4cm@R=1.8cm{
(\imath_p\wedge \imath_q)^{(2)}&
\ar@{|->}[l]  P(\imath_p\wedge\imath_q)\ar@{|->}[dr]\\
\imath_p^{(2)}\wedge \imath_q^{(2)} \ar@{|->}[u]& P(\imath_p)\wedge P(\imath_q)\ar@{|->}[l]\ar@{|..>}[u]
&\sum x^{p+q-k}\otimes\Squ^k(u\wedge v)\\
&&(\sum x^{p-i}\otimes\Squ^i(u))\wedge(\sum x^{q-j}\otimes\Squ^j(v))\ar@{|..>}[u]\ar@{|->}[ul];[]
}\]
If $\delta^*$ does the right thing, then we're winners. It must, as in the universal case, $i^*$ is injective.
\item Write $\Squ\,u=\sum \Squ^iu$. The Cartan formula says $\Squ(uv)=(\Squ\,u)(\Squ\,v)$.
\item Exercise: show that $\Squ^0 s = s$, where $s$ is the generator of $\widetilde H^1 (S^1) = \Z_2$.
\item $\Squ^k$ is stable --- it commutes with suspension. (The suspension is ``smash with $s$''\footnote{$\widetilde H^q (X)=\widetilde H^q (X)\otimes\widetilde H^1(S^1)\xrightarrow{\ \wedge e\ }\widetilde H^{q+1} (\Sigma X)$}, and so $\Squ^k(\sigma u) = \Squ^k(u \wedge e) = \Squ^k u \wedge \Squ^0 e = \sigma \Squ^k u$.)
In particular, it is additive.
\item 
$\Squ^0: \widetilde H^q (X) \to \widetilde H^q (X)$ is the identity. (The map $\widetilde H^q(K^q)\to\widetilde H^q(S^q)$ induced by the fundamental class $s^{\wedge q}\in\widetilde H^q(S^q)$ is an isomorphism. Thus it's enough to check this on $s^{\wedge q}$, which follows from the exercise and Cartan.)
\item
$\Squ^1$ is the Bockstein.
\item The Adem relations are satisfied. For $a<2b$: 
\[
\Squ^a \Squ^b = \sum_{j=0}^{\lfloor a/2\rfloor} \binom{b-j-1}{a-2j} \Squ^{a+b-j} \Squ^j.
\]
Note that $a<2b\implies a+b-j\geq2j$.
\item $\Squ^{i_1}\cdots\Squ^{i_r}$ is admissible if $i_1\geq 2i_2,\,\i_2\geq 2i_3$, etc. The Adem relations let us write any such product as a sum of admissible products.
\item Let $\Steen$ be the free unital $\Z_2$-algebra on symbols $\Squ^i$ for $i>0$, mod the Adem relations.
\begin{itemise}
\item $\Steen$ acts on $\widetilde H^*(X)$ --- the free algebra obviously does, and this action descends to $\Steen$.
\item $\Steen$ has a basis the admissible products.
\item $\Steen$ is generated by the $\Squ^{2^i}$, which are indecomposable. (Do it by induction: use admissibility and Adem.)
\end{itemise}
\item Application: Suppose that $\widetilde H^*(X;\Z_2)=\Z_2[x]$, with $|x|=q$. Then $q=2^i$. (For $x^2=\Squ^{q}x\neq0$, but the intermediate cohomology groups between $q$ and $2q$ are zero.)
\item Application: Suppose that $f:S^{2n-1}\to S^n$ has odd Hopf invariant. Then $n=2^i$.
\end{itemize}

\pagebreak
\end{SteenrodTalk}
\begin{SerreCohModTwoEilMacLane}
\KanSemResponse
{``Cohomologie modulo 2 des complexes d'Eilenberg-MacLane'' --- J.-P.\ Serre}
\subsection*{Cohomology Operations}
\begin{itemise}
\item $H^p(K(\pi,q);G)$ is in bijection with natural transformations $\widetilde H^{q}(\DASH;\pi)\to\widetilde H^{p}(\DASH;G)$.
\item The groups $H^{n+q}(K(\pi,q);G)$ are stable when $q>n$. Elements thereof are called ``stable cohomology operations''. The Steenrod algebra is the algebra of stable cohomology operations $H^*(\DASH;\Z_2)$. To see this, we should calculate $H^{n+q}(K(\pi,q);G)$ for large $q$.
\item We'll use the path-loop fibration $K(\Z_2,q)\to *\to K(\Z_2,q+1)$. By induction:
\[\text{$H^*(K(\Z_2,q);\Z_2)=\Z_2[\Squ^I(\imath_q)]$, over those admissible $I$ with excess less than $q$.}\]
\item Need Borel's theorem:
\begin{itemize}\squishlist
\item A graded ring $R$ over $\Z_2$ has an order set $x_1,x_2,\ldots$ of homogeneous elements as a \emph{simple system of generators} if the monomials $\{x_{i_1}\cdots x_{i_r}:i_1<\cdots<i_r\}$ form a $\Z_2$-basis, and each graded part is finite dimensional.
\item
Suppose $F\rightarrow E\downarrow B$ is a fibre space with $E$ acyclic, and that $H^*(F;\Z_2)$ has a simple system $\{x_\alpha\}$ of transgressive generators. Then $H^*(B;\Z_2)$ is a polynomial ring in the $\{\tau(x_\alpha)\}$.
\end{itemize}
\item When $q=1$: have $\RP^\infty$, the poly alg on $\Squ^\emptyset\imath_1$.
\item Inductive step:
\begin{itemize}\squishlist
\item There's a simple system of generators $(\Squ^I\imath_q)^{2^r}$.
\item This equals $\Squ^{2^{r-1}(q+n(I)),\ldots,2(q+n(I)),(q+n(I))}\Squ^I\imath_q$.
\item As $I$ runs over all admissibles with $e(I)<q$, ${2^{r-1}(q+n(I)),\ldots,2(q+n(I)),(q+n(I))},I$ runs over all with $e(I)<q+1$.
\end{itemize}
\item What about:
\[\text{$H^*(K(\Z_{2^h},q);\Z_2)=\Z_2[\Squ^I_h(\imath_q)]$, over those admissible $I$ with excess less than $q$.}\]
Here, we swap $\Squ^1$ for $\Squ^1_h$, the $h^\text{th}$ Bockstein operator.

Here, the starting case is different: $H^*(K(\Z_{2^h},1);\Z_2)=\Lambda(u_1)\otimes\Z_2[\Squ_h^1(u_1)]$.
\item What about:
\[\text{$H^*(K(\Z,q);\Z_2)=\Z_2[\Squ^I(\imath_q)]$, over admissible $I$, $e(I)<q$, $I_\text{last}>1$.}\]
Here, the starting case is different: $H^*(K(\Z,1);\Z_2)=\Lambda(u_1)$.
\end{itemise}
\subsection*{Stable Operations}
Let's consider $H^{n+q}(K(\Z_2,q);G)$ for $q>n$. Basis corresponding to admissible sequences with degree $n$ (and excess less than $q$, but the excess condition is vacuous). Thus you get the whole degree $n$ part of the Steenrod algebra.
\subsection*{$H^p(K(\pi,q);\Z_2)$ for $\pi$ finitely generated abelian}
Using the K\"unneth formula to turn $\oplus$ to $\otimes$, only need to know it for $\Z$ and $\Z_{p^r}$. Know when $p=2$. If $p$ is odd, then ($q>1$) you can use mod $\calC$ theory (where $\calC$ is the class of ``abelian torsion groups of finite exponent $n$ coprime to $p$'').

Note that if one is interested in stable groups, the K\"unneth formula turns $\oplus$ to $\oplus$.
\subsection*{Poincar\'e Polynomials}
Let $\theta(\pi;q)$ be the Poincar\'e polynomial for $H^*(K(\pi,q),\Z_2)$.
\subsubsection*{The case $\pi=\Z_{2^h}$}
\begin{itemize}\squishlist
\item Have poly alg on an element of dimension $q+n(I)$ for each admissible $I$ with $e(I)<q$.
\begin{align*}
\theta(\Z_2;q)&=\prod_{e(I)<q}(1-t^{q+n(I)})^{-1}\\
&=\prod_{h_1\geq h_2\geq\cdots\geq h_{q}=0}(1-t^{\sum2^{h_j}})^{-1}
\end{align*}
\item To see the last equality, take a sequence $I=(i_1,\ldots,i_r)$.
\begin{itemize}\squishlist
\item Let $\alpha_1=i_1-2i_2$,\ldots, $\alpha_r=i_r$.
\item Let $\alpha_0=q-1-\sum\alpha_i$.
\item $n(I)=\sum (2^i-1)\alpha_i=\sum_{i\geq0}2^i\alpha_i-(q-1)$.
\item Thus generators of dimension $N$ are in bijection with $(\alpha_0,\ldots,\alpha_r)$ with sum $q-1$ and
\[N=1+2^1\alpha_1+\cdots+2^r\alpha_r.\]
There are $q$ summands, each a power of $2$. Let $h_1,\ldots, h_q$ be the powers, from biggest to smallest.
\end{itemize}
\end{itemize}
\subsubsection*{The case $\pi=\Z$}
\begin{itemize}\squishlist
\item Have poly alg on an element of dimension $q+n(I)$ where $e(I)<q$, $I_\text{last}\neq1$ (for $q>1$).
\begin{align*}
\theta(\Z;q)&=\mathop{\prod_{e(I)<q,\ I_\text{last}\neq1}}(1-t^{q+n(I)})^{-1}\\
&=\prod_{h_1> h_2\geq\cdots\geq h_{q-1}=0}(1-t^{\sum2^{h_j}})^{-1}
\end{align*}
\begin{itemize}\squishlist
\item Now need $\alpha_r>1$, so that $h_1=h_2$.
\item But then we can always swap $2^{h_1}+2^{h_2}$ for $2^{h_1+1}$.
\item Thus we get one less summand and a strict last inequality.
\end{itemize}
\item
\begin{itemize}\squishlist
\item $\frac{\theta(\Z_2;q-1)}{\theta(\Z;q)}$ has those indexed by $h_1$ to $h_{q-1}$ with $h_1=h_2$.
\item One can swap $2^{h_1}+2^{h_2}$ for $2^{h_1+1}$ again, and get the product for $\theta(\Z;q-1)$. Thus
\[\theta(\Z;q)=\frac{\theta(\Z_2;q-1)\cdot\theta(\Z_2;q-3)\cdots}{\theta(\Z_2;q-2)\cdot\theta(\Z_2;q-4)\cdots}\]
\end{itemize}
\end{itemize}
\pagebreak
\subsubsection*{Convergence}
\begin{itemize}\squishlist
\item These converge for $|t|<1$.
\item They have a ``dominant singularity'' at $t=1$.
\begin{itemize}\squishlist
\item Write $t=1-2^{-x}$, so $t\to1$ slowly as $x\to\infty$. Write:
\[\phi(\pi;q)(x)=\log_2(\theta(\pi;q)(t)).\]
\item Serre calculates:
\[\phi(\Z_2;q)\sim x^q/q!\text{\qquad and\qquad}\phi(\Z;q)\sim x^{q-1}/(q-1)!\qquad\qquad\]
\item As $\oplus\mapsto\otimes$ on $H^*$, $\oplus\mapsto\oplus$ on $\phi$. Thus:
\begin{thm*}
Suppose $\pi$ is f.g. with $s$ $\Z$ summands and $r$ $\Z_{2^h}$ summands. Then:
\[\phi(\pi;q)\sim \begin{cases}
r\cdot x^q/q! & r\geq1,\\
s\cdot x^{q-1}/(q-1)! & r=0,\,s\geq1,\\
0& r=s=0.
\end{cases}\]
\end{thm*}
\end{itemize}
\end{itemize}
\subsubsection*{Topological Application}
\begin{thm*}
Suppose $X$ is simply connected and:
\begin{enumerate}\squishlist
\item The $H_i(X;\Z)$ are finitely generated;
\item The $H_i(X;\Z_2)$ vanish for $i\gg0$;
\item $H_i(X;\Z_2)$ is nonzero for some $i\gg0$.
\end{enumerate}
Then $\pi_i(X)$ has a $\Z$ or $\Z_2$ subgroup for infinitely many $i$.
\end{thm*}
\begin{exmp*}
In $S^n$ there are finitely many $\Z$ summands, and thus infinitely many $\Z_2$ summands.
\end{exmp*}
\begin{proof}
\begin{itemize}\squishlist
\item By Hurewicz\footnote{Suppose that $X$ is simply connected, and that a Serre class $\scrC$ satisfies \textup{2A} and \textup{3}. Then if $\pi_i(X)\in\scrC$ for $i<n$ then $H_i(X)\in\scrC$ for $i<n$ and $h:\pi_n(X)\to H_n(X)$ is a $\scrC$-isomorphism.} mod $\calC$,\footnote{$\calC$ is the class of ``finitely generated abelian groups''} the homotopy groups are finitely generated. So it's enough to see that infinitely many $\pi_i\otimes\Z_2$ are nonzero.
\item For a contradiction, suppose that $q$ is maximal such that $\pi_q\otimes \Z_2\neq0$.
\item Let $j$ be the least positive integer such that $H_j(X;\Z_2)\neq0$.
\item By Hurewicz mod $\calC$,\footnote{$\calC$ is the class of ``abelian torsion groups of finite exponent $n$ coprime to $2$''} $j$ is the first index such that $\pi_j\otimes \Z_2\neq0$. Thus $q\geq j\geq2$.
\item Let $X_i=(X,i)$ be $X$ with the first $i-1$ homotopy groups killed, and let $A(t)$ be the Poincar\'e polynomial for $H^*(A;\Z_2)$, for any space $A$ where these are finite dimensional.
\item There's a fibration\footnote{Build $X_{i+1}$ from $X_{i}$ by attaching cells to $X_{i}$ to make it a $K(\pi_i,i)$. Then have inclusion $X_{i}\to K(\pi_i,i)$ whose fibre we call $X_{i+1}$.} $X_{q+1}\to X_q\to K(\pi_q,q)$. $H^*(X_{q+1};\Z_2)$ is trivial, by Hurewicz mod $\calC$. Thus the SSS gives $X_{q}(t)=\theta(\pi_q;q)$.
\item %The mod 2 Serre spectral sequence shows that $H^*(X_q;\Z_2)=H^*(K(\Pi,q);\Z_2)$.
Now there are fibrations:
%\[X_{q+1}\to X_{q}\to K(\pi_{q};q),\]
\[K(\pi_{q-1},q-2)\to X_{q}\to X_{q-1},\]
\[K(\pi_{q-2},q-3)\to X_{q-1}\to X_{q-2},\]
\[K(\pi_{q-3},q-4)\to X_{q-2}\to X_{q-3},\cdots\]
\[K(\pi_{2},1)\to X_{3}\to X_2.\]
\item Whenever $F\to E\to B$ is a fibration with $B$ simply connected, we have, on \emph{every coefficient}, $E(t)\leq F(t)\cdot B(t)$ (if all the terms make sense). The $E_2$ page has Poincar\'e poly $F(t)\cdot B(t)$, but the differentials cut down dimensions.
\item Thus, on coefficients:
\begin{alignat*}{2}
\theta(\pi_q;q)&= X_q(t)\\
&\leq X_{q-1}(t)\cdot\theta(\pi_{q-1};q-2)\leq\cdots\\
&\leq X(t)\cdot \prod_{1<i<q}\theta(\pi_i;i-1).
\end{alignat*}
\item $X(t)$ is a polynomial, so its values on $[0,1]$ are bounded by $h$. Taking $\log_2$:
\[\phi(\pi_q;q)(x)\leq\log_2h+\sum_{i=2}^{q-1}\phi(\pi_i;i-1)(x).\]
This is a contradiction --- the terms on the right grow at most as fast as $x^{q-2}$ as $x\to\infty$, but those on the right grow as either $x^q$ or $x^{q-1}$.\qedhere
\end{itemize}
\end{proof}

\pagebreak
\end{SerreCohModTwoEilMacLane}
\begin{CharacteristicClasses}
\KanSemResponse
{``Characteristic Classes'' --- Milnor \& Stasheff}
I took this reading assignment as a chance to read in more detail much of what I had read earlier, but not understood as well. At the time of writing I've read the first 11 chapters of the text.

It was nice to see the Stiefel-Whitney classes applied early in the exposition - a bit of motivation never goes astray. In particular, the Cartan formula can be used to compare the classes of the tangent and normal bundles (given an immersion into Euclidean space), and one can obtain lower bounds in this way for the dimension of the ambient space. One is finding invariants of the twistedness of the normal bundle, and something very twisted needs room to be so. In particular for $\RP^n$, when $n$ is a power of two, this gives a lower bound which is as strong as possible.

Of the things that I have enjoyed particularly, one thing that stands out was the way that the cohomology of the Grassmannian $BO(n)$ was calculated, and the way that this calculation hints at the structures underlying the Stiefel-Whitney classes.

The approach is to assume the existence of (possibly non-unique) Stiefel-Whitney classes, and then to prove that the classes of the universal bundle are algebraically independent, by calculating on a well understood example. The Stiefel-Whitney classes of the universal $n$-bundle then generate a polynomial algebra on one generator in each degree from 1 to $n$, putting a lower bound on the dimension of $BO(n)$. Next, a cell structure is put on $BO(n)$, minimal in the very strong sense that there are no differentials in the cellular cochain complex with $\Z_2$ coefficients. This can be shown simply by noting that the dimension in degree $i$ of the polynomial subalgebra equals the number of cells of dimension $i$.

Anyway, this proof really did four things. It gave an explicit cell structure on the Grassmannian (which I didn't know), it gave a painless calculation of the mod 2 cohomology thereof (although it is only painless now that I am more comfortable with the proof of the properties of the squares), and in the meantime, it even said something about the attaching maps --- they all have even degree (after collapsing onto lower cells). Finally, it showed the uniqueness of the Stiefel-Whitney classes.

\subsubsection*{The Thom isomorphism}
This reading gave me a chance to think a little more about the Thom isomorphism, which comes up in chapter 9. Perhaps while the following text lacks clarity, and possibly correctness, it is an honest representation of the way I currently view this material.

My current preferred way of thinking about it works when the base $M^m$ of an (oriented) $R^n$-bundle $E$ is a closed (oriented) $m$-manifold. Let $D$ and $S$ be the disk and sphere bundles of $E$. Then we have Poincar\'e duality on the pair $(D,S)$ (an (oriented) $m+n$-manifold with boundary), and the Thom class $u\in H^n(D,S)$ is dual to the homology class $[B]\in H_m(D,S)$.

Given a relative $k$-chain $s=\sum_{i}n_i\sigma_i$ in $(D,S)$, it can be deformed so that each simplex $\sigma_i:\Delta^k\to D$ is smooth near $B$ and transverse to $B$, in which case, $\sigma_i^{-1}(B)$ is an $n-k$-dimensional submanifold of $\Delta^k$. This submanifold can be triangulated, yielding an $(n-k)$-chain in $B$. Adding these up over the simplices $\sigma_i$, we obtain an $n-k$-chain $\overline s$ in $B$. 

Now a cohomology class in $H^k(D,S)$, being equivalent to a class in $H^k(E,E_0)$, can be taken to be supported along $B$. The value on $s$ of a cohomology class supported along $B$ should be determined by $\overline s$ and a sign related to the orientation with which $s$ intersected $B$. Thus, such a cohomology class should be the cup product of a class in $H^{n-k}(B)$ and the Thom class (which detects with orientation the number of intersections of an $n$-chain with $B$).

I suppose the purpose of writing these paragraphs was first to subject them to your scrutiny and second to claim that I can now better understand the Euler class. The Euler class is the restriction $u|_D$ of the Thom class to $D$, viewed as an element of $H^n(B)$ (as $B$ and $D$ are homotopic). As such it is the image under the Thom isomorphism of $u\cup u\in H^{2n}(D,S)$. $u\cup u$ is dual under Poincar\'e duality to the self-intersection of the submanifold $B$.

In the setting where $n=m$ and ($B$ is connected), this shows that $u\cup u$ is the self-intersection number of $B$ multiplied by the generator of $H^{2n}(D,S)$, explaining the relationship between the Euler class of the tangent bundle and the Euler characteristic of the manifold $B$.

A couple more cool things that I feel like mentioning:
\begin{itemize}
\item It is shown that the dual class (in $H^k(A)$) to an embedding $M\to A$ (of codimension $k$) of a closed smooth submanifold restricts to the Euler (or top SW) class of $M$. This is used to show that $\RP^{2^i}$ will not smoothly embed as a closed subset of $\R^{2^{i+1}-1}$.
\item The derivation of Poincar\'e duality (with field coefficients) using the diagonal class of the embedding $M\to M\times M$. I'd never thought about the diagonal class until now, and now I even have a formula for it!
\end{itemize}



%In this case, the Euler class, which is 

%Any relative $k$-chain in $(D,S)$ should be homotopic to one with the following property: whenever it exits $S$, 

%somehow it is transverse to $B$ in $n$-dimensions, and in the remaining $k-n$ dimensions, it runs parallel to $B$. Obviously this makes no sense. But anyway, 
%It seems to me then that the point of the Thom isomorphism is that any

\pagebreak
\end{CharacteristicClasses}
\begin{MilnorGeometricRealization}
\KanSemResponse
{``The Geometric Realization of a Semi-simplicial Complex'' --- John Milnor}
What I have gained most from this paper, and from Guozhen's talk, is the realisation that simplicial sets and their realisations are not too confusing. Until now, I had a latent belief that the geometric realisation was a strange, `thick' object, somehow with more `mass' than I desired. Instead, it turns out that the relations imply that every point in the realisation belongs to the interior of a unique nondegenerate simplex --- it's all very well organised in there.

In fact, the construction that Milnor uses of the realisation is really very simple --- one just takes a disjoint union of the realisations of all of the simplices, and glues `faces' onto faces, and `degenerate simplices' degenerately onto simplices. The cells of the resulting CW-complex are exactly the simplices who were lucky enough not to get glued degenerately onto a lower dimensional one. Moreover, it is not strange to include degenerate simplices in our definition of simplicial set --- their presence mimics the situation in the singular simplicial set of a topological space (and I suspect there are even better rationales around for this) --- when calculating singular homology groups, we certainly don't discard degenerate simplices before forming the singular chain complex, even though they don't appear to tell us anything about the geometry of the space that the nondegenerate simplices did not. Again, perhaps this discussion seemed a little silly, but it was honest!

I suppose now that one has realised that the geometric realisation isn't such a nasty process, one might expect that it should be compatible with reasonable operations on simplicial sets, such as products. There's a little point here to be made. The natural map $\eta:|K\times K'|\to|K|\times|K'|$ is a bijection, as Milnor verifies by constructing a set-theoretic inverse $\overline\eta$. He constructs $\overline\eta$ through a careful examination of the cells of $|K\times K'|$. Moreover, the map $\overline\eta$ is continuous on each product cell of $|K|\times|K'|$. Thus as long as $|K|\times|K'|$ actually inherits the structure of a CW-complex from its product cells, $\overline\eta$ is a homeomorphism. Fortunately, this happens under various conditions on the `size' of $K$ and $K'$.

I like this point, because it is an example of why CW-complexes are so good to work with. This gave me pause to think about why `compactly generated' is a good adjective to place in front of the noun `space'. The continuity of a map out of such a space is determined `locally', which in this case means `over compact subsets'. 

We would like to see that the `total singular complex' functor $S$ is right adjoint to the realisation functor:
\[|\DASH|:\sSet\longleftrightarrow\Top:S\]
My tendency is to try to understand such adjunctions as this by analogy --- there may be better ways --- and I suppose one analogy would be with the adjunction between the free and forgetful functors in various settings. The realisation should be the way to build a topological space which is as `free' as possible using the simplices to hand (while satisfying the relations between then), while applying $S$ should be the clumsiest way to return a simplicial complex given a space. Now write $\phi$ and $\theta$ for the maps
\[\phi:\Top(|T|,X)\cong\sSet(T,SX): \theta.\]
The appropriate definition of $\phi$ is obvious, but $\theta$ is a little bit trickier, and to understand it, one needs to understand the counit map $|SX|\to X$. The formula for this counit it clear enough, and chasing through this point of view has made me understand the whole picture far better.

Guozhen, in his talk, gave a perspective on this adjunction which doesn't feature in the paper. It runs as follows: to check that the above adjunction holds, it is enough to observe that $|\DASH|$ preserves colimits (apparently this holds `almost by definition', which I'm beginning to believe as I write this response). Then one checks the adjunction holds when $T=\Delta^n$, and then notes that for any $T\in\sSet$ we may write
\[K=\colim_{\Delta/K}\Delta^n,\]
where $\Delta/K$ is the category of simplices $\Delta^n$ over $K$. I still need to really sit down with this and work out the details.

What's cool about this is the idea that $\sSet$ is built from simplices, just as the category of CW-complexes, and so we can understand colimit-preserving functors from $\sSet$ by examining them on simplices. In fact, that the functor $|\DASH|$ is left adjoint to $S$ turned out to be equivalent to the fact that it preserved colimits.

In fact, after writing thus far, one notes that if $F:\calC\to\calD$ is a functor which preserves colimits, and $G:\calD\to\calC$ is any functor, then these are adjoint \Iff there is a collection $\{c_i\}$ of objects of $\calC$, generating $\calC$ under colimits, such that $\calD(Fc_i,d)\cong\calC(c_i,Gd)$. The functor $S$ is of course perfectly designed to play the role of $G$.

At this late stage, one should say a few words about the Dold-Kan correspondence, and about Eilenberg-MacLane complexes. Let it just be said that after this reading and talk, this material seems much less distant.

\pagebreak
\end{MilnorGeometricRealization}
\begin{ThomGlobalPropertiesPartOne}
\KanSemResponse
{``Some Global Properties of Differentiable manifolds'' --- R.\ Thom --- Part I}
This reading response will discuss material up to the end of chapter II.

The first chapter of this paper consists of a series of results that one would sincerely hope exist. Although I will admit that I haven't studied the proofs of these results, I'm glad that the proofs exist. What seems most crucial to me is the existence of a tubular neighbourhood isomorphic to the disk bundle of the normal bundle, in order to discuss Poincar\'e duality using the Thom isomorphism and a Thom collapse map.

The question addressed in chapter II is whether or not a class $z\in H_{n-k}(V^n)$ is realisable by means of a submanifold \emph{whose normal bundle has structure group reducible to $G$}, for $G$ a closed subgroup of $O(k)$. At first, I thought that the restriction on structure group sounded a little hard to handle, involving questions of factorisation through various morphisms $BG\to BO(k)$. Thom shows, however, that this is no great hindrance, at least when $G$ is such that the cohomology of $BG$ can be calculated.
Moreover, he provides a single framework in which to pose the unoriented, oriented and framed questions all at once.

Anyway, once one has seen why the question is so natural, each step the proof of theorem II.1\footnote{That $z$ is realisable by such a submanifold \Iff the Poincar\'e dual is pulled back from the Thom class on $M(G)$.} seems equally natural. However, I don't yet have a very good feel for why theorem II.1 should really be the case. Every part of the argument stands to reason --- one takes a submanifold, classifies the normal bundle, observes that the Thom class pulls across to the Poincar\'e dual, etc., but for me there's still a level of intuition missing.

There are definitely some good results which follow from this. In particular, there is a cool result about realising homology classes $z\in H_{n-k}(V^n)$ with $n-k<n/2$ by \emph{framed} submanifolds (when $V^n$ is oriented). In this case, there is some $N=N(n,k)$ such that $Nz$ can be realised thus for any such $z$. I think that this is a surprising result.\footnote{Perhaps what should be considered surprising though is the result of Serre from which this is derived.}

At this point, I must defer further discussions to the next response. I'll make sure that I cover the rest before the next talk --- things have been particularly hectic this week. Sorry, Haynes.

\pagebreak
\end{ThomGlobalPropertiesPartOne}
\begin{JeremyThomTalk}
\KanSemResponse
{Jeremy's Talk on Thom's paper}
\renewcommand{\Steen}{\scrA_2}
We'd really like to calculate the homotopy groups
\[\pi_*({MBO}(r)),\qquad \pi_*({MBSO}(r)).\]
We'll do the first, which is easier, and that's all we have time for.

Let's just start trying to understand the cohomology thereof and the Steenrod operations thereupon (the cohomology should be the most accessible topological invariant). We know: 
\[H^{r+i}(MBO(r))\cong H^i(BO(r))=\Z_2[\text{Stiefel-Whitney classes}]\]
Although we know the cohomology ring, we aren't completely sure yet as the action of the steenrod algebra. Anyway, we know that
\[H^i(BO(r))\cong H^i(BO(s))\text{ where $i\leq r,s$}\]
so $H^{r+i}(MBO(r))$ is independent of $r$ for $i\leq r$, and we define $H^i(MBO)$ to be this stable value. This connotes that we're taking a spectrum, but we're not so sophisticated. We can define $H^*(MBO)=\bigoplus H^i(MBO)$. For example, $H^0(MBO)=H^r(MBO(r))$ for large $r$, and this is just a one-dimensional $\F_2$-vector space generated by the Thom class $U$.

\subsection*{Digression on the Steenrod algebra}
Let $\Steen$ be the Steenrod algebra. For a space $X$, $H^*(X)$ is as graded $\Steen$-module. A nice space $X$ to use here is $X=\RP^\infty\times\cdots\times\RP^\infty$. We have a map
\funcdef{\omega:\Steen}{H^*(X)}{a}{a(u_1\otimes\cdots\otimes u_n)}
$\omega$ is injective in degree $\leq n$, and this is how we see that the Adem relations are the only relations. This is the simplest case where we see the linear independence of a bunch of the admissible squares.

\begin{prop*}
Have $\psi:\Steen\to\Steen\otimes\Steen$ the unique algebra homomorphism\footnote{To check it we need to use a bit of knowledge about the Steenrod.} taking $\Squ^i$ to the Whitney sum $\sum \Squ^{i_1}\otimes\Squ^{i_2}$ --- the comultiplication map.
\end{prop*}
\begin{proof}
Let $A$ be the free associative algebra on the $\Squ^i$. There's a projection $\pi:A\to\Steen$. The definition of $\psi$ definitely extends to a map $A\to\Steen\otimes\Steen$. We need to check that $\ker\pi$ is contained in $\ker\psi$. But:
\[\xymatrix{
A\ar[r]^{\psi\quad}\ar[d]_\pi&\Steen\otimes\Steen
\ar[r]^{\omega\otimes\omega\qquad}&H^*(X)\otimes H^*(X)\ar[r]^{\quad\sim}&H^*(X\times X)\\
\Steen\ar[rrru]_{\omega\times\omega}\ar@{..>}[ur]
}\]
And $\omega\otimes\omega$ is injective in a range which depends on how many factors we choose in $X$.
\end{proof}
Also have a counit map $\epsilon:\Steen\to\Z_2$ which takes $\Squ^0\mapsto 1$ and $\Squ^i\to0$ for $i>0$. These satisfy coherence conditions:
\[\xymatrix{\Steen\ar[r]\ar[d]&\Steen\otimes\Steen\ar[d]&\\
\Steen\otimes\Steen\ar[r]&\Steen\otimes\Steen\otimes\Steen}\]
This is enough to make $\Steen$ a bialgebra. As $\epsilon$ is an isomorphism between the degree 0 part and the field:
\begin{fact*}
$\Steen$ is a connected associative, coassociative and cocommutative Hopf algebra.
\end{fact*}
\begin{rmk*}
Haynes said some things which I only half remember. Something like:
``If a bialgebra is connected, there's exactly one antipode making it a group object in coalgebras. If the bialgebra is either commutative or cocommutative, then the antipode is an involution.''

\end{rmk*}
\noindent 
\subsection*{Back to $MBO(r)$}

We should note that there's a Whitney sum map 
\[BO(r)\times BO(s)\to BO(r+s)\]
giving
\[MBO(r)\wedge MBO(s)\to MBO(r+s)\]
and on cohomology
\[\psi:\widetilde H^* (MBO(s+r))\to \widetilde H^*( MBO(s))\otimes \widetilde H^* (MBO(r))\]
\[\epsilon: H^*(MBO)\to\F_2\qquad 1\mapsto 1\]
This structure makes $H^*(MBO)$ a coalgebra\footnote{Should be associative because Whitney sums are!} and a left $\Steen$-module. $a\in\Steen$ acts on $H^{n+r}(MBO(r))$ (for $r$ very large) via the natural action. This should be compatible with the stabilisation process. There's one sticky point --- cupping with the Thom class is \emph{not} natural, but we'll apply this twice, so what we end up with \emph{is} correct.
\begin{thm*}
Let $A$ be a connected Hopf algebra, and let $M$ be a connected co-algebra which is also an $A$-module. Suppose also that $\psi:M\to M\otimes M$ is an $A$-module homomorphism. Let $\nu:A\to M$ be defined by $\nu(a)=a\cdot1$, where $1\in M^0$ is the unit (well defined by connectivity). Suppose that $\nu$ is injective. Then $M$ is a free $A$-module.
\end{thm*}
\begin{proof}[Sketch of proof]
Do the dumbest thing in the world. Define $N=M/A_+M$, where $A_+$ is the positive degree part of $A$. Choose any vector space splitting $f:N\to M$. Define $C:A\otimes N\to M$ by $a\otimes n\mapsto af(n)$. Once you see that this is an isomorphism, you're done, and you can show this by induction on degree.
\end{proof}
\noindent To apply this to our example, we should check:
\begin{prop*}
$\nu:\Steen\to H^*(MBO)$ sending $a\mapsto a(U)$ is injective.
\end{prop*}
\begin{proof}
Assume that $a\in\Steen^n$ (the $n\fourth$ degree part) has $\nu(a)=0$. Think of this as happening in:
\[H^r(MBO(r))\to H^{r+n}(MBO(r))\qquad U\mapsto a(U).\]
As this is the universal case, it's enough to pull all this back, and examine
\[H^r(MBO(1)\wedge\cdots\wedge MBO(1))\to H^{r+n}(MBO(1)\times\cdots\times MBO(1))\]
This is a general thing --- the splitting principle --- these constructions don't see the difference between the split and nonsplit cases. $U=x_1\cdots x_r$ where $x_i$ is the Thom class of $MBO(1)$. Note that $MBO(1)=\RP^\infty$. Now write $a=\sum \Squ^I$ in the admissible sequence basis. Now how does $\Squ^I$ hit $U$? Inductively, we see
\[a(U)=\sum x_1^{2^{v_1}}\cdots x_r^{2^{v_r}}\]
where $v_i=i_i-2i_{i+1}$. This implies that $\Squ^i$ is injective.
\end{proof}
 Now note that $H^*(MBO$) is a free $\Steen$-module, due to the proposition and theorem above.
\subsection*{Some consequences}
Lets derive some consequences. When we destabilise, this says:
\[H^{n+r}(MBO(r))\text{ is a free $\Steen$-module for $r\geq n$.}\]
Moreover, (as some stuff is zero):
\[H^{i}(MBO(r))\text{ is a free $\Steen$-module for $i\leq 2r$ .}\]
Let's choose a basis $x_1,\ldots,x_m$ for $H^*(MBO(r))$ in the range $*\leq 2r$.
Suppose $x_i$ lives in dimension $n_i$, $r\leq n_i\leq 2r$. Each $x_i$ induces a map $\phi_i:MBO(r)\to K(\Z_2,n_i)$, these maps can be glued together to get
\[\phi:MBO(r)\to \prod_{i=1}^n K(\Z_2,n_i)=K.\]
Consider
\[ H^j(K)=\bigoplus_{\sum j_\alpha=j}H^{j_1}(K(\Z_2,n_i))\otimes\cdots\otimes H^{j_n}(K(\Z_2,n_i)).\]
For $j\leq 2r$, the only way we can get anything here is when all but one of the $j_i$ are zero, so
\[H^j(K)\cong \bigoplus_{i=1}^m H^j(K(\Z_2,n_i))\cong \bigoplus_{i=1}^m \Steen^{j-n_i}\]
Now $\phi^*:H^j(K)\to H^j(MBO(r))$ sends
\[(a_1,\ldots,a_m)\mapsto a_1x_1+ a_2x_2+ \cdots+ a_mx_m\]
and because this stuff is free, in terms of a bases, this is an isomorphism. That is, $\phi$ is a $\Z_2$-cohomology isomorphism in the range $j\leq 2r$.

Now the cohomology of $MBO(r)$ is universally zero in mod $p$ coefficients.\footnote{c.f.\ Milnor-Stasheff. One can use Pontrjagin classes and Steenrod powers.}  Thus we have an iso with mod $p$ coefficients (of zero objects). Thus $\phi$ is a homology isomorphism in this range. As everything is simply connected, it's a $\pi_*$-equivalence in this range! In particular,
\[\frakN_n=\varinjlim \pi_{n+r}(MBO(r))\]
And it's going to follow, just by counting, that:
\begin{cor*}
For large $r$, $\dim \pi_{n+r}(MBO(r))=p_{nd}(n)$, the number of partitions of $n$ that include no numbers one less than a power of 2.
\end{cor*}
\subsection*{Applications}
We have seen that the Hurewicz homomorphism from $\pi_*(MBO(r))\to H^*(MBO(r))$ is injective for $\ast< 2r$. So (using the duality of homology and cohomology)
\[\varinjlim_r \pi_{n+r}(MBO(r))\to H^{n+r}(MBO(r))\]
is an injection from the cobordism groups of an $n$-manifold. This tells us that the Stiefel-Whitney numbers of a manifold determine its cobordism class.

Now $p_{nd}(3)=0$, so that every closed $3$-manifold is null-cobordant. This was once considered very hard.

\pagebreak
\end{JeremyThomTalk}
\begin{ThomGlobalPropertiesPartTwo}
\KanSemResponse
{``Some Global Properties of Differentiable manifolds'' --- R.\ Thom --- Part II}
There are a number of different candidates for the moral of this paper, which is perhaps a reason it seems to be considered a classic. I'll try to reveal those which I've perceived in this response.

Thom constructs, as discussed in Krishanu's talk, a bijection between the set $L_{n-k}(V)$ of $L$-cobordism classes of $(n-k)$-dimensional submanifolds on $V$ (an $n$-dimensional manifold) and the homotopy classes of maps $[V,MO(k)]$. Now $L$-equivalence classes of $r$-dimensional submanifolds on $S^n$, for large $n$, are just the same as cobordism classes of $r$-manifolds. In particular, we would like to calculate $[S^n,MO(r+n)]$ for $n$ large --- this will be the unoriented cobordism group $\frakN^r$.

Unfortunately, calculating these homotopy groups threatens to be very tricky indeed. We know the cohomology algebra of $MO(m)$ --- by the Thom isomorphism it is essentially a shift of that of $BO(m)$, which is a polynomial algebra on the Stiefel-Whitney classes of the universal $m$-bundle. However, here, it would seem hard to know how to turn this information into homotopical information. It's enough to give a map to a space with known homotopy groups which is an isomorphism on $H^*$ in a range, but one would have to guess the space, and describe the map.

Now Thom determined earlier the cohomology algebra of $MO(k)$, being isomorphic to the ideal of the cohomology algebra of $BO(k)$ generated by the top Stiefel-Whitney class. This was done geometrically by comparing the map $BO(k-1)\to BO(k)$ with the map from $EO(k)\times_{O(k)}S^{k-1}\to BO(k)$. Thus he has an understanding of the basis of the cohomology groups, from which he can construct a collection of classifying maps to Eilenberg-MacLane spaces. Thus he gets a map from $MBO(k)$ to a product of Eilenberg-MacLane spaces. 

Now this map is actually a cohomology isomorphism in a range, as can be determined using the action of the Steenrod algebra $\Steen$. I have been having trouble fully digesting this part of the argument, although Jeremy's explanation in his practice talk was helpful. I suppose the moral here is that the action of the Steenrod algebra provides an ample serving of extra structure for our purposes, as it does in a variety of problems we have seen.

Here is a moment where I'm not sure how one would anticipate the result proven. In particular, it seems strange that the spectrum $MO$ should be a wedge of Eilenberg-MacLane spectra. I'm not sure what it really means for this to be the case, although Michael Andrews points out to me that it shows $MO$ to be quite a weak cohomology theory (while apparently $MU$ is much more interesting).

My main problem with this paper is that I have found it hard to read. I found myself swimming in the middle sections for far too long, and becoming a little anxious to put it down, which is a shame. Perhaps I'd have found progress a little easier with a briefer presentation. Moreover, I feel that I could have benefited more from this paper if it had followed through with its threats to discuss the topic in terms of spectra.

I'm sure I'll have an opportunity to take a closer look at this material post-Kan seminar, perhaps from a more modern viewpoint.

\pagebreak
\end{ThomGlobalPropertiesPartTwo}
\begin{SignatureThmTalk}
\KanSemResponse
{The Signature Theorem}
%\begin{enumerate}
%\item
%Fix $n$. Seek to bound the rank of $\Omega_n$.
%\begin{itemise}
%\item Choose $k,p\geq n+2$.
%\item Let $\widetilde \gamma$ be the universal oriented $k$-bundle over $\widetilde G=B(SO(k))=\widetilde G_k(\R^\infty)$.
%\item Let $\widetilde \gamma'$ be the tautologous $k$-bundle over $\widetilde G'=\widetilde G_k(\R^{k+p})$.
%\item $\widetilde G'$ is a subcomplex of $\widetilde G$ containing the $(n+1)$-skeleton.\footnote{For the number of $n$-cells in $G_k(\R^{k+p})$ equals the number of partitions of $n$ into at most $k$ integers, each of which is at most $p$.}
%\item We thus have maps:
%\[\Omega_n\leftfibration\pi_{n+k}(T(\widetilde\gamma'))\overset{\calC}{\to}H_{n+k}(T(\widetilde\gamma'))\cong H_{n}(\widetilde G')\cong H_n(\widetilde G)\]
%\item By the universal coefficient theorem:
%\[0\to\Ext(H_{n-1}(\widetilde G),\Z[1/2])\to H^{n}(\widetilde G;\Z[1/2])\to \Hom(H_n(\widetilde G),\Z[1/2])\to0.\]
%Now the Ext group is finite as $H_{n-1}(\widetilde G)$ is finitely generated,\footnote{In fact, $\Ext(\Z/(2^r(2s+1))\Z,\Z[1/2])\cong \Z/(2s+1)\Z$, and of course $\Ext(\Z,\Z[1/2])=0$.} so:
%%\[p(n/4)=\rank_{\Z[1/2]}\left[H^n(\widetilde G,\Z[1/2])\right]=\rank_{\Z[1/2]}\left[\Hom(H_n(\widetilde G),\Z[1/2])\right]=\rank_\Z H_n(\widetilde G).\]
%\begin{alignat*}{2}
%\rank_\Z \left[H_n(\widetilde G)\right]&=\rank_{\Z[1/2]}\left[\Hom(H_n(\widetilde G),\Z[1/2])\right]&\quad&\text{(as $\Z[1/2]$ is torsion free)}\\
%&=\rank_{\Z[1/2]}\left[H^n(\widetilde G,\Z[1/2])\right]&&\text{(as the $\Ext$ group is finite)}\\
%&=p(n/4).
%\end{alignat*}
%So, for all $n$, $\Omega_n$ is finitely generated, with rank $p(n/4)$. $\Omega_*\otimes\Q$ is polynomial with generators the even complex projective spaces.
%\end{itemise}
%\end{enumerate}
%\pagebreak

\subsection*{Remarks on oriented $4k$-dimensional manifolds:}
\begin{enumerate}
\item There is a nondegenerate symmetric bilinear form on $H^{2k}(M^{4k})$.
\item The signature is called the signature $\sigma(M^{4k})$ of $M^{4k}$. 
\item The signature is a \emph{genus} --- a ring homomorphism from $\Omega_*\to\Q$.
\item $\sigma$ must factor through $\Omega_*\to\Omega_*\otimes\Q$. We'll study $\Omega_*\otimes \Q$ using Pontrjagin numbers.
\end{enumerate}
\subsection*{Chern and Pontrjagin classes}
\begin{itemise}
\item An $n$-dimensional complex vector bundle $\omega\downarrow X$ has Chern classes: characteristic classes $c_i(\omega)\in H^{2i}(X)$, $i=0,\ldots,n$.
\item These satisfy a Cartan formula, $c_0=1$, and a normalisation, and are pulled back from:
\[H^*(BU(n);\Z)=\Z[c_1(\gamma^n),c_2(\gamma^n),\ldots,c_n(\gamma^n)]\text{\quad with $|c_i|=2i$.}\]
\item Define the Pontrjagin classes of an $n$-dimensional real vector bundle $\xi\downarrow X$ to be the even Chern classes of the complexification $\xi\otimes\C$, with a sign:
\[p_i(\xi)=(-1)^ic_{2i}(\xi\otimes\C)\in H^{4i}(X;\Z).\]
We could abbreviate $p(\xi):=c_{\pm ev}(\xi\otimes\C)$.
\item Ignore the odd Chern classes --- they have order 2, as a complexification is isomorphic to its conjugate.
\item The total Pontrjagin class inherits a Cartan formula which holds modulo elements of order 2.
\item If $M^{4k}$ is an oriented $4k$-manifold, for each partition $I=(i_1,\ldots,i_r)\in\calP(k)$, we have:
\[p_I[M]=p_{i_1}\cdots p_{i_r}[M]=\langle p_{i_1}(\tau_M)\cdots p_{i_r}(\tau_M),[M]\rangle\textit{\ \ $\leftarrow$ Pontrjagin number}\]
\item \textbf{Pontrjagin numbers of $\CP^{2n}$:} Write $H^*(\CP^{2n})=\Z[a]/a^{2n+1}$.\\
We know:
$c(\tau)=(1+a)^{2n+1}$ and $c(\overline\tau)=(1-a)^{2n+1}$, so that $c(\tau\oplus\overline\tau)=(1-a^2)^{2n+1}$.
%\[p(\tau)=c_{ev}(\tau\otimes\C)=c_{ev}(\tau\oplus\overline\tau)=\sum c_{2j_1}(\tau)c_{2j_2}(\tau)\]
%Now $c(\tau)=(1+a)^{n+1}\text{ and }c(\overline\tau)=(1-a)^{n+1}\text{, so that }$
\[p(\tau):=c_{\pm ev}(\tau\otimes\C)=c_{\pm ev}(\tau\oplus\overline\tau)=(1+a^2)^{2n+1}.\]
Unravelling this, we obtain:
$p_I[\CP^{2n}]={2n+1\choose i_1}\cdots{2n+1\choose i_r}.$
\item We have that:
\[H^*(B(SO(k));\Z[1/2])=\Z[1/2][p_1,p_2,\ldots,p_{\lfloor k/2\rfloor},e]/
\begin{cases}
e=0;&\ \text{$k$ odd};\\
e^2=p_{k/2};&\ \text{$k$ even},
\end{cases}\]
where $p_i$ has dimension $4i$, and $e$ has dimension $k$. Thus, this group has rank $p(n/4)$ in dimension $n$, for any $n<k$.

\end{itemise}

\pagebreak
\subsection*{Symmetric functions}
\begin{itemise}
\item Choose indeterminates $t_1,\ldots,t_n$. Let $\sigma_1,\ldots,\sigma_n$ be the symmetric polynomials.
\[\prod(1+t_i)=1+\sum\sigma_i.\]
\item For each partition $I=(i_1,\ldots,i_r)\in\calP(k)$ with length $r\leq n$, can form
\[\sum t_1^{i_1}\cdots t_r^{i_r}=:s_I(\sigma_1,\ldots,\sigma_n)\]
so that $s_I$ is a polynomial in $n$ variables of degree $k$ if its $i^\text{th}$ input has degree $i$.
\item Note that $s_{(n)}(\sigma_1,\ldots,\sigma_n)=\sigma_n$. By convention, $s_{()}(\sigma_1,\ldots,\sigma_n)=1$.
\item If $\xi\downarrow X$ is a real vector bundle and $I\in\calP(k)$, write 
\[s_I(p(\xi)):=s_I(p_1(\xi),\ldots,p_r(\xi))\in H^{4k}(X).\]
\item For $M$ a $4n$-manifold and $I\in\calP(n)$, write
\[s_I(p)[M]:=\langle s_I(p(\tau_M)),[M]\rangle.\]
\item $s_{(n)}(p)[\CP^{2n}]=2n+1$. As $p(\tau)=(1+a^2)^{2n+1}$, each of $t_1,\ldots,t_{2n+1}$  represents $a^2$.\\
In particular, $s_{(n)}(\sigma_1,\ldots,\sigma_{2n+1}):=\sum t^n_{i_1}$ should then take the value $(2n+1)(a^2)^n$.
\end{itemise}

\subsection*{Why is $\sigma$ a genus?}
That is is additive is easy. Suppose given $V^n$ and $W^m$.
If $m+n$ is not divisible by $4$, we're clearly done, so write $4k=m+n$. If $m$ and $n$ not both divisible by four, need to show $\sigma(V\times M)=0$. Otherwise, need $\sigma(V)\sigma(W)$. Always use real coefficients. Define
\[B=\mathop{\bigoplus_{s=0}^{2k}}_{s\neq n/2}H^{s}(V)\otimes H^{2k-s}(W).\]
Give a basis $\{v_i^s\}$ of the modules $H^s(V)$ that appear in this sum, so that $\langle v_i^s\cup v_j^{n-s},V\rangle=\delta_{ij}$. Give a similar basis $\{w_i^s\}$ of the modules $H^s(W)$ appearing. Then $B$ has a basis $\{v_i^s\otimes w_j^{2k-s}\}$, and the only nontrivial pairings between these basis elements are:
\[(v_i^s\otimes w_j^{2k-s})\cup(v_i^{n-s}\otimes w_j^{2k-n+s})=(-1)^{n}=(v_i^{n-s}\otimes w_j^{2k-n+s})\cup(v_i^s\otimes w_j^{2k-s}).\]
So we get antidiagonal $2\times2$ matrices down the diagonal, and can thus disregard $B$.

We are left to deal with $A=H^{m/2}(V)\otimes H^{n/2}(W)$. If $m$ and $n$ are odd, we're done. If $m$ and $n$ are multiples of four, the signatures multiply, and we're done (note that everything is in even dimensions). If $m$ and $n$ are congruent to $2$ (mod $4$), we have the tensor product of skew-symmetric forms, which have zeros on the diagonal.

To see that $\sigma$ vanishes on boundaries, we have, by the same argument as Michael gave on Wednesday, if $j:M^{4k}\to V$ is the inclusion of the boundary, an equation 
\[\dim[\im j^*]=1/2\cdot\dim [H^{2k}(M)],\text{ where }j^*:H^{2k}(V)\to H^{2k}(M)\]
However, $\langle j^*(x)\cup j^*(y),[M]\rangle=\langle x\cup y,j_*[M]\rangle=0$ as $M$ bounds, so we have a totally isotropic subspace of half the total dimension. Apparently this only happens when the signature is zero.
\pagebreak
\subsection*{A lower bound on the rank of $\Omega_*$}
%\[M\to BSO(n)\overset{\otimes\C}{\to} BU(n)\from (BU(1))^n\]

%%%%I had the following in when I was planning to shorten everything heavily
%%%%%%%%%%%%%%%%%%%%%%%% BEGIN
%\[\xymatrix@R=.2cm@C=2.2cm{
%BSO(n)\ar[r]^{\otimes\C}&BU(n)&\ar[l](BU(1))^n\\
%H^*(BSO(n))&\ar[l]\Z[x_1,\ldots,x_n]^{\Sigma_n}\ar@{^{(}->}[r]&\Z[x_1,\ldots,x_n]\\
%p_i&(-1)^ic_{2i}\ar@{|->}[r]\ar@{|->}[l]&(-1)^i\sigma_{2i}\\
%p_I&(-1)^{|I|}c_{2i_1}\cdots c_{2i_r}\ar@{|->}[r]\ar@{|->}[l]&\overline p_I:=(-1)^{|I|}\sigma_{2i_1}\cdots \sigma_{2i_r}\\
%s_I&(-1)^{|I|}s_I(c_{2},\ldots,c_{n})\ar@{|->}[r]\ar@{|->}[l]&\overline s_I:=(-1)^{|I|} s_I(\sigma_{2},\sigma_4,\ldots,\sigma_{2\lfloor n/2\rfloor})\\
%}\]
%The $\{\overline p_I:I\in\calP(k)\}$ are linearly independent elements of $H^{4k}(BU(1)^n)$. One can choose another basis $\{\overline s_I:I\in\calP(k)\}$ for this span, to obtain new characteristic classes $s_I\in H^{4k}(BSO(n))$. The $p_I$ and the $s_I$ carry all the same information, but the $s_I$ have certain favourable properties.
%%%%%%%%%%%%%%%%%%%%%%%% END


%
%
%In calculating Pontrjagin numbers of an orientable $4n$-manifold, we classify the tangent bundle via a map $M\to BSO(4n)\to BU(4n)$, and pull back the even Chern classes. Pontrjagin classes $p_I(\widetilde\gamma)$ of the tautological bundle, for $I\in\calP(n)$.
%
%There are certain linear combinations $\{s_I:I\in\calP(k)\}$ of the Pontrjagin classes $\{p_I:I\in\calP(k)\}$, defined combinatorially, with the following favourable properties:
%%\begin{itemise}
%%\item $s_{(k)}[\CP^{2k}]=2k+1$.
%%\item The $s_I$ are obtained from the $p_I$ by multiplying by an invertible $p(k)\times p(k)$ matrix.
%%\item There is a formula (derived from the Cartan formula and formal properties of the polynomials $s_I$), stating that if $J=(j_1,\ldots,j_q)\in\calP(n)$:
%%\[s_I[M^J]=\mathop{\sum_{I_1\cdots I_q=I}}_{I_k\in\calP(j_k)}\prod_{k=1}^q s_{I_k}[M^{4j_k}].\]
%%\end{itemise}
\begin{thm*}[16.8]
Suppose that $M^4,\ldots,M^{4n}$ are oriented manifolds with $s_{(k)}(p)[M^{4k}]\neq0$. Define $M^J=M^{4j_i}\times\cdots\times M^{4j_s}$ for $J\in\calP(n)$. Then the $p(n)\times p(n)$ matrix
\[\left[p_I[M^J]\right]_{I,J}\]
of Pontrjagin numbers is non-singular.
\end{thm*}
\begin{rmk*}
Pontrjagin numbers are oriented cobordism invariant.
\end{rmk*}
\begin{proof}[Proof of Remark]
Suppose that $M^{4k}=\partial V$, with inclusion $j:M\to V$. Then $j^*\tau_V=1\oplus\tau_M$, so that $p(\tau_M)=j^*p(\tau_V)$. But $\langle j^*(p(\tau_V)),[M]\rangle=\langle p(\tau_V),j_*[M]\rangle=0$, as $[M]$ is a boundary.
\end{proof}
\noindent As it is readily calculated that $s_{(k)}[\CP^{2k}]=2k+1$, we obtain:
\begin{cor*}
The manifolds $\CP^2,\CP^4,\CP^6,\ldots$ are algebraically independent in $\Omega_*$. In particular, they generate a polynomial subalgebra whose rank in dimension $n$ is $p(n/4)$.
\end{cor*}
\begin{proof}[proof of corollary]
A homogeneous polynomial relation of dimension $4n$ is a linear relation between the manifolds $M^J$ for $J\in\calP(n)$, using $M^{4i}=\CP^{2i}$. This would give a linear relation between their vectors of Pontrjagin numbers, which is impossible.
\end{proof}
\begin{proof}[Proof of Theorem \lparen16.8\rparen]\hfil
It follows from readily verifiable (but time-costly) properties of the $s_I$ and the Cartan formula that, if $J=(j_1,\ldots,j_q)\in\calP(n)$:
\[s_I(p)[M^J]=\mathop{\sum_{I_1\cdots I_q=I}}_{I_k\in\calP(j_k)}\prod_{k=1}^q s_{I_k}(p)[M^{4j_k}].\]
Thus, if $I$ does not refine $J$, this quantity is zero, and if $I=J$, this quantity is nonzero, being the product of certain of the $s_{(k)}(p)[M^{4k}]$. Thus the matrix is triangular with nonzero diagonal, w.r.t.\ refinement.
\end{proof}
\pagebreak
\subsection*{Some preliminaries before proving the converse}
%\item State and prove the theorem about finite $(k-1)$-connected complexes.
\begin{thm*}[18.3]
Let $X$ be a finite $(k-1)$-connected complex ($k\geq2$). Then the Hurewicz homomorphism is a $\calC$-isomorphism up to dimension $2k-2$.
\end{thm*}
\begin{proof}Let $\calK$ be the collection of $(k-1)$-connected complexes for which the theorem holds.
\begin{itemise}
\item $S^n\in\calK$, by a calculation of Serre.
\item If $X,Y\in\calK$, then $X\times Y$ and $X\vee Y$ are both in $\calK$. To see this, note that $X\wedge Y$ is $(2k-1)$-connected, and use the relative Hurewicz theorem\footnote{Suppose that $(X,A)$ is an $(n-1)$-connected pair of path-connected spaces with $n\geq2$ and $A\neq\emptyset$. Then $H_i(X,A)=0$ for $i<n$ and $h:\pi_n(X,A)\to H_n(X,A)$ is an isomorphism.} to see that the groups $\pi_i(X\times Y,X\vee Y)$ are zero for $i<2k$. Now examine the homotopy-homology ladder (with $i\leq2k-2$), in light of the K\"unneth theorem:
\[\xymatrix@R=.4cm{
0\ar[d]&&0\ar[d]\\
\pi_i(X\vee Y)\ar[d]\ar[rr]&&H_i(X\vee Y)\ar[d]\\
\ar[d]\pi_i(X\times Y)\ar@{=}[rdd]\ar[rr]&&H_i(X\times Y)\ar@{=}[rdd]\ar[d]\\
0&&0\\
&\pi_i(X)\oplus\pi_i(Y)\ar[rr]^{\calC\text{-iso}\oplus\calC\text{-iso}}&&H_i(X)\oplus H_i(Y)
}\]
\item Suppose that $X$ is an arbitrary finite $(k-1)$-connected complex. Choose a homogeneous basis $f_i:S^{r_i}\to X$ for $\bigoplus_{i=k}^{2k-1}\pi_*(X)$, and let $f:\vee S^{r_i}\to X$ be the resulting map.
\[\xymatrix{
\ar[d]_h\pi_*(\vee S^{r_i})\ar[r]^{\pi_*f}&\pi_*(X)\ar[d]_{h'}\\
\pi_*(\vee S^{r_i})\ar[r]^{H_*f}&H_*(X)
}\]
As $h$ is a $\calC$-iso to dimension $2k-2$, we see that $\pi_*f$ is a $\calC$-iso to dimension $2k-2$, and a $\calC$-epi in dimension $2k-1$. By Whitehead's theorem mod $\scrC$,%
\footnote{Whitehead's theorem mod $\scrC$:
Suppose that $f:A\to X$ is a map of simply connected spaces that induces an isomorphism on $\pi_2$. Then TFAE:\\
\phantom{Space}1. $f_*:H_i(X)\to H_i(Y)$ is $\scrC$-iso for $i<n$ and $\scrC$-epi for $i=n$; and\\
\phantom{Space}2. $f_\#:\pi_i(X)\to \pi_i(Y)$ is $\scrC$-iso for $i<n$ and $\scrC$-epi for $i=n$.}
$H_*f$ is a $\calC$-iso to dimension $2k-2$. Thus $h'$ must be a $\calC$-iso  to dimension $2k-2$, and $X\in\calK$.\qedhere
\end{itemise}
\end{proof}

\pagebreak
\subsection*{An upper bound on the rank of $\Omega_n$}
Fix $n$. Seek to bound the rank of $\Omega_n$.
\begin{itemise}
\item Choose $k,p\geq n+2$.
\item Let $\widetilde \gamma$ be the universal oriented $k$-bundle over $\widetilde G=B(SO(k))=\widetilde G_k(\R^\infty)$.
\item Let $\widetilde \gamma'$ be the tautologous $k$-bundle over $\widetilde G'=\widetilde G_k(\R^{k+p})$.
\item $\widetilde G'$ is a subcomplex of $\widetilde G$ containing the $(n+1)$-skeleton.\footnote{For the number of $n$-cells in $G_k(\R^{k+p})$ equals the number of partitions of $n$ into at most $k$ integers, each of which is at most $p$.}
\item We thus have maps:
\[\Omega_n\leftfibration\pi_{n+k}(T(\widetilde\gamma'))\overset{\calC}{\to}H_{n+k}(T(\widetilde\gamma'))\cong H_{n}(\widetilde G')\cong H_n(\widetilde G)\]
\item By the universal coefficient theorem:
\[0\to\Ext(H_{n-1}(\widetilde G),\Z[1/2])\to H^{n}(\widetilde G;\Z[1/2])\to \Hom(H_n(\widetilde G),\Z[1/2])\to0.\]
Now the Ext group is finite as $H_{n-1}(\widetilde G)$ is finitely generated,\footnote{In fact, $\Ext(\Z/(2^r(2s+1))\Z,\Z[1/2])\cong \Z/(2s+1)\Z$, and of course $\Ext(\Z,\Z[1/2])=0$.} so:
%\[p(n/4)=\rank_{\Z[1/2]}\left[H^n(\widetilde G,\Z[1/2])\right]=\rank_{\Z[1/2]}\left[\Hom(H_n(\widetilde G),\Z[1/2])\right]=\rank_\Z H_n(\widetilde G).\]
\begin{alignat*}{2}
\rank_\Z \left[H_n(\widetilde G)\right]&=\rank_{\Z[1/2]}\left[\Hom(H_n(\widetilde G),\Z[1/2])\right]&\quad&\text{(as $\Z[1/2]$ is torsion free)}\\
&=\rank_{\Z[1/2]}\left[H^n(\widetilde G,\Z[1/2])\right]&&\text{(as the $\Ext$ group is finite)}\\
&=p(n/4).
\end{alignat*}
So, for all $n$, $\Omega_n$ is finitely generated, with rank at most $p(n/4)$. We conclude:
\end{itemise}
\begin{thm*}
$\Omega_*\otimes\Q$ is polynomial with generators the even complex projective spaces.
\end{thm*}
\subsection*{Multiplicative sequences}
\begin{itemise}
\item Let $x_1,x_2,\ldots$ be a sequence of indeterminates with $|x_i|=i$.
\item Consider a sequence $K_*$ of polynomials (with $\Q$ coefficients):
\[K_1(x_1), K_2(x_1,x_2),\ldots\]
such that $|K_i|=i$.
\item Suppose that $A^*$ is a unital graded $\Q$-algebra. Write $A^\times$ for the group (under multiplication) of formal sequences $1+a_1+a_2+\cdots$, with leading term $1$. 
\item Then $K_*$ defines a function $K:A^\times\to A^\times$ which is `polynomial':
\[a=1+a_1+a_2+\cdots\overset{K}{\mapsto}1+K_1(a_1)+K_2(a_1,a_2)+\cdots\]
\item $K_*$ is called a multiplicative sequence if $K:A^\times\to A^\times$ is a group automorphism for all unital graded $\Q$-algebras $A^*$.
\end{itemise}
\subsection*{The $K$-genus}
We can obtain a ring homomorphism $K:\Omega_*\to \Q$ as follows.
\begin{itemise}
\item Choose $M^m$ a representative of a cobordism class. 
\item If $m$ is not divisible by 4, set $K(M^m)=0$.
\item If $m=4n$, write $K[M]=\langle K_n(p_1,\ldots,p_n),M^{4n}\rangle$.
\end{itemise}
That this function is additive is obvious. That it vanishes on oriented boundaries follows since Pontrjagin numbers do. That it is multiplicative follows from the Cartan formula which holds (mod order two elements), as $K$ is multiplicative.\footnote{We have $p(M\times M)\equiv p(M)\times p(M')$ (mod elements of order 2), and so $K(p\times p')=K(p)\times K(p')$ (operating in the algebra $H^*(M\times M')$). Then $K[M\times M']=\langle K(p\times p'),\mu\times\mu'\rangle=(-1)^{mm'}\langle K(p),\mu\rangle\langle K(p'),\mu'\rangle=K[M] K[M'].$}
\subsection*{Classification of multiplicative sequences}
\begin{lem*}[Hirzebruch]\hfil
\begin{itemise}
\item Multiplicative sequences are in bijective correspondence with elements of $\Q[[t]]$.
\item To obtain a power series from its multiplicative sequence, one considers the $\Q$-algebra $\Q[t]$, and evaluates $K(1+t)$.
\item If $K$ belongs to $f(t)$, and $a\in A^1$, we always have $K(1+a_1)=f(a_1)$.
\end{itemise}
\end{lem*}
\noindent Note that in our application, $A^1$ will be $H^4(M)$, where $M$ is some $4k$-manifold.
\subsection*{The signature theorem}
\begin{thm*}[Hirzebruch signature theorem]
Let $L_*$ be the multiplicative sequence corresponding to the power series
\[\sqrt t/\tanh\sqrt t=\sum_{k\geq0}\frac{(-1)^{k-1} 2^{2k}B_k}{(2k)!}t^k,\]
so that, for example,
\[L_4=\frac{1}{14175}\left(381\sigma_4-71\sigma_3\sigma_1
-19\sigma_2^2+22\sigma_2\sigma_1^2-3\sigma_1^4\right).\]
Then the signature $\sigma(M^{4k})$ equals the $L$-genus $L[M^{4k}]$.
\end{thm*}
\noindent(For example, for a smooth compact oriented 16-manifold, its signature is the above $\Q$-linear combination of Pontrjagin numbers.)
\begin{proof}\hfil
\begin{itemise}
\item It's enough to check the generators $\CP^{2k}$ of $\Omega\otimes\Q$, whose signature is obviously one (after getting the right orientation convention for $\CP^{2k}$).
\item Let $p=p(\tau)=(1+a^2)^{2k+1}$ be the total Pontrjagin class of $\CP^{2k}$.
\item $L(p)=L(1+a^2)^{2k+1}=(a/\tanh a)^{2k+1}$.
\item Thus we need only check that the coefficient of $a^{2k}$ in $(a/\tanh a)^{2k+1}$ is one --- an exercise in the theory of residues.\qedhere
\end{itemise}
\end{proof}

\pagebreak
\end{SignatureThmTalk}
\begin{KervaireNoDifferentiableStructure}
\KanSemResponse
{``A Manifold which does not admit any Differentiable Structure'' --- M.\ Kervaire}
This paper seems to me to be quite remarkable. It consists of the application of a large collection of techniques, in ways perhaps a little unmotivated, to show the existence of a pathological object. Not only does Kervaire give us ``A Manifold which does not admit any Differentiable Structure'', he throws in a set of steak knives and an exotic 9-sphere.

It seems that there is a disparity between the dense computational character of the text, and the relative simplicity of the construction of the manifold $M_0$. Perhaps this is to be expected --- if there were any manifolds not admitting a differentiable structure, there should be an abundance of them, and it need not be a surprise that one can be found by a relatively simple construction. That being said, the non-existence of a differentiable structure sounds like the type of statement that needs to be proven by intricate calculation --- this would explain the density of the exposition. Moreover, the nature of the search for pathological objects explains some of the seemingly arbitrary features of the example given (for example, the appearance of the number `10') and of the calculations (for example, the attachment of a 6-cell to $\Omega S^6$ along a map of degree two).

Reading this paper has given me an opportunity to acquaint myself with the basics of obstruction theory, which seems like a powerful tool when in the right hands. Anyway, it's first use is, for any $X\in H^5(M)$ (where $M$ is a 4-connected 10-manifold) to define a map $f:M\to\Omega:=\Omega S^6$ such that the generator $e_1$ of $H^5(\Omega)$ maps to $X$. Here, one sees that the first difficult obstruction is in dimension 10 --- its coefficients as in $\pi_9(\Omega)=\pi_{10}(S^6)$, which is zero (as it's in the Stable $4$-stem). Lucky?

Anyway, once such a map $f$ is found, one can form $f^*(u_2)$, where $u_2$ is the generator of $H^{10}(\Omega;\Z_2)$. We denote this $\phi(x)$, where $x$ is the reduction mod 2 of $X$. If 5 were an even number, then $\phi(x)$ would equal the reduction mod 2 of $X^2$ (as then the cohomology of $\Omega$ would be divided polynomial). However, 5 is not even, so that we have some trickier `operation'. There must be some good way to understand this geometrically, although I don't know what that would be. However, perhaps if we knew we were searching for an `operation' of this type, $\Omega$ would be a somewhat feasible choice --- the generators $e_1$ and $e_2$ of $H^5(\Omega)$ and $H^{10}(\Omega)$ are not related by a polynomial identity, but by $\pi^*(e_2)=e_2\otimes1+1\otimes e_2+e_1\otimes e_1$, where $\pi^*$ is the Pontrjagin coproduct.

The next thing that struck me in this paper was the use of (or more the existence of) Milnor's technique for killing homotopy groups of a (framed) manifold by ``spherical modifications''. In fact, the idea is very simple. Given a low-dimensional non-trivial homotopy class $S^r\to M$, that one would like to kill, one deforms and fattens it into a codimension 0 embedding $S^r\times D^{q+1}$ into $M$, and then surgically replaces this copy of $D^{r+1}\times S^{q}$. Of course, then, the `hole' given by the homotopy class is filled, and if $r$ is small enough, the new hole is one of strictly higher dimension. That's a clever idea. Moreover, as we kill the lower homotopy groups, the homology of the manifold becomes simpler more quickly than one might expect, by Poincar\'e duality.

It was a pleasure to see the results of the Thom paper appearing in this context. I need practice before feeling completely at easy with them. On thing which I have found a little weird is the way that the Thom collapse works for framed manifolds. It has a rather different flavour than with a non-trivial structure group, as the classifying map can be taken to be to a point. I don't know why this is bugging me, but everything will eventually be alright.

Anyway, there were a number of prerequisites for this paper which I lacked. In particular, I'd like to learn about the handlebody decompositions that Michael discussed in his practise talk, and how they are dualisable. This, along with a better understanding of Morse theory, would give me more confidence with some of the details, for example, in the calculation showing that the Kervaire invariant of $M_0$ is in fact one.

\pagebreak
\end{KervaireNoDifferentiableStructure}
\begin{JandrKervaire}
\KanSemResponse
{Michael Andrews' Talk on Kervaire's paper}
The plan:
\begin{enumerate}
\item Construct $\Phi$ a (homotopy type) invariant of closed 4-connected 10-manifolds.\footnote{Simply connected implies oriented!}
\item If $M$ is smooth, then $\Phi(M)=0$.
\item Construct $M_0$ such that $\Phi(M_0)=1$.
\end{enumerate}
\begin{question}
Is this the standard Kervaire invariant?
\end{question}
The normal Kervaire invariant is defined for $(4k-2)$-dim framed manifolds. Define a quadratic refinement $q:H^{2k+1}(M;\Z)\to\Z_2$ of the cup product pairing (i.e.\ a function satisfying $q(x+y)=q(x)+q(y)+(x\smile y)[M]$). Define $\Phi=\ArfInvariant(q)$.\footnote{Arf invariant returns which value is taken on more often by $q$.}

To define $q$, take $M^n\subset \R^{n+N}$ giving $S^{n+N}\to\text{Th}(\nu)=\Sigma^NM_+$
\[\left(q:M_+\to K\right)\mapsto(S^{n+M}\to\Sigma^NM_+\to\Sigma^NK)\]
this gives an element of $\pi_n^s(K)=\Z_2$. ($K=K(\Z_2,2k+1)$)

\subsection*{Defining $\Phi$}
Define $\phi_0:H^5(M)\to\Z_2$ by letting $e_1,e_2$ be the generators of $H^5(\Omega S^6),H^{10}(\Omega S^6)$, and $u_2$ be the generator $H^{10}(\Omega S^6;\Z_2)$. For $X\in H^5(M)$:
\begin{enumerate}
\item Find $f_X:M\to\Omega$ such that $f_X^*(e_1)=X$.
\item Define $\phi_0(M):=f^*_X(u_2)[M]$.
\end{enumerate}
Can show that $\phi_0$ descends (nothing tricky here, by Poincar\'e duality...) to
\[\phi:H^5(M;\Z_2)\to\Z_2.\]
Let $\Phi$ be $\ArfInvariant(\phi)$, i.e.\ $\sum\phi(x_i)\phi(y_i)$, where the $x_i,y_i$ are a symplectic basis.
\subsection*{Constructing $M_0$}
Start with $S^5$, and let $\tau$ be its tangent bundle.  Pull back in two ways, with the factors $D^5$ playing opposite roles:
\[\xymatrix{
D^5\times D^5\ar[d]_{\pi_1}\ar[r]&D(\tau)_1\ar[d]\\
D^5\ar[r]^{\text{top hemi}}&S^5
}\qquad
\xymatrix{
D^5\times D^5\ar[d]_{\pi_2}\ar[r]&D(\tau)_2\ar[d]\\
D^5\ar[r]^{\text{top hemi}}&S^5
}\]
Push out:
\[\xymatrix{
\ar[d]D^5\times D^5\ar[r]&D(\tau)_1\ar[d]\\
D(\tau)_2\ar[r]&W
}\]
That is, $W$ is obtained by plumbing two copies of the disc bundle.
Need to do some smoothing of the boundary $\partial W$, (unless you're Milnor, who's really smart somehow).

By Morse theory, $\partial W=S^9$ (Milnor writes down a Morse function with only two critical points, and applies a theorem of Milnor).\footnote{To construct it, you note that the tangent bundle of $S^5$ has a non-vanishing section, so we can reduce the structure group to $SO(4)$. This comes in handy somehow.} 
Let $M_0$ be the cofibre of $S^9\to W$. Our proof will then also show that $\partial W$ must be exotic.

\subsection*{Showing that $\Phi(M_0)=1$:}
Now to $\Phi(M_0)=1$. The 0-sections of the two disk bundles meet transversely at a point. We have $S^5\vee S^5\to W$, which is a homotopy equivalence. So adding a 10-cell, $M_0$ has a CW-structure with one 0-cell, two 5-cells and one 10-cell (which shows that it's 4-connected). Haynes: ``so it looks a lot like $S^5\times S^5$, a whole lot! I think then you're plumbing by \emph{not} swapping the roles of the disks.''

The copies of $S^5$ give generators of $H_5(M_0)=\Z\oplus\Z$. Let $X$ and $Y$ be the Poincar\'e dual generators. Let $x$ and $y$ be their reductions mod $2$, and let $K=\text{Th}(\tau)$. Now $x\cdot y=1$ (using the intersection pairing), so that $\{x,y\}$ is a symplectic basis. We need then to show that $\phi(x)\phi(y)=1$.

Let's take $f_X:M_0\to K$ which collapses everything outside $D(\tau)_X$ (the copy of $D(\tau)$ corresponding to $X$) to the basepoint and takes $D(\tau)_X\to K=D(\tau)/S(\tau)$. This $D(\tau)_X$ is a tubular neighbourhood of the $S^5$ corresponding to $X$, so that $f_X^*(t)=X$, where $t$ is the Thom class. By a local degree argument, $f_X^*$ is an iso on $H^{10}$. We do the same to construct $f_Y$.

If $K$ were equal to $\Omega S^6$, we'd be done. But it's not! Now $K$ has the form $S^5\cup_f e^{10}$. Here, $f=[\imath_5,\imath_5]$, the Whitehead square. Now $\Omega S^6$ has a cell structure (Michael likes to think about this using Morse theory) with cells of dimension $5k$ for $k\in\N$, so that the 10-skeleton is $S^5\cup e^{10}$ too, and it turns out that you also get the Whitehead square. Thus $K\subset \Omega S^6$ is the 14-skeleton, and you're a winner. 

By the way, we're using $\Omega S^6$ because it's an H-space, which will be useful when one wants to prove the quadratic refinement formula. We'll form $f_{X+Y}$ as the composite:
\[\xymatrix{
M\ar[r]^{M\quad}&M\times M\ar[r]^{\ f_X\times f_Y}&\Omega\times\Omega\ar[r]&\Omega.
}\]
\subsection*{Is $\Phi$ well defined?}
We've gotta show that $f_X$ does exist. We do obstruction theory:

If $(X,A)$ is a relative CW-complex, for each $(n+1)$-cell $\sigma$, let $g_\sigma:S^n\to X$ be the attaching map. Given $f:X^n\to Y$, one would like to extend this to the $(n+1)$-skeleton. To make the following make sense, one needs $\pi_1(Y)$'s action on $\pi_n(Y)$ to be trivial, in order to disregard basepoints. Let's also have $Y$ connected.

All we really want is for the compositions of $f$ with the $g_\sigma$ to be null. We can define:
\[c(f)\in C^{n+1}_{CW}(X,A;\pi_n(Y))\text{\ given by }c(f)(\sigma)=[f\circ g_\sigma]\]
This last thing is supposed to be an element of $\pi_n(Y)$, but the basepoints are mucked up. Who cares --- $Y$ is connected and $\pi_1$ acts trivially.
\begin{thm*}
\begin{itemise}
\item $c(f)=0\iff f$ can be extended to $X^{n+1}$.
\item $c(f)$ is a cocycle, and $[c(f)]=0$ \Iff $f|_{X^{n-1}}$ can be extended to $X^{n+1}$.
\end{itemise}
\end{thm*}
\noindent There's a bit left to do for well-defined-ness. We'll talk about why it descends to mod 2 cohomology. How come? Well, the relation
\[\phi_0(X+Y)=\phi_0(X)+\phi_0(Y)+x\cdot y\]
is proven by considering a composite
\[\xymatrix{
M\ar[r]^{M\quad}&M\times M\ar[r]^{\ f_X\times f_Y}&\Omega\times\Omega\ar[r]&\Omega
}\]
and exploiting our knowledge of the Pontrjagin coproduct on the cohomology of $\Omega$.
\subsection*{$M$ smooth implies $\Phi(M)=0$}
Just use Milnor's paper repeatedly. Milnor has a tool for killing homotopy groups of smooth manifolds. Suppose we're given an $N$-manifold $M^N$ and $[f]\in\pi_n(M)$. Have $S^n\to M$, change it to get an embedding $S^n\times D^{N-n}\cofibration M$ (by the tubular neighbourhood theorem, to do this is to embed $S^n$ in $M$ with trivial normal bundle). Then apply surgery, to get
\[(M-(S^n\times D^{N-n}))\cup (D^{n+1}\times S^{N-n-1})\]
Two problems, you create a new hole, so can only do half the dimensions, i.e.\ you want $n<N/2$ or so. Also, you need a $\pi$-manifold, in order to choose the embedding (still only get certain dimensions --- if you can embed $M$ into a Euclidean space, with trivial normal bundle, then you can embed $S$ inside with trivial normal bundle \rednote{how?}).

Using differential topology, we can make $S^n\to M$ an embedding. Moreover, we can embed $M\subset \R^N$ with trivial normal bundle, by as $M$ is a $\pi$-manifold. Then the normal bundle of $S^n$ in $\R^N$ is trivial, as the embedding of $S^n$ in $\R^N$ is isotopic to the standard embedding ($N$ being large). Thus the normal bundle of $S^n$ in $M$ is stably trivial. However, by the proof of the theorem soon to follow on $\pi$-manifolds, as the dimension of $S^n$ is less than the dimension of its normal bundle in $M$, its normal bundle must be trivial.


 The killing procedure preserves $\pi$-manifold-ness. Milnor does all this very well.
\begin{defn*}
A $\pi$-manifold is one whose tangent bundle $\tau$ has $\tau\oplus\epsilon$ trivial.
\end{defn*}
\noindent This obviously implies that $\tau$ is stably trivial. It also shows that $\nu_M$ is stably trivial. In fact all these three are equivalent.
\begin{thm*}
If $M$ is a smooth $n$-manifold with tangent bundle $\tau$, TFAE:
\begin{itemise}\squishlist
\itm[(a)] $M$ is a $\pi$-manifold.
\itm[(b)] $\tau$ is stably trivial.
\itm[(c)] $\tau\oplus\epsilon$ is trivial.
\itm[(d)] The stable normal bundle of $M$ is trivial.
\end{itemise}
Moreover, if $M$ is connected with non-empty connected boundary, these are all equivalent to:
\begin{itemise}\squishlist
\itm[(e)] $\tau$ is trivial.
\end{itemise}
\end{thm*}
\begin{proof}
Consider the following diagram, in which the `$\Gamma$' shapes are all fibrations. Let $M\to BO(n)$ classify the tangent bundle.
\[\xymatrix{
M\ar[r]&BO(n)\ar[d]&S^n\ar[l]\\
&\ar@{..>}[d]BO(n+1)&\ar[l] S^{n+1}\\
&\ar[d]BO(N-1)&\ar[l] S^{N-1}\\
&BO(N)
}\]
We only need to prove that (b) implies (c) (and that (b) implies (e) under the extra hypothesis). Assuming (b), we see that the composite $M\to BO(N)$ is trivial. Thus, it factors through $S^{N-1}\to BO(N)$. However, by connectivity, the map to $S^{N-1}$ must be null, so that in fact, $M\to BO(N-1)$ is null. This argument works all the way up the ladder, to show that $M\to BO(n+1)$ is null, giving (c). If $M$ has a nonempty connected boundary, it is homotopy equivalent to a CW-complex of lower dimension (using a handlebody decomposition argument), so that we can take this argument one step further. \bluenote{Identify the spectral sequence here...}
\end{proof}
\subsection*{There are three big things left to do:}
\begin{enumerate}
\item A smooth 4-connected 10-manifold is $\pi$ (obstruction theory, involving the injectivity of the $J$ homomorphism on $\pi_9$). This is the only place we use smoothness, in order to apply Milnor's procedure.
\item Given $(M^{10},f_n)$, get element $\alpha(M,f_n)\in\pi_{n+10}(S^n)$, and if this element is zero, the the Kervaire invariant $\Phi(M)$ is zero.
\begin{proof}
The first thing you do is note that Thom shows that there's a $(V^{11},F)$ framed with $\partial V=M$. Can assume $V$ is connected (as $M$ is). Then the tangent bundle of $V$ is trivial, in particular, it's a $\pi$-manifold.

By applying Milnor's killing process, can assume $V$ is 4-connected. Then Poincar\'e duality shows that $V$ only has nonzero cohomology in dimensions 0, 5, 6, 11. We have the following beautiful diagram from Poincar\'e duality, with $\Z_2$ coefficients:
\[\xymatrix{
H^5(V)\ar[d]^\sim\ar[r]^{i^*}& H^5(M)\ar[d]^\sim\ar[r]^\delta&H^6(V,M)\ar[d]^\sim\\
H_6(V,M)\ar[r]^\partial&H_5(M)\ar[r]^{i_*}&H_5(V)
}\]
(There's a sign in this diagram with $\Z$ coefficients, depending on your sign conventions!)
Now we have the following equalities:
\begin{alignat*}{2}
\dim\ker(\delta)&=\dim\im(i^*)&\quad&\text{(exactness)}\\
&=\dim\im (i_*)&&\text{(as $i_*$ and $i^*$ are dual)}\\
&=\dim\im(\delta)&&\text{(by commutativity)}\\
&=\dim H^5(M)-\dim\ker(\delta)&&\text{(rank-nullity)}
\end{alignat*}
Thus, $\ker(\delta)$ is a subspace of $H^5(M)$, of half the dimension. Moreover, suppose that $x,y\in\ker(\delta)=\im(i^*)$. Then $x\cup y\in\im(i^*)$, and as $H^{10}(V)=0$ (by Poincar\'e duality), this shows that $x\cup y=0$. Thus, $\ker(\delta)$ is a ``Lagrangian subspace'', and so can be extended to a symplectic basis. In particular, it'll be enough to show that $\phi(x)=0$ for all $x\in\ker(\delta)$.

Haynes: ``This is an important general idea: that if $M$ is the boundary of $V$, the the kernel of $\delta$ gives a Lagrangian subspace.''

So, choose $x\in\ker(\delta)$, and lift it to an integral class $X\in H^5(M)$. Choose $f_X:M\to \Omega$ such that $f_X^*(e_1)=X$. We need only show that $f_X^*(u_2)=0$.

Now if we could form a diagram
\[\xymatrix{
M\ar[r]\ar[d]&\Omega\ar[d]\\
V\ar[r]&\Omega^*
}\]
such that $H^{10}(\Omega;\Z_2)\to H^{10}(\Omega^*;\Z_2)$ is an isomorphism, we'd be happy, since $H^{10}(V;\Z_2)=0$. (Note that it's too hard to do this when $\Omega^*=\Omega$.) If we had a choice of $\Omega^*$, we still need to fill in the map $V\to \Omega^*$, and we approach this problem with obstruction theory. We have a map $M\to\Omega^*$ which we need to extend to all of $V$. Note that $(V,M)$ can be triangulated into a relative CW-complex. Now $\Omega$ has a cell structure with one cell in each dimension a multiple of $5$. We'll form $\Omega^*$ by attaching a 6-cell a map of degree 2 to the 5-cell. The map $\Omega\to\Omega^*$ is obviously an isomorphism on $H^{10}$. Somehow, he manages to pin down all the obstructions up to the one in dimension 10, and then brings in a lemma to show that the map can be extended to the (unique) 11-cell of $V$ (This 11-cell can be unique by the dualisability of handlebody decompositions --- Morse theory).
\end{proof}
\item The lemma taking two pages to prove, with $\beta\circ\alpha$ in the statement.
\end{enumerate}

\pagebreak
\end{JandrKervaire}
\begin{HopfInvOne}
\KanSemResponse
{``$K$-theory and the Hopf Invariant'' --- Adams \& Atiyah}
I had seen this proof of the Hopf Invariant one theorem before --- reading it again gave me an opportunity to collect my thoughts. In particular, now that I have a little more experience in dealing with the Atiyah-Hirzebruch spectral sequences, I am more confident with some of the steps in the proof.

A first point I'll discuss is the following statement:
\[\text{If }u\in K_{2n}(X),\text{ then }\psi^k(u)\cong k^nu\pmod{K_{2n+1}(X)},\text{ where $K_q(x):=\ker\{K(X)\to K(X^{q-1})\}$.}\]
Although once a mysterious fact, this now seems obvious, in light of the construction of the Atiyah-Hirzebruch spectral sequence:
\[\xymatrix@!0@C=1.6cm@R=1.28cm{%@C=.5cm{
\makebox[0cm][r]{$\cdots$\ }
K^*(X^{2n-1})\ar[rr];[]&&
K^*(X^{2n})\ar[rr];[]\ar[ld];[]&&
\cdots\ar[rr];[]&&
K^*(X)\\
&K^*(X^{2n}/X^{2n-1})\ar@{~>}[ul];[]&&\\
0\ar@{|->}[rr];[]&&\widetilde u\ar@{|->}[rrrr];[]^c\ar@{|->}[ld];[]&&&&u\\
&x
}\]
Now the long composite $c$ induces an isomorphism (on $K^0$) between $K_{2n}(X)/K_{2n+1}(X)$ and the subquotient $E_\infty^{2n,-2n}$ of $K^*(X^{2n}/X^{2n-1})$. The permanent cycle $x$, an element of the $K$-theory of a wedge of copies of $S^{2n}$, has $\phi^{k}(x)=k^nx$. The point which I am trying to emphasise is that this information passes to information about $u$ \emph{because the isomorphism $K_{2n}(X)/K_{2n+1}(X)\simeq E_\infty^{2n,-2n}$ is induced by genuine maps of topological spaces}. Thus (unless I am mistaken) we see that the Adams operations are compatible with the Atiyah-Hirzebruch spectral sequence.

Now we need only examine the $E_2$ page of the spectral sequence, (when $n=2m$ is even), and note that it must collapse. From here the only interesting part of the proof is the \emph{fact} that it exists.

Adams and Atiyah go on to note the existence of the Chern character, and its interaction with the Adams operations. I haven't seen this material before, so I though I'd work out the exercise on Chern character in Milnor/Stasheff, and see how it relates here. Doing so reinforced my understanding of the importance of the splitting principle. One gives a formula for the Chern character in terms of the Chern classes of a bundle, and this formula is additive, by properties of the polynomials $s_I(\sigma_1,\ldots,\sigma_n)$. One also notes, using Girard's formula for $s_{(n)}(\sigma_1,\ldots,\sigma_n)$, that $\ch(\scrL)=\exp(c_1(\scrL))$ for a line bundle $\scrL$. Now, one would like to see that the character is characterised by these properties, and is multiplicative, but both of these are immediate from the splitting principle --- fantastic.

Finally, one can note formula (2.1):
\[\text{writing }\ch x=\sum a_{2m}\text{ for }a_{2m}\in H^{2m}(X;\Q),\text{ we have }\ch (\psi^kx)=\sum k^ma_{2m}.\]
This also follows from the splitting principle, as one notes that $c_1(\psi^k(\scrL))=c_1(\scrL^{\otimes k})=kc_1(\scrL)$.

Anyway, at this point, the authors use the Chern character ``to identify $K(X)\otimes \Q$ with $\oplus H^{2m}(X;\Q)$''. This seemed very strange to me. \textit{Here was a long discussion of the various arguments for this to be the case. It has now been removed --- we discussed this in detail the other day.} 

%Wouldn't there by differentials in the Atiyah-Hirzebruch spectral sequence which make this impossible? Dustin explained that he could explain this to me over a beer when I better understand formal group laws (and the concept of Landweber exactness). He later remarked that it should be sufficient to check the isomorphism on spheres, if one can write the Chern character as a morphism of cohomology theories. This seems impossible to me --- we are trying to prove that
%\[\widetilde K(\DASH)\otimes\Q\to\bigoplus_{n\geq0} \widetilde H^{2n}(\DASH;\Q),\]
%is an isomorphism, but the extension of the left hand side to a cohomology theory is periodic, while the extension of the right hand side is not.
%
%I had wanted to say that $\widetilde K(\DASH)\otimes\Q$ and $\bigoplus_{n\geq0} \widetilde H^{2n}(\DASH;\Q)$, are both representable functors (by pointed maps into a CW-complex), and there is a morphism between their representing objects, given by the Chern character. The representing objects have trivial $\pi_1$, as both functors take value zero on $S^1$. That morphism induces an isomorphism on homotopy groups, as we can see that the Chern character is an isomorphism on spheres. Anyway, I think here we can't conclude that we have an isomorphism (using Whitehead's theorem), as the representing objects aren't connected!

%%Would it be correct to say the following? $\widetilde K(\DASH)$ is representable by a CW-complex (by Brown representability), and if we have a pullback diagram and a surjection in $\AbGp$ as follows: 
%%\[\xymatrix{
%%\widetilde K(U\cup V)\ar@{->>}[r]&\text{pullback}\ar[d]\ar[r]&\widetilde K(U)\ar[d]\\
%%&\widetilde K(V)\ar[r]&\widetilde K(U\cap V)
%%}\]
%%we have the same after tensoring with $\Q$, so that the Mayer-Vietoris axiom holds for $\widetilde K(\DASH)\otimes \Q$. The wedge axiom obviously holds for $\widetilde K(\DASH)\otimes\Q$. Thus $\widetilde K(\DASH)\otimes\Q$ is representable (by homotopy classes of pointed maps to some CW-complex $C$). The  Chern character gives a natural transformation:
%%\[\widetilde K(\DASH)\otimes\Q\to\bigoplus_{n\geq0} \widetilde H^{2n}(\DASH;\Q),\]
%%the latter being representable by $D=\prod K(\Q,2n)$. We only need to check that this is an isomorphism on spheres. Here it gets weird, maybe... I want to say that there's an iso on spheres so we're done by Whitehead, but the representing spaces aren't connected!  
%
%In the meantime, a bunch of other arguments have come up --- I mentioned to Michael that the differentials in the Atiyah-Hirzebruch spectral sequence must vanish --- he claims that the AHSS measures the failure of the representing spectrum to be a product of Eilenberg-MacLane spectra. Moreover, Marcus claims that any $\Q$-local spectrum is a product of Eilenberg-MacLane spectra. It would seem that these two claims would imply that the AHSS is rationally degenerate, and so make the fact that the Chern character is a rational isomorphism seem more likely.
%\emph{Let the speculation cease.}


%\begin{itemise}
%\item Any rational spectrum is a product of Eilenberg-MacLane spectra.
%\item Well, one could see this for the one we have by noting that the rational cohomology of $BU(n)$ is pretty simple, urm, it's like the same as that of the rational Eilenberg MacLane space! Whoa, really?.... then what?
%\item Point is that the differentials in the AHSS measure the failure of a spectrum to be a product of Eilenberg-MacLane spectra. Thus, the AHSS is degenerate rationally, giving the result. Does the Chern character give a map of spectral sequences? I suppose that the answer is `yes'. How would one format this?
%\end{itemise}

Anyway, I have a number of rather pressing responsibilities at the moment, and I has been some time since I picked up the paper in question (and wrote whatever lies above), so I won't go deeper into it at this juncture. I will however, make brief comments on a little material from Markus' first talk.

What really had never crossed my mind (and I think is often swept under the rug) is the subtlety involved when defining operations on $K$-theory using operations on vector bundles. Markus emphasised this well in his talk. As the representing space for $K$-theory fails to be compact, $K$-theory classes thereupon in fact cannot be represented by vector bundles, so chaos threatens to ensue. As there is no $\varprojlim^1$ term at the key moment, we are in the clear --- the problem that I never knew existed is in fact no problem at all.

Finally, I was glad to hear that there is a splitting principle for $K$-theory. In fact, the same projective bundle construction works in order to split off line bundles one at a time.  Let $V\downarrow X$ be a $\C^n$-bundle, then pull $V$ back along the map $\pi:\PP(V)\to X$, which is itself a bundle of complex projective spaces $\CP^{n-1}$. This map splits off a line bundle.

The injectivity of $K^*(\pi)$ seems to want to be proven using the same technique as the proof of the standard splitting principle. That is, I'd like to say the $K$-theory of $\CP^{n-1}$ is $\Z[L]/(L-1)^{n}$, and these $K$-theory classes are all pulled back from the classes of the tautological bundle on $\PP(V)$ (note also that the $K$-theory of the fibre is invariant under the action of $\pi_1(X)$). So the ingredients are there, and we can apply the generalised Serre spectral sequence to see that the $K(\PP(V))$ is in fact a free module over $K(X)$.

The only problem here for me is that I haven't seen the derivation of the generalised Serre spectral sequence:
\[H^s(X;K^t(\CP^{n-1}))\implies K^{s+t}(\PP(V)).\]
I shouldn't complain here, as I don't carry the derivation of the multiplicative structure of the SSS in my head anyway.


\pagebreak
\end{HopfInvOne}
\begin{MarkusKtheoryPractice}
\KanSemResponse
{Markus's Practice talk on $K$-theory}
\subsection*{$K$-theory}
Let $X$ be a compact topological space. Define $\Vect(X)$ to be the set of isomorphism classes of finite dimensional complex vector bundles on $X$. $\Vect(X)$ is a commutative monoid under Whitney sum. Define $K(X):=\GrthGrp(\Vect(X))$.

For $X$ pointed, we define $\widetilde K(X):=\ker\{K(X)\to K(\ast)\}$. This map is actually split by $\ast\to X\to \ast$, so that $K(X)\simeq K(\ast)\oplus\widetilde K(X)$. (You also thus have $\widetilde K(X)=\coker\{K(\ast)\to K(X)\}$).

Assume $X$ connected. Then (the colimit is via stabilisation) 
\[\widetilde K(X)\simeq \colim_n\Vect_n(X)\cong \colim[X,BU(n)]\cong[X,BU]\cong[X,BU]_*\]
So for compact connected $X$, it's representable. For $X$ not necessarily connected, but still compact, we get that $\widetilde K(X)\cong[X,\Z\times BU]_*$.

We now \emph{define} $\widetilde K(X):=[X,\Z\times BU]_*$ and $\widetilde K(X):=[X,\Z\times BU]$ for arbitrary $X$. For general $X$, the vector bundle construction is \emph{not} the same as this. [example: the tautologous bundle on $\CP^\infty$ doesn't have an inverse?] [example: the thing represented by the identity map of $BU$.]
\subsection*{Multiplication}
If $X$, $Y$ are compact, then the tensor product induces maps $K(X)\otimes K(Y)\to K(X\times Y)$, and $\widetilde K(X)\otimes \widetilde K(Y)\to\widetilde K(X\wedge Y)$. If $X=Y$ we can pull this back to an internal product. [currently unsure as to how to extend this to arbitrary $X$ and $Y$.]

Now we have a representable functor, and a representing object gives half a cohomology theory. We just need to make it into an infinite loop space. Something far better happens:
\[\Omega U\simeq\Z\times BU,\text{ so that }\Omega^2(\Z\times BU)\cong \Z\times BU.\]
This implies that $\widetilde K(X)\cong \widetilde K(S^2\wedge X)$, where the isomorphism is given by smashing with the Bott class $[H]-1\in\widetilde K(S^2)$, where $H$ is the taut bundle on $\CP^1$. Moreover, $K(S^2)=\Z[H]/(H-1)^2$. It follows that $\widetilde K$ is the $0^\text{th}$ part of a 2-periodic reduced cohomology theory: $\widetilde K^{-1}(X):=[X,U]_*$.

Problem: $\Z\times BU$ is not compact. This makes it harder to construct operations on $K$-theory --- we don't actually have vector bundles at the representing object. We use the fact that $BU=\colim \text{Gr}_n(\C^{2n})=G_{n,2n}$. We'll show that the $K$-theory of $BU$ is the inverse limit of that of the $G_{n,2n}$. There's a Milnor exact sequence:
\[\xymatrix{
0\ar[r]&\lim^1 K^{-1}(G_{n,2n})\ar[r]&K^0(BU)\ar[r]&\lim K^0(G_{n,2n})\ar[r]&0}\]
We do this using the AHSS, $H^s(G_{n,2n},K^t(\ast))\implies K^{s+t}(G_{n,2n})$, as this collapses, due to a chequerboard pattern, and $G_{n,2n}$ has no odd $k$-theory. [If $t$ is odd, have coefficients zero (as $U$ is simply connected). If $s$ is odd, get zero, as $G_{n,2n}$ has only cells in even dimensions.] Thus we get $K^0(BU)\cong \varinjlim K^0(G_{n,2n})$.

Thus given a transformation which works on compact spaces, we can extend to a transformation on all spaces.
\subsection*{$K$-theory operations}
For $X$ compact, $V$ a vector bundle over $X$, let $\lambda^n(V)$ be the $n^\text{th}$ exterior power of $V$. This is \emph{not} additive. Let $\lambda(V)$ be the formal sum $\sum_{n=0}^\infty \lambda^n(V) t^n\in K(X)[[t]]$. This lands in the multiplicative group of elements with leading coefficient 1. This $\lambda$ is a monoid homomorphism from $\Vect(X)$ to this multiplicative group, which thus extends to $K(X)$. This follows from the equation %$\lambda^n(V\oplus W)=\oplus \lambda ^{

Now let $\lambda^i$ be the $i^\text{th}$ component of $\lambda(X)$, $\lambda^i:K(X)\to K(X)$. These don't have great properties.
\subsection*{Adams operations}
We define, for $x\in K(Y)$, $\psi(x)=\rank(x)-t\frac{d}{dt}\log \lambda_{-t}(x)\in K(X)[[t]]$, expanding the log via $\log(t)=\sum_{n=0}^\infty \frac{(-1)^{n+1}}{n}(t-1)^n$. Let $\psi^i(x)$ be the $i^\text{th}$ component.
\begin{itemise}
\item The $\psi^i$ are additive.
\item $\psi^i(\scrL)=\scrL^{\otimes i}$ for line bundles $\scrL$.
\item The $\psi^i$ are multiplicative (thus ring homomorphism).
\item $\psi^i\otimes\psi^j=\psi^{ij}$.
\item $\psi^i(u)=i^nu$ for all $u\in\widetilde K(S^{2n})$.
\end{itemise}

\pagebreak
\end{MarkusKtheoryPractice}
\begin{MarkusKtheory}
\KanSemResponse
{Markus's talk on $K$-theory}
\subsection*{$K$-theory}
Let $X$ be a compact topological space. Define $\Vect(X)$ to be the set of isomorphism classes of finite dimensional complex vector bundles on $X$. $\Vect(X)$ is a commutative monoid under Whitney sum.

Define $K(X):=\GrthGrp(\Vect(X))$.
For $X$ pointed, we define $\widetilde K(X):=\ker\{K(X)\to K(\ast)\}$. This map is actually split by $\ast\to X\to \ast$, so that $K(X)\simeq K(\ast)\oplus\widetilde K(X)$. (You also thus have $\widetilde K(X)=\coker\{K(\ast)\to K(X)\}$).

Assume $X$ connected. Then (the colimit is via stabilisation by adding trivial bundles) 
\[\widetilde K(X)\simeq \colim_n\Vect_n(X)\cong \colim[X,BU(n)]\cong[X,BU]\cong[X,BU]_*\]
Here the second last isomorphism holds as $X$ is compact, and the last isomorphism holds since $BU$ is simply connected.
So for compact connected $X$, it's representable. For $X$ not necessarily connected, but still compact, we get that $\widetilde K(X)\cong[X,\Z\times BU]_*$ and $K(X)\cong[X,\Z\times BU]$.

We now \emph{define} $\widetilde K(X):=[X,\Z\times BU]_*$ and $\widetilde K(X):=[X,\Z\times BU]$ for arbitrary $X$. Note that for general $X$, the vector bundle construction is \emph{not} the same as this. 
\begin{exmp*} The tautologous bundle on $\CP^\infty$ doesn't have an inverse.
\end{exmp*} 
\noindent[Example: the thing represented by the identity map of $BU$.]
\subsection*{Multiplication}
If $X$, $Y$ are compact, then the tensor product induces pairings $K(X)\otimes K(Y)\to K(X\times Y)$, and $\widetilde K(X)\otimes \widetilde K(Y)\to\widetilde K(X\wedge Y)$. If $X=Y$ we can pull this back to an internal product.

We'll get a multiplication on $\Z\times BU$ (more twisted than you might think). Suppose $X$ is compact and connected and have $X\to BU$, can think of this as a map $X\to n\times BU$ for any $n$. This factors via $X\to BU(n)$, giving a bundle $V$ of dimension $n$... If $w$ of dim $m$, ummm should pull back to $0$-dim stuff. 
\[([V]-[\epsilon^n])([W]-[\epsilon^m])=[V\otimes W\oplus\epsilon^{mn}]-[W^n\oplus V^m].\]
So get $m\times BU\times n\times BU\to mn\times BU$ which doesn't agree with $BU\times BU\to BU$ under the obvious identification.

 [currently unsure as to how to extend this to arbitrary $X$ and $Y$.]

Now we have a representable functor, and a representing object gives half a cohomology theory. We just need to make it into an infinite loop space. In fact, it already is:
\[\Omega U\simeq\Z\times BU,\text{ so that }\Omega^2(\Z\times BU)\cong \Z\times BU.\]
This implies that $\widetilde K(X)\cong \widetilde K(S^2\wedge X)$, where the isomorphism is given by smashing with the Bott class $[H]-1\in\widetilde K(S^2)$, where $H$ is the taut bundle on $\CP^1$. Moreover\footnote{To see why $(H-1)^2=0$, one could note that it is a statement about 2-dimensional vector bundles on $S^2$, and there are only trivial ones. Alternatively, note that this is a reduced class, and any product of two reduced classes on a suspension is zero.}, $K(S^2)=\Z[H]/(H-1)^2$. It follows that $\widetilde K$ is the $0^\text{th}$ part of a 2-periodic reduced cohomology theory: $\widetilde K^{-1}(X):=[X,U]_*$.

\subsection*{The difficulty with forming operations on $K$-theory}
\textbf{Problem:} The representing object $\Z\times BU$ is not compact. This makes it harder to construct operations on $K$-theory --- we don't actually have vector bundles at the representing object.

In order to circumvent this, we'll show\footnote{One can use the AHSS directly to show that $K^{-1}(\Z\times BU)=0$.} that 
\begin{thm*}
$K(Z\times BU)=\varprojlim K(G_n)$, where $G_n:=\{-n,\ldots,n\}\times \Gr_{n}(\C^{2n})$.
\end{thm*}
\begin{proof}
We use the fact that $\Z\times BU=\colim G_n$. We'll show that the $K$-theory of $\Z\times BU$ is the inverse limit of that of the $G_n$. There's a Milnor exact sequence:
\[\xymatrix{
0\ar[r]&\varprojlim^1 K^{-1}(G_{n})\ar[r]&K^0(\Z\times BU)\ar[r]&\varprojlim K^0(G_{n})\ar[r]&0}\]
We do this using the AHSS, $H^s(G_{n},K^t(\ast))\implies K^{s+t}(G_{n})$, as this collapses, due to a checker-board pattern\footnote{%
Note that $K^{-1}(*)=[*,U]=0$, and $G_n$ only has cells in even dimensions.}, and $G_{n}$ has no odd $k$-theory. Thus the $\varprojlim^1$ vanishes.
\end{proof}
\noindent Thus, given a transformation defined on the $K$-theory of compact spaces, we can extend to a transformation on all $K$-theory, by taking the universal class in $K^0(\Z\times BU)$, representing it by a sequence in the above inverse limit, and applying the transformation to the terms individually.

Similarly, $K(\Z\times BU\times\Z\times BU)\cong\lim K(G_n\times G_n)$, so can extend pairings on compact spaces to pairings overall. I.e.\ we can use this to construct a multiplication on $\Z\times BU$ which classifies a multiplication of the $K$-theory of compact spaces.
\subsection*{Splitting principle}
If $X$ is a space and $V$ a vector bundle on $X$, then there is a space $Y$ and a map $f:Y\to X$ such that $f^*V$ is a sum of line bundles, and $f^*:K(X)\to K(Y)$ is injective.
[The standard construction using the projectivisation, splitting off one bundle at a time, works in $K$-theory, too.]
\subsection*{$K$-theory operations}
For $X$ compact, $V$ a vector bundle over $X$, let $\lambda^n(V)$ be the $n^\text{th}$ exterior power of $V$. This is \emph{not} additive. Let $\lambda(V)$ be the formal sum $\sum_{n=0}^\infty \lambda^n(V) t^n\in K(X)[[t]]$. This lands in the multiplicative group of elements with leading coefficient 1. This $\lambda$ is a monoid homomorphism from $\Vect(X)$ to this multiplicative group, which thus extends to $K(X)$. This follows from the equation $\lambda^n(V\oplus W)=\bigoplus_{i+j=n} \lambda^i V\otimes\lambda^jW$.

Now let $\lambda^i$ be the $i^\text{th}$ component of $\lambda$, thus $\lambda^i$ is a natural transformation $K(X)\to K(X)$. These don't have great properties [components are neither additive nor multiplicative].
\subsection*{Adams operations}
We define, for $x\in K(Y)$, $\psi(x)=\rank(x)-t\frac{d}{dt}\log \lambda_{-t}(x)\in K(X)[[t]]$, expanding the log via $\log(t)=\sum_{n=1}^\infty \frac{(-1)^{n+1}}{n}(t-1)^n$. Call this the total Adams operation. Let $\psi^i(x)$ be the $i^\text{th}$ component. The Adams operations have the following properties:
\begin{enumerate}\squishlist
\item The $\psi^i$ are additive.
\item $\psi^i(\scrL)=\scrL^{\otimes i}$ for line bundles $\scrL$.
\item The $\psi^i$ are multiplicative (in particular, they are ring homomorphisms).
\item $\psi^i\circ\psi^j=\psi^{ij}$.
\item $\psi^i(u)=i^nu$ for all $u\in\widetilde K(S^{2n})$ (in particular, they are unstable).
\end{enumerate}
\begin{proof}
By the splitting principle, the first two properties characterise.
\begin{enumerate}\squishlist
\item This holds by construction.
\item We compute. Let $V$ be a line bundle. Then $\lambda([V])=1+Vt$, so 
\[\log \lambda_{-t}[V]=\sum_{n=1}^\infty \frac{(-1)^{n+1}}{n}(-Vt)^n,\]
 and we get $\psi([V])=\sum [V]^nt^n$.
\item We can check this on line bundles by 2. But then it also holds for sums of these, so we can apply the splitting principle.
\item This is true on line bundles, and both sides are additive, so they must agree everywhere, as the first two properties characterise.
\item Starting with $n=1$, we know that $x=[H]-1$ is a generator of $\widetilde K(S^2)$, and $\psi^i(x)=\psi^i((x+1)-1)=(x+1)^i-1=ix$. In general, to increase $n$, we have $\psi^i$ multiplicative in terms of the inner operation, but this implies that it's multiplicative in terms of the smash operation.\qedhere
\end{enumerate}
\end{proof}

\pagebreak
\end{MarkusKtheory}
\begin{GuozhenEqKthyPractice}
\KanSemResponse
{Guozhen's Practice Talk on ``Equivariant $K$-theory and completion''}
Our main theorem will be that $K(BG)=\widehat{R(G)}$, for a compact Lie group $G$.

\begin{defn*}Let $G$ be a compact Lie group, and $X$ a compact $G$-space. Define the monoid of $G$-vector bundles over $X$ to be the set of (iso classes) of vector bundles on $X$ so that $G$ acts on the total space. Taking the group completion under Whitney sum, we obtain $K_G(X)$. The tensor product makes this a ring. We can define compactly supported $K$-theory, denoted $K^0_G(X)$, to be the K-theory of the one point compactification of $X$, and define relative equivariant $K$-theory as usual.
\end{defn*}
\begin{enumerate}\squishlist
\item $K_G(*)=R(G)$. 
\item If $G$ acts trivially on $X$, we have $K_G(X)=K(X)\otimes R(G)$.
\item If $G$ acts freely on $X$, $K_G(X)=K(X/G)$. More generally, if $N$ is a normal subgroup of $G$, and $N$ cats freely on $X$, $K_G(X)=K_{G/N}(X/N)$.
\item If $H\subset G$, then $K_H(X)\simeq K_G(X\times_H G)$, so that we can always recover the $K$-theory of a smaller group by studying a larger group.
\end{enumerate}
Let $V$ be a (finite-dimensional) $G$-vector space, and $X$ a compact $G$-space. Define a class $\lambda[V]\in K^0_G(V)$ (compactly supported --- i.e. K-theory of a sphere). We construct a complex on $V$:
\[0\to\lambda^0(V)\overset{\wedge v}{\to}\Lambda^1(V)\to\cdots\to \lambda^n(V)\]
Here, $\lambda^i(V)$ is the $i^\text{th}$ exterior algebra. This above is a complex on $V$, and $\wedge v$ is the operation which sends $w\mapsto w\wedge v$.

This complex is exact when $v\neq 0$, as can be checked on a basis. Thus, this complex defines an element $\lambda(V)$ of $K_G(V,V-0)$. (For this, we take the alternating sum, $\sum (-1)^i \lambda^i(V)$ as elements of $K$-theory, and note that this is trivial wherever the complex was exact.)
\rednote{[Here, I asked a dumb question, and the following went on the board: $0\to W^{\text{even}}\xrightarrow{d +d^*} W^{\text{odd}}\to 0$.]}
Let $\lambda^*(V)$ be the complex conjugate of $\lambda(V)$.
\begin{thm*}[Bott periodicity]
$K_G(X)\to K_G(X\times V)$ is an isomorphism, where $W\mapsto W\otimes\lambda^*(V)$.
\end{thm*}
\noindent\bluenote{[This is exactly the same as having an isomorphism $\widetilde K_G(X_+)\to K_G(V_+\wedge X_+)$. To make this sound more like traditional Bott periodicity, there must be some way to relate this to having an isomorphism $\widetilde K_G(X_+)\to K_G(S^V\wedge X_+)$.]}
\begin{cor*}[Thom isomorphism]
If $V$ is a $G$-vector bundle over $X$, then $K_G(X)\to K_G(V)$ is an isomorphism, also sending $w\mapsto w\otimes \lambda^*[V]$.
\end{cor*}
\begin{proof}
Let $Y$ be the $U(n)$-principle bundle associated with $V$. We have $K_{G\times U(n)}(Y)=K_G(X)$, as $Y$ is $U(n)$-free. On the other hand, \rednote{[this is broken, see tomorrow]}
%\[K_{G\times U(n)}(Y)\cong K_{G\times U(n)}(Y\times\C^n)=K_G(Y\times_{U(n)} \C^n)=K_G(V).\]
%(Used freeness, twice)
\begin{alignat*}{2}
K_G(X)&=K_{G\times U(n)}(Y)&&\text{as $Y$ is $U(n)$-free}\\
&=K_{G\times U(n)}(Y\times\C^n)&\qquad&\\
&=K_G(Y\times_{U(n)} \C^n)&&\text{}\\
&=K_G(V)&&\text{}
\end{alignat*}
\end{proof}
\subsection*{The theorem}
In general, $K_G(X)$ is a finite module over $R(G)$ when $X$ is a finite complex. \rednote{[There was some doubt about this during the lecture. It might seem enough to verify it for the cells $G/H\times D^n$.]} Let $I_G\subset R(G)$ be the augmentation ideal, the kernel of the dimension map $R(G)\to\Z$.
\begin{thm*}
If $X$ is a finite $G$-CW-complex \rednote{[we need $K_G(X)$ to be a finite $R(G)$-module, and we didn't know if this follows from the finite $G$-CW-complex assumption]} then $K(X\times_G EG)$ is the $I_G$-adic completion of $K_G(X)$, that is $\varprojlim K_G(X)/I_G^n$.
\end{thm*}
\begin{proof}
We'll prove that $K_G(X)\to K_G(X\times EG)$ is the completion.
\begin{enumerate}
\item When $G=T^n$.
\begin{lem*}
Let $G\to T^n$ be a homomorphism, let $X$ be a $T^n$-space, so that $G$ acts on $X$ through this homomorphism. Then there's an isomorphism $K_G(X\times ET^n)\simeq K_G(X)_{\hat I_{T^n}}$.
\end{lem*}
\begin{proof}[Proof of lemma]
For $n=1$, have $ET=S^\infty=\varinjlim S^{2n-1}\subset \C^n$. $T=U(1)$ acts as usual. We've got a pair $(D^{2n},S^{2n-1})$ acted on by $G$. Use the Thom iso, so get
\[K_G(X\times(D^{2n},S^{2n-1}))\cong K_G(X)\qquad    \text{Thom class}\longleftrightarrow 1\]
%\[\xymatrix{
%K_G(X)\ar[r]^{\lambda^*(C^n)}&\ar@{=}[d]K_G(X)\ar[r]&K_G(X\times S^{2n-1})\ar[r]& K_G(X\times(D^{2n},S^{2n-1}))\\
%&K_G(X\times D^{2n}
%\]
Now $\lambda^*(C^n)=(\lambda^*(C))^n=(1-\rho)^n=\xi^n$. Have SES
\[\xymatrix{
0\ar[r] &K_G(X)/\xi^n\ar[r] &K_G(X\times S^{2n-1})\ar[r]& {_{\xi}K_G}\ar[r]&0
}\]
As in paper for the rest of $n=1$.

We do the rest by induction:
\[K_G(X\times ET^n)=K_G(X\times ET^{n-1}\times ET)=K_G(X\times ET^{n-1})_{\hat I_T}=(K_G(X)_{\hat I_{T^{n-1}}})_{\hat I_{T}}\]
This is $K_G(X)_{I_{T^n}}$ as desired.
\end{proof}
\item $G=U(n)$: we have $T^n\subset U(n)$, and 
\[\xymatrix{
K_{U(n)}(X)\ar[r]\ar[d]& K_{U(n)}(EU(n))\ar[d]\\
\ar[r]K_{T^n}(X)&K_{T^n}(X\times EU(n))
}\]
\rednote{From here, it all gets iffy. I haven't read this stuff, just copied it from my notes, hoping to improve it tomorrow. Sorry.}
Claim that the second row is a direct summand of the first. As the topologies are the same, somehow this works out.

Now we'll prove $K_{U(n)}(X)$ is a direct summand of $K_{T^n}(X)$. We also have
\[K_{U(n)}(X)\to K_{U(n)}(X\times_{T^n} U(n))=K_{T^n}(X).\]  
We have a fibration
\[\xymatrix{U(n)/T^n\ar[r]&X\times_{T^n}U(n)\ar[d]\\
&X}\]
Now $U(n)/T^n=\GL(N,\C)/H$ with $H$ upper triangular, so that $U(n)/T^n$ is a complex manifold. On a complex manifold, we have the Dolbeault complex. Now for any $W$ over $X\times_{T^n}U(n)$, we define
\[d_*(W)=\text{index}\{W\otimes\text{the Dolbeault complex of the fibre}\}\]
We'd like to prove that $d_*$ gives a $K_{U(n)}(X)$-module morphism $K_{U(n)}(X\times_{T^n}U(n))\to K_{U(n)}(X)$. One checks that we've exhibited $K_{U(n)}(X)$ as a direct summand.
\item The general case: we embed $G\cofibration U(n)$ for some $n$. Then we have $K_G(X)=K_{U(n)}(X\times_G U(n))$. Moreover
\[K_G(X\times EG)=K_G(X\times EU(n))=K_{U(n)}(X\times_G\times U(n)\times EU(n)).\qedhere\]
\end{enumerate}
\end{proof}

\pagebreak
\end{GuozhenEqKthyPractice}
\begin{AtiyahSegalEqKthy}
\KanSemResponse
{``Equivariant $K$-theory and completion'' --- Atiyah \& Segal}
As a first and not-so-poignant comment, it seems fortuitous that the authors introduce the category of pro-objects. Model structures on the category of pro-objects in a model category appears to be the subject of an upcoming seminar, so I've had a head start there! I suppose the reason for the appearance of pro-objects in this paper is as Markus explained --- it's difficult to work with the $K$-theory of non-compact spaces, so we need to choose an exhaustive increasing family of compact subsets. We'd also rather like to readily be able to compare the outcomes when we choose different families, and all this is built into the language of pro-objects.

Anyway, I've spent a fair bit of time reading this paper, and thinking about equivariant $K$-theory, and it's been fairly stimulating. Although I've learnt a lot, somehow I don't have all that much to write.

Michael and I stared for quite a while today at the morphism $R(G)\to K_G(EG)$, which according to the main theorem should be the completion map. We couldn't figure out why this should be the case. To be more specific, the theorem states that the maps
\[\alpha_n: R(G)/I_G^n R(G)\to K^*_G(E_G^n),\]
where $E_G^n$ is an increasing family of compact subsets of $EG$ with union all of $EG$, are an isomorphism of pro-rings. Even this statement has proven difficult to interpret, although the proof is convincing. Perhaps you have a good way to think about it. There must be something useful to say about the short exact sequences on page 6, but I just don't really know what it is.

There are, nonetheless, a large number of things I learned from reading. For example, there's an isomorphism $K_G(X)\simeq K(X)\otimes R(G)$ when $X$ is a space with trivial $G$-action, thanks to Schur's lemma, really. I was stuck for a way to figure this out for quite a while, until Michael pointed it out of page 15.

I suppose another thing gained from this paper is an opportune moment to consider the Thom isomorphism for $K$-theory. There's a particularly key use of it in understanding the maps $\alpha_n$ in the case of the circle, as the class $(1-\rho)$ generates the augmentation ideal. Really, before deriving the Thom isomorphism, one should understand equivariant Bott periodicity, and although I haven't even read a proof of non-equivariant Bott periodicity, this paper is `a good invitation'.

Milnor's model for $EG$ is great --- the direct limit of repeated joins of the group. I suppose that this works for a compact Lie group, as such a group is in particular a CW-complex, and for CW-complexes $X$ and $Y$, we have $X*Y=\Sigma(X\wedge Y)$, so that the connectivity keeps increasing. One wonders if this construction applies to non-Lie groups, for example.

There may be plenty more to say about this paper, but perhaps at the moment I'm not the one to say it.

\textbf{}

\noindent [Question: Does a vector bundle on $X$ induce the structure of $\pi_1X$-module on its fibre? I thought about a possible construction but didn't have time to decide whether it was well defined. If there were a connection here, there might correspondingly be something to say about the morphism $R(G)\to K(BG)$, at least when $G$ is discrete, so that $\pi_1(BG)=G$.]

\pagebreak
\end{AtiyahSegalEqKthy}
\begin{GuozhenEqKthy}
\KanSemResponse
{Guozhen's Talk on ``Equivariant $K$-theory and completion''}
Our main theorem will be that $K(BG)=\widehat{R(G)}$, for a compact Lie group $G$.

\begin{defn*}Let $G$ be a compact Lie group, and $X$ a compact $G$-space. Define the monoid of $G$-vector bundles over $X$ to be the set of (iso classes) of vector bundles on $X$ so that $G$ acts on the total space. Taking the group completion under Whitney sum, we obtain $K_G(X)$. The tensor product makes this a ring. We can define compactly supported $K$-theory, denoted $K^c_G(X)$, to be the K-theory of the one point compactification of $X$, and define relative equivariant $K$-theory as usual.
\end{defn*}
\begin{enumerate}\squishlist
\item $K_G(*)=R(G)$. 
\item If $G$ acts trivially on $X$, we have $K_G(X)=K(X)\otimes R(G)$ (by Schur's lemma).
\item If $G$ acts freely on $X$, $K_G(X)=K(X/G)$. More generally, if $N$ is a normal subgroup of $G$, and $N$ cats freely on $X$, $K_G(X)=K_{G/N}(X/N)$.
\item Given $H\to G$, get a map $K_G\to K_H$.
\item If $H\subset G$, then $K_H(X)\simeq K_G(X\times_H G)$, so that we can always recover the $K$-theory of a smaller group by studying a larger group.
\end{enumerate}
Let $V$ be an $n$-dimensional $G$-vector space, and $X$ a compact $G$-space. Define a class $\lambda[V]\in K^0_G(V)$ (compactly supported --- i.e. K-theory of a sphere). We construct a complex on $V$:
\[0\to\lambda^0(V)\overset{\wedge v}{\to}\Lambda^1(V)\to\cdots\to \lambda^n(V)\to0\]
Here, $\lambda^i(V)$ is the $i^\text{th}$ exterior algebra. This above is a complex on $V$, and $\wedge v$ is the operation which sends $w\mapsto w\wedge v$.

This complex is exact when $v\neq 0$, as can be checked on a basis. Thus, this complex defines an element $\lambda(V)$ of $K_G(V,V-0)=K^c_G(V)$. (For this, we take the alternating sum, $\sum (-1)^i \lambda^i(V)$ as elements of $K$-theory, and note that this is trivial wherever the complex was exact.)
%\rednote{[Here, I asked a dumb question, and the following went on the board: $0\to W^{\text{even}}\xrightarrow{d +d^*} W^{\text{odd}}\to 0$.]}
Let $\lambda^*(V)$ be the complex conjugate of $\lambda(V)$.
\begin{thm*}[Bott periodicity]
$K_G(X)\to K_G^c(X\times V)$ is an isomorphism, where $W\mapsto W\otimes\lambda^*(V)$. Moreover, this is a $K_G(X)$-module isomorphism.
\end{thm*}
\noindent Note here that if $X$ is compact, the compactly supported $K$-theory of $X\times V$ is the reduced $K$-theory of the Thom space. This holds by definition of the compactly supported $K$-theory. Note that as in the standard Thom isomorphism, we have $\lambda^*(V\oplus W)=\lambda^*(V)\cdot\lambda^*(W)$, where the dot is the external smash product.
%\noindent\bluenote{[This is exactly the same as having an isomorphism $\widetilde K_G(X_+)\to K_G(V_+\wedge X_+)$. To make this sound more like traditional Bott periodicity, there must be some way to relate this to having an isomorphism $\widetilde K_G(X_+)\to K_G(S^V\wedge X_+)$.]}
\begin{cor*}[Thom isomorphism]
If $V$ is a $G$-vector bundle over $X$, then $K_G(X)\to K_G(V)$ is an isomorphism, also sending $w\mapsto w\otimes \lambda^*[V]$.
\end{cor*}
\begin{proof}Broken.
\end{proof}
%\begin{proof}
%Let $Y$ be the $U(n)$-principle bundle associated with $V$. We have $K_{G\times U(n)}(Y)=K_G(X)$, as $Y$ is $U(n)$-free. On the other hand, \rednote{[this is broken, see tomorrow]}
%\begin{alignat*}{2}
%K_G(X)&=K_{G\times U(n)}(Y)&&\text{as $Y$ is $U(n)$-free}\\
%&=K_{G\times U(n)}(Y\times\C^n)&\qquad&\\
%&=K_G(Y\times_{U(n)} \C^n)&&\text{}\\
%&=K_G(V)&&\text{}
%\end{alignat*}
%\end{proof}
\subsection*{The main theorem}
If $G$ is compact Lie, $R(G)$ is noetherian, containing the augmentation ideal $I_G$, the kernel of the dimension map $R(G)\to\Z$. If $X$ is a finite  $G$-CW-complex (built out of cells $G/H\times D^n$), $K_G(X)$ is a finite module, as $K_G(G/H)=R(H)$ is a finite $R(G)$-module.

In general, $K_G(X)$ is a finite module over $R(G)$ when $X$ is a finite complex. %\rednote{[There was some doubt about this during the lecture. It might seem enough to verify it for the cells $G/H\times D^n$.]} 
Let $I_G\subset R(G)$ be the augmentation ideal, the kernel of the dimension map $R(G)\to\Z$.

We have a map $K_G(X)\to K_G(X\times EG)=K(X\times_G EG)$ induced by the projection.
\begin{thm*}
If $X$ is a finite $G$-CW-complex\footnote{In the paper, we only need $K_G(X)$ to be a finite $R(G)$-module.},
% \rednote{[we need $K_G(X)$ to be a finite $R(G)$-module, and we didn't know if this follows from the finite $G$-CW-complex assumption]} 
then $K(X\times_G EG)$ is the $I_G$-adic completion of $K_G(X)$. That is, $K(X\times_G EG)=\varprojlim K_G(X)/I_G^nK_G(X)$.
\end{thm*}
\begin{proof}
We have a map $K_G(X)\to K_G(X\times EG)=K(X\times_G EG)$ induced by the projection.
We'll prove that $K_G(X)\to K_G(X\times EG)$ is the completion, in the following four steps.
\begin{enumerate}
\item When $G=T^1$.
\begin{lem*}
Let $G\to T$ be a homomorphism, let $X$ be a $T$-space, so that $G$ acts on $X$ through this homomorphism. Then there's an isomorphism $K_G(X\times ET)\simeq K_G(X)_{\hat I_{T}}$.
\end{lem*}
\begin{proof}[Proof of lemma]
Have $ET=S^\infty=\varinjlim S^{2n-1}\subset \C^n$. $T=U(1)$ acts as usual. We've got a pair $(D^{2n},S^{2n-1})$ acted on by $G$.
 Use the Thom iso, so get
\[K_G(X\times(D^{2n},S^{2n-1}))\cong K_G(X)\qquad    \text{Thom class}\longleftrightarrow 1\]
%\[\xymatrix{
%K_G(X)\ar[r]^{\lambda^*(C^n)}&\ar@{=}[d]K_G(X)\ar[r]&K_G(X\times S^{2n-1})\ar[r]& K_G(X\times(D^{2n},S^{2n-1}))\\
%&K_G(X\times D^{2n}
%\]
Get a LES
\[K_G(X\times pr)\to K_G(X\times D)\to K_G(X\times S)\to K_G^{-1}(X\times pr)\to K_G^{-1}(X\times D)\]
Identifying the first and second terms with $K_G(X)$, the map between them is multiplication by the Thom class. Now the Thom class is multiplicative, so it's the $N^\text{th}$ power of the Thom class of $\C$, which is $1-\rho$ (as in the traditional case).
Now $\lambda^*(C^n)=(\lambda^*(C))^n=(1-\rho)^n=\xi^n$. Have SES
\[\xymatrix{
0\ar[r] &K_G(X)/\xi^n\ar[r] &K_G(X\times S^{2n-1})\ar[r]& {_{\xi^n}K_G^1}\ar[r]&0
}\]
where the last term is the kernel of multiplication by $\xi^n$.

In the special case $\varprojlim^1$ vanishes and the $\varprojlim$ vanishes (by finite module over noetherian ring-ness), so $K_G(X\times S^\infty)\cong \varprojlim K_G(X\times S^{2n-1})$. But the inverse limit of the third terms vanishes. [Get ML because of finiteness]
\end{proof}
\item When $G=T^n$, $n>1$.
\begin{lem*}
Let $G\to T^n$ be a homomorphism, let $X$ be a $T^n$-space, so that $G$ acts on $X$ through this homomorphism. Then there's an isomorphism $K_G(X\times ET^n)\simeq K_G(X)_{\hat I_{T^n}}$.
\end{lem*}
\begin{proof}[Proof of lemma]
We do the rest by induction:
\[K_G(X\times ET^n)=K_G(X\times ET^{n-1}\times ET)=K_G(X\times ET^{n-1})_{\hat I_T}=(K_G(X)_{\hat I_{T^{n-1}}})_{\hat I_{T}}\]
We know that $R(T^n)=\Z[X_1^{\pm1},\ldots,X_n^{\pm1}]$, so that
this is $K_G(X)_{I_{T^n}}$ as desired.
\end{proof}
\item $G=U(n)$: we have $T^n\subset U(n)$ the diagonal maximal torus, and the following commuting diagram, in which the vertical maps are the restrictions, and the bottom map is known to be the completion, by part 2. We take $EU(n)$ and $ET^n$ to be the same topological space.
\[\xymatrix{
K_{U(n)}(X)\ar[r]\ar[d]^{\pi^*}& K_{U(n)}(X\times EU(n))\ar[d]\\
\ar[r]K_{T^n}(X)&K_{T^n}(X\times ET^n)
}\]
We'll show that the second row is a direct summand of the first. This will suffice, as the $I_{Un}$ and the $I_{Tn}$ topologies coincide, as $R(U(n))\to R(T^n)$ is a finite extension --- note that $R(U(n))$ is the invariants of the Laurent series ring $R(T^n)$ under the action of the symmetric group.

Now we'll prove $K_{U(n)}(X)$ is a direct summand of $K_{T^n}(X)$. We also have
\[K_{U(n)}(X)\to K_{U(n)}(X\times_{T^n} U(n))=K_{T^n}(X).\]  
We have a fibration
\[\xymatrix{U(n)/T^n\ar[r]&X\times_{T^n}U(n)\ar[d]\\
&X}\]
Now $U(n)/T^n=\GL(N,\C)/H$ with $H$ is the upper triangular matrices --- a flag variety, so that $U(n)/T^n$ is a complex manifold, and moreover a rational variety (as a flag variety has a CW-decomposition and the top cell gives a rational map to complex projective space).


Now for some index theory. On a complex manifold, we have the Dolbeault complex:
\[0\to \Omega^{0,0}\overset{\overline\partial}{\to} \Omega^{0,1}\overset{\overline\partial}{\to} \Omega^{0,2}\overset{\overline\partial}{\to} \Omega^{0,3}\overset{\overline\partial}{\to}\cdots\]
 Now for any vector bundle $W$ over $X\times_{T^n}U(n)$, we define
\[d_*(W)=\text{index}\{W\otimes\text{the Dolbeault complex of the fibre}\}\]
Take $D=\bar\partial+\bar\partial^*:W\otimes \Omega^{0,\text{even}}\to W\otimes \Omega^{0,\text{odd}}$. We can take the index of $D$:
\[d(W)=\ker(D)-\coker(D)\]
Not that these aren't actually vector bundles, but somehow their difference is, and in fact gives a well defined element of $K$-theory. Thus we obtain $d:K_G(X\times_{T^n}U(n))\to K_G(X)$. Now $d$ preserves the module structure, so if we prove that $D(\pi^*(1))$, we'll see that $d\pi^*=\id$. 

Now $D(\pi^*(1))$ is the index of the Dolbeault complex $\sum (-1)^i H^i(U/T,\calO)$, but the \rednote{something} numbers are birational invariants, so get $h_i=\delta_{i0}$. This implies that $d(\pi^*(1))=1$.
%
%We'd like to prove that $d_*$ gives a $K_{U(n)}(X)$-module morphism $K_{U(n)}(X\times_{T^n}U(n))\to K_{U(n)}(X)$. One checks that we've exhibited $K_{U(n)}(X)$ as a direct summand.
\item The general case: we embed $G\cofibration U(n)$ for some $n$. Then we have $K_G(X)=K_{U(n)}(X\times_G U(n))$. Moreover
%\[K_G(X\times EG)=K_G(X\times EU(n))=K_{U(n)}(X\times_G\times U(n)\times EU(n)).\]
\begin{alignat*}{2}
K(X\times_G EG)&= K(X\times_G EU(n))\\
&=K(X\times_G U(n)\times_{U(n)}EU(n))
\end{alignat*}
Using again the fact that $R(G)$ is finite over $R(U(n))$ to see that the topologies are right, we conclude that the theorem holds for general $G$.\qedhere
\end{enumerate}
\end{proof}
\subsection*{Comments and Questions after the lecture}
If $G=U(n)$, $R(G)=\Z[\lambda_1,\ldots,\lambda_n,\lambda_n^{-1}]$ ($\lambda_n$ is the determinant, which is invertible). Then $K(BU(n))=\Z[[u_1,\ldots,u_n]]$, and the $\varprojlim^1$ is zero, so $K(BU)$ is the inverse limit of all these.

\pagebreak
\end{GuozhenEqKthy}
\begin{MarkusHopfInvOnePractice}
\KanSemResponse
{Markus's Practice talk on Hopf Invariant One}
Given $f:S^{2n-1}(S^n)$, for $n>1$, the cone $C(f)$ of $f$ has a CW-structure with one cell in dimensions 0, $n$ and $2n$. Its cohomology ring has a copy of $\Z$ in each of these dimensions. Let $b\in H^{2n}(C(f);\Z)$ be the generator corresponding to a fixed orientation of $S^2n$. Then if $a$ is any generator of $H^{n}(C(f);\Z)$, we have $a\cup a=h(f)\cdot b$ for some unique $h(f)\in\Z$. This is the Hopf invariant. It is necessarily zero unless $n$ is even.

This descends to a map $h:pi_{2n-1}(S^n)\to\Z$.
\begin{lem*}
h is a group homomorphism.
\end{lem*}
\begin{proof} All the maps in the diagram on the left are isomorphisms on $H^n$. Their action on $H^{2n}$ is recorded on the right:
\[\xymatrix@!0@R=.9cm@C=2.3cm{
&&C(f)\ar[ld]\\
C(f+g)\ar[r]&C(f\vee g)\\
&&C(g)\ar[lu]
}\qquad\xymatrix@!0@R=.9cm@C=2.3cm{
&&\Z\ar[ld];[]^{(1,0)}\\
\Z\ar[r];[]_{(1,1)}&\Z\\
&&\Z\ar[lu];[]_{(0,1)}
}\]
\end{proof}
\begin{prop*}
If $n$ is even, then $2\in\im(h)$.
\end{prop*}
\begin{prop*}
If $S^{n-1}$ is parallelisable, then $\pi_{2n-1}(S^n)$ contains an element of Hopf invariant one.
\end{prop*}
\begin{thm*}
If $\pi_{2n-1}(S^n)$ contains an element of Hopf invariant one, then $n=2,4,8$.
\end{thm*}
\noindent (Note we can construct these parallelisations via multiplications from division algebras.)

Our main tool for constructing maps of non-trivial Hopf invariant is the following. Suppose given a map $\phi:S^{n-1}\times S^{n-1}\to S^{n-1}$. Fixing a basepoint in $S^{n-1}$, we can embed
\[S^{n-1}\vee S^{n-1}\to S^{n-1}\times S^{n-1}\overset{\phi}{\to} S^{n-1}\]
Call this composite $\phi_1\vee\phi_2$. Let $\alpha,\beta$ be the degrees of these maps.

We use the Hopf construction to get a map $S^{2n-1}\to S^n$, by inducing a morphism of pushout diagrams:
\[\xymatrix{%@!0@R=.9cm@C=4cm{
S^{n-1}\times S^{n-1}\ar[r]\ar[d]\ar[rd]&S^{n-1}\times CS^{n-1}\ar[rd]\\
CS^{n-1}\times S^{n-1}\ar[rd]&S^{n-1}\ar[r]\ar[d]&CS^{n-1}\\
&CS^{n-1}
}\]
This induces a map $f:S^{2n-1}\to S^n$. We will not prove the following:
\begin{prop*}
The Hopf invariant of $f$ is equal to $\pm\alpha\beta$.
\end{prop*}

\begin{proof}[Proof of first proposition]
The clutching function of the tangent bundle of $S^n$ is
\[\widetilde\phi:S^{n-1}\to\Omega^{n-1}S^{n-1}\]
sends $x$ to ``reflection along the hyperplane orthogonal to $x$''. Let $\phi:S\times S\to S$ be the adjoint of this (the unpointed version).

Fixing one component, we're just measuring the degree of a reflection on $S^{n-1}$, which is $-1$ (as $n$ is even).

The other map is ``$x\mapsto x(bpt)$''. Well, any point has two preimages, and they have the same local degree, as the antipode has degree $(-1)^n=1$, so we have total degree either $\pm2$.
\end{proof}
\begin{proof}[Proof of second proposition]
Let $S^{n-1}$ be parallelisable. This means that there is a section $s$ of (the fibre bundle) $O(n)\overset{pr_1}{\to}S^{n-1}$.
\[\xymatrix{S^{n-1}\times S^{n-1}\ar[r]\ar[d]_{s\times 1}&S^{n-1}\\
O(n)\times S^{n-1}\ar[ur]_{act}&
}\]
Then $e_1$ is a right unit of this multiplication. Replacing $s$ by $s(e_1)^{-1}s$, we still have a section, and $e_1$ is now an identity for the top multiplication. But now, $\phi_1,\phi_2$ are the identity, and so have degree 1, so that the resulting Hopf construction has Hopf invariant one.
\end{proof}
\begin{proof}[Proof of theorem]
Assume $n$ is even, as we may.
The Hopf invariant, having been defined in singular cohomology, needs to be translated to $K$-theory. We use the Atiyah-Hirzebruch spectral sequence:
\[\widetilde H^s(C(f),K^t(\ast))\implies \widetilde K^{s+t}(C(f)).\]
As $n$ is even, there can be no non-trivial differentials by $E_2$. So 
\[\widetilde K(C(f))\cong\widetilde H^{2n}(C(f);\Z)\oplus \widetilde H^n(C(f);\Z).\]
Now there is a multiplication on the whole $K$-theory spectrum, and the spectral sequence is multiplicative. So, squaring in $K$-theory corresponds to squaring in singular cohomology.

Now recall the Adams operations from the previous lecture. Note also that
\begin{lem*}
For a prime $p$, $\phi^p(x)\equiv x^p$ modulo multiples of $p$.
\end{lem*}
\begin{proof}
This holds for line bundles. Then use the fact that $(x_1+x_2+\cdots+x_n)^p\equiv x_1^p+\cdots x_n^p$, and apply the splitting principle.
\end{proof}
We want to show that $a^2\cong 0\pmod{2}$ if $n\neq2,4,8$. This is equivalent to showing that $\phi^2(a)\cong 0\pmod{2}$. Let $n=2m$. Now
\[\psi^2(a)=2^ma+\mu b\]
by the naturality of the AHSS. We also note that
\[\psi^3(a)=3^ma+\nu b.\]
As $\psi^3$ and $\psi^2$ commute, have
\[3^m(2^ma+\mu b)+\nu 2^{2m}b=2^n(3^ma+\nu b)+\mu2^{2m}b\]
So $2^m(2^m-1)\nu= 3^m(3^{m}-1)\mu$.
Assume that $\mu$ is odd. Then $2^m|(3^{m}-1)$, but, this only happens when $m=1,2,4$, by elementary number theory.
\end{proof}


\pagebreak
\end{MarkusHopfInvOnePractice}
\begin{MarkusHopfInvOne}
\KanSemResponse
{Markus's talk on the Hopf Invariant One problem}
Given $f:S^{2n-1}(S^n)$, for $n>1$, the cone $C(f)$ of $f$ has a CW-structure with one cell in dimensions 0, $n$ and $2n$. Its cohomology ring has a copy of $\Z$ in each of these dimensions. Let $b\in H^{2n}(C(f);\Z)$ be the generator corresponding to a fixed orientation of $S^{2n}$. Then if $a$ is any generator of $H^{n}(C(f);\Z)$, we have $a\cup a=h(f)\cdot b$ for some unique $h(f)\in\Z$. This is the Hopf invariant.

This descends to a map $h:\pi_{2n-1}(S^n)\to\Z$, which necessarily zero unless $n$ is even.
\begin{lem*}
h is a group homomorphism $h:\pi_{2n-1}(S^n)\to\Z$.
\end{lem*}
\begin{proof} All the maps in the diagram on the left are isomorphisms on $H^n$. Their action on $H^{2n}$ is recorded on the right:
\[\xymatrix@!0@R=.9cm@C=2.3cm{
&&C(f)\ar[ld]\\
C(f+g)\ar[r]&C(f\vee g)\\
&&C(g)\ar[lu]
}\qquad\xymatrix@!0@R=.9cm@C=2.3cm{
&&\Z\ar[ld];[]^{(1,0)}\\
\Z\ar[r];[]_{(1,1)}&\Z\oplus\Z\\
&&\Z\ar[lu];[]_{(0,1)}
}\]
\end{proof}
\begin{prop*}
If $n$ is even, then $2\in\im(h)$.
\end{prop*}
\begin{prop*}
If $S^{n-1}$ is parallelisable, then $\pi_{2n-1}(S^n)$ contains an element of Hopf invariant one.
\end{prop*}
\begin{thm*}
If $\pi_{2n-1}(S^n)$ contains an element of Hopf invariant one, then $n=2,4,8$.
\end{thm*}
\noindent (Note we can construct these parallelisations via multiplications from division algebras.)

Our main tool for constructing maps of non-trivial Hopf invariant is the `Hopf construction'. Suppose given a map $\phi:S^{n-1}\times S^{n-1}\to S^{n-1}$. Fixing a basepoint in $S^{n-1}$, we can embed
\[S^{n-1}\vee S^{n-1}\to S^{n-1}\times S^{n-1}\overset{\phi}{\to} S^{n-1}\]
Call this composite $\phi_1\vee\phi_2$. Let $\alpha,\beta$ be the degrees of these maps.

We use the Hopf construction to get a map $S^{2n-1}\to S^n$, by inducing a morphism of pushout diagrams:
\[\xymatrix{%@!0@R=.9cm@C=4cm{
S^{n-1}\times S^{n-1}\ar[r]\ar[d]\ar[rd]&S^{n-1}\times CS^{n-1}\ar[rd]\\
CS^{n-1}\times S^{n-1}\ar[rd]&S^{n-1}\ar[r]\ar[d]&CS^{n-1}\\
&CS^{n-1}
}\]
This induces a map $f:S^{2n-1}\to S^n$ (note that the first pushout is a boundary decomposition of $D^n\times D^n$). We will not prove the following:
\begin{prop*}
The Hopf invariant of $f$ is equal to $\pm\alpha\beta$.
\end{prop*}

\begin{proof}[Proof of first proposition]
The clutching function of the tangent bundle of $S^n$ is the adjoint of
\[\widetilde\phi:S^{n-1}\to\Map(S^{n-1},S^{n-1})\]
which sends $x$ to ``reflection along the hyperplane orthogonal to $x$''. Let $\phi:S^{n-1}\times S^{n-1}\to S^{n-1}$ be the adjoint of this.

Fixing one component, we're just measuring the degree of a reflection on $S^{n-1}$, which is $-1$ (as $n$ is even).

The other map is ``$x\mapsto \widetilde\phi(x)(bpt)$''. Well, any point has two preimages, and they have the same local degree, as the antipode has degree $(-1)^n=1$, so we have total degree either $\pm2$.
\end{proof}
\begin{proof}[Proof of second proposition]
Let $S^{n-1}$ be parallelisable. This means that there is a section $s$ of (the fibre bundle) $O(n)\overset{pr_1}{\to}S^{n-1}$ (using that the normal bundle of $S^{n-1}$ in $\R^n$ is trivial).
\[\xymatrix{S^{n-1}\times S^{n-1}\ar[r]^\phi\ar[d]_{s\times 1}&S^{n-1}\\
O(n)\times S^{n-1}\ar[ur]_{act}&
}\]
Then $e_1$ is a right unit of this `multiplication' $\phi$. Replacing $s$ by $s(e_1)^{-1}s$, we still have a section, and $e_1$ is now an identity for $\phi$. But now, $\phi_1,\phi_2$ are the identity, and so have degree 1, so that the resulting Hopf construction has Hopf invariant one.
\end{proof}
\noindent This, taken with the theorem, also shows that the only spheres which are $H$-spaces are $S^1$, $S^3$ and $S^7$, putting the required bound on the number of real division algebras that can be!
\begin{proof}[Proof of theorem]
Assume $n$ is even, as we may.
The Hopf invariant, having been defined in singular cohomology, needs to be translated to $K$-theory. We use the Atiyah-Hirzebruch spectral sequence:
\[\widetilde H^s(C(f),K^t(\ast))\implies \widetilde K^{s+t}(C(f)).\]
As $n$ is even, there can be no non-trivial differentials by $E_2$. So 
\[\widetilde K(C(f))\cong\widetilde H^{2n}(C(f);\Z)\oplus \widetilde H^n(C(f);\Z).\]
In fact this holds as a ring\footnote{using the multiplicatication on $K$-theory, and that the spectral sequence is multiplicative}, where we take an associated graded object on the left hand side. However, one notes that the error terms lie in \emph{lower} gradings, which are zero, so we actually get a ring isomorphism. So, squaring in $K$-theory corresponds to squaring in singular cohomology. That is, if $a,b$ are the relevant generators  in $K$-theory, we still have $a^2=h(f)\cdot b$.

Recall the Adams operations from the previous lecture. One thing that wasn't mentioned:
\begin{lem*}
For a prime $p$, $\phi^p(x)\equiv x^p$ modulo multiples of $p$.
\end{lem*}
\begin{proof}
This holds for line bundles with equality. Then use the fact that %$(x_1+x_2+\cdots+x_n)^p\equiv x_1^p+\cdots x_n^p$,
\[(x_1+x_2+\cdots+x_n)^p\equiv x_1^p+\cdots x_n^p+p\cdot f(\lambda^1(x),\ldots,\lambda^p(x)),\] (which follows by considering symmetric poynomials...) and apply the splitting principle.
\end{proof}
We want to show that $a^2\cong 0\pmod{2}$ if $n\neq2,4,8$. This is equivalent to showing that $\phi^2(a)\cong 0\pmod{2}$. Let $n=2m$. Now
\[\psi^2(a)=2^ma+\mu b\]
by the naturality of the AHSS. We also note that
\[\psi^3(a)=3^ma+\nu b.\]
As $\psi^3$ and $\psi^2$ commute (and $b$ actually comes from a spherical class), have
\[3^m(2^ma+\mu b)+\nu 2^{2m}b=2^n(3^ma+\nu b)+\mu2^{2m}b\]
So $2^m(2^m-1)\nu= 3^m(3^{m}-1)\mu$.
If $\mu$ is odd then $2^m|(3^{m}-1)$, but this only happens when $m=1,2,4$, by elementary number theory.
\end{proof}


\pagebreak
\end{MarkusHopfInvOne}
\begin{AdamsStableHtpy}
\KanSemResponse
{``Stable homotopy and generalised homology'' --- Adams}
My intention during this reading assignment has been to read as much as I can afford of part III of Adams' book.

My first observation when reading the definition and discussion in \S2 of spectra, is that the definitions of CW-spectra and of $\Omega$-spectra seem far more natural than when I first heard them. One note that an $\Omega$-spectrum is a spectrum which receives as many `functions of spectra' as it possibly can, while a CW-spectrum is one out of which there are as many functions as possible (as extending maps from a subcomplex is relatively simple). Having recently found out more about model categories, one hopes that these can be made the classes of fibrant and cofibrant objects respectively, or at least are closely related to these classes of spectra. Indeed, Adams notes that the functions between two spectra are not in general sufficient unless the domain is $\Omega$ and the codomain is CW.

I suppose that it is the case that when one allows morphisms to be defined only on cofinal objects, one is inverting the inclusions of cofinal subcomplexes. I wouldn't think that inverting the class of inclusions of cofinal subcomplexes is sufficient, though, to get the right homotopy category. However, it seems that that is what's being claimed, in Adams' definition of the homotopy category of spectra! I need to more reading, or to obtain further guidance, in order to answer this.

It is interesting to wonder whether this presentation would have been different had Model categories been around --- they arrived soon after, didn't they? 

Another thing which I liked is the method given for replacing a CW-spectrum with an $\Omega_0$-spectrum. One applies Brown representability (BR) to construct the spaces of a spectrum. Then, fortunately, there's a J.H.C.\ Whitehead theorem with which to compare the result with the original spectrum. Similarly, BR appears in order to prove that the stable category has arbitrary coproducts. 

Everything here seems to hint at how BR belies the whole idea of spectra. At the moment, I'm a little too close to the canvas to see everything laid out in its splendour, but there are themes that are coming through. Spaces give cohomology theories via the $Q=\Omega^\infty\Sigma^\infty=\varinjlim \Omega^n\Sigma^n$ construction, and cohomology theories give infinite loop spaces via BR. These two processes are adjoint maps, and the endofunctor $\text{BR}\circ Q$ on topological spaces is the `free infinite loop space' monad! On the other hand, $Q\circ\text{BR}$ is something I don't understand.

Around this point, one runs into the challenges of \S4. That is, Adams treats the smash product of spectra. I have two complaints. Firstly, this part of the book was devilishly disheartening to read, perhaps a necessary evil\footnote{Although certainly an evil that I cannot overcome with my current time constraints.}. Secondly, I have started to empathise with people's suggestions that a construction of the smash product at the level of the homotopy category is not as desirable as a construction in the model category. It's the first complaint which has proven overwhelming.

At least it is becoming a little clearer to me why one might expect the smash product structure to be difficult in this derivation. The non-commutativity of the sphere sequence is the problem! This perspective has the difficulty for me that it is not completely obvious that spectra were ever supposed to be $S$-modules in the first place (for $S$ some sphere sequence in a category of sequences). One project on the agenda is that of convincing myself that this is the natural concept. One say that it is `obvious', but would this just be sophistry?

Moving on, one comes to duality. Spanier-Whitehead duality is still a little mysterious to me, but the construction of dualities in the stable category is not too difficult (once the smash product is around) --- just another application of the power of BR, taken with Yoneda's handy lemma.


\pagebreak
\end{AdamsStableHtpy}

\begin{KrishanuAdamsStableHtpy}
\KanSemResponse
{Krishanu's talk on Stable homotopy and generalised homology}
A \emph{spectrum} $E$ is a sequence of pointed spaces $E_n$ with maps $SE_n\to E_{n+1}$. A \emph{CW-spectrum} is such in which the spaces are CW-complexes and the maps are CW-inclusions.

Let $E^*$ be a \emph{generalised cohomology theory}. That is, to each CW-pair and integer $n$ an abelian group, contravariant in the pair, with excision $E^n(X,A)=E^n(X-C,A-C)$, connecting maps $E^n(A)\to E^{n+1}(X,A)$ giving a long exact sequence, and a wedge axiom.
We'll define $\widetilde E^n(X)=E^n(X,\ast)$.

Let $E_n$ be the space representing $\widetilde E^n(\DASH)$. Then we have
\[E^n(X,\ast)\cong E^{n+1}(CX,X)\cong E^{n+1}(SX,CX)\cong E^{n+1}(SX,\ast)\]
Thus we have a natural isomorphism $[X,E_n]\to[SX,E_{n+1}]\cong [X,\Omega E_{n+1}]$, and by Yoneda's lemma, we have maps $E_n\to\Omega E_{n+1}$, which are homotopy equivalences. This gives an $\Omega$-spectrum, which can be made a CW-spectrum (i.e.\ with CW-inclusion structure maps) using a telescoping argument.

On the other hand, suppose we have a spectrum $E$, we get a cohomology theory by defining (for finite CW-complexes $X$ only):
\[E^n(X)=\varinjlim([X,E_n]\to[SX,E_{n+1}]\to\cdots)\]
We'll also define
\[\pi_r(E)=\varinjlim(\pi_{n+r}(E_n)).\]

Define a \emph{function} $E\to F$ to be the strictest possible notion of map --- levelwise commuting pointed maps. Call $E'\subset E$ dense (or cofinal) if every cell of $E$ is eventually in $E'$. A \emph{map} is an equivalence class of functions defined on cofinal subspectra.

Define $\Cyl(E)=E\wedge I_+$ (you can always smash with a space on the right, as it doesn't interfere with the suspensions). A map $\Cyl(E)\to F$ gives a homotopy of maps of spectra. A \emph{morphism} is a homotopy class of maps. A morphism of degree $n$ is a homotopy equivalence class of maps which lower degree by $n$.

We now define
\[E^n(X)=[X,E]_{-n}\text{ \ and \ }E_n(X)=[S,E\wedge X]_n.\]
Note that homology has a wedge axiom because $S$ is compact!
[By the way, this generalises my normal concept of singular homology --- try $E$ an Eilenberg-MacLane spectrum and $X$ a sphere to check this. In particular, smashing with an Eilenberg-MacLane spectrum turns homology to stable homotopy.]

I'd rather like a commutative, associative monoidal product, so that given $E\wedge F\to G$, we obtain a natural pairing $E^s(X)\otimes F^s(X)\to G^{r+s}(X)$ for any space $X$. In particular, given $E\wedge E\to E$ we get a product on cohomology. For example, we can use the maps $MO(n)\wedge MO(m)\to MO(n+m)$ to get a map $MO\wedge MO\to MO$.
\subsection*{Universal coefficients theorem}
Adams shows that cofibre sequences and fibre sequences coincide, and that smash products preserve cofibre sequences.
Given a free resolution $0\to R\to F\to G\to0$ of a group $G$, form the cofibre sequence $\bigvee S\to\bigvee S\to MG$, to get the Moore spectrum. Define $EG:=E\wedge MG$. 
\begin{thm*}[Universal coefficients]
There is a short exact sequence:
\[0\to E_n(F)\otimes G\to(EG)_n(F)\to\Tor^\Z(E_{n-1}(F),G)\to0\]
\end{thm*}
\begin{proof}
We get a long exact sequence:
\[\xymatrix{
[S,\bigvee_\alpha E\wedge F]_n\ar[r]\ar@{=}[d]&[S,\bigvee_\beta E\wedge F]_n\ar[r]&[S, EG\wedge F]_n\ar[r]&[S,\bigvee_\alpha E\wedge F]_{n-1}&\\%\ar[r]&[S,\bigvee_\beta E\wedge F]_{n-1}\\
\Z^A\otimes[S,E\wedge F]\ar[r]&\Z^B\otimes[S,E\wedge F]\ar@{=}[u]
}\]
Splitting this into short exact sequences gives the UCT.
\end{proof}
Consider now the case $G=\Q$ and $F=S$. Then there's no $\Tor$ term, and we get
\[\pi_*(E)\otimes\Q\weakequiv\pi_n(E\Q).\]
Consider the map $S\to H$ representing a generator of $\pi_0(H)=\Z$. This gives a commutative diagram
\[\xymatrix{
\ar[d]\pi_*(S)\otimes\Q\ar[r]&\pi_n(S\Q)\ar[d]\\
\pi_*(H)\otimes\Q\ar[r]&\pi_n(H\Q)
}\]
But $\pi_n(S)\otimes\Q=0$ for $n\neq0$, so the left vertical is an iso. The horizontals are isomorphisms by the UCT, thus the right vertical is an iso.

\noindent[Haynes: can always write $E=\colim \Sigma^{-n}E_n$...

There are maps of spectra which induce zero on every homology theory. Call these `phantoms'. This doesn't happen for cohomology theories.]

\pagebreak
\end{KrishanuAdamsStableHtpy}
\begin{JeremyQuillenPractice}
\KanSemResponse
{Jeremy's Practice talk on Quillen}
We showed that for a compact Lie group $G$, $K(G)=\widehat{R(G)}$, an object which is purely algebraic.
\begin{question}
Can we describe $H^*(BG,\Z/p\Z)$ algebraically?
\end{question}
We'll assume these coefficients from now on.
\begin{conjecture}[Atiyah-Swan]
The Krull dimension of $H^*(BG)_{red}$ is equal to the maximal rank of an elementary abelian $p$-subgroup of $G$.
\end{conjecture}
\noindent Recall that an elementary abelian $p$-subgroup of rank $r$ is a subgroup of the form $(\Z/p\Z)^r$.

By analogy with the approach of Atiyah and Segal, instead of studying $H^*(BG)$, we should look at 
\[H^*_G(X):=H^*(EG\times_G X)\text{ for a $G$-space $X$}.\]
Then we're interested in $H^*_G(\ast)=H^*(BG)$, which we abbreviate as $H_G^*$.

Suppose that $u:G\to G'$ is a homomorphism of compact Lie groups and $f:X\to X'$ is $u$-equivariant, so that $f(gx)=u(g)f(x)$. Now we can draw a diagram:
\[\xymatrix{
(EG\times EG')\times_G X\ar[r]^{\ \ \ p_2}\ar[d]_{p_1}^{\text{$H^*$-iso}}&EG'\times_GX\ar[d]_{\overline f}\\
EG\times_G X&EG'\times_{G'}X'
}\]
$p_1$ has contractible fibre $EG'$, and so there is a natural map $H^*_{G'}(X')\to H^*_{G}(X)$. Suppose moreover that $v:EG\to EG'$ is $u$-equivariant, we have a map ${\overline{v\times f}}$ as follows:
\[\xymatrix{
(EG\times EG')\times_G X\ar[r]^{\ \ \ p_2}\ar[d]_{p_1}&EG'\times_GX\ar[d]_{\overline f}\\
\ar@/_1ex/[u]_{\text{`use $v$'}}\ar@{-->}[r]^{\overline{v\times f}}EG\times_G X&EG'\times_{G'}X'
}\]
In this diagram, we use $v$ to obtain a section of $p_1$, which must also be a cohomology isomorphism. Then the two paths from $EG\times_G X$ to $EG'\times_{G'}X'$ are equal, so that $\overline{v\times f}^*:H^*_{G'}(X')\to H^*_{G}(X)$ is the same map as we described earlier.
\subsection*{Conjugation invariance}
Fix $g_0\in G$, and suppose that $u:G\to G$ is given by $g\mapsto g_0gg_0^{-1}$. For any $G$-space $X$ we have a $u$-equivariant map $f:X\to X$ given by $x\mapsto g_0x$. We can contemplate the induced map $H^*_G(X)\to H^*_G(X)$. We claim that it is the identity.

To see this, note that we have $v:EG\to EG$ given by $p\mapsto g_0p$. Now the map $v\times f:(p,x)\mapsto(g_0p,g_0x)$ has $\overline{v\times f}$ \emph{equal} to the identity, so that so that conjugation has no effect on equivariant cohomology.
\subsection*{The Leray Spectral Sequence}
We have a fibre bundle
\[\xymatrix{
X\ar[r]&EG\times_G X\ar[d]_p\\
&BG
}\]
Unfortunately, the Serre spectral sequence isn't all that useful in general. Instead, use the map:
\[\xymatrix{
EG\times_G X\ar[r]_{\ \ \ p_2}&X/G
}\]
The fibre at $y\in X/G$ is $EG\times_G\calO$, where $\calO$ is the orbit of $y$, which looks radically different for varying $y$, so we need something more robust! We'll use the Leray spectral sequence for sheaf cohomology.
\[E_2^{st}=H^s(X/G,\scrH_G^t)\implies H_G^{s+t}(X).\]
The sheaf $\scrH_G^t$ is the sheafification of
\[U\mapsto H^t(\pi^{-1}(U))\]
The stalk at $y\in X/G$ is just $H^t_G(\calO)$.
\begin{exmp*}
If $G$ acts freely on $X$, then $H_G^*(\calO)=H_G^*(G)=H^t(EG)=0$ for $i\neq0$, so that the stalks are zero, and the sheaves are zero. Thus $H^*(X/G)=H_G^*(X)$.
\end{exmp*}
\subsection*{Nilpotence of elements of positive filtration}
We would like a setting in which we can say something interesting about:
\[E_2^{st}=H^s(X/G,\scrH_G^t)\implies H_G^{s+t}(X).\]
Suppose that $X$ is a $G$-CW-complex with bounded dimension. Anything with positive filtration, cup-product it with itself enough and it'll die. So everything with positive filtration will die! We normally look at the edge hom
\[\phi:H^*_G(X)\to E^{0*}_2.\]
If $s\in E_r^{0*}$, then $s^p\in E_{r+1}^{0*}$, as automatically, $d_r(s^p)=ps^{p-1} d_r(s)=0$. This means that for any $s\in E_2^{0*}$, $s^{p^d}\in E_\infty^{0*}$. This is good, as we're only interested in the Krull dimension, so we should view any element whose power lives in the image as good enough.

Thus, the edge hom $\phi:H^*_G(X)\to H^0(X/G,\scrH^*_G)$ is an $F$-isomorphism. That is, $\ker\phi$ is nilpotent, and all $s\in H^0(X/G,\scrH^*_G)$ has some power $s^{p^d}\in\im\phi$.

Quillen's big result here looks like...
\[H^*_G(X)\overset{Fiso}{\to} H^0(X/G,\scrH_G^t)\to \calA_G^*(X)\]
where $\calA_G^*(X)$ is the ring of compatible families of functions $f_A:X^A\to H^*_A$ where $A$ runs over all elementary abelian $p$-subgroups of $G$.

What is the map $H^0(X/G,\scrH_G^t)\to \calA_G^*(X)$? Choose $s\in H^0(X/G,\scrH_G^t)$. We give $s_A:X^A\to H_A^t$ by mapping $x$ (whose image in $X/G$ is $\overline x$) to the image of the value $s(\overline x)$ of $s$ in the stalk at $\overline x$ under the map $H^t_G(GX)\to H^t_A$ induced by $A\subset G$ and $\{x\}\to G_x$.

The compatibility conditions are as follows. Suppose that $A,A'$ are elementary $p$-subs and $g\in G$ are such that $g^{-1}Ag\subset A'$. Then $s_A(ga')=\theta^*s_{A'}(a')$ where $\theta$ sends $k\in A$ to $g^{-1}kg\in A'$. These are the conditions on a compatible family.
\begin{prop*}
If $X$ is `locally nice', $H^0(X/G,\scrH^*_G)\to\calA_G^*(X)$ (a map of anti-commutative $\Z/p\Z$-algebras) is and isomorphism if every isotropy group is an elementary $p$-subgroup.
\end{prop*}
\begin{proof}
Interpret $\calA_G^*(X)$ as global sections of a sheaf on $X/G$. Somehow get something like...
\[H_G^t(Gx)\to H^t_{G_x}.\]
\end{proof}
Now embed $G$ in $U(n)$, and let $F$ denote $U(n)/S$ where $S$ is the group generated by all elements of order dividing $p$


\pagebreak
\end{JeremyQuillenPractice}
\begin{QuillenSpectrumOfEqCohomology}
\KanSemResponse
{``The spectrum of an equivariant cohomology ring, I'' --- Quillen}
Just as I found myself drowning in a sea of work last weekend, I find myself drowning in a sea of words when trying to read this paper. I'll do my best.

I suppose that the title is what it is because when one calculates $H^*_G(X)$ up to $F$-isomorphism, one obtains the topological space underlying $\spec H^*_G(X)$ completely.

So the first thing which I need to learn more about is the Leray spectral sequence. It seems to be an instance of the Grothendieck spectral sequence for computing the derived functor of a composite functor, of which I saw a description of a while ago. In particular, I should be able to proceed with confidence. However, I'd rather like to understand the sheaf $\scrH_G^t$ on $X/G$. This is the image of the constant sheaf under the $t^\text{th}$ derived functor of the map $(p_2)_*:\Shf(EG\times_G X)\to \Shf(X/G)$ induced by $p_2:EG\times_G X\to X/G$.

(Apparently,) this is the sheaf associated to the presheaf $U\mapsto H^t(p_2^{-1}(U))$ on $X/G$. This is quite a strange presheaf. It takes an open subset of $X/G$ to the cohomology of a corresponding open subset of the homotopy orbits! For why? Perhaps I'm just being too optimistic, expecting a way to understand this.

We see in the next section another use of an embedding of $G$ in a unitary group. This time we're trying to show that if $H^*(X)$ is finitely generated over a fixed noetherian coefficient ring $\Lambda$, then so is $H^*_G(X)$. This will follow by showing that $H^*_G(X)$ is finite over  $H^*_{U(n)}(\ast)$, where $U(n)$ is a unitary group in which we've embedded $G$, as the cohomology of $BU(n)$ is well understood. This is shown by an application of the Leray spectral sequence, once it is observed that $H^*_G(X)$ coincides with $H^*_{U(n)}(U(n)\times_G X)$. It seems, after seeing a little of this type of material recently in the Atiyah-Segal paper, that these methods are fairly ubiquitous in the equivariant setting.

I would like to learn more and to write more tonight, but must be realistic. I think that things will become clearer when Jeremy speaks about this tomorrow.

\pagebreak
\end{QuillenSpectrumOfEqCohomology}
\begin{JeremyQuillen}
\KanSemResponse
{Jeremy's talk on Quillen}
We showed that for a compact Lie group $G$, $K(G)=\widehat{R(G)}$, an object which is purely algebraic.
\begin{question}
Can we describe $H^*(BG,\Z/p\Z)$ algebraically?
\end{question}
We'll assume these coefficients from now on.
\begin{conjecture}[Atiyah-Swan]
The Krull dimension of $H^*(BG)_{red}$ is equal to the maximal rank of an elementary abelian $p$-subgroup of $G$.
\end{conjecture}
\noindent Recall that an elementary abelian $p$-subgroup of rank $r$ is a subgroup of the form $(\Z/p\Z)^r$.

By analogy with the approach of Atiyah and Segal, instead of studying $H^*(BG)$, we should look at 
\[H^*_G(X):=H^*(EG\times_G X)\text{ for a $G$-space $X$}.\]
[This is the default definition of an equivariant cohomology thoery given a normal cohomology theory. For $K$-theory, you can do better, and you get a completion on the left...]
Then we're interested in $H^*_G(\ast)=H^*(BG)$, which we abbreviate as $H_G^*$ following Quillen.
\subsection*{Some elementary remarks on equivariant cohomology}
Suppose that $u:G\to G'$ is a homomorphism of compact Lie groups and $f:X\to X'$ is $u$-equivariant, so that $f(gx)=u(g)f(x)$. Now we can draw a diagram:
\[\xymatrix{
(EG\times EG')\times_G X\ar[r]^{\ \ \ p_2}\ar[d]_{p_1}^{\text{$H^*$-iso}}&EG'\times_GX\ar[d]_{\overline f}\\
EG\times_G X&EG'\times_{G'}X'
}\]
$p_1$ has contractible fibre $EG'$, and so there is a natural map $H^*_{G'}(X')\to H^*_{G}(X)$. Suppose moreover that $v:EG\to EG'$ is $u$-equivariant, we have a map ${\overline{v\times f}}$ as follows:
\[\xymatrix{
(EG\times EG')\times_G X\ar[r]^{\ \ \ p_2}\ar[d]_{p_1}&EG'\times_GX\ar[d]_{\overline f}\\
\ar@/_1ex/[u]_{\text{`use $v$'}}\ar@{-->}[r]^{\overline{v\times f}}EG\times_G X&EG'\times_{G'}X'
}\]
In this diagram, we use $v$ to obtain a section of $p_1$, which must also be a cohomology isomorphism. Then the two paths from $EG\times_G X$ to $EG'\times_{G'}X'$ are equal, so that $\overline{v\times f}^*:H^*_{G'}(X')\to H^*_{G}(X)$ is the same map as we described earlier.
\subsection*{Conjugation invariance}
Fix $g_0\in G$, and suppose that $u:G\to G$ is given by $g\mapsto g_0gg_0^{-1}$. For any $G$-space $X$ we have a $u$-equivariant map $f:X\to X$ given by $x\mapsto g_0x$. We can contemplate the induced map $H^*_G(X)\to H^*_G(X)$. We claim that it is the identity.

To see this, note that we have $v:EG\to EG$ given by $p\mapsto g_0p$. Now the map $v\times f:(p,x)\mapsto(g_0p,g_0x)$ has $\overline{v\times f}$ \emph{equal} to the identity, so that so that conjugation has no effect on equivariant cohomology.
\subsection*{The Leray Spectral Sequence}
We have a fibre bundle
\[\xymatrix{
X\ar[r]&EG\times_G X\ar[d]_p\\
&BG
}\]
Unfortunately, the Serre spectral sequence isn't all that useful in general. Instead, we'll use the map:
\[\xymatrix{
EG\times_G X\ar[r]_{\ \ \ p_2}&X/G
}\]
The fibre at $y\in X/G$ is $EG\times_G\calO$, where $\calO$ is the orbit of $y$, which looks radically different for varying $y$, so we need something more robust! We'll use the Leray spectral sequence for sheaf cohomology.
\[E_2^{st}=H^s(X/G,\scrH_G^t)\implies H_G^{s+t}(X).\]
The sheaf $\scrH_G^t$ is the sheafification of
\[U\mapsto H^t(\pi^{-1}(U))\]
The stalk at $y\in X/G$ is just $H^t_G(\calO)$. [Which is also $H^t(BG_x)$, with $G_x$ the stabiliser of $x$, a preimage of $y$.]
\begin{exmp*}
If $G$ acts freely on $X$, then $H_G^*(\calO)=H_G^*(G)=H^t(EG)=0$ for $i\neq0$, so that the stalks are zero, and the sheaves are zero. Thus $H^*(X/G)=H_G^*(X)$.
\end{exmp*}
\subsection*{Nilpotence of elements of positive filtration}
We would like a setting in which we can say something interesting about:
\[E_2^{st}=H^s(X/G,\scrH_G^t)\implies H_G^{s+t}(X).\]
Suppose that $X$ is a $G$-CW-complex with bounded dimension. [Actually, $X$ anything of bounded topological dimension.] Anything with positive filtration, cup-product it with itself enough and it'll die. So everything with positive filtration will die!

Now if $s\in E_r^{0*}$, then $s^p\in E_{r+1}^{0*}$, as automatically, $d_r(s^p)=ps^{p-1} d_r(s)=0$. This means that for any $s\in E_2^{0*}$, $s^{p^d}\in E_\infty^{0*}$. This is good, as we're only interested in the Krull dimension, so we should view any element whose power lives in the image as good enough.

Thus, the edge homomorphism $\phi:H^*_G(X)\to H^0(X/G,\scrH^*_G)$ is an $F$-isomorphism. That is, $\ker\phi$ consists only of nilpotent elements, and all $s\in H^0(X/G,\scrH^*_G)$ have some power $s^{p^d}\in\im\phi$.

This is good, as $F$-isomorphism preserves the set of prime ideals.
\subsection*{Onwards!}
Quillen's result here is that the following composite is an $F$-isomorphism:
\[H^*_G(X)\overset{Fiso}{\to} H^0(X/G,\scrH_G^t)\to\calA_G^*(X)\]
where $\calA_G^*(X)$ is the ring of `compatible families' of locally constant functions $f_A:X^A\to H^*_A$ where $A$ runs over all elementary abelian $p$-subgroups of $G$.

The compatibility conditions are as follows. Suppose that $A,A'$ are elementary $p$-subgroups and $g\in G$ are such that $g^{-1}Ag\subset A'$. Then $s_A(ga')=\theta^*s_{A'}(a')$ where $\theta$ sends $k\in A$ to $g^{-1}kg\in A'$, ($\theta^*:H^t_{A'}\to H^t_A$). These are the conditions on a compatible family.

What is the map $H^0(X/G,\scrH_G^t)\to \calA_G^*(X)$? Choose a section $s\in H^0(X/G,\scrH_G^t)$. We give $s_A:X^A\to H_A^t$ by mapping $x$ (whose image in $X/G$ is $\overline x$) to the image of the value $s(\overline x)$ of $s$ in the stalk at $\overline x$ under the map $H^t_G(Gx)\to H^t_A$ induced by $A\subset G$ and $\{x\}\cofibration Gx$.
\subsection*{An example}
$H^*(BO(n),\Z/p\Z)$ has Krull dimension $n$. We should compare this to that of $\calA_{O(n)}^*(pt)$, which is the set of maps $pt\to H^*_A$. There's a natural elementary abelian 2-subgroup $S$. 
\begin{claim}
Every $A\subset O(n)$ is subconjugate to $S$.
\end{claim}
\begin{proof}
Any $a\in A$ has order $2$, so that $a^T=a^{-1}=a$, so that $a$ is diagonalisable in $O(n)$. As this is abelian, all the elements of $A$ are simultaneously diagonalisable.
\end{proof}
Thus $\calA^*_{O(n)}\simeq (H^*_{(\Z/2\Z)^n})^{\Sigma_n}$, which has the correct Krull dimension.
\subsection*{The proof}
\begin{prop*}
If $X$ is `locally nice', $H^0(X/G,\scrH^*_G)\to\calA_G^*(X)$ (a map of anti-commutative $\Z/p\Z$-algebras) is a \emph{genuine} isomorphism if every isotropy group of $X$ is an elementary abelian $p$-subgroup.
\end{prop*}
\begin{proof}
Given $S_A:X^A\to H^t_A$, we can take $\overline x\in X/G$, and lift it to $x\in X$, and apply $s_{G_x}$ to obtain an element of $H^t_{G_x}$, which is the stalk of $\scrH^t_G$. Quillen checks that these glue to give a genuine section, giving an inverse.
\end{proof}
We now have a sequence
\[H^*_G(X)\overset{Fiso}{\to} H^0(X/G,\scrH_G^t)\to\calA_G^*(X)\]
and we wish to show that the composite is an $F$-iso.

Now embed $G$ in $U(n)$, and let $F$ denote $U(n)/S$ where $S$ is the group generated by all elements of order dividing $p$. Now the idea is
\[\xymatrix{
 H^*_G(X)\ar[d]\ar[r]&\ar[d] H^*_G(X\times F)\ar@<.35ex>[r]\ar@<-.35ex>[r]&\ar[d] H^*_G(X\times F\times F)\\
 \calA^*_G(X)\ar[r]& \calA^*_G(X\times F)\ar@<.35ex>[r]\ar@<-.35ex>[r]& \calA^*_G(X\times F\times F)
}\]
Now isotropy subgroups of $X\times F$ are automatically elementary $p$-subgroups. So we get $F$-isos in the right two columns (being the composite of an $F$-iso and a genuine composite). Moreover, Quillen checks that the rows are exact, so that the first row is an $F$-iso, and we're done.

\pagebreak
\end{JeremyQuillen}
\begin{JandrGammaPractice}
\KanSemResponse
{Jandr's practice talk on $\Gamma$-categories}
The neat idea --- from a category with coproducts, one can construct a spectrum.

there are three ways to motivate the constructions. Let's talk about the first. Let $X$ be a set. Then I can define the free monoid on $S$, $\coprod X^n/\Sigma_n$. Similarly, we could ask for the free commutative algebra on a vectorspace $V$.

Given vector spaces, I can embed them in a homotopy theory (chain complexes) and then instead take homotopy orbits, so I might take, for $X$ a space:
\[M:=\coprod E\Sigma_n\times X^n/\Sigma_n.\]
[This is the underlying space of a $\Gamma$-space, and we'll see what the multiplication should be later, but you can see it derectly...] This is so good, that it's $A_\infty$. That is, the associahedrons can be filled for all multiplications.

\subsection*{$\Gamma$-spaces}
$\Gamma$ is the category whose objects are finite sets, and whose morphisms from $S\to T$ are functions on power sets $\calP(S)\to\calP(T)$ preserving disjoint unions. [$\Gamma^\op$ is actually just pointed finite sets!]

A $\Gamma$-space is a contravariant functor $\Gamma^\op\to\Top$, with
\begin{enumerate}\squishlist
\item $A(\underline{0})\simeq*$
\item $A(n)\to A(1)^n$ is a homotopy equivalence, being induced by $i_k:1\to n$ sending $\{1\}\mapsto\{k\}$.
\end{enumerate}
$A(1)$ is supposed to be the `underlying space' of a category-like thing.

The composition law is defined up to homotopy, via
\[\xymatrix{
%a1& a2& a3& a4& a5& a6\\
%b1& b2& b3& b4& b5& b6\\
\ar[d]^{\simeq} A(2)\ar[r]^{\mu^*}& A(1)& c3& c4& 1\ar[r]& 2\\
 A(1)^2& d2& d3& d4& \{1\}\ar@{|->}[r]& \{1,2\}\\
%e1& e2& e3& e4& e5& e6\\
%f1& f2& f3& f4& f5& f6\\
}\]
The associativity is encoded by the square
\[\xymatrix{
%a1& a2& a3& a4& a5& a6\\
%b1& b2& b3& b4& b5& b6\\
 1\ar[d]\ar[r]&2\ar[d]\\
 2\ar[r]& 3
%e1& e2& e3& e4& e5& e6\\
%f1& f2& f3& f4& f5& f6\\
}\]
Now let $M$ be a topological abelian monoid. Let $A(S)=M^S$, and given $\theta\in\Gamma(S,T)$, write
\[(\theta^*m)_s=\sum_{t\in \theta(s)}m_t.\]
This gives a $\Gamma$-space.

Now there's a functor $\Delta\to\Gamma$. $\Delta$ stores all the associativity formulae, while $\Gamma$ stores all commutativity and associativity formulae. The functor sends
$[m]\mapsto \underline{m}$, and $\{i\}\mapsto \{j\in\underline{n}:f(i-1)<j\leq f(i)\}$ ??

$\partial_j$ corresponds to multiplying the $j$ and $j+1$th element.
\subsection*{Realisation and classifying spaces}
Given $A:\Gamma^\op\to\Top$ a $\Gamma$-space, $|A|$ means the goemetric realisation as a simplicial space.

Have
\[\Gamma^\op\times\Gamma^\op\to\Gamma^\op,\text{ gvien by }(S,T)\mapsto S\times T\]
Given a $\Gamma$-space $A$, get $\Gamma^\op\to\Top$ given by $S \mapsto |T\mapsto A(S\times T)|$, and we call this the classifying space $BA$, another $\Gamma$-space.

$A(1)$, $BA(1)$, \ldots is a spectrum called $\mathbb{B}A$. We need structure maps, in particular a map $\Sigma A(1)\to BA(1)=|A|$. Now in fact, $A(0)\subset A(1)$ as a retract (using the unique map $1\to 0$), so we can choose any basepoint in $A(0)$, which is contractible. Some more comments happened in order to construct this map, and we have a functor
\[\mathbb{B}:\Gamma\Top\to\scrS.\]
\begin{prop*}
If $A(1)$ is $k$-connected, then $BA(1)$ is $(k+1)$-connected. If $A(1)$ has a homotopy inverse (i.e.\ $A$ is group like, \Iff $\pi_1A$ is a group, since some technical condition holds), then $A(1)\weakequiv \Omega BA(1)$.
\end{prop*}
\noindent In particular, $\pi_0(B^nA(1))$ is a group for $n\geq1$, so that $B^nA(1)\weakequiv \Omega B^{n+1}A(1)$ is a weak equivalence, so that we almost get an $\Omega$-spectrum --- the only problem is the first map. Moreover, $A(1)\to\Omega BA(1)$ is the group completion (a general principle --- take $B$ then loops of a monoid).
\subsection*{$\Gamma$-categories}
A $\Gamma$-category is a functor $\calC:\Gamma^{\op}\to\Top\Cat$ ($\Top\Cat$ being the category of topological categories) satisfying the same two axioms.
\begin{cor*}
The functor $S\mapsto|\calC(S)|$ is a $\Gamma$-space.
\end{cor*}
If $\calC$ is a category with coproducts, one might hope to get a map $\coprod:\calC\times\calC\to\calC$. Given all the choices, you only get commutativity and associativity up to coherent isomorphisms. The trick is, if $S$ is a finite set, to let $\calC(S)$ be the category whose objects are coproduct-preserving functors $\calP(S)\to\calC$ (whiere the power set is a poset with inclusions the maps), and whose morphisms are isomorphisms of functors, i.e.\ natural equivalences.

An object in $\calC(2)$ is $A\rightarrow C\leftarrow B$ where this is a coproduct diagram. I've got a map $\calC(2)\to \calC(1)$, which sends this to $\calC$. I've also got the map $\calC(2)\to\calC(1)^2$, which returns $(A,B)$. We've gotten
\funcdef{\Gamma^\op}{\Cat}{S}{\calC(S)}
This is a $\Gamma$-category, so we get a $\Gamma$-space, and thus a spectrum!
\begin{exmp*}
Let $G$ be a topological abelian group. Actually, make it strictly an abelian group. I have a $\Gamma$-space $A(S)=G^S$. Then $\mathbb{B}A=HG$, as $A(1)=G$, so that $\Omega BA(1)=G$, etc...
\end{exmp*}
\begin{exmp*}[Finite sets]
Let the category of finite sets be $f\Set$.
Choose a model with one object for each $n\geq0$. Then coproducts are unique. Then $F\Set(1)=\coprod\Sigma_n$ of the symmetric groups viewed as categories. Then $|F\Set(1)|=\coprod  B\Sigma_n$, which is a topological monoid (maybe it strictifies). Let's call the associated $\Gamma$-space $B\Sigma$ (assoc to the cat with coprods). 
\end{exmp*}
\begin{exmp*}
Let $\calC$ be the category of finitely generated projective $R$-modules. Get a spectrum $K_R=\pi_*(K_R)$, algebraic $K$-theory.
\end{exmp*}
I'd like to construct a `free' functor $F:\Top\to\Gamma\Top$. It turns out that $FX(1)=|\calC_X(1)|=\coprod E\Sigma_n\times X^n/\Sigma^n$.

There's a functor $\mathbb{A}:\scrS\to\Gamma\Top$, such that
\[\mathbb{A}X(T)=\scrS(S^T,X)\]
This is a $\Gamma$-space, since
\[AX(n)=\scrS(S^{\times n},X)\simeq \scrS(S^{\vee n},X)=\scrS(S,X)^n=AX(1)^n.\]
\begin{thm*}
There's an adjunction $B:\Gamma\Top\leftrightarrow \scrS:A$.
\begin{enumerate}\squishlist
\item $BA$ is connective for all $A$;
\item $AX$ is grouplike for all $X$;
\item $A\to \mathbb{A}\mathbb{B}A$ is an equivalence (of $\Gamma$-spaces) for all grouplike $A$;
\item$\mathbb{B}\mathbb{A}X\to X$ is an equivalence for all connective spectra $X$.
\end{enumerate}
\end{thm*}


\pagebreak
\end{JandrGammaPractice}
\begin{JandrGammaSpaces}
\KanSemResponse
{Jandr's talk on $\Gamma$-categories}
The neat idea --- from a category with coproducts, one can construct a spectrum.


\subsection*{$\Gamma$-categories}
A $\Gamma$-category is a functor $\calC:\Gamma^{\op}\to\Top\Cat$ ($\Top\Cat$ being the category of topological categories) satisfying the axioms:
\begin{enumerate}\squishlist
\item $A(\underline{0})\simeq*$
\item $A(n)\to A(1)^n$ is a homotopy equivalence, being induced by $i_k:1\to n$ sending $\{1\}\mapsto\{k\}$.
\end{enumerate}




there are three ways to motivate the constructions. Let's talk about the first. Let $X$ be a set. Then I can define the free monoid on $S$, $\coprod X^n/\Sigma_n$. Similarly, we could ask for the free commutative algebra on a vectorspace $V$.

Given vector spaces, I can embed them in a homotopy theory (chain complexes) and then instead take homotopy orbits, so I might take, for $X$ a space:
\[M:=\coprod E\Sigma_n\times X^n/\Sigma_n.\]
[This is the underlying space of a $\Gamma$-space, and we'll see what the multiplication should be later, but you can see it derectly...] This is so good, that it's $A_\infty$. That is, the associahedrons can be filled for all multiplications.

\subsection*{$\Gamma$-spaces}
$\Gamma$ is the category whose objects are finite sets, and whose morphisms from $S\to T$ are functions on power sets $\calP(S)\to\calP(T)$ preserving disjoint unions. [$\Gamma^\op$ is actually just pointed finite sets!]

A $\Gamma$-space is a contravariant functor $\Gamma^\op\to\Top$, with
\begin{enumerate}\squishlist
\item $A(\underline{0})\simeq*$
\item $A(n)\to A(1)^n$ is a homotopy equivalence, being induced by $i_k:1\to n$ sending $\{1\}\mapsto\{k\}$.
\end{enumerate}
$A(1)$ is supposed to be the `underlying space' of a category-like thing.

The composition law is defined up to homotopy, via
\[\xymatrix{
%a1& a2& a3& a4& a5& a6\\
%b1& b2& b3& b4& b5& b6\\
\ar[d]^{\simeq} A(2)\ar[r]^{\mu^*}& A(1)& c3& c4& 1\ar[r]& 2\\
 A(1)^2& d2& d3& d4& \{1\}\ar@{|->}[r]& \{1,2\}\\
%e1& e2& e3& e4& e5& e6\\
%f1& f2& f3& f4& f5& f6\\
}\]
The associativity is encoded by the square
\[\xymatrix{
%a1& a2& a3& a4& a5& a6\\
%b1& b2& b3& b4& b5& b6\\
 1\ar[d]\ar[r]&2\ar[d]\\
 2\ar[r]& 3
%e1& e2& e3& e4& e5& e6\\
%f1& f2& f3& f4& f5& f6\\
}\]
Now let $M$ be a topological abelian monoid. Let $A(S)=M^S$, and given $\theta\in\Gamma(S,T)$, write
\[(\theta^*m)_s=\sum_{t\in \theta(s)}m_t.\]
This gives a $\Gamma$-space.

Now there's a functor $\Delta\to\Gamma$ which is an inclusion except at the  very bottom, where the two face maps get identified. $\Delta$ stores all the associativity formulae, while $\Gamma$ stores all commutativity and associativity formulae. The functor sends
$[m]\mapsto \underline{m}$, and a map $f:[m]\to[n]$ to $\{i\}\mapsto \{j\in\underline{n}:f(i-1)<j\leq f(i)\}$.

$\partial_j$ corresponds to multiplying the $j$ and $j+1$th element.
\subsection*{Realisation and classifying spaces}
Given $A:\Gamma^\op\to\Top$ a $\Gamma$-space, $|A|$ means the goemetric realisation as a simplicial space.

Have
\[\Gamma^\op\times\Gamma^\op\to\Gamma^\op,\text{ gvien by }(S,T)\mapsto S\times T\]
Given a $\Gamma$-space $A$, get $\Gamma^\op\to\Top$ given by $S \mapsto |T\mapsto A(S\times T)|$, and we call this the classifying space $BA$, another $\Gamma$-space.

$A(1)$, $BA(1)$, \ldots is a spectrum called $\mathbb{B}A$. We need structure maps, in particular a map $\Sigma A(1)\to BA(1)=|A|$. Now in fact, $A(0)\subset A(1)$ as a retract (using the unique map $1\to 0$), so we can choose any basepoint in $A(0)$, which is contractible. Some more comments happened in order to construct this map, and we have a functor
\[\mathbb{B}:\Gamma\Top\to\scrS.\]
\begin{prop*}
If $A(1)$ is $k$-connected, then $BA(1)$ is $(k+1)$-connected. If $A(1)$ has a homotopy inverse (i.e.\ $A$ is group like, \Iff $\pi_1A$ is a group, since some technical condition holds), then $A(1)\weakequiv \Omega BA(1)$.
\end{prop*}
\noindent In particular, $\pi_0(B^nA(1))$ is a group for $n\geq1$, so that $B^nA(1)\weakequiv \Omega B^{n+1}A(1)$ is a weak equivalence, so that we almost get an $\Omega$-spectrum --- the only problem is the first map. Moreover, $A(1)\to\Omega BA(1)$ is the group completion (a general principle --- take $B$ then loops of a monoid).

\begin{cor*}
The functor $S\mapsto|\calC(S)|$ is a $\Gamma$-space.
\end{cor*}
If $\calC$ is a category with coproducts, one might hope to get a map $\coprod:\calC\times\calC\to\calC$. Given all the choices, you only get commutativity and associativity up to coherent isomorphisms. The trick is, if $S$ is a finite set, to let $\calC(S)$ be the category whose objects are coproduct-preserving functors $\calP(S)\to\calC$ (whiere the power set is a poset with inclusions the maps), and whose morphisms are isomorphisms of functors, i.e.\ natural equivalences.

An object in $\calC(2)$ is $A\rightarrow C\leftarrow B$ where this is a coproduct diagram. I've got a map $\calC(2)\to \calC(1)$, which sends this to $\calC$. I've also got the map $\calC(2)\to\calC(1)^2$, which returns $(A,B)$. We've gotten
\funcdef{\Gamma^\op}{\Cat}{S}{\calC(S)}
This is a $\Gamma$-category, so we get a $\Gamma$-space, and thus a spectrum!
\begin{exmp*}
Let $G$ be a topological abelian group. Actually, make it strictly an abelian group. I have a $\Gamma$-space $A(S)=G^S$. Then $\mathbb{B}A=HG$, as $A(1)=G$, so that $\Omega BA(1)=G$, etc...
\end{exmp*}
\begin{exmp*}[Finite sets]
Let the category of finite sets be $f\Set$.
Choose a model with one object for each $n\geq0$. Then coproducts are unique. Then $F\Set(1)=\coprod\Sigma_n$ of the symmetric groups viewed as categories. Then $|F\Set(1)|=\coprod  B\Sigma_n$, which is a topological monoid (maybe it strictifies). Let's call the associated $\Gamma$-space $B\Sigma$ (assoc to the cat with coprods). 
\end{exmp*}
\begin{exmp*}
Let $\calC$ be the category of finitely generated projective $R$-modules. Get a spectrum $K_R=\pi_*(K_R)$, algebraic $K$-theory.
\end{exmp*}
I'd like to construct a `free' functor $F:\Top\to\Gamma\Top$. It turns out that $FX(1)=|\calC_X(1)|=\coprod E\Sigma_n\times X^n/\Sigma^n$.

There's a functor $\mathbb{A}:\scrS\to\Gamma\Top$, such that
\[\mathbb{A}X(T)=\scrS(S^T,X)\]
This is a $\Gamma$-space, since
\[AX(n)=\scrS(S^{\times n},X)\simeq \scrS(S^{\vee n},X)=\scrS(S,X)^n=AX(1)^n.\]
\begin{thm*}
There's an adjunction $B:\Gamma\Top\leftrightarrow \scrS:A$.
\begin{enumerate}\squishlist
\item $BA$ is connective for all $A$;
\item $AX$ is grouplike for all $X$;
\item $A\to \mathbb{A}\mathbb{B}A$ is an equivalence (of $\Gamma$-spaces) for all grouplike $A$;
\item$\mathbb{B}\mathbb{A}X\to X$ is an equivalence for all connective spectra $X$.
\end{enumerate}
\end{thm*}


\pagebreak
\end{JandrGammaSpaces}
\begin{SegalCategoriesAndCohomologyTheories}
\KanSemResponse
{``Categories and Cohomology Theories'' --- G.\ Segal}
I've read as must of this paper as I can get through in a short time, and have enjoyed it. It's nice to read a paper which is less `computational' and more `conceptual', after some of the more difficult reads of late. Perhaps you'd disagree with my choice of `words'.

Obviously, everything in the paper fits together very well indeed. My difficulty at the present time is then, as always, to figure out how the definitions can be motivated. The realisation of a $\Gamma$-space is not too difficult, I suppose. It's as if we were gluing copies of $A(1)^n\times \Delta^n$ together along the maps given by multiplication, were $A(1)$ a monoid. If this were exactly what was being done, we'd end up with a space whose construction mimics that of the classifying space of the monoid via the bar construction. Instead, we use $A(n)$, a fatter version, and turn out something which is \emph{like} the bar construction for the space $A(1)$, but with some extra flab corresponding to the excess of $A(n)$ over $A(1)^n$.

After saying this, it becomes easier to interpret the formula for the classifying space of a $\Gamma$-space $A$ ($BA$ is the $\Gamma$-space whose terms are $BA(S):=|T\mapsto A(S\times T)|$, for $S$ a finite set). Then $BA(1)$ is just the realisation of $A$, and the rest of the terms ($|S|>1$) are pretty relaxed about the process. Perhaps when one becomes deeply comfortable with the bar construction, and is thinking about de-strictifying the notion of monoid, this is actually not such a difficult definition to give.

At this point, Segal notes that there's a natural map $SA(1)\to BA(1)$, which must correspond to the map $M\to\Omega BM$ when $M$ is a monoid. All this makes me wish I had more time at present to think hard about the strict case, but alas, I do not.

Anyway, I'm getting a feel for these $\Gamma$-spaces fairly quickly, I suppose because they're not so difficult. I'll get a lot out of Michael's repeat performance tomorrow, and will pick this up again after quals. As I don't currently have much to say that's coherent, I'll stop here for now.
%One notes that the `$\pi_0$ problem' that you explained in relation to the theory of operads is present here, too. The spectrum associated with a $\Gamma$-space $A$ is only an $\Omega$-spectrum \Iff $A(1)$ has a homotopy inverse --- it is grouplike.



%Note that the $\pi_1$ problem is exactly summarised... you get an $\Omega$-spectrum iff its grouplikey, else you get a completion type thingo......
\pagebreak
\end{SegalCategoriesAndCohomologyTheories}
\begin{Operads}
\begin{itemise}
\item \textbf{Operad:} upointed $\scrC(j)$ with right $\Sigma_j$-action. $\scrC(0)=*$, $1\in\scrC(1)$, 
\[\textup{composition maps:\qquad }\gamma:\scrC(k)\times\left(\scrC(j_1)\times\cdots\times\scrC(j_k)\right)
\to\scrC\left(\sum j_i\right)\]
satisfying associativity, $\gamma(c;(1,\ldots,1))=c$, equivariance formula.
\item An operad is $\Sigma$-free if it is so level-wise.
\item Define local equivalence as levelwise weak equivalence.
\hrule
\item \textbf{$\scrC$-space:} $X$ with a map $\theta(c):X^j\to X$ for each $c\in \scrC(j)$, so that: the composition maps are compatible with composition of operations; $\theta(1)=\id$; and an equivariance is satisfied.
\item There's a category of $\scrC$-spaces, $\scrC[\scrT]$.
\hrule
\item Let $\scrM(j)=\Sigma_j$ and $\scrN(j)=*$. 
\item $\scrM$-spaces are monoids --- $\theta(\id_{\Sigma_2})$ is an associative multiplication, basepoint the identity.
\begin{itemize}\squishlist
\item An $A_\infty$ operad is one with a local equivalence to $\scrM$ (automatically $\Sigma$-free). Algebras over these have multiplications ``associative up to all higher homotopy''.
\end{itemize}
\item $\scrN$-spaces are commutative monoids --- the equivariance forces commutativity. 
\begin{itemize}\squishlist
\item An $E_\infty$-operad is a $\Sigma$-free operad with a local equivalence to $\scrN$. Algebras over these have multiplications ``associative and commutative up to all higher homotopy''.
\end{itemize}
\hrule
\item \textbf{Monad:} a monoid object in $\Fun(\scrT,\scrT)$.\qquad  $CC\nt C$, $1_\scrT\nt C$:
\[\xymatrix@C=1.5cm{
CX\ar[r]^{C(\eta_X)}&CCX\ar[d]_{\mu_X}&CX\ar[l]_{\eta_{CX}}\\
&CX\ar@{=}[lu]\ar@{=}[ru]
}
\text{ and }
\xymatrix@C=1.5cm{
CCCX\ar[r]^{\mu_{CX}}\ar[d]_{C(\mu_X)}&CCX\ar[d]_{\mu_X}\\
CCX\ar[r]^{\mu_X}&CX
}
\]
\item \textbf{$C$-algebra:} $X\in\scrT$ with associative, unital action map $CX\to X$.
\item There's a category of $C$-algebras, $C[\scrT]$.
\hrule
\item Want $\mathsf{Operads}\to\mathsf{Monads_\scrT}$ so that $\scrC$-space structures are the same as $C$-algebra structures.
\item Define $CX:=\coprod\scrC(j)\times X^j/\approx$, where
\begin{alignat*}{2}
(\gamma(c;(1,\ldots,1,*,1,\ldots,1)),(y_1,\ldots,y_{j-1}))&\approx(c,(y_1,\ldots,y_{i-1},*,y_i,\ldots,y_{j-1}))&\qquad&\text{($c\in\scrC(j)$)}
\\(c\sigma,\underline{y})&\approx (c,\sigma\underline{y})\makebox[0cm][l]{\qquad ($\underline{y}\in X^j$, $\sigma\in\Sigma_j$, $c\in\scrC(j)$)}%&\qquad&\text{()}
%\\&=%&\qquad&\text{()}
%\\&=%&\qquad&\text{()}
%\\&=%&\qquad&\text{()}
\end{alignat*}
\begin{itemize}\squishlist
\item $C$ is a monad, with the desired property.
\item $CX$ is a $C$-algebra via $CC\nt C$ and there's an adjunction \smash{$\xymatrix@R=.3cm@C=2cm{
\scrT  \ar@<.6ex>[r]^{C}&
C[\scrT]  \ar@<.4ex>[l]^{\textup{forget}}
}$}
\item $C$ is the ``free $C$-algebra''/``free $\scrC$-space'' functor.
\end{itemize}
\item Given any adjunction \smash{$\xymatrix@R=.3cm@C=2cm{
\scrT  \ar@<.6ex>[r]^{Q}&
\scrV  \ar@<.4ex>[l]^{U}
}$}, $UQ$ is a monad.
\item \textbf{Prop 3.4} Suppose that $\phi:\scrC_1\to\scrC_2$ is a local equivalence of $\Sigma$-free operads such that either $\pi_0(\scrC_i(j))=\Sigma_j$ for all $j$, or $\pi_0(\scrC_i(j))=*$ for all $j$. Then for $X$ connected, $C_1X\weakequiv C_2X$.
\hrule
\item Define the little $n$-cubes operad, even for $n$ infinite.
\item Discuss $C_nX$.
\item Show that $C_n$ acts on $n$-fold loop spaces.
\hrule
\item \textbf{Approximation:} If $X$ is connected, there is a weak equivalence $C_nX\to\Omega^n\Sigma^nX$:
\[C_nX\overset{C_n\eta_n}{\to}C_n\Omega^nS^nX\overset{\theta_n}{\to}\Omega^nS^nX\]
\item \textbf{Recognition:} 
Suppose that $\scrD$ is $\Sigma$-free, and $\pi:\scrD\to\scrC_n$ is a local equivalence. Let $X$ be a connected $D$-algebra. Then $X$ is weakly equivalent to an $n$-fold loop space.
\hrule
\item Suppose $D:\calT\to\calT$ is a monad. A ``$D$-functor to $\calV$'' is a functor $F:\calT\to \calV$ with a right action of $D$, that is: $\lambda:FD\nt F$ satisfying unitality and associativity.
\item Let $\calB(\calT,\calV)$ be the category whose objects are triples $(F,D,X)$, where $D$ is a monad in $\calT$, $X$ is a $D$-algebra (in $\calT$), and $F$ is a $D$-functor to $\calV$. Define a functor
\[B_*:\calB(\calT,\calV)\to s\calV\textup{\quad by\quad}B_q(F,D,X)=FD^qX\]
There are $n+1$ different positions to insert the unit of $D$, giving the maps $s_i$ for $1\leq i\leq n$. There are $q+1$ different `action maps' with which to form a map $FD^qX\to FD^{q-1}X$: 
\begin{itemise}
\item the right action of $D$ on $F$ induces $\partial_0$;
\item the $n-1$ different multiplications on $D$ induce $\partial_i$ for $1\leq i\leq n-1$;
\item the left action of $D$ on $X$ induces $\partial_q$.
\end{itemise}
\item \textbf{Proposition 9.8:} $X_*$ is a strong deformation retract of $B_*(D,D,X)$ in $s\calT$, with a functorial right inverse.
\hrule
\item Suppose given a map of monads $D\nt\Omega^n\Sigma^n$.
\begin{itemize}\squishlist
\item The adjoint $\Sigma^nD\nt\Sigma^n$ makes $\Sigma^n$ a $D$-functor.
\item Applying $\Omega^n$ we obtain a $D$-functor structure on $\Omega^n\Sigma^n$.
\item The map $D\nt\Omega^n\Sigma^n$ is in fact a map of $D$-functors (drawing $D$-functor actions horizontal). Here, the quadrilateral commutes as we have a map of monads, and the bottom composite is the adjoint defining the $D$-functors:
\[\xymatrix{
DD\ar@{=>}[rr]\ar@{==>}[rd]\ar@{=>}[d]&&D\ar@{=>}[d]\\
\Omega^n\Sigma^nD\ar@{=>}[r]&
\Omega^n\Sigma^n\Omega^n\Sigma^n\ar@{=>}[r]&\Omega^n\Sigma^n
}\]
\hrule
\end{itemize}


%\item Suppose that $\pi:\scrD\to\scrC_n$ is a morphism of operads. Then $\Sigma^n:\scrT\to\scrT$ is a $\scrD$-functor, via the adjoint $\Sigma^nD\nt\Sigma^n$ of
%\[DY\overset{\pi}{\to}C_nY\overset{\alpha_n}{\to}\Omega^n\Sigma^nY.\]
%\item $\Omega^n\Sigma^n$ is then a $D$-functor simply by applying $\Omega^n$.

\end{itemise}


\begin{thm*}[13.1]
Suppose that $\scrD$ is $\Sigma$-free, and $\pi:\scrD\to\scrC_n$ is a local equivalence. Let $X$ be a $D$-algebra with structure map $\xi:DX\to X$. Then we have maps (of $D$-algebras), which are all weak equivalences except possibly $B(\alpha_n\pi,1,1)$:
\[\quad \xymatrix@C=1.8cm{
\makebox[0cm][r]{$X=$\,}\ar@{=}[d]{|X_*|}&\ar[l]_{\epsilon(\xi)\qquad}^{\textup{sdr\qquad}}
{|B_*(D,D,X)|}\ar[r]^{B_*(\alpha_n\pi,1,1)\quad }
&{|B_*(\Omega^n\Sigma^n,D,X)|}\ar@{=}[r]^{\textup{out}}&{|\Omega^n_*B_*(\Sigma^n,D,X)|}\ar[d]^{\gamma^n}_\sim\\
\makebox[0cm][r]{$X=$\,}{|X_*|}\ar[ur]_(.8){\tau(\zeta)}^(.4)\sim&
&&\Omega^n{|B_*(\Sigma^n,D,X)|}
}\]
%The maps in $sD[\scrT]$ drawn at right induce their counterparts in $D[\scrT]$ via geometric realisation, and the adjoint $\Sigma^nX\to B(\Sigma^n,D,X)$ of the dashed composite map is simply $|\tau_*(\id_{\Sigma^nX})|$. 
When $X$ is connected, $B(\alpha_n\pi,1,1)$ is a weak equivalence, so that we have constructed a functorial $n$-fold delooping of connected $D$-algebras. Finally, $B(\Sigma^n,D,X)$ is $(m+n)$-connected whenever $X$ is $m$-connected.
\end{thm*}
\begin{proof}
\begin{itemise}
\item The left hand triangle is the realisation of strong deformation retraction in $s\scrT$.
\item Suppose that $X$ is connected.
\begin{itemize}\squishlist
\item The levelwise maps are $DD^{q}X\overset{\pi}{\to} C_nD^{q}X\overset{\alpha_n}{\to} \Omega^n\Sigma^nD^qX$.
\item Then so is $D^qX$ is connected. [It's e.t.p.\ $DX$ is so, but this is kinda obvious, as $\pi_0D(j)$ is $*$ when $n>1$ and $\Sigma_j$ when $n=1$.]
\item The approximation theorem shows that $\alpha_n$ is a weak equivalence. That $\pi$ is a local equivalence of $\Sigma$-free operads, $\pi$ is a weak equivalence.
\item The realisation is an $H$-map of connected $H$-spaces, and under these circumstances (with some properness assumptions on the simplicial spaces) we have that the realisation is a weak equivalence.
\end{itemize}
\item We can bring the $\Omega^n$ out of the bar construction, as it plays no role in the $D$-functor structure of $\Omega^n\Sigma^n$.
\item There's always a map $\gamma:|\Omega_*Y|\to\Omega|Y|$.
\begin{itemize}\squishlist
\item It's a weak equivalence when each $Y_p$ is connected (and $Y$ is proper).
\item The worst space we need to be connected is $\Omega^{n-1}\Sigma^nD^qX$. It is.
\end{itemize}
\item Just comment the last bits.\footnote{The connectivity of $B(\Sigma^n,D,X)$ follows by theorem 11.12. The composite is a map of $D$-algebras, as long as $\gamma^n$ is. It happens to be so, but \emph{only as $D$ comes from an operad}.}\qedhere
\end{itemise}
\end{proof}
In fact, the delooping is unique:
\begin{cor*}
If
\[X\overset{f}{\from}X'\overset{g}{\to}\Omega^nY\]
 is a pair of weak homotopy equivalences of connected $D$-algebras, with $Y$ $n$-connected, then the diagram
\[B(S^n,D,X)\overset{B(1,1,f)}{\from}B(S^n,D,X')\overset{g^{\textup{adj}}}{\to}Y\]
displays a weak homotopy equivalence between $Y$ and $B(S^n,D,X)$.
\end{cor*}

\pagebreak
\section*{References}
\begin{prop*}[3.10]
If $\scrC$ is an $A_\infty$ operad an $\scrC'$ is any operad over $\scrM$, then
$\pi_2:\scrC\nabla\scrC'\to\scrC'$ is a local $\Sigma$-equivalence. If instead
$\scrC$ is $E_\infty$, and $\scrC'$ is any $\Sigma$-free operad,
$\pi_2:\scrC\times\scrC'\to\scrC'$ is a local equivalence of $\Sigma$-free
operads.
\end{prop*}\noindent
This shows that every $E_\infty$ space is an $A_\infty$ space, as if $\scrC$
is $E_\infty$, the map $\scrC\times\scrM\to\scrM$ is, by 3.7 and 3.10,
the structure map of an $A_\infty$ operad. 

\hrule

\begin{thm*}[11.12]
Fix $n\geq0$. If $X$ is a strictly proper simplicial space such that $X_q$ is $(n-q)$-connected for all $q\leq n$, then $|X|$ is $n$-connected.
\end{thm*}
\begin{thm*}[11.13]
Let $f:X\to Y$ be a simplicial map between strictly proper simplicial spaces. Assume that each $f_q$ is a weak equivalence and that either both $|X|$ and $|Y|$ are simply connected or $|f|$ is a connected map of connected $H$-spaces. Then $|f|$ is a weak homotopy equivalence.
\end{thm*}
\hrule
\begin{thm*}[12.2]
Let $C:\calT\to\calT$ be the monad associated with an operad $\scrC$. Then there is a natural homeomorphism $\nu:|C_*X|\to C|X|$ for $X\in s\calT$. Moreover, the monad action diagrams commute, so that geometric realisation defines a functor $sC[\calT]\to C[\calT]$.
\end{thm*}
\begin{proof}
Points of $|C_*X|$ are of the form $|[c,x_1,\ldots,x_j],u|$, for $u\in\Delta_q$, $x_i\in X_q$ and $c\in\calC(j)$. One can map this point to $[c,|x_1,u|,\ldots,|x_j,u|]$.
\end{proof}
\noindent Note that this fails for algebras over $\Omega^n\Sigma^n$, so that $C$ must be associated with an operad. This is a very important example. Otherwise it seems as though we could have gotten away with never mentioning operads.
\begin{thm*}[12.3]
For $X\in s\scrT$, there is a natural map $\gamma:|\Omega_*X|\to\Omega|X|$. If $X$ is proper\footnote{A simplicial space is proper if the inclusion of the image $sX_q$ in $X_{q+1}$ of the degeneracies $s_i:X_q\to S_{q+1}$ gives an NDR pair $(X_{q+1},sX_q)$ for each $q$. May has an appendix full of ways to ensure that properness hypotheses are satisfied --- I have not yet read this material.} and each $X_q$ is connected, then $\gamma$ is a weak equivalence.
\end{thm*}
\begin{thm*}[12.4]
For $X\in s\scrT$, the iteration $\gamma^n:|\Omega^n_*X|\to\Omega^n|X|$ is a morphism of $C_n$-algebras, and the following diagram is commutative:
\[\xymatrix@C=2cm{
|(C_n)_*X|\ar[r]^{\nu}\ar[d]_{|(\alpha_n)_*|}&C_n|X|\ar[d]^{\alpha_n}\\
|\Omega^n_*\sigma^n_*X|\ar[r]^{\Omega^n\tau^n\circ\gamma^n}&\Omega^n\Sigma^n|X|
}\]
\end{thm*}

\end{Operads}
\begin{MarkusLocalisationPractice}
\KanSemResponse
{Localisation of spectra with respect to a homology theory}
Let $E$ be a spectrum, then $E_*(\DASH):=\pi_*(\DASH\sprod E)$. We'll work in the stable homotopy category. I'd like
\[\xymatrix@R=.3cm@C=2cm{
\mathsf{shc}  \ar@<.6ex>[r]^{F}&
\mathsf{shc}[\textup{$E_*$-isos}^{-1}] \ar@<.4ex>[l]^{G}\\
%X \ar@{|->}[r] & FX\\
%GY             & Y \ar@{|->}[l]\\
}\]
We'll have the right adjoint fully faithful, so that we can actually simply work in a subcategory. The image will be the $E_*$-local spectra.
\begin{defn*}
A spectrum $X$ is $E_*$-local if every $E_*$-isomorphism $Y\to Z$ induces an isomorphism $[Z,X]\to[Y,X]$. A spectrum is $E_*$-acyclic if $E_*(X)=0$.
\end{defn*}
\begin{lem*}
A spectrum is $E_*$-local if it admits no maps from $E_*$-acyclic spectra.
\end{lem*}
\begin{thm*}
For every spectrum $X$ there's and $E_*$-local spectrum $X_E$ and an $E_*$-iso $X\overset{\eta_X}{\to} X_E$. This is the unit of the above adjunction.
\end{thm*}
We can form a triangle (i.e.\ cofibre sequence):
\[{_E}X\overset{\theta_X}{\to}X\overset{\eta_X}{\to}X_E\to\Sigma{{_E}X}\]
with ${_E}X$ acyclic, $\eta_X$ is the initial example of a map from $X$ to an $E_*$-local spectrum, and $_EX\to X$ is the terminal example of a map from an acyclic to $X$. [$\eta_x$ is also terminal for $E$-isos out of $X$.]

The spectrum is not determined by its localisation, for example $E$, $E\vee E$ and $\Sigma^nE$ all give the same localisation.
\begin{defn*}
We call $E$ and $F$ equivalent if they give the same localisation. This is equivalent to giving the same acyclic objects. The equivalence classes here \emph{form a set}. This set is partially ordered, as we can say $\langle E\rangle\leq\langle E'\rangle$ if every $E'$-local is also $E$-local.
\end{defn*}
\subsection*{Moore spectra}

\begin{exmp*} Given $A\in\AbGp$, $\mathbb{S}A$ is the Moore spectrum of $A$. There is a short exact sequence, for $X$ any spectrum
\[0\to\pi_*X\otimes A\to\pi_*(X\sprod\mathbb{S}A)=\mathbb{S}A_*(X)\to\Tor(\pi_{*-1}(X),A)\to0\]
and
\[0\to\Ext(A,\pi_{*+1}(X))\to[\mathbb{S}A,X]_*\to\Hom(A,\pi_*(X))\to0\]
Take cofiber sequence defining, and map in or out.

Now $\mathbb{S}A$ is connective and $H_*(\mathbb{S}A)=\begin{cases}
A,&\textup{if }*=0;\\
0,&\textup{otherwise}.
\end{cases}$ This characterises it completely, by the Hurewicz theorem and whitehead theorem.

Note that the second SES shows that there might be more maps between Moore spectra which do not come from maps of groups, but every map of groups can be realised.
\end{exmp*}
By the first SES, $\mathbb{S}A$-acyclicness depends only on its homotopy groups. To detect the equivalence class of the Moore spectrum of a group in the above poset, it's enough to consider...

The class of $\mathbb{S}A$-acyclic spectra only depends on whether $A$ is a torsion group and the set of primes it is uniquely divisible by. Thus, it suffices to examine, for $J$ a set of primes,
\[A=\Z_{(J)}\textup{\quad and\quad }A=\oplus_{p\in J}\Z/p\]
\begin{enumerate}\squishlist
\item If $A=\Z_{(J)}$ then $X_{\mathbb{S}A}=X\sprod \mathbb{S}A$, and $\pi_*(X_{\mathbb{S}A})=\pi_*X\otimes A$. A spectrum is local if its homotopy is already a module over $\Z_{(J)}$.
\begin{proof}
$\mathbb{S}A$ is a ring spectrum, and on $\pi_0$ it induces the ring multiplication of $A$. Then the unit $\mathbb{S}A\weakequiv\mathbb{S}A\sprod\mathbb{S}A$ is actually an isomorphism. As this is a right inverse for the multiplication, we have $\mathbb{S}A\sprod\mathbb{S}A\weakequiv\mathbb{S}A$. In particular, the map $X\to X\sprod \mathbb{S}A$ is an $\mathbb{S}A$-iso...
\end{proof}
\item If $A=\Z/p$, then $X_{\mathbb{S}A}\cong F(\Sigma^{-1}\mathbb{S}\Z/p^\infty,X)$.\footnote{The function spectrum is right adjoint to smash, constructed by Brown representability.} There is a short exact sequence
\[0\to\Ext(\Z/p^\infty,\pi_{*}X)\to\pi_*(X_{\mathbb{S}A})\to\Hom(\Z/p^\infty,\pi_{*+1}X)\]
Thus, if $\pi_*X$ is finitely generated for all $*$, then $\pi_*(X_{\mathbb{S}A})=\pi_{*}X\otimes\widehat{\Z_p}$.
A spectrum is $\mathbb{S}A$-local if $\Ext(\Z[1/p],\pi_*X)=0$ and $\Hom(\Z[1/p],\pi_*X)=0$ (which is called being Ext-$p$-complete.
\end{enumerate}
\begin{thm*}
If $A=\oplus_{p\in J}\Z/p$ then $X_{\mathbb{S}A}=\prod_{p\in J}X_{\mathbb{S}\Z/p}$.
\end{thm*}
\begin{thm*}
If $E$ and $X$ are connective, then $X_E\cong X_{\mathbb{S}\pi_*E}$.
\end{thm*}
\begin{defn*}
Let $K$ be $K$-theory. Then $\pi_*K$ is $\Z$ in even degrees, so a localisation as in the previous theorem would do nothing. But in fact localising w.r.t.\ $K$-theory is smashing with $S^B$, 
\end{defn*}


\pagebreak
\end{MarkusLocalisationPractice}
\begin{BousfieldLoc}
\KanSemResponse
{``Localization of spectra with respect to homology'' --- Bousfield}
I have only had the time to read the first few pages of this paper, and of course to attend Markus' practice talk. That said, there are a few interesting things to note.

The first thing which I note from this paper is just the existence of a nice notion of localisation. I suppose that I knew that one existed after having listened to a Juvitop talk on Bousfield localisation, but there are some particularly good aspects of this one. It's good that the localisation with respect to $\mathbb{S}G$ is determined only by the ``type of acyclicity'' of $G$; it's good that there's an $E_*$-Whitehead theorem; etc.

Perhaps here one should expect these good properties from generalities on Bousfield localisation, but I am still not sufficiently versed in the Bousfield localisation of model categories to have expected these things before reading them. One other thing that is striking on first sight is the fact that the localisation is a full subcategory of the stable homotopy category. This I perhaps \emph{should} have expected --- maps in the homotopy category are equivalence classes of maps in the model category between various replacements. If we find a replacement in the localised model category, it started as a replacement in the unlocalised model category, so the maps in the homotopy category should be unaltered.


\end{BousfieldLoc}
\begin{MarkusLocalisation}
\KanSemResponse
{Localisation of spectra with respect to a homology theory}
Let $E$ be a spectrum, then $E_*(\DASH):=\pi_*(\DASH\sprod E)$. We'll work in the stable homotopy category. I'd like an adjunction
\[\xymatrix@R=.3cm@C=2cm{
\mathsf{shc}  \ar@<.6ex>[r]^{F}&
\mathsf{shc}[\textup{$E_*$-isos}^{-1}] \ar@<.4ex>[l]^{G}\\
%X \ar@{|->}[r] & FX\\
%GY             & Y \ar@{|->}[l]\\
}\]
We'll have the right adjoint fully faithful, so that we can actually simply work in a subcategory. The image will be the $E_*$-local spectra.
\begin{defn*}
A spectrum $X$ is $E_*$-local if every $E_*$-isomorphism $Y\to Z$ induces an isomorphism $[Z,X]\to[Y,X]$. A spectrum is $E_*$-acyclic if $E_*(X)=0$.
\end{defn*}
\begin{lem*}
A spectrum is $E_*$-local if it admits no maps from $E_*$-acyclic spectra.
\end{lem*}
\begin{thm*}
For every spectrum $X$ there's and $E_*$-local spectrum $X_E$ and an $E_*$-iso $X\overset{\eta_X}{\to} X_E$. This is the unit of the above adjunction.
\end{thm*}
We can form a triangle (i.e.\ cofibre sequence):
\[{_E}X\overset{\theta_X}{\to}X\overset{\eta_X}{\to}X_E\to\Sigma{{_E}X}\]
with ${_E}X$ acyclic, $\eta_X$ is the initial example of a map from $X$ to an $E_*$-local spectrum, and $_EX\to X$ is the terminal example of a map from an acyclic to $X$. [$\eta_x$ is also terminal for $E$-isos out of $X$.]

The spectrum is not determined by its localisation, for example $E$, $E\vee E$ and $\Sigma^nE$ all give the same localisation.
\begin{defn*}
We call $E$ and $F$ equivalent if they give the same localisation. This is equivalent to giving the same acyclic objects. The equivalence classes here \emph{form a set}. This set is partially ordered, as we can say $\langle E\rangle\leq\langle E'\rangle$ if every $E'$-local is also $E$-local.
\end{defn*}
\subsection*{Moore spectra}

\begin{exmp*} Given $A\in\AbGp$, $\mathbb{S}A$ is the Moore spectrum of $A$. There is a short exact sequence, for $X$ any spectrum
\[0\to\pi_*X\otimes A\to\pi_*(X\sprod\mathbb{S}A)=\mathbb{S}A_*(X)\to\Tor(\pi_{*-1}(X),A)\to0\]
and
\[0\to\Ext(A,\pi_{*+1}(X))\to[\mathbb{S}A,X]_*\to\Hom(A,\pi_*(X))\to0\]
Take cofiber sequence defining, and map in or out.

Now $\mathbb{S}A$ is connective and $H_*(\mathbb{S}A)=\begin{cases}
A,&\textup{if }*=0;\\
0,&\textup{otherwise}.
\end{cases}$ This characterises it completely, by the Hurewicz theorem and whitehead theorem.

Note that the second SES shows that there might be more maps between Moore spectra which do not come from maps of groups, but every map of groups can be realised.
\end{exmp*}
By the first SES, $\mathbb{S}A$-acyclicness depends only on its homotopy groups. To detect the equivalence class of the Moore spectrum of a group in the above poset, it's enough to consider...

The class of $\mathbb{S}A$-acyclic spectra only depends on whether $A$ is a torsion group and the set of primes it is uniquely divisible by. Thus, it suffices to examine, for $J$ a set of primes,
\[A=\Z_{(J)}\textup{\quad and\quad }A=\oplus_{p\in J}\Z/p\]
\begin{enumerate}\squishlist
\item If $A=\Z_{(J)}$ then $X_{\mathbb{S}A}=X\sprod \mathbb{S}A$, and $\pi_*(X_{\mathbb{S}A})=\pi_*X\otimes A$. A spectrum is $\mathbb{S}A_*$-local if its homotopy is already a module over $\Z_{(J)}$.
\begin{proof}
$\mathbb{S}A$ is a ring spectrum, and on $\pi_0$ it induces the ring multiplication of $A$. Then the unit $\mathbb{S}A\weakequiv\mathbb{S}A\sprod\mathbb{S}A$ is actually an isomorphism. As this is a right inverse for the multiplication, we have $\mathbb{S}A\sprod\mathbb{S}A\weakequiv\mathbb{S}A$. In particular, the map $X\to X\sprod \mathbb{S}A$ is an $\mathbb{S}A$-iso...
\end{proof}
\item If $A=\Z/p$, then $X_{\mathbb{S}A}\cong F(\Sigma^{-1}\mathbb{S}(\Z/p^\infty),X)$.\footnote{The function spectrum is right adjoint to smash, constructed by Brown representability.} There is a short exact sequence
\[0\to\Ext(\Z/p^\infty,\pi_{*}X)\to\pi_*(X_{\mathbb{S}A})\to\Hom(\Z/p^\infty,\pi_{*+1}X)\]
Thus, if $\pi_*X$ is finitely generated for all $*$, then $\pi_*(X_{\mathbb{S}A})=\pi_{*}X\otimes\widehat{\Z_p}$ (as the Hom set is zero).
A spectrum is $\mathbb{S}A$-local if $\Ext(\Z[1/p],\pi_*X)=0$ and $\Hom(\Z[1/p],\pi_*X)=0$ (which is called being Ext-$p$-complete).
\begin{proof}
Consider the short exact
\[\xymatrix{
0\ar[r]&
\Z\ar[r]&
\Z[1/p]\ar[r]&
\Z/p^\infty\ar[r]&
0
}\]
this can be realised on moore spectra, giving a triangle
\[\xymatrix{
\Sigma^{-1}\mathbb{S}(\Z/p^\infty)\ar[r]&
\mathbb{S}\ar[r]&
\mathbb{S}\Z[1/p]\ar[r]&
\mathbb{S}\Z/p^\infty
}\]
applying the function spectrum, we still have a cofibre sequence
It'll be enough to prove that $F(\mathbb{S}\Z/p^\infty,X)$ is local and $F(\mathbb{S},X)$ is acyclic. To see the localness, for maps in from an acyclic $Y$, we get, by adjunction, $[Y\sprod \Sigma^{-1}\mathbb{S}\Z/p^\infty,X]$ but the source is acyclic, as this only depends on div type.

The next is similar.
\end{proof}
Note that $F(\sigma^{-1}\mathbb{S}\Z/p^\infty)$ is the holim of a sequence
\[\xymatrix{
X\sprod \mathbb{S}\Z/p&\ar[l]X\sprod \mathbb{S}\Z/p^2&\ar[l]X\sprod \mathbb{S}\Z/p^3&\ar[l]
}\]
Note that the hocolim of 
\[\xymatrix{
 \mathbb{S}\Z/p&\ar[l];[] \mathbb{S}\Z/p^2&\ar[l];[] \mathbb{S}\Z/p^3&
}\]
is $\mathbb{S}\Z/p^\infty$.
Applying function spectra, we get a holim for $F(\Sigma^{-1}\mathbb{S}\Z/p^\infty,X)$,
\end{enumerate}
\begin{thm*}
If $A=\oplus_{p\in J}\Z/p$ then $X_{\mathbb{S}A}=\prod_{p\in J}X_{\mathbb{S}\Z/p}$.
\end{thm*}
\begin{thm*}
If $E$ and $X$ are connective, then $X_E\cong X_{\mathbb{S}\pi_*E}$.
\end{thm*}
\begin{defn*}
Let $K$ be $K$-theory. Then $\pi_*K$ is $\Z$ in even degrees, so a localisation as in the previous theorem would do nothing. But in fact localising w.r.t.\ $K$-theory is smashing with $S^B$, 
\end{defn*}


\pagebreak
\end{MarkusLocalisation}
\begin{MoorePostnikovSystems}
\KanSemResponse
{``Semi-simplicial complexes and Postnikov systems'' --- J.\ Moore}
Reading this paper has been a pleasure, as it lays out parsimoniously a series of analogies between Kan complexes and spaces. My reading of this paper is the first thing that has justified to me that Kan complexes could/should be the fibrant objects in a model category structure on $s\Set$ Quillen equivalent to the category of spaces.

It seems to me that Kan complexes are rather more convenient than spaces for the study of homotopy groups. For example, it is possible to pare back a Kan complex $X$ to a minimal one $X'$, a deformation retract, after which the homotopy groups $\pi_n(X,x)$ (by lemma 1.20) are just the \emph{set} of maps from $(\Delta^n,\partial\Delta^n)$ to $X'$. I don't know how useful this is for calculation, but it's a nice idea. I can't think of an analogy for this in $\Top$. Given a space $X$, one can take its simplicial complex, and then trim this back to a minimal Kan complex, which seems to summarise exactly the data of 'points', 'paths', 'homotopies of paths', etc.

One thing which is immediately striking is how easy it is to describe the Postnikov stages of a Kan complex. It's not just easy to describe them, but even to see why the description works.

I had intended to write more, on the Dold-Kan correspondence (which is pretty surprising I suppose), but I'm deeply exhausted after the move.
\pagebreak
\end{MoorePostnikovSystems}
\begin{KrishanuKanComplexes}
\KanSemResponse
{``Kan Complexes, etc.'' --- Krishanu}
A Kan fibration is a map of simplicial sets with the RLP with respect to all of the inclusions of horns $\Lambda_n^k\cofibration \Delta_n$. A Kan complex is any simplicial set whose unique map to the point is a Kan fibration. As such, it is clear that $Y$ is Kan implies that $Y^X$ is Kan for any $X$.

Note that we define the mapping simplicial set by the formula $(Y^X)_n=\Hom(X\times\Delta_n,Y)$. To understand this, recall that by Yoneda's lemma, the $n$-simplices of a simplicial set are in bijection with maps from the standard $n$-simplex, and then recall the adjunction.

The Kan condition shows that we can compose paths, etc, so that the homotopy groups $\pi_n(Y,y):=\pi_0((Y,y)^{(\Delta_n,\partial\Delta_n)})$ are indeed groups.

There's a fantastic way to form the Postnikov tower for a Kan complex $Y$. Let $P^n Y$ be formed from $Y$ by identifying two simplices whenever all of their faces of dimension at most $n$ coincide. Using Yoneda's lemma, one sees that the homotopy groups of $P^nY$ vanish above dimension $n$.

\begin{prop*}
A simplicial group is a Kan complex. Moreover, its homotopy groups are all abelian, and multiplication in the homotopy groups coincides with multiplication in the group.
\end{prop*}
\begin{thm*}[Dold-Kan]
The categories $s\mathsf{Ab}$ and $\mathsf{Ch}_{\geq0}$ are equivalent, with a commuting diagram:
\[\xymatrix{
\ar[rd]_{\pi_*}s\mathsf{Ab}\ar@<.1cm>[rr]^{N_*}&&
\mathsf{Ch}_{\geq0}\ar@<1mm>[ll]^\Gamma\ar[ld]^{H_*}\\
&\mathsf{Ab}
}\]
\end{thm*}
\noindent One can use this theorem to give nice simplicial models for Eilenberg-MacLane spaces, which have a multiplication which strictifies the loopspace structure\footnote{Recall that the realisation of a product of simplicial sets is homeomorphic to the product of the realisations, so the realisation of a simplicial group is a group.}.

To construct $N_*$ (the \emph{normalised chain complex}\footnote{Haynes ramarked that the simplicial abelian groups which are \emph{minimal} Kan complexes are exactly those for which the differentials all vanish in the normalised chain complex.}), one defines, for $A_*$ a simplicial abelian group:
\[N_n(A)=\bigcap_{i=0}^{n-1}\ker(\partial_i),\textup{ and uses $\partial_n$ for the differential.}\]


\pagebreak
\end{KrishanuKanComplexes}
\begin{JeremyModelPractice}
\KanSemResponse
{``Model Categories and some other stuff.'' --- Jeremy}
\subsection*{Philosophy}
Given a category $\calC$, what does it mean to equip $\calC$ with a homotopy theory? Ideas:

\begin{itemise}
\item Equip $\calC$ with 2-morphisms (as homotopies), and so obtain the notion of $\infty$-categories.
\item Take the classical homotopy category, $\ho\calC$, where one simply identifies homotopic morphisms. This sounds like a bad idea --- we shouldn't just identify isomorphic things!
\item Quillen: form a model category, to generalise the homotopy theories of spaces and chain complexes...
\end{itemise}
Not every higher category comes from a model category, and the model category should be thought of as a (non-canonical) presentation of the homotopy theory coming from a higher category.
\begin{defn*}
Fix a category $\calC$. Say that $u:A\to Z$ has the LLP with respect to $f:Y\to W$ (equivalently that $f$ has the RLP wrt $u$) if every commuting square as follows has a lift:
\[\xymatrix{
A\ar[r]\xycofib[d]_u&Y\ar@{->>}[d]^f\\
Z\ar@{-->}[ru]\ar[r]&W
}\]
If $u$ has the LLP wrt $f$, write $u\pitchfork f$. Given a class $M$ of maps in $\calC$, define $^\pitchfork M$ to be the class of all maps with the LLP wrt every morphism in $M$, and define $ M^\pitchfork$ to be the class of all maps with the RLP wrt every morphism in $M$.
\end{defn*}
\begin{defn*}
A weak factorisation system $(L,R)$ in $\calC$ is...
\begin{enumerate}\squishlist
\item Every $f:A\to B$ factors as $A\overset{p}{\to} Z\overset{u}{\to} B$ with $p\in R$ and $u\in L$.
\item $\calL={^\pitchfork R}$, $R=L^\pitchfork$
\end{enumerate}
\end{defn*}
\begin{defn*}
A model category $\calC$ is a category with weak equivs $W\subset\Mor\calC$ satisfying
\begin{enumerate}\squishlist
\item Two out of three.
\item There exist classes $\textup{cof,fib}\subset\Mor\calC$ such that the following two pairs are weak factorisation systems: $(\textup{cof},\textup{fib}\cap W)$, $(\textup{cof}\cap W,\textup{fib})$.
\item $\calC$ is complete and cocomplete.
\end{enumerate}
\end{defn*}\begin{prop*}
If $(L,R)$ is a weak factorisation system, $L$ is closed under pushouts and arbitrary coproducts, and $R$ is closed under pullbacks and products. Both are closed under retracts and both contain all isomorphisms. ($L$ is closed under transfinite composition...)
\end{prop*}
Given a model category $\calC$, as it is complete and cocomplete, $\calC$ contains an initial object $\emptyset$, and a terminal object $*$. Call an object $A$ of $\calC$:
\begin{enumerate}\squishlist
\item cofibrant if $\emptyset\to A$ is a cofibration
\item fibrant if $A\to *$ is a fibration
\end{enumerate}
The most important objects are the cofibrant-fibrant objects, and the other objects in the category `model' these.

\subsection*{Homotopy theory}
If $A,B$ are fib-cof, then one can form a factorisation $A\sqcup A\cofibration C \acyclicfibration A$ ($C$ a ``cylinder object''), and this way, we have the usual method of defining homotopy.% So to define homotopy, we could restrict to the full subcategory consisting of cof-fib objects, and ... hrm... (I think the hootopy cat should be the localisation to the thingo... schmeh)

\begin{defn*}
The correct notion of `morphism of homotopy theories' is that of Quillen adjunction:
\[\xymatrix@R=.3cm@C=2cm{
\calC  \ar@<.6ex>[r]^{L}&
\calD  \ar@<.4ex>[l]^{R}
}\]
such that $L$ preserves cofibrations and acyclic cofibrations, and $R$ preserves fibrations and acyclic fibrations.
\end{defn*}
\begin{lem*}[Ken Brown]
$L$ preserves weak equivalences between cofibrant objects.
\end{lem*}
\begin{proof}
Given $f:A\weakequiv B$, chase the following diagram:
\[\xymatrix@R=0cm{
&A\xycofib[dr]\\
\emptyset\xycofib[ur]\xycofib[dr]&&A\sqcup B\xycofib[r]&C\xyfib[r]^\sim&B\\
&B\xycofib[ur]
}\qedhere\]
\end{proof}
\noindent This implies that the adjunction induces maps of homotopy categories.
\subsection*{Producing model structures}
The following is easy to check, easy to compute with, and very prevalent:
\begin{defn*}
A combinatorial model category is a model category $\calC$, which is 
\begin{enumerate}\squishlist
\item \textbf{Locally presentable:} 
\begin{itemise}
\item Locally small (the hom classes are sets);
\item There exists a set $S$ of objects generating $\calC$ under colimits such that every object of $S$ is ``small''. 
\begin{itemise}
\item An object $s$ is small \Iff it is small with respect to some regular cardinal $\kappa$. $s$ is small w.r.t.\ a regular cardinal $\kappa$ (and thus w.r.t.\ all larger regular cardinals) if the following natural map is always bijective.
Suppose we have a poset $J$ such that every subposet $L\subset J$ of cardinality $<\kappa$ has an upper bound in $J$ (a $\kappa$-filtered category). Then for all diagrams $J\to \calC$, there is a natural map 
\[\varinjlim_J \Hom_\calC(X,F(j))\to\Hom(X,\varinjlim_J F(j)).\]
This is the map which we require to be a bijection.
\end{itemise}
\end{itemise}
\item Cofibrantly generated: there exist sets $I,J$ such that $I$ and $J$ generate the cofibrations and trivial cofibrations respectively.
\begin{itemise}
\item If $I$ is a set, write $\textup{call}(I)$ for the class of things which can be obtained using pushouts and (transfinite) compositions. Then write $\textup{cof}(I)$ for the set of retracts of $\textup{cell}(I)$. This is what we mean by generate.
\end{itemise}

\end{enumerate}
In simplicial sets, $I$ is the set of inclusions $\partial_i\Delta^n\to \Delta^n$.


\end{defn*}
\begin{thm*}[Jeff Smith]
Suppose that $\calC$ is locally presentable and $\textup{Arr}_W(\calC)\subset\textup{Arr}(\calC)$ is an accessible inclusion. Given weak equivalences $W$ satisfying two out of three, and generating cofibrations $I$ such that $\textup{cof}(I)\cap W$ is closed under pushout and transfinite composition. Suppose also that $I^\pitchfork\subset W$.

Then $\calC$ is a model category.
\end{thm*}

\pagebreak
\end{JeremyModelPractice}
\begin{JeremyModelCategories}
\KanSemResponse
{``Model Categories and some other stuff'' --- Jeremy}
\subsection*{Philosophy}
Given a category $\calC$, what does it mean to equip $\calC$ with a homotopy theory? Ideas:

\begin{itemise}
\item Equip $\calC$ with 2-morphisms (as homotopies), and so obtain the notion of $\infty$-categories.
\item Take the classical homotopy category, $\ho\calC$, where one simply identifies homotopic morphisms. This sounds like a bad idea --- we shouldn't just identify isomorphic things!
\item Quillen: form a model category, to generalise the homotopy theories of spaces and chain complexes. This is a strong thing to do --- to give a model structure can be difficult, but once you have one, it's a very powerful computational tool.
\end{itemise}
Not every higher category comes from a model category, and the model category should be thought of as a (non-canonical) presentation of the homotopy theory coming from a higher category.
\begin{defn*}
Fix a category $\calC$. Say that $u:A\to Z$ has the LLP with respect to $f:Y\to W$ (equivalently that $f$ has the RLP wrt $u$) if every commuting square as follows has a lift:
\[\xymatrix{
A\ar[r]\xycofib[d]_u&Y\ar@{->>}[d]^f\\
Z\ar@{-->}[ru]\ar[r]&W
}\]
If $u$ has the LLP wrt $f$, write $u\pitchfork f$. Given a class $M$ of maps in $\calC$, define $^\pitchfork M$ to be the class of all maps with the LLP wrt every morphism in $M$ (the \emph{left compliment}), and define $ M^\pitchfork$ to be the class of all maps with the RLP wrt every morphism in $M$ (the \emph{right compliment}).
\end{defn*}
\begin{defn*}
A weak factorisation system in $\calC$ is a pair $(L,R)$ of subclasses of $\Mor\calC$ such that:
\begin{enumerate}\squishlist
\item Every $f:A\to B$ factors as $A\overset{p}{\to} Z\overset{u}{\to} B$ with $p\in R$ and $u\in L$.
\item $\calL={^\pitchfork R}$, $R=L^\pitchfork$
\end{enumerate}
\end{defn*}
\begin{prop*}
For any subclass $M\subset\Mor\calC$, the left complement ${^\pitchfork M}$:
\begin{enumerate}\squishlist
\item contains all isomorphisms;
\item is closed under retracts;
\item is closed under coproducts (taken in the arrow category);
\item is closed under co-base change (with respect to arbitrary maps);
\item is closed under transfinite composition.\footnote{Given a sequence of maps $x_0\to x_1\to\cdots$, the transfinite composition is the map $x_0\to\varinjlim x_i$. In fact, we can form a transfinite induction with any ordinal in place of $\omega$.}
\end{enumerate}
\end{prop*}
\begin{defn*}
A model category $\calC$ is a category with weak equivs $W\subset\Mor\calC$ satisfying
\begin{enumerate}\squishlist
\item Two out of three.
\item There exist classes $\textup{cof,fib}\subset\Mor\calC$ such that the following two pairs are weak factorisation systems: $(\textup{cof},\textup{fib}\cap W)$, $(\textup{cof}\cap W,\textup{fib})$.
\item $\calC$ is complete and cocomplete.
\end{enumerate}
\noindent The power of such a structure comes from the interaction of the two weak factorisation systems. Note that the class $W$ is determined as the set of composites of an acyclic cofibration and an acyclic fibration.
\begin{exmps*}
\begin{enumerate}\squishlist
\item $s\mathsf{Set}$, with cofibrations the monomorphisms, fibrations the Kan fibrations, and weak equivalences trickier.
\item $\mathsf{Top}$, with cofibrations the retracts of generalised CW pairs, fibrations the (not necessarily surjective) Serre fibrations, and weak equivalences the normal weak equivalences.
\item $\mathsf{Top}$, with an alternative model structure is which the cofibrations the NDR-pairs, the fibrations are the Hurewicz fibrations, and the weak equivalences the homotopy equivalences.\footnote{Although perhaps not as useful as the other model structure, this structure is nice because one can specify the three classes very explicitly.}
\end{enumerate}

\end{exmps*}

\end{defn*}\begin{prop*}
If $(L,R)$ is a weak factorisation system, $L$ is closed under pushouts and arbitrary coproducts, and $R$ is closed under pullbacks and products. Both are closed under retracts and both contain all isomorphisms. ($L$ is closed under transfinite composition...)
\end{prop*}
Given a model category $\calC$, as it is complete and cocomplete, $\calC$ contains an initial object $\emptyset$ (the empty limit), and a terminal object $*$ (the empty colimit). Call an object $A$ of $\calC$:
\begin{enumerate}\squishlist
\item cofibrant if $\emptyset\to A$ is a cofibration
\item fibrant if $A\to *$ is a fibration
\end{enumerate}
The most important objects are the cofibrant-fibrant objects, and the other objects in the category `model' these. In $s\mathsf{Set}$, these are the Kan complexes, and in $\mathsf{Top}$, these are the retracts of CW complexes.

\subsection*{Homotopy theory}
Given a category $\calC$ with weak equivalences, we could think about $\ho\calC=\calC[W^{-1}]$. To construct this naiively, we need to `pass to a higher universe'. 

An alternative would be to form the full subcategory $\calC_{\textup{cf}}$ of fibrant-cofibrant objects of $\calC$, and take the homotopy category $\ho\calC_{\textup{cf}}$. This turns out to be an equivalent category, but we have not avoided the set-theoretic difficulties.

Instead, we proceed as in $\mathsf{Top}$. If $A,B$ are fib-cof, then one can form a factorisation $A\sqcup A\cofibration C \acyclicfibration A$ ($C$ a ``cylinder object''), and this way, we have the usual method of defining homotopy. That is, we say that $f,g:A\sqcup A\to B$ are homotopic if $f\sqcup g:A\sqcup A\to B$ factors through $C$. We can this way define $\ho\calC:=\calC_{\textup{cf}}/\sim$, and we see that this definition gives an \emph{equivalent} category to those above, but without set-theoretic difficulties.

% So to define homotopy, we could restrict to the full subcategory consisting of cof-fib objects, and ... hrm... (I think the hootopy cat should be the localisation to the thingo... schmeh)
\subsection*{Morphisms of model categories}
%The correct notion of `morphism of homotopy theories' is that of Quillen adjunction:
\begin{defn*}
An adjunction \smash{$\xymatrix@R=.3cm@C=1cm{
\calC  \ar@<.6ex>[r]^{F}&
\calD  \ar@<.4ex>[l]^{G}
}$} is called a \emph{Quillen adjunction} if any of the following equivalent conditions hold:
\begin{enumerate}\squishlist
\item $F$ preserves cofibrations and acyclic cofibrations;
\item $G$ preserves fibrations and acyclic fibrations; or
\item $F$ preserves cofibrations and $G$ preserves fibrations.
\end{enumerate}
It is called a \emph{Quillen equivalence} if in addition, a map $F(c)\to d$ is a weak equivalence \Iff its adjoint $c\to G(d)$ is a weak equivalence.
\end{defn*}
\begin{exmp*} There is a Quillen equivalence:
\[\xymatrix@R=.3cm@C=2cm{
s\mathsf{Set}  \ar@<.6ex>[r]^{|\cdot|}&
\mathsf{Top}  \ar@<.4ex>[l]^{\textup{Sing}}\\
%X \ar@{|->}[r] & FX\\
%GY             & Y \ar@{|->}[l]
}\]
In fact, the realisation of a Kan fibration is a Serre fibration (the content of a paper of Quillen), so that this adjunction preserves more than is to be expected.
\end{exmp*}
\begin{lem*}[Ken Brown]
$F$ preserves weak equivalences between cofibrant objects.
\end{lem*}
\begin{proof}
Given $f:A\weakequiv B$, chase the following diagram:
\[\xymatrix@R=0cm{
&A\xycofib[dr]\\
\emptyset\xycofib[ur]\xycofib[dr]&&A\sqcup B\xycofib[r]&C\xyfib[r]^\sim&B\\
&B\xycofib[ur]
}\qedhere\]
\end{proof}
\noindent This implies that the adjunction induces maps of homotopy categories. Specifically, it shows that the derived functors are well defined on Hom-sets, generalising the fundamental theorem of homological algebra.
\begin{thm*}
A Quillen adjunction induces a total derived adjunction \smash{$\xymatrix@R=.3cm@C=1cm{
\ho\calC  \ar@<.6ex>[r]^{LF}&
\ho\calD  \ar@<.4ex>[l]^{RG}
}$} which is an equivalence when the Quillen adjunction is a Quillen equivalence.
\end{thm*}

\subsection*{Producing model structures}
The hypotheses in the following definition are easy to check. Combinatorial model categories are easy to compute with, and very prevalent:
\begin{defn*}
A combinatorial model category is a model category $\calC$, which is:
\begin{enumerate}\squishlist
\item \textbf{Locally presentable:} 
\begin{itemise}
\item Locally small (the hom classes are sets);
\item There exists a set $S$ of objects generating $\calC$ under colimits such that every object of $S$ is ``small''. 
\begin{itemise}
\item An object $s$ is small \Iff it is small with respect to some regular cardinal $\kappa$. $s$ is small w.r.t.\ a regular cardinal $\kappa$ (and thus w.r.t.\ all larger regular cardinals) if the following natural map is always bijective.
Suppose we have a poset $J$ such that every subposet $L\subset J$ of cardinality $<\kappa$ has an upper bound in $J$ (a $\kappa$-filtered category). Then for all diagrams $J\to \calC$, there is a natural map 
\[\varinjlim_J \Hom_\calC(X,F(j))\to\Hom(X,\varinjlim_J F(j)).\]
This is the map which we require to be a bijection.
\end{itemise}
\end{itemise}
\item \textbf{Cofibrantly generated}: there exist sets $I,J$ such that $I$ and $J$ `generate' the cofibrations and trivial cofibrations respectively.
\begin{itemise}
\item If $I$ is a set, write $\textup{call}(I)$ for the class of things which can be obtained using pushouts and (transfinite) compositions. Then write $\textup{cof}(I)$ for the set of retracts of $\textup{cell}(I)$. This is what we mean by generate.
\end{itemise}
For example, in $s\mathsf{Set}$, $I$ can be taken to be the set of boundary inclusions $\partial\Delta^n\to \Delta^n$ of simplices.
\end{enumerate}
\end{defn*}
\begin{thm*}[Jeff Smith]
Suppose that $\calC$ is a locally presentable bicomplete category with a specified \textbf{set} $I\subset\Mor\calC$, and \textbf{class} $W\subset\Mor\calC$.\footnote{The hypotheses up to this footnote are not difficult to check.} Suppose also that:
\begin{enumerate}\squishlist
\item $\textup{Arr}_W(\calC)\subset\textup{Arr}(\calC)$ is an accessible inclusion;
\item $W$ satisfies two out of three;
\item $I^\pitchfork\subset W$; and
\item $\textup{cof}(I)\cap W$ is closed under co-base changes and transfinite composition.
\end{enumerate}
Then $\calC$ is a model category with weak equivalences $W$ and cofibrations $\textup{cof}(I)$.
\end{thm*}
\begin{proof}
The small object argument.
\end{proof}
\noindent The class of such model categories is a very favorable place in which to work. In particular, there is a nice theory of Bousfield localisation for such model categories.
\pagebreak
\end{JeremyModelCategories}
\begin{JeremyRationalHomotopyPractice}
\KanSemResponse
{``Rational Homotopy Theory'' --- Jeremy}
The idea: homotopy theory is hard, but using Bousfield localisation, we might be able to make it more tractable, albeit less sensitive.

We'll study homotopy groups tensored with $\Q$. As it's strange to try and form $\pi_1(X)\otimes\Q$ when $\pi_1(X)$ is non-abelian, we'll only study simply connected pointed spaces $X$.

That is, we'd like to use the category $\mathsf{Top}_2$, consisting of simply-connected pointed spaces. However, this cannot be a model category --- it does not admit (even finite) limits.\footnote{The equaliser of the two hemisphere inclusions $\xymatrix{D^2\ar@<1mm>[r]\ar@<-.5mm>[r]&S^2}$ is the circle.}
In order to fix this problem, we'll study the category $s\mathsf{Set}_2$ of ``2-reduced simplicial sets'':. 
\begin{defn*}
The category $s\mathsf{Set}_2$ of ``2-reduced simplicial sets'' is the full subcategory of $s\mathsf{Set}$ whose objects are those simplicial sets with exactly one 0-simplex and one 1-simplex.
\end{defn*}
\noindent This is not so inappropriate, as given a simply-connected space, a minimal subcomplex of the singular complex will be an object of this category.
\begin{prop*}
There is a model structure on $s\mathsf{Set}_2$, whose cofibrations are monomorphisms, whose weak equivalences are as in $s\mathsf{Set}$ (maps whose realisation is a weak equivalence), and whose fibrations are those Kan fibrations with the right lifting property with respect to $*\to\Delta^1/\partial\Delta^1$. 
\end{prop*}
\subsection*{Bousfield localisation}
Suppose that $M$ is a model category (e.g.\ $s\mathsf{Set}_2$), and $S$ is a class in $\Mor M$, (e.g.\ the rational equivalences). We can construct a new model category $M_{\textup{loc}}$ with the same objects, same cofibrations, more weak equivalences (including the elements of $S$) and fewer fibrations. 
\begin{prop*}
The Bousfield localisation has the following properties:
\begin{enumerate}\squishlist
\item $(M_{\textup{loc}})_{\textup{cf}}$ embeds in $M_{\textup{cf}}$ as a subcategory.
\item Every weak equivalence in  $(M_{\textup{loc}})_{\textup{cf}}$ is in fact a weak equivalence in $M$.
\item There's a Quillen adjunction:
\smash{$\xymatrix@R=.3cm@C=1cm{
M  \ar@<.6ex>[r]^{\id}&
M_\textup{loc}  \ar@<.4ex>[l]^{\id}\\
%X \ar@{|->}[r] & FX\\
%GY             & Y \ar@{|->}[l]
}$}
\end{enumerate}
\end{prop*}
\begin{defn*}
$X$ is \emph{rational} if $\pi_*(X)$ is a $\Q$-vector space (i.e.\ if $X$ is cofibrant-fibrant in the localised model structure \rednote{In what category?}.)
\end{defn*}
The following demonstrates why the problem of computing rationally is so tractable.
\begin{thm*}
The following are equivalent:
\begin{enumerate}\squishlist
\item $X$ is rational;
\item $\pi_*(X)$ is a $\Q$-vector space;
\item $\widetilde{H}_{*}(X;\Z)$ is a $\Q$-vector space;
\item $\widetilde{H}_{*}(\Omega X;\Z)$ is a $\Q$-vector space;
\end{enumerate}
\end{thm*}
\begin{proof}
$2\iff 3$ is the only interesting part. For this, we note that ${H}^{*}(K(\Q,1);\F_p)$ is trivial for all primes $p$, which implies, via the SSS, that ${H}^{*}(K(\Q,n);\F_p)$ is trivial for all primes $p$. To finish to proof, we look hard at the Postnikov tower.
\end{proof}
\begin{exmps*}
\begin{enumerate}\squishlist
\item The rational circle: the mapping telescope of the maps
\[S^1\overset{1}{\to}S^1\overset{2}{\to}S^1\overset{3}{\to}S^1\overset{4}{\to}\cdots\]
Each stage $X_r$ of the telescope retracts onto $S^1$, but the map $X_r\to X_{r+1}$ is multiplication by $r$. As homotopy (and homology) commute with colimits along cofibrations, we must then obtain a rational space, in fact a $K(\Q,1)$.
\item This can also be done with higher spheres, to obtain a spcae $S^n_0$.
\item We can form a ``rational $(n+1)$-disk'' via $D_0^{n+1}:=CS^n_0$. Given a CW-complex $X$, we can then replace every cell with rational cells, to obtain a space $X_0$ which is rational. There's a map $X\to X_0$, universal for maps from $X$ to a rational space \rednote{category of spaces here?}. Moreover, $X\to X_0$ is cofibrant-fibrant replacement. \rednote{again, what cat?}
\end{enumerate}
\end{exmps*}
\begin{thm*}
A map $f:X\to Y$ is a rational equivalence \Iff $H_{*}(f;\Q)$ is an isomorphism.
\end{thm*}
\begin{proof}
CW-approximate $X$ and $Y$. Use the universal property to obtain a map $X_0\to Y_0$. Use the above theorem: $2\iff 3$.
\end{proof}
\subsection*{A zig-zag of Quillen equivalences}
We would like to find a purely algebraic category to which $s\mathsf{Set}_2$ is Quillen equivalent. Quillen obtains the following Quillen equivalences, in which the left adjoint is always drawn as the top arrow:
\[\xymatrix@R=.3cm@C=1cm{
(s\mathsf{Set}_2)_\textup{loc}  \ar@<.6ex>[r]^{G}
&(s\mathsf{Gp}_1)_\textup{loc}  \ar@<.4ex>[l]^{\overline{W}}\ar@<.6ex>[r]^{\hat\Q}
&s\mathsf{CHA}_1  \ar@<.4ex>[l]^{\scrG}\ar@<-.4ex>[r]_{\scrP}
&s\mathsf{LA}_1  \ar@<-.6ex>[l]_{\hat U}\ar@<-.4ex>[r]_{N}
&\mathsf{DGL}_1  \ar@<-.6ex>[l]_{N^*}\ar@<-.4ex>[r]_{\scrC}
&\mathsf{DGC}_2  \ar@<-.6ex>[l]_{\scrL}
}\]
These categories are, in order:
\begin{itemise}
\item the localisation of 2-reduced simplicial sets, 
\item the localisation of reduced simplicial groups (simplicial groups $G_*$ s.t.\ $G_0=\{e\}$),
\item reduced simplicial complete Hopf algebras over $\Q$ ($R_*$ s.t.\ $R_0\simeq\Q$).
\item reduced simplicial Lie algebras over $\Q$.
\item reduced differential graded Lie algebras over $\Q$.
\item 2-reduced differential graded (cocommutative coassociative) coalgebras over $\Q$.
\end{itemise}
Moreover, there is an adjunction {$\xymatrix@R=.3cm@C=1.5cm{
\mathsf{Top}_2  \ar@<-.4ex>[r];[]_{|\cdot|}&
s\mathsf{Set}_2  \ar@<-.6ex>[l];[]_{E_2\Sing}
}
$} inducing an isomorphism on rational homotopy categories, even though $\mathsf{Top}_2$ is not model. The functor $E_2:s\mathsf{Set}\to s\mathsf{Set}_2$ is the 2-reduction of a simplicial set, which returns the sub-simplicial set of simplices whose 1-skeleton lies at the basepoint.
\subsection*{The five adjunctions}
\begin{enumerate}\squishlist
\item $G$ is the Kan loop group, and $\overline{W}$ is its right adjoint, analogous to the classifying space functor.
\item Given a group $G$, one can form the rational group algebra $A=\Q[G]$, which is a cocommutative Hopf algebra with coproduct sending $g\mapsto g\otimes g$, and counit sending $g\mapsto1$. We can recover $G$ from $A$ as the set of grouplike elements: $\{x\in A\,|\,\Delta x=x\otimes x,\,\epsilon x=1\}$.

In general, taking the grouplike elements of a cocommutative Hopf algebra returns a group (the antipode provides the inverse of a grouplike element), providing a functor $\scrG:\mathsf{CHA}\to\mathsf{Gp}$.

\INDENT The adjoint functor in the other direction is the group algebra functor followed by completion at the augmentation ideal.
\item Given a Lie algebra $L$, one can form the universal enveloping algebra $U(L)$. This can be made into a Hopf algebra by defining $\Delta(l)=l\otimes1+1\otimes l$, and $\epsilon(l)=0$. We can recover the Lie algebra as those elements satisfying these two equations.

\INDENT In general, the set of elements satisfying these equations forms a Lie algebra, providing another of the above functors. Its adjoint is the universal enveloping algebra followed by completion.
\item This is some kind of beefed up Dold-Kan correspondence, which restricts to an equivalence between $s(\mathsf{Mod}_\Q)$ and $\mathsf{Ch}^{\Q}_{\geq0}$. The challenge is to drag the extra structure from one side to the other. This is the part where we really need to be working in characteristic zero. You need to do something clever with the Eilenberg-Zilber map.
\item A little mysterious at the moment.
\end{enumerate}

\pagebreak
\end{JeremyRationalHomotopyPractice}
\begin{KTheoryQuillen}
\KanSemResponse
{``Higher algebraic K-theory, I'' --- Quillen}
There was a lot to pick up from this paper. Firstly, it's nice to see a better way to view the concept of $K$-theory. It's an invariant associated with a category.

\pagebreak
\end{KTheoryQuillen}
\begin{RationalHomotopyQuillen}
\KanSemResponse
{``Rational homotopy theory'' --- Quillen}
This paper gives me my first ever reason to be fond of a variety of algebraic constructions facts. For example, the fact that the universal enveloping algebra of a Lie algebra is a Hopf algebra, or the fact that the grouplike elements of a Hopf algebra form a group!

The idea is that all of these constructions form a chain of Quillen equivalences between the category of 2-reduced simplicial sets after Bousfield localising at the rational equivalences, and a purely algebraic category. It seems that `purely algebraic' means `devoid of the word simplicial', which is good enough for me.

To link this back to spaces, Quillen forms an adjunction between the category of simply connected spaces and that of 2-reduced simplicial sets. This cannot be a Quillen adjunction, as the former category is not a model category. However, the former still has a notion of weak equivalence, and thus a notion of homotopy category. He proves directly that there are induced maps on homotopy categories which are inverse equivalences. Anyway, it's a little difficult to quickly pick out exactly the model structures on all of these categories from part I --- but Quillen goes on in II to explain all the model structures involved.

Worthy of note is the construction of a model category structure on $s\scrA$, where $\scrA$ is a category with finite limits and with enough projectives, satisfying one of two conditions. (Here, projective can be given a definition which makes sense in any category). In fact, this can be extended to give a model structure on the $r$-reduced category, $(s\scrA)_r$.

The construction of the adjunctions and the verification of their properties takes a significant part of this paper. What is also very interesting, and not to be overlooked, is sections I.5 and I.6, in which Quillen relates these constructions to other structures ---  the Whitehead product, and the coalgebra structure on homology.

Although now I must put this paper down in favour of other more onerous duties, I'll revisit it over the break in more detail.



\pagebreak
\end{RationalHomotopyQuillen}

\end{document}


















% !TEX root = z_output/_KanSeminar.tex
%%%%%%%%%%%%%%%%%%%%%%%%%%%%%%%%%%%%%%%%%%%%%%%%%%%%%%%%%%%%%%%%%%%%%%%%%%%%%%%%
%%%%%%%%%%%%%%%%%%%%%%%%%%% 80 characters %%%%%%%%%%%%%%%%%%%%%%%%%%%%%%%%%%%%%%
%%%%%%%%%%%%%%%%%%%%%%%%%%%%%%%%%%%%%%%%%%%%%%%%%%%%%%%%%%%%%%%%%%%%%%%%%%%%%%%%
\documentclass[11pt]{article}
\usepackage{fullpage}
\usepackage{amsmath,amsthm,amssymb}
\usepackage{mathrsfs,nicefrac}
\usepackage{amssymb}
\usepackage{epsfig}
\usepackage[all,2cell]{xy}
\usepackage{sseq}
\usepackage{tocloft}
\usepackage{cancel}
\usepackage[strict]{changepage}
\usepackage{color}
\usepackage{tikz}
\usepackage{extpfeil}
\usepackage{version}
\usepackage{framed}
\definecolor{shadecolor}{rgb}{.925,0.925,0.925}

%\usepackage{ifthen}
%Used for disabling hyperref
\ifx\dontloadhyperref\undefined
%\usepackage[pdftex,pdfborder={0 0 0 [1 1]}]{hyperref}
\usepackage[pdftex,pdfborder={0 0 .5 [1 1]}]{hyperref}
\else
\providecommand{\texorpdfstring}[2]{#1}
\fi
%>>>>>>>>>>>>>>>>>>>>>>>>>>>>>>
%<<<        Versions        <<<
%>>>>>>>>>>>>>>>>>>>>>>>>>>>>>>
%Add in the following line to include all the versions.
%\def\excludeversion#1{\includeversion{#1}}

%>>>>>>>>>>>>>>>>>>>>>>>>>>>>>>
%<<<       Better ToC       <<<
%>>>>>>>>>>>>>>>>>>>>>>>>>>>>>>
\setlength{\cftbeforesecskip}{0.5ex}

%>>>>>>>>>>>>>>>>>>>>>>>>>>>>>>
%<<<      Hyperref mod      <<<
%>>>>>>>>>>>>>>>>>>>>>>>>>>>>>>

%needs more testing
\newcounter{dummyforrefstepcounter}
\newcommand{\labelRIGHTHERE}[1]
{\refstepcounter{dummyforrefstepcounter}\label{#1}}


%>>>>>>>>>>>>>>>>>>>>>>>>>>>>>>
%<<<  Theorem Environments  <<<
%>>>>>>>>>>>>>>>>>>>>>>>>>>>>>>
\ifx\dontloaddefinitionsoftheoremenvironments\undefined
\theoremstyle{plain}
\newtheorem{thm}{Theorem}[section]
\newtheorem*{thm*}{Theorem}
\newtheorem{lem}[thm]{Lemma}
\newtheorem*{lem*}{Lemma}
\newtheorem{prop}[thm]{Proposition}
\newtheorem*{prop*}{Proposition}
\newtheorem{cor}[thm]{Corollary}
\newtheorem*{cor*}{Corollary}
\newtheorem{defprop}[thm]{Definition-Proposition}
\newtheorem*{punchline}{Punchline}
\newtheorem*{conjecture}{Conjecture}
\newtheorem*{claim}{Claim}

\theoremstyle{definition}
\newtheorem{defn}{Definition}[section]
\newtheorem*{defn*}{Definition}
\newtheorem{exmp}{Example}[section]
\newtheorem*{exmp*}{Example}
\newtheorem*{exmps*}{Examples}
\newtheorem*{nonexmp*}{Non-example}
\newtheorem{asspt}{Assumption}[section]
\newtheorem{notation}{Notation}[section]
\newtheorem{exercise}{Exercise}[section]
\newtheorem*{fact*}{Fact}
\newtheorem*{rmk*}{Remark}
\newtheorem{fact}{Fact}
\newtheorem*{aside}{Aside}
\newtheorem*{question}{Question}
\newtheorem*{answer}{Answer}

\else\relax\fi

%>>>>>>>>>>>>>>>>>>>>>>>>>>>>>>
%<<<      Fields, etc.      <<<
%>>>>>>>>>>>>>>>>>>>>>>>>>>>>>>
\DeclareSymbolFont{AMSb}{U}{msb}{m}{n}
\DeclareMathSymbol{\N}{\mathbin}{AMSb}{"4E}
\DeclareMathSymbol{\Octonions}{\mathbin}{AMSb}{"4F}
\DeclareMathSymbol{\Z}{\mathbin}{AMSb}{"5A}
\DeclareMathSymbol{\R}{\mathbin}{AMSb}{"52}
\DeclareMathSymbol{\Q}{\mathbin}{AMSb}{"51}
\DeclareMathSymbol{\PP}{\mathbin}{AMSb}{"50}
\DeclareMathSymbol{\I}{\mathbin}{AMSb}{"49}
\DeclareMathSymbol{\C}{\mathbin}{AMSb}{"43}
\DeclareMathSymbol{\A}{\mathbin}{AMSb}{"41}
\DeclareMathSymbol{\F}{\mathbin}{AMSb}{"46}
\DeclareMathSymbol{\G}{\mathbin}{AMSb}{"47}
\DeclareMathSymbol{\Quaternions}{\mathbin}{AMSb}{"48}


%>>>>>>>>>>>>>>>>>>>>>>>>>>>>>>
%<<<       Operators        <<<
%>>>>>>>>>>>>>>>>>>>>>>>>>>>>>>
\DeclareMathOperator{\ad}{\textbf{ad}}
\DeclareMathOperator{\coker}{coker}
\renewcommand{\ker}{\textup{ker}\,}
\DeclareMathOperator{\End}{End}
\DeclareMathOperator{\Aut}{Aut}
\DeclareMathOperator{\Hom}{Hom}
\DeclareMathOperator{\Maps}{Maps}
\DeclareMathOperator{\Mor}{Mor}
\DeclareMathOperator{\Gal}{Gal}
\DeclareMathOperator{\Ext}{Ext}
\DeclareMathOperator{\Tor}{Tor}
\DeclareMathOperator{\Map}{Map}
\DeclareMathOperator{\Der}{Der}
\DeclareMathOperator{\Rad}{Rad}
\DeclareMathOperator{\rank}{rank}
\DeclareMathOperator{\ArfInvariant}{Arf}
\DeclareMathOperator{\KervaireInvariant}{Ker}
\DeclareMathOperator{\im}{im}
\DeclareMathOperator{\coim}{coim}
\DeclareMathOperator{\trace}{tr}
\DeclareMathOperator{\supp}{supp}
\DeclareMathOperator{\ann}{ann}
\DeclareMathOperator{\spec}{Spec}
\DeclareMathOperator{\SPEC}{\textbf{Spec}}
\DeclareMathOperator{\proj}{Proj}
\DeclareMathOperator{\PROJ}{\textbf{Proj}}
\DeclareMathOperator{\fiber}{F}
\DeclareMathOperator{\cofiber}{C}
\DeclareMathOperator{\cone}{cone}
\DeclareMathOperator{\skel}{sk}
\DeclareMathOperator{\coskel}{cosk}
\DeclareMathOperator{\conn}{conn}
\DeclareMathOperator{\colim}{colim}
\DeclareMathOperator{\limit}{lim}
\DeclareMathOperator{\ch}{ch}
\DeclareMathOperator{\Vect}{Vect}
\DeclareMathOperator{\GrthGrp}{GrthGp}
\DeclareMathOperator{\Sym}{Sym}
\DeclareMathOperator{\Prob}{\mathbb{P}}
\DeclareMathOperator{\Exp}{\mathbb{E}}
\DeclareMathOperator{\GeomMean}{\mathbb{G}}
\DeclareMathOperator{\Var}{Var}
\DeclareMathOperator{\Cov}{Cov}
\DeclareMathOperator{\Sp}{Sp}
\DeclareMathOperator{\Seq}{Seq}
\DeclareMathOperator{\Cyl}{Cyl}
\DeclareMathOperator{\Ev}{Ev}
\DeclareMathOperator{\sh}{sh}
\DeclareMathOperator{\intHom}{\underline{Hom}}
\DeclareMathOperator{\Frac}{frac}



%>>>>>>>>>>>>>>>>>>>>>>>>>>>>>>
%<<<   Cohomology Theories  <<<
%>>>>>>>>>>>>>>>>>>>>>>>>>>>>>>
\DeclareMathOperator{\KR}{{K\R}}
\DeclareMathOperator{\KO}{{KO}}
\DeclareMathOperator{\K}{{K}}
\DeclareMathOperator{\OmegaO}{{\Omega_{\Octonions}}}

%>>>>>>>>>>>>>>>>>>>>>>>>>>>>>>
%<<<   Algebraic Geometry   <<<
%>>>>>>>>>>>>>>>>>>>>>>>>>>>>>>
\DeclareMathOperator{\Spec}{Spec}
\DeclareMathOperator{\Proj}{Proj}
\DeclareMathOperator{\Sing}{Sing}
\DeclareMathOperator{\shfHom}{\mathscr{H}\textit{\!\!om}}
\DeclareMathOperator{\WeilDivisors}{{Div}}
\DeclareMathOperator{\CartierDivisors}{{CaDiv}}
\DeclareMathOperator{\PrincipalWeilDivisors}{{PrDiv}}
\DeclareMathOperator{\LocallyPrincipalWeilDivisors}{{LPDiv}}
\DeclareMathOperator{\PrincipalCartierDivisors}{{PrCaDiv}}
\DeclareMathOperator{\DivisorClass}{{Cl}}
\DeclareMathOperator{\CartierClass}{{CaCl}}
\DeclareMathOperator{\Picard}{{Pic}}
\DeclareMathOperator{\Frob}{Frob}


%>>>>>>>>>>>>>>>>>>>>>>>>>>>>>>
%<<<  Mathematical Objects  <<<
%>>>>>>>>>>>>>>>>>>>>>>>>>>>>>>
\newcommand{\sll}{\mathfrak{sl}}
\newcommand{\gl}{\mathfrak{gl}}
\newcommand{\GL}{\mbox{GL}}
\newcommand{\PGL}{\mbox{PGL}}
\newcommand{\SL}{\mbox{SL}}
\newcommand{\Mat}{\mbox{Mat}}
\newcommand{\Gr}{\textup{Gr}}
\newcommand{\Squ}{\textup{Sq}}
\newcommand{\catSet}{\textit{Sets}}
\newcommand{\RP}{{\R\PP}}
\newcommand{\CP}{{\C\PP}}
\newcommand{\Steen}{\mathscr{A}}
\newcommand{\Orth}{\textup{\textbf{O}}}

%>>>>>>>>>>>>>>>>>>>>>>>>>>>>>>
%<<<  Mathematical Symbols  <<<
%>>>>>>>>>>>>>>>>>>>>>>>>>>>>>>
\newcommand{\DASH}{\textup{---}}
\newcommand{\op}{\textup{op}}
\newcommand{\CW}{\textup{CW}}
\newcommand{\ob}{\textup{ob}\,}
\newcommand{\ho}{\textup{ho}}
\newcommand{\st}{\textup{st}}
\newcommand{\id}{\textup{id}}
\newcommand{\Bullet}{\ensuremath{\bullet} }
\newcommand{\sprod}{\wedge}

%>>>>>>>>>>>>>>>>>>>>>>>>>>>>>>
%<<<      Some Arrows       <<<
%>>>>>>>>>>>>>>>>>>>>>>>>>>>>>>
\newcommand{\nt}{\Longrightarrow}
\let\shortmapsto\mapsto
\let\mapsto\longmapsto
\newcommand{\mapsfrom}{\,\reflectbox{$\mapsto$}\ }
\newcommand{\bigrightsquig}{\scalebox{2}{\ensuremath{\rightsquigarrow}}}
\newcommand{\bigleftsquig}{\reflectbox{\scalebox{2}{\ensuremath{\rightsquigarrow}}}}

%\newcommand{\cofibration}{\xhookrightarrow{\phantom{\ \,{\sim\!}\ \ }}}
%\newcommand{\fibration}{\xtwoheadrightarrow{\phantom{\sim\!}}}
%\newcommand{\acycliccofibration}{\xhookrightarrow{\ \,{\sim\!}\ \ }}
%\newcommand{\acyclicfibration}{\xtwoheadrightarrow{\sim\!}}
%\newcommand{\leftcofibration}{\xhookleftarrow{\phantom{\ \,{\sim\!}\ \ }}}
%\newcommand{\leftfibration}{\xtwoheadleftarrow{\phantom{\sim\!}}}
%\newcommand{\leftacycliccofibration}{\xhookleftarrow{\ \ {\sim\!}\,\ }}
%\newcommand{\leftacyclicfibration}{\xtwoheadleftarrow{\sim\!}}
%\newcommand{\weakequiv}{\xrightarrow{\ \,\sim\,\ }}
%\newcommand{\leftweakequiv}{\xleftarrow{\ \,\sim\,\ }}

\newcommand{\cofibration}
{\xhookrightarrow{\phantom{\ \,{\raisebox{-.3ex}[0ex][0ex]{\scriptsize$\sim$}\!}\ \ }}}
\newcommand{\fibration}
{\xtwoheadrightarrow{\phantom{\raisebox{-.3ex}[0ex][0ex]{\scriptsize$\sim$}\!}}}
\newcommand{\acycliccofibration}
{\xhookrightarrow{\ \,{\raisebox{-.55ex}[0ex][0ex]{\scriptsize$\sim$}\!}\ \ }}
\newcommand{\acyclicfibration}
{\xtwoheadrightarrow{\raisebox{-.6ex}[0ex][0ex]{\scriptsize$\sim$}\!}}
\newcommand{\leftcofibration}
{\xhookleftarrow{\phantom{\ \,{\raisebox{-.3ex}[0ex][0ex]{\scriptsize$\sim$}\!}\ \ }}}
\newcommand{\leftfibration}
{\xtwoheadleftarrow{\phantom{\raisebox{-.3ex}[0ex][0ex]{\scriptsize$\sim$}\!}}}
\newcommand{\leftacycliccofibration}
{\xhookleftarrow{\ \ {\raisebox{-.55ex}[0ex][0ex]{\scriptsize$\sim$}\!}\,\ }}
\newcommand{\leftacyclicfibration}
{\xtwoheadleftarrow{\raisebox{-.6ex}[0ex][0ex]{\scriptsize$\sim$}\!}}
\newcommand{\weakequiv}
{\xrightarrow{\ \,\raisebox{-.3ex}[0ex][0ex]{\scriptsize$\sim$}\,\ }}
\newcommand{\leftweakequiv}
{\xleftarrow{\ \,\raisebox{-.3ex}[0ex][0ex]{\scriptsize$\sim$}\,\ }}

%>>>>>>>>>>>>>>>>>>>>>>>>>>>>>>
%<<<    xymatrix Arrows     <<<
%>>>>>>>>>>>>>>>>>>>>>>>>>>>>>>
\newdir{ >}{{}*!/-5pt/@{>}}
\newcommand{\xycof}{\ar@{ >->}}
\newcommand{\xycofib}{\ar@{^{(}->}}
\newcommand{\xycofibdown}{\ar@{_{(}->}}
\newcommand{\xyfib}{\ar@{->>}}
\newcommand{\xymapsto}{\ar@{|->}}

%>>>>>>>>>>>>>>>>>>>>>>>>>>>>>>
%<<<     Greek Letters      <<<
%>>>>>>>>>>>>>>>>>>>>>>>>>>>>>>
%\newcommand{\oldphi}{\phi}
%\renewcommand{\phi}{\varphi}
\let\oldphi\phi
\let\phi\varphi
\renewcommand{\to}{\longrightarrow}
\newcommand{\from}{\longleftarrow}
\newcommand{\eps}{\varepsilon}

%>>>>>>>>>>>>>>>>>>>>>>>>>>>>>>
%<<<  1st-4th & parentheses <<<
%>>>>>>>>>>>>>>>>>>>>>>>>>>>>>>
\newcommand{\first}{^\text{st}}
\newcommand{\second}{^\text{nd}}
\newcommand{\third}{^\text{rd}}
\newcommand{\fourth}{^\text{th}}
\newcommand{\ZEROTH}{$0^\text{th}$ }
\newcommand{\FIRST}{$1^\text{st}$ }
\newcommand{\SECOND}{$2^\text{nd}$ }
\newcommand{\THIRD}{$3^\text{rd}$ }
\newcommand{\FOURTH}{$4^\text{th}$ }
\newcommand{\iTH}{$i^\text{th}$ }
\newcommand{\jTH}{$j^\text{th}$ }
\newcommand{\nTH}{$n^\text{th}$ }

%>>>>>>>>>>>>>>>>>>>>>>>>>>>>>>
%<<<    upright commands    <<<
%>>>>>>>>>>>>>>>>>>>>>>>>>>>>>>
\newcommand{\upcol}{\textup{:}}
\newcommand{\upsemi}{\textup{;}}
\providecommand{\lparen}{\textup{(}}
\providecommand{\rparen}{\textup{)}}
\renewcommand{\lparen}{\textup{(}}
\renewcommand{\rparen}{\textup{)}}
\newcommand{\Iff}{\emph{iff} }

%>>>>>>>>>>>>>>>>>>>>>>>>>>>>>>
%<<<     Environments       <<<
%>>>>>>>>>>>>>>>>>>>>>>>>>>>>>>
\newcommand{\squishlist}
{ %\setlength{\topsep}{100pt} doesn't seem to do anything.
  \setlength{\itemsep}{.5pt}
  \setlength{\parskip}{0pt}
  \setlength{\parsep}{0pt}}
\newenvironment{itemise}{
\begin{list}{\textup{$\rightsquigarrow$}}
   {  \setlength{\topsep}{1mm}
      \setlength{\itemsep}{1pt}
      \setlength{\parskip}{0pt}
      \setlength{\parsep}{0pt}
   }
}{\end{list}\vspace{-.1cm}}
\newcommand{\INDENT}{\textbf{}\phantom{space}}
\renewcommand{\INDENT}{\rule{.7cm}{0cm}}

\newcommand{\itm}[1][$\rightsquigarrow$]{\item[{\makebox[.5cm][c]{\textup{#1}}}]}


%\newcommand{\rednote}[1]{{\color{red}#1}\makebox[0cm][l]{\scalebox{.1}{rednote}}}
%\newcommand{\bluenote}[1]{{\color{blue}#1}\makebox[0cm][l]{\scalebox{.1}{rednote}}}

\newcommand{\rednote}[1]
{{\color{red}#1}\makebox[0cm][l]{\scalebox{.1}{\rotatebox{90}{?????}}}}
\newcommand{\bluenote}[1]
{{\color{blue}#1}\makebox[0cm][l]{\scalebox{.1}{\rotatebox{90}{?????}}}}


\newcommand{\funcdef}[4]{\begin{align*}
#1&\to #2\\
#3&\mapsto#4
\end{align*}}

%\newcommand{\comment}[1]{}

%>>>>>>>>>>>>>>>>>>>>>>>>>>>>>>
%<<<       Categories       <<<
%>>>>>>>>>>>>>>>>>>>>>>>>>>>>>>
\newcommand{\Ens}{{\mathscr{E}ns}}
\DeclareMathOperator{\Sheaves}{{\mathsf{Shf}}}
\DeclareMathOperator{\Presheaves}{{\mathsf{PreShf}}}
\DeclareMathOperator{\Psh}{{\mathsf{Psh}}}
\DeclareMathOperator{\Shf}{{\mathsf{Shf}}}
\DeclareMathOperator{\Varieties}{{\mathsf{Var}}}
\DeclareMathOperator{\Schemes}{{\mathsf{Sch}}}
\DeclareMathOperator{\Rings}{{\mathsf{Rings}}}
\DeclareMathOperator{\AbGp}{{\mathsf{AbGp}}}
\DeclareMathOperator{\Modules}{{\mathsf{\!-Mod}}}
\DeclareMathOperator{\fgModules}{{\mathsf{\!-Mod}^{\textup{fg}}}}
\DeclareMathOperator{\QuasiCoherent}{{\mathsf{QCoh}}}
\DeclareMathOperator{\Coherent}{{\mathsf{Coh}}}
\DeclareMathOperator{\GSW}{{\mathcal{SW}^G}}
\DeclareMathOperator{\Burnside}{{\mathsf{Burn}}}
\DeclareMathOperator{\GSet}{{G\mathsf{Set}}}
\DeclareMathOperator{\FinGSet}{{G\mathsf{Set}^\textup{fin}}}
\DeclareMathOperator{\HSet}{{H\mathsf{Set}}}
\DeclareMathOperator{\Cat}{{\mathsf{Cat}}}
\DeclareMathOperator{\Fun}{{\mathsf{Fun}}}
\DeclareMathOperator{\Orb}{{\mathsf{Orb}}}
\DeclareMathOperator{\Set}{{\mathsf{Set}}}
\DeclareMathOperator{\sSet}{{\mathsf{sSet}}}
\DeclareMathOperator{\Top}{{\mathsf{Top}}}
\DeclareMathOperator{\GSpectra}{{G-\mathsf{Spectra}}}
\DeclareMathOperator{\Lan}{Lan}
\DeclareMathOperator{\Ran}{Ran}

%>>>>>>>>>>>>>>>>>>>>>>>>>>>>>>
%<<<     Script Letters     <<<
%>>>>>>>>>>>>>>>>>>>>>>>>>>>>>>
\newcommand{\scrQ}{\mathscr{Q}}
\newcommand{\scrW}{\mathscr{W}}
\newcommand{\scrE}{\mathscr{E}}
\newcommand{\scrR}{\mathscr{R}}
\newcommand{\scrT}{\mathscr{T}}
\newcommand{\scrY}{\mathscr{Y}}
\newcommand{\scrU}{\mathscr{U}}
\newcommand{\scrI}{\mathscr{I}}
\newcommand{\scrO}{\mathscr{O}}
\newcommand{\scrP}{\mathscr{P}}
\newcommand{\scrA}{\mathscr{A}}
\newcommand{\scrS}{\mathscr{S}}
\newcommand{\scrD}{\mathscr{D}}
\newcommand{\scrF}{\mathscr{F}}
\newcommand{\scrG}{\mathscr{G}}
\newcommand{\scrH}{\mathscr{H}}
\newcommand{\scrJ}{\mathscr{J}}
\newcommand{\scrK}{\mathscr{K}}
\newcommand{\scrL}{\mathscr{L}}
\newcommand{\scrZ}{\mathscr{Z}}
\newcommand{\scrX}{\mathscr{X}}
\newcommand{\scrC}{\mathscr{C}}
\newcommand{\scrV}{\mathscr{V}}
\newcommand{\scrB}{\mathscr{B}}
\newcommand{\scrN}{\mathscr{N}}
\newcommand{\scrM}{\mathscr{M}}

%>>>>>>>>>>>>>>>>>>>>>>>>>>>>>>
%<<<     Fractur Letters    <<<
%>>>>>>>>>>>>>>>>>>>>>>>>>>>>>>
\newcommand{\frakQ}{\mathfrak{Q}}
\newcommand{\frakW}{\mathfrak{W}}
\newcommand{\frakE}{\mathfrak{E}}
\newcommand{\frakR}{\mathfrak{R}}
\newcommand{\frakT}{\mathfrak{T}}
\newcommand{\frakY}{\mathfrak{Y}}
\newcommand{\frakU}{\mathfrak{U}}
\newcommand{\frakI}{\mathfrak{I}}
\newcommand{\frakO}{\mathfrak{O}}
\newcommand{\frakP}{\mathfrak{P}}
\newcommand{\frakA}{\mathfrak{A}}
\newcommand{\frakS}{\mathfrak{S}}
\newcommand{\frakD}{\mathfrak{D}}
\newcommand{\frakF}{\mathfrak{F}}
\newcommand{\frakG}{\mathfrak{G}}
\newcommand{\frakH}{\mathfrak{H}}
\newcommand{\frakJ}{\mathfrak{J}}
\newcommand{\frakK}{\mathfrak{K}}
\newcommand{\frakL}{\mathfrak{L}}
\newcommand{\frakZ}{\mathfrak{Z}}
\newcommand{\frakX}{\mathfrak{X}}
\newcommand{\frakC}{\mathfrak{C}}
\newcommand{\frakV}{\mathfrak{V}}
\newcommand{\frakB}{\mathfrak{B}}
\newcommand{\frakN}{\mathfrak{N}}
\newcommand{\frakM}{\mathfrak{M}}

\newcommand{\frakq}{\mathfrak{q}}
\newcommand{\frakw}{\mathfrak{w}}
\newcommand{\frake}{\mathfrak{e}}
\newcommand{\frakr}{\mathfrak{r}}
\newcommand{\frakt}{\mathfrak{t}}
\newcommand{\fraky}{\mathfrak{y}}
\newcommand{\fraku}{\mathfrak{u}}
\newcommand{\fraki}{\mathfrak{i}}
\newcommand{\frako}{\mathfrak{o}}
\newcommand{\frakp}{\mathfrak{p}}
\newcommand{\fraka}{\mathfrak{a}}
\newcommand{\fraks}{\mathfrak{s}}
\newcommand{\frakd}{\mathfrak{d}}
\newcommand{\frakf}{\mathfrak{f}}
\newcommand{\frakg}{\mathfrak{g}}
\newcommand{\frakh}{\mathfrak{h}}
\newcommand{\frakj}{\mathfrak{j}}
\newcommand{\frakk}{\mathfrak{k}}
\newcommand{\frakl}{\mathfrak{l}}
\newcommand{\frakz}{\mathfrak{z}}
\newcommand{\frakx}{\mathfrak{x}}
\newcommand{\frakc}{\mathfrak{c}}
\newcommand{\frakv}{\mathfrak{v}}
\newcommand{\frakb}{\mathfrak{b}}
\newcommand{\frakn}{\mathfrak{n}}
\newcommand{\frakm}{\mathfrak{m}}

%>>>>>>>>>>>>>>>>>>>>>>>>>>>>>>
%<<<  Caligraphic Letters   <<<
%>>>>>>>>>>>>>>>>>>>>>>>>>>>>>>
\newcommand{\calQ}{\mathcal{Q}}
\newcommand{\calW}{\mathcal{W}}
\newcommand{\calE}{\mathcal{E}}
\newcommand{\calR}{\mathcal{R}}
\newcommand{\calT}{\mathcal{T}}
\newcommand{\calY}{\mathcal{Y}}
\newcommand{\calU}{\mathcal{U}}
\newcommand{\calI}{\mathcal{I}}
\newcommand{\calO}{\mathcal{O}}
\newcommand{\calP}{\mathcal{P}}
\newcommand{\calA}{\mathcal{A}}
\newcommand{\calS}{\mathcal{S}}
\newcommand{\calD}{\mathcal{D}}
\newcommand{\calF}{\mathcal{F}}
\newcommand{\calG}{\mathcal{G}}
\newcommand{\calH}{\mathcal{H}}
\newcommand{\calJ}{\mathcal{J}}
\newcommand{\calK}{\mathcal{K}}
\newcommand{\calL}{\mathcal{L}}
\newcommand{\calZ}{\mathcal{Z}}
\newcommand{\calX}{\mathcal{X}}
\newcommand{\calC}{\mathcal{C}}
\newcommand{\calV}{\mathcal{V}}
\newcommand{\calB}{\mathcal{B}}
\newcommand{\calN}{\mathcal{N}}
\newcommand{\calM}{\mathcal{M}}

%>>>>>>>>>>>>>>>>>>>>>>>>>>>>>>
%<<<<<<<<<DEPRECIATED<<<<<<<<<<
%>>>>>>>>>>>>>>>>>>>>>>>>>>>>>>

%%% From Kac's template
% 1-inch margins, from fullpage.sty by H.Partl, Version 2, Dec. 15, 1988.
%\topmargin 0pt
%\advance \topmargin by -\headheight
%\advance \topmargin by -\headsep
%\textheight 9.1in
%\oddsidemargin 0pt
%\evensidemargin \oddsidemargin
%\marginparwidth 0.5in
%\textwidth 6.5in
%
%\parindent 0in
%\parskip 1.5ex
%%\renewcommand{\baselinestretch}{1.25}

%%% From the net
%\newcommand{\pullbackcorner}[1][dr]{\save*!/#1+1.2pc/#1:(1,-1)@^{|-}\restore}
%\newcommand{\pushoutcorner}[1][dr]{\save*!/#1-1.2pc/#1:(-1,1)@^{|-}\restore}









\usepackage{fancyhdr}
\usepackage{lipsum}
\usepackage{comment}
\rfoot{\footnotesize Michael Donovan}


\newcommand{\TalkToOutput}{5} %`all' or an integer

\newcommand{\KanSemResponse}[3]%{number}{title}{response}
{
\ifthenelse{\equal{\TalkToOutput}{all} \OR \equal{\TalkToOutput}{#1}}
{
\thispagestyle{fancy}
\subsection*{#2}
#3

\pagebreak
}
{}
}

\begin{document}
\KanSemResponse{1}
{``Homologie singuli\`ere des espaces fibr\'es'' --- J.-P.\ Serre}
{
The first part of this paper consists of a clear exposition of the methods used thereafter to produce spectral sequences. All the technology that is to be used is displayed with such clarity that one feels almost equipped to finish the writing oneself. (Strangely,) I hadn't really been familiar with the derivation of a spectral sequence from a filtered differential graded algebra, as I had seen it before I had started to become comforatable with spectral sequences. Thus, reading this derivation was a pleasure.

I had been happy with the `births and deaths' interpretation of the Serre spectral sequence which we discussed, but I hadn't noticed that the differentials therein could be so closely connected with the actual differentials in a singlular (actually cubical) chain complex for the total space. This really does seem reasonable, on reflection.

As I have used the Serre spectral sequence before, it was a pleasure to see how Serre sets up the machinery. For example, the discussion in chapter 1 of the transgression greatly clarifies the standard description of the transgression in the Serre spectral sequence. To me, this description had been a black box, and the derivation given has relieved me of the notion that something complicated was going on.

In the second chapter, Serre defines cubical homology, which seems perfectly suited to his goals. Given a Serre fibration, he filters the cubical chains $u$ on the total space according to how deep a projection $I^n\to I^p$ the projection of $u$ to the base factors through. This has the virtue that a $p+q$-cube which lies in the $p^\text{th}$ filtration induces a $p$-cube on the base and a $q$-cube on the fibre (after assuming the base and fibre connected and adjusting the cubical complexes of total and base spaces to only include those with all corners at the basepoints). The induced chain homotopy between the $E_0$ page and the tensor product of chains on the base with those on the fibre make it much easier to properly understand the early pages of the spectral sequence.

The local coefficients are built seamlessly into the picture --- I had until now been afraid of them. In particular, after reading this, I am happier with the proof I may present on Wednesday regarding the homology of the $\pi$-adic construction, which uses the Serre spectral sequence with base space $B\pi$.

From this point the paper turns to applications, many familiar, some unfamiliar. On particularly charming idea is to take a space and repeatedly take the universal cover and then apply $\Omega$, to drag down the homotopy groups. It is this type of operation which Serre seems to apply often with admirable confidence. Anyhow, this process is used to show that the rational homotopy of odd spheres is zero, which is really a fantastic fact. This too relies on Serre's mod $\calC$ theory.

Along the same lines, it is shown where the first $p$-torsion arises in the homotopy groups of a sphere. Both of these proofs have evaded my complete understanding, as they are in French, and when I have a little more time, I will try to decode them.

Although I'm looking forward to reading about these applications in more detail, along with the other applications that I have not mentioned, I think that the main thing I will get out of this paper is clarity regarding the workings of the Serre spectral sequence. I felt recently that I had for the first time understood the differentials, using the births/deaths analogy which we discussed, coming from an exact triangle. Although this perspective is great when the coefficients are trivial, the $E_1$ page is not really canonical, and the multiplicative structure is a mystery. Now, however, this paper has opened up a route to understanding which is perhaps more direct, ``natural from $E_0$'' (as Michael Andrews will emphasise), and in which the multiplicative structure is finally transparent.
}
\KanSemResponse{2}
{A derivation of the Steenrod Squares}
{
Choose $p$ prime, $n>0$, $\pi\subseteq \Sigma_n$. Fix spaces $E\pi\overset{r}{\to} B\pi$, and basepoints $e\mapsto b$.
\[\xymatrix@R=.3cm{
X^{(n)}\ar[r]^i\ar@{=}[d]&D_\pi X\ar@{=}[d]&\ar[l]_jB\pi_+\wedge X\ar@{=}[d]\\
X^n/FW\ar[r]^{\text{incl}}&\frac{E\pi\times_\pi X^n}{E\pi\times_\pi FW}&
\ar[l]_\Delta\frac{B\pi\times X}{B\pi\times\{*\}}\\
y\ar@{|->}[r]&(e,y),\ (f,\Delta x)&\ar@{|->}[l](r(f),x)
}
\xymatrix@R=.3cm{
&&\widetilde H^q(X)\ar[lld]_{u\mapsto u^{\wedge n}}\ar[ld]^{P_\pi}
\ar[dd]_{\text{SQ}}\\
\widetilde H^{nq}(X^{(n)})&\ar[l]^{i^*}\widetilde H^{nq}(D_\pi X)
\ar[rd]^{j^*}\\
&&\widetilde H^{nq}(B\pi_+\wedge X)
}
\]
If $n=p=2$ and $\pi=\Z_2$, we have $\widetilde H^{nq}(B\pi_+\wedge X)=\Z_2[x]\otimes \widetilde H^*(X)$. Thus we can write $\text{SQ}(u)=\sum x^{q-i}\otimes\Squ^iu$, where $\Squ^iu\in \widetilde H^{q+i}(X)$.

We are working over a field, so that homology and cohomology are dual. Thus to understand a cohomology class, it is enough to understand how it pairs with homology classes. So, we should take $s\in H_{q+i}(X)$ and calculate $(\Squ^iu)(s)$. To do so, one takes $t_{q-i}\in H_{q-i}(\RP^\infty)$ dual to $x^{q-i}$,\footnote{$(x^{q-i}\otimes \Squ^i(u))(t_{q-i}\otimes s)=\Squ^i(u)(s)$.}
and calculates the pairing of $P_\pi(u)$ with $j^*(t_{q-i}\otimes s)$.

\subsection*{Thom class vs.\ Steenrod power}
If $\xi$ is an ($R$-oriented) $\R^n$-bundle, get an $n$-disk bundle $D(\xi)$ and an $(n-1)$-sphere sub-bundle $S(\xi)$.
\[\xymatrix@!0@R=12mm@C=2.5cm{
\iota\\
\tau(\xi)\ar@{|->}[u]
}
\raisebox{-.6cm}{\qquad in $H^n(\DASH;R)$ of\qquad}
\xymatrix@!0@R=12mm@C=2.5cm{
(D^n,S^{n-1})\ar[d]\\
(D(\xi),S(\xi))
}\]
Very similarly:
\[\xymatrix@!0@R=12mm@C=2.5cm{
u^{(n)}\\
P_\pi u\ar@{|->}[u]\ar@{|->}[d]\\
\text{SQ}u
}
\raisebox{-.6cm}{\qquad in $H^n(\DASH;R)$ of\qquad}
\xymatrix@!0@R=12mm@C=2.5cm{
(X^n,FW)\ar[d]^i\\
(E\pi\times_\pi X^n,E\pi\times_\pi FW)\\
(B\pi\times X,B\pi\times\{*\})\ar[u]_j
}\]

\pagebreak
\subsection*{Construction of Steenrod Power $P_\pi$}
\begin{lem*}\label{lemaboutpiadic}
Suppose that $\widetilde H^i (X) = 0$ for $i < q$ and that $\widetilde H^q (X)$ is finite dimensional.  Then,
\[\widetilde H^i(D_\pi X)=
\begin{cases}0;&\textup{if }i<nq;\\(\widetilde H^q(X)^{\otimes n})^\pi;&\textup{if }i=nq,%;\\???;&\textup{if }i>nq
\end{cases}
\]
and $i_X^*:\widetilde H^{nq}(D_\pi X)\to \widetilde H^{np}(X^{(n)})$ is the inclusion of the $\pi$-invariants $(\widetilde H^q(X)^{\otimes n})^\pi\subset\widetilde H^q(X)^{\otimes n}$.
\end{lem*}
\begin{proof}
%One could produce a CW-complex for $D_\pi X$ having no cells of dimension less than $nq$. No, one cannot! What if $X$ is a $K(X/3,d)$ and the field is $X/2$? Then $H^*(X;Z/2)=0$, but there's plenty of cells.
We have a map (drawn with dotted arrows) of bundle-subbundle pairs:
\[\xymatrix@!0@C=1.7cm@R=.8cm{
F_{n-1}X^n\ar[dd]\ar[rd]\ar@{.>}[rr]&&F_{n-1}X^n\ar'[d][dd]\ar[rd]\\
&X^n\ar[dd]\ar@{.>}[rr]&&\ar[dd]X^n\\
F_{n-1}X^n\ar[dd]\ar[rd]\ar@{.>}'[r][rr]&&E\pi\times_\pi F_{n-1}X^n\ar'[d][dd]\ar[rd]\\
&X^n\ar[dd]\ar@{.>}[rr]&&\ar[dd]E\pi\times_\pi X^n\\
\{b\}\ar@{.>}'[r][rr]\ar@{=}[rd]&&B\pi\ar@{=}[rd]\\
&\{b\}\ar@{.>}[rr]&&B\pi
}\]
\begin{itemize}\squishlist
\item $H^i(X^n,FW)=\widetilde H^i(X^{(n)})$ is zero for $i<nq$, $H^*(X)^{\otimes n}$ when $i=nq$, and the identification is equivariant w.r.t.\ permutations.
\item $E_2^{s,t} = H^s(B\pi; \{H^t(X^n, F_{n-1} X^n)\}) \Rightarrow \widetilde H^{s+t}(D_\pi X)$ is zero below row $nq$.
\item Have edge homomorphism: 
\[\xymatrix{
H^{nq}(E\pi\times_\pi X^n,E\pi\times_\pi FW)\ar@{->>}[r]^{\text{\qquad\qquad\qquad \ \ (iso)}} &E^{0,nq}_\infty\ar@{^{(}->}[r]^{\text{(iso)}}&E^{0,nq}_2\ar@{^{(}->}[r]^{\text{(inv)}}&E_1^{0,nq}\ar@{=}[r]&H^q(X^n,FW)
}\]
To get from $E_1^{0,nq}$ to $E_2^{0,nq}$ (after which the group is stable) one takes the $\pi$-invariants (recall that in Serre's model, one can take $E_1^{0,nq}$ to be $C^0(B)\otimes H^{nq}(X^n,F_{n-1}X^n)$, and only use cubes with corners at the basepoint).\qedhere
\end{itemize}
\end{proof}
\begin{itemise}
\item If $X=K_q=K(\F_p,q)$, this applies. Moreover, $K^q(K(\F_p,q);\F_p)$ is the field $\F_p\langle\imath_q\rangle$.
\item Thus $i^*:\widetilde H^{nq}(D_\pi K_q)\to \widetilde H^{np}(K_q^{(n)})$ is an isomorphism. There is then a unique $P_\pi\imath_q$ mapping to $\imath_q^{\wedge n}$ under $i^*$.
\item Now suppose that $u:X\to K_q$ represents $u\in \widetilde H^q(X)$. Need
\[\xymatrix{
\widetilde H^q(X)\ar[r]^{P_\pi}&\widetilde H^q(D_\pi X)\\
\widetilde H^q(K_q)\ar[r]^{P_\pi}\ar[u]^{u^*}&\widetilde H^q(D_\pi K_q)\ar[u]_{(D_\pi u)^*}
}\]
Thus must have $P_\pi u=(D_\pi u)^*P_\pi\imath_q$. This map is natural, is the only way to extend $P_\mu$ from what it needs to be on $K_q$, and satisfies the formula, since it is satisfied in the universal case.
\end{itemise}

\pagebreak

\subsection*{Properties}
\begin{itemize}
\item $\Squ^i u=0$ for $i<0$ (universality) and for $i>q$ (no negative powers of $x$).
\item $\Squ^q u=u^2$.
\[\xymatrix@R=.6cm{
B\pi_+ \wedge X \ar[r]^j & D_\pi X \\
S^0 \wedge X  \ar[r]^\Delta\ar[u]^{k\wedge 1} & X^{(2)}\ar[u]^i
}
\raisebox{-.5cm}{\textup{\qquad inducing:\qquad}}
\xymatrix@R=.6cm{
\sum_{i=0}^q x^{q-i} \otimes \Squ^i u\ar@{|->}[d]&
\ar@{|->}[l]  Pu \ar@{|->}[d]\\
u\smile u & u^{\wedge2}\ar@{|->}[l]
}\raisebox{-.5cm}{\textup{\qquad on $\widetilde H^{2q}(\DASH)$.\qquad}}
\]
\item Cartan formula. Have map $\delta:D_\pi(X\wedge Y)\to D_\pi(X)\wedge D_\pi(Y)$ which is $\Delta$ on $E_\pi$ parts.
\[\xymatrix@!0@C=4.4cm@R=1.8cm{
%\xymatrix{
(X\wedge Y)^{(2)}\ar[r]^i\ar[d]_T&D_\pi(X\wedge Y)\ar[d]^{\delta}_{(\Delta_{E\pi_+})}\\
X^{(2)}\wedge Y^{(2)}\ar[r]^{i\wedge i}&D_\pi X\wedge D_\pi Y
&B\pi_+\wedge X\wedge Y\ar[ul]_j\ar[d]^{\Delta_{B\pi_+}}\\
&&B\pi_+\wedge X\wedge B\pi_+\wedge Y\ar[ul]_{j\wedge j}
}\]
\[\qquad\qquad\xymatrix@!0@C=4.4cm@R=1.8cm{
(\imath_p\wedge \imath_q)^{(2)}&
\ar@{|->}[l]  P(\imath_p\wedge\imath_q)\ar@{|->}[dr]\\
\imath_p^{(2)}\wedge \imath_q^{(2)} \ar@{|->}[u]& P(\imath_p)\wedge P(\imath_q)\ar@{|->}[l]\ar@{|..>}[u]
&\sum x^{p+q-k}\otimes\Squ^k(u\wedge v)\\
&&(\sum x^{p-i}\otimes\Squ^i(u))\wedge(\sum x^{q-j}\otimes\Squ^j(v))\ar@{|..>}[u]\ar@{|->}[ul];[]
}\]
If $\delta^*$ does the right thing, then we're winners. It must, as in the universal case, $i^*$ is injective.
\item Write $\Squ\,u=\sum \Squ^iu$. The Cartan formula says $\Squ(uv)=(\Squ\,u)(\Squ\,v)$.
\item Exercise: show that $\Squ^0 s = s$, where $s$ is the generator of $\widetilde H^1 (S^1) = \Z_2$.
\item $\Squ^k$ is stable --- it commutes with suspension. (The suspension is ``smash with $s$''\footnote{$\widetilde H^q (X)=\widetilde H^q (X)\otimes\widetilde H^1(S^1)\xrightarrow{\ \wedge e\ }\widetilde H^{q+1} (\Sigma X)$}, and so $\Squ^k(\sigma u) = \Squ^k(u \wedge e) = \Squ^k u \wedge \Squ^0 e = \sigma \Squ^k u$.)
In particular, it is additive.
\item 
$\Squ^0: \widetilde H^q (X) \to \widetilde H^q (X)$ is the identity. (The map $\widetilde H^q(K^q)\to\widetilde H^q(S^q)$ induced by the fundamental class $s^{\wedge q}\in\widetilde H^q(S^q)$ is an isomorphism. Thus it's enough to check this on $s^{\wedge q}$, which follows from the exercise and Cartan.)
\item
$\Squ^1$ is the Bockstein.
\item The Adem relations are satisfied. For $a<2b$: 
\[
\Squ^a \Squ^b = \sum_{j=0}^{\lfloor a/2\rfloor} \binom{b-j-1}{a-2j} \Squ^{a+b-j} \Squ^j.
\]
Note that $a<2b\implies a+b-j\geq2j$.
\item $\Squ^{i_1}\cdots\Squ^{i_r}$ is admmissible if $i_1\geq 2i_2,\,\i_2\geq 2i_3$, etc. The Adem relations let us write any such product as a sum of admissible products.
\item Let $\Steen$ be the free unital $\Z_2$-algebra on symbols $\Squ^i$ for $i>0$, mod the Adem relations.
\begin{itemise}
\item $\Steen$ acts on $\widetilde H^*(X)$ --- the free algebra obviously does, and this action descends to $\Steen$.
\item $\Steen$ has a basis the admissible products.
\item $\Steen$ is generated by the $\Squ^{2^i}$, which are indecomposable. (Do it by induction: use admissibility and Adem.)
\end{itemise}
\item Application: Suppose that $\widetilde H^*(X;\Z_2)=\Z_2[x]$, with $|x|=q$. Then $q=2^i$. (For $x^2=\Squ^{q}x\neq0$, but the intermidiate cohomology groups between $q$ and $2q$ are zero.)
\item Application: Suppose that $f:S^{2n-1}\to S^n$ has odd Hopf invariant. Then $n=2^i$.
\end{itemize}
}
\KanSemResponse{3}
{``Cohomologie modulo 2 des complexes d'Eilenberg-MacLane'' --- J.-P.\ Serre}
{
\subsection*{Cohomology Operations}
\begin{itemise}
\item $H^p(K(\pi,q);G)$ is in bijection with natural transformations $\widetilde H^{q}(\DASH;\pi)\to\widetilde H^{p}(\DASH;G)$.
\item The groups $H^{n+q}(K(\pi,q);G)$ are stable when $q>n$. Elements thereof are called ``stable cohomology operations''. The Steenrod algebra is the algebra of stable cohomology operations $H^*(\DASH;\Z_2)$. To see this, we should calculate $H^{n+q}(K(\pi,q);G)$ for large $q$.
\item We'll use the path-loop fibation $K(\Z_2,q)\to *\to K(\Z_2,q+1)$. By induction:
\[\text{$H^*(K(\Z_2,q);\Z_2)=\Z_2[\Squ^I(\imath_q)]$, over those admissible $I$ with excess less than $q$.}\]
\item Need Borel's theorem:
\begin{itemize}\squishlist
\item A graded ring $R$ over $\Z_2$ has an order set $x_1,x_2,\ldots$ of homogeneous elements as a \emph{simple system of generators} if the monomials $\{x_{i_1}\cdots x_{i_r}:i_1<\cdots<i_r\}$ form a $\Z_2$-basis, and each graded part is finite dimensional.
\item
Suppose $F\rightarrow E\downarrow B$ is a fiber space with $E$ acyclic, and that $H^*(F;\Z_2)$ has a simple system $\{x_\alpha\}$ of transgressive generators. Then $H^*(B;\Z_2)$ is a polynomial ring in the $\{\tau(x_\alpha)\}$.
\end{itemize}
\item When $q=1$: have $\RP^\infty$, the poly alg on $\Squ^\emptyset\imath_1$.
\item Inductive step:
\begin{itemize}\squishlist
\item There's a simple system of generators $(\Squ^I\imath_q)^{2^r}$.
\item This equals $\Squ^{2^{r-1}(q+n(I)),\ldots,2(q+n(I)),(q+n(I))}\Squ^I\imath_q$.
\item As $I$ runs over all admissibles with $e(I)<q$, ${2^{r-1}(q+n(I)),\ldots,2(q+n(I)),(q+n(I))},I$ runs over all with $e(I)<q+1$.
\end{itemize}
\item What about:
\[\text{$H^*(K(\Z_{2^h},q);\Z_2)=\Z_2[\Squ^I_h(\imath_q)]$, over those admissible $I$ with excess less than $q$.}\]
Here, we swap $\Squ^1$ for $\Squ^1_h$, the $h^\text{th}$ Bockstein operator.

Here, the starting case is different: $H^*(K(\Z_{2^h},1);\Z_2)=\Lambda(u_1)\otimes\Z_2[\Squ_h^1(u_1)]$.
\item What about:
\[\text{$H^*(K(\Z,q);\Z_2)=\Z_2[\Squ^I(\imath_q)]$, over admissible $I$, $e(I)<q$, $I_\text{last}>1$.}\]
Here, the starting case is different: $H^*(K(\Z,1);\Z_2)=\Lambda(u_1)$.
\end{itemise}
\subsection*{Stable Operations}
Let's consider $H^{n+q}(K(\Z_2,q);G)$ for $q>n$. Basis corresponding to admissible sequences with degree $n$ (and excess less than $q$, but the excess condition is vacuous). Thus you get the whole degree $n$ part of the Steenrod algebra.
\subsection*{$H^p(K(\pi,q);\Z_2)$ for $\pi$ finitely generated abelian}
Using the K\"unneth formula to turn $\oplus$ to $\otimes$, only need to know it for $\Z$ and $\Z_{p^r}$. Know when $p=2$. If $p$ is odd, then ($q>1$) you can use mod $\calC$ theory (where $\calC$ is the class of ``abelian torsion groups of finite exponent $n$ coprime to $p$'').

Note that if one is interested in stable groups, the K\"unneth formula turns $\oplus$ to $\oplus$.
\subsection*{Poincare Polynomials}
Let $\theta(\pi;q)$ be the Poincare polynomial for $H^*(K(\pi,q),\Z_2)$.
\subsubsection*{The case $\pi=\Z_{2^h}$}
\begin{itemize}\squishlist
\item Have poly alg on an element of dimension $q+n(I)$ for each admissible $I$ with $e(I)<q$.
\begin{align*}
\theta(\Z_2;q)&=\prod_{e(I)<q}(1-t^{q+n(I)})^{-1}\\
&=\prod_{h_1\geq h_2\geq\cdots\geq h_{q}=0}(1-t^{\sum2^{h_j}})^{-1}
\end{align*}
\item To see the last equality, take a sequence $I=(i_1,\ldots,i_r)$.
\begin{itemize}\squishlist
\item Let $\alpha_1=i_1-2i_2$,\ldots, $\alpha_r=i_r$.
\item Let $\alpha_0=q-1-\sum\alpha_i$.
\item $n(I)=\sum (2^i-1)\alpha_i=\sum_{i\geq0}2^i\alpha_i-(q-1)$.
\item Thus generators of dimension $N$ are in bijection with $(\alpha_0,\ldots,\alpha_r)$ with sum $q-1$ and
\[N=1+2^1\alpha_1+\cdots+2^r\alpha_r.\]
There are $q$ summands, each a power of $2$. Let $h_1,\ldots, h_q$ be the powers, from biggest to smallest.
\end{itemize}
\end{itemize}
\subsubsection*{The case $\pi=\Z$}
\begin{itemize}\squishlist
\item Have poly alg on an element of dimension $q+n(I)$ where $e(I)<q$, $I_\text{last}\neq1$ (for $q>1$).
\begin{align*}
\theta(\Z;q)&=\mathop{\prod_{e(I)<q,\ I_\text{last}\neq1}}(1-t^{q+n(I)})^{-1}\\
&=\prod_{h_1> h_2\geq\cdots\geq h_{q-1}=0}(1-t^{\sum2^{h_j}})^{-1}
\end{align*}
\begin{itemize}\squishlist
\item Now need $\alpha_r>1$, so that $h_1=h_2$.
\item But then we can always swap $2^{h_1}+2^{h_2}$ for $2^{h_1+1}$.
\item Thus we get one less summand and a strict last inequality.
\end{itemize}
\item
\begin{itemize}\squishlist
\item $\frac{\theta(\Z_2;q-1)}{\theta(\Z;q)}$ has those indexed by $h_1$ to $h_{q-1}$ with $h_1=h_2$.
\item One can swap $2^{h_1}+2^{h_2}$ for $2^{h_1+1}$ again, and get the product for $\theta(\Z;q-1)$. Thus
\[\theta(\Z;q)=\frac{\theta(\Z_2;q-1)\cdot\theta(\Z_2;q-3)\cdots}{\theta(\Z_2;q-2)\cdot\theta(\Z_2;q-4)\cdots}\]
\end{itemize}
\end{itemize}
\pagebreak
\subsubsection*{Convergence}
\begin{itemize}\squishlist
\item These converge for $|t|<1$.
\item They have a ``dominant singularity'' at $t=1$.
\begin{itemize}\squishlist
\item Write $t=1-2^{-x}$, so $t\to1$ slowly as $x\to\infty$. Write:
\[\phi(\pi;q)(x)=\log_2(\theta(\pi;q)(t)).\]
\item Serre calculates:
\[\phi(\Z_2;q)\sim x^q/q!\text{\qquad and\qquad}\phi(\Z;q)\sim x^{q-1}/(q-1)!\qquad\qquad\]
\item As $\oplus\mapsto\otimes$ on $H^*$, $\oplus\mapsto\oplus$ on $\phi$. Thus:
\begin{thm*}
Suppose $\pi$ is f.g. with $s$ $\Z$ summands and $r$ $\Z_{2^h}$ summands. Then:
\[\phi(\pi;q)\sim \begin{cases}
r\cdot x^q/q! & r\geq1,\\
s\cdot x^{q-1}/(q-1)! & r=0,\,s\geq1,\\
0& r=s=0.
\end{cases}\]
\end{thm*}
\end{itemize}
\end{itemize}
\subsubsection*{Topological Application}
\begin{thm*}
Suppose $X$ is simply connected and:
\begin{enumerate}\squishlist
\item The $H_i(X;\Z)$ are finitely generated;
\item The $H_i(X;\Z_2)$ vanish for $i\gg0$;
\item $H_i(X;\Z_2)$ is nonzero for some $i\gg0$.
\end{enumerate}
Then $\pi_i(X)$ has a $\Z$ or $\Z_2$ subgroup for infinitely many $i$.
\end{thm*}
\begin{exmp*}
In $S^n$ there are finitely many $\Z$ summands, and thus infinitely many $\Z_2$ summands.
\end{exmp*}
\begin{proof}
\begin{itemize}\squishlist
\item By Hurewicz\footnote{Suppose that $X$ is simply connected, and that a Serre class $\scrC$ satisfies \textup{2A} and \textup{3}. Then if $\pi_i(X)\in\scrC$ for $i<n$ then $H_i(X)\in\scrC$ for $i<n$ and $h:\pi_n(X)\to H_n(X)$ is a $\scrC$-isomorphism.} mod $\calC$,\footnote{$\calC$ is the class of ``finitely generated abelian groups''} the homotopy groups are finitely generated. So it's enough to see that infinitely many $\pi_i\otimes\Z_2$ are nonzero.
\item For a contradiction, suppose that $q$ is maximal such that $\pi_q\otimes \Z_2\neq0$.
\item Let $j$ be the least positive integer such that $H_j(X;\Z_2)\neq0$.
\item By Hurewicz mod $\calC$,\footnote{$\calC$ is the class of ``abelian torsion groups of finite exponent $n$ coprime to $2$''} $j$ is the first index such that $\pi_j\otimes \Z_2\neq0$. Thus $q\geq j\geq2$.
\item Let $X_i=(X,i)$ be $X$ with the first $i-1$ homotopy groups killed, and let $A(t)$ be the Poincare polynomial for $H^*(A;\Z_2)$, for any space $A$ where these are finite dimensional.
\item There's a fibration\footnote{Build $X_{i+1}$ from $X_{i}$ by attaching cells to $X_{i}$ to make it a $K(\pi_i,i)$. Then have inclusion $X_{i}\to K(\pi_i,i)$ whose fibre we call $X_{i+1}$.} $X_{q+1}\to X_q\to K(\pi_q,q)$. $H^*(X_{q+1};\Z_2)$ is trivial, by Hurewicz mod $\calC$. Thus the SSS gives $X_{q}(t)=\theta(\pi_q;q)$.
\item %The mod 2 Serre spectral sequence shows that $H^*(X_q;\Z_2)=H^*(K(\Pi,q);\Z_2)$.
Now there are fibrations:
%\[X_{q+1}\to X_{q}\to K(\pi_{q};q),\]
\[K(\pi_{q-1},q-2)\to X_{q}\to X_{q-1},\]
\[K(\pi_{q-2},q-3)\to X_{q-1}\to X_{q-2},\]
\[K(\pi_{q-3},q-4)\to X_{q-2}\to X_{q-3},\cdots\]
\[K(\pi_{2},1)\to X_{3}\to X_2.\]
\item Whenever $F\to E\to B$ is a fibration with $B$ simply connected, we have, on \emph{every coefficient}, $E(t)\leq F(t)\cdot B(t)$ (if all the terms make sense). The $E_2$ page has Poincare poly $F(t)\cdot B(t)$, but the differentials cut down dimensions.
\item Thus, on coefficients:
\begin{alignat*}{2}
\theta(\pi_q;q)&= X_q(t)\\
&\leq X_{q-1}(t)\cdot\theta(\pi_{q-1};q-2)\leq\cdots\\
&\leq X(t)\cdot \prod_{1<i<q}\theta(\pi_i;i-1).
\end{alignat*}
\item $X(t)$ is a polynomial, so its values on $[0,1]$ are bounded by $h$. Taking $\log_2$:
\[\phi(\pi_q;q)(x)\leq\log_2h+\sum_{i=2}^{q-1}\phi(\pi_i;i-1)(x).\]
This is a contradiction --- the terms on the right grow at most as fast as $x^{q-2}$ as $x\to\infty$, but those on the right grow as either $x^q$ or $x^{q-1}$.\qedhere
\end{itemize}
\end{proof}
}
\KanSemResponse{4}
{``Characteristic Classes'' --- Milnor and Stasheff}
{
I took this reading assingment as a chance to read in more detail much of what I had read earlier, but not understood as well. At the time of writing I've read the first 11 chapters of the text.

It was nice to see the Stiefel-Whitney classes applied early in the exposition - a bit of motivation never goes astray. In particular, the Cartan formula can be used to compare the classes of the tangent and normal bundles (given an immersion into Euclidean space), and one can obtain lower bounds in this way for the dimension of the ambient space. One is finding invariants of the twistedness of the normal bundle, and something very twisted needs room to be so. In particular for $\RP^n$, when $n$ is a power of two, this gives a lower bound which is as strong as possible.

Of the things that I have enjoyed particularly, one thing that stands out was the way that the cohomology of the Grassmannian $BO(n)$ was calculated, and the way that this calculation hints at the structures underlying the Stiefel-Whitney classes.

The approach is to assume the existence of (possibly non-unique) Stiefel-Whitney classes, and then to prove that the classes of the universal bundle are algebraically independent, by calculating on a well understood example. The Stiefel-Whitney classes of the universal $n$-bundle then generate a polynomial algebra on one generator in each degree from 1 to $n$, putting a lower bound on the dimension of $BO(n)$. Next, a cell structure is put on $BO(n)$, minimal in the very strong sense that there are no differentials in the cellular cochain complex with $\Z_2$ coefficients. This can be shown simply by noting that the dimension in degree $i$ of the polynomial subalgebra equals the number of cells of dimension $i$.

Anyway, this proof really did four things. It gave an explicit cell structure on the Grassmannian (which I didn't know), it gave a painless calculation of the mod 2 cohomology thereof (although it is only painless now that I am more comfortable with the proof of the properties of the squares), and in the meantime, it even said something about the attaching maps --- they all have even degree (after collapsing onto lower cells). Finally, it showed the uniqueness of the Stiefel-Whitney classes.

\subsubsection*{The Thom isomorphism}
This reading gave me a chance to think a little more about the Thom isomorphism, which comes up in chapter 9. Perhaps while the following text lacks clarity, and possibly correctness, it is an honest representation of the way I currently view this material.

My current preferred way of thinking about it works when the base $M^m$ of an (oriented) $R^n$-bundle $E$ is a closed (oriented) $m$-manifold. Let $D$ and $S$ be the disk and sphere bundles of $E$. Then we have Poincare duality on the pair $(D,S)$ (an (oriented) $m+n$-manifold with boundary), and the Thom class $u\in H^n(D,S)$ is dual to the homology class $[B]\in H_m(D,S)$.

Given a relative $k$-chain $s=\sum_{i}n_i\sigma_i$ in $(D,S)$, it can be deformed so that each simplex $\sigma_i:\Delta^k\to D$ is smooth near $B$ and transverse to $B$, in which case, $\sigma_i^{-1}(B)$ is an $n-k$-dimensional submanifold of $\Delta^k$. This submanifold can be triangulated, yielding an $(n-k)$-chain in $B$. Adding these up over the simplices $\sigma_i$, we obtain an $n-k$-chain $\overline s$ in $B$. 

Now a cohomology class in $H^k(D,S)$, being equivalent to a class in $H^k(E,E_0)$, can be taken to be supported along $B$. The value on $s$ of a cohomology class supported along $B$ should be determined by $\overline s$ and a sign related to the orientation with which $s$ intersected $B$. Thus, such a cohomology class should be the cup product of a class in $H^{n-k}(B)$ and the Thom class (which detects with orientation the number of intersections of an $n$-chain with $B$).

I suppose the purpose of writing these paragraphs was first to subject them to your scrutiny and second to claim that I can now better understand the Euler class. The Euler class is the restriction $u|_D$ of the Thom class to $D$, viewed as an element of $H^n(B)$ (as $B$ and $D$ are homotopic). As such it is the image under the Thom isomorphism of $u\cup u\in H^{2n}(D,S)$. $u\cup u$ is dual under Poincare duality to the self-intersection of the submanifold $B$.

In the setting where $n=m$ and ($B$ is connected), this shows that $u\cup u$ is the self-intersection number of $B$ multiplied by the generator of $H^{2n}(D,S)$, explaining the relationship between the Euler class of the tangent bundle and the Euler characteristic of the manifold $B$.

A couple more cool things that I feel like mentioning:
\begin{itemize}
\item It is shown that the dual class (in $H^k(A)$) to an embedding $M\to A$ (of codimension $k$) of a closed smooth submanifold restricts to the Euler (or top SW) class of $M$. This is used to show that $\RP^{2^i}$ will not smoothly embed as a closed subset of $\R^{2^{i+1}-1}$.
\item The derivation of Poincare duality (with field coefficients) using the diagonal class of the embedding $M\to M\times M$. I'd never thought about the diagonal class until now, and now I even have a formula for it!
\end{itemize}



%In this case, the Euler class, which is 

%Any relative $k$-chain in $(D,S)$ should be homotopic to one with the following property: whenever it exits $S$, 

%somehow it is transverse to $B$ in $n$-dimensions, and in the remaining $k-n$ dimensions, it runs parallel to $B$. Obviously this makes no sense. But anyway, 
%It seems to me then that the point of the Thom isomorphism is that any
}
\KanSemResponse{5}
{``The Geometric Realization of a Semi-simplicial Complex'' --- John Milnor}
{
What I have gained most from this paper, and from Guozhen's talk, is the realisation that simplicial sets and their realisations are not too confusing. Until now, I had a latent belief that the geometric realisation was a strange, `thick' object, somehow with more `mass' than I desired. Instead, it turns out that the relations imply that every point in the realisation belongs to the interior of a unique nondegenerate simplex --- it's all very well organised in there.

In fact, the construction that Milnor uses of the realisation is really very simple --- one just takes a disjoint union of the realisations of all of the simplices, and glues `faces' onto faces, and `degenerate simplices' degenerately onto simplices. The cells of the resulting CW-complex are exactly the simplices who were lucky enough not to get glued degenerately onto a lower dimensional one. Moreover, it is not strange to include degenerate simplices in our definition of simplicial set --- their presence mimics the situation in the singular simplicial set of a topological space (and I suspect there are even better rationales around for this) --- when calculating singular homology groups, we certainly don't discard degenerate simplices before forming the singular chain complex, even though they don't appear to tell us anything about the geometry of the space that the nondegenerate simplices did not. Again, perhaps this discussion seemed a little silly, but it was honest!

I suppose now that one has realised that the geometric realisation isn't such a nasty process, one might expect that it should be compatible with reasonable operations on simplicial sets, such as products. There's a little point here to be made. The natural map $\eta:|K\times K'|\to|K|\times|K'|$ is a bijection, as Milnor verifies by constructing a set-theoretic inverse $\overline\eta$. He constructs $\overline\eta$ through a careful examination of the cells of $|K\times K'|$. Moreover, the map $\overline\eta$ is continuous on each product cell of $|K|\times|K'|$. Thus as long as $|K|\times|K'|$ actually inherits the structure of a CW-complex from its product cells, $\overline\eta$ is a homeomorphism. Fortunately, this happens under various conditions on the `size' of $K$ and $K'$.

I like this point, because it is an example of why CW-complexes are so good to work with. This gave me pause to think about why `compactly generated' is a good adjective to place in front of the noun `space'. The continuity of a map out of such a space is determined `locally', which in this case means `over compact subsets'. 

We would like to see that the `total singular complex' functor $S$ is right adjoint to the realisation functor:
\[|\DASH|:\sSet\longleftrightarrow\Top:S\]
My tendency is to try to understand such adjunctions as this by analogy --- there may be better ways --- and I suppose one analogy would be with the adjunction between the free and forgetful functors in various settings. The realisation should be the way to build a topological space which is as `free' as possible using the simplices to hand (while satisfying the relations between then), while applying $S$ should be the clumsiest way to return a simplicial complex given a space. Now write $\phi$ and $\theta$ for the maps
\[\phi:\Top(|T|,X)\cong\sSet(T,SX): \theta.\]
The appropriate definition of $\phi$ is obvious, but $\theta$ is a little bit trickier, and to understand it, one needs to understand the counit map $|SX|\to X$. The formula for this counit it clear enough, and chasing through this point of view has made me understand the whole picture far better.

Guozhen, in his talk, gave a perspective on this adjunction which doesn't feature in the paper. It runs as follows: to check that the above adjunction holds, it is enough to observe that $|\DASH|$ preserves colimits (apparently this holds `almost by definition', which I'm beginning to believe as I write this response). Then one checks the adjunction holds when $T=\Delta^n$, and then notes that for any $T\in\sSet$ we may write
\[K=\colim_{\Delta/K}\Delta^n,\]
where $\Delta/K$ is the category of simplices $\Delta^n$ over $K$. I still need to really sit down with this and work out the details.

What's cool about this is the idea that $\sSet$ is built from simplices, just as the category of CW-complexes, and so we can understand colimit-preserving functors from $\sSet$ by examining them on simplices. In fact, that the functor $|\DASH|$ is left adjoint to $S$ turned out to be equivalent to the fact that it preserved colimits.

In fact, after writing thus far, one notes that if $F:\calC\to\calD$ is a functor which preserves colimits, and $G:\calD\to\calC$ is any functor, then these are adjoint \Iff there is a collection $\{c_i\}$ of objects of $\calC$, generating $\calC$ under colimits, such that $\calD(Fc_i,d)\cong\calC(c_i,Gd)$. The functor $S$ is of course perfectly designed to play the role of $G$.

At this late stage, one should say a few words about the Dold-Kan correspondence, and about Eilenberg-MacLane complexes. Let it just be said that after this reading and talk, this material seems much less distant.
}
\end{document}



















% !TEX root = z_output/_EquivStableHtpy.tex
%%%%%%%%%%%%%%%%%%%%%%%%%%%%%%%%%%%%%%%%%%%%%%%%%%%%%%%%%%%%%%%%%%%%%%%%%%%%%%%%
%%%%%%%%%%%%%%%%%%%%%%%%%%% 80 characters %%%%%%%%%%%%%%%%%%%%%%%%%%%%%%%%%%%%%%
%%%%%%%%%%%%%%%%%%%%%%%%%%%%%%%%%%%%%%%%%%%%%%%%%%%%%%%%%%%%%%%%%%%%%%%%%%%%%%%%
\documentclass[11pt]{article}
\usepackage{fullpage}
\usepackage{amsmath,amsthm,amssymb}
\usepackage{mathrsfs,nicefrac}
\usepackage{amssymb}
\usepackage{epsfig}
\usepackage[all,2cell]{xy}
\usepackage{sseq}
\usepackage{tocloft}
\usepackage{cancel}
\usepackage[strict]{changepage}
\usepackage{color}
\usepackage{tikz}
\usepackage{extpfeil}
\usepackage{version}
\usepackage{framed}
\definecolor{shadecolor}{rgb}{.925,0.925,0.925}

%\usepackage{ifthen}
%Used for disabling hyperref
\ifx\dontloadhyperref\undefined
%\usepackage[pdftex,pdfborder={0 0 0 [1 1]}]{hyperref}
\usepackage[pdftex,pdfborder={0 0 .5 [1 1]}]{hyperref}
\else
\providecommand{\texorpdfstring}[2]{#1}
\fi
%>>>>>>>>>>>>>>>>>>>>>>>>>>>>>>
%<<<        Versions        <<<
%>>>>>>>>>>>>>>>>>>>>>>>>>>>>>>
%Add in the following line to include all the versions.
%\def\excludeversion#1{\includeversion{#1}}

%>>>>>>>>>>>>>>>>>>>>>>>>>>>>>>
%<<<       Better ToC       <<<
%>>>>>>>>>>>>>>>>>>>>>>>>>>>>>>
\setlength{\cftbeforesecskip}{0.5ex}

%>>>>>>>>>>>>>>>>>>>>>>>>>>>>>>
%<<<      Hyperref mod      <<<
%>>>>>>>>>>>>>>>>>>>>>>>>>>>>>>

%needs more testing
\newcounter{dummyforrefstepcounter}
\newcommand{\labelRIGHTHERE}[1]
{\refstepcounter{dummyforrefstepcounter}\label{#1}}


%>>>>>>>>>>>>>>>>>>>>>>>>>>>>>>
%<<<  Theorem Environments  <<<
%>>>>>>>>>>>>>>>>>>>>>>>>>>>>>>
\ifx\dontloaddefinitionsoftheoremenvironments\undefined
\theoremstyle{plain}
\newtheorem{thm}{Theorem}[section]
\newtheorem*{thm*}{Theorem}
\newtheorem{lem}[thm]{Lemma}
\newtheorem*{lem*}{Lemma}
\newtheorem{prop}[thm]{Proposition}
\newtheorem*{prop*}{Proposition}
\newtheorem{cor}[thm]{Corollary}
\newtheorem*{cor*}{Corollary}
\newtheorem{defprop}[thm]{Definition-Proposition}
\newtheorem*{punchline}{Punchline}
\newtheorem*{conjecture}{Conjecture}
\newtheorem*{claim}{Claim}

\theoremstyle{definition}
\newtheorem{defn}{Definition}[section]
\newtheorem*{defn*}{Definition}
\newtheorem{exmp}{Example}[section]
\newtheorem*{exmp*}{Example}
\newtheorem*{exmps*}{Examples}
\newtheorem*{nonexmp*}{Non-example}
\newtheorem{asspt}{Assumption}[section]
\newtheorem{notation}{Notation}[section]
\newtheorem{exercise}{Exercise}[section]
\newtheorem*{fact*}{Fact}
\newtheorem*{rmk*}{Remark}
\newtheorem{fact}{Fact}
\newtheorem*{aside}{Aside}
\newtheorem*{question}{Question}
\newtheorem*{answer}{Answer}

\else\relax\fi

%>>>>>>>>>>>>>>>>>>>>>>>>>>>>>>
%<<<      Fields, etc.      <<<
%>>>>>>>>>>>>>>>>>>>>>>>>>>>>>>
\DeclareSymbolFont{AMSb}{U}{msb}{m}{n}
\DeclareMathSymbol{\N}{\mathbin}{AMSb}{"4E}
\DeclareMathSymbol{\Octonions}{\mathbin}{AMSb}{"4F}
\DeclareMathSymbol{\Z}{\mathbin}{AMSb}{"5A}
\DeclareMathSymbol{\R}{\mathbin}{AMSb}{"52}
\DeclareMathSymbol{\Q}{\mathbin}{AMSb}{"51}
\DeclareMathSymbol{\PP}{\mathbin}{AMSb}{"50}
\DeclareMathSymbol{\I}{\mathbin}{AMSb}{"49}
\DeclareMathSymbol{\C}{\mathbin}{AMSb}{"43}
\DeclareMathSymbol{\A}{\mathbin}{AMSb}{"41}
\DeclareMathSymbol{\F}{\mathbin}{AMSb}{"46}
\DeclareMathSymbol{\G}{\mathbin}{AMSb}{"47}
\DeclareMathSymbol{\Quaternions}{\mathbin}{AMSb}{"48}


%>>>>>>>>>>>>>>>>>>>>>>>>>>>>>>
%<<<       Operators        <<<
%>>>>>>>>>>>>>>>>>>>>>>>>>>>>>>
\DeclareMathOperator{\ad}{\textbf{ad}}
\DeclareMathOperator{\coker}{coker}
\renewcommand{\ker}{\textup{ker}\,}
\DeclareMathOperator{\End}{End}
\DeclareMathOperator{\Aut}{Aut}
\DeclareMathOperator{\Hom}{Hom}
\DeclareMathOperator{\Maps}{Maps}
\DeclareMathOperator{\Mor}{Mor}
\DeclareMathOperator{\Gal}{Gal}
\DeclareMathOperator{\Ext}{Ext}
\DeclareMathOperator{\Tor}{Tor}
\DeclareMathOperator{\Map}{Map}
\DeclareMathOperator{\Der}{Der}
\DeclareMathOperator{\Rad}{Rad}
\DeclareMathOperator{\rank}{rank}
\DeclareMathOperator{\ArfInvariant}{Arf}
\DeclareMathOperator{\KervaireInvariant}{Ker}
\DeclareMathOperator{\im}{im}
\DeclareMathOperator{\coim}{coim}
\DeclareMathOperator{\trace}{tr}
\DeclareMathOperator{\supp}{supp}
\DeclareMathOperator{\ann}{ann}
\DeclareMathOperator{\spec}{Spec}
\DeclareMathOperator{\SPEC}{\textbf{Spec}}
\DeclareMathOperator{\proj}{Proj}
\DeclareMathOperator{\PROJ}{\textbf{Proj}}
\DeclareMathOperator{\fiber}{F}
\DeclareMathOperator{\cofiber}{C}
\DeclareMathOperator{\cone}{cone}
\DeclareMathOperator{\skel}{sk}
\DeclareMathOperator{\coskel}{cosk}
\DeclareMathOperator{\conn}{conn}
\DeclareMathOperator{\colim}{colim}
\DeclareMathOperator{\limit}{lim}
\DeclareMathOperator{\ch}{ch}
\DeclareMathOperator{\Vect}{Vect}
\DeclareMathOperator{\GrthGrp}{GrthGp}
\DeclareMathOperator{\Sym}{Sym}
\DeclareMathOperator{\Prob}{\mathbb{P}}
\DeclareMathOperator{\Exp}{\mathbb{E}}
\DeclareMathOperator{\GeomMean}{\mathbb{G}}
\DeclareMathOperator{\Var}{Var}
\DeclareMathOperator{\Cov}{Cov}
\DeclareMathOperator{\Sp}{Sp}
\DeclareMathOperator{\Seq}{Seq}
\DeclareMathOperator{\Cyl}{Cyl}
\DeclareMathOperator{\Ev}{Ev}
\DeclareMathOperator{\sh}{sh}
\DeclareMathOperator{\intHom}{\underline{Hom}}
\DeclareMathOperator{\Frac}{frac}



%>>>>>>>>>>>>>>>>>>>>>>>>>>>>>>
%<<<   Cohomology Theories  <<<
%>>>>>>>>>>>>>>>>>>>>>>>>>>>>>>
\DeclareMathOperator{\KR}{{K\R}}
\DeclareMathOperator{\KO}{{KO}}
\DeclareMathOperator{\K}{{K}}
\DeclareMathOperator{\OmegaO}{{\Omega_{\Octonions}}}

%>>>>>>>>>>>>>>>>>>>>>>>>>>>>>>
%<<<   Algebraic Geometry   <<<
%>>>>>>>>>>>>>>>>>>>>>>>>>>>>>>
\DeclareMathOperator{\Spec}{Spec}
\DeclareMathOperator{\Proj}{Proj}
\DeclareMathOperator{\Sing}{Sing}
\DeclareMathOperator{\shfHom}{\mathscr{H}\textit{\!\!om}}
\DeclareMathOperator{\WeilDivisors}{{Div}}
\DeclareMathOperator{\CartierDivisors}{{CaDiv}}
\DeclareMathOperator{\PrincipalWeilDivisors}{{PrDiv}}
\DeclareMathOperator{\LocallyPrincipalWeilDivisors}{{LPDiv}}
\DeclareMathOperator{\PrincipalCartierDivisors}{{PrCaDiv}}
\DeclareMathOperator{\DivisorClass}{{Cl}}
\DeclareMathOperator{\CartierClass}{{CaCl}}
\DeclareMathOperator{\Picard}{{Pic}}
\DeclareMathOperator{\Frob}{Frob}


%>>>>>>>>>>>>>>>>>>>>>>>>>>>>>>
%<<<  Mathematical Objects  <<<
%>>>>>>>>>>>>>>>>>>>>>>>>>>>>>>
\newcommand{\sll}{\mathfrak{sl}}
\newcommand{\gl}{\mathfrak{gl}}
\newcommand{\GL}{\mbox{GL}}
\newcommand{\PGL}{\mbox{PGL}}
\newcommand{\SL}{\mbox{SL}}
\newcommand{\Mat}{\mbox{Mat}}
\newcommand{\Gr}{\textup{Gr}}
\newcommand{\Squ}{\textup{Sq}}
\newcommand{\catSet}{\textit{Sets}}
\newcommand{\RP}{{\R\PP}}
\newcommand{\CP}{{\C\PP}}
\newcommand{\Steen}{\mathscr{A}}
\newcommand{\Orth}{\textup{\textbf{O}}}

%>>>>>>>>>>>>>>>>>>>>>>>>>>>>>>
%<<<  Mathematical Symbols  <<<
%>>>>>>>>>>>>>>>>>>>>>>>>>>>>>>
\newcommand{\DASH}{\textup{---}}
\newcommand{\op}{\textup{op}}
\newcommand{\CW}{\textup{CW}}
\newcommand{\ob}{\textup{ob}\,}
\newcommand{\ho}{\textup{ho}}
\newcommand{\st}{\textup{st}}
\newcommand{\id}{\textup{id}}
\newcommand{\Bullet}{\ensuremath{\bullet} }
\newcommand{\sprod}{\wedge}

%>>>>>>>>>>>>>>>>>>>>>>>>>>>>>>
%<<<      Some Arrows       <<<
%>>>>>>>>>>>>>>>>>>>>>>>>>>>>>>
\newcommand{\nt}{\Longrightarrow}
\let\shortmapsto\mapsto
\let\mapsto\longmapsto
\newcommand{\mapsfrom}{\,\reflectbox{$\mapsto$}\ }
\newcommand{\bigrightsquig}{\scalebox{2}{\ensuremath{\rightsquigarrow}}}
\newcommand{\bigleftsquig}{\reflectbox{\scalebox{2}{\ensuremath{\rightsquigarrow}}}}

%\newcommand{\cofibration}{\xhookrightarrow{\phantom{\ \,{\sim\!}\ \ }}}
%\newcommand{\fibration}{\xtwoheadrightarrow{\phantom{\sim\!}}}
%\newcommand{\acycliccofibration}{\xhookrightarrow{\ \,{\sim\!}\ \ }}
%\newcommand{\acyclicfibration}{\xtwoheadrightarrow{\sim\!}}
%\newcommand{\leftcofibration}{\xhookleftarrow{\phantom{\ \,{\sim\!}\ \ }}}
%\newcommand{\leftfibration}{\xtwoheadleftarrow{\phantom{\sim\!}}}
%\newcommand{\leftacycliccofibration}{\xhookleftarrow{\ \ {\sim\!}\,\ }}
%\newcommand{\leftacyclicfibration}{\xtwoheadleftarrow{\sim\!}}
%\newcommand{\weakequiv}{\xrightarrow{\ \,\sim\,\ }}
%\newcommand{\leftweakequiv}{\xleftarrow{\ \,\sim\,\ }}

\newcommand{\cofibration}
{\xhookrightarrow{\phantom{\ \,{\raisebox{-.3ex}[0ex][0ex]{\scriptsize$\sim$}\!}\ \ }}}
\newcommand{\fibration}
{\xtwoheadrightarrow{\phantom{\raisebox{-.3ex}[0ex][0ex]{\scriptsize$\sim$}\!}}}
\newcommand{\acycliccofibration}
{\xhookrightarrow{\ \,{\raisebox{-.55ex}[0ex][0ex]{\scriptsize$\sim$}\!}\ \ }}
\newcommand{\acyclicfibration}
{\xtwoheadrightarrow{\raisebox{-.6ex}[0ex][0ex]{\scriptsize$\sim$}\!}}
\newcommand{\leftcofibration}
{\xhookleftarrow{\phantom{\ \,{\raisebox{-.3ex}[0ex][0ex]{\scriptsize$\sim$}\!}\ \ }}}
\newcommand{\leftfibration}
{\xtwoheadleftarrow{\phantom{\raisebox{-.3ex}[0ex][0ex]{\scriptsize$\sim$}\!}}}
\newcommand{\leftacycliccofibration}
{\xhookleftarrow{\ \ {\raisebox{-.55ex}[0ex][0ex]{\scriptsize$\sim$}\!}\,\ }}
\newcommand{\leftacyclicfibration}
{\xtwoheadleftarrow{\raisebox{-.6ex}[0ex][0ex]{\scriptsize$\sim$}\!}}
\newcommand{\weakequiv}
{\xrightarrow{\ \,\raisebox{-.3ex}[0ex][0ex]{\scriptsize$\sim$}\,\ }}
\newcommand{\leftweakequiv}
{\xleftarrow{\ \,\raisebox{-.3ex}[0ex][0ex]{\scriptsize$\sim$}\,\ }}

%>>>>>>>>>>>>>>>>>>>>>>>>>>>>>>
%<<<    xymatrix Arrows     <<<
%>>>>>>>>>>>>>>>>>>>>>>>>>>>>>>
\newdir{ >}{{}*!/-5pt/@{>}}
\newcommand{\xycof}{\ar@{ >->}}
\newcommand{\xycofib}{\ar@{^{(}->}}
\newcommand{\xycofibdown}{\ar@{_{(}->}}
\newcommand{\xyfib}{\ar@{->>}}
\newcommand{\xymapsto}{\ar@{|->}}

%>>>>>>>>>>>>>>>>>>>>>>>>>>>>>>
%<<<     Greek Letters      <<<
%>>>>>>>>>>>>>>>>>>>>>>>>>>>>>>
%\newcommand{\oldphi}{\phi}
%\renewcommand{\phi}{\varphi}
\let\oldphi\phi
\let\phi\varphi
\renewcommand{\to}{\longrightarrow}
\newcommand{\from}{\longleftarrow}
\newcommand{\eps}{\varepsilon}

%>>>>>>>>>>>>>>>>>>>>>>>>>>>>>>
%<<<  1st-4th & parentheses <<<
%>>>>>>>>>>>>>>>>>>>>>>>>>>>>>>
\newcommand{\first}{^\text{st}}
\newcommand{\second}{^\text{nd}}
\newcommand{\third}{^\text{rd}}
\newcommand{\fourth}{^\text{th}}
\newcommand{\ZEROTH}{$0^\text{th}$ }
\newcommand{\FIRST}{$1^\text{st}$ }
\newcommand{\SECOND}{$2^\text{nd}$ }
\newcommand{\THIRD}{$3^\text{rd}$ }
\newcommand{\FOURTH}{$4^\text{th}$ }
\newcommand{\iTH}{$i^\text{th}$ }
\newcommand{\jTH}{$j^\text{th}$ }
\newcommand{\nTH}{$n^\text{th}$ }

%>>>>>>>>>>>>>>>>>>>>>>>>>>>>>>
%<<<    upright commands    <<<
%>>>>>>>>>>>>>>>>>>>>>>>>>>>>>>
\newcommand{\upcol}{\textup{:}}
\newcommand{\upsemi}{\textup{;}}
\providecommand{\lparen}{\textup{(}}
\providecommand{\rparen}{\textup{)}}
\renewcommand{\lparen}{\textup{(}}
\renewcommand{\rparen}{\textup{)}}
\newcommand{\Iff}{\emph{iff} }

%>>>>>>>>>>>>>>>>>>>>>>>>>>>>>>
%<<<     Environments       <<<
%>>>>>>>>>>>>>>>>>>>>>>>>>>>>>>
\newcommand{\squishlist}
{ %\setlength{\topsep}{100pt} doesn't seem to do anything.
  \setlength{\itemsep}{.5pt}
  \setlength{\parskip}{0pt}
  \setlength{\parsep}{0pt}}
\newenvironment{itemise}{
\begin{list}{\textup{$\rightsquigarrow$}}
   {  \setlength{\topsep}{1mm}
      \setlength{\itemsep}{1pt}
      \setlength{\parskip}{0pt}
      \setlength{\parsep}{0pt}
   }
}{\end{list}\vspace{-.1cm}}
\newcommand{\INDENT}{\textbf{}\phantom{space}}
\renewcommand{\INDENT}{\rule{.7cm}{0cm}}

\newcommand{\itm}[1][$\rightsquigarrow$]{\item[{\makebox[.5cm][c]{\textup{#1}}}]}


%\newcommand{\rednote}[1]{{\color{red}#1}\makebox[0cm][l]{\scalebox{.1}{rednote}}}
%\newcommand{\bluenote}[1]{{\color{blue}#1}\makebox[0cm][l]{\scalebox{.1}{rednote}}}

\newcommand{\rednote}[1]
{{\color{red}#1}\makebox[0cm][l]{\scalebox{.1}{\rotatebox{90}{?????}}}}
\newcommand{\bluenote}[1]
{{\color{blue}#1}\makebox[0cm][l]{\scalebox{.1}{\rotatebox{90}{?????}}}}


\newcommand{\funcdef}[4]{\begin{align*}
#1&\to #2\\
#3&\mapsto#4
\end{align*}}

%\newcommand{\comment}[1]{}

%>>>>>>>>>>>>>>>>>>>>>>>>>>>>>>
%<<<       Categories       <<<
%>>>>>>>>>>>>>>>>>>>>>>>>>>>>>>
\newcommand{\Ens}{{\mathscr{E}ns}}
\DeclareMathOperator{\Sheaves}{{\mathsf{Shf}}}
\DeclareMathOperator{\Presheaves}{{\mathsf{PreShf}}}
\DeclareMathOperator{\Psh}{{\mathsf{Psh}}}
\DeclareMathOperator{\Shf}{{\mathsf{Shf}}}
\DeclareMathOperator{\Varieties}{{\mathsf{Var}}}
\DeclareMathOperator{\Schemes}{{\mathsf{Sch}}}
\DeclareMathOperator{\Rings}{{\mathsf{Rings}}}
\DeclareMathOperator{\AbGp}{{\mathsf{AbGp}}}
\DeclareMathOperator{\Modules}{{\mathsf{\!-Mod}}}
\DeclareMathOperator{\fgModules}{{\mathsf{\!-Mod}^{\textup{fg}}}}
\DeclareMathOperator{\QuasiCoherent}{{\mathsf{QCoh}}}
\DeclareMathOperator{\Coherent}{{\mathsf{Coh}}}
\DeclareMathOperator{\GSW}{{\mathcal{SW}^G}}
\DeclareMathOperator{\Burnside}{{\mathsf{Burn}}}
\DeclareMathOperator{\GSet}{{G\mathsf{Set}}}
\DeclareMathOperator{\FinGSet}{{G\mathsf{Set}^\textup{fin}}}
\DeclareMathOperator{\HSet}{{H\mathsf{Set}}}
\DeclareMathOperator{\Cat}{{\mathsf{Cat}}}
\DeclareMathOperator{\Fun}{{\mathsf{Fun}}}
\DeclareMathOperator{\Orb}{{\mathsf{Orb}}}
\DeclareMathOperator{\Set}{{\mathsf{Set}}}
\DeclareMathOperator{\sSet}{{\mathsf{sSet}}}
\DeclareMathOperator{\Top}{{\mathsf{Top}}}
\DeclareMathOperator{\GSpectra}{{G-\mathsf{Spectra}}}
\DeclareMathOperator{\Lan}{Lan}
\DeclareMathOperator{\Ran}{Ran}

%>>>>>>>>>>>>>>>>>>>>>>>>>>>>>>
%<<<     Script Letters     <<<
%>>>>>>>>>>>>>>>>>>>>>>>>>>>>>>
\newcommand{\scrQ}{\mathscr{Q}}
\newcommand{\scrW}{\mathscr{W}}
\newcommand{\scrE}{\mathscr{E}}
\newcommand{\scrR}{\mathscr{R}}
\newcommand{\scrT}{\mathscr{T}}
\newcommand{\scrY}{\mathscr{Y}}
\newcommand{\scrU}{\mathscr{U}}
\newcommand{\scrI}{\mathscr{I}}
\newcommand{\scrO}{\mathscr{O}}
\newcommand{\scrP}{\mathscr{P}}
\newcommand{\scrA}{\mathscr{A}}
\newcommand{\scrS}{\mathscr{S}}
\newcommand{\scrD}{\mathscr{D}}
\newcommand{\scrF}{\mathscr{F}}
\newcommand{\scrG}{\mathscr{G}}
\newcommand{\scrH}{\mathscr{H}}
\newcommand{\scrJ}{\mathscr{J}}
\newcommand{\scrK}{\mathscr{K}}
\newcommand{\scrL}{\mathscr{L}}
\newcommand{\scrZ}{\mathscr{Z}}
\newcommand{\scrX}{\mathscr{X}}
\newcommand{\scrC}{\mathscr{C}}
\newcommand{\scrV}{\mathscr{V}}
\newcommand{\scrB}{\mathscr{B}}
\newcommand{\scrN}{\mathscr{N}}
\newcommand{\scrM}{\mathscr{M}}

%>>>>>>>>>>>>>>>>>>>>>>>>>>>>>>
%<<<     Fractur Letters    <<<
%>>>>>>>>>>>>>>>>>>>>>>>>>>>>>>
\newcommand{\frakQ}{\mathfrak{Q}}
\newcommand{\frakW}{\mathfrak{W}}
\newcommand{\frakE}{\mathfrak{E}}
\newcommand{\frakR}{\mathfrak{R}}
\newcommand{\frakT}{\mathfrak{T}}
\newcommand{\frakY}{\mathfrak{Y}}
\newcommand{\frakU}{\mathfrak{U}}
\newcommand{\frakI}{\mathfrak{I}}
\newcommand{\frakO}{\mathfrak{O}}
\newcommand{\frakP}{\mathfrak{P}}
\newcommand{\frakA}{\mathfrak{A}}
\newcommand{\frakS}{\mathfrak{S}}
\newcommand{\frakD}{\mathfrak{D}}
\newcommand{\frakF}{\mathfrak{F}}
\newcommand{\frakG}{\mathfrak{G}}
\newcommand{\frakH}{\mathfrak{H}}
\newcommand{\frakJ}{\mathfrak{J}}
\newcommand{\frakK}{\mathfrak{K}}
\newcommand{\frakL}{\mathfrak{L}}
\newcommand{\frakZ}{\mathfrak{Z}}
\newcommand{\frakX}{\mathfrak{X}}
\newcommand{\frakC}{\mathfrak{C}}
\newcommand{\frakV}{\mathfrak{V}}
\newcommand{\frakB}{\mathfrak{B}}
\newcommand{\frakN}{\mathfrak{N}}
\newcommand{\frakM}{\mathfrak{M}}

\newcommand{\frakq}{\mathfrak{q}}
\newcommand{\frakw}{\mathfrak{w}}
\newcommand{\frake}{\mathfrak{e}}
\newcommand{\frakr}{\mathfrak{r}}
\newcommand{\frakt}{\mathfrak{t}}
\newcommand{\fraky}{\mathfrak{y}}
\newcommand{\fraku}{\mathfrak{u}}
\newcommand{\fraki}{\mathfrak{i}}
\newcommand{\frako}{\mathfrak{o}}
\newcommand{\frakp}{\mathfrak{p}}
\newcommand{\fraka}{\mathfrak{a}}
\newcommand{\fraks}{\mathfrak{s}}
\newcommand{\frakd}{\mathfrak{d}}
\newcommand{\frakf}{\mathfrak{f}}
\newcommand{\frakg}{\mathfrak{g}}
\newcommand{\frakh}{\mathfrak{h}}
\newcommand{\frakj}{\mathfrak{j}}
\newcommand{\frakk}{\mathfrak{k}}
\newcommand{\frakl}{\mathfrak{l}}
\newcommand{\frakz}{\mathfrak{z}}
\newcommand{\frakx}{\mathfrak{x}}
\newcommand{\frakc}{\mathfrak{c}}
\newcommand{\frakv}{\mathfrak{v}}
\newcommand{\frakb}{\mathfrak{b}}
\newcommand{\frakn}{\mathfrak{n}}
\newcommand{\frakm}{\mathfrak{m}}

%>>>>>>>>>>>>>>>>>>>>>>>>>>>>>>
%<<<  Caligraphic Letters   <<<
%>>>>>>>>>>>>>>>>>>>>>>>>>>>>>>
\newcommand{\calQ}{\mathcal{Q}}
\newcommand{\calW}{\mathcal{W}}
\newcommand{\calE}{\mathcal{E}}
\newcommand{\calR}{\mathcal{R}}
\newcommand{\calT}{\mathcal{T}}
\newcommand{\calY}{\mathcal{Y}}
\newcommand{\calU}{\mathcal{U}}
\newcommand{\calI}{\mathcal{I}}
\newcommand{\calO}{\mathcal{O}}
\newcommand{\calP}{\mathcal{P}}
\newcommand{\calA}{\mathcal{A}}
\newcommand{\calS}{\mathcal{S}}
\newcommand{\calD}{\mathcal{D}}
\newcommand{\calF}{\mathcal{F}}
\newcommand{\calG}{\mathcal{G}}
\newcommand{\calH}{\mathcal{H}}
\newcommand{\calJ}{\mathcal{J}}
\newcommand{\calK}{\mathcal{K}}
\newcommand{\calL}{\mathcal{L}}
\newcommand{\calZ}{\mathcal{Z}}
\newcommand{\calX}{\mathcal{X}}
\newcommand{\calC}{\mathcal{C}}
\newcommand{\calV}{\mathcal{V}}
\newcommand{\calB}{\mathcal{B}}
\newcommand{\calN}{\mathcal{N}}
\newcommand{\calM}{\mathcal{M}}

%>>>>>>>>>>>>>>>>>>>>>>>>>>>>>>
%<<<<<<<<<DEPRECIATED<<<<<<<<<<
%>>>>>>>>>>>>>>>>>>>>>>>>>>>>>>

%%% From Kac's template
% 1-inch margins, from fullpage.sty by H.Partl, Version 2, Dec. 15, 1988.
%\topmargin 0pt
%\advance \topmargin by -\headheight
%\advance \topmargin by -\headsep
%\textheight 9.1in
%\oddsidemargin 0pt
%\evensidemargin \oddsidemargin
%\marginparwidth 0.5in
%\textwidth 6.5in
%
%\parindent 0in
%\parskip 1.5ex
%%\renewcommand{\baselinestretch}{1.25}

%%% From the net
%\newcommand{\pullbackcorner}[1][dr]{\save*!/#1+1.2pc/#1:(1,-1)@^{|-}\restore}
%\newcommand{\pushoutcorner}[1][dr]{\save*!/#1-1.2pc/#1:(-1,1)@^{|-}\restore}










\newcommand{\rednote}[1]{{\color{red}#1}\scalebox{.1}{rednote}}
\newcommand{\bluenote}[1]{{\color{blue}#1}\scalebox{.1}{rednote}}

\newcommand{\funcdef}[4]{\begin{align*}
#1&\to #2\\
#3&\mapsto#4
\end{align*}}

\newcommand{\NewLecture}[3]{\section{#1 {\small(#2/#3/2011)}}}
\newcommand{\Extracurricular}[1]{
\section*{#1 {\small(extracurricular)}}
}
\makeatletter
\def\@seccntformat#1{Lecture \csname the#1\endcsname.\quad}
\makeatother
\title{Equivariant Stable Homotopy Theory}
\author{Mike Hopkins}
\date{}

\begin{document}
\maketitle
\begin{abstract}
These notes are based on a lecture course given by Mike Hopkins at Harvard in
which he discussed the solution of Arf-Kervaire invariant problem. This theorem
was proved by Mike Hill, Mike Hopkins and Doug Ravanel after $50$ years of
mystery.
\end{abstract}
\NewLecture{}{Wednesday 31}{8}
\begin{thm*}[Hill, Hopkins, Ravanel]
If $M$ is a stably framed smooth manifold of Kervaire invariant one, then $\dim
M$ is one of: 2, 6, 14, 30, 62 or 126.
\end{thm*}
\subsection*{Pontryagin (1930s)}
Studied smooth maps $S^{n+k}\to S^n$ in terms
of the preimage $f^{-1}(x_0)$, where $x_0$ is a regular value. As $x$ is
regular, $M_0=f^{-1}(x_0)$ is a smooth manifold.

Moreover, given a choice of basis of $T_{x_0}S^n$, we obtain a trivialisation of
the normal bundle of $M$ in $S^{n+k}$. Thus, $M$ is ``stably framed''. Also, if
$x_0$ and $x_1$ are both regular values, then (choosing a path from $x_0$ to
$x_1$, satisfying some decent condition), we get a cobordism $N$ from $M_0$ to
$M_1$. That is, $\partial N=M_0\sqcup M_1$ ``as stably framed manifolds''.
Pontryagin found:
\[\Maps(S^{n+k},S^n)/\text{htpy}\longleftrightarrow\{\text{stably framed
$k$-manifolds + details...}\}/\text{cobordism}.\]
We neglect to mention what the details are at this point. Instead, we stabilise:
the suspension homomorphism is compatible with the regular value construction,
and one obtains:
\[\pi_k^\text{st}S^0\longleftrightarrow\{\text{stably framed
$k$-manifolds}\}/\text{cobordism}.\]
Note that we are only considering compact manifolds in our discussion.
\subsection*{When \texorpdfstring{$k=0$}{k=0}:}
Any $0$-manifold embeds in $\R$, and a framing there means a choice of
orientation at each point. Opposing points cancel, and the lone cobordism
invariant of stably framed $0$-manifolds is the number of points, counted with
sign.

Of course, this corresponds to the fact that $\pi_0^\text{st}S^0=\Z$.
\subsection*{When \texorpdfstring{$k=1$}{k=1}:}
We can think of two framings of $S^1$. First is the obvious, were one views
$S^1$ as the boundary of $D^2\in \R^2$, and choose a nonvanishing section for
the normal bundle which simply points outward at every point. This represents
zero, as it extends to the disk which $S^1$ bounds.

Next there is a framing where one embeds $S^1$ as a figure $8$ in $\R^2$,
perturbs the two segments of the crossing away from each other into the third
dimension. On of the sections always points in directions parallel to the plane,
while the other section always points upward.

Apparently, these two are different, reflecting that $\pi_2^\text{st}=\Z/2$.

\bluenote{%\color{blue}
Since the tangent bundle of a circle is trivial, the normal bundle of
the circle embedded into Euclidean space is too. The framings of a circle
in $\mathbb{R}^{n+1}$ are just given by the fundamental group of $O(n)$.
For the non-trivial framing I prefer to think of a circle embedded in
3-space in the standard manner. The circle is a roller-coaster ride and as
you move once round the circle you turn once clockwise relative to your
direction of travel. You could have turned around $n$ times in either
direction. This shows $\pi_3(S^2)=\mathbb{Z}$. Once you're embedded in
$4$-space, however, you can slide between the roller-coaster ride taking
you twice around and $0$ times around. So $\pi_4(S^3)=\mathbb{Z}_2$.}

\rednote{ Cool --- I'd like to glue this in appropriately, but would like to
know the exact statement of Pontryagin's result first. We should discuss this further some time.}

\subsection*{When \texorpdfstring{$k=2$}{k=2}:}
Suppose that a manifold is embedded in $\R^n$, and given a framing of the normal
bundle. This framing orients the normal bundle, and using the ambient
orientation of $\R^n$, this induces an orientation on the tangent bundle. Thus,
we only need to consider \emph{oriented} 2-manifolds.

Now suppose that the framing $\overrightarrow{e}$ on $S^2$ represents zero. If
$\overrightarrow{e_1}$ is another framing, then they are related by
$\overrightarrow{e_1}=T\overrightarrow{e}$ for $T:S^2\to\GL_N(\R)$. Now the
second homotopy group of a Lie group is known to be zero, so that $S^2$ with
\emph{any} framing represents zero.

%If that seemed a little complicated, then it is clear that the framings of
%$S^2$, embedded in the standard way in $\R^{n+2}$, correspond to homotopy
%classes of maps $S^2\to O(n)$. That
\bluenote{%\color{blue} 
I would have said the following. Embedding $S^2$ in
$\mathbb{R}^{n+2}$ in the standard way it is clear that
the framings of $S^2$ are given by $\pi_2(O(n))=0$. So the standard framing
of $S^2\subset\mathbb{R}^3$ is the only one.}

\rednote{I have the following thought: it's not obvious right now that
framings actually correspond bijectively with $\pi_2(O(n))$ --- we can see
though that there's a surjection from $\pi_2(O(n))$ to the set of framings. It
there actually always a bijection between $[X,O(n)]$ and the set of framings?
(This is all getting a little silly by now.)}



Now we'd like to deal with a surface $\Sigma\subset\R^N$ of higher genus. The
idea is to use surgery. That is, choose a circle $M$ on $\Sigma$. We can cut
along $M$, and glue a disc to either side of the cut to form a new surface
$\Sigma'$ with genus one lower. However, we need the framing induced on $M$ to
extend over the disk, for the following reason. Suppose that the framing on $M$
does extend. Then we can imagine that $\Sigma'$, as a subspace of $\R^N$, is
literally sitting \emph{within} $\Sigma$. Then there is an unframed cobordism
between $\Sigma$ and $\Sigma'$ given by filling in the space in between the two
manifolds to get a threefold with boundary. As the framing of $M$ extends to its
bounding disk, we can make this unframed cobordism into a framed cobordism.

Thus, when such a circle $M$ exists, we can perform framed surgery, to reduce
the genus by one. If such a circle always exists, then all framed 2-manifolds
will represent zero, so that $\pi_2^\text{st}S^0=0$.

Pontryagin gave the following argument. There is a function
$\phi:H_1(\Sigma;\Z/2)\to\Z/2$, in which $[M]$ maps to the framed cobordism
class of $M$, where here $M$ is a $1$-manifold in whose fundamental class we are
interested. Now if $\Sigma$ has positive genus, $H_1(\Sigma;\Z/2)$ has dimension
exceeding one, so that $\ker\phi$ is nonzero. That is, one can always find a
circle $M$ along which to perform framed surgery!

The flaw in Pontryagin's argument is that $\phi$ is not linear. Instead it is
quadratic. More specifically, $\phi$ is a quadratic refinement of the
intersection pairing:
\[\phi(x+y)-\phi(x)-\phi(y)=I(x,y).\]
The truth of the matter is that the function $\phi$, being quadratic, has an Arf
invariant $\ArfInvariant(\phi)\in\Z/2$, and it is this invariant which gives the
identification with $\pi_2^\text{st}S^0=\Z/2$.
\begin{question}
In which dimensions can every element of $\pi_k^\text{st}S^0$ be represented by
a homotopy sphere? (As in dimension one, but not in dimension 2.)
\end{question}
\begin{answer}
In all dimensions but 2, 6, 14, 30, 62, and possibly 126.
\end{answer}
\subsection*{Kervaire (1960)}He constructed a map
$\phi:H_{2k+1}(M^{4k+2};\Z/2)\to\Z/2$ which was a quadratic refinement of the
intersection pairing, and defined $\Phi(M)=\ArfInvariant(\phi)$. This is the
Kervaire invariant. He also showed that if $M$ is a smooth manifold of dimension
10 or 18, then $\Phi(M)=0$.

On the other hand, he constructed a 10-manifold $M$ with $\Phi(M)=1$, and thus
produced a manifold with no smooth structure. To construct the manifold $M$, let
$X$ be a certain gluing of two copies of the disk bundle of the tangent bundle
of $S^5$. Then let $N=\partial X\simeq S^9$, and define $M=X\cup\cone(N)$. Then
$M$ is a topological 10-manifold with Kervaire invariant one.
\begin{question}
In which dimensions can $\Phi(M)$ be nonzero (when $M$ is smooth and stably
framed)?
\end{question}
\begin{answer}
Only in dimensions 2, 6, 14, 30, 62, and possibly 126. 
\end{answer}
\subsection*{Browder}
Browder cleared a path to answering these questions by showing:
\begin{itemise}
\item $\Phi(M)=0$ unless $\dim M=2^{j+1}-2$;
\item An $M$ exists with $\Phi(M)\neq0$ \Iff there is some
$\theta_j\in\pi_{2^{j+1}-2}^\text{st}(S^0)$ represented by $h_j^2$ at the $E^2$
term of the Adams spectral sequence.
\end{itemise}
It is in these terms that we will study this question in this course.
\NewLecture{}{Friday 2}{9}
The HHR result states that an $\theta_j$ does not exist when $j\geq7$. There are
four main parts of the proof.
\begin{itemise}
\item[(1)] Construct a multiplicative cohomology theory $\Omega$.
\item[(2)] Prove the \textbf{detection theorem}: If $\theta_j$ exists, it is
detected as a nonzero image in $\Omega_{2^{j+1}-2}(S^0)$.

\INDENT To explain this, we note that there is a distinguished element
$1\in\Omega_0(S^0)$. In particular, given $\theta_j:S^{2^{j+1}-2}\to S^0$ (a map
of spectra), we are interested in $\theta_j^*(1)\in\Omega_{2^{j+1}-2}(S^0)$:
\[\qquad\xymatrix{
\makebox[0cm][r]{$\Omega_{2^{j+1}-2}(S^0)=$\ }\Omega^0(S^{2^{j+1}-2})&
\ar[l]_{\theta_j^*}\Omega^0(S^{0})\ni 1
}\]
That is, we plan to understand the Kervaire invariant by understanding a
``degree'' in the cohomology theory $\Omega$.

One might rephrase the above by saying that the composite
\[\Sigma^{2^{j+1}-2}\mathbb{S}\overset{\theta_j}{\to}\mathbb{S}\overset{\eta}
{\to}\Omega,\]
an element of $\pi_{2^{j+1}-2}(\Omega)$, is non-zero (where $\eta$ is the unit
of the multiplicative theory $\Omega$).
\item[(3)] Prove the \textbf{gap theorem}: that $\pi_i(\Omega)=0$ for $-4<i<0$.
\item[(4)] Prove the \textbf{periodicity theorem}: that
$\pi_i(\Omega)=\pi_{i+256}(\Omega)$ for all $i$.
\end{itemise}
These steps together prove the theorem. Suppose that $\theta_j$ exists. Then
$\pi_{2^{j+1}-2}(\Omega)\neq0$, by (2). However, if $j\geq7$, this group is
zero, by (3) and (4).

For an example of a theory like this, we need look no further than $\KO$. In
fact, the gap here is somehow ubiquitous --- by the end of the course, we'll
have a general construction in which we see these phenomena, and $\KO$ will
arise in this general setting.

The proof of the detection theorem that we will give is not so natural. This is
because we do not understand the Kervaire invariant very geometrically. If it
was given by, say, integrating some characteristic classes of the tangent bundle
of a manifold, then we might have a more natural proof. {\small(This is the
least exciting part of the course, and will be left until the end, in case
something needs to be jettisoned.)}

Note that these techniques were not set up to solve the Arf-Kervaire invariant
problem, more to develop better understanding of the homotopy groups of sphere.
Mike described what happened: ``It was as if we were trying to invent a highly
wear-resistant material, but discovered that when we wore it into the bathroom,
we became invisible.''

\subsection*{Atiyah ``\texorpdfstring{$\K$}{K}-theory and reality''}
In this paper, Atiyah introduces the $C_2$-equivariant cohomology theory $\KR$
called `real $\K$-theory' (Mike Hopkins would prefer if it were denoted
${\K_\C}$). This is a hybrid of $\K$ and $\KO$. In fact, given a space $X$ with
a $C_2$-action, $C_2$ acts on $\KR^*(X)$, and the fixed points are isomorphic to
$\KO^*(X)$. On the other hand, $\KR^*(X\times C_2)$ is isomorphic to $\K(X)$.

Moreover, Atiyah used complex Bott periodicity to prove real Bott periodicity
--- that $\KO$ is periodic. That is, classical Bott periodicity states:
\[\K^0(X\wedge S^2)\cong \K^0(X).\]
The same proof applies (\textbf{is this what was said?}) to show that:
\[\KR^0(X\wedge S^{\rho_2})=\KR^0(X).\]
Finally, he used this to imply the classical periodicity theorem, $\K^0(X\wedge
S^8)=\K^0(X)$.

In the above, $S^{\rho_2}$ needs some explanation. Given a representation $\rho$
of a finite group $G$ on a finite dimensional real vector space $V$, define
$S^{\rho}$ to be the one point compactification of $V$, with the obvious
$G$-action. Here, $\rho_2$ stands for the regular representation of $C_2$.
Note also that $S^{\rho_2}$ can be viewed as $\C\cup\{\infty\}$, where $\C$ has
a $C_2$-action given by complex conjugation.\footnote{Complex conjugation on
$\C$ is simply reflection across the real line; writing $C_2=\{\imath,x\}$,
multiplication by $x$ on $\R C_2$ is also just a reflection --- consider the
basis $\{\imath+x,\imath-x\}$.}

\subsection*{HHR's analogue}
HHR construct an analogue of this situation, in which $\KO$ is replaced with
$\Omega$, and $\KR$ (which should be called $\K_\C$) is replaced with $\OmegaO$.
Now by construction, he has
\[\OmegaO^0(X\wedge S^{\rho_8})\cong\OmegaO^0(X),\]
and, inspired by Atiyah's work, this is used to prove that $\Omega^0(X\wedge
S^{256})=\Omega^0(X)$.

We should discuss the use of the character $\Octonions$, which refers to the
octonions. In the $\KR$ story, $C_2$ is relevant as it is the Galois group
$\Gal(\C/\R)$. The point is that in the HHR construction, $C_8$ is acting as a
`Galois group' $\Gal(\Octonions/\R)$.

To justify such a claim, we would need an action of $C_8$ on $\Octonions$, whose
fixed points are $\R$, and for which a Hilbert basis theorem holds. That is,
$\Octonions$ is isomorphic to $\rho_8$ as a $C_8$-representation over $\R$.
Unfortunately, no such action exists, however:
\begin{fact*}
There is an antiautomorphism $\sigma:\Octonions\to \Octonions$, unique up to
inner automorphisms, such that $\sigma^8=\id$, making $\Octonions\cong\rho_8$ as
a $C_8$-representation. Moreover, we get a Galois correspondence.
\[\begin{array}{c|c|c|c}
\{\id\}&\langle\sigma^4\rangle&\langle\sigma^2\rangle&C_8\\\hline
\Octonions&\Quaternions&\C&\R
\end{array}\]
\end{fact*}
\subsection*{The slice tower}
This is an analogue of the Postnikov tower of spectra, perhaps the main
innovation given by HHR's work. Recall that for a spectrum $E$, there are
fibrations $F^n\to P^nE\to P^{n-1}E$, and compatible maps $E\to P^nE$ such that:
\begin{itemize}\squishlist
\item $\pi_{i}(P^n E)=0$ for $i>n$;
\item the maps $E\to P^nE$ induce isomorphisms on $\pi_{i}$ for $i\leq n$;
\end{itemize}
Then by the long exact sequence in homotopy, $F^n=H(\pi_n(E))\wedge S^n$. This
data is called a Postnikov tower for $E$.
The associated graded spectrum is the product $\prod_nH(\pi_n(E))\wedge S^n$.
%Note that this associated graded spectrum is the most na\"ive spectrum with the
%same coefficients as $E$.

If $E=\K$ is the $\K$-theory spectrum, then one might say that its associated
graded spectrum as a particularly na\"ive 2-periodic spectrum --- writing it as
$\prod_n {H\Z}\wedge S^{2n}$ makes this periodicity obvious. $\KR$, however, has
a period which is not a number, but a representation, and there is a some tower,
the slice tower for $\KR$, whose associated graded is the product $\prod_n
{H\Z}\wedge S^{n\rho_2}$.

Note that the spectra ${H\Z}\wedge S^{n\rho_2}$ are not Eilenberg-MacLane
spectra in equivariant homotopy, but something different. Constructing the slice
tower was `perhaps the main innovation' given by HHR.

\Extracurricular{$\KR$: A brief discussion}
%\section*{$\KR$: A brief discussion {\small(extracurricular)}}
\subsection*{A first approach}
\begin{defn*} A space $X$ is said to be a real space if it is equipped with an an involution $\tau$, i.e. if it is a $\mathbb{Z}_2$-space. We write $\overline{x}$ for $\tau(x)$.
\end{defn*}
\begin{defn*} A real vector bundle over a real space $X$ is a complex vector bundle $p:E\to X$ satisfying the following conditions: $E$ is a real space; $p$ is $\mathbb{Z}_2$-equivariant; we obtain anti-linear isomorphisms $E_x \to E_{\overline{x}}$ by restricting the involution on $E$.
\end{defn*}
\begin{defn*} 
A morphism between real vector bundles over a real space $X$ is a homomorphism of complex vector bundles commuting with the involutions.
\end{defn*}
\begin{defn*}
The Grothendieck group of the category of real vector bundles over a real space X is denoted by $\KR(X)$.
\end{defn*}
Note that if $X$ is a real space with trivial involution and $E$ is a real vector bundle over $X$, then the involution on $E$ restricts to an involution on each fibre $E_{x}=E_{\overline{x}}$. This involution is linear over $\mathbb{R}$ and so each fibre decomposes as a direct sum of the $+1$-eigenspace and the $-1$-eigenspace. A real vector bundle over $X$ (viewed as a space) gives a real vector bundle over $X$ (viewed as a real space) by complexification --- the involution is complex conjugation. A real vector bundle over $X$ (viewed as a real space) gives a real vector bundle over $X$ (viewed as a space) by taking the $+1$-eigenspace. These constructions are inverse to one another, and are natural, so that $\KR(X)=\KO(X)$ whenever $X$ is a real space with trivial involution.

Suppose we start with a based space $X$. We can equip $X\coprod X$ \rednote{(do we mean $\vee$?)} with the involution which swaps the factors. Complex vector bundles over $X$ are in bijection with real vector bundles over $X\coprod X$, so that $\K(X)=\KR(X\coprod X)$.
\subsection*{An alternative approach}
Note that $U$ has an involution given by complex conjugation and its fixed points are $O$. Thus $BU$ becomes a real space with fixed points $BO$; \bluenote{(to discuss: why does $B$ commute with fixed points)} equip $\mathbb{Z}$ with the trivial involution. This allows us to make the following
\begin{defn*} Let $X$ be a based real space (the basepoint is fixed by the involution). We define $\widetilde{\KR}(X)$ to be the group $[X,\mathbb{Z}\times BU]_{\mathbb{Z}_2}$ \rednote{(what's the difference between a lower and an upper $\Z_2$ decoration?} of $\mathbb{Z}_2$-homotopy classes of $\mathbb{Z}_2$-maps.
\end{defn*}
Suppose $X$ is a based real space with trivial involution. Then \[\widetilde{\KR}(X)=[X,\mathbb{Z}\times BU]_{\mathbb{Z}_2}=[X,\mathbb{Z}\times BU^{\mathbb{Z}_2}]=[X,\mathbb{Z}\times BO]=\widetilde{\KO}(X)\]
Suppose $X$ is a based space. We can equip $X\vee X$ with the involution which swaps the factors. Then \[\widetilde{\KR}(X\vee X)=[X\vee X,\mathbb{Z}\times BU]_{\mathbb{Z}_2}=[X,\mathbb{Z}\times BU]=\widetilde{\K}(X).\]
\subsection*{Periodicity}
Let $S^{2,1}$ \bluenote{(perhaps now we'd like to change this notation to $S^{\rho_{\Z_2}}$ or something)} denote the one point compactification of $\mathbb{R}^2$ equipped with the involution $(x,y)\mapsto (x,-y)$.

\begin{thm*}
Let $X$ be a real space. Then we have a natural isomorphism
\[\widetilde{\KR}(S^{2,1}\wedge X)\cong\widetilde{\KR}(X).\] Equivalently, we have an equivariant homotopy equivalence:
\[\Omega^{2,1}(\mathbb{Z}\times BU)\simeq\mathbb{Z}\times BU.\]
\end{thm*}
\noindent This allows us to define a $\mathbb{Z}_2$-equivariant spectrum $\KR$ with $\KR^{\mathbb{Z}_2}=\KO$.

\Extracurricular{Some Useful Adjunctions}
Let $G$ be a topological group.
Let $\mathscr{U}$ denote the category of (compactly generated, weakly Hausdorff) spaces with continuous maps, and let $G\mathscr{U}$ denote the category of left $G$-spaces with $G$-equivariant continuous maps.

Let $X$ and $Y$ be $G$-spaces. Then $\mathscr{U}(X,Y)$ is a $G$-space: for $f\in\mathscr{U}(X,Y)$, we define $g\cdot f$ by  $(g\cdot f)(x)=g(f(g^{-1}x))$; we see $\mathscr{U}(X,Y)^G=G\mathscr{U}(X,Y)$.
\subsubsection*{Adjunctions between $\scrU$ and $G\scrU$}
We have a functor $1_G:\mathscr{U}\to G\mathscr{U}$ which equips a space $X$ with a trivial $G$-action, and which acts as the identity on morphisms.
We also have functors (`the $G$-invariants' and `the $G$-orbits'):
\[(\DASH)^G:G\mathscr{U}\to\mathscr{U}, \hspace{3pt} X\mapsto X^G; \hspace{10pt}
(\DASH)/G:G\mathscr{U}\to\mathscr{U}, \hspace{3pt} X\mapsto X/G\]
\begin{thm*} $1_G$ is left adjoint to $(\DASH)^G$ and right adjoint to $(\DASH)/G$:
\[(\DASH)/G:G\scrU\longleftrightarrow\scrU: 1_G:\scrU\longleftrightarrow G\scrU: (\DASH)^G.\]
\end{thm*}
\begin{proof} \bluenote{(More a restatement)} Let $K$ be a space, which we endow with a trivial $G$-action and $X$ be a $G$-space. Then we have natural homeomorphisms
\[G\mathscr{U}(K,X)\cong\mathscr{U}(K,X^G);\hspace{10pt}
\mathscr{U}(X/G,K)\cong G\mathscr{U}(X,K).\qedhere\]
\end{proof}
\subsubsection*{Adjunctions between $H\scrU$ and $G\scrU$, for a subgroup $H\subset G$}
Let $H$ be a subgroup of $G$ and let $Y$ be an $H$-space. Then we have the forgetful functor
\[U:G\mathscr{U}\to H\mathscr{U}.\]
We define $G\times_H Y$ to be the coequaliser of the maps $G\times H\times Y\to G\times Y$ given by the right action of $H$ on $G$ and the left action of $G$ on $Y$. The left action of $G$ on itself makes $G\times_H Y$ into a $G$-space.

We define $\textup{Map}_H(G,Y)$ to be the subset of $\mathscr{U}(G,Y)$ defined by
\[\{f:G\to Y: f=l_h\circ f\circ r_h\}.\]
The action $g\cdot f=f\circ l_{g^{-1}}$ makes $\textup{Map}_H(G,Y)$ into a $G$-space.

These constructions define two functors:
\[G\times_H(\DASH):H\mathscr{U}\to G\mathscr{U}; \hspace{10pt}
\textup{Map}_H(G,\DASH):H\mathscr{U}\to G\mathscr{U}\]
\begin{thm*} $U$ is right adjoint to $G\times_H(\DASH)$ and left adjoint to $\textup{Map}_H(G,\DASH)$:
\[G\times_H(\DASH):G\scrU\longleftrightarrow\scrU: U:\scrU\longleftrightarrow G\scrU: \Map_H(G,\DASH).\]
\end{thm*}
\begin{proof} Let $X$ be a $G$-space and $Y$ be an $H$-space. We wish to define natural isomorphisms
\[G\mathscr{U}(G\times_H Y,X)\cong H\mathscr{U}(Y,X);\hspace{10pt}
H\mathscr{U}(X,Y)\cong G\mathscr{U}(X,\textup{Map}_H(G,Y)).\]
Given a $G$-map $f:G\times_H Y\to X$, we can define an $H$-map $Y\to X$ by $y\mapsto f[e,y]$. Given an $H$-map $f:Y\to X$ we can define a $G$-map $G\times_H Y\to X$ by $[g,y]\mapsto gf(y)$. We note that the unit $Y\to G\times_H Y$ of the adjunction is $y\mapsto [e,y]$, and the counit $G\times_H X\to X$ is $[g,x]\mapsto x$. Defining the adjunction in terms of these maps makes it clear that the adjunction is actually a homeomorphism of topological spaces.

The unit $X\to\textup{Map}_H(G,X)$ and counit $\textup{Map}_H(G,Y)\to Y$ of the other adjunction are given by $x\mapsto (g\mapsto g^{-1}x)$ and $f\mapsto f(e)$ respectively. Again, we obtain a homeomorphism.
\end{proof}
\begin{thm*} If $X$ is a $G$-space then we have G-homeomorphisms:
\[G\times_H X\cong (G/H)\times X;\text{\qquad and\qquad}
\textup{Map}_H(G,X)\cong\mathscr{U}(G/H,X).\]
\end{thm*}
\begin{proof}
Define
\[G\times_H X\to (G/H)\times X,\hspace{3pt}[g,x]\to ([g],gx)\]
and
\[\textup{Map}_H(G,X)\to\mathscr{U}(G/H,X),\hspace{3pt}f\mapsto([g]\mapsto g(f(g))).\qedhere\]
\end{proof}
\begin{thm*} $G/H\times(\DASH):\mathscr{U}\to G\mathscr{U}$ is left adjoint to $(\DASH)^H:G\mathscr{U}\to\mathscr{U}$.
\end{thm*}
\begin{proof}
Let $K$ be a space which we endow with a trivial $G$-action and $X$ be a $G$-space. Then
\[G\mathscr{U}(G/H\times K, X)\cong G\mathscr{U}(G\times_H K,X)\cong
H\mathscr{U}(K,X)\cong\mathscr{U}(K,X^H),\]
and the isomorphism is given by $(G/H\times K\overset{f}{\to} X)\mapsto (k\mapsto f([e],k))$.
\end{proof}
\begin{thm*}
$(\DASH)/H:G\mathscr{U}\to\mathscr{U}$ is left adjoint to $\mathscr{U}(G/H,\DASH):\mathscr{U}\to G\mathscr{U}$.
\end{thm*}
\begin{proof}
Let $K$ be a space which we endow with a trivial $G$-action and $X$ be a $G$-space. Then
\[\mathscr{U}(X/H,K)\cong H\mathscr{U}(X,K)\cong G\mathscr{U}(X,\textup{Map}_H(G,K))\cong G\mathscr{U}(X,\mathscr{U}(G/H,K)).\qedhere\]
\end{proof}
\begin{thm*}
The adjunction $\mathscr{U}(X\times Y,Z)\cong\mathscr{U}(X,\mathscr{U}(Y,Z))$ restricts to an adjunction $G\mathscr{U}(X\times Y,Z)\cong G\mathscr{U}(X,\mathscr{U}(Y,Z))$ for $G$-spaces $X,Y$ and $Z$.
\end{thm*}
\subsubsection*{With basepoints}
Let $\mathscr{T}$ denote the category of based spaces with basepoint preserving continuous maps and let $G\mathscr{T}$ denote the category of based $G$-spaces with $G$-equivariant basepoint preserving continuous maps.
All of the above continues to hold in the based setting. We replace $\mathscr{U}$, $\times$, $G$, $(G/H)$, and $\textup{Map}_H$ with $\mathscr{T}$, $\wedge$, $G_+$, $(G/H)_+$, and $\textup{Map}^*_H$, respectively.

In particular, given a representation $V$ of $G$, we have
$G\mathscr{T}(X\wedge S^V, Y)\cong G\mathscr{T}(X,\mathscr{T}(S^V,Y))$. We write $\Sigma^V X$ for $X\wedge S^V$ and $\Omega^V X$ for the $G$-space $\mathscr{T}(S^V,Y)$.

Finally, we note that all these adjunctions are sufficiently natural that they pass to homotopy classes of maps between spaces.


\NewLecture{}{Wednesday 7}{9}
One reference to try is Adams' ``Prerequisites for Carlsson's work''.

Now let $G$ b e a finite group (could probably be compact Lie), and consider the category of spaces with a left $G$-action. By spaces we mean compactly generated weak Hausdorff spaces.
\begin{itemise}
\item A space is compactly generated if it is the colimit of its compact subspaces, i.e.\ a subset is open \Iff its intersection with each compact subset is open therein.
\item A space $X$ is weak Hausdorff if the diagonal $X\to X\times X$ is closed, where we give $X$ the compactly generated topology.
\item Write $X_n$ for the real line with the segment $[-1/n,1/n]$ doubled. The colimit of the $X_n$ in topological spaces is the line with the origin doubled, whereas, in weak Hausdorff spaces, it is just the line.
\end{itemise}
Alternatively, a space could be a simplicial set.

Now there are two reasonable categories to mention here, $\scrT^G$ and $\scrT_G$, both of whose objects are just left $G$-spaces. We write $\scrT^G(X,Y)$ for $G$-equivariant maps, and $\scrT_G(X,Y)$ for the $G$-space of non-equivariant maps.

To approach the homotopy theory of $G$-spaces, there are two approaches. There is that of abstract homotopy theory, through the construction of a model category. There are also the more computational aspects, which we discuss now.
\subsection*{$G$-CW-complexes}
A $G$-CW-complex should be something built from equivariant cells. But what are they? Consider two motivating examples:
\begin{enumerate}\squishlist
\item The circle is a $\Z_2$ space, the action via horizontal reflection. Call this $E_1$.
\item The circle is a $\Z_2$ space, the action via the antipodal map. Call this $E_2$.
\end{enumerate}
\textbf{First idea:} Perhaps a cell should be the unit ball in a finite dimensional unitary\footnote{As $G$ is finite, all finite dimensional representations can be made unitary.} representation of $G$.
In this case, $E_1$ is $D(V)/S(V)$, where $V$ is the sign representation. However, $E_2$ can't be of the form $D(V)/S(V)$, so we're not quite satisfied with this na\"ive approach.

\noindent \textbf{Second idea:} We could define a $G$-cell to be a space of the form $S\times \Delta[n]$, where $S$ is a discrete (finite) $G$-set. We could motivate this as follows. If $X_\cdot$ is a $G$-simplicial set (i.e.\ a simplicial $G$-set, a simplicial set in which each set is a $G$-set and each map is $G$-equivariant) then we would like $|X|$ to be a $G$-CW-complex. Well $|X|$ is formed in skeleta, by the pushout diagram:
\[\xymatrix@R=.6cm{
\ar[d]X'_n\times\partial\Delta[n]\ar@{^{(}->}[r]&X'_n\times\Delta[n]\ar[d]\\
|\Skel^{n-1}X|\ar[r]&|\Skel^nX|
}\]
where $X'_n$ is the $G$-set of nondegenerate $n$-simplices. Thus:
\begin{defn*}
A $G$-CW-complex is a $G$-space built from $G$-cells $S\times D^n$, with $S$ a discrete \bluenote{(finite?)} $G$-set.\footnote{In the literature, it is common to restrict to cells of the form $G/H\times D^n$, but why bother?}
\end{defn*}
Note then that $E_1$ then has two $0$-cells, $\{*\}\times D^0$, and one 1-cell, $\Z_2\times D^1$, and now, $E_2$ has one $0$-cell, $\Z_2\times D_0$, and one 1-cell, $\Z_2\times D^1$.
\subsection*{Homotopy Groups}
Now we'd like to study $[X,Y]^G=\pi_0(\scrT^G(X,Y))$. \rednote{(right?)} We'd really like a Whitehead theorem --- an analogue of the fact that a weak equivalence of CW-complexes is a homotopy equivalence. For a $G$-analogue, we'll need to know what replaces homotopy groups!

We'll definitely need a condition which guarantees that
\[[S\times S^{n-1},X]^G\to[S\times S^{n-1},Y]^G\text{\ \ is a bijection.}\]
If we know that this holds for $S=S_1$ and for $S=S_2$, then it will hold when $S=S_1\sqcup S_2$, so we'll only need to ensure that it holds when $S=G/H$. Now by the above adjunctions (as $S^{n-1}$ has the trivial action):
\[[G/H\times S^{n-1},X]^G=[S^{n-1},X^H].\]
So the natural $G$-Whitehead theorem to hope for is
\begin{thm*}
A map $X\to Y$ of $G$-CW-complexes is an equivariant homotopy equivalence \Iff for all $H\subset G$ and all $n$,
\[[S^{n-1},X^H]\to[S^{n-1},Y^H]\text{\ \ is a bijection}.\]
\end{thm*}
\noindent Consistent with this theorem, we define:
\begin{defn*}
A map $X\to Y$ of $G$-spaces is a weak equivalence if for all $H\subset G$, the induced map $X^H\to Y^H$ is a weak equivalence.
\end{defn*}
\noindent
Thus, the $G$-equivariant homotopy theory of spaces is describable in terms of the ordinary homotopy theory of the $H$-fixed point spaces for all $H\subset G$.  For example:
\begin{thm*}
If $X$ is a $G$-CW-complex, then $X^H$ is a CW-complex. Suppose that $\dim X^H<n(H)$, for $n$ a function on conjugacy classes of subgroups of $G$. Suppose that $Y$ is a $G$-space, and $\pi_i(Y^H)=0$ for $i\leq n(H)$. Then $[X,Y]^G=*$.
\end{thm*}
\NewLecture{}{Friday 9}{9}
Now really lecture notes here --- more a summary of some relevant material.
\subsection*{Homotopy Groups}
Suppose that $X$ is a based space and $Y$ is based CW-complex. If we try to define a based map $f:Y\to X$ by induction on the skeleta of $Y$, the obstructions lie in the homotopy groups of $X$. If $X$ is a based $G$-space and $Y$ is a based $G$-CW-complex, we would like to find obstructions to defining equivariant based maps $f:Y\to X$ by induction on the skeleta of $Y$. This desire motivates the definition of the homotopy groups of $X$:
%\[\underline{\pi}_n(X):(\GSet)^{\textup{op}}\to\AbGp,
%\hspace{4pt}
%\vartheta\mapsto [\vartheta_+\wedge S^n,X]^G\]
\funcdef{\underline{\pi}_n(X):(\GSet)^{\textup{op}}}{\AbGp}
{\vartheta}{[\vartheta_+\wedge S^n,X]^G}
Note that $\underline{\pi}_n(X)$ sends coproducts to products, and that a based equivariant map $f:X\to X'$ induces a natural transformation $f_*:\underline{\pi}_n(X)\to\underline{\pi}_n(X')$. In particular, $\underline{\pi}_n$ is a functor
\[G\mathscr{T}\to \Cat_{\coprod\shortmapsto\prod}((\GSet)^{\textup{op}},\AbGp).\]
Here, the target is the full subcategory of the category of functors $(\GSet)^{\textup{op}}\to\AbGp$ consisting of functors which send coproducts to products.
\begin{thm*}
As in ordinary homotopy theory, we have $\underline{\pi}_n(X)\simeq\underline{\pi}_{n-1}(\Omega X)$.
\end{thm*}
\begin{proof}
$\underline{\pi}_n(X)=[(\DASH)_+\wedge S^n,X]^G\cong
[(\DASH)_+\wedge S^{n-1},\Omega X]^G=\underline{\pi}_{n-1}(\Omega X).$
\end{proof}
\begin{thm*}
For any $\vartheta\in \GSet$, and for any $H\subset G$, we have:
\[\underline{\pi}_n(X)_{\vartheta}\cong\pi_n(G\mathscr{U}(\vartheta,X))
\text{ \ and\ \ }\underline{\pi}_n(X)_{G/H}\cong\pi_n(X^H).\]
\end{thm*}
\begin{proof} For the first claim:
\[[\vartheta_+\wedge S^n,X]^G\cong[S^n,\mathscr{T}(\vartheta_+,X)]^G
\cong[S^n,\mathscr{U}(\vartheta,X)^G]
=[S^n,G\mathscr{U}(\vartheta,X)]
=\pi_n(G\mathscr{U}(\vartheta,X))\]
so that $\underline{\pi}_n(X)_{\vartheta}\cong
\pi_n(G\mathscr{U}(\vartheta,X))$. Since $G\mathscr{U}(G/H,X)=X^H$, we obtain
$\underline{\pi}_n(X)_{G/H}\cong\pi_n(X^H)$.
Alternatively, we can see this by noting
$\underline{\pi}_n(X)_{G/H}=[G/H_+\wedge S^n,X]^G\cong [S^n,X^H]=\pi_n(X^H).$
\end{proof}

Let $\Orb_G$ denote the category consisting of subgroups of $G$ with morphisms generated by conjugations and inclusions. We have a fully faithful functor
\funcdef{\Orb_G}{\GSet}{H}{G/H}
%\[\Orb_G\to \GSet,\hspace{4pt} H\to G/H.\]
which is injective on objects, so we can think of $\Orb_G$ as a full subcategory of $\GSet$. Taking this point of view, we see that the inclusion of the full subcategory category generated by $\Orb_G$ under coproducts is an equivalence of categories. Thus, the functor above induces an equivalence of functor categories
\[\Cat((\Orb_G)^{\textup{op}},\AbGp)\overset{\sim}{\longleftarrow}\Cat_{\coprod\shortmapsto\prod}((\GSet)^{\textup{op}},\AbGp).\]
This demonstrates that the homotopy groups $\underline{\pi}_n(X)$ are determined by their restriction to the subcategory $\Orb_G$.

\subsection*{Eilenberg MacLane Spaces}
Let
\[M:(\GSet)^{\textup{op}}\to\AbGp\]
be a functor taking coproducts to products. We wish to construct a $G$-CW-complex $Y$ such that $\underline{\pi}_n(Y)=M$ and $\underline{\pi}_i(Y)=0$ when $i\neq n$. Such a $G$-CW-complex will then be denoted by $K(M,n)$.

Let $\vartheta$ and $\psi=\coprod_{i\in I}\psi_i$ be $G$-sets, where $\vartheta$ and each $\psi_i$ is of the form $G/H$ and let
\[X=\psi_+\wedge S^n={\textstyle\bigvee_{\!\psi}} S^n.\]
Then, as $\vartheta\simeq G/H$, a $G$-map $\vartheta\to \bigvee_{\!\psi} S^n$ is determined by a point of $S^n$ and an $H$-invariant element of $\psi$. However, $\psi^H=\GSet(\vartheta,\psi)$, so that:
\[G\mathscr{U}(\vartheta,X)=G\mathscr{U}(\vartheta,{\textstyle\bigvee_{\!\psi}} S^n)=\mathop{\textstyle{\bigvee}}_{\GSet(\vartheta,\psi)} S^n.\]
However, even when $\vartheta$ is not transitive, this equality must hold. Hence, for any $\vartheta\in \GSet$:
\[\underline{\pi}_n(X)_{\vartheta}\cong\pi_n(G\mathscr{U}(\vartheta,X))
\cong\pi_n\left(\mathop{\textstyle{\bigvee}}_{\GSet(\vartheta,\psi)} S^n\right)= \bigoplus_{\GSet(\vartheta,\psi)}\pi_n(S^n)=
\bigoplus_{\GSet(\vartheta,\psi)}\mathbb{Z}.\]
We could also write: %and so \rednote{(go from here)}
\[\underline{\pi}_n(X)_{\vartheta}\cong\bigoplus_{i\in I}\bigoplus_{\GSet(\vartheta,\psi_i)}\mathbb{Z}\]
By results of the previous subsection we may view $M$ and $\underline{\pi}_n(X)$ as functors
\[(\Orb_G)^{\textup{op}}\to\AbGp,\]
and we have just demonstrated that $\underline{\pi}_n(X)$ is the composite of the functors
\[\coprod_{i\in I}\GSet(G/(\DASH),\psi_i):(\Orb_G)^{\textup{op}}\to\Set\text{ \ and \ }\mathbb{Z}\{\DASH\}:\Set\to\AbGp.\]

We argue that after a suitable choice of $\psi$ we can find a natural transformation
\[\mathbb{Z}\left\{\coprod_{i\in I}\GSet(G/(\DASH),\psi_i)\right\}\implies M\]
which is termwise surjective. Using the adjunction $\mathbb{Z}\{\DASH\}:\Set\to\AbGp:U$, where $U$ is the forgetful functor, this is equivalent to defining a natural transformation
\[\coprod_{i\in I}\GSet(G/(\DASH),\psi_i)\implies UM\]
which maps to a set of generators termwise. By the universal property of coproducts this is equivalent to defining a natural transformation
\[\GSet(G/(\DASH),\psi_i)\implies UM\]
for each $i\in I$ and by Yoneda's lemma, this is equivalent to specifying an element of $UM_{\psi_i}=M_{\psi_i}$ for each $i\in I$; this element determines where $\id_{\psi_i}$ is mapped to. We now see that by choosing generators of each $M_{G/H}$ and taking a copy of $G/H$ for each gives a suitable $\psi$.

Applying the same argument to the kernel we can construct a termwise exact sequence of natural transformations
\[\mathbb{Z}\left\{\coprod_{i\in I}\GSet(G/(\DASH),\psi_i)\right\}
\overset{\alpha}{\implies}
\mathbb{Z}\{\GSet(G/(\DASH),\psi')\}\implies
M\implies 0\]
where $\psi'=\coprod_{i\in I}\psi'_i$ is a $G$-set and each $\psi'_i$ is of the form $G/H$.

Let $X'=\psi'_+\wedge S^n=\bigvee_{\psi'} S^n$ and consider the commuting diagram
\[\xymatrix@C=2.4cm{
\underline{\pi}_n(X)_{\vartheta}\ar@{.>}[r]\ar[d]^{\cong}&
\underline{\pi}_n(X')_{\vartheta}\ar[d]^{\cong}\\
\mathbb{Z}\{\coprod_{i\in I}\GSet(\vartheta,\psi_i)\}=
{\displaystyle\bigoplus_{\GSet(\vartheta,\psi)}}\mathbb{Z}\ar[r]^-{\alpha_{\vartheta}} &{\displaystyle\bigoplus_{\GSet(\vartheta,\psi')}}\mathbb{Z}}\]
We would like to construct an equivariant map $f:X\to X'$ such that the top horizontal map is induced by $f$. By the argument above the bottom horizontal map is determined by where it maps each $\id_{\psi_i}$, i.e. it is determined by an element of
\[\prod_{i\in I}\bigoplus_{\GSet(\psi_i,\psi')}\mathbb{Z}.\]
Now
\[[X,X']^G=[\psi_+\wedge S^n,X']^G\cong\pi_n(G\mathscr{U}(\psi,X'))\cong
\prod_{i\in I}\pi_n(G\mathscr{U}(\psi_i,X'))
\cong\prod_{i\in I}\bigoplus_{\GSet(\psi_i,\psi')}\mathbb{Z}\]
We take an $f\in[X,X']^G$ corresponding to the element determining the natural transformation. There is a big commuting diagram lurking somewhere, which tells us that this $f$ induces the top horizontal map.

Now define $Y$ to be $X'\cup_f CX$. One then checks that $Y^H=X'^H\cup_{f^H} CX^H$ and standard homotopy theory shows $\pi_i(Y^H)=0$ for $i<n$ and that
\[\pi_n(X^H)\overset{f^H_*}{\to}\pi_n(X'^H)\overset{j^H_*}{\to}\pi_n(Y^H)\to 0\]
is exact. Hence $\underline{\pi}_n(Y)=M$. We can kill of the higher homotopy groups in a similar manner to finish constructing the Eilenberg-MacLane space.

At this point in the argument I realised a lot of the machinery used above is probably unnecessary and I think there is an easier, more explicit way to proceed. Hopefully, my hard work here will prove valuable once we stabilise.

As usual we have $\Omega K(M,n)\simeq K(M,n-1)$, equivariantly.

One should believe that we can construct Postnikov towers in $G$-equivariant homotopy theory in an identical way to which we do in ordinary homotopy theory.







\subsection*{Cohomology}
Everything that follows will be in a reduced world. For a based $G$-space $X$ we define
\[H^n_G(X;M)=[X,K(M,n)]^G.\]
One checks that for an equivariant map $i:A\to X$,
\[A\overset{i}{\to}X\overset{j}{\to}X\cup_i CA\]
is a $G$-cofibre sequence. Because $\Omega K(M,n)\simeq K(M,n-1)$, we have a suspension isomorphism and for any equivariant map $i:A\to X$ we have a long exact sequence
\[\ldots\to H^n_G(X)\overset{i^*}{\to}H^n_G(A)\overset{\delta}{\to}
H^{n+1}_G(X\cup_i CA)\overset{j^*}{\to} H^{n+1}_G(X)\to\ldots\]
In the usual way one can construct an unreduced cohomology theory and get an exact sequence of pairs and one can use this to calculate the cohomology of CW-complexes. We simply note that if $X$ is a based $G$-CW-complex and $X^{(n)}$ denotes the $n$-skeleton. We have
\[\tilde{C}^n_{\textup{cell}}(X;M)=[X^{(n)}/X^{(n-1)},K(M,n)]^G\]
$X^{(n)}/X^{(n-1)}=\vartheta_+\wedge S^n$ for some $G$-set. So
\[\tilde{C}^n_{\textup{cell}}(X;M)=[\vartheta_+\wedge S^n,K(M,n)]^G=\pi_n(K(M,n))_{\vartheta}=M_{\vartheta}.\]
\begin{exmp*}
Define
\[M:(\GSet)^{\textup{op}}\to\AbGp,
\hspace{4pt}
\vartheta\mapsto \GSet(\vartheta,\mathbb{Z})\]
where $\mathbb{Z}$ has the trivial $G$-action. Then $M_{G/H}=
\GSet(G/H,\mathbb{Z})=\mathbb{Z}$.

Let $K(\mathbb{Z},n)$ be a the usual Eilenberg-MacLane space equipped with a trivial $G$-action. Then for $i\neq n$
\[\underline{\pi}_i(K(\mathbb{Z},n))_{G/H}\cong
\pi_i(K(\mathbb{Z},n)^H)=\pi_i(K(\mathbb{Z},n))=0\]
and
\[\underline{\pi}_n(K(\mathbb{Z},n))_{G/H}\cong
\pi_n(K(\mathbb{Z},n))=\mathbb{Z}=M_{G/H}\]
so $K(\mathbb{Z},n)=K(M,n)$. When $X$ is a $G$-space
\[H^n_G(X;M)=[X;K(M,n)]^G=[X;K(\mathbb{Z},n)]^G
\cong[X/G,K(\mathbb{Z},n)]=H^n(X/G;\mathbb{Z}).\]
\end{exmp*}
\begin{exmp*}
When $X$ is a $G$-space
\[H^n(X;\mathbb{Z})=[X,K(\mathbb{Z},n)]
\cong[X,\mathscr{T}(G_+,K(\mathbb{Z},n))]^G=
[X,\mathscr{U}(G,K(\mathbb{Z},n))]^G.\]
Now
\[\mathscr{U}(G,K(\mathbb{Z},n))^H=H\mathscr{U}(G,K(\mathbb{Z},n))
\cong\mathscr{U}(H\backslash G,K(\mathbb{Z},n))\]
and mapping from a discrete set is just taking a product so
\[\underline{\pi}_n(\mathscr{U}(G,K(\mathbb{Z},n)))_{G/H}\cong
\pi_n(\mathscr{U}(G,K(\mathbb{Z},n))^H)\cong
\Set(H\backslash G,\pi_n(K(\mathbb{Z},n)))\cong\Set(G/H,\mathbb{Z})\]
and we see $H^n(X;\mathbb{Z})\cong H^n_G(X;M)$ where
\[M:(\GSet)^{\textup{op}}\to\AbGp,
\hspace{4pt}
\vartheta\mapsto\Set(\vartheta,\mathbb{Z}).\]
\end{exmp*}






\subsection*{Other approaches to Eilenberg-MacLane spaces}
Let $M:(\Orb_G)^{\textup{op}}\to\AbGp$ be a functor. Let
\[\{x^i_H:i\in I_H\}\]
be a set of generators for $M_H$.

Let $X=\bigvee_{H\leq G}\bigvee_{i\in I_H}G/H_+\wedge S^n$. It is straightforward to construct a surjection
\[\pi_n(X^H)\to M_H.\]
It is harder, but possible to choose surjections, which are compatible. Let $M'_H\leq\pi_n(X^H)$ be the kernel of each surjection. Construct $X'$ in the same way. This gives an equivariant map $X'\to X$ and we take the cofibre.

The latter approach was an attempt to be more concrete. The following approach is more abstract.

We have a functor
\[R:G\mathscr{T}\to\Cat
((\GSet)^{\textup{op}},\mathscr{T}),
\hspace{4pt}
X\mapsto(\vartheta\mapsto G\mathscr{U}(\vartheta,X)).\]
$\pi_n$ induces a functor
\[\Pi_n:\Cat
((\GSet)^{\textup{op}},\mathscr{T})\to
\Cat
((\GSet)^{\textup{op}},\AbGp)\]
and $\Pi_n\circ R=\underline{\pi}_n$.

$R$ is right adjoint to a functor $L$ and together these functors give a Quillen equivalence. We have a natural construction of noneqivariant Eilenberg MacLane space. Usually this involves simplicial sets. In particular, we have a functor
\[K(\DASH,n):\AbGp\to\mathscr{T}\]
such that composing with $\pi_n:\mathscr{T}\to\AbGp$ we get the identity functor. Thus we have an induced functor
\[\tilde{K}(\DASH,n):\Cat((\GSet)^{\textup{op}},\AbGp)\to
\Cat((\GSet)^{\textup{op}},\mathscr{T})\]
such that $\Pi_n\circ\tilde{K}(\DASH,n)$ is the identity functor.
\[\xymatrixcolsep{5pc}\xymatrix{
\Cat
((\GSet)^{\textup{op}},\AbGp)
\ar[r]^{\tilde{K}(\DASH,n)}\ar[rd]_{K(\DASH,n)}
&\Cat
((\GSet)^{\textup{op}},\mathscr{T})\ar[r]^{\Pi_n}\ar@<-1ex>[d]_L
&\Cat
((\GSet)^{\textup{op}},\AbGp)\\
&G\mathscr{T}\ar@<-1ex>[u]_R\ar[ru]_{\underline{\pi}_n}&}\]
$L\circ\tilde{K}(\DASH,n)$ gives the required functor
\[K(\DASH,n):\Cat((\GSet)^{\textup{op}},\AbGp)\to
G\mathscr{T}.\]
\subsection*{Suspension theorems}
Classically, under certain conditions on $\dim X$ and $\conn Y$, the suspension induces an isomorphism $[X,Y]\to[S^n\wedge X,S^n\wedge Y]$. Actually, the real question here is ``how connected is $Y\to \Omega^nS^n\wedge Y$'', and the answer is ``$(2k-1)$-connected when $Y$ is $(k-1)$-connected''.

The equivariant analogue would be that
\[[X,Y]^G\to [S^V\wedge X,S^V\wedge Y]^G\]
is an isomorphism, where $V$ is some representation of $G$.
\begin{exmp*}
Suppose $G=\Z_2$ and $V=\R$ (the trivial representation). Then consider $Y\to\Omega S^1\wedge Y$. When $Y$ is $(k-1)$-connected, this is $(2k-1)$-connected.
Now $(\Omega S^1\wedge Y)^H=\Omega S^1\wedge Y^H$, and when if $Y^H$ is $(k-1)$-connected, the map on $H$ invariants is $(2k-1)$-connected.
\end{exmp*}
\begin{exmp*}
Suppose $G=\Z_2$ and $V=\sigma$ (the sign representation). Consider $Y\to \Omega^\sigma S^\sigma\wedge Y$ (here, the $\Omega^\sigma$ stands for ``equivariant maps from $S^\sigma$'').
\end{exmp*}
\NewLecture{}{Monday 12}{9} Not yet.
\NewLecture{}{Wednesday 14}{9}
The question for this lecture is: ``what is the stable $0$-stem''?
\subsection*{Degree $S$ maps, for $S$ a $G$-set}
To obtain a degree $d$ map in standard homotopy theory, we choose a subset $S\subset S^n$ of size $d$, and choose a small ball around each point. Let $Bs$ be the union of the interiors of the $d$ small balls. Then a degree $d$ map is given by the composite:
\[S^n\to S^n/(S^n-Bs)=\textstyle{\bigvee}_{\!s\in S}S^n\to S^n.\]
For an equivariant analogue, suppose that $S$ is a finite $G$-set. Choose an equivariant embedding $S\subset S^V$ for some $G$-representation $V$. Now choose small balls around each point compatibly under the action of $G$, and use the same composite: \bluenote{(It's weird here to have $S^V$ after the =.)}
\[S^V\to S^V/(S^V-Bs)=\textstyle{\bigvee}_{\!s\in S}S^V\to S^V.\]
\begin{exmp*}
$G=\Z_2$, $S=\Z_2$. Let $V$ be the sign representation, so that $S^V$ is the circle with $\Z_2$ acting by reflection. Embed $S$ in $S^V$ as a pair of opposite points.
\end{exmp*}
\subsection*{The transfer map associated with a map $S\to T$ of $G$-sets}
Suppose that $p:S\to T$. We'll construct a \emph{stable} map $p^!:T_+\to S_+$.  To this end, we embed $S$ in $V$ for some unitary $G$-representation $V$ (and call the embedding $e:S\to V$). Now choose some $\epsilon>0$ such that $\|e(s)-e(s')\|\geq 2\epsilon$ for all $s\neq s'$, and let $\calD(V)$ be the open ball of radius $\epsilon$ around the origin in $V$. Then, we have a diagram:
\[\xymatrix{
S\ar[d]^i\ar[r]^{p\times e}&T\times V\\
S\times \calD(V)\ar[ur]_{\widetilde p}
}\]
where we define $i(s)=(s,0)$, and $\widetilde p(s,v)=(p(s),e(s)+v)$. Then $\widetilde p$ is an equivariant open inclusion, and as such, induces an equivariant map $\widetilde p^!:(T\times V)_+\to (S\times \calD(V))_+$ on one point compactifications. We observe that this is the required map:
\[p^!:T_+\wedge S^V\to S_+\wedge S^V\]
\begin{prop*}
A pullback (left) in $\FinGSet$ induces a commuting diagram (right) in $\GSW$:
\[\xymatrix@!0@R=12mm@C=1.5cm{
\ar[d]_qT'\ar[r]^j&S'\ar[d]^p\\
T\ar[r]^i&S
}\qquad\qquad\xymatrix@!0@R=12mm@C=1.5cm{
\ar[d];[]^{q^!}T'_+\ar[r]^j&S'_+\ar[d];[]_{p^!}\\
T_+\ar[r]^i&S_+
}\]
\end{prop*}
\subsection*{The Burnside category $\Burnside_G$ of $G$}
Let the objects of $\Burnside_G$ be finite $G$-sets, and let the maps $\Burnside_G(S_0,S_1)$ be the abelian group generated by equivalence classes of equivariant roofs (a.k.a.\ equivariant correspondences):
\[\xymatrix@R=5mm{&T\ar[dl]\ar[dr]\\S_0&&S_1}\]
where roofs are equivalent if there is an equivariant commuting diagram:
\[\xymatrix@R=5mm{&T\ar[dl]\ar[drr]\ar[r]^\simeq&T'\ar[dll]\ar[dr]
\\S_0&&&S_1}\]
and we impose the relation:
\[
\left[\raisebox{.5cm}[.8cm][0cm] {$\xymatrix@R=5mm{&T\ar[dl]\ar[dr]\\S_0&&S_1}$}\right]
+
\left[\raisebox{.5cm}[.8cm][0cm] {$\xymatrix@R=5mm{&T'\ar[dl]\ar[dr]\\S_0&&S_1}$}\right]
=\left[\raisebox{.5cm}[.8cm][0cm] {$\xymatrix@R=5mm{&T\sqcup T'\ar[dl]\ar[dr]\\S_0&&S_1}$}\right]
%=\left[\raisebox{.45cm}[.8cm][0cm] {$\xymatrix@R=5mm{&S\sqcup S''\ar[dl]\ar[dr]\\S&&T}$}\right]
\]
We compose morphisms via taking the evident pullback:
\[\xymatrix@R=5mm{
&&U\ar[dl]\ar[dr]\\
&T_{01}\ar[dl]\ar[dr]&&T_{12}\ar[dl]\ar[dr]\\
S_0&&S_1&&S_2
}\]
Now there is a functor $\FinGSet\to\Burnside_G$, which is the identity on objects, and on morphisms, takes $f:S\to T$ to the roof:
\[\xymatrix@R=5mm{&S\ar@{=}[dl]\ar[dr]^f\\S&&T}\]
\begin{prop*}
The transfer map gives a functor $\Burnside_G\to\GSW$ extending the functor $\FinGSet\to\GSW$ given by $S\mapsto S_+$. This functor sends a roof $S_0\overset{p}{\longleftarrow} T\overset{r}{\longrightarrow}S_1$ to the composite $r_* p^!$.
\end{prop*}
\begin{proof}
We only need to check that the maps on morphisms are compatible with composition, but that this exactly the previous proposition.
\end{proof}
\begin{thm*}
The functor $\Burnside_G\to \GSW$ is fully faithful: $\Burnside_G(S,T)\overset{\simeq}{\to}\{S_+,T_+\}^G$.
\end{thm*}
\begin{exmp*}
Suppose $S=T=*$. Then:
\begin{align*}
\Burnside_G(*,*)&=\Z\{\text{finite $G$-sets}\}/({\displaystyle\sqcup\leftrightarrow\text{sum}})\\
&=A(G),\text{ the Burnside ring of $G$.}
\end{align*}
\end{exmp*}
\begin{cor*}
The stable $0$-stem, $\{S^0,S^0\}^G$, equals $A(G)$.
\end{cor*}
Now $A(G)$ is the free abelian group on generators $G/H$, where $H$ runs through conjugacy classes of subgroups of $H$. The multiplication in $A(G)$ is given by cartesian product of $G$-sets. For each conjugacy class of subgroups $H$, there is a ring homomorphism $A(G)\to\Z$ given by $S\mapsto \# S^H$. This induces a ring homomorphism
\[A(G)\to\prod_\text{conj classes of $H$}\Z.\]
This map is injective, and is an isomorphism after inverting $|G|$.
\begin{exmp*}
Suppose $G=\Z_2$. Then $A(G)=\Z\langle\bullet\rangle\oplus \Z\langle\bullet\leftrightarrow\bullet\rangle$. We have a ring homomorphism $A(G)\to\Z\oplus\Z$ given by $S\mapsto(\#S^{\Z_2},\#S)$. In general, all we know about these two numbers is that their difference must be even.
\end{exmp*}
\subsection*{Spanier-Whitehead Duality}
\begin{defn*}
A duality between objects $X$ and $Y$ in $\GSW$ is to be a pair of morphisms $X\wedge Y\to S^0$ and $S^0\to X\wedge Y$ such that the following diagrams commute (of course up to stable homotopy):
\[\xymatrix{
S^0\wedge X\ar[d]\ar[dr]^\simeq\\
X\wedge Y\wedge X\ar[r]&X\wedge S^0
}
\qquad\xymatrix{
Y\wedge S^0\ar[d]\ar[dr]^\simeq\\
Y\wedge X\wedge Y\ar[r]&S^0\wedge Y
}\]
\end{defn*}
Given a such duality, there is a natural bijection
$\{A\wedge X,B\}^G\to\{A,B\wedge Y\}^G$ given by
\[a\mapsto\left\{A=A\wedge S^0\to A\wedge X\wedge Y\overset{a\wedge1}{\to}B\wedge Y\right\}.\]
Now it is in fact the case that every object in $\Burnside_G$ is self-dual. As the functor above is a symmetric monoidal functor, this shows that the image in $\GSW$ of an element of $\Burnside_G$ is self dual. To understand the duality in $\Burnside_G$, consider the roofs:
\[\xymatrix@R=5mm{&S\ar[dl]_\Delta\ar[dr]^f\\S\times S&&\ast}
\qquad\raisebox{-.5cm}{\text{and}}\qquad
\xymatrix@R=5mm{&S\ar[dl]_f\ar[dr]^\Delta\\\ast&&S\times S}\]
The diagram which shows the self duality of $S$ in $\Burnside_G$ is then:
\[\xymatrix@R=5mm{
&&S\ar[ld]_\Delta\ar[rd]^\Delta\\
&S\times S\ar[ld]_{\ast\times1}\ar[rd]^{\Delta\times1}&&S\times S\ar[ld]_{1\times\Delta}\ar[rd]^{1\times\ast}\\
\ast\times S&&S\times S\times S&&S\times\ast
}\]
The self-duality of $S_+$ which this induces explains why $\Burnside_G(S,T)$ is symmetric in $S$ and $T$.

\rednote{(The proof of the main theorem was started, but we ran out of time).}
\NewLecture{}{Friday 16}{9} Not yet. In other notebook.
\NewLecture{}{Monday 19}{9} Recall that
% if $\GSW$ is the $G$-Spanier-Whitehead category,
we showed that the functor
\funcdef{\Burnside_G}{\GSW}{S}{S_+}
% $S\mapsto S_+$ from $\Burn_G$ to $\GSW$
is fully faithful. We also noted that the group $[S^n,X]^G$ is actually a value of a functor:
\funcdef{\underline{\pi}_n(X):(\FinGSet)^\op}{\AbGp}{S}{[S^n\wedge S_+,X]^G}
This functor is a ``coefficient system'' --- a contravariant functor $(\FinGSet)^\op\to\AbGp$ which sends coproducts to products.

Stabilising, we obtain Mackey functors:
\funcdef{\underline{\pi}_n^\st(X):(\Burnside_G)^\op}{\AbGp}{S}{\{S^n\wedge S_+,X\}^G}
A Mackey functor is just a contravariant additive functor $(\Burnside_G)^\op\to\AbGp$.

We showed that every coefficient system arises as a homotopy group --- we even formed a $K(A,n)$ for $A$ a coefficient system.
\subsection*{An alternative definition of Mackey Functors}
A Mackey functor $M$ may be described as a pair of functors $M_*:\FinGSet\to\AbGp$ and $M^*:(\FinGSet)^\op\to\AbGp$, which take the same values on objects:
\[M_*(S)=M^*(S)=M(S)\]
and such that for every pullback square (left) the induced diagram (right) commutes:
\[\xymatrix@!0@R=12mm@C=1.5cm{
\ar[d]_qT'\ar[r]^j&S'\ar[d]^p\\
T\ar[r]^i&S
}\qquad\qquad\xymatrix@!0@R=12mm@C=2.05cm{
\ar[d];[]^{q^*}M(T')\ar[r]^{j_*}&M(S')\ar[d];[]_{p^*}\\
M(T)\ar[r]^{i_*}&M(S)
}\]
\subsection*{Every Mackey functor is a stable homotopy group}
Our next goal is to show that every Mackey functor $M$ occurs as a stable homotopy group, and at the same time, to construct $K(M,n)$, and its delooping $K(M,n+V)$ for any representation $V$. By its delooping, I mean that $K(M,n)=\Omega^VK(M,n+V)$.

Let's begin. Fix a Mackey functor $M$. Choose a representation $W$ sufficiently large that $[S^W,S^W]^G\overset{\cong}{\to}\{S^0,S^0\}^G$ is an isomorphism. Nay, even better, choose $W$ so large that
\[[S^W,S^W]^G\overset{\cong}{\to} [S^W\wedge S^V,S^W\wedge S^V]^G\]
is an isomorphism for all $V$. For this $W$, we have
\[[S_+\wedge S^W,T_+\wedge S^W]^G\overset{\cong}{\to}\{S_+,T_+\}^G=\Burnside_G(S,T)\]
for all $S,T\in\Burnside_G$. \rednote{(why?)}
Now if one fixes $T$, and views $S$ as a variable, this shows that the stable homotopy group $\underline\pi_0^\st (T_+)$ is the Mackey functor represented by $T$. Thus any representable Mackey functor arises as a homotopy group.

Now we'll attack $M$. We can write $M$ as a cokernel of a map $P\to Q$, where $P$ and $Q$ are both sums of representable Mackey functors. We could produce such $P$ and $Q$ by the standard overkill.\footnote{
By `the standard overkill', one might mean:
\[Q=\bigoplus_{S\in \FinGSet}\bigoplus_{x\in M(S)}\Burnside_G(\DASH,S).\]
To produce a map $r:Q(T)\to M(T)$, we need a map $\Burnside_G(T,S)\to M(T)$ for all $S$ and $x\in M(S)$. A map $f\in \Burnside_G(T,S)$ induces $M(f):M(S)\to M(T)$, and $M(f)(x)\in M(T)$ is taken as $r(f)$.
We can then produce $P$ the same way, mapping onto the kernel.}

Now we saw that any (arbitrary) sum $P$ of representable Mackey functors occurs as $\underline\pi_0(T_+)$ for some (possibly infinite) discrete $G$-set $T$. So write $P=\underline\pi_0(T_+)$ and $Q=\underline\pi_0(U_+)$.

\rednote{By some Yoneda-ish argument or something}, we obtain a stable map $T_+\to S_+$. As $W$ is large enough, this stable map is realised after smashing with $S^W$, and we can form a cofibre sequence:
\[S^W\wedge T_+\to S^W\wedge U_+\to X_1\to \Sigma S^W\wedge T_+\]
This gives an exact sequence, which we view as a sequence of Mackey functors in the variable $S$:
\[[S^W\wedge S_+,S^W\wedge T_+]^G\to[S^W\wedge S_+,S^W\wedge U_+]^G\to [S^W\wedge S_+,X_1]^G\to [S^W\wedge S_+,\Sigma S^W\wedge T_+]^G\]
As this left morphism in this sequence is just $P\to Q$, the third Mackey functor will be $M$ if we can show that the fourth functor is zero. But we can see this \bluenote{(the reasoning here differs from what's in the lecture, which I felt was incomplete --- am I correct?)} by noting that for any $H\subset G$, the $H$-fixed points $(S^W\wedge S_+)^H=(S^{W^H}\wedge S^H_+)$ have the same dimension as $W^H$, while the $H$-fixed points $(\Sigma S^W\wedge T_+)^H=(\Sigma S^{W^H}\wedge T^H_+)$ are $(\dim W^H)$-connected.

Thus we have shown that $M=\underline\pi_0(S^{-W}\wedge X_1)$, so that $M$ is a stable homotopy group. \textbf{But wait!} $S^{-W}$ is not an object of the category $\GSW$! We're going to need a larger category containing these desuspensions. \rednote{(Is this the point here?)}

Instead, we have $M=[S^W\wedge S_+,X_1]$, so we could say that $M$ is $\underline\pi_W(X_1)$. In this sense, $M$ is a stable homotopy group.
\subsection*{Attempted construction of $K(M,n)$}
\rednote{\textbf{JANDR}: Here it gets colloquial --- I'm just summarising the point so you can check if I'm making sense.}

Now that we have $X_1$, we can form the cone on all maps of the form $S^1\wedge S^W\wedge T_+\to X_1$, to obtain a space $X_2$. We can do this over and over to get something $K(M,W)$-like. However, the category $\GSW$ obviously doesn't have negative spheres to desuspend it all down to (say) a $K(M,0)$, and it doesn't even have enough stuff to make the colimits (i.e.\ to cone all that stuff off)! That's the problem.
\subsection*{Problems with $\GSW$}
\begin{itemize}
\item If $f\in\{X,Y\}^G$, we can't form the cofibre $Y\cup CX$! We can form a cofibre sequence
\[S^W\wedge X\to S^W\wedge Y\to \text{cone}\]
but the third term is $S^W\wedge Y\cup CX$. Everything would be ok if we had an object $S^{-W}$, and could define $Y\cup CX=S^{-W}\wedge \text{cofib}(S^W\wedge X\to S^W\wedge Y)$.

Note that we could add objects $S^{-W}\wedge X$ to the category simply by specifying what it means to give a map into them. Of course the right specification is clear.
\item We don't have colimits. We need colimits of sequences like:
\[X_0\to S^{W_1}\wedge X_1\to S^{W_2}\wedge X_2\to\cdot.\]
If we formally added an object $X$ as the colimit of this sequence, as $X$ is supposed to be the colimit, we already know that we should be defining
\[\{X,Y\}^G=\varprojlim\{S^{W_j}\wedge X_j,Y\}^G.\]
Moreover, we could explain how to map in by specifying that mapping in commutes with the colimit, i.e.:
\[\{Y,X\}^G=\varinjlim\{Y,S^{W_j}\wedge X_j\}^G.\]
\end{itemize}
The homotopy category of $G$-CW-spectra is the category obtained by adding these objects --- negative spheres and colimits.
\NewLecture{}{Friday 23}{9}
Today we'll outline two approaches to constructing equivariant spectra.
\subsection*{Approach 1.}
Fix a $G$-universe $\scrU$ --- a countably infinite dimensional $G$-inner-product space containing each finite dimensional representation of $G$ with infinite multiplicity (this remedy for the set-theoretical difficulties is due to J.P.\ May). For example, $\scrU$ could be the inner product space sum of countably many copies of the regular representation.

\begin{defn*}
A $G$-spectrum indexed over $U$ is then to be a collection of spaces 
\[\{X_V\,|\,V\subset\calU\text{ is finite dimensional and $G$-invariant}\}\]
along with (unstable) maps $S^{W-V}\wedge S_V\to X_W$ whenever $V\subseteq W$ (where $W-V=W\cup V^\perp$), which are associative, in the sense that the following diagram commutes:
\[\xymatrix{
\ar[d]S^U\wedge S^V\wedge X_W\ar[r]&S^U\wedge X_{V\oplus W}\ar[d]\\
S^{U\oplus V}\wedge X_W\ar[r]&X_{U\oplus V\oplus W}
}\]
\end{defn*}
We should understand this by analogy with Adams' spectra. There, a spectrum $X$ is a sequence of spaces $X_n$ with maps $\Sigma X_n\to X_{n+1}$. Then $X$ is equal to the direct limit $S^{-n}\wedge X_n$.
%imagine here that $X_V$ is actually a desuspension by $S^V$ of

Note that in general an arbitrary directed system in $\GSW$ includes some stable maps --- maps which only appear after some suspensions. So it would seem more natural to allow the structure maps to be stable maps, not just unstable as we have stipulated. However, it seems that it suffices to look at the more specialised systems where all of the maps are unstable:
\subsection*{Approach 2. Orthogonal $G$-spectra}
Suppose that $V$ and $W$ are finite dimensional representations of $G$, each equipped with nondegenerate invariant positive definite inner products. Define $\calO(V,W)$ to the the Stiefel manifold of (non-equivariant) linear isometries $V\to W$. Now $G$ acts on $\calO(V,W)$ via $(gf)(v)=g(f(g^{-1}v))$, so that the fixed points are the equivariant isometric embeddings $V\to W$.

Write $\underline{W-V}$ for the vector bundle over $\calO(V,W)$ whose fibre over $f:V\to W$ is $W-f(V)$. Finally, define:
\begin{alignat*}{2}
I_G(V,W):=&\textup{\,Thom}(\calO(V,W),\underline{V-W})\\
=&(\underline{V-W})_+,\text{ \ (the one point compactification, as $\calO(V,W)$ is compact)}\\
=&\text{\,an amalgamation of the spheres $S^{W-f(V)}$}
\end{alignat*}
This is a $G$-space, because everything in sight has a $G$-action.
\begin{defn*}
An orthogonal $G$-spectrum $X$ is a collection of $G$-spaces $X_V$, one for each orthogonal $G$-representation, with $G$-equivariant maps $I_G(V,W)\wedge X_V\to X_W$, associative in the sense that the following diagram commutes:
\[\xymatrix{
\ar[d]I_G(V,W)\wedge I_G(U,V)\wedge X_U\ar[r]&I_G(V,W)\wedge X_{V}\ar[d]\\
I_G(U,W)\wedge X_U\ar[r]&X_{W}
}\]
The map in the left column is described as follows. Composition gives a map $c:\calO(V,W)\times\calO(U,V)\to\calO(U,W)$. Moreover, $c^*(\underline{W-U})=\underline{W-V}\times\underline{V-U}$ (as all the maps involved are linear isometries, and thus injective). The induced map on Thom complexes gives the left vertical.
\end{defn*}

Let $\scrT_G$ be the category whose objects are pointed $G$-spaces, and whose morphisms are pointed (possibly) non-equivariant maps. This category is enriched over $\scrT^G$, the category of $G$-spaces with equivariant maps.

Now a $G$-equivariant map $I_G(V,W)\wedge X_V\to X_W$ corresponds to a $G$-equivariant map $I_G(V,W)\to \scrT_G(X_V,X_W)$. That is, for each isometry $f:V\to W$, and each $x\in W-f(V)$, we give a (non-equivariant) map $X_V\to X_W$, such that as $\|x\|$ increases, the map $X_V\to X_W$ tends to the constant map, and we do all this equivariantly. \rednote{(I don't really get the point of the `$x$' in this.)}

Thus an orthogonal $G$-spectrum is the same as a functor between categories enriched over $\scrT_G$:
\[I_G\to\scrT_G\]
where $I_G$ is the category whose objects are finite dimensional $G$-representations, whose morphisms are given by the Thom complex constructed above, and whose composition is given by the smash product map constructed. \rednote{I haven't thought long about this, but assume it's what he's getting at.}
\subsection*{The obvious notion of a map}
In either model, there is an obvious notion of map. That is, we give a map $f_V:X_V\to Y_V$ for each $V$, and demand the obvious compatibilities. This approach is incorrect. We'll work with the first approach in order to give examples.
\begin{exmp*}
For any pointed $G$-space $T$, we can form a $G$-spectrum $\Sigma^\infty T$ with $(\Sigma^\infty T)_V=S^V\wedge T$. This process is supposed to give an embedding $\GSW\to\GSpectra$. To see this heuristically, note that $\Sigma^\infty T=\colim S^{-V}\wedge S^V\wedge T=\colim T$, the colimit of the constant diagram $T$. 

However, to define a spectrum map $\Sigma^\infty S^0\to \Sigma^\infty S^0$, we require a map $f_0:S^0\to S^0$, and this determines everything else via the compatibility with the structure map $S^V\wedge(\Sigma^\infty S^0)_0\to(\Sigma^\infty S^0)_V$ (drawn vertically):
\[\xymatrix@R=.6cm{ 
\ar[d]^\simeq S^V\wedge S^0\ar[r]^{\id\wedge f_0}&S^V\wedge S^0\ar[d]^\simeq\\
S^V\wedge S^0\ar[r]^{f_V}&S^V\wedge S^0
}\]
Thus $\GSpectra(S^0,S^0)=\scrT^G(S^0,S^0)=\{\id\}$, which is a long way from $\Burnside_G(*,*)=A(G)$.
\end{exmp*}
Fix now a representation $U$ of $G$. Define a $G$-spectrum $(S)$ by stipulating
\[\begin{cases} *,&U\not\subset W,\\ S^W,&U\subset W,\end{cases}\]
with the obvious structure maps. This is suppose to correspond to $S^0$ as well --- we have just truncated the beginning of the colimit. But $\GSpectra((S),S^0)$ is now the same as $\scrT^G(S^U,S^U)$, so for $U\gg0$, this will calculate $\{S^0,S^0\}^G$.

The message to take from this is that we should introduce a class of weak equivalences, so that the homotopy theory will not see the difference between $\Sigma^\infty S^0$ and $(S)$. Once we've given the correct model structure, the homotopy category will be correct.
\end{document}
















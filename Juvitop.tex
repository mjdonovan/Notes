% !TEX root = z_output/_Juvitop.tex
%%%%%%%%%%%%%%%%%%%%%%%%%%%%%%%%%%%%%%%%%%%%%%%%%%%%%%%%%%%%%%%%%%%%%%%%%%%%%%%%
%%%%%%%%%%%%%%%%%%%%%%%%%%% 80 characters %%%%%%%%%%%%%%%%%%%%%%%%%%%%%%%%%%%%%%
%%%%%%%%%%%%%%%%%%%%%%%%%%%%%%%%%%%%%%%%%%%%%%%%%%%%%%%%%%%%%%%%%%%%%%%%%%%%%%%%
\documentclass[11pt]{article}
\usepackage{fullpage}
\usepackage{amsmath,amsthm,amssymb}
\usepackage{mathrsfs,nicefrac}
\usepackage{amssymb}
\usepackage{epsfig}
\usepackage[all,2cell]{xy}
\usepackage{sseq}
\usepackage{tocloft}
\usepackage{cancel}
\usepackage[strict]{changepage}
\usepackage{color}
\usepackage{tikz}
\usepackage{extpfeil}
\usepackage{version}
\usepackage{framed}
\definecolor{shadecolor}{rgb}{.925,0.925,0.925}

%\usepackage{ifthen}
%Used for disabling hyperref
\ifx\dontloadhyperref\undefined
%\usepackage[pdftex,pdfborder={0 0 0 [1 1]}]{hyperref}
\usepackage[pdftex,pdfborder={0 0 .5 [1 1]}]{hyperref}
\else
\providecommand{\texorpdfstring}[2]{#1}
\fi
%>>>>>>>>>>>>>>>>>>>>>>>>>>>>>>
%<<<        Versions        <<<
%>>>>>>>>>>>>>>>>>>>>>>>>>>>>>>
%Add in the following line to include all the versions.
%\def\excludeversion#1{\includeversion{#1}}

%>>>>>>>>>>>>>>>>>>>>>>>>>>>>>>
%<<<       Better ToC       <<<
%>>>>>>>>>>>>>>>>>>>>>>>>>>>>>>
\setlength{\cftbeforesecskip}{0.5ex}

%>>>>>>>>>>>>>>>>>>>>>>>>>>>>>>
%<<<      Hyperref mod      <<<
%>>>>>>>>>>>>>>>>>>>>>>>>>>>>>>

%needs more testing
\newcounter{dummyforrefstepcounter}
\newcommand{\labelRIGHTHERE}[1]
{\refstepcounter{dummyforrefstepcounter}\label{#1}}


%>>>>>>>>>>>>>>>>>>>>>>>>>>>>>>
%<<<  Theorem Environments  <<<
%>>>>>>>>>>>>>>>>>>>>>>>>>>>>>>
\ifx\dontloaddefinitionsoftheoremenvironments\undefined
\theoremstyle{plain}
\newtheorem{thm}{Theorem}[section]
\newtheorem*{thm*}{Theorem}
\newtheorem{lem}[thm]{Lemma}
\newtheorem*{lem*}{Lemma}
\newtheorem{prop}[thm]{Proposition}
\newtheorem*{prop*}{Proposition}
\newtheorem{cor}[thm]{Corollary}
\newtheorem*{cor*}{Corollary}
\newtheorem{defprop}[thm]{Definition-Proposition}
\newtheorem*{punchline}{Punchline}
\newtheorem*{conjecture}{Conjecture}
\newtheorem*{claim}{Claim}

\theoremstyle{definition}
\newtheorem{defn}{Definition}[section]
\newtheorem*{defn*}{Definition}
\newtheorem{exmp}{Example}[section]
\newtheorem*{exmp*}{Example}
\newtheorem*{exmps*}{Examples}
\newtheorem*{nonexmp*}{Non-example}
\newtheorem{asspt}{Assumption}[section]
\newtheorem{notation}{Notation}[section]
\newtheorem{exercise}{Exercise}[section]
\newtheorem*{fact*}{Fact}
\newtheorem*{rmk*}{Remark}
\newtheorem{fact}{Fact}
\newtheorem*{aside}{Aside}
\newtheorem*{question}{Question}
\newtheorem*{answer}{Answer}

\else\relax\fi

%>>>>>>>>>>>>>>>>>>>>>>>>>>>>>>
%<<<      Fields, etc.      <<<
%>>>>>>>>>>>>>>>>>>>>>>>>>>>>>>
\DeclareSymbolFont{AMSb}{U}{msb}{m}{n}
\DeclareMathSymbol{\N}{\mathbin}{AMSb}{"4E}
\DeclareMathSymbol{\Octonions}{\mathbin}{AMSb}{"4F}
\DeclareMathSymbol{\Z}{\mathbin}{AMSb}{"5A}
\DeclareMathSymbol{\R}{\mathbin}{AMSb}{"52}
\DeclareMathSymbol{\Q}{\mathbin}{AMSb}{"51}
\DeclareMathSymbol{\PP}{\mathbin}{AMSb}{"50}
\DeclareMathSymbol{\I}{\mathbin}{AMSb}{"49}
\DeclareMathSymbol{\C}{\mathbin}{AMSb}{"43}
\DeclareMathSymbol{\A}{\mathbin}{AMSb}{"41}
\DeclareMathSymbol{\F}{\mathbin}{AMSb}{"46}
\DeclareMathSymbol{\G}{\mathbin}{AMSb}{"47}
\DeclareMathSymbol{\Quaternions}{\mathbin}{AMSb}{"48}


%>>>>>>>>>>>>>>>>>>>>>>>>>>>>>>
%<<<       Operators        <<<
%>>>>>>>>>>>>>>>>>>>>>>>>>>>>>>
\DeclareMathOperator{\ad}{\textbf{ad}}
\DeclareMathOperator{\coker}{coker}
\renewcommand{\ker}{\textup{ker}\,}
\DeclareMathOperator{\End}{End}
\DeclareMathOperator{\Aut}{Aut}
\DeclareMathOperator{\Hom}{Hom}
\DeclareMathOperator{\Maps}{Maps}
\DeclareMathOperator{\Mor}{Mor}
\DeclareMathOperator{\Gal}{Gal}
\DeclareMathOperator{\Ext}{Ext}
\DeclareMathOperator{\Tor}{Tor}
\DeclareMathOperator{\Map}{Map}
\DeclareMathOperator{\Der}{Der}
\DeclareMathOperator{\Rad}{Rad}
\DeclareMathOperator{\rank}{rank}
\DeclareMathOperator{\ArfInvariant}{Arf}
\DeclareMathOperator{\KervaireInvariant}{Ker}
\DeclareMathOperator{\im}{im}
\DeclareMathOperator{\coim}{coim}
\DeclareMathOperator{\trace}{tr}
\DeclareMathOperator{\supp}{supp}
\DeclareMathOperator{\ann}{ann}
\DeclareMathOperator{\spec}{Spec}
\DeclareMathOperator{\SPEC}{\textbf{Spec}}
\DeclareMathOperator{\proj}{Proj}
\DeclareMathOperator{\PROJ}{\textbf{Proj}}
\DeclareMathOperator{\fiber}{F}
\DeclareMathOperator{\cofiber}{C}
\DeclareMathOperator{\cone}{cone}
\DeclareMathOperator{\skel}{sk}
\DeclareMathOperator{\coskel}{cosk}
\DeclareMathOperator{\conn}{conn}
\DeclareMathOperator{\colim}{colim}
\DeclareMathOperator{\limit}{lim}
\DeclareMathOperator{\ch}{ch}
\DeclareMathOperator{\Vect}{Vect}
\DeclareMathOperator{\GrthGrp}{GrthGp}
\DeclareMathOperator{\Sym}{Sym}
\DeclareMathOperator{\Prob}{\mathbb{P}}
\DeclareMathOperator{\Exp}{\mathbb{E}}
\DeclareMathOperator{\GeomMean}{\mathbb{G}}
\DeclareMathOperator{\Var}{Var}
\DeclareMathOperator{\Cov}{Cov}
\DeclareMathOperator{\Sp}{Sp}
\DeclareMathOperator{\Seq}{Seq}
\DeclareMathOperator{\Cyl}{Cyl}
\DeclareMathOperator{\Ev}{Ev}
\DeclareMathOperator{\sh}{sh}
\DeclareMathOperator{\intHom}{\underline{Hom}}
\DeclareMathOperator{\Frac}{frac}



%>>>>>>>>>>>>>>>>>>>>>>>>>>>>>>
%<<<   Cohomology Theories  <<<
%>>>>>>>>>>>>>>>>>>>>>>>>>>>>>>
\DeclareMathOperator{\KR}{{K\R}}
\DeclareMathOperator{\KO}{{KO}}
\DeclareMathOperator{\K}{{K}}
\DeclareMathOperator{\OmegaO}{{\Omega_{\Octonions}}}

%>>>>>>>>>>>>>>>>>>>>>>>>>>>>>>
%<<<   Algebraic Geometry   <<<
%>>>>>>>>>>>>>>>>>>>>>>>>>>>>>>
\DeclareMathOperator{\Spec}{Spec}
\DeclareMathOperator{\Proj}{Proj}
\DeclareMathOperator{\Sing}{Sing}
\DeclareMathOperator{\shfHom}{\mathscr{H}\textit{\!\!om}}
\DeclareMathOperator{\WeilDivisors}{{Div}}
\DeclareMathOperator{\CartierDivisors}{{CaDiv}}
\DeclareMathOperator{\PrincipalWeilDivisors}{{PrDiv}}
\DeclareMathOperator{\LocallyPrincipalWeilDivisors}{{LPDiv}}
\DeclareMathOperator{\PrincipalCartierDivisors}{{PrCaDiv}}
\DeclareMathOperator{\DivisorClass}{{Cl}}
\DeclareMathOperator{\CartierClass}{{CaCl}}
\DeclareMathOperator{\Picard}{{Pic}}
\DeclareMathOperator{\Frob}{Frob}


%>>>>>>>>>>>>>>>>>>>>>>>>>>>>>>
%<<<  Mathematical Objects  <<<
%>>>>>>>>>>>>>>>>>>>>>>>>>>>>>>
\newcommand{\sll}{\mathfrak{sl}}
\newcommand{\gl}{\mathfrak{gl}}
\newcommand{\GL}{\mbox{GL}}
\newcommand{\PGL}{\mbox{PGL}}
\newcommand{\SL}{\mbox{SL}}
\newcommand{\Mat}{\mbox{Mat}}
\newcommand{\Gr}{\textup{Gr}}
\newcommand{\Squ}{\textup{Sq}}
\newcommand{\catSet}{\textit{Sets}}
\newcommand{\RP}{{\R\PP}}
\newcommand{\CP}{{\C\PP}}
\newcommand{\Steen}{\mathscr{A}}
\newcommand{\Orth}{\textup{\textbf{O}}}

%>>>>>>>>>>>>>>>>>>>>>>>>>>>>>>
%<<<  Mathematical Symbols  <<<
%>>>>>>>>>>>>>>>>>>>>>>>>>>>>>>
\newcommand{\DASH}{\textup{---}}
\newcommand{\op}{\textup{op}}
\newcommand{\CW}{\textup{CW}}
\newcommand{\ob}{\textup{ob}\,}
\newcommand{\ho}{\textup{ho}}
\newcommand{\st}{\textup{st}}
\newcommand{\id}{\textup{id}}
\newcommand{\Bullet}{\ensuremath{\bullet} }
\newcommand{\sprod}{\wedge}

%>>>>>>>>>>>>>>>>>>>>>>>>>>>>>>
%<<<      Some Arrows       <<<
%>>>>>>>>>>>>>>>>>>>>>>>>>>>>>>
\newcommand{\nt}{\Longrightarrow}
\let\shortmapsto\mapsto
\let\mapsto\longmapsto
\newcommand{\mapsfrom}{\,\reflectbox{$\mapsto$}\ }
\newcommand{\bigrightsquig}{\scalebox{2}{\ensuremath{\rightsquigarrow}}}
\newcommand{\bigleftsquig}{\reflectbox{\scalebox{2}{\ensuremath{\rightsquigarrow}}}}

%\newcommand{\cofibration}{\xhookrightarrow{\phantom{\ \,{\sim\!}\ \ }}}
%\newcommand{\fibration}{\xtwoheadrightarrow{\phantom{\sim\!}}}
%\newcommand{\acycliccofibration}{\xhookrightarrow{\ \,{\sim\!}\ \ }}
%\newcommand{\acyclicfibration}{\xtwoheadrightarrow{\sim\!}}
%\newcommand{\leftcofibration}{\xhookleftarrow{\phantom{\ \,{\sim\!}\ \ }}}
%\newcommand{\leftfibration}{\xtwoheadleftarrow{\phantom{\sim\!}}}
%\newcommand{\leftacycliccofibration}{\xhookleftarrow{\ \ {\sim\!}\,\ }}
%\newcommand{\leftacyclicfibration}{\xtwoheadleftarrow{\sim\!}}
%\newcommand{\weakequiv}{\xrightarrow{\ \,\sim\,\ }}
%\newcommand{\leftweakequiv}{\xleftarrow{\ \,\sim\,\ }}

\newcommand{\cofibration}
{\xhookrightarrow{\phantom{\ \,{\raisebox{-.3ex}[0ex][0ex]{\scriptsize$\sim$}\!}\ \ }}}
\newcommand{\fibration}
{\xtwoheadrightarrow{\phantom{\raisebox{-.3ex}[0ex][0ex]{\scriptsize$\sim$}\!}}}
\newcommand{\acycliccofibration}
{\xhookrightarrow{\ \,{\raisebox{-.55ex}[0ex][0ex]{\scriptsize$\sim$}\!}\ \ }}
\newcommand{\acyclicfibration}
{\xtwoheadrightarrow{\raisebox{-.6ex}[0ex][0ex]{\scriptsize$\sim$}\!}}
\newcommand{\leftcofibration}
{\xhookleftarrow{\phantom{\ \,{\raisebox{-.3ex}[0ex][0ex]{\scriptsize$\sim$}\!}\ \ }}}
\newcommand{\leftfibration}
{\xtwoheadleftarrow{\phantom{\raisebox{-.3ex}[0ex][0ex]{\scriptsize$\sim$}\!}}}
\newcommand{\leftacycliccofibration}
{\xhookleftarrow{\ \ {\raisebox{-.55ex}[0ex][0ex]{\scriptsize$\sim$}\!}\,\ }}
\newcommand{\leftacyclicfibration}
{\xtwoheadleftarrow{\raisebox{-.6ex}[0ex][0ex]{\scriptsize$\sim$}\!}}
\newcommand{\weakequiv}
{\xrightarrow{\ \,\raisebox{-.3ex}[0ex][0ex]{\scriptsize$\sim$}\,\ }}
\newcommand{\leftweakequiv}
{\xleftarrow{\ \,\raisebox{-.3ex}[0ex][0ex]{\scriptsize$\sim$}\,\ }}

%>>>>>>>>>>>>>>>>>>>>>>>>>>>>>>
%<<<    xymatrix Arrows     <<<
%>>>>>>>>>>>>>>>>>>>>>>>>>>>>>>
\newdir{ >}{{}*!/-5pt/@{>}}
\newcommand{\xycof}{\ar@{ >->}}
\newcommand{\xycofib}{\ar@{^{(}->}}
\newcommand{\xycofibdown}{\ar@{_{(}->}}
\newcommand{\xyfib}{\ar@{->>}}
\newcommand{\xymapsto}{\ar@{|->}}

%>>>>>>>>>>>>>>>>>>>>>>>>>>>>>>
%<<<     Greek Letters      <<<
%>>>>>>>>>>>>>>>>>>>>>>>>>>>>>>
%\newcommand{\oldphi}{\phi}
%\renewcommand{\phi}{\varphi}
\let\oldphi\phi
\let\phi\varphi
\renewcommand{\to}{\longrightarrow}
\newcommand{\from}{\longleftarrow}
\newcommand{\eps}{\varepsilon}

%>>>>>>>>>>>>>>>>>>>>>>>>>>>>>>
%<<<  1st-4th & parentheses <<<
%>>>>>>>>>>>>>>>>>>>>>>>>>>>>>>
\newcommand{\first}{^\text{st}}
\newcommand{\second}{^\text{nd}}
\newcommand{\third}{^\text{rd}}
\newcommand{\fourth}{^\text{th}}
\newcommand{\ZEROTH}{$0^\text{th}$ }
\newcommand{\FIRST}{$1^\text{st}$ }
\newcommand{\SECOND}{$2^\text{nd}$ }
\newcommand{\THIRD}{$3^\text{rd}$ }
\newcommand{\FOURTH}{$4^\text{th}$ }
\newcommand{\iTH}{$i^\text{th}$ }
\newcommand{\jTH}{$j^\text{th}$ }
\newcommand{\nTH}{$n^\text{th}$ }

%>>>>>>>>>>>>>>>>>>>>>>>>>>>>>>
%<<<    upright commands    <<<
%>>>>>>>>>>>>>>>>>>>>>>>>>>>>>>
\newcommand{\upcol}{\textup{:}}
\newcommand{\upsemi}{\textup{;}}
\providecommand{\lparen}{\textup{(}}
\providecommand{\rparen}{\textup{)}}
\renewcommand{\lparen}{\textup{(}}
\renewcommand{\rparen}{\textup{)}}
\newcommand{\Iff}{\emph{iff} }

%>>>>>>>>>>>>>>>>>>>>>>>>>>>>>>
%<<<     Environments       <<<
%>>>>>>>>>>>>>>>>>>>>>>>>>>>>>>
\newcommand{\squishlist}
{ %\setlength{\topsep}{100pt} doesn't seem to do anything.
  \setlength{\itemsep}{.5pt}
  \setlength{\parskip}{0pt}
  \setlength{\parsep}{0pt}}
\newenvironment{itemise}{
\begin{list}{\textup{$\rightsquigarrow$}}
   {  \setlength{\topsep}{1mm}
      \setlength{\itemsep}{1pt}
      \setlength{\parskip}{0pt}
      \setlength{\parsep}{0pt}
   }
}{\end{list}\vspace{-.1cm}}
\newcommand{\INDENT}{\textbf{}\phantom{space}}
\renewcommand{\INDENT}{\rule{.7cm}{0cm}}

\newcommand{\itm}[1][$\rightsquigarrow$]{\item[{\makebox[.5cm][c]{\textup{#1}}}]}


%\newcommand{\rednote}[1]{{\color{red}#1}\makebox[0cm][l]{\scalebox{.1}{rednote}}}
%\newcommand{\bluenote}[1]{{\color{blue}#1}\makebox[0cm][l]{\scalebox{.1}{rednote}}}

\newcommand{\rednote}[1]
{{\color{red}#1}\makebox[0cm][l]{\scalebox{.1}{\rotatebox{90}{?????}}}}
\newcommand{\bluenote}[1]
{{\color{blue}#1}\makebox[0cm][l]{\scalebox{.1}{\rotatebox{90}{?????}}}}


\newcommand{\funcdef}[4]{\begin{align*}
#1&\to #2\\
#3&\mapsto#4
\end{align*}}

%\newcommand{\comment}[1]{}

%>>>>>>>>>>>>>>>>>>>>>>>>>>>>>>
%<<<       Categories       <<<
%>>>>>>>>>>>>>>>>>>>>>>>>>>>>>>
\newcommand{\Ens}{{\mathscr{E}ns}}
\DeclareMathOperator{\Sheaves}{{\mathsf{Shf}}}
\DeclareMathOperator{\Presheaves}{{\mathsf{PreShf}}}
\DeclareMathOperator{\Psh}{{\mathsf{Psh}}}
\DeclareMathOperator{\Shf}{{\mathsf{Shf}}}
\DeclareMathOperator{\Varieties}{{\mathsf{Var}}}
\DeclareMathOperator{\Schemes}{{\mathsf{Sch}}}
\DeclareMathOperator{\Rings}{{\mathsf{Rings}}}
\DeclareMathOperator{\AbGp}{{\mathsf{AbGp}}}
\DeclareMathOperator{\Modules}{{\mathsf{\!-Mod}}}
\DeclareMathOperator{\fgModules}{{\mathsf{\!-Mod}^{\textup{fg}}}}
\DeclareMathOperator{\QuasiCoherent}{{\mathsf{QCoh}}}
\DeclareMathOperator{\Coherent}{{\mathsf{Coh}}}
\DeclareMathOperator{\GSW}{{\mathcal{SW}^G}}
\DeclareMathOperator{\Burnside}{{\mathsf{Burn}}}
\DeclareMathOperator{\GSet}{{G\mathsf{Set}}}
\DeclareMathOperator{\FinGSet}{{G\mathsf{Set}^\textup{fin}}}
\DeclareMathOperator{\HSet}{{H\mathsf{Set}}}
\DeclareMathOperator{\Cat}{{\mathsf{Cat}}}
\DeclareMathOperator{\Fun}{{\mathsf{Fun}}}
\DeclareMathOperator{\Orb}{{\mathsf{Orb}}}
\DeclareMathOperator{\Set}{{\mathsf{Set}}}
\DeclareMathOperator{\sSet}{{\mathsf{sSet}}}
\DeclareMathOperator{\Top}{{\mathsf{Top}}}
\DeclareMathOperator{\GSpectra}{{G-\mathsf{Spectra}}}
\DeclareMathOperator{\Lan}{Lan}
\DeclareMathOperator{\Ran}{Ran}

%>>>>>>>>>>>>>>>>>>>>>>>>>>>>>>
%<<<     Script Letters     <<<
%>>>>>>>>>>>>>>>>>>>>>>>>>>>>>>
\newcommand{\scrQ}{\mathscr{Q}}
\newcommand{\scrW}{\mathscr{W}}
\newcommand{\scrE}{\mathscr{E}}
\newcommand{\scrR}{\mathscr{R}}
\newcommand{\scrT}{\mathscr{T}}
\newcommand{\scrY}{\mathscr{Y}}
\newcommand{\scrU}{\mathscr{U}}
\newcommand{\scrI}{\mathscr{I}}
\newcommand{\scrO}{\mathscr{O}}
\newcommand{\scrP}{\mathscr{P}}
\newcommand{\scrA}{\mathscr{A}}
\newcommand{\scrS}{\mathscr{S}}
\newcommand{\scrD}{\mathscr{D}}
\newcommand{\scrF}{\mathscr{F}}
\newcommand{\scrG}{\mathscr{G}}
\newcommand{\scrH}{\mathscr{H}}
\newcommand{\scrJ}{\mathscr{J}}
\newcommand{\scrK}{\mathscr{K}}
\newcommand{\scrL}{\mathscr{L}}
\newcommand{\scrZ}{\mathscr{Z}}
\newcommand{\scrX}{\mathscr{X}}
\newcommand{\scrC}{\mathscr{C}}
\newcommand{\scrV}{\mathscr{V}}
\newcommand{\scrB}{\mathscr{B}}
\newcommand{\scrN}{\mathscr{N}}
\newcommand{\scrM}{\mathscr{M}}

%>>>>>>>>>>>>>>>>>>>>>>>>>>>>>>
%<<<     Fractur Letters    <<<
%>>>>>>>>>>>>>>>>>>>>>>>>>>>>>>
\newcommand{\frakQ}{\mathfrak{Q}}
\newcommand{\frakW}{\mathfrak{W}}
\newcommand{\frakE}{\mathfrak{E}}
\newcommand{\frakR}{\mathfrak{R}}
\newcommand{\frakT}{\mathfrak{T}}
\newcommand{\frakY}{\mathfrak{Y}}
\newcommand{\frakU}{\mathfrak{U}}
\newcommand{\frakI}{\mathfrak{I}}
\newcommand{\frakO}{\mathfrak{O}}
\newcommand{\frakP}{\mathfrak{P}}
\newcommand{\frakA}{\mathfrak{A}}
\newcommand{\frakS}{\mathfrak{S}}
\newcommand{\frakD}{\mathfrak{D}}
\newcommand{\frakF}{\mathfrak{F}}
\newcommand{\frakG}{\mathfrak{G}}
\newcommand{\frakH}{\mathfrak{H}}
\newcommand{\frakJ}{\mathfrak{J}}
\newcommand{\frakK}{\mathfrak{K}}
\newcommand{\frakL}{\mathfrak{L}}
\newcommand{\frakZ}{\mathfrak{Z}}
\newcommand{\frakX}{\mathfrak{X}}
\newcommand{\frakC}{\mathfrak{C}}
\newcommand{\frakV}{\mathfrak{V}}
\newcommand{\frakB}{\mathfrak{B}}
\newcommand{\frakN}{\mathfrak{N}}
\newcommand{\frakM}{\mathfrak{M}}

\newcommand{\frakq}{\mathfrak{q}}
\newcommand{\frakw}{\mathfrak{w}}
\newcommand{\frake}{\mathfrak{e}}
\newcommand{\frakr}{\mathfrak{r}}
\newcommand{\frakt}{\mathfrak{t}}
\newcommand{\fraky}{\mathfrak{y}}
\newcommand{\fraku}{\mathfrak{u}}
\newcommand{\fraki}{\mathfrak{i}}
\newcommand{\frako}{\mathfrak{o}}
\newcommand{\frakp}{\mathfrak{p}}
\newcommand{\fraka}{\mathfrak{a}}
\newcommand{\fraks}{\mathfrak{s}}
\newcommand{\frakd}{\mathfrak{d}}
\newcommand{\frakf}{\mathfrak{f}}
\newcommand{\frakg}{\mathfrak{g}}
\newcommand{\frakh}{\mathfrak{h}}
\newcommand{\frakj}{\mathfrak{j}}
\newcommand{\frakk}{\mathfrak{k}}
\newcommand{\frakl}{\mathfrak{l}}
\newcommand{\frakz}{\mathfrak{z}}
\newcommand{\frakx}{\mathfrak{x}}
\newcommand{\frakc}{\mathfrak{c}}
\newcommand{\frakv}{\mathfrak{v}}
\newcommand{\frakb}{\mathfrak{b}}
\newcommand{\frakn}{\mathfrak{n}}
\newcommand{\frakm}{\mathfrak{m}}

%>>>>>>>>>>>>>>>>>>>>>>>>>>>>>>
%<<<  Caligraphic Letters   <<<
%>>>>>>>>>>>>>>>>>>>>>>>>>>>>>>
\newcommand{\calQ}{\mathcal{Q}}
\newcommand{\calW}{\mathcal{W}}
\newcommand{\calE}{\mathcal{E}}
\newcommand{\calR}{\mathcal{R}}
\newcommand{\calT}{\mathcal{T}}
\newcommand{\calY}{\mathcal{Y}}
\newcommand{\calU}{\mathcal{U}}
\newcommand{\calI}{\mathcal{I}}
\newcommand{\calO}{\mathcal{O}}
\newcommand{\calP}{\mathcal{P}}
\newcommand{\calA}{\mathcal{A}}
\newcommand{\calS}{\mathcal{S}}
\newcommand{\calD}{\mathcal{D}}
\newcommand{\calF}{\mathcal{F}}
\newcommand{\calG}{\mathcal{G}}
\newcommand{\calH}{\mathcal{H}}
\newcommand{\calJ}{\mathcal{J}}
\newcommand{\calK}{\mathcal{K}}
\newcommand{\calL}{\mathcal{L}}
\newcommand{\calZ}{\mathcal{Z}}
\newcommand{\calX}{\mathcal{X}}
\newcommand{\calC}{\mathcal{C}}
\newcommand{\calV}{\mathcal{V}}
\newcommand{\calB}{\mathcal{B}}
\newcommand{\calN}{\mathcal{N}}
\newcommand{\calM}{\mathcal{M}}

%>>>>>>>>>>>>>>>>>>>>>>>>>>>>>>
%<<<<<<<<<DEPRECIATED<<<<<<<<<<
%>>>>>>>>>>>>>>>>>>>>>>>>>>>>>>

%%% From Kac's template
% 1-inch margins, from fullpage.sty by H.Partl, Version 2, Dec. 15, 1988.
%\topmargin 0pt
%\advance \topmargin by -\headheight
%\advance \topmargin by -\headsep
%\textheight 9.1in
%\oddsidemargin 0pt
%\evensidemargin \oddsidemargin
%\marginparwidth 0.5in
%\textwidth 6.5in
%
%\parindent 0in
%\parskip 1.5ex
%%\renewcommand{\baselinestretch}{1.25}

%%% From the net
%\newcommand{\pullbackcorner}[1][dr]{\save*!/#1+1.2pc/#1:(1,-1)@^{|-}\restore}
%\newcommand{\pushoutcorner}[1][dr]{\save*!/#1-1.2pc/#1:(-1,1)@^{|-}\restore}









\usepackage{fancyhdr}

\includeversion{PartOne}
\includeversion{PartTwo}
\includeversion{PartThree}
\includeversion{PartFour}
\includeversion{PartFive}
\includeversion{PartSix}
\includeversion{PartSeven}
\includeversion{PartEight}
\includeversion{PartNine}
\includeversion{PartTen}

%Juvitop Fall 2011
\excludeversion{SaulSimplicialLocalisation}
\excludeversion{AlexandreBausfieldLocalisation}
\includeversion{InnaSimplicalModelCats}

%Babytop Fall 2011
\excludeversion{GeoffroyTopologicalHochschildHomology}




\rfoot{\footnotesize Michael Donovan}
\newcommand{\KanSemResponse}[1]
{
\thispagestyle{fancy}
\subsection*{#1}
}

\begin{document}
\begin{SaulSimplicialLocalisation}
\KanSemResponse
{``Simplicial Localisation'' --- Saul Glasman --- 6/10/2011}
\begin{abstract}
The message of simplicial localisation is that trying to invert 
morphisms in a category catapults the would-be-inverter into homotopy 
theory, whether they want to be there or not. This accounts, in part, 
for the ubiquity of homotopical concepts in modern mathematics. First 
I'll give a brief and bracing refresher on simplicial sets for those 
whose heads are not yet simplicial. I'll discuss two perspectives on 
simplicial localisation, first presenting the useful and picturesque 
hammock localisation and then teaching you how to take a free resolution 
of a category. I'll make a few remarks on how great this is and apply it 
to the theory of model categories. Most of the material on this talk is 
based on three seminal 1980 papers by Dwyer and Kan.
\end{abstract}
Group completion \rednote{(do it levelwise?)}, classifying spaces, operads, etc.\ are all better in the category of simplicial sets. We have a Quillen adjunction:
\[|\DASH|:\sSet\longleftrightarrow\Top:\Sing\]
These form a Quillen equivalence.
\begin{exmp*}
The nerve of a category $\calC$ is the simplicial set whose $k$-simplices are chains of $k$ composable morphisms. Forming the realisation of the nerve of a group (as a category) you get the classifying space.
\end{exmp*}
\textbf{Drawback:} Not everything is fibrant (unlike in $\Top$).

This is bad, for example if you'd like to calculate, say, self maps of $S^1$, if we view $S^1$ simply as the boundary of the $2$-simplex.
The fibrant objects in $\sSet$ are the Kan complexes, ``characterised by the fact that they are enormous''.
\subsection*{Simplicial Localisation}
Given a category $\calC$ and a class $\scrW$ of ``weak equivalences'' in $\calC$. $\scrW$ should have the 2 of 3 property, and maybe some other stuff. We want to localise $\calC\to\calC[\scrW^{-1}]$ with the obvious universal property.

Let's write a candidate for $\calC[\scrW^{-1}]$. The objects are the same, while the maps are equivalence classes of zigzags (\rednote{under some relation} where we at least allow ourselves to add in pairs of identities):
\[\xymatrix{
X\ar[r]&A_0&\ar[l]_\sim A_1\ar[r]&A_2&\ar[l]_\sim\cdots\ar[r]&A_{2n}&A_{2n+1}\ar[l]_\sim\ar[r]&Y
}\]
There are computational problems here. The reason for this difficulty is that we've skipped a step.
\begin{defn*}
A simplicial category is one in which $\Map(X,Y)$ is a simplicial set, composition is a morphism of simplicial sets. Actually we could just say that a simplicial category is a functor $\Delta^\op\to\Set$, such that for any $I,J\in\Delta$, $S(I)$ and $S(J)$ have the same objects, and $S(I)\to S(J)$ is the identity on objects. The morphisms are free to do as they please.
\end{defn*}
There's a universal simplicial category $L^\scrW C$ in which $\scrW$ becomes invertible. I.e.\ if $\calD$ is a simplicial category, and we have a functor sending $\scrW$ to isomorphisms, we get a unique (up to homotopy) factorisation
\[\xymatrix{
\ar[d]\calC\ar[r]&\calD\\
L^\scrW \calC\ar@{..>}[ur]
}\]
In particular, $\calD$ could be $\calC[\scrW^{-1}]$, so we have filled in our missing step.
\[\xymatrix{
\ar[d]\calC\ar[r]&\calC[\scrW^{-1}]\\
L^\scrW \calC\ar@{..>}[ur]
}\]
Moreover, the map on morphisms given by this factorisation is the projection to $\pi_0$ (after we realise the simplicial set of morphisms $L^\scrW \calC(X,Y)$).
\subsection*{Constructing the simplicial localisation}
Suppose we have $X,Y\in\ob\calC$. The $0$-simplices in $\Mor(X,Y)$ are just zig-zags (up to an equivalence relation in which we can add in identities):
\[\xymatrix{
X\ar[r]&A_0&\ar[l]_\sim A_1\ar[r]&A_2&\ar[l]_\sim\cdots\ar[r]&A_{2n}&A_{2n+1}\ar[l]_\sim\ar[r]&Y
}\]
$1$-simplices are going to be diagrams like:
\[\xymatrix@R=.4cm{
&A_0\ar[dd]^\sim&\ar[l]_\sim A_1\ar[dd]^\sim\ar[r]&A_2\ar[dd]^\sim&\ar[l]_\sim\cdots\ar[r]&A_{2n}\ar[dd]^\sim&A_{2n+1}\ar[dd]^\sim\ar[l]_\sim\ar[rd]\\
X\ar[rd]\ar[ur]&&&&&&&Y\\
&B_0&\ar[l]_\sim B_1\ar[r]&B_2&\ar[l]_\sim\cdots\ar[r]&B_{2n}&B_{2n+1}\ar[l]_\sim\ar[ru]
}\]
Higher simplices come from carrying on in this fashion --- adding more and more rows to the diagram. A $k$-simplex will be a hammock of width $k+1$.
Thus $\Mor(X,Y)$ is just the nerve of the category of zig-zags from $X$ to $Y$ and weak equivalences of zig-zags.

\subsection*{In a model category}
I can get a factorisation
\[\xymatrix{
X&\ar@{->>}[l]_\sim X'\ar[r]&Y'\ar@{_{(}->}[r];[]_\sim&Y
}\]
Where $X'$ is cofibrant, and $Y'$ is fibrant. So actually, we only need hammocks as in diag 3.
\subsection*{Another model for the simplicial localisation}
There's a forgetful functor $\Cat\to\mathsf{DirectedGraphs}$, which has a left adjoint, called `Free'. The free category on a directed graph has morphisms given by composable words in the edges of the graph. Now suppose I want to localise $\calC=\text{Free}(V\cup W)$, where $V$ and $W$ are graphs with no shared edges (but probably with shared vertices). Then $\calC[W^{-1}]=\text{AlmostFree}(V\cup W^{\pm1})$:  the morphisms are reduced composable words in the arrows of $V$ and of $W\cup W^{-1}$. That's easy, and if we could replace a category with something like this, we'd win.

Given $\calC$, I can form $F(\calC)$, the free category on $\calC$, by applying forget, then free. $F$ is a monad, and I have natural transformations $FF\to F$ and $1\to F$ satisfying the monoid axioms. So I'll construct a simplicial category $M$, whose category of $k$-simplices is $F^{k+1}(\calC)$, and whose faces and degeneracies are given by the monad maps.

\begin{fact}
$M$ is equivalent to $\calC$ as  simplical category.
\end{fact}
\noindent But now to invert $W$ is achievable levelwise. $M\to M[W^{-1}]$ is the simplicial localisation.
\subsection*{A simplicial model category}
should be a model category enriched in simplicial sets, which already has the `right' space of morphisms, at least between its cofibrant and fibrant objects.
\begin{defn*}
A simplicial model category is a model category, and a simplicial category, which: 
\begin{enumerate}
\item is tensored and cotensored over $\sSet$. [So that given $X\in \calC$ and $K\in\sSet$, can take $X\otimes K$ and $\Hom(K,X)$, and there's an adjunction].
\item Given a cofibration $A\cofibration B$ and a fibration $X\fibration Y$, there is a natural map:
\[c:\Hom(B,X)\to\Hom(A,X)\times_{\Hom(A,Y)}\Hom(B,Y).\]
The codomain is the space of ways to complete the following into a commuting square:
\[\xymatrix@R=.6cm{
A\ar@{^{(}->}[d]&X\ar@{->>}[d]\\
B&Y
}\]
We demand that $c$ is a fibration, and if either $X\fibration Y$ or $A\cofibration B$ is acyclic, that $c$ is an acyclic fibration.
\end{enumerate}
\begin{punchline}
If $\calC$ is a simplicial model category, and $\calC'$ is the underlying category (take $0$-simplices as morphisms), then $\calC$ and $L^\scrW(\calC')$ are equivalent and simplicial categories.\footnote{$\scrW$ is the $0$-simplices that are weak equivalences.}
\end{punchline}
\subsection*{Post talk ramblings}
A weak equivalence of simplicial categories is a zig-zag of (simplicial) functors $F_i:\calC_i\to\calC_{i+1}$ such that $\pi_0F_i$ is a weak equivalence (of spaces), and for each $X,Y\in\ob(\calC_i)$, $\Mor(X,Y)\to\Mor(F_iX,F_iY)$ is a weak equivalence in the model category of simplicial sets.
\end{defn*}


\pagebreak
\end{SaulSimplicialLocalisation}
\begin{AlexandreBausfieldLocalisation}
\KanSemResponse
{``Localization of Model Categories'' --- Luis Alexandre Pereira --- 13/10/2011}
\begin{abstract}
The problem of localizing a model category $M$ is that of enlarging the 
class of weak equivalences to include an additonal set $S$ of maps while 
still obtaining a model structure $S^{-1}M$, preferably one that can be 
related to the original model structure on $M$ (meaning a Quillen 
adjunction is desirable). While such a localization needs not always 
exist, there are certain hairy technical conditions (which nonetheless 
hold in most model categories of interest) on $M$ that always allow one to 
form $S^{-1}M$.
This is the so-called left Bousfield localization, whose base category 
is $M$ itself and whose cofibrations are precisely those of $M$, with an 
enlarged class of weak equivalences.

After discussing this process and 
some general ideas of how and why it works (this will include a rare 
sighting of the definition of the ubiquitous Reedy model structures of 
diagrams), we will illustrate it's importance with a wealth of examples: 
Bousfield's localization of spaces with respect to homology (the primal 
example), Dugger's hocolim localization of $sM$, showing a wide class of 
model categories can be made simplicial, Resk's complete segal spaces (a 
model for $(\infty,1)$-categories) obtained from localizing simplicial spaces, 
and Dugger's theory of presentations of model categories, which are 
localizations of the so-called universal model categories.
\end{abstract}
\begin{PartOne}
Let $M$ be a model category. This structure supports the inversion of the weak equivalences. Have $\ho M(X,Y)=M(X_C,Y_F)/\sim$. What if we want to further localise $\ho M$, by a buch of maps $S$. Want:
\[\xymatrix{
M\ar[d]\ar@{..>}[r]&N\ar[d]\\
\ho M\ar[r]&\ho M[S^{-1}]
}\]
We'll get a quillen pair, and there'll be a choice of which direction the adjuction goes:
\[L:M\longleftrightarrow N:R\]
We'll usually choose version that makes $L$ the left adjoint!

\begin{defn*}
A localisation of $M$ on a class $S$ of maps should be a quillen pair $L:M\to N$ such that $L$ sends $S$ to weak equivalences, and is initial with this property.
\end{defn*}
The first guess is to take $N=M$, and enlarge the weak equivalences by $S$ as necessary,  and, as we want the maps $M\to M$ to be the identity, we try to use the same fibrations. (NB: Left Quillen functors preserve cofibrations, so they have to increase. It'd be nice if we could keep the class of cofibrations unchanged).
\begin{prop*}
If such a model structure exists, it satisfies the universal property.
\end{prop*}
\begin{defn*}
These model structures are called left Bausfeild localisations.
\end{defn*}
\begin{defn*}
Let $S$ be a set of maps. Then $X\in M$ is called $S$-local if $X$ is fibrant and $\underline M(B,X)\to\underline M(A,X)$ is a weak equivalence for all $A\to B$ in $S$.
\end{defn*}
\begin{defn*}
A map $\overline A\to\overline B$ is called an $S$-local equivalence if $\underline M(\overline B,X)\to\underline M(\overline A,X)$ is a weak equivalence for all $X\in M$ $S$-local.
\end{defn*}
Now we can redefine left Bausfield localisation, calling it $L_SM$:
\begin{itemise}
\item The cofibrations are the same as those of $M$.
\item the weak equivalences are the $S$-local equivalences.
\item the fibrations are those with the right lifting property.
\item Note that the fibrant objects are now the $S$-local objects.
\end{itemise}
\subsection*{Homotopy function complexes (with a discussion of Reedy model categories)}
$\underline(X,Y)$ is supposed to be such that $\pi_0=\ho M(X,Y)$. One way to think about it is via Saul's talk. This is the simplicial mapping space. However, there are set-theoretical problems, for large categories, and model categories are always large --- a small category doesn't have enough limits and colimits.

We thus need a new definition. Suppose that $M$ is a simplicial model category. Then define $\underline M(X,Y)$ by $\underline M(X,Y)_m=M(X_C\otimes\Delta^m,Y_F)$. 

But how can we define $\underline M(X,Y)$ for an arbitrary model category? We need analogues of the cosimplicial object $X^*$:
\[\xymatrix{X_c\ar@<.6ex>[r]\ar@<-.6ex>[r]&\ar[l] X_C\otimes\Delta^1
\ar[r]\ar@<1.2ex>[r]\ar@<-1.2ex>[r]&
\ar@<.6ex>[l]\ar@<-.6ex>[l]X_C\otimes\Delta^2
}\]

The first bit should be a cylinder object, giving $X\sqcup X\cofibration X\otimes \Delta^1\weakequiv X$. Ummm, in general, want $X\otimes\partial \Delta^m\cofibration X\otimes\Delta^M\weakequiv X$ each time.

This is the same as saying that $X^*$ is a cofibrant replacement of the constant diagram at $X$ in the Reedy model structure on $cM$, the category of cosimplical objects in $M$, i.e.\ functors $\Delta\to M$.
\subsubsection*{The Reedy model structure}
Consider $\Delta$, and two subcategories: $\Delta^+$ the injecive maps, and $\Delta^-$ the injective maps. Maps in $\Delta$ factor uniquely by a map in $\Delta^1$ then a map in $\Delta^+$.
\[\xymatrix{
\Bullet\ar[rr]\ar[rd]_{-}&&\Bullet\\
&\Bullet\ar[ru]_{+}
}\]
Now we can define three model structure on $cM$:
\begin{enumerate}
\item $(cM)_{proj}$ in which the weak equivalences (fibrations) are the termwise weak equivalences (fibrations).
\item $(cM)_{inj}$ in which the weak equivalences (cofibrations) are the termwise weak equivalences (cofibrations).
\item $(cM)_{Reedy}$. Here the weak equivalences are the termwise weak equivalences. The cofibrations are the natural transformations whose restrictions to $\Delta^+$ are cofibrations in the projective model structure on $M^{\Delta^+}$. The fibrations are the natural transformations whose restrictions to $\Delta^-$ are fibrations in the injective model structure on $M^{\Delta^-}$. 
\end{enumerate}
\end{PartOne}
\begin{PartTwo}
The upshot: we can define $\underline M(A,X)=M((A^*)_c,X_f)$.
\begin{prop*}
For $M$ a model category, a map $A\to B$ is a weak equivalence \Iff the $\underline M(B,X)\to\underline M(A,X)$ are weak equivalences for all $X$.
\end{prop*}
Certainly, if $a\to B$ is to be a weak equivalence in $L_SM$, then it must be true that $\underline{L_SM}(B,X)\to\underline{L_SM}(A,X)$ is a weak equivalence. But:
$\underline{L_SM}(B,X)=\underline M(B,X)$, provided that $X$ is $L_SM$-fibrant.
\subsubsection*{Do the $L_SM$ exist?}
Fortunately, the answer is often `yes'.
\begin{thm*}
$L_SM$ always exists provided ($S$ is a set, and) $M$ is
\begin{enumerate}\squishlist
\item left proper combinatorial; or
\item left proper cellular.
\end{enumerate}
\end{thm*}
\begin{defn*}
It's left proper if when you push out a weak equiv along a cofibration, you get a weak equiv.
\end{defn*}
\begin{defn*}
It's combinatorial if it's cofibrantly generated and satisfies some set-theoretic conditions. Cellular is similar.
\end{defn*}
\begin{defn*}
It's cofibrantly generated if there is a \textbf{set} $I$ of cofibrations and a set $J$ of trivial fibrations so that $I$ determines the trivial fibrations by the right lifting property, and $J$ does the same for the (cofibrations?). (the full classes alway determine this stuff, but are proper classes (in general, at least))
\end{defn*}
\begin{exmp*}
\begin{itemise}
\item $\Top$ is cellular
\item $\sSet$ is both cellular and combinatorial
\item $(M^C)_{proj}$ exists and has whatever of these properties $M$ has. \rednote{(What's this?)}
\item $(M^R)_{reedy}$ exists and has whatever of these properties $M$ has.
\end{itemise}
\end{exmp*}
Lets have some examples
\begin{enumerate}
\item Let $M$ be $\sSet$ or $\Top$, and let $S$ be the homology isomorphisms.
\item ($m$-types) Let $M=\sSet$, and let $S=\partial\Delta^{m+2}\to \Delta^{m+2}$.

What now are the fibrant objects? Should have $\Map(\Delta^{n+2},X)\weakequiv\Map(S^{n+1},X)$, which implies that $\pi_{n+1}(X)=0$.

If you smash $\partial\Delta^{m+2}\to \Delta^{m+2}$ with $\partial\Delta^0\to\Delta^0$, you get that the higher inclusions $S^{m+2}\cofibration \Delta^{m+3}$ are weak equivalences, and so on. So, $L_SM$ is the homotopy theory of $m$-types, and fibrant replacement consists of taking the $m^\text{th}$ stage of the Postnikov tower.
\item (Dugger) He gets a Quillen equivalence $c_*:M\longleftrightarrow sM_{hc}:(\DASH)_0$, where the right hand side is $((sM)_{reedy})_{\textit{hocolim equivalences}}$.
\item (Rezk's complete Segal spaces) $M=sSet^{\Delta^\op}$ (the same as bisimplical set, but we want to distinguish a coordinate). A Segal space is $X_*$ such that there's a map $X_m\to X_1\times_{X_0}\cdots\times_{X_0}X_1$. Ummm $X_0$ is supposed to be the space of objects in an $(\infty,1)$-cat. $X_1$ is supposed to be the space of arrows, $X_2$ is the space of composable pairs of arrows, etc. I.e.\ you take a simplex and you look at its spine. (?context?) These `Segal spaces' are supposed to represent $(\infty,1)$-categories.

Ummmmmmmm localise with respect to $F_{\Delta^1}\sqcup_{F_{\Delta^0}}\cdots\sqcup_{F_{\Delta^0}}F_{\Delta^1}\to F_{\Delta^n}$.
\end{enumerate}
\end{PartTwo}
\pagebreak
\end{AlexandreBausfieldLocalisation}
\begin{GeoffroyTopologicalHochschildHomology}
\newcommand{\THH}{\textup{THH}}
\newcommand{\HH}{\textup{HH}}
\KanSemResponse
{``Topological Hochschild Homology'' --- Geoffroy Horel --- 18/10/2011}
\begin{abstract}
In this talk we will try to give an overview of THH
(Topological Hochschild Homology) which is the analogue in the category of
spectra of good-old Hochschild homology for associative algebra over field.
We will give two different construction of THH. The first one through the
cyclic bar construction has the advantage of being a straightforward
generalization of the algebraic version. The second one through
factorization homology is more interesting as it describes THH as an
example inside a large family of constructions indexed by framed manifolds.
Finally if time permits we will introduce the Bockstedt spectral sequence
and make an explicit computation of THH(KU).
\end{abstract}
\subsection*{Hochschild Homology}
Suppose $k$ is a commutative ring and $A$ an associative $k$-algebra. I can define $A\otimes_k A^\op$, and $(A\otimes_k A^\op)^\op=A\otimes_k A^\op$, so that left and right modules coincide. We'll call such things bimodules:
\[MOD_{A\otimes_k A^\op}:=Bimod{A-A}\]
For $M\in Bimod{A-A}$, 
\[\HH_*(A,M):=\Tor^{A\otimes_k A^\op}(A,M)\]
and 
\[\HH_*(A):=\Tor^{A\otimes_k A^\op}(A,A).\]
There's a nice projective resolution called:
\subsubsection*{Cyclic bar construction}
Suppose $A$ is prjective over $k$. Define $CB_*^k(A)\in Mod_k^{\Delta^\op}$, $CB_a(A)=A^{\otimes n+1}$.
\[d_i:CB_m(A)\to CB_{m-1}(A) (i\in\{0,m+1\})\]
\[d_0(a_0\otimes a_{m+1})=a_0a_1\otimes a_2\otimes\cdots\]
\[d_m(a_0\otimes a_{m+1})=a_0\otimes a_1\otimes\cdots\otimes a_{m-1}a_{m}\]
\[d_{m+1}(a_0\otimes a_{m+1})=a_ma_0\otimes a_1\otimes\cdots\otimes a_{m-1}\]
Taking the alternating sum of the faces, get a chain complex, and
$H_*(CD_*(A))=\HH_*(A)$.
Why do people care about Hochschild homology?
\begin{enumerate}
\item The Hochschild-Kostant-Rosenbern theorem:
\begin{thm*}
If $k\to A$ is a smooth map of commutative algebras then there is a map $\Omega_k^*(A)\weakequiv \HH_*^k(A)$ (left is deRham) which is an isomorphism of DGAs. In degree zero, have the identity $A\to\HH_0(A)=A$. In degree one:
\[\Omega^1(A)=\{a\cdot db:a,b\in A\}/(a\cdot d(bc)=ac\cdot db+ab\cdot dc)\]
There's a map $\Omega^1(A)\weakequiv\HH_1(A)=A\otimes A/\langle\rangle$, sending $a\otimes b\mapsto a\cdot db$.

Now $\HH_*^k$ is the right analogue for the DeRham differential forms wh

Now we have a deRham differential $\Omega^*(A)\to\Omega^{*+1}(A)$, and there's a Connes differential $\HH_*^k(A)\to \HH_{*+1}^k(A)$, which is preserved by the isomorphism. That's a plus.
\end{thm*}
An easy computation is:
$\HH_*^k(k[x_1,x_2,\ldots,x_n])=k[x_1,\ldots,x_n]\otimes_k E(\sigma x_1,\ldots,\sigma x_n)$, with $|\sigma x_i|=1$ --- these correspond to $d x_1,\ldots, d x_n$.
\end{enumerate}
\subsection*{THH}
Let's say we're working with a symmetric monoidal model category of spectra, call it $(\spec,\otimes,S)$. (You choose.) Well suppose $R$ is a cofibrant commutative algebra in $\spec$. There is a symmetric monoidal model category of $R$-modules $Mod_R$, with an adjunction $\spec\longleftrightarrow Mod_R$. 
%Thus I can talk about associative algebras in $R$-models, and this has a model structure, 
If $A$ is an associative algebra in $Mod_R$, I can still define $CB_m(A)=A^{\otimes m+1}$

My first definition of $\THH$ will be the left derived functor
\[\THH^R(A)=L|CB_*(A)|=|CB_*(A^c)|.\]
Let's give a definition where the circle action is more clear, and where we can obviously plug in an $A_\infty$-algebra:
\begin{defn*} 
Let $Fin$ be the category of finite sets. Let $E_1$ be the topological category whose objects are finite sets, and $E_1(X,Y)=\coprod_{f\in Fin(X,Y)}\prod_{y\in Y} \scrE_1(f^{-1}(y))$., where $\scrE_1(X)$, for $X$ a finite set, is the space of configurations of intervals in $(0,1)$, with the intervals labeled by $X$.
 This is monoidall under disjoint union.

An $A_\infty$-algbra is a monoidal topological functor $A^{\otimes\DASH}:E_1\to\spec$. (This $\scrE_1$ is a model of the $A_\infty$-operad).

Let's define a functor $C:E_1^\op\to\Top$ sending $X$ to a configuration of intervals in the circle, labeled by the elements of $X$. It's straightforward to see how to define the action of the functor on maps: need $C(Y)\times E_1(X,Y)\to C(X)$, i.e.\ for all $f:X\to Y$ a map, via the obvious operadic type map.
\end{defn*}
Can take the homotopy coend $C\otimes_{E_1}^L A^{\otimes\DASH}=:\THH(A)$. This has an obvious $S^1$-action, as the circle acts on $C$.
\subsection*{B\"ockstedt spectral sequence}
Suppose $K_*$ is a multiplicative homology theory with a K\"unneth isomorphism.
\[E^1_{s,t}=K_t(CB_s(A))\implies K_{s+t}(\THH(A))\]
The $d^1$ differential is the alternating sum of somesuch. As $K$ has a K\"unneth isomorphism, you realise that
\[E^1_{s,*}=\HH^{K_*}_s(K_*A)\implies K_*(\THH(A)).\]
As an application, define $KU$ to be the $p$-completed complex $k$-theory spectrum. Also, Morava $K(1)_*$-theory is a summand of $KU/p$. Then you can do some stuff.













\pagebreak
\end{GeoffroyTopologicalHochschildHomology}
\begin{InnaSimplicalModelCats}
\KanSemResponse
{``(Simplicial) (model) categories'' --- Inna Zakharevich --- 19/10/2011}
\begin{abstract}
We've talked about model categories, we've talked about simplicial categories. Now we must tackle the two monsters... together. This talk will be about the interactions of the simplicial structure and the model structure on the category of simplicial objects in a model category: $SM7$, the Reedy structure, and (if we have time) the $E^2$ model structure.
\end{abstract}
We had the rather complicated hammock localisation:
\[\text{Model Cats}\overset{L_H}{\to}\text{Simplicial Cats}\]
If $\calC$ is a model category with a simplicial enrichment, we want $\pi_0\underline\Hom(A,B)$ to be the set of homotopy classes $\Hom_{\ho\calC}(A,B)$ of maps $A$ to $B$.

We'd rather like $L_H(A,B)\simeq \Hom_\calC(A,B)$, maybe with some conditions of cofibrancy, etc.

In this talk, $\underline\Hom$ refers to a space of morphisms, but I didn't notice this for a while, so some underlines might be missing.
\subsection*{Axiom SM7}
\[\xymatrix{
A\ar@{-->}[r]^\alpha\ar@{^{(}->}[d]_i&X\ar@{->>}[d]^P\\
B\ar@{-->}[r]_\beta\ar@{..>}[ur]&Y
}\]
Have, for $i,P$ fixed, a function 
 \[\Hom(B,X)\xrightarrow{(i^*,P_*)}\Hom(B,Y)\times_{\Hom(A,Y)}\Hom(X,Y)\]
expressing the fact that ``for every lift you get a diagram''. Note that
we didn't say that $i$ or $P$ had to by acyclic.

\noindent \textbf{\Bullet SM7:} $(i^*,P_*)$ is always a fibration. If either $i$ or $P$ is acyclic, then so is $(i^*,P_*)$.
\begin{itemise}
\item We want the lift to be well defined, at least up to a contractible choice, when it is guaranteed to exist by the model category axioms.
\item You want there to be some continuity in the answer to the question ``what lifts exist for $(\alpha,\beta)$'' to vary smoothly in $(\alpha,\beta)$, where $\alpha:A\to X$ and $\beta:B\to Y$ make the diagram commute. For example, if no lift exists, and you deform the diagram slightly, you don't want to suddenly see a lift.
\end{itemise}
\begin{defn*}
A \emph{simplicial model category} is a model category enriched over $\sSet$ satisfying SM7.
\end{defn*}
\noindent There are the best things that are both simplicial and model.
\subsection*{How to simplicially enrich $\sSet$}
Well, we start be setting 
\[\underline{\Hom}(A,B)_0:=\Hom(A,B).\]
We really want the 1-simplices to be homotopies, so define:
\[\underline{\Hom}(A,B)_1:=\Hom(A\times\Delta^1,B).\]
We carry on this way, using $\Delta^k$ instead of $(\Delta^1)^k$, the former being more simplicial:
\[\underline{\Hom}(A,B)_k:=\Hom(A\times\Delta^k,B).\]
\begin{thm*}[Quillen?]
This works, making $\sSet$ a simplicial model category.
\end{thm*}
\begin{proof} See Goerss and Jardine, chapter II.
\end{proof}
We can generalise this. Suppose that $\calC$ has all coproducts (for example, if $\calC$ is a model category). We construct a functor $\calC\times\sSet\xrightarrow{\otimes}s\calC$, where $s\calC=[\Delta^\op,\calC]$, by setting\footnote{A common alternative way to think of this: it's the diagonal in the bisimplicial set $(X\otimes K)_{ik}=\coprod_{K_i}X_j$.} \rednote{(any further identification in this colimit?)}
\[(X\otimes K)_n=\coprod_{K_n}X_n.\]
Now in general, given a functor $\calC\times\sSet\xrightarrow{\otimes}\calC$, (satisfying some conditions), we can give a simplicial enrichment on $\calC$, via the recipe:
\[\underline{\Hom}(A,B)_k:=\Hom(A\otimes\Delta^k,B).\]
All the simplicial structure here comes from the $\Delta^k$ coordinate.

\subsection*{$\calC$ is a model category}
Consider $s\calC$. We have a simplicial structure on $s\calC$ (from previous). We also have the projective and injective model structures
\subsection*{Why do we care about model structures on diagram categories?}
Suppose we have a diagram, like $D$: $\Bullet\longrightarrow\Bullet\longleftarrow\Bullet$.  We have the functor
\[\text{pullback}:\Top^D\to\Top.\]
Both of these are model categories, but this isn't a Quillen functor [i.e.\ doesn't preserve weak equivs]. Pullback is not compatible with homotopy.\footnote{For example, the pullback of $\ast\longrightarrow S^1\longleftarrow\ast$ is either a point or the empty set.} \emph{However}, if the two maps in are fibrations, even if you change either the objects or the maps up to homotopy, you get a weak equiv on pullbacks.\footnote{In spaces, you get lucky and only need one to be a fibration.}

There is a model structure on $\Top^D$, in particular, it's the injective structure you need for pullbacks. \rednote{(I don't get how some of the comments hereabouts are relating to each other.)}

%We'll look at a theorem of Reedy which will work for any reedy diagram category. 
The injective and projective structures aren't right \rednote{(for what?)}. To demonstrate this, I'll try to characterise the \rednote{(acyclic?)} cofibrations on the model category $s\calC$ given the projective model structure, between objects $A_*$ and $B_*$.

To get a lift, we at least need one on the $0$-level, so we'll need $A_0\to B_0$ to be an acyclic cofibration in $\calC$:
\[\xymatrix{
A_0\ar[r]\xycofib[d]^\simeq&X_0\ar@{->>}[d]\\
B_0\ar[r]\ar@{..>}[ur]&Y_0
}\]
Let's try to go a little higher. The lift on the next level is a rather more complicated piece of cheese:
\[\xymatrix@!0@R=8mm@C=1cm{
A_0\ar[dddd]\ar[rrrrrrr]\xycofibdown[drrr]&&&&&&&X_0\ar[dddd]\\
&&&B_0\ar@{..>}[rrrru]\ar[ldd]\ar[dddd]&&&\\
\\
&&\text{Push}\ar@{-->}[ddr]\\%^(.3)\sim\\
A_1\ar[rrrrrrr]\ar[drrr]\ar[rru]&&&&&&&X_1\\
&&&B_1\ar@{..>}[rrrru]
}\]
Somehow, things work when the dashed map on the left face from the pushout to $B_1$ is a weak equivalence. \rednote{(I haven't figured this out yet.)} Anyway, this is all getting complicated rather quickly, so we should move on instead to:
\subsection*{The Reedy model structure}
\begin{thm*}
We have a model structure on $s\calC$, where the weak equivalences are levelwise, and
\[\textup{proj}\weakequiv\textup{Reedy}\weakequiv\textup{inj}\]
are left Quillen equivalences.\footnote{Reedy's paper on this is hard to find, Goerss and Jardine's account is hard to read.}
\end{thm*}
\noindent The downside is that the Reedy structure is not simplicial.
\begin{thm*}
If $\calC\otimes \sSet\to \calC$ ($\calC$ a model category) satisfies, for all $A\cofibration B\in\calC$:
\begin{enumerate}\squishlist
\item $(A\otimes \Delta^n)\cup_{A\otimes\partial\Delta^n}(B\otimes\partial\Delta^n)\to B\otimes\Delta^n$, the ``box product''\footnote{Suppose that $\calC=\Top$ and $\Top\otimes\sSet\to\Top$ is $A\otimes K\mapsto A\times|K|$. Suppose given a cofibration $A\cofibration B$. The ``box product'' maps out of $A\times D^n\cup B\times S^{n-1}$, a cylender $B\times D^n$ with $B\setminus A\times (D^n)^\circ$ removed. It is then the inclusion of this into the full cylender $B\times D^n$. \bluenote{It seems likely that this functor satisfies the criteria of the theorem.}}, is a cofibration, which is acyclic whenever $i$ is acyclic; and
\item The same box product for $\Delta^0\acycliccofibration\Delta^1$ is always an acyclic cofibration;
\end{enumerate}
then $\calC$ is a simplicial model category.
\end{thm*}
\noindent In the Reedy structure, it's the easier condition, 2, that fails.

There's another structure, the $E^2$ model structure, for which this works, in a paper by Dwyer, Kan and Stover. They say $X\weakequiv Y$ is a weak equivalence \Iff $\pi_i(\pi_j X)\overset{\cong}{\to}\pi_i(\pi_j  Y)$ is iso for all $i$, $j$ (the homotopy of a simplicial group). (That is, we get an $E^2$-page iso (??)).
\subsection*{The Reedy structure}
\begin{defn*}[Formal definition] Let $\Delta_n$ be the full subcat of $\Delta$ of objects $\leq n$. Have inclusion $\Delta_n\to\Delta$, giving
\[\calC^{\Delta^{op}}\to\calC^{\Delta^{op}_n}\]
This functor has both a left and right adjoint:
\begin{itemise}
\item The left adjoint is the $n^\text{th}$ skeleton $\skel_n$. This is the object in which you throw out all the nondegenerate simplices whose dimension exceeds $n$. Note that the skeleton is the minimal object with the correct cells at dimensions $n$ and below, which suits the fact that it is a left adjoint --- it should at least like supporting maps.
\item The right adjoint is the $n^\text{th}$ coskeleton $\coskel_n$. This is the object in which you add in all possible $(n+1)$-simplices, then all possible $(n+2)$-simplices, and so on forever. Note that the coskeleton is the maximal object with the correct cells at dimensions $n$ and below, which suits the fact that it is a right adjoint --- it should at least like recieving maps.
\end{itemise}
Now let $L_nX=(\skel_{n-1}X)_n\in\calC$, and $M_nX=(\coskel_{n-1}X)_n\in\calC$ ``latching and matching''. Have $L_nX\to X_n$ and $X_n\to M_nX$. The cofibrations are the maps $A\to B$ where $A_n\sqcup_{L_nA}L_nB\to B_n$ is a cofibration for all $n$.
The fibrations are the things $A_n\to M_nA\times_{M_nB}B_n$ are fibrations for all $n$.
\end{defn*}
\begin{cor*}
These maps are acyclic for all $n$ \Iff $A\to B$ is.
\end{cor*}
The pushout is the map from the first latching object. This model structure is self-dual, halving the number of proofs!

Q: Why's $s\calC$ interesting?
A: If you wanna define Andre-Quillen cohomology, you really need simplicial objects, which are relevant to resolutions of things in some settings.
\pagebreak
\end{InnaSimplicalModelCats}
\end{document}




























% !TEX root = z_output/_Juvitop.tex
%%%%%%%%%%%%%%%%%%%%%%%%%%%%%%%%%%%%%%%%%%%%%%%%%%%%%%%%%%%%%%%%%%%%%%%%%%%%%%%%
%%%%%%%%%%%%%%%%%%%%%%%%%%% 80 characters %%%%%%%%%%%%%%%%%%%%%%%%%%%%%%%%%%%%%%
%%%%%%%%%%%%%%%%%%%%%%%%%%%%%%%%%%%%%%%%%%%%%%%%%%%%%%%%%%%%%%%%%%%%%%%%%%%%%%%%
\documentclass[11pt]{article}
\usepackage{fullpage}
\usepackage{amsmath,amsthm,amssymb}
\usepackage{mathrsfs,nicefrac}
\usepackage{amssymb}
\usepackage{epsfig}
\usepackage[all]{xy}
\usepackage{sseq}
\usepackage{tocloft}
\usepackage{cancel}
\usepackage[strict]{changepage}
\usepackage{color}
\usepackage{tikz}
\usepackage{extpfeil}
\usepackage{version}
%\usepackage{ifthen}
%Used for disabling hyperref
\ifx\dontloadhyperref\undefined
%\usepackage[pdftex,pdfborder={0 0 0 [1 1]}]{hyperref}
\usepackage[pdftex,pdfborder={0 0 .5 [1 1]}]{hyperref}
\else
\providecommand{\texorpdfstring}[2]{#1}
\fi

%>>>>>>>>>>>>>>>>>>>>>>>>>>>>>>
%<<<       Better ToC       <<<
%>>>>>>>>>>>>>>>>>>>>>>>>>>>>>>
\setlength{\cftbeforesecskip}{0.5ex}

%>>>>>>>>>>>>>>>>>>>>>>>>>>>>>>
%<<<      Hyperref mod      <<<
%>>>>>>>>>>>>>>>>>>>>>>>>>>>>>>

%needs more testing
\newcounter{dummyforrefstepcounter}
\newcommand{\labelRIGHTHERE}[1]
{\refstepcounter{dummyforrefstepcounter}\label{#1}}


%>>>>>>>>>>>>>>>>>>>>>>>>>>>>>>
%<<<  Theorem Environments  <<<
%>>>>>>>>>>>>>>>>>>>>>>>>>>>>>>
\ifx\dontloaddefinitionsoftheoremenvironments\undefined
\theoremstyle{plain}
\newtheorem{thm}{Theorem}[section]
\newtheorem*{thm*}{Theorem}
\newtheorem{lem}[thm]{Lemma}
\newtheorem*{lem*}{Lemma}
\newtheorem{prop}[thm]{Proposition}
\newtheorem*{prop*}{Proposition}
\newtheorem{cor}[thm]{Corollary}
\newtheorem*{cor*}{Corollary}
\newtheorem{defprop}[thm]{Definition-Proposition}
\newtheorem*{punchline}{Punchline}

\theoremstyle{definition}
\newtheorem{defn}{Definition}[section]
\newtheorem*{defn*}{Definition}
\newtheorem{exmp}{Example}[section]
\newtheorem*{exmp*}{Example}
\newtheorem{asspt}{Assumption}[section]
\newtheorem{notation}{Notation}[section]
\newtheorem{exercise}{Exercise}[section]
\newtheorem*{fact*}{Fact}
\newtheorem*{rmk*}{Remark}
\newtheorem{fact}{Fact}
\newtheorem*{aside}{Aside}
\newtheorem*{question}{Question}
\newtheorem*{answer}{Answer}

\else\relax\fi

%>>>>>>>>>>>>>>>>>>>>>>>>>>>>>>
%<<<      Fields, etc.      <<<
%>>>>>>>>>>>>>>>>>>>>>>>>>>>>>>
\DeclareSymbolFont{AMSb}{U}{msb}{m}{n}
\DeclareMathSymbol{\N}{\mathbin}{AMSb}{"4E}
\DeclareMathSymbol{\Octonions}{\mathbin}{AMSb}{"4F}
\DeclareMathSymbol{\Z}{\mathbin}{AMSb}{"5A}
\DeclareMathSymbol{\R}{\mathbin}{AMSb}{"52}
\DeclareMathSymbol{\Q}{\mathbin}{AMSb}{"51}
\DeclareMathSymbol{\PP}{\mathbin}{AMSb}{"50}
\DeclareMathSymbol{\I}{\mathbin}{AMSb}{"49}
\DeclareMathSymbol{\C}{\mathbin}{AMSb}{"43}
\DeclareMathSymbol{\A}{\mathbin}{AMSb}{"41}
\DeclareMathSymbol{\F}{\mathbin}{AMSb}{"46}
\DeclareMathSymbol{\Quaternions}{\mathbin}{AMSb}{"48}


%>>>>>>>>>>>>>>>>>>>>>>>>>>>>>>
%<<<       Operators        <<<
%>>>>>>>>>>>>>>>>>>>>>>>>>>>>>>
\DeclareMathOperator{\ad}{\textbf{ad}}
\DeclareMathOperator{\coker}{coker}
\renewcommand{\ker}{\textup{ker}\,}
\DeclareMathOperator{\End}{End}
\DeclareMathOperator{\Aut}{Aut}
\DeclareMathOperator{\Hom}{Hom}
\DeclareMathOperator{\Maps}{Maps}
\DeclareMathOperator{\Mor}{Mor}
\DeclareMathOperator{\Gal}{Gal}
\DeclareMathOperator{\Ext}{Ext}
\DeclareMathOperator{\Tor}{Tor}
\DeclareMathOperator{\Map}{Map}
\DeclareMathOperator{\Der}{Der}
\DeclareMathOperator{\Rad}{Rad}
\DeclareMathOperator{\rank}{rank}
\DeclareMathOperator{\ArfInvariant}{Arf}
\DeclareMathOperator{\KervaireInvariant}{Ker}
\DeclareMathOperator{\im}{im}
\DeclareMathOperator{\coim}{coim}
\DeclareMathOperator{\trace}{tr}
\DeclareMathOperator{\supp}{supp}
\DeclareMathOperator{\ann}{ann}
\DeclareMathOperator{\spec}{Spec}
\DeclareMathOperator{\proj}{Proj}
\DeclareMathOperator{\fiber}{F}
\DeclareMathOperator{\cofiber}{C}
\DeclareMathOperator{\cone}{cone}
\DeclareMathOperator{\Skel}{Sk}
\DeclareMathOperator{\conn}{conn}
\DeclareMathOperator{\colim}{colim}
\DeclareMathOperator{\limit}{lim}

%>>>>>>>>>>>>>>>>>>>>>>>>>>>>>>
%<<<   Cohomology Theories  <<<
%>>>>>>>>>>>>>>>>>>>>>>>>>>>>>>
\DeclareMathOperator{\KR}{{K\R}}
\DeclareMathOperator{\KO}{{KO}}
\DeclareMathOperator{\K}{{K}}
\DeclareMathOperator{\OmegaO}{{\Omega_{\Octonions}}}

%>>>>>>>>>>>>>>>>>>>>>>>>>>>>>>
%<<<   Algebraic Geometry   <<<
%>>>>>>>>>>>>>>>>>>>>>>>>>>>>>>
\DeclareMathOperator{\Spec}{Spec}
\DeclareMathOperator{\Proj}{Proj}
\DeclareMathOperator{\Sing}{Sing}
\DeclareMathOperator{\shfHom}{\mathscr{H}\textit{\!\!om}}
\newcommand{\WeilDivisors}{\textup{Div}}
\newcommand{\CartierDivisors}{\textup{CaDiv}}
\newcommand{\PrincipalWeilDivisors}{\textup{PrDiv}}
\newcommand{\LocallyPrincipalWeilDivisors}{\textup{LPDiv}}
\newcommand{\PrincipalCartierDivisors}{\textup{PrCaDiv}}
\newcommand{\DivisorClass}{\textup{Cl}}
\newcommand{\CartierClass}{\textup{CaCl}}
\newcommand{\Picard}{\textup{Pic}}
\DeclareMathOperator{\Frob}{Frob}


%>>>>>>>>>>>>>>>>>>>>>>>>>>>>>>
%<<<  Mathematical Objects  <<<
%>>>>>>>>>>>>>>>>>>>>>>>>>>>>>>
\newcommand{\sll}{\mathfrak{sl}}
\newcommand{\gl}{\mathfrak{gl}}
\newcommand{\GL}{\mbox{GL}}
\newcommand{\PGL}{\mbox{PGL}}
\newcommand{\SL}{\mbox{SL}}
\newcommand{\Mat}{\mbox{Mat}}
\newcommand{\Gr}{\textup{Gr}}
\newcommand{\Squ}{\textup{Sq}}
\newcommand{\catSet}{\textit{Sets}}
\newcommand{\RP}{{\R\PP}}
\newcommand{\CP}{{\C\PP}}
\newcommand{\Steen}{\mathscr{A}}
\newcommand{\Orth}{\textup{\textbf{O}}}

%>>>>>>>>>>>>>>>>>>>>>>>>>>>>>>
%<<<  Mathematical Symbols  <<<
%>>>>>>>>>>>>>>>>>>>>>>>>>>>>>>
\newcommand{\DASH}{\textup{---}}
\newcommand{\op}{\textup{op}}
\newcommand{\ob}{\textup{ob}\,}
\newcommand{\ho}{\textup{ho}}
\newcommand{\st}{\textup{st}}
\newcommand{\id}{\textup{id}}
\newcommand{\Bullet}{\ensuremath{\bullet} }

%>>>>>>>>>>>>>>>>>>>>>>>>>>>>>>
%<<<      Some Arrows       <<<
%>>>>>>>>>>>>>>>>>>>>>>>>>>>>>>
\let\shortmapsto\mapsto
\let\mapsto\longmapsto
\newcommand{\mapsfrom}{\,\reflectbox{$\mapsto$}\ }
\newcommand{\bigrightsquig}{\scalebox{2}{\ensuremath{\rightsquigarrow}}}
\newcommand{\bigleftsquig}{\reflectbox{\scalebox{2}{\ensuremath{\rightsquigarrow}}}}

%\newcommand{\cofibration}{\xhookrightarrow{\phantom{\ \,{\sim\!}\ \ }}}
%\newcommand{\fibration}{\xtwoheadrightarrow{\phantom{\sim\!}}}
%\newcommand{\acycliccofibration}{\xhookrightarrow{\ \,{\sim\!}\ \ }}
%\newcommand{\acyclicfibration}{\xtwoheadrightarrow{\sim\!}}
%\newcommand{\leftcofibration}{\xhookleftarrow{\phantom{\ \,{\sim\!}\ \ }}}
%\newcommand{\leftfibration}{\xtwoheadleftarrow{\phantom{\sim\!}}}
%\newcommand{\leftacycliccofibration}{\xhookleftarrow{\ \ {\sim\!}\,\ }}
%\newcommand{\leftacyclicfibration}{\xtwoheadleftarrow{\sim\!}}
%\newcommand{\weakequiv}{\xrightarrow{\ \,\sim\,\ }}
%\newcommand{\leftweakequiv}{\xleftarrow{\ \,\sim\,\ }}

\newcommand{\cofibration}
{\xhookrightarrow{\phantom{\ \,{\raisebox{-.3ex}[0ex][0ex]{\scriptsize$\sim$}\!}\ \ }}}
\newcommand{\fibration}
{\xtwoheadrightarrow{\phantom{\raisebox{-.3ex}[0ex][0ex]{\scriptsize$\sim$}\!}}}
\newcommand{\acycliccofibration}
{\xhookrightarrow{\ \,{\raisebox{-.55ex}[0ex][0ex]{\scriptsize$\sim$}\!}\ \ }}
\newcommand{\acyclicfibration}
{\xtwoheadrightarrow{\raisebox{-.6ex}[0ex][0ex]{\scriptsize$\sim$}\!}}
\newcommand{\leftcofibration}
{\xhookleftarrow{\phantom{\ \,{\raisebox{-.3ex}[0ex][0ex]{\scriptsize$\sim$}\!}\ \ }}}
\newcommand{\leftfibration}
{\xtwoheadleftarrow{\phantom{\raisebox{-.3ex}[0ex][0ex]{\scriptsize$\sim$}\!}}}
\newcommand{\leftacycliccofibration}
{\xhookleftarrow{\ \ {\raisebox{-.55ex}[0ex][0ex]{\scriptsize$\sim$}\!}\,\ }}
\newcommand{\leftacyclicfibration}
{\xtwoheadleftarrow{\raisebox{-.6ex}[0ex][0ex]{\scriptsize$\sim$}\!}}
\newcommand{\weakequiv}
{\xrightarrow{\ \,\raisebox{-.3ex}[0ex][0ex]{\scriptsize$\sim$}\,\ }}
\newcommand{\leftweakequiv}
{\xleftarrow{\ \,\raisebox{-.3ex}[0ex][0ex]{\scriptsize$\sim$}\,\ }}


%>>>>>>>>>>>>>>>>>>>>>>>>>>>>>>
%<<<     Greek Letters      <<<
%>>>>>>>>>>>>>>>>>>>>>>>>>>>>>>
%\newcommand{\oldphi}{\phi}
%\renewcommand{\phi}{\varphi}
\let\oldphi\phi
\let\phi\varphi
\renewcommand{\to}{\longrightarrow}
\newcommand{\eps}{\varepsilon}

%>>>>>>>>>>>>>>>>>>>>>>>>>>>>>>
%<<<  1st-4th & parentheses <<<
%>>>>>>>>>>>>>>>>>>>>>>>>>>>>>>
\newcommand{\first}{^\text{st}}
\newcommand{\second}{^\text{nd}}
\newcommand{\third}{^\text{rd}}
\newcommand{\fourth}{^\text{th}}
\newcommand{\ZEROTH}{$0^\text{th}$ }
\newcommand{\FIRST}{$1^\text{st}$ }
\newcommand{\SECOND}{$2^\text{nd}$ }
\newcommand{\THIRD}{$3^\text{rd}$ }
\newcommand{\FOURTH}{$4^\text{th}$ }
\newcommand{\iTH}{$i^\text{th}$ }
\newcommand{\jTH}{$j^\text{th}$ }
\newcommand{\nTH}{$n^\text{th}$ }

%>>>>>>>>>>>>>>>>>>>>>>>>>>>>>>
%<<<    upright commands    <<<
%>>>>>>>>>>>>>>>>>>>>>>>>>>>>>>
\newcommand{\upcol}{\textup{:}}
\newcommand{\upsemi}{\textup{;}}
\providecommand{\lparen}{\textup{(}}
\providecommand{\rparen}{\textup{)}}
\renewcommand{\lparen}{\textup{(}}
\renewcommand{\rparen}{\textup{)}}
\newcommand{\Iff}{\emph{iff} }

%>>>>>>>>>>>>>>>>>>>>>>>>>>>>>>
%<<<     Environments       <<<
%>>>>>>>>>>>>>>>>>>>>>>>>>>>>>>
\newcommand{\squishlist}
{ %\setlength{\topsep}{100pt} doesn't seem to do anything.
  \setlength{\itemsep}{.5pt}
  \setlength{\parskip}{0pt}
  \setlength{\parsep}{0pt}}
\newenvironment{itemise}{
\begin{list}{\textup{$\rightsquigarrow$}}
   {  \setlength{\topsep}{1mm}
      \setlength{\itemsep}{1pt}
      \setlength{\parskip}{0pt}
      \setlength{\parsep}{0pt}
   }
}{\end{list}\vspace{-.1cm}}
\newcommand{\INDENT}{\textbf{}\phantom{space}}
\renewcommand{\INDENT}{\rule{.7cm}{0cm}}

\newcommand{\itm}[1][$\rightsquigarrow$]{\item[{\makebox[.5cm][c]{\textup{#1}}}]}

\newcommand{\rednote}[1]{{\color{red}#1}\scalebox{.1}{rednote}}
\newcommand{\bluenote}[1]{{\color{blue}#1}\scalebox{.1}{rednote}}
\newcommand{\funcdef}[4]{\begin{align*}
#1&\to #2\\
#3&\mapsto#4
\end{align*}}

%\newcommand{\comment}[1]{}

%>>>>>>>>>>>>>>>>>>>>>>>>>>>>>>
%<<<       Categories       <<<
%>>>>>>>>>>>>>>>>>>>>>>>>>>>>>>
\newcommand{\Ens}{{\mathscr{E}ns}}
\DeclareMathOperator{\Sheaves}{{\mathsf{Shf}}}
\DeclareMathOperator{\Presheaves}{{\mathsf{PreShf}}}
\DeclareMathOperator{\Varieties}{{\mathsf{Var}}}
\DeclareMathOperator{\Schemes}{{\mathsf{Sch}}}
\DeclareMathOperator{\Rings}{{\mathsf{Rings}}}
\DeclareMathOperator{\AbGp}{{\mathsf{AbGp}}}
\DeclareMathOperator{\Modules}{{\mathsf{\!-Mod}}}
\DeclareMathOperator{\QuasiCoherent}{{\mathsf{QCoh}}}
\DeclareMathOperator{\Coherent}{{\mathsf{Coh}}}
\DeclareMathOperator{\GSW}{{\mathcal{SW}^G}}
\DeclareMathOperator{\Burnside}{{\mathsf{Burn}}}
\DeclareMathOperator{\GSet}{{G\mathsf{Set}}}
\DeclareMathOperator{\FinGSet}{{G\mathsf{Set}^\textup{fin}}}
\DeclareMathOperator{\HSet}{{H\mathsf{Set}}}
\DeclareMathOperator{\Cat}{{\mathsf{Cat}}}
\DeclareMathOperator{\Orb}{{\mathsf{Orb}}}
\DeclareMathOperator{\Set}{{\mathsf{Set}}}
\DeclareMathOperator{\sSet}{{\mathsf{sSet}}}
\DeclareMathOperator{\Top}{{\mathsf{Top}}}
\DeclareMathOperator{\GSpectra}{{G-\mathsf{Spectra}}}

%>>>>>>>>>>>>>>>>>>>>>>>>>>>>>>
%<<<     Script Letters     <<<
%>>>>>>>>>>>>>>>>>>>>>>>>>>>>>>
\newcommand{\scrQ}{\mathscr{Q}}
\newcommand{\scrW}{\mathscr{W}}
\newcommand{\scrE}{\mathscr{E}}
\newcommand{\scrR}{\mathscr{R}}
\newcommand{\scrT}{\mathscr{T}}
\newcommand{\scrY}{\mathscr{Y}}
\newcommand{\scrU}{\mathscr{U}}
\newcommand{\scrI}{\mathscr{I}}
\newcommand{\scrO}{\mathscr{O}}
\newcommand{\scrP}{\mathscr{P}}
\newcommand{\scrA}{\mathscr{A}}
\newcommand{\scrS}{\mathscr{S}}
\newcommand{\scrD}{\mathscr{D}}
\newcommand{\scrF}{\mathscr{F}}
\newcommand{\scrG}{\mathscr{G}}
\newcommand{\scrH}{\mathscr{H}}
\newcommand{\scrJ}{\mathscr{J}}
\newcommand{\scrK}{\mathscr{K}}
\newcommand{\scrL}{\mathscr{L}}
\newcommand{\scrZ}{\mathscr{Z}}
\newcommand{\scrX}{\mathscr{X}}
\newcommand{\scrC}{\mathscr{C}}
\newcommand{\scrV}{\mathscr{V}}
\newcommand{\scrB}{\mathscr{B}}
\newcommand{\scrN}{\mathscr{N}}
\newcommand{\scrM}{\mathscr{M}}

%>>>>>>>>>>>>>>>>>>>>>>>>>>>>>>
%<<<     Fractur Letters    <<<
%>>>>>>>>>>>>>>>>>>>>>>>>>>>>>>
\newcommand{\frakQ}{\mathfrak{Q}}
\newcommand{\frakW}{\mathfrak{W}}
\newcommand{\frakE}{\mathfrak{E}}
\newcommand{\frakR}{\mathfrak{R}}
\newcommand{\frakT}{\mathfrak{T}}
\newcommand{\frakY}{\mathfrak{Y}}
\newcommand{\frakU}{\mathfrak{U}}
\newcommand{\frakI}{\mathfrak{I}}
\newcommand{\frakO}{\mathfrak{O}}
\newcommand{\frakP}{\mathfrak{P}}
\newcommand{\frakA}{\mathfrak{A}}
\newcommand{\frakS}{\mathfrak{S}}
\newcommand{\frakD}{\mathfrak{D}}
\newcommand{\frakF}{\mathfrak{F}}
\newcommand{\frakG}{\mathfrak{G}}
\newcommand{\frakH}{\mathfrak{H}}
\newcommand{\frakJ}{\mathfrak{J}}
\newcommand{\frakK}{\mathfrak{K}}
\newcommand{\frakL}{\mathfrak{L}}
\newcommand{\frakZ}{\mathfrak{Z}}
\newcommand{\frakX}{\mathfrak{X}}
\newcommand{\frakC}{\mathfrak{C}}
\newcommand{\frakV}{\mathfrak{V}}
\newcommand{\frakB}{\mathfrak{B}}
\newcommand{\frakN}{\mathfrak{N}}
\newcommand{\frakM}{\mathfrak{M}}

\newcommand{\frakq}{\mathfrak{q}}
\newcommand{\frakw}{\mathfrak{w}}
\newcommand{\frake}{\mathfrak{e}}
\newcommand{\frakr}{\mathfrak{r}}
\newcommand{\frakt}{\mathfrak{t}}
\newcommand{\fraky}{\mathfrak{y}}
\newcommand{\fraku}{\mathfrak{u}}
\newcommand{\fraki}{\mathfrak{i}}
\newcommand{\frako}{\mathfrak{o}}
\newcommand{\frakp}{\mathfrak{p}}
\newcommand{\fraka}{\mathfrak{a}}
\newcommand{\fraks}{\mathfrak{s}}
\newcommand{\frakd}{\mathfrak{d}}
\newcommand{\frakf}{\mathfrak{f}}
\newcommand{\frakg}{\mathfrak{g}}
\newcommand{\frakh}{\mathfrak{h}}
\newcommand{\frakj}{\mathfrak{j}}
\newcommand{\frakk}{\mathfrak{k}}
\newcommand{\frakl}{\mathfrak{l}}
\newcommand{\frakz}{\mathfrak{z}}
\newcommand{\frakx}{\mathfrak{x}}
\newcommand{\frakc}{\mathfrak{c}}
\newcommand{\frakv}{\mathfrak{v}}
\newcommand{\frakb}{\mathfrak{b}}
\newcommand{\frakn}{\mathfrak{n}}
\newcommand{\frakm}{\mathfrak{m}}

%>>>>>>>>>>>>>>>>>>>>>>>>>>>>>>
%<<<  Caligraphic Letters   <<<
%>>>>>>>>>>>>>>>>>>>>>>>>>>>>>>
\newcommand{\calQ}{\mathcal{Q}}
\newcommand{\calW}{\mathcal{W}}
\newcommand{\calE}{\mathcal{E}}
\newcommand{\calR}{\mathcal{R}}
\newcommand{\calT}{\mathcal{T}}
\newcommand{\calY}{\mathcal{Y}}
\newcommand{\calU}{\mathcal{U}}
\newcommand{\calI}{\mathcal{I}}
\newcommand{\calO}{\mathcal{O}}
\newcommand{\calP}{\mathcal{P}}
\newcommand{\calA}{\mathcal{A}}
\newcommand{\calS}{\mathcal{S}}
\newcommand{\calD}{\mathcal{D}}
\newcommand{\calF}{\mathcal{F}}
\newcommand{\calG}{\mathcal{G}}
\newcommand{\calH}{\mathcal{H}}
\newcommand{\calJ}{\mathcal{J}}
\newcommand{\calK}{\mathcal{K}}
\newcommand{\calL}{\mathcal{L}}
\newcommand{\calZ}{\mathcal{Z}}
\newcommand{\calX}{\mathcal{X}}
\newcommand{\calC}{\mathcal{C}}
\newcommand{\calV}{\mathcal{V}}
\newcommand{\calB}{\mathcal{B}}
\newcommand{\calN}{\mathcal{N}}
\newcommand{\calM}{\mathcal{M}}

%>>>>>>>>>>>>>>>>>>>>>>>>>>>>>>
%<<<<<<<<<DEPRECIATED<<<<<<<<<<
%>>>>>>>>>>>>>>>>>>>>>>>>>>>>>>

%%% From Kac's template
% 1-inch margins, from fullpage.sty by H.Partl, Version 2, Dec. 15, 1988.
%\topmargin 0pt
%\advance \topmargin by -\headheight
%\advance \topmargin by -\headsep
%\textheight 9.1in
%\oddsidemargin 0pt
%\evensidemargin \oddsidemargin
%\marginparwidth 0.5in
%\textwidth 6.5in
%
%\parindent 0in
%\parskip 1.5ex
%%\renewcommand{\baselinestretch}{1.25}

%%% From the net
%\newcommand{\pullbackcorner}[1][dr]{\save*!/#1+1.2pc/#1:(1,-1)@^{|-}\restore}
%\newcommand{\pushoutcorner}[1][dr]{\save*!/#1-1.2pc/#1:(-1,1)@^{|-}\restore}









\usepackage{fancyhdr}

\excludeversion{SaulJuvitop}
\includeversion{AlexandreBausfieldLocalisation}

\rfoot{\footnotesize Michael Donovan}
\newcommand{\KanSemResponse}[1]
{
\thispagestyle{fancy}
\subsection*{#1}
}

\begin{document}
\begin{SaulJuvitop}
\KanSemResponse
{``Simplicial Localisation'' --- Saul Glasman --- 6/10/2011}
\begin{abstract}
The message of simplicial localisation is that trying to invert 
morphisms in a category catapults the would-be-inverter into homotopy 
theory, whether they want to be there or not. This accounts, in part, 
for the ubiquity of homotopical concepts in modern mathematics. First 
I'll give a brief and bracing refresher on simplicial sets for those 
whose heads are not yet simplicial. I'll discuss two perspectives on 
simplicial localisation, first presenting the useful and picturesque 
hammock localisation and then teaching you how to take a free resolution 
of a category. I'll make a few remarks on how great this is and apply it 
to the theory of model categories. Most of the material on this talk is 
based on three seminal 1980 papers by Dwyer and Kan.
\end{abstract}
Group completion \rednote{(do it levelwise?)}, classifying spaces, operads, etc.\ are all better in the category of simplicial sets. We have a Quillen adjunction:
\[|\DASH|:\sSet\longleftrightarrow\Top:\Sing\]
These form a Quillen equivalence.
\begin{exmp*}
The nerve of a category $\calC$ is the simplicial set whose $k$-simplices are chains of $k$ composable morphisms. Forming the realisation of the nerve of a group (as a category) you get the classifying space.
\end{exmp*}
\textbf{Drawback:} Not everything is fibrant (unlike in $\Top$).

This is bad, for example if you'd like to calculate, say, self maps of $S^1$, if we view $S^1$ simply as the boundary of the $2$-simplex.
The fibrant objects in $\sSet$ are the Kan complexes, ``characterised by the fact that they are enormous''.
\subsection*{Simplicial Localisation}
Given a category $\calC$ and a class $\scrW$ of ``weak equivalences'' in $\calC$. $\scrW$ should have the 2 of 3 property, and maybe some other stuff. We want to localise $\calC\to\calC[\scrW^{-1}]$ with the obvious universal property.

Let's write a candidate for $\calC[\scrW^{-1}]$. The objects are the same, while the maps are equivalence classes of zigzags (\rednote{under some relation} where we at least allow ourselves to add in pairs of identities):
\[\xymatrix{
X\ar[r]&A_0&\ar[l]_\sim A_1\ar[r]&A_2&\ar[l]_\sim\cdots\ar[r]&A_{2n}&A_{2n+1}\ar[l]_\sim\ar[r]&Y
}\]
There are computational problems here. The reason for this difficulty is that we've skipped a step.
\begin{defn*}
A simplicial category is one in which $\Map(X,Y)$ is a simplicial set, composition is a morphism of simplicial sets. Actually we could just say that a simplicial category is a functor $\Delta^\op\to\Set$, such that for any $I,J\in\Delta$, $S(I)$ and $S(J)$ have the same objects, and $S(I)\to S(J)$ is the identity on objects. The morphisms are free to do as they please.
\end{defn*}
There's a universal simplicial category $L^\scrW C$ in which $\scrW$ becomes invertible. I.e.\ if $\calD$ is a simplicial category, and we have a functor sending $\scrW$ to isomorphisms, we get a unique (up to homotopy) factorisation
\[\xymatrix{
\ar[d]\calC\ar[r]&\calD\\
L^\scrW \calC\ar@{..>}[ur]
}\]
In particular, $\calD$ could be $\calC[\scrW^{-1}]$, so we have filled in our missing step.
\[\xymatrix{
\ar[d]\calC\ar[r]&\calC[\scrW^{-1}]\\
L^\scrW \calC\ar@{..>}[ur]
}\]
Moreover, the map on morphisms given by this factorisation is the projection to $\pi_0$ (after we realise the simplicial set of morphisms $L^\scrW \calC(X,Y)$).
\subsection*{Constructing the simplicial localisation}
Suppose we have $X,Y\in\ob\calC$. The $0$-simplices in $\Mor(X,Y)$ are just zig-zags (up to an equivalence relation in which we can add in identities):
\[\xymatrix{
X\ar[r]&A_0&\ar[l]_\sim A_1\ar[r]&A_2&\ar[l]_\sim\cdots\ar[r]&A_{2n}&A_{2n+1}\ar[l]_\sim\ar[r]&Y
}\]
$1$-simplices are going to be diagrams like:
\[\xymatrix@R=.4cm{
&A_0\ar[dd]^\sim&\ar[l]_\sim A_1\ar[dd]^\sim\ar[r]&A_2\ar[dd]^\sim&\ar[l]_\sim\cdots\ar[r]&A_{2n}\ar[dd]^\sim&A_{2n+1}\ar[dd]^\sim\ar[l]_\sim\ar[rd]\\
X\ar[rd]\ar[ur]&&&&&&&Y\\
&B_0&\ar[l]_\sim B_1\ar[r]&B_2&\ar[l]_\sim\cdots\ar[r]&B_{2n}&B_{2n+1}\ar[l]_\sim\ar[ru]
}\]
Higher simplices come from carrying on in this fashion --- adding more and more rows to the diagram. A $k$-simplex will be a hammock of width $k+1$.
Thus $\Mor(X,Y)$ is just the nerve of the category of zig-zags from $X$ to $Y$ and weak equivalences of zig-zags.

\subsection*{In a model category}
I can get a factorisation
\[\xymatrix{
X&\ar@{->>}[l]_\sim X'\ar[r]&Y'\ar@{_{(}->}[r];[]_\sim&Y
}\]
Where $X'$ is cofibrant, and $Y'$ is fibrant. So actually, we only need hammocks as in diag 3.
\subsection*{Another model for the simplicial localisation}
There's a forgetful functor $\Cat\to\mathsf{DirectedGraphs}$, which has a left adjoint, called `Free'. The free category on a directed graph has morphisms given by composable words in the edges of the graph. Now suppose I want to localise $\calC=\text{Free}(V\cup W)$, where $V$ and $W$ are graphs with no shared edges (but probably with shared vertices). Then $\calC[W^{-1}]=\text{AlmostFree}(V\cup W^{\pm1})$:  the morphisms are reduced composable words in the arrows of $V$ and of $W\cup W^{-1}$. That's easy, and if we could replace a category with something like this, we'd win.

Given $\calC$, I can form $F(\calC)$, the free category on $\calC$, by applying forget, then free. $F$ is a monad, and I have natural transformations $FF\to F$ and $1\to F$ satisfying the monoid axioms. So I'll construct a simplicial category $M$, whose category of $k$-simplices is $F^{k+1}(\calC)$, and whose faces and degeneracies are given by the monad maps.

\begin{fact}
$M$ is equivalent to $\calC$ as  simplical category.
\end{fact}
\noindent But now to invert $W$ is achievable levelwise. $M\to M[W^{-1}]$ is the simplicial localisation.
\subsection*{A simplicial model category}
should be a model category enriched in simplicial sets, which already has the `right' space of morphisms, at least between its cofibrant and fibrant objects.
\begin{defn*}
A simplicial model category is a model category, and a simplicial category, which: 
\begin{enumerate}
\item is tensored and cotensored over $\sSet$. [So that given $X\in \calC$ and $K\in\sSet$, can take $X\otimes K$ and $\Hom(K,X)$, and there's an adjunction].
\item Given a cofibration $A\cofibration B$ and a fibration $X\fibration Y$, there is a natural map:
\[c:\Hom(B,X)\to\Hom(A,X)\times_{\Hom(A,Y)}\Hom(B,Y).\]
The codomain is the space of ways to complete the following into a commuting square:
\[\xymatrix@R=.6cm{
A\ar@{^{(}->}[d]&X\ar@{->>}[d]\\
B&Y
}\]
We demand that $c$ is a fibration, and if either $X\fibration Y$ or $A\cofibration B$ is acyclic, that $c$ is an acyclic fibration.
\end{enumerate}
\begin{punchline}
If $\calC$ is a simplicial model category, and $\calC'$ is the underlying category (take $0$-simplices as morphisms), then $\calC$ and $L^\scrW(\calC')$ are equivalent and simplicial categories.\footnote{$\scrW$ is the $0$-simplices that are weak equivalences.}
\end{punchline}
\subsection*{Post talk ramblings}
A weak equivalence of simplicial categories is a zig-zag of (simplicial) functors $F_i:\calC_i\to\calC_{i+1}$ such that $\pi_0F_i$ is a weak equivalence (of spaces), and for each $X,Y\in\ob(\calC_i)$, $\Mor(X,Y)\to\Mor(F_iX,F_iY)$ is a weak equivalence in the model category of simplicial sets.
\end{defn*}


\pagebreak
\end{SaulJuvitop}
\begin{AlexandreBausfieldLocalisation}
\KanSemResponse
{``Localization of Model Categories'' --- Luis Alexandre Pereira --- 13/10/2011}
\begin{abstract}
The problem of localizing a model category $M$ is that of enlarging the 
class of weak equivalences to include an additonal set $S$ of maps while 
still obtaining a model structure $S^{-1}M$, preferably one that can be 
related to the original model structure on $M$ (meaning a Quillen 
adjunction is desirable). While such a localization needs not always 
exist, there are certain hairy technical conditions (which nonetheless 
hold in most model categories of interest) on $M$ that always allow one to 
form $S^{-1}M$.
This is the so-called left Bousfield localization, whose base category 
is $M$ itself and whose cofibrations are precisely those of $M$, with an 
enlarged class of weak equivalences.

After discussing this process and 
some general ideas of how and why it works (this will include a rare 
sighting of the definition of the ubiquitous Reedy model structures of 
diagrams), we will illustrate it's importance with a wealth of examples: 
Bousfield's localization of spaces with respect to homology (the primal 
example), Dugger's hocolim localization of $sM$, showing a wide class of 
model categories can be made simplicial, Resk's complete segal spaces (a 
model for $(\infty,1)$-categories) obtained from localizing simplicial spaces, 
and Dugger's theory of presentations of model categories, which are 
localizations of the so-called universal model categories.
\end{abstract}


\pagebreak
\end{AlexandreBausfieldLocalisation}
\end{document}




























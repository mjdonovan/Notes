% !TEX root = z_output/conn_simp_algs_are_complete.tex
\documentclass[11pt]{amsart}
\usepackage{fullpage}
\usepackage{amsmath,amsthm,amssymb}
\usepackage{mathrsfs,nicefrac}
\usepackage{amssymb}
\usepackage{epsfig}
\usepackage[all,2cell]{xy}
\usepackage{sseq}
\usepackage{tocloft}
\usepackage{cancel}
\usepackage[strict]{changepage}
\usepackage{color}
\usepackage{tikz}
\usepackage{extpfeil}
\usepackage{version}
\usepackage{framed}
\definecolor{shadecolor}{rgb}{.925,0.925,0.925}

%\usepackage{ifthen}
%Used for disabling hyperref
\ifx\dontloadhyperref\undefined
%\usepackage[pdftex,pdfborder={0 0 0 [1 1]}]{hyperref}
\usepackage[pdftex,pdfborder={0 0 .5 [1 1]}]{hyperref}
\else
\providecommand{\texorpdfstring}[2]{#1}
\fi
%>>>>>>>>>>>>>>>>>>>>>>>>>>>>>>
%<<<        Versions        <<<
%>>>>>>>>>>>>>>>>>>>>>>>>>>>>>>
%Add in the following line to include all the versions.
%\def\excludeversion#1{\includeversion{#1}}

%>>>>>>>>>>>>>>>>>>>>>>>>>>>>>>
%<<<       Better ToC       <<<
%>>>>>>>>>>>>>>>>>>>>>>>>>>>>>>
\setlength{\cftbeforesecskip}{0.5ex}

%>>>>>>>>>>>>>>>>>>>>>>>>>>>>>>
%<<<      Hyperref mod      <<<
%>>>>>>>>>>>>>>>>>>>>>>>>>>>>>>

%needs more testing
\newcounter{dummyforrefstepcounter}
\newcommand{\labelRIGHTHERE}[1]
{\refstepcounter{dummyforrefstepcounter}\label{#1}}


%>>>>>>>>>>>>>>>>>>>>>>>>>>>>>>
%<<<  Theorem Environments  <<<
%>>>>>>>>>>>>>>>>>>>>>>>>>>>>>>
\ifx\dontloaddefinitionsoftheoremenvironments\undefined
\theoremstyle{plain}
\newtheorem{thm}{Theorem}[section]
\newtheorem*{thm*}{Theorem}
\newtheorem{lem}[thm]{Lemma}
\newtheorem*{lem*}{Lemma}
\newtheorem{prop}[thm]{Proposition}
\newtheorem*{prop*}{Proposition}
\newtheorem{cor}[thm]{Corollary}
\newtheorem*{cor*}{Corollary}
\newtheorem{defprop}[thm]{Definition-Proposition}
\newtheorem*{punchline}{Punchline}
\newtheorem*{conjecture}{Conjecture}
\newtheorem*{claim}{Claim}

\theoremstyle{definition}
\newtheorem{defn}{Definition}[section]
\newtheorem*{defn*}{Definition}
\newtheorem{exmp}{Example}[section]
\newtheorem*{exmp*}{Example}
\newtheorem*{exmps*}{Examples}
\newtheorem*{nonexmp*}{Non-example}
\newtheorem{asspt}{Assumption}[section]
\newtheorem{notation}{Notation}[section]
\newtheorem{exercise}{Exercise}[section]
\newtheorem*{fact*}{Fact}
\newtheorem*{rmk*}{Remark}
\newtheorem{fact}{Fact}
\newtheorem*{aside}{Aside}
\newtheorem*{question}{Question}
\newtheorem*{answer}{Answer}

\else\relax\fi

%>>>>>>>>>>>>>>>>>>>>>>>>>>>>>>
%<<<      Fields, etc.      <<<
%>>>>>>>>>>>>>>>>>>>>>>>>>>>>>>
\DeclareSymbolFont{AMSb}{U}{msb}{m}{n}
\DeclareMathSymbol{\N}{\mathbin}{AMSb}{"4E}
\DeclareMathSymbol{\Octonions}{\mathbin}{AMSb}{"4F}
\DeclareMathSymbol{\Z}{\mathbin}{AMSb}{"5A}
\DeclareMathSymbol{\R}{\mathbin}{AMSb}{"52}
\DeclareMathSymbol{\Q}{\mathbin}{AMSb}{"51}
\DeclareMathSymbol{\PP}{\mathbin}{AMSb}{"50}
\DeclareMathSymbol{\I}{\mathbin}{AMSb}{"49}
\DeclareMathSymbol{\C}{\mathbin}{AMSb}{"43}
\DeclareMathSymbol{\A}{\mathbin}{AMSb}{"41}
\DeclareMathSymbol{\F}{\mathbin}{AMSb}{"46}
\DeclareMathSymbol{\G}{\mathbin}{AMSb}{"47}
\DeclareMathSymbol{\Quaternions}{\mathbin}{AMSb}{"48}


%>>>>>>>>>>>>>>>>>>>>>>>>>>>>>>
%<<<       Operators        <<<
%>>>>>>>>>>>>>>>>>>>>>>>>>>>>>>
\DeclareMathOperator{\ad}{\textbf{ad}}
\DeclareMathOperator{\coker}{coker}
\renewcommand{\ker}{\textup{ker}\,}
\DeclareMathOperator{\End}{End}
\DeclareMathOperator{\Aut}{Aut}
\DeclareMathOperator{\Hom}{Hom}
\DeclareMathOperator{\Maps}{Maps}
\DeclareMathOperator{\Mor}{Mor}
\DeclareMathOperator{\Gal}{Gal}
\DeclareMathOperator{\Ext}{Ext}
\DeclareMathOperator{\Tor}{Tor}
\DeclareMathOperator{\Map}{Map}
\DeclareMathOperator{\Der}{Der}
\DeclareMathOperator{\Rad}{Rad}
\DeclareMathOperator{\rank}{rank}
\DeclareMathOperator{\ArfInvariant}{Arf}
\DeclareMathOperator{\KervaireInvariant}{Ker}
\DeclareMathOperator{\im}{im}
\DeclareMathOperator{\coim}{coim}
\DeclareMathOperator{\trace}{tr}
\DeclareMathOperator{\supp}{supp}
\DeclareMathOperator{\ann}{ann}
\DeclareMathOperator{\spec}{Spec}
\DeclareMathOperator{\SPEC}{\textbf{Spec}}
\DeclareMathOperator{\proj}{Proj}
\DeclareMathOperator{\PROJ}{\textbf{Proj}}
\DeclareMathOperator{\fiber}{F}
\DeclareMathOperator{\cofiber}{C}
\DeclareMathOperator{\cone}{cone}
\DeclareMathOperator{\skel}{sk}
\DeclareMathOperator{\coskel}{cosk}
\DeclareMathOperator{\conn}{conn}
\DeclareMathOperator{\colim}{colim}
\DeclareMathOperator{\limit}{lim}
\DeclareMathOperator{\ch}{ch}
\DeclareMathOperator{\Vect}{Vect}
\DeclareMathOperator{\GrthGrp}{GrthGp}
\DeclareMathOperator{\Sym}{Sym}
\DeclareMathOperator{\Prob}{\mathbb{P}}
\DeclareMathOperator{\Exp}{\mathbb{E}}
\DeclareMathOperator{\GeomMean}{\mathbb{G}}
\DeclareMathOperator{\Var}{Var}
\DeclareMathOperator{\Cov}{Cov}
\DeclareMathOperator{\Sp}{Sp}
\DeclareMathOperator{\Seq}{Seq}
\DeclareMathOperator{\Cyl}{Cyl}
\DeclareMathOperator{\Ev}{Ev}
\DeclareMathOperator{\sh}{sh}
\DeclareMathOperator{\intHom}{\underline{Hom}}
\DeclareMathOperator{\Frac}{frac}



%>>>>>>>>>>>>>>>>>>>>>>>>>>>>>>
%<<<   Cohomology Theories  <<<
%>>>>>>>>>>>>>>>>>>>>>>>>>>>>>>
\DeclareMathOperator{\KR}{{K\R}}
\DeclareMathOperator{\KO}{{KO}}
\DeclareMathOperator{\K}{{K}}
\DeclareMathOperator{\OmegaO}{{\Omega_{\Octonions}}}

%>>>>>>>>>>>>>>>>>>>>>>>>>>>>>>
%<<<   Algebraic Geometry   <<<
%>>>>>>>>>>>>>>>>>>>>>>>>>>>>>>
\DeclareMathOperator{\Spec}{Spec}
\DeclareMathOperator{\Proj}{Proj}
\DeclareMathOperator{\Sing}{Sing}
\DeclareMathOperator{\shfHom}{\mathscr{H}\textit{\!\!om}}
\DeclareMathOperator{\WeilDivisors}{{Div}}
\DeclareMathOperator{\CartierDivisors}{{CaDiv}}
\DeclareMathOperator{\PrincipalWeilDivisors}{{PrDiv}}
\DeclareMathOperator{\LocallyPrincipalWeilDivisors}{{LPDiv}}
\DeclareMathOperator{\PrincipalCartierDivisors}{{PrCaDiv}}
\DeclareMathOperator{\DivisorClass}{{Cl}}
\DeclareMathOperator{\CartierClass}{{CaCl}}
\DeclareMathOperator{\Picard}{{Pic}}
\DeclareMathOperator{\Frob}{Frob}


%>>>>>>>>>>>>>>>>>>>>>>>>>>>>>>
%<<<  Mathematical Objects  <<<
%>>>>>>>>>>>>>>>>>>>>>>>>>>>>>>
\newcommand{\sll}{\mathfrak{sl}}
\newcommand{\gl}{\mathfrak{gl}}
\newcommand{\GL}{\mbox{GL}}
\newcommand{\PGL}{\mbox{PGL}}
\newcommand{\SL}{\mbox{SL}}
\newcommand{\Mat}{\mbox{Mat}}
\newcommand{\Gr}{\textup{Gr}}
\newcommand{\Squ}{\textup{Sq}}
\newcommand{\catSet}{\textit{Sets}}
\newcommand{\RP}{{\R\PP}}
\newcommand{\CP}{{\C\PP}}
\newcommand{\Steen}{\mathscr{A}}
\newcommand{\Orth}{\textup{\textbf{O}}}

%>>>>>>>>>>>>>>>>>>>>>>>>>>>>>>
%<<<  Mathematical Symbols  <<<
%>>>>>>>>>>>>>>>>>>>>>>>>>>>>>>
\newcommand{\DASH}{\textup{---}}
\newcommand{\op}{\textup{op}}
\newcommand{\CW}{\textup{CW}}
\newcommand{\ob}{\textup{ob}\,}
\newcommand{\ho}{\textup{ho}}
\newcommand{\st}{\textup{st}}
\newcommand{\id}{\textup{id}}
\newcommand{\Bullet}{\ensuremath{\bullet} }
\newcommand{\sprod}{\wedge}

%>>>>>>>>>>>>>>>>>>>>>>>>>>>>>>
%<<<      Some Arrows       <<<
%>>>>>>>>>>>>>>>>>>>>>>>>>>>>>>
\newcommand{\nt}{\Longrightarrow}
\let\shortmapsto\mapsto
\let\mapsto\longmapsto
\newcommand{\mapsfrom}{\,\reflectbox{$\mapsto$}\ }
\newcommand{\bigrightsquig}{\scalebox{2}{\ensuremath{\rightsquigarrow}}}
\newcommand{\bigleftsquig}{\reflectbox{\scalebox{2}{\ensuremath{\rightsquigarrow}}}}

%\newcommand{\cofibration}{\xhookrightarrow{\phantom{\ \,{\sim\!}\ \ }}}
%\newcommand{\fibration}{\xtwoheadrightarrow{\phantom{\sim\!}}}
%\newcommand{\acycliccofibration}{\xhookrightarrow{\ \,{\sim\!}\ \ }}
%\newcommand{\acyclicfibration}{\xtwoheadrightarrow{\sim\!}}
%\newcommand{\leftcofibration}{\xhookleftarrow{\phantom{\ \,{\sim\!}\ \ }}}
%\newcommand{\leftfibration}{\xtwoheadleftarrow{\phantom{\sim\!}}}
%\newcommand{\leftacycliccofibration}{\xhookleftarrow{\ \ {\sim\!}\,\ }}
%\newcommand{\leftacyclicfibration}{\xtwoheadleftarrow{\sim\!}}
%\newcommand{\weakequiv}{\xrightarrow{\ \,\sim\,\ }}
%\newcommand{\leftweakequiv}{\xleftarrow{\ \,\sim\,\ }}

\newcommand{\cofibration}
{\xhookrightarrow{\phantom{\ \,{\raisebox{-.3ex}[0ex][0ex]{\scriptsize$\sim$}\!}\ \ }}}
\newcommand{\fibration}
{\xtwoheadrightarrow{\phantom{\raisebox{-.3ex}[0ex][0ex]{\scriptsize$\sim$}\!}}}
\newcommand{\acycliccofibration}
{\xhookrightarrow{\ \,{\raisebox{-.55ex}[0ex][0ex]{\scriptsize$\sim$}\!}\ \ }}
\newcommand{\acyclicfibration}
{\xtwoheadrightarrow{\raisebox{-.6ex}[0ex][0ex]{\scriptsize$\sim$}\!}}
\newcommand{\leftcofibration}
{\xhookleftarrow{\phantom{\ \,{\raisebox{-.3ex}[0ex][0ex]{\scriptsize$\sim$}\!}\ \ }}}
\newcommand{\leftfibration}
{\xtwoheadleftarrow{\phantom{\raisebox{-.3ex}[0ex][0ex]{\scriptsize$\sim$}\!}}}
\newcommand{\leftacycliccofibration}
{\xhookleftarrow{\ \ {\raisebox{-.55ex}[0ex][0ex]{\scriptsize$\sim$}\!}\,\ }}
\newcommand{\leftacyclicfibration}
{\xtwoheadleftarrow{\raisebox{-.6ex}[0ex][0ex]{\scriptsize$\sim$}\!}}
\newcommand{\weakequiv}
{\xrightarrow{\ \,\raisebox{-.3ex}[0ex][0ex]{\scriptsize$\sim$}\,\ }}
\newcommand{\leftweakequiv}
{\xleftarrow{\ \,\raisebox{-.3ex}[0ex][0ex]{\scriptsize$\sim$}\,\ }}

%>>>>>>>>>>>>>>>>>>>>>>>>>>>>>>
%<<<    xymatrix Arrows     <<<
%>>>>>>>>>>>>>>>>>>>>>>>>>>>>>>
\newdir{ >}{{}*!/-5pt/@{>}}
\newcommand{\xycof}{\ar@{ >->}}
\newcommand{\xycofib}{\ar@{^{(}->}}
\newcommand{\xycofibdown}{\ar@{_{(}->}}
\newcommand{\xyfib}{\ar@{->>}}
\newcommand{\xymapsto}{\ar@{|->}}

%>>>>>>>>>>>>>>>>>>>>>>>>>>>>>>
%<<<     Greek Letters      <<<
%>>>>>>>>>>>>>>>>>>>>>>>>>>>>>>
%\newcommand{\oldphi}{\phi}
%\renewcommand{\phi}{\varphi}
\let\oldphi\phi
\let\phi\varphi
\renewcommand{\to}{\longrightarrow}
\newcommand{\from}{\longleftarrow}
\newcommand{\eps}{\varepsilon}

%>>>>>>>>>>>>>>>>>>>>>>>>>>>>>>
%<<<  1st-4th & parentheses <<<
%>>>>>>>>>>>>>>>>>>>>>>>>>>>>>>
\newcommand{\first}{^\text{st}}
\newcommand{\second}{^\text{nd}}
\newcommand{\third}{^\text{rd}}
\newcommand{\fourth}{^\text{th}}
\newcommand{\ZEROTH}{$0^\text{th}$ }
\newcommand{\FIRST}{$1^\text{st}$ }
\newcommand{\SECOND}{$2^\text{nd}$ }
\newcommand{\THIRD}{$3^\text{rd}$ }
\newcommand{\FOURTH}{$4^\text{th}$ }
\newcommand{\iTH}{$i^\text{th}$ }
\newcommand{\jTH}{$j^\text{th}$ }
\newcommand{\nTH}{$n^\text{th}$ }

%>>>>>>>>>>>>>>>>>>>>>>>>>>>>>>
%<<<    upright commands    <<<
%>>>>>>>>>>>>>>>>>>>>>>>>>>>>>>
\newcommand{\upcol}{\textup{:}}
\newcommand{\upsemi}{\textup{;}}
\providecommand{\lparen}{\textup{(}}
\providecommand{\rparen}{\textup{)}}
\renewcommand{\lparen}{\textup{(}}
\renewcommand{\rparen}{\textup{)}}
\newcommand{\Iff}{\emph{iff} }

%>>>>>>>>>>>>>>>>>>>>>>>>>>>>>>
%<<<     Environments       <<<
%>>>>>>>>>>>>>>>>>>>>>>>>>>>>>>
\newcommand{\squishlist}
{ %\setlength{\topsep}{100pt} doesn't seem to do anything.
  \setlength{\itemsep}{.5pt}
  \setlength{\parskip}{0pt}
  \setlength{\parsep}{0pt}}
\newenvironment{itemise}{
\begin{list}{\textup{$\rightsquigarrow$}}
   {  \setlength{\topsep}{1mm}
      \setlength{\itemsep}{1pt}
      \setlength{\parskip}{0pt}
      \setlength{\parsep}{0pt}
   }
}{\end{list}\vspace{-.1cm}}
\newcommand{\INDENT}{\textbf{}\phantom{space}}
\renewcommand{\INDENT}{\rule{.7cm}{0cm}}

\newcommand{\itm}[1][$\rightsquigarrow$]{\item[{\makebox[.5cm][c]{\textup{#1}}}]}


%\newcommand{\rednote}[1]{{\color{red}#1}\makebox[0cm][l]{\scalebox{.1}{rednote}}}
%\newcommand{\bluenote}[1]{{\color{blue}#1}\makebox[0cm][l]{\scalebox{.1}{rednote}}}

\newcommand{\rednote}[1]
{{\color{red}#1}\makebox[0cm][l]{\scalebox{.1}{\rotatebox{90}{?????}}}}
\newcommand{\bluenote}[1]
{{\color{blue}#1}\makebox[0cm][l]{\scalebox{.1}{\rotatebox{90}{?????}}}}


\newcommand{\funcdef}[4]{\begin{align*}
#1&\to #2\\
#3&\mapsto#4
\end{align*}}

%\newcommand{\comment}[1]{}

%>>>>>>>>>>>>>>>>>>>>>>>>>>>>>>
%<<<       Categories       <<<
%>>>>>>>>>>>>>>>>>>>>>>>>>>>>>>
\newcommand{\Ens}{{\mathscr{E}ns}}
\DeclareMathOperator{\Sheaves}{{\mathsf{Shf}}}
\DeclareMathOperator{\Presheaves}{{\mathsf{PreShf}}}
\DeclareMathOperator{\Psh}{{\mathsf{Psh}}}
\DeclareMathOperator{\Shf}{{\mathsf{Shf}}}
\DeclareMathOperator{\Varieties}{{\mathsf{Var}}}
\DeclareMathOperator{\Schemes}{{\mathsf{Sch}}}
\DeclareMathOperator{\Rings}{{\mathsf{Rings}}}
\DeclareMathOperator{\AbGp}{{\mathsf{AbGp}}}
\DeclareMathOperator{\Modules}{{\mathsf{\!-Mod}}}
\DeclareMathOperator{\fgModules}{{\mathsf{\!-Mod}^{\textup{fg}}}}
\DeclareMathOperator{\QuasiCoherent}{{\mathsf{QCoh}}}
\DeclareMathOperator{\Coherent}{{\mathsf{Coh}}}
\DeclareMathOperator{\GSW}{{\mathcal{SW}^G}}
\DeclareMathOperator{\Burnside}{{\mathsf{Burn}}}
\DeclareMathOperator{\GSet}{{G\mathsf{Set}}}
\DeclareMathOperator{\FinGSet}{{G\mathsf{Set}^\textup{fin}}}
\DeclareMathOperator{\HSet}{{H\mathsf{Set}}}
\DeclareMathOperator{\Cat}{{\mathsf{Cat}}}
\DeclareMathOperator{\Fun}{{\mathsf{Fun}}}
\DeclareMathOperator{\Orb}{{\mathsf{Orb}}}
\DeclareMathOperator{\Set}{{\mathsf{Set}}}
\DeclareMathOperator{\sSet}{{\mathsf{sSet}}}
\DeclareMathOperator{\Top}{{\mathsf{Top}}}
\DeclareMathOperator{\GSpectra}{{G-\mathsf{Spectra}}}
\DeclareMathOperator{\Lan}{Lan}
\DeclareMathOperator{\Ran}{Ran}

%>>>>>>>>>>>>>>>>>>>>>>>>>>>>>>
%<<<     Script Letters     <<<
%>>>>>>>>>>>>>>>>>>>>>>>>>>>>>>
\newcommand{\scrQ}{\mathscr{Q}}
\newcommand{\scrW}{\mathscr{W}}
\newcommand{\scrE}{\mathscr{E}}
\newcommand{\scrR}{\mathscr{R}}
\newcommand{\scrT}{\mathscr{T}}
\newcommand{\scrY}{\mathscr{Y}}
\newcommand{\scrU}{\mathscr{U}}
\newcommand{\scrI}{\mathscr{I}}
\newcommand{\scrO}{\mathscr{O}}
\newcommand{\scrP}{\mathscr{P}}
\newcommand{\scrA}{\mathscr{A}}
\newcommand{\scrS}{\mathscr{S}}
\newcommand{\scrD}{\mathscr{D}}
\newcommand{\scrF}{\mathscr{F}}
\newcommand{\scrG}{\mathscr{G}}
\newcommand{\scrH}{\mathscr{H}}
\newcommand{\scrJ}{\mathscr{J}}
\newcommand{\scrK}{\mathscr{K}}
\newcommand{\scrL}{\mathscr{L}}
\newcommand{\scrZ}{\mathscr{Z}}
\newcommand{\scrX}{\mathscr{X}}
\newcommand{\scrC}{\mathscr{C}}
\newcommand{\scrV}{\mathscr{V}}
\newcommand{\scrB}{\mathscr{B}}
\newcommand{\scrN}{\mathscr{N}}
\newcommand{\scrM}{\mathscr{M}}

%>>>>>>>>>>>>>>>>>>>>>>>>>>>>>>
%<<<     Fractur Letters    <<<
%>>>>>>>>>>>>>>>>>>>>>>>>>>>>>>
\newcommand{\frakQ}{\mathfrak{Q}}
\newcommand{\frakW}{\mathfrak{W}}
\newcommand{\frakE}{\mathfrak{E}}
\newcommand{\frakR}{\mathfrak{R}}
\newcommand{\frakT}{\mathfrak{T}}
\newcommand{\frakY}{\mathfrak{Y}}
\newcommand{\frakU}{\mathfrak{U}}
\newcommand{\frakI}{\mathfrak{I}}
\newcommand{\frakO}{\mathfrak{O}}
\newcommand{\frakP}{\mathfrak{P}}
\newcommand{\frakA}{\mathfrak{A}}
\newcommand{\frakS}{\mathfrak{S}}
\newcommand{\frakD}{\mathfrak{D}}
\newcommand{\frakF}{\mathfrak{F}}
\newcommand{\frakG}{\mathfrak{G}}
\newcommand{\frakH}{\mathfrak{H}}
\newcommand{\frakJ}{\mathfrak{J}}
\newcommand{\frakK}{\mathfrak{K}}
\newcommand{\frakL}{\mathfrak{L}}
\newcommand{\frakZ}{\mathfrak{Z}}
\newcommand{\frakX}{\mathfrak{X}}
\newcommand{\frakC}{\mathfrak{C}}
\newcommand{\frakV}{\mathfrak{V}}
\newcommand{\frakB}{\mathfrak{B}}
\newcommand{\frakN}{\mathfrak{N}}
\newcommand{\frakM}{\mathfrak{M}}

\newcommand{\frakq}{\mathfrak{q}}
\newcommand{\frakw}{\mathfrak{w}}
\newcommand{\frake}{\mathfrak{e}}
\newcommand{\frakr}{\mathfrak{r}}
\newcommand{\frakt}{\mathfrak{t}}
\newcommand{\fraky}{\mathfrak{y}}
\newcommand{\fraku}{\mathfrak{u}}
\newcommand{\fraki}{\mathfrak{i}}
\newcommand{\frako}{\mathfrak{o}}
\newcommand{\frakp}{\mathfrak{p}}
\newcommand{\fraka}{\mathfrak{a}}
\newcommand{\fraks}{\mathfrak{s}}
\newcommand{\frakd}{\mathfrak{d}}
\newcommand{\frakf}{\mathfrak{f}}
\newcommand{\frakg}{\mathfrak{g}}
\newcommand{\frakh}{\mathfrak{h}}
\newcommand{\frakj}{\mathfrak{j}}
\newcommand{\frakk}{\mathfrak{k}}
\newcommand{\frakl}{\mathfrak{l}}
\newcommand{\frakz}{\mathfrak{z}}
\newcommand{\frakx}{\mathfrak{x}}
\newcommand{\frakc}{\mathfrak{c}}
\newcommand{\frakv}{\mathfrak{v}}
\newcommand{\frakb}{\mathfrak{b}}
\newcommand{\frakn}{\mathfrak{n}}
\newcommand{\frakm}{\mathfrak{m}}

%>>>>>>>>>>>>>>>>>>>>>>>>>>>>>>
%<<<  Caligraphic Letters   <<<
%>>>>>>>>>>>>>>>>>>>>>>>>>>>>>>
\newcommand{\calQ}{\mathcal{Q}}
\newcommand{\calW}{\mathcal{W}}
\newcommand{\calE}{\mathcal{E}}
\newcommand{\calR}{\mathcal{R}}
\newcommand{\calT}{\mathcal{T}}
\newcommand{\calY}{\mathcal{Y}}
\newcommand{\calU}{\mathcal{U}}
\newcommand{\calI}{\mathcal{I}}
\newcommand{\calO}{\mathcal{O}}
\newcommand{\calP}{\mathcal{P}}
\newcommand{\calA}{\mathcal{A}}
\newcommand{\calS}{\mathcal{S}}
\newcommand{\calD}{\mathcal{D}}
\newcommand{\calF}{\mathcal{F}}
\newcommand{\calG}{\mathcal{G}}
\newcommand{\calH}{\mathcal{H}}
\newcommand{\calJ}{\mathcal{J}}
\newcommand{\calK}{\mathcal{K}}
\newcommand{\calL}{\mathcal{L}}
\newcommand{\calZ}{\mathcal{Z}}
\newcommand{\calX}{\mathcal{X}}
\newcommand{\calC}{\mathcal{C}}
\newcommand{\calV}{\mathcal{V}}
\newcommand{\calB}{\mathcal{B}}
\newcommand{\calN}{\mathcal{N}}
\newcommand{\calM}{\mathcal{M}}

%>>>>>>>>>>>>>>>>>>>>>>>>>>>>>>
%<<<<<<<<<DEPRECIATED<<<<<<<<<<
%>>>>>>>>>>>>>>>>>>>>>>>>>>>>>>

%%% From Kac's template
% 1-inch margins, from fullpage.sty by H.Partl, Version 2, Dec. 15, 1988.
%\topmargin 0pt
%\advance \topmargin by -\headheight
%\advance \topmargin by -\headsep
%\textheight 9.1in
%\oddsidemargin 0pt
%\evensidemargin \oddsidemargin
%\marginparwidth 0.5in
%\textwidth 6.5in
%
%\parindent 0in
%\parskip 1.5ex
%%\renewcommand{\baselinestretch}{1.25}

%%% From the net
%\newcommand{\pullbackcorner}[1][dr]{\save*!/#1+1.2pc/#1:(1,-1)@^{|-}\restore}
%\newcommand{\pushoutcorner}[1][dr]{\save*!/#1-1.2pc/#1:(-1,1)@^{|-}\restore}









\usepackage{framed}
%\usepackage{biblatex} 
\usepackage[style=numeric,%citestyle=numeric,
url=false,doi=false,isbn=false,eprint=false]{biblatex}%
\hypersetup{colorlinks=false,pdfborder={0 0 0}}



\makeatletter
\renewcommand{\@seccntformat}[1]{\csname the#1\endcsname.\quad}
\makeatother


\theoremstyle{plain}
\newtheorem{theorem}{Theorem}
\newtheorem*{completenesstheorem}{Completeness Theorem}
\newtheorem{twistinglemma}[thm]{Twisting Lemma}
\renewcommand*{\thetheorem}{\Alph{theorem}}
\newtheorem{conjectureAlpha}{Conjecture}
\renewcommand*{\theconjectureAlpha}{\Alph{conjectureAlpha}}



\newcommand{\NOFULLPAGE}{\relax}

\newcommand{\Sq}{\mathrm{Sq}}
\newcommand{\Comm}{\calC}

\bibliography{../../Dropbox/logbook/_LOGBOOK/papers}


\title{Connected simplicial algebras are Andr\'e-Quillen complete}
\author{Michael Donovan}

\begin{document}
\maketitle

\begin{abstract}
We modify a classical construction of Bousfield and Kan to define the Adams tower of a simplicial nonunital algebra over a field $k$. We relate this construction to Radulescu-Banu's cosimplicial resolution, and prove that the all connected simplicial algebras are complete with respect to Andr\'e-Quillen homology. This is a convergence result for the unstable Adams spectral sequence for commutative algebras over $k$.
\end{abstract}

The classical unstable Adams spectral sequence was defined in two ways by Bousfield and Kan. In \cite{BousKanSSeq.pdf} they define an Adams tower for a reduced simplicial set, and define the Adams spectral sequence to be the homotopy spectral sequence of this tower. The tower is built by \emph{deriving functors with respect to a ring}.
 Later, in \cite{BK_pairings_products.pdf}, they make another construction of the spectral sequence using cosimplicial methods, namely, they construct \emph{the resolution of a space with respect to a ring}, and take the spectral sequence of this cosimplicial space.

In this paper, we will work in the category $s\calC$ of simplicial non-unital algebras over a field $k$. For any $X\in s\calC$, Radulescu-Banu \cite{radelescuBanu.pdf} gave a cosimplicial resolution $\calX^\bullet$ of $X$, analogous to the resolution of a space with respect to a ring. Having done so, he defines the \emph{completion of $X$ with respect to Andr\'e-Quillen homology} to be the simplicial algebra
\[X^{\hat\,}:=\textup{Tot}(\calX^\bullet),\]
and defines the absolute Bousfield-Kan spectral sequence of $X$ to be the spectral sequence of the cosimplicial object $\calX^\bullet$. Its target is $\pi_*X^{\hat\,}$. 

More precisely, Radulescu-Banu constructs a comonad structure on the cofibrant replacement functor $c:s\calC\to s\calC$ arising from Quillen's small object argument, and mixes $c$ into the standard cobar construction arising from the adjunction $Q:s\calC\rightleftarrows s\calV:K$, obtaining a coaugmented cosimplicial object 
\[
\vcenter{
\def\labelstyle{\scriptstyle}
\xymatrix@C=1.5cm@1{
cX\,
\ar[r]
&
\,cKQcX\,
\ar[r];[]
&
\,c(KQc)^2X\,
\ar@<-1ex>[l];[]
\ar@<+1ex>[l];[]
\ar@<+1ex>[r];[]
\ar@<-1ex>[r];[]
&
\,c(KQc)^3X\,\makebox[0cm][l]{\,$\cdots $}
\ar[l];[]
\ar@<-2ex>[l];[]
\ar@<+2ex>[l];[]
}}.\]
In particular, there is a natural completion map $cX\to\textup{Tot}(\calX^\bullet)$, and we say that $X$ is \emph{complete with respect to Andr\'e-Quillen homology} when this map is an equivalence.
For background, general theory, and another approach, see Blum-Riehl and Bous03. The purpose of this paper is to prove the following
\begin{completenesstheorem}\label{completenesstheorem}
Suppose that $k$ is a field, and that $X$ is a connected simplicial $k$-algebra. Then the completion map $X\to X{\hat{\,}_\textup{\!\!AQ}}$ is an equivalence.
\end{completenesstheorem}
The failure of the completion map $X\to X^{\hat\,}$ to be an equivalence is tied to a construction we give in section \ref{sec:derWRTab} of an Adams tower over $X$. Although this construction is analogous to Bousfield and Kan's original construction in classical topology, just as in Radulescu-Banu's work, we must be careful to ensure that our constructions are homotopically correct, as not all simplicial algebras are cofibrant. In section \ref{sec:relnWithRB} we explain the relationship between this Adams tower and Radulescu-Banu's cosimplicial resolution. In section \ref{sec:connectivityAnalysis} we perform a connectivity analysis, analogous to Bousfield and Kan's in the classical case, confirming the completeness theorem. Appendix \ref{sec:ItSimpBar} contains a construction required for the connectivity analysis of section \ref{sec:connectivityAnalysis}.


\begin{shaded}
\tiny
an Adams spectral sequence for simplicial algebras, by finding an appropriate cosimplicial resolution of a simplicial algebra with respect to Andr\'e-Quillen homology. Thus, he obtained a spectral sequence with target the homotopy of a certain completion $X{\hat\,}$ of $X$.
\begin{itemise}
\setlength{\parindent}{.25in}
\item The classical unstable Adams spectral sequence was defined in two ways by Bousfield and Kan. In \cite{BousKanSSeq.pdf} they define an Adams tower for a reduced simplicial set, and define the Adams spectral sequence to be the homotopy spectral sequence of this tower. The tower is built by \emph{deriving functors with respect to $R$}.
 Later, in \cite{BK_pairings_products.pdf}, they make another construction of the spectral sequence using cosimplicial methods, namely, they construct \emph{the resolution of a space with respect to $R$}, and take the spectral sequence of this cosimplicial space.
\item The first construction lends itself well to analysis of the completion map $X\to X{\hat\,}$, while the second construction lends itself to the description of the $E^2$ page by homotopical algebra, and to the construction of a Whitehead product on the Adams spectral sequence.
\item Radulescu-Banu \cite{radelescuBanu.pdf} constructed an Adams spectral sequence for simplicial algebras, by finding an appropriate cosimplicial resolution of a simplicial algebra with respect to Andr\'e-Quillen homology. Thus, he obtained a spectral sequence with target the homotopy of a certain completion $X{\hat\,}$ of $X$.

The purpose of this paper is to analyse the completion map $X\to X{\hat\,}$ by reconstructing Radulescu-Banu's spectral sequence by a method analogous to Bousfield and Kan's first method in the classical case. That is, by \emph{deriving functors with respect to Andr\'e-Quillen homology}. We will be able to prove theorem \ref{completenesstheorem}, conjectured by Radelescu-Banu. We will also prove that three different ways to define the spectral sequence coincide.
\end{itemise}

\end{shaded}


\section{Functors derived with respect to abelianisation and the Adams tower}\label{sec:derWRTab}
In this paper we work in the category $\calC$ of nonunital commutative algebras over a field $k$. These are simply $k$-vector spaces $A$ equipped with a commutative and associative pairing $A\otimes A\to A$. The category $s\calC$ of simplicial algebras is a model category (cf.\ [quillen]), and we let $c:s\calC\to s\calC$ be a functorial cofibrant replacement produced by Quillen's small object argument.

%\newcommand{\dupdown}[2]{D^{\smash{#1}}_{\smash{#2}}}
\newcommand{\dupdown}[2]{D_{\smash{#1}}}
\newcommand{\caldup}[1]{\calD_{\smash{#1}}}
\newcommand{\caldupdown}[2]{\calD^{\smash{#1}}_{\smash{#2}}}



For any functor $F:s\calC\to s\calC$, we'll now define the $r^\textup{th}$ derivation $\dupdown{r}{b}F$ of $F$ with respect to Andr\'e-Quillen homology. The definition is recursive:
\begin{alignat*}{2}
(\dupdown{0}{b}F)(X)
&:=
F(cX)%
\\
(\dupdown{s}{b}F)(X)
&:=
\hofib((\dupdown{s-1}{c}F)(cX)\xrightarrow{(\dupdown{s-1}{c}F)(\eta_{cX})} (\dupdown{s-1}{c}F)(KQcX))
\end{alignat*}
where $\eta$ is the unit of the adjunction $Q\dashv K$, i.e.\ the natural surjection onto indecomposables, and $\hofib$ is any fixed  functorial construction of the homotopy fiber. These functors fit into a tower, via the following composite natural transformation:
\[\delta:\left((\dupdown{s}{c}F)(X)\to (\dupdown{s-1}{c}F)(cX)\overset{(\dupdown{s-1}{c}F)(\epsilon)}{\to} (\dupdown{s-1}{c}F)(X)\right).\]
For clarity, we'll record the construction of $(\dupdown{1}{c}F)(X)$, $(\dupdown{2}{c}F)(X)$ and the map between them in the following diagram in which every composable pair of parallel arrows is defined to be a homotopy fiber sequence.
%@=50pt
\[\def\labelstyle{\scriptstyle}
\xymatrix@!0@R=30pt@C=50pt{
(\dupdown{2}{c}F)(X)\ar[dr]\ar[dd]_\delta\\
&(\dupdown{1}{c}F)(cX) \ar[rr]\ar[rd]\ar[dd]^(.75){(\dupdown{1}{c}F)(\epsilon)}         &           &FcccX \ar[rr]\ar[rd]\ar'[d][dd]^{Fcc\epsilon}         &           &   FcKQccX \ar'[d][dd]^{FcKQc\epsilon}           \ar[dr]  &                  \\
(\dupdown{1}{c}F)(X)\ar[dr]&        &  (\dupdown{1}{c}F)(KQcX) \ar[rr] \ar[dd]  &                     &  FccKQcX \ar[rr] \ar[dd]  &             & FcKQcKQcX
         \ar[dd] \\
&(\dupdown{1}{c}F)(X) \ar'[r][rr] \ar[dr] &        &   FccX \ar'[r][rr] \ar[dr] &        &   FcKQcX
\ar[dr] &                   \\
&        &   0 \ar[rr]      &               &   0 \ar[rr]      &                    &
0
}\]
We have thus constructed a tower
\[\xymatrix@R=4mm{
\cdots 
\ar[r]
&%r1c1
(\dupdown{2}{c}F)X
\ar[r]
&%r1c3
%r1c4
(\dupdown{1}{c}F)X
\ar[r]&%r1c5
(\dupdown{0}{c}F)X=FcX,
}\]
which is natural in in the object $X$ and the functor $F$.
The functors $\dupdown{r}{c}F$ are homotopical as long as $F$ preserves weak equivalences between cofibrant objects. Employing the shorthand
\[\dupdown{s}{c}X:=(\dupdown{s}{c}I)X,\]
we define \emph{the Adams tower of $X$} to be the tower
\[\xymatrix@R=4mm{
\cdots 
\ar[r]
&%r1c1
\dupdown{2}{c}X
\ar[r]
&%r1c3
%r1c4
\dupdown{1}{c}X
\ar[r]&%r1c5
\dupdown{0}{c}X=cX.
}\]

\section{Relationship with Radulescu-Banu's cosimplicial resolution}\label{sec:relnWithRB}
\begin{itemise}
\setlength{\parindent}{.25in}
\item Explain RB's resolution.
\item The spectral sequence may be defined to be either the spectral sequence of the bicomplex or the spectral sequence of the Tot tower, thanks to \cite[Lemma 2.2]{BousfieldHSSCS.pdf}.
\item By results of \cite[{X.4.9}]{YellowMonster}, all (truncated) cosimplicial groups are Reedy fibrant.
\item It converges to the totalisation of the resolution, which we make the definition of $X{\hat\,}$.
\item Explain that Blum-Riehl and Bousfield together give a partial description of the $E^2$.
\end{itemise}
\begin{itemise}
\setlength{\parindent}{.25in}
\item As explained by Sinha, we can model the Tot tower of the Radulescu-Banu resolution by the tower of homotopy limits induced by the inclusions
\[\cdots\to P_0[n-1]\to P_0[n]\to P_0[n+1]\to\cdots \to P_0[\infty]\]
\item Then we can pull back the tower under $\scrX^{-1}$ to a tower over $\scrX^{-1}$, and the terms are a definition of the total homotopy fiber of a cube.
\item Note that the derived functors wrt Quillen homology coincide with the alternative definition of the total homotopy fiber. That's the point, really.
\end{itemise}



\section{Connectivity estimates in the Adams tower}
\label{sec:connectivityAnalysis}
We wish to prove the following connectivity result for the Adams tower:
\begin{prop}\label{convergenceProp}
For integers $t\geq2$ and $q\geq0$, the map $\pi_q(\dupdown{2t+q-1}{c}X)\to\pi_q(\dupdown{t}{c}X)$ is zero.
\end{prop}
Indeed, in light of the results of section \ref{sec:relnWithRB}, proposition \ref{convergenceProp} implies the completeness theorem:
\begin{proof}[Proof of completeness theorem]
If you take the fibers of the maps from $cX$ to the Tot tower you'll get the Adams tower. Nothing gets down to $\pi_*(cX)$ because of proposition \ref{convergenceProp}.
\end{proof}

For this purpose we now define a somewhat less homotopical version $\caldup{s}F$ of the derivations $\dupdown{s}{b}F$ of $F$. There will three differences: in the definition of $(\caldup{s}F)(X)$ there will be one fewer cofibrant replacement applied, we will use the standard simplicial bar construction $b$ instead of the SOA functor $c$, and we will take \emph{strict} fibers, not \emph{homotopy} fibers. Again, the definition is recursive:
\begin{alignat*}{2}
(\caldup{0}F)(X)
&:=
F(X)%
\\
(\caldup{s}F)(X)
&:=
\ker((\caldup{s-1}F)(bX)\xrightarrow{(\caldup{s-1}F)(\eta_{bX})} (\caldup{s-1}F)(KQbX))
\end{alignat*}
These functors will not be homotopical unless $F$ is homotopical itself. We define \emph{the modified Adams tower of $X$} to be the tower
\[\xymatrix@R=4mm{
\cdots 
\ar[r]
&%r1c1
\caldup{2}X
\ar[r]
&%r1c3
%r1c4
\caldup{1}X
\ar[r]&%r1c5
\caldup{0}X=X,
}\]
where $\caldup{s}X$ is again shorthand for $(\caldup{s}I)X$, and the tower maps $\delta$ are defined as before.
\begin{prop}\label{prop:modifiedAdamsTower}
There's a natural zig-zag of weak equivalences of towers between the Adams tower of $X$ and the modified Adams tower of $X$.
\end{prop}
\begin{proof}
Let $\mathsf{CR(s\calC)}$ be the category of cofibrant replacement functors in $s\calC$. That is, an object of $\mathsf{CR(s\calC)}$ is a pair, $(f,\epsilon)$, such that $f:s\calC\to s\calC$ is a functor whose image consists only of cofibrant objects, and $\epsilon:f\Rightarrow I$ is a natural acyclic fibration. Morphisms in $\mathsf{CR(s\calC)}$ are natural transformations which commute with the augmentations. Then we may define, for any $(f,\epsilon)\in\mathsf{CR(s\calC)}$ and $F:s\calC\to s\calC$, the derivations of $F$ \emph{using replacement $f$}:
\begin{alignat*}{2}
(D_{\smash{0}}^{\smash{f}}F)(X)
:=
F(fX),\quad %
(D_{\smash{s}}^{\smash{f}}F)(X)
:=
\hofib((D_{\smash{s-1}}^{\smash{f}}F)(fX)\to (D_{\smash{s-1}}^{\smash{f}}F)(KQfX)).
\end{alignat*}
These functors are natural in $f$, so that a morphism in $\mathsf{CR(s\calC)}$ induces a weak equivalence of towers. Our proposed zig-zag of towers is: %Now there is a zig-zag in $\mathsf{CR(s\calC)}$ from the SOA functor $c$ to the bar construction $b$, as both receive a map from the cofibrant replacement $b\circ c$. Thus we need only produce a zig-zag between the towers $\{(D_{\smash{s}}^{\smash{b}}I)X\}$ and $\{(\caldup{s}I)X\}$
\[D_{\smash{s}}= D_{\smash{s}}^{\smash{c}}I\overset{\alpha}{\from}D_{\smash{s}}^{\smash{b\circ c}}\!I\overset{\beta}{\to}D_{\smash{s}}^{\smash{b}}I\overset{\gamma}{\from}\caldup{s}b\overset{\caldup{s}\epsilon}{\to}\caldup{s}I=\caldup{s}\]
The map $\caldup{s}\epsilon$ is a natural weak equivalence, as are the maps $\alpha$ and $\beta$, which are induced by the evident maps of cofibrant replacement functors. 

The remaining map $\gamma$ is defined recursively using the inclusion of the strict fiber in the homotopy fiber. That is, $\gamma_s:(\caldup{s}b)X\to D_{\smash{s}}^{\smash{b}}X$ is the identity of $bX$ when $s=0$, and for $s>0$, it is:
\[\xymatrix@R=4mm{
\makebox[0cm][r]{$(\caldup{s}b)X:=\,$}\ker((\caldup{s-1}b)(bX)\to (\caldup{s-1}b)(KQbX))\ar[d]^-{\textup{incl.}}
\\%r1c1
\hofib((\caldup{s-1}b)(bX)\to(\caldup{s-1}b)(KQbX))\ar[d]^-{\hofib(\gamma_{s-1},\gamma_{s-1})}
\\%r2c1
\hofib((D_{\smash{s-1}}^{\smash{b}}I)(bX)\to (D_{\smash{s-1}}^{\smash{b}}I)(KQbX))\makebox[0cm][l]{$\,=:D_{\smash{s}}^{\smash{b}}X$}%r3c1
}\]
We will prove that $\gamma_s$ is a weak equivalence by induction on $s$. The base case is trivial as $\gamma_0$ is an identity map. To complete the induction step, it will suffice to show that the inclusion above is a weak equivalence, which will follow from lemma \ref{towerWithPowers}, as all epimorphisms are fibrations [miller?] and the unit map $\eta:I\to KQ$ is an epimorphism.
\end{proof}
Now let $P^s:s\calC\to s\calC$ be the ``$s^\textup{th}$ power'' functor, the prolongation of the endofunctor $Y\mapsto Y^s$ of $\calC$, where $Y^s=\textup{im}(\textup{mult}:Y^{\otimes s}\to Y)$.
\begin{lem}\label{towerWithPowers}
The functors $\caldup{r}$, $\caldup{r}b$ and $\caldup{r}P^{s}$ preserve surjective maps and there is a commuting diagram of functors:
\[\xymatrix@R=4mm{
\cdots 
\ar[r]
&%r1c1
\caldup{r}
\ar[r]
\ar[d]
&%r1c2
\cdots \ar[r]
&%r1c4
\caldup{2}
\ar[r]
\ar[d]
&%r1c4
\caldup{1}
\ar[r]
\ar[d]&%r1c5
\caldup{0}
\ar@{=}[d]
\\%r1c6
\cdots
\ar[r]
&%r2c1
P^{r+1}
\ar[r]
&%r2c3
\cdots 
\ar[r]&%r2c4
P^3
\ar[r]
&P^2
\ar[r]
&%r2c5
I%r2c6
}\]
\end{lem}
\begin{shaded}\tiny
\begin{proof}
Before we give this proof, we'll need a little notation.
If $F$ is the prolongation \textbf{almost} of a functor $\calC\to\calC$, one notes that $(\caldup{s}F)$ can be defined one level at a time. That is,  we may define functors $(\caldupdown{s}{n}F):\calC\to\calC$ such that $(\caldup{s}F)(X)_n=(\caldupdown{s}{n}F)(X_n)$, by defining $(\caldupdown{0}{n}F):=F$, and
\[(\caldupdown{s}{n}F)(Y):=\ker((\caldupdown{s-1}{n}F)(T^{n+1}Y)\to (\caldupdown{s-1}{n}F)(QT^{n+1}Y)).\]
The claims of the lemma follow from the following facts, which hold for any $Y\in\calC$, and integers $r,n\geq0$ and $s>0$:
\begin{enumerate}
\squishlist
\setlength{\parindent}{.25in}
\item[a)] $(\caldupdown{r}{n}I)Y\subset (\calT_1T^n_1)\cdots (\calT_rT^n_r)Y$ is the subset consisting of those mega\-monomials which have a doubling in each of the $\calT_1,\ldots,\calT_r$ functors.
\item[b)] $(\caldupdown{r}{n}b)Y\subset (\calT T^n)(\calT_1T^n_1)\cdots (\calT_rT^n_r)Y$ is the subset consisting of those mega\-monomials which have a doubling in each of the $\calT_1,\ldots,\calT_r$ functors.
\item[c)] $(\caldupdown{r}{n}P^{s})Y\subset (\calT_1T^n_1)\cdots (\calT_rT^n_r)Y$ is the subset consisting of those mega\-monomials which have a doubling in each of the $\calT_2,\ldots,\calT_r$ functors, and a power of $s$ in the $\calT_1$ functor.
\end{enumerate}
(a) is trivial when $r=0$. Supposing an inductive hypothesis:
\begin{alignat*}{2}
(\caldupdown{r}{n}I)(Y)
&:=
\ker((\caldupdown{r-1}{n}I)(\calT_r T_r^n Y)\to (\caldupdown{r-1}{n}I) (Q\calT_rT_r^n X_n))%
\\
&=
(\caldupdown{r-1}{n}I)(\calT_r T_r^n Y)\cap\langle \textup{meganomials with a doubling in $\calT_r$}\rangle\\
&=
\langle \textup{meganomials with a doubling in $\calT_1,\ldots,\calT_{r-1}$}\rangle\cap\langle \textup{doubling in $\calT_r$}\rangle\\
&=
\langle \textup{meganomials with a doubling in $\calT_1,\ldots,\calT_{r-1},\calT_{r}$}\rangle
\end{alignat*}
The proofs of (b) and (c) are essentially the same as the proof of (a).
\end{proof}
\end{shaded}
\begin{lem}\label{connectivityOfDerivedPowers}
For $A\in s\Comm$ connected, and any $t\geq1$ and $s\geq2$, $(\caldup{t}P^{s})(A)$ is $(s-t)$-connected.
\end{lem}
\begin{proof}
We'll prove this by induction on $t$. When $t=1$ we have:
\[(\caldup{1}P^{s})A:=\ker(P^{s}(bA)\to P^{s}(QbA))=P^{s}(bA),\]
so we must show that $P^s(bA)$ is $(s-1)$-connected. For this we use a truncation of Quillen's fundamental spectral sequence, as presented in \cite[thm 6.2]{MR1089001}. That is, filter $P^s(bA)$ by the ideals $P^{s+p}(bA)$ for $p\geq0$. Then there is a convergent spectral sequence:
\[E^1_{p,q}= \pi_q\bigl((QbA)^{\otimes (s+p)}_{\Sigma_{s+p}}\bigr)\implies \pi_q(P^s(bA))\]
Now the simplicial vector space $QbA$ is connected, as $\pi_0(QbA)=Q(\pi_0bA)=Q(0)=0$. The $t=1$ case now follows from the general fact that if $V$ is a connected simplicial vector space then $V^{\otimes n}_{\Sigma_n}$ is $(n-1)$-connected. (Bousfield? Obvious? Whatever.)

Now let $t\geq2$ and suppose by induction that $(\caldup{t-1}P^{s})(B)$ is $(s-(t-1))$-connected for any connected $B$ and any $s\geq2$. Then by lemma \ref{towerWithPowers}, there's a short exact sequence:
\[\xymatrix{
0\ar[r]&
(\caldup{t}P^{s})(A)\ar[r]&
(\caldup{t-1}P^{s})(cA)\ar[r]&
(\caldup{t-1}P^{s})(QcA)\ar[r]&
0
}\]
in which the rightmost two objects are each $(s-t+1)$-connected. The associated long exact sequence shows that $(\caldup{t}P^{s})(A)$ is $(s-t)$-connected.
\end{proof}
Before we can give the proof of proposition \ref{convergenceProp}, we need the following \emph{twisting lemma}, analogous to that of [BandK]. Before stating it, we note that $\caldup{s}\caldup{t}(X)$ and $\caldup{s+t}X$ are equal by construction.
\begin{twistinglemma}
\label{DsDt=Dt+s}
The maps $\caldup{i}\delta:\caldup{n}X\to \caldup{n-1}X$ are homotopic for $0\leq i< n$.
\end{twistinglemma}
\begin{proof}
We may reindex the twisting lemma as follows: the maps 
\[\caldup{s}\delta,\caldup{s-1}\delta:\caldup{s+t}X\to \caldup{s+t-1}X\]
are homotopic whenever $s,t\geq1$. Now $\caldup{s+t}X$ is constructed as the subalgebra
\[\caldup{s+t}X:= \bigcap_{i=1}^{s+t}\ker\!\left(b^{s+t-i}\eta b^{i}:b^{s+t}X\to b^{s+t-i}KQb^{i}X\right)\]
of the iterated bar construction $b^{s+t}X$, and there are commuting squares
\[\xymatrix@R=7mm{
b^{s+t}X
\ar[r]_-{b^s\epsilon b^{t-1}}
&%r1c1
b^{s+t-1}X\\%r1c2
\caldup{s+t}X
\ar[r]^-{\caldup{s}\delta}
\ar[u]
&%r2c1
\caldup{s+t-1}X
\ar[u]
%r2c2
}\raisebox{-6mm}{\textup{\quad and\quad }}
\xymatrix@R=7mm{
b^{s+t}X
\ar[r]_-{b^{s-1}\epsilon b^{t}}
&%r1c1
b^{s+t-1}X\\%r1c2
\caldup{s+t}X
\ar[r]^-{\caldup{s-1}\delta}
\ar[u]
&%r2c1
\caldup{s+t-1}X
\ar[u]
%r2c2
}\]
Our method is to construct an explicit simplicial homotopy between the maps $b^s\epsilon b^{t-1}$ and $b^{s-1}\epsilon b^{t}$, and to simply note that the homotopy restricts to a homotopy between $\caldup{s}\delta$ and $\caldup{s-1}\delta$. We defer the construction of this homotopy until proposition \ref{IteratedBarConstructionHomotopy}.
\end{proof}




\begin{proof}[Proof of proposition \ref{convergenceProp}]
%The first claim is \ref{towerIsHomotopical}. 
By proposition \ref{prop:modifiedAdamsTower}, it is enough to prove that $\pi_q(\caldup{2t+q-1}X)\to\pi_q(\caldup{t}X)$ is zero.
Apply $\caldup{t}\DASH$ to the diagram of functors constructed in \ref{towerWithPowers} and apply the result to $X$ to obtain a commuting diagram of functors
\[\xymatrix@R=4mm{
\caldup{2t+q-1}X
\ar[r]^-{\caldup{t}\delta}
\ar[d]
&
\cdots \ar[r]^-{\caldup{t}\delta}
&%r1c4
\caldup{t+1}X
\ar[r]^-{\caldup{t}\delta}
\ar[d]&%r1c5
\caldup{t}X
\ar@{=}[d]
\\%r1c6
\caldup{t}P^{t+q}X
\ar[r]
&%r2c3
\cdots 
\ar[r]&%r2c4
\caldup{t}P^2X
\ar[r]
&%r2c5
\caldup{t}P^1X%r2c6
}\]
By the twisting lemma, \ref{DsDt=Dt+s}, the composite along the top row is homotopic to the map of interest, and factors through $(\caldup{t}P^{t+q})(A)$, which is $q$-connected by lemma \ref{connectivityOfDerivedPowers}.
\end{proof}







\appendix
\section{Iterated simplicial bar constructions}\label{sec:ItSimpBar}
\newcommand{\algCat}{\calA}
\newcommand{\trip}[3]{{#1}_{\smash{#2}}^{\smash{#3}}}
In this section we will be concerned with iterated simplicial bar constructions. Before getting into the details, let's establish a little notation. Fix a category of universal algebras $\algCat$. For any simplicial object $X$ in $\algCat$, we'll write 
\[\trip{d}{i,q}{X}:X_q\to X_{q-1}\textup{\ and\ }\trip{s}{i,q}{X}:X_q\to X_{q+1}\]
for the $i^\textup{th}$ face and degeneracy maps out of $X_q$. Suppose $F,G\in\algCat^\algCat$ are endofunctors, $\Phi:F\to G$ is a natural transformation, and $A,B\in\algCat$ are objects. Write $[\Phi]:\algCat(A,B)\to\algCat(FA,GB)$ for the operator sending $m:A\to B$ to the diagonal composite in the commuting square
\[\xymatrix@R=4mm{
FA
\ar[r]^-{\Phi_A}
\ar[d]_-{Fm}
\ar[dr]|-{[\Phi]m}
&%r1c1
GA
\ar[d]^-{Gm}
\\%r1c2
FB
\ar[r]_-{\Phi_B}
&%r2c1
GB
%r2c2
}\]
Write $T:\algCat\to \algCat$ for the comonad of the adjunction $\textup{free}:\algCat\rightleftarrows\mathsf{Vect}:\textup{forget}$. Then there is an (augmented) simplicial endofunctor, $\frakb\in s\algCat^\algCat$, derived from the unit and counit of the adjunction:
\[\vcenter{
\def\labelstyle{\scriptstyle}
\xymatrix@C=1.65cm@1{
{\ I\,}
&
\,T^1\,
\ar@{..>}[l]|(.65){\frakd_{0,0}}
\ar[r]|(.65){\fraks_{0,0}}
&
\,T^2\,
\ar@<-1ex>[l]|(.65){\frakd_{0,1}}
\ar@<+1ex>[l]|(.65){\frakd_{1,1}}
\ar@<+1ex>[r]|(.65){\fraks_{0,1}}
\ar@<-1ex>[r]|(.65){\fraks_{1,1}}
&
\,T^3\,
\ar[l]|(.65){\frakd_{1,2}}
\ar@<-2ex>[l]|(.65){\frakd_{0,2}}
\ar@<+2ex>[l]|(.65){\frakd_{2,2}}
\ar[r]|(.65){\fraks_{1,2}}
\ar@<+2ex>[r]|(.65){\fraks_{0,2}}
\ar@<-2ex>[r]|(.65){\fraks_{2,2}}
&
\,T^4\,\makebox[0cm][l]{\,$\cdots $}
\ar@<-3ex>[l]|(.65){\frakd_{0,3}}
\ar@<-1ex>[l]|(.65){\frakd_{1,3}}
\ar@<+1ex>[l]|(.65){\frakd_{2,3}}
\ar@<+3ex>[l]|(.65){\frakd_{3,3}}
}}\]
The simplicial bar construction is the cofibrant replacement functor $(b,\epsilon)$ on $s\algCat$ which is the diagonal of the bisimplicial object obtained by application of $\frakb$ levelwise. That is, for $X\in s\algCat$, $bX$ is the simplicial object with $(bX)_q:=T^{q+1}X_q$, and with
\[\trip{d}{i,q}{bX}:=[\frakd_{i,q}]\trip{d}{i,q}{X}.\]
The augmentation $\epsilon:b\to I$ is defined on level $q$ by 
\[\epsilon_q=\frakd_{0,0}\frakd_{0,1}\cdots \frakd_{0,q}:T^{q+1}\to I\]
\begin{prop}\label{IteratedBarConstructionHomotopy}
For any integers $s,t\geq1$, there is a simplicial homotopy $b^{s-1}\epsilon_{(b^{t}X)}\simeq b^{s}\epsilon_{(b^{t-1}X)}$ of maps $b^{s+t}X\to b^{s+t-1}X$, natural in $X\in s\algCat$.
\end{prop}
\begin{proof}
Write $K=b^{s+t}X$ and $L=b^{s+t-1}X$ for the source and target of these maps respectively. Noting the formulae
\[[\frakd_{iq}]^2= [\frakd_{q+i,2q}\circ\frakd_{i,2q+1}]\ \textup{and}\  [\fraks_{iq}]^2= [\fraks_{q+i+2,2q+2}\circ\fraks_{i,2q+1}],\]
we can describe the simplicial structure maps in $K$ and $L$ as follows:
\begin{alignat*}{2}
\trip{d}{iq}{L}&=[\frakd_{iq}]^{s+t-1}\trip{d}{iq}{X}\\
\trip{s}{iq}{L}&=[\fraks_{iq}]^{s+t-1}\trip{s}{iq}{X}\\
\trip{d}{iq}{K}&=[\frakd_{iq}]^{s-1}[\frakd_{q+i,2q}\circ\frakd_{i,2q+1}][\frakd_{iq}]^{t-1}\trip{d}{iq}{X}\\
\trip{s}{iq}{K}&=[\fraks_{iq}]^{s-1}[\fraks_{q+i+2,2q+2}\circ\fraks_{i,2q+1}][\fraks_{iq}]^{t-1}\trip{s}{iq}{X}
\end{alignat*}
We can now state an explicit simplicial homotopy between the two maps of interest. Using precisely the notation of \cite[\S5]{MaySimpObj.pdf}, we define $\trip{h}{jq}{}:K_q\to L_{q+1}$, for $0\leq i\leq q$, by the formula
\[\trip{h}{jq}{}:=[\fraks_{jq}]^{s-1}[\frakd_{j+1,q+2}\circ\cdots \circ\frakd_{j+1,2q+1}][\fraks_{jq}]^{t-1}\trip{s}{jq}{X}\]
If the $\trip{h}{jq}{}$ in fact form a valid homotopy $f\simeq g$, then: %$f_q=\trip{d}{0,q+1}{L}\trip{h}{0,q}{}$, and $f_q=\trip{d}{q+1,q+1}{L}\trip{h}{q,q}{}$. We may calculate these composites as follows:
\begin{alignat*}{2}
f_q&=\trip{d}{0,q+1}{L}\trip{h}{0,q}{}\\
&=
[\frakd_{0,q+1}\fraks_{0q}]^{s-1}[\frakd_{0,q+1}\frakd_{1,q+2}\circ\cdots \circ\frakd_{1,2q+1}][\frakd_{0,q+1}\fraks_{0q}]^{t-1}(\trip{d}{0,q+1}{X}\trip{s}{0q}{X})%
\\
&=
[\trip{\textup{id}}{T^{q+1}}{}]^{s-1}[\frakd_{0,q+1}\frakd_{0,q+2}\circ\cdots \circ\frakd_{0,2q+1}][\trip{\textup{id}}{T^{q+1}}{}]^{t-1}\trip{\textup{id}}{X_q}{}%
\end{alignat*}
which is exactly the action of  $b^{s-1}\epsilon_{(b^{t}X)}$ in level $q$. Similarly,
\begin{alignat*}{2}
g_q&=\trip{d}{q+1,q+1}{L}\trip{h}{q,q}{}\\
&=
[\frakd_{q+1,q+1}\fraks_{qq}]^{s-1}[\frakd_{q+1,q+1}\frakd_{q+1,q+2}\circ\cdots \circ\frakd_{q+1,2q+1}][\frakd_{q+1,q+1}\fraks_{qq}]^{t-1}(\trip{d}{q+1,q+1}{X}\trip{s}{qq}{X})
\\
&=
[\trip{\textup{id}}{T^{q+1}}{}]^{s-1}[\frakd_{q+1,q+1}\frakd_{q+1,q+2}\circ\cdots \circ\frakd_{q+1,2q+1}][\trip{\textup{id}}{T^{q+1}}{}]^{t-1}\trip{\textup{id}}{X_q}{}%
\end{alignat*}
is exactly the action of $b^{s}\epsilon_{(b^{t-1}X)}$ in level $q$.
All that remains is to verify that the $h_{i,q}$ satisfy the defining identities for the notion of simplicial homotopy. While forming each of the composites of interest, none of the groupings provided by the square brackets are disturbed. Thus, each identity (labeled 1-5) can be checked one piece (labeled a-b) at a time:
{\renewcommand{\circ}{\relax}
\begin{enumerate}\squishlist
\setlength{\parindent}{.25in}
\item $\trip{d}{i,q+1}{L}\circ \trip{h}{j,q}{}=\trip{h}{j-1,q-1}{}\circ \trip{d}{i,q}{K}$ whenever $0\leq i<j\leq q$
\begin{enumerate}\squishlist
\setlength{\parindent}{.25in}
\item $\trip{d}{i,q+1}{X}\circ \trip{s}{j,q}{X}=\trip{s}{j-1,q-1}{X}\circ \trip{d}{i,q}{X}$ and $\trip{\frakd}{i,q+1}{}\circ \trip{\fraks}{j,q}{}=\trip{\fraks}{j-1,q-1}{}\circ \trip{\frakd}{i,q}{}$
\item 
$\frakd_{i,q+1}\circ
\frakd_{j+1,q+2}\circ\cdots \circ\frakd_{j+1,2q+1}=
\frakd_{j,q+1}\circ\cdots \circ\frakd_{j,2q-1}\circ
\frakd_{q+i,2q}\circ\frakd_{i,2q+1}$
\end{enumerate}
\item $\trip{d}{j+1,q+1}{L}\circ \trip{h}{j,q}{}=\trip{d}{j+1,q+1}{L}\circ \trip{h}{j+1,q}{}$ whenever $0\leq j\leq q-1$
\begin{enumerate}\squishlist
\setlength{\parindent}{.25in}
\item $\trip{d}{j+1,q+1}{X}\circ \trip{s}{j,q}{X}=\trip{d}{j+1,q+1}{X}\circ \trip{s}{j+1,q}{X}$ and $\trip{\frakd}{j+1,q+1}{}\circ \trip{\fraks}{j,q}{}=\trip{\frakd}{j+1,q+1}{}\circ \trip{\fraks}{j+1,q}{}$
\item 
$\frakd_{j+1,q+1}\circ
\frakd_{j+1,q+2}\circ\cdots \circ\frakd_{j+1,2q+1}=
\frakd_{j+1,q+1}\circ
\frakd_{j+2,q+2}\circ\cdots \circ\frakd_{j+2,2q+1}$
\end{enumerate}
\item $\trip{d}{i,q+1}{L}\circ \trip{h}{j,q}{}=\trip{h}{j,q-1}{}\circ \trip{d}{i-1,q}{K}$ whenever $0\leq j<i-1\leq q$
\begin{enumerate}\squishlist
\setlength{\parindent}{.25in}
\item $\trip{d}{i,q+1}{X}\circ \trip{s}{j,q}{X}=\trip{s}{j,q-1}{X}\circ \trip{d}{i-1,q}{X}$ and $\trip{\frakd}{i,q+1}{}\circ \trip{\fraks}{j,q}{}=\trip{\fraks}{j,q-1}{}\circ \trip{\frakd}{i-1,q}{}$
\item 
$\frakd_{i,q+1}\circ
\frakd_{j+1,q+2}\circ\cdots \circ\frakd_{j+1,2q+1}=
\frakd_{j+1,q+1}\circ\cdots \circ\frakd_{j+1,2q-1}\circ
\frakd_{q+i-1,2q}\circ\frakd_{i-1,2q+1}$
\end{enumerate}
\item $\trip{s}{i,q+1}{L}\circ \trip{h}{j,q}{}=\trip{h}{j+1,q+1}{}\circ \trip{s}{i,q}{K}$ whenever $0\leq i\leq j\leq q$
\begin{enumerate}\squishlist
\setlength{\parindent}{.25in}
\item $\trip{s}{i,q+1}{X}\circ \trip{s}{j,q}{X}=\trip{s}{j+1,q+1}{X}\circ \trip{s}{i,q}{X}$ and $\trip{\fraks}{i,q+1}{}\circ \trip{\fraks}{j,q}{}=\trip{\fraks}{j+1,q+1}{}\circ \trip{\fraks}{i,q}{}$
\item 
$\fraks_{i,q+1}\circ
\frakd_{j+1,q+2}\circ\cdots \circ\frakd_{j+1,2q+1}=
\frakd_{j+2,q+3}\circ\cdots \circ\frakd_{j+2,2q+3}\circ
\fraks_{q+i+2,2q+2}\circ\fraks_{i,2q+1}$
\end{enumerate}
\item $\trip{s}{i,q+1}{L}\circ \trip{h}{j,q}{}=\trip{h}{j,q+1}{}\circ \trip{s}{i-1,q}{K}$ whenever $0\leq j<i\leq q+1$
\begin{enumerate}\squishlist
\setlength{\parindent}{.25in}
\item $\trip{s}{i,q+1}{X}\circ \trip{s}{j,q}{X}=\trip{s}{j,q+1}{X}\circ \trip{s}{i-1,q}{X}$ and $\trip{\fraks}{i,q+1}{}\circ \trip{\fraks}{j,q}{}=\trip{\fraks}{j,q+1}{}\circ \trip{\fraks}{i-1,q}{}$
\item 
$\fraks_{i,q+1}\circ
\frakd_{j+1,q+2}\circ\cdots \circ\frakd_{j+1,2q+1}=
\frakd_{j+1,q+3}\circ\cdots \circ\frakd_{j+1,2q+3}\circ
\fraks_{q+i+1,2q+2}\circ\fraks_{i-1,2q+1}$
\end{enumerate}
\end{enumerate}
}
\noindent Each of these equations follows from the simplicial identities.
\end{proof}



\printbibliography

\end{document}

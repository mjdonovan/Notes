% !TEX root = z_output/conn_simp_algs_are_complete.tex
\documentclass[11pt]{amsart}
\usepackage{fullpage}
\usepackage{amsmath,amsthm,amssymb}
\usepackage{mathrsfs,nicefrac}
\usepackage{amssymb}
\usepackage{epsfig}
\usepackage[all,2cell]{xy}
\usepackage{sseq}
\usepackage{tocloft}
\usepackage{cancel}
\usepackage[strict]{changepage}
\usepackage{color}
\usepackage{tikz}
\usepackage{extpfeil}
\usepackage{version}
\usepackage{framed}
\definecolor{shadecolor}{rgb}{.925,0.925,0.925}

%\usepackage{ifthen}
%Used for disabling hyperref
\ifx\dontloadhyperref\undefined
%\usepackage[pdftex,pdfborder={0 0 0 [1 1]}]{hyperref}
\usepackage[pdftex,pdfborder={0 0 .5 [1 1]}]{hyperref}
\else
\providecommand{\texorpdfstring}[2]{#1}
\fi
%>>>>>>>>>>>>>>>>>>>>>>>>>>>>>>
%<<<        Versions        <<<
%>>>>>>>>>>>>>>>>>>>>>>>>>>>>>>
%Add in the following line to include all the versions.
%\def\excludeversion#1{\includeversion{#1}}

%>>>>>>>>>>>>>>>>>>>>>>>>>>>>>>
%<<<       Better ToC       <<<
%>>>>>>>>>>>>>>>>>>>>>>>>>>>>>>
\setlength{\cftbeforesecskip}{0.5ex}

%>>>>>>>>>>>>>>>>>>>>>>>>>>>>>>
%<<<      Hyperref mod      <<<
%>>>>>>>>>>>>>>>>>>>>>>>>>>>>>>

%needs more testing
\newcounter{dummyforrefstepcounter}
\newcommand{\labelRIGHTHERE}[1]
{\refstepcounter{dummyforrefstepcounter}\label{#1}}


%>>>>>>>>>>>>>>>>>>>>>>>>>>>>>>
%<<<  Theorem Environments  <<<
%>>>>>>>>>>>>>>>>>>>>>>>>>>>>>>
\ifx\dontloaddefinitionsoftheoremenvironments\undefined
\theoremstyle{plain}
\newtheorem{thm}{Theorem}[section]
\newtheorem*{thm*}{Theorem}
\newtheorem{lem}[thm]{Lemma}
\newtheorem*{lem*}{Lemma}
\newtheorem{prop}[thm]{Proposition}
\newtheorem*{prop*}{Proposition}
\newtheorem{cor}[thm]{Corollary}
\newtheorem*{cor*}{Corollary}
\newtheorem{defprop}[thm]{Definition-Proposition}
\newtheorem*{punchline}{Punchline}
\newtheorem*{conjecture}{Conjecture}
\newtheorem*{claim}{Claim}

\theoremstyle{definition}
\newtheorem{defn}{Definition}[section]
\newtheorem*{defn*}{Definition}
\newtheorem{exmp}{Example}[section]
\newtheorem*{exmp*}{Example}
\newtheorem*{exmps*}{Examples}
\newtheorem*{nonexmp*}{Non-example}
\newtheorem{asspt}{Assumption}[section]
\newtheorem{notation}{Notation}[section]
\newtheorem{exercise}{Exercise}[section]
\newtheorem*{fact*}{Fact}
\newtheorem*{rmk*}{Remark}
\newtheorem{fact}{Fact}
\newtheorem*{aside}{Aside}
\newtheorem*{question}{Question}
\newtheorem*{answer}{Answer}

\else\relax\fi

%>>>>>>>>>>>>>>>>>>>>>>>>>>>>>>
%<<<      Fields, etc.      <<<
%>>>>>>>>>>>>>>>>>>>>>>>>>>>>>>
\DeclareSymbolFont{AMSb}{U}{msb}{m}{n}
\DeclareMathSymbol{\N}{\mathbin}{AMSb}{"4E}
\DeclareMathSymbol{\Octonions}{\mathbin}{AMSb}{"4F}
\DeclareMathSymbol{\Z}{\mathbin}{AMSb}{"5A}
\DeclareMathSymbol{\R}{\mathbin}{AMSb}{"52}
\DeclareMathSymbol{\Q}{\mathbin}{AMSb}{"51}
\DeclareMathSymbol{\PP}{\mathbin}{AMSb}{"50}
\DeclareMathSymbol{\I}{\mathbin}{AMSb}{"49}
\DeclareMathSymbol{\C}{\mathbin}{AMSb}{"43}
\DeclareMathSymbol{\A}{\mathbin}{AMSb}{"41}
\DeclareMathSymbol{\F}{\mathbin}{AMSb}{"46}
\DeclareMathSymbol{\G}{\mathbin}{AMSb}{"47}
\DeclareMathSymbol{\Quaternions}{\mathbin}{AMSb}{"48}


%>>>>>>>>>>>>>>>>>>>>>>>>>>>>>>
%<<<       Operators        <<<
%>>>>>>>>>>>>>>>>>>>>>>>>>>>>>>
\DeclareMathOperator{\ad}{\textbf{ad}}
\DeclareMathOperator{\coker}{coker}
\renewcommand{\ker}{\textup{ker}\,}
\DeclareMathOperator{\End}{End}
\DeclareMathOperator{\Aut}{Aut}
\DeclareMathOperator{\Hom}{Hom}
\DeclareMathOperator{\Maps}{Maps}
\DeclareMathOperator{\Mor}{Mor}
\DeclareMathOperator{\Gal}{Gal}
\DeclareMathOperator{\Ext}{Ext}
\DeclareMathOperator{\Tor}{Tor}
\DeclareMathOperator{\Map}{Map}
\DeclareMathOperator{\Der}{Der}
\DeclareMathOperator{\Rad}{Rad}
\DeclareMathOperator{\rank}{rank}
\DeclareMathOperator{\ArfInvariant}{Arf}
\DeclareMathOperator{\KervaireInvariant}{Ker}
\DeclareMathOperator{\im}{im}
\DeclareMathOperator{\coim}{coim}
\DeclareMathOperator{\trace}{tr}
\DeclareMathOperator{\supp}{supp}
\DeclareMathOperator{\ann}{ann}
\DeclareMathOperator{\spec}{Spec}
\DeclareMathOperator{\SPEC}{\textbf{Spec}}
\DeclareMathOperator{\proj}{Proj}
\DeclareMathOperator{\PROJ}{\textbf{Proj}}
\DeclareMathOperator{\fiber}{F}
\DeclareMathOperator{\cofiber}{C}
\DeclareMathOperator{\cone}{cone}
\DeclareMathOperator{\skel}{sk}
\DeclareMathOperator{\coskel}{cosk}
\DeclareMathOperator{\conn}{conn}
\DeclareMathOperator{\colim}{colim}
\DeclareMathOperator{\limit}{lim}
\DeclareMathOperator{\ch}{ch}
\DeclareMathOperator{\Vect}{Vect}
\DeclareMathOperator{\GrthGrp}{GrthGp}
\DeclareMathOperator{\Sym}{Sym}
\DeclareMathOperator{\Prob}{\mathbb{P}}
\DeclareMathOperator{\Exp}{\mathbb{E}}
\DeclareMathOperator{\GeomMean}{\mathbb{G}}
\DeclareMathOperator{\Var}{Var}
\DeclareMathOperator{\Cov}{Cov}
\DeclareMathOperator{\Sp}{Sp}
\DeclareMathOperator{\Seq}{Seq}
\DeclareMathOperator{\Cyl}{Cyl}
\DeclareMathOperator{\Ev}{Ev}
\DeclareMathOperator{\sh}{sh}
\DeclareMathOperator{\intHom}{\underline{Hom}}
\DeclareMathOperator{\Frac}{frac}



%>>>>>>>>>>>>>>>>>>>>>>>>>>>>>>
%<<<   Cohomology Theories  <<<
%>>>>>>>>>>>>>>>>>>>>>>>>>>>>>>
\DeclareMathOperator{\KR}{{K\R}}
\DeclareMathOperator{\KO}{{KO}}
\DeclareMathOperator{\K}{{K}}
\DeclareMathOperator{\OmegaO}{{\Omega_{\Octonions}}}

%>>>>>>>>>>>>>>>>>>>>>>>>>>>>>>
%<<<   Algebraic Geometry   <<<
%>>>>>>>>>>>>>>>>>>>>>>>>>>>>>>
\DeclareMathOperator{\Spec}{Spec}
\DeclareMathOperator{\Proj}{Proj}
\DeclareMathOperator{\Sing}{Sing}
\DeclareMathOperator{\shfHom}{\mathscr{H}\textit{\!\!om}}
\DeclareMathOperator{\WeilDivisors}{{Div}}
\DeclareMathOperator{\CartierDivisors}{{CaDiv}}
\DeclareMathOperator{\PrincipalWeilDivisors}{{PrDiv}}
\DeclareMathOperator{\LocallyPrincipalWeilDivisors}{{LPDiv}}
\DeclareMathOperator{\PrincipalCartierDivisors}{{PrCaDiv}}
\DeclareMathOperator{\DivisorClass}{{Cl}}
\DeclareMathOperator{\CartierClass}{{CaCl}}
\DeclareMathOperator{\Picard}{{Pic}}
\DeclareMathOperator{\Frob}{Frob}


%>>>>>>>>>>>>>>>>>>>>>>>>>>>>>>
%<<<  Mathematical Objects  <<<
%>>>>>>>>>>>>>>>>>>>>>>>>>>>>>>
\newcommand{\sll}{\mathfrak{sl}}
\newcommand{\gl}{\mathfrak{gl}}
\newcommand{\GL}{\mbox{GL}}
\newcommand{\PGL}{\mbox{PGL}}
\newcommand{\SL}{\mbox{SL}}
\newcommand{\Mat}{\mbox{Mat}}
\newcommand{\Gr}{\textup{Gr}}
\newcommand{\Squ}{\textup{Sq}}
\newcommand{\catSet}{\textit{Sets}}
\newcommand{\RP}{{\R\PP}}
\newcommand{\CP}{{\C\PP}}
\newcommand{\Steen}{\mathscr{A}}
\newcommand{\Orth}{\textup{\textbf{O}}}

%>>>>>>>>>>>>>>>>>>>>>>>>>>>>>>
%<<<  Mathematical Symbols  <<<
%>>>>>>>>>>>>>>>>>>>>>>>>>>>>>>
\newcommand{\DASH}{\textup{---}}
\newcommand{\op}{\textup{op}}
\newcommand{\CW}{\textup{CW}}
\newcommand{\ob}{\textup{ob}\,}
\newcommand{\ho}{\textup{ho}}
\newcommand{\st}{\textup{st}}
\newcommand{\id}{\textup{id}}
\newcommand{\Bullet}{\ensuremath{\bullet} }
\newcommand{\sprod}{\wedge}

%>>>>>>>>>>>>>>>>>>>>>>>>>>>>>>
%<<<      Some Arrows       <<<
%>>>>>>>>>>>>>>>>>>>>>>>>>>>>>>
\newcommand{\nt}{\Longrightarrow}
\let\shortmapsto\mapsto
\let\mapsto\longmapsto
\newcommand{\mapsfrom}{\,\reflectbox{$\mapsto$}\ }
\newcommand{\bigrightsquig}{\scalebox{2}{\ensuremath{\rightsquigarrow}}}
\newcommand{\bigleftsquig}{\reflectbox{\scalebox{2}{\ensuremath{\rightsquigarrow}}}}

%\newcommand{\cofibration}{\xhookrightarrow{\phantom{\ \,{\sim\!}\ \ }}}
%\newcommand{\fibration}{\xtwoheadrightarrow{\phantom{\sim\!}}}
%\newcommand{\acycliccofibration}{\xhookrightarrow{\ \,{\sim\!}\ \ }}
%\newcommand{\acyclicfibration}{\xtwoheadrightarrow{\sim\!}}
%\newcommand{\leftcofibration}{\xhookleftarrow{\phantom{\ \,{\sim\!}\ \ }}}
%\newcommand{\leftfibration}{\xtwoheadleftarrow{\phantom{\sim\!}}}
%\newcommand{\leftacycliccofibration}{\xhookleftarrow{\ \ {\sim\!}\,\ }}
%\newcommand{\leftacyclicfibration}{\xtwoheadleftarrow{\sim\!}}
%\newcommand{\weakequiv}{\xrightarrow{\ \,\sim\,\ }}
%\newcommand{\leftweakequiv}{\xleftarrow{\ \,\sim\,\ }}

\newcommand{\cofibration}
{\xhookrightarrow{\phantom{\ \,{\raisebox{-.3ex}[0ex][0ex]{\scriptsize$\sim$}\!}\ \ }}}
\newcommand{\fibration}
{\xtwoheadrightarrow{\phantom{\raisebox{-.3ex}[0ex][0ex]{\scriptsize$\sim$}\!}}}
\newcommand{\acycliccofibration}
{\xhookrightarrow{\ \,{\raisebox{-.55ex}[0ex][0ex]{\scriptsize$\sim$}\!}\ \ }}
\newcommand{\acyclicfibration}
{\xtwoheadrightarrow{\raisebox{-.6ex}[0ex][0ex]{\scriptsize$\sim$}\!}}
\newcommand{\leftcofibration}
{\xhookleftarrow{\phantom{\ \,{\raisebox{-.3ex}[0ex][0ex]{\scriptsize$\sim$}\!}\ \ }}}
\newcommand{\leftfibration}
{\xtwoheadleftarrow{\phantom{\raisebox{-.3ex}[0ex][0ex]{\scriptsize$\sim$}\!}}}
\newcommand{\leftacycliccofibration}
{\xhookleftarrow{\ \ {\raisebox{-.55ex}[0ex][0ex]{\scriptsize$\sim$}\!}\,\ }}
\newcommand{\leftacyclicfibration}
{\xtwoheadleftarrow{\raisebox{-.6ex}[0ex][0ex]{\scriptsize$\sim$}\!}}
\newcommand{\weakequiv}
{\xrightarrow{\ \,\raisebox{-.3ex}[0ex][0ex]{\scriptsize$\sim$}\,\ }}
\newcommand{\leftweakequiv}
{\xleftarrow{\ \,\raisebox{-.3ex}[0ex][0ex]{\scriptsize$\sim$}\,\ }}

%>>>>>>>>>>>>>>>>>>>>>>>>>>>>>>
%<<<    xymatrix Arrows     <<<
%>>>>>>>>>>>>>>>>>>>>>>>>>>>>>>
\newdir{ >}{{}*!/-5pt/@{>}}
\newcommand{\xycof}{\ar@{ >->}}
\newcommand{\xycofib}{\ar@{^{(}->}}
\newcommand{\xycofibdown}{\ar@{_{(}->}}
\newcommand{\xyfib}{\ar@{->>}}
\newcommand{\xymapsto}{\ar@{|->}}

%>>>>>>>>>>>>>>>>>>>>>>>>>>>>>>
%<<<     Greek Letters      <<<
%>>>>>>>>>>>>>>>>>>>>>>>>>>>>>>
%\newcommand{\oldphi}{\phi}
%\renewcommand{\phi}{\varphi}
\let\oldphi\phi
\let\phi\varphi
\renewcommand{\to}{\longrightarrow}
\newcommand{\from}{\longleftarrow}
\newcommand{\eps}{\varepsilon}

%>>>>>>>>>>>>>>>>>>>>>>>>>>>>>>
%<<<  1st-4th & parentheses <<<
%>>>>>>>>>>>>>>>>>>>>>>>>>>>>>>
\newcommand{\first}{^\text{st}}
\newcommand{\second}{^\text{nd}}
\newcommand{\third}{^\text{rd}}
\newcommand{\fourth}{^\text{th}}
\newcommand{\ZEROTH}{$0^\text{th}$ }
\newcommand{\FIRST}{$1^\text{st}$ }
\newcommand{\SECOND}{$2^\text{nd}$ }
\newcommand{\THIRD}{$3^\text{rd}$ }
\newcommand{\FOURTH}{$4^\text{th}$ }
\newcommand{\iTH}{$i^\text{th}$ }
\newcommand{\jTH}{$j^\text{th}$ }
\newcommand{\nTH}{$n^\text{th}$ }

%>>>>>>>>>>>>>>>>>>>>>>>>>>>>>>
%<<<    upright commands    <<<
%>>>>>>>>>>>>>>>>>>>>>>>>>>>>>>
\newcommand{\upcol}{\textup{:}}
\newcommand{\upsemi}{\textup{;}}
\providecommand{\lparen}{\textup{(}}
\providecommand{\rparen}{\textup{)}}
\renewcommand{\lparen}{\textup{(}}
\renewcommand{\rparen}{\textup{)}}
\newcommand{\Iff}{\emph{iff} }

%>>>>>>>>>>>>>>>>>>>>>>>>>>>>>>
%<<<     Environments       <<<
%>>>>>>>>>>>>>>>>>>>>>>>>>>>>>>
\newcommand{\squishlist}
{ %\setlength{\topsep}{100pt} doesn't seem to do anything.
  \setlength{\itemsep}{.5pt}
  \setlength{\parskip}{0pt}
  \setlength{\parsep}{0pt}}
\newenvironment{itemise}{
\begin{list}{\textup{$\rightsquigarrow$}}
   {  \setlength{\topsep}{1mm}
      \setlength{\itemsep}{1pt}
      \setlength{\parskip}{0pt}
      \setlength{\parsep}{0pt}
   }
}{\end{list}\vspace{-.1cm}}
\newcommand{\INDENT}{\textbf{}\phantom{space}}
\renewcommand{\INDENT}{\rule{.7cm}{0cm}}

\newcommand{\itm}[1][$\rightsquigarrow$]{\item[{\makebox[.5cm][c]{\textup{#1}}}]}


%\newcommand{\rednote}[1]{{\color{red}#1}\makebox[0cm][l]{\scalebox{.1}{rednote}}}
%\newcommand{\bluenote}[1]{{\color{blue}#1}\makebox[0cm][l]{\scalebox{.1}{rednote}}}

\newcommand{\rednote}[1]
{{\color{red}#1}\makebox[0cm][l]{\scalebox{.1}{\rotatebox{90}{?????}}}}
\newcommand{\bluenote}[1]
{{\color{blue}#1}\makebox[0cm][l]{\scalebox{.1}{\rotatebox{90}{?????}}}}


\newcommand{\funcdef}[4]{\begin{align*}
#1&\to #2\\
#3&\mapsto#4
\end{align*}}

%\newcommand{\comment}[1]{}

%>>>>>>>>>>>>>>>>>>>>>>>>>>>>>>
%<<<       Categories       <<<
%>>>>>>>>>>>>>>>>>>>>>>>>>>>>>>
\newcommand{\Ens}{{\mathscr{E}ns}}
\DeclareMathOperator{\Sheaves}{{\mathsf{Shf}}}
\DeclareMathOperator{\Presheaves}{{\mathsf{PreShf}}}
\DeclareMathOperator{\Psh}{{\mathsf{Psh}}}
\DeclareMathOperator{\Shf}{{\mathsf{Shf}}}
\DeclareMathOperator{\Varieties}{{\mathsf{Var}}}
\DeclareMathOperator{\Schemes}{{\mathsf{Sch}}}
\DeclareMathOperator{\Rings}{{\mathsf{Rings}}}
\DeclareMathOperator{\AbGp}{{\mathsf{AbGp}}}
\DeclareMathOperator{\Modules}{{\mathsf{\!-Mod}}}
\DeclareMathOperator{\fgModules}{{\mathsf{\!-Mod}^{\textup{fg}}}}
\DeclareMathOperator{\QuasiCoherent}{{\mathsf{QCoh}}}
\DeclareMathOperator{\Coherent}{{\mathsf{Coh}}}
\DeclareMathOperator{\GSW}{{\mathcal{SW}^G}}
\DeclareMathOperator{\Burnside}{{\mathsf{Burn}}}
\DeclareMathOperator{\GSet}{{G\mathsf{Set}}}
\DeclareMathOperator{\FinGSet}{{G\mathsf{Set}^\textup{fin}}}
\DeclareMathOperator{\HSet}{{H\mathsf{Set}}}
\DeclareMathOperator{\Cat}{{\mathsf{Cat}}}
\DeclareMathOperator{\Fun}{{\mathsf{Fun}}}
\DeclareMathOperator{\Orb}{{\mathsf{Orb}}}
\DeclareMathOperator{\Set}{{\mathsf{Set}}}
\DeclareMathOperator{\sSet}{{\mathsf{sSet}}}
\DeclareMathOperator{\Top}{{\mathsf{Top}}}
\DeclareMathOperator{\GSpectra}{{G-\mathsf{Spectra}}}
\DeclareMathOperator{\Lan}{Lan}
\DeclareMathOperator{\Ran}{Ran}

%>>>>>>>>>>>>>>>>>>>>>>>>>>>>>>
%<<<     Script Letters     <<<
%>>>>>>>>>>>>>>>>>>>>>>>>>>>>>>
\newcommand{\scrQ}{\mathscr{Q}}
\newcommand{\scrW}{\mathscr{W}}
\newcommand{\scrE}{\mathscr{E}}
\newcommand{\scrR}{\mathscr{R}}
\newcommand{\scrT}{\mathscr{T}}
\newcommand{\scrY}{\mathscr{Y}}
\newcommand{\scrU}{\mathscr{U}}
\newcommand{\scrI}{\mathscr{I}}
\newcommand{\scrO}{\mathscr{O}}
\newcommand{\scrP}{\mathscr{P}}
\newcommand{\scrA}{\mathscr{A}}
\newcommand{\scrS}{\mathscr{S}}
\newcommand{\scrD}{\mathscr{D}}
\newcommand{\scrF}{\mathscr{F}}
\newcommand{\scrG}{\mathscr{G}}
\newcommand{\scrH}{\mathscr{H}}
\newcommand{\scrJ}{\mathscr{J}}
\newcommand{\scrK}{\mathscr{K}}
\newcommand{\scrL}{\mathscr{L}}
\newcommand{\scrZ}{\mathscr{Z}}
\newcommand{\scrX}{\mathscr{X}}
\newcommand{\scrC}{\mathscr{C}}
\newcommand{\scrV}{\mathscr{V}}
\newcommand{\scrB}{\mathscr{B}}
\newcommand{\scrN}{\mathscr{N}}
\newcommand{\scrM}{\mathscr{M}}

%>>>>>>>>>>>>>>>>>>>>>>>>>>>>>>
%<<<     Fractur Letters    <<<
%>>>>>>>>>>>>>>>>>>>>>>>>>>>>>>
\newcommand{\frakQ}{\mathfrak{Q}}
\newcommand{\frakW}{\mathfrak{W}}
\newcommand{\frakE}{\mathfrak{E}}
\newcommand{\frakR}{\mathfrak{R}}
\newcommand{\frakT}{\mathfrak{T}}
\newcommand{\frakY}{\mathfrak{Y}}
\newcommand{\frakU}{\mathfrak{U}}
\newcommand{\frakI}{\mathfrak{I}}
\newcommand{\frakO}{\mathfrak{O}}
\newcommand{\frakP}{\mathfrak{P}}
\newcommand{\frakA}{\mathfrak{A}}
\newcommand{\frakS}{\mathfrak{S}}
\newcommand{\frakD}{\mathfrak{D}}
\newcommand{\frakF}{\mathfrak{F}}
\newcommand{\frakG}{\mathfrak{G}}
\newcommand{\frakH}{\mathfrak{H}}
\newcommand{\frakJ}{\mathfrak{J}}
\newcommand{\frakK}{\mathfrak{K}}
\newcommand{\frakL}{\mathfrak{L}}
\newcommand{\frakZ}{\mathfrak{Z}}
\newcommand{\frakX}{\mathfrak{X}}
\newcommand{\frakC}{\mathfrak{C}}
\newcommand{\frakV}{\mathfrak{V}}
\newcommand{\frakB}{\mathfrak{B}}
\newcommand{\frakN}{\mathfrak{N}}
\newcommand{\frakM}{\mathfrak{M}}

\newcommand{\frakq}{\mathfrak{q}}
\newcommand{\frakw}{\mathfrak{w}}
\newcommand{\frake}{\mathfrak{e}}
\newcommand{\frakr}{\mathfrak{r}}
\newcommand{\frakt}{\mathfrak{t}}
\newcommand{\fraky}{\mathfrak{y}}
\newcommand{\fraku}{\mathfrak{u}}
\newcommand{\fraki}{\mathfrak{i}}
\newcommand{\frako}{\mathfrak{o}}
\newcommand{\frakp}{\mathfrak{p}}
\newcommand{\fraka}{\mathfrak{a}}
\newcommand{\fraks}{\mathfrak{s}}
\newcommand{\frakd}{\mathfrak{d}}
\newcommand{\frakf}{\mathfrak{f}}
\newcommand{\frakg}{\mathfrak{g}}
\newcommand{\frakh}{\mathfrak{h}}
\newcommand{\frakj}{\mathfrak{j}}
\newcommand{\frakk}{\mathfrak{k}}
\newcommand{\frakl}{\mathfrak{l}}
\newcommand{\frakz}{\mathfrak{z}}
\newcommand{\frakx}{\mathfrak{x}}
\newcommand{\frakc}{\mathfrak{c}}
\newcommand{\frakv}{\mathfrak{v}}
\newcommand{\frakb}{\mathfrak{b}}
\newcommand{\frakn}{\mathfrak{n}}
\newcommand{\frakm}{\mathfrak{m}}

%>>>>>>>>>>>>>>>>>>>>>>>>>>>>>>
%<<<  Caligraphic Letters   <<<
%>>>>>>>>>>>>>>>>>>>>>>>>>>>>>>
\newcommand{\calQ}{\mathcal{Q}}
\newcommand{\calW}{\mathcal{W}}
\newcommand{\calE}{\mathcal{E}}
\newcommand{\calR}{\mathcal{R}}
\newcommand{\calT}{\mathcal{T}}
\newcommand{\calY}{\mathcal{Y}}
\newcommand{\calU}{\mathcal{U}}
\newcommand{\calI}{\mathcal{I}}
\newcommand{\calO}{\mathcal{O}}
\newcommand{\calP}{\mathcal{P}}
\newcommand{\calA}{\mathcal{A}}
\newcommand{\calS}{\mathcal{S}}
\newcommand{\calD}{\mathcal{D}}
\newcommand{\calF}{\mathcal{F}}
\newcommand{\calG}{\mathcal{G}}
\newcommand{\calH}{\mathcal{H}}
\newcommand{\calJ}{\mathcal{J}}
\newcommand{\calK}{\mathcal{K}}
\newcommand{\calL}{\mathcal{L}}
\newcommand{\calZ}{\mathcal{Z}}
\newcommand{\calX}{\mathcal{X}}
\newcommand{\calC}{\mathcal{C}}
\newcommand{\calV}{\mathcal{V}}
\newcommand{\calB}{\mathcal{B}}
\newcommand{\calN}{\mathcal{N}}
\newcommand{\calM}{\mathcal{M}}

%>>>>>>>>>>>>>>>>>>>>>>>>>>>>>>
%<<<<<<<<<DEPRECIATED<<<<<<<<<<
%>>>>>>>>>>>>>>>>>>>>>>>>>>>>>>

%%% From Kac's template
% 1-inch margins, from fullpage.sty by H.Partl, Version 2, Dec. 15, 1988.
%\topmargin 0pt
%\advance \topmargin by -\headheight
%\advance \topmargin by -\headsep
%\textheight 9.1in
%\oddsidemargin 0pt
%\evensidemargin \oddsidemargin
%\marginparwidth 0.5in
%\textwidth 6.5in
%
%\parindent 0in
%\parskip 1.5ex
%%\renewcommand{\baselinestretch}{1.25}

%%% From the net
%\newcommand{\pullbackcorner}[1][dr]{\save*!/#1+1.2pc/#1:(1,-1)@^{|-}\restore}
%\newcommand{\pushoutcorner}[1][dr]{\save*!/#1-1.2pc/#1:(-1,1)@^{|-}\restore}









\usepackage{framed}
%\usepackage{biblatex} 
\usepackage[style=numeric,%citestyle=numeric,
url=false,doi=false,isbn=false,eprint=false]{biblatex}%
\hypersetup{colorlinks=false,pdfborder={0 0 0}}



\makeatletter
\renewcommand{\@seccntformat}[1]{\csname the#1\endcsname.\quad}
\makeatother


\theoremstyle{plain}
\newtheorem{theorem}{Theorem}
\newtheorem*{completenesstheorem}{Completeness Theorem}
\newtheorem{twistinglemma}[thm]{Twisting Lemma}
\renewcommand*{\thetheorem}{\Alph{theorem}}
\newtheorem{conjectureAlpha}{Conjecture}
\renewcommand*{\theconjectureAlpha}{\Alph{conjectureAlpha}}



\newcommand{\NOFULLPAGE}{\relax}

\newcommand{\Sq}{\mathrm{Sq}}
\newcommand{\Comm}{\calC}

\bibliography{../../Dropbox/logbook/_LOGBOOK/papers}


\title{Connected simplicial algebras are Andr\'e-Quillen complete}
\author{Michael Donovan}





%\newcommand{\dupdown}[2]{D^{\smash{#1}}_{\smash{#2}}}
\newcommand{\dupdown}[2]{D_{\smash{#1}}}
\newcommand{\caldup}[1]{\calD_{\smash{#1}}}
\newcommand{\caldupdown}[2]{\calD^{\smash{#1}}_{\smash{#2}}}

\begin{document}

\maketitle

\begin{abstract}
We modify a classical construction of Bousfield and Kan \cite{BousKanSSeq.pdf} to define the Adams tower of a simplicial nonunital commutative algebra over a field $k$. We relate this construction to Radulescu-Banu's cosimplicial resolution \cite{Radulescu-Banu.pdf}, and prove that all connected simplicial algebras are complete with respect to Andr\'e-Quillen homology. This is a convergence result for the unstable Adams spectral sequence for commutative algebras over $k$.
\end{abstract}

Let $s\calC$ denote the simplicial model category \cite{QuillenHomAlg.pdf} of simplicial non-unital commutative algebras over a field $k$, and let $X$ be an object of $s\calC$. Radulescu-Banu \cite{Radulescu-Banu.pdf} has given a cosimplicial resolution $\calX^\bullet$ of $X$ by generalised Eilenberg-Mac Lane objects, and defines the \emph{completion of $X$ with respect to Andr\'e-Quillen homology} to be the totalization $X\hat{\ }:=\textup{Tot}(\calX^\bullet)$. The purpose of the present work is to prove the following conjecture of Radulescu-Banu:
\begin{completenesstheorem}\label{completenesstheorem}
If $X$ is a connected simplicial $k$-algebra, then $X$ is naturally equivalent to its completion $X\hat{\ }$.
\end{completenesstheorem}
Radulescu-Banu's completion functor $X\mapsto X\hat{\ }$ is the analogue of Bousfield and Kan's $R$-completion functor on topological spaces \cite{BousKanSSeq.pdf}, which has proven extremely useful in classical homotopy theory. As in the classical case, the homotopy of the completion $X\hat{\ }$ is the target of an unstable Adams spectral sequence. The completeness theorem may be viewed as a convergence result for this spectral sequence, which we will study in detail in forthcoming work.

In fact, Bousfield and Kan have defined the unstable Adams spectral sequence of a space in two different ways. Their earlier approach \cite{BK_pairings.pdf} was to define the \emph{derivation of a functor with respect to a ring}. This approach constructs the \emph{Adams tower} over the space in question, and lends itself well to connectivity analyses. Their latter approach, \cite{BousKanSSeq.pdf}, to give a cosimplicial resolution of a space by simplicial $R$-modules, lends itself more to the analysis of the $E^2$ page, and is directly analogous to Radulescu-Banu's construction.

Since the release of \cite{BK_pairings.pdf} and \cite{BousKanSSeq.pdf}, the relationship between the two approaches has been clarified by the introduction of the theory of cubical diagrams to homotopy theory [Munson-Voli\'c]. Our approach to proving the completeness theorem will be to define the Adams tower of a simplicial algebra using a construction analogous to Bousfield and Kan in \cite{BK_pairings.pdf} (Section \ref{sec:derWRTab}), and to use the general theory of cubical diagrams to relate it to Radulescu-Banu's construction (Section \ref{sec:relnWithRB}). In Section \ref{sec:connectivityAnalysis}, we perform the necessary connectivity estimates in the Adams tower in order to prove the completeness theorem. 

As Radulescu-Banu observed, there is an additional difficulty in constructing the Bousfield-Kan cosimplicial resolution of a simplicial algebra which is not present in the classical context. Namely, since not all simplicial algebras are cofibrant, the naive cosimplicial resolution will not be homotopically correct. Radulescu-Banu's innovation was to explain that the cofibrant replacement functor $c:s\calC\to s\calC$ constructed by Quillen's small object argument \cite{QuillenHomAlg.pdf} admits a comonad diagonal $\psi:c\to cc$, and can thus be mixed into the cosimplicial resolution, making it homotopically correct. 

Accordingly, we must mix the SOA functor $c$ into our definition of the Adams tower over a simplicial algebra so that it relates as desired to Radulescu-Banu's resolution. The application of these cofibrant replacement functors adds to the difficulty of proving the connectivity estimates of Section \ref{sec:connectivityAnalysis}. We circumvent this difficulty by shifting to the standard comonadic bar construction.

In Section \ref{introToRBwork} we will introduce Radulescu-Banu's cosimplicial resolution and explain a little terminology. In the appendix we will state and prove a useful result on iterated simplicial bar constructions.

The author would like to thank Haynes Miller for his guidance over the course of this research, and John Harper for explaining to him the use of cubical diagrams in connectivity analyses such as that of Section \ref{sec:connectivityAnalysis}.


\section{Radulescu-Banu's completion functor}\label{introToRBwork}
A non-unital commutative $k$-algebra is a $k$-vector space $Y$ equipped with an associative and commutative map $\mu:Y\otimes Y\to Y$. We will refer to such objects simply as \emph{algebras}. It can be shown that if $Y\in\calC$ is a categorical group object, then it is in fact a zero-square algebra, that is  $Y^2=\im(\mu)=0$, and the group map $Y\times Y\to Y$ is simply the vector space addition. Thus, the abelianization adjunction for $\calC$ can be modeled as $Q:s\calC\rightleftarrows s\mathsf{Vect}:K$, in which the left adjoint is the \emph{abelianization} or \emph{indecomposables} functor
\[Q:\calC\to \mathsf{Vect},\qquad Y\mapsto Y/Y^2,\]
and the right adjoint is the \emph{zero-square} functor
\[K:\mathsf{Vect}\to\calC,\qquad V\mapsto V\textup{ with $V^2$ set to zero}.\]
The Andr\'e-Quillen homology of a simplicial algebra $X$ is defined as the homotopy groups of its left derived abelianization:
\[H_*X:=\pi_*(QcX),\]
where $c:s\calC\to s\calC$ is the cofibrant replacement from Quillen's SOA. For an excellent introduction to these ideas, see \cite[\S4]{MR1089001}. Radulescu-Banu constructed \cite{Radulescu-Banu.pdf} a comonad diagonal $\psi:c\to cc$, in order to define the coface maps in a (coaugmented) cosimplicial object:
\[\makebox[0cm][r]{\,$\calX^\bullet:\qquad $}\vcenter{
\def\labelstyle{\scriptstyle}
\xymatrix@C=1.5cm@1{
cX\,
\ar[r]
&
\,cKQcX\,
\ar[r];[]
&
\,c(KQc)^2X\,
\ar@<-1ex>[l];[]
\ar@<+1ex>[l];[]
\ar@<+1ex>[r];[]
\ar@<-1ex>[r];[]
&
\,c(KQc)^3X\,\makebox[0cm][l]{\,$\cdots. $}
\ar[l];[]
\ar@<-2ex>[l];[]
\ar@<+2ex>[l];[]
}}\]
The construction is explained and generalized by Blumberg and Riehl \cite{BlumRiehlResolutions.pdf}. The definition of the coface and codegeneracy maps is similar to that in the monadic resolution of $X$ using the adjunction $Q\dashv K$, however, the coface maps must create an extra copy of $c$, and the codegeneracies must destroy a copy.  Instead of simply using the unit and counit of the the adjunction respectively, one uses the composites
\[c\overset{\psi}{\to}cc\overset{c\eta c}{\to}cKQc\textup{\quad and\quad }QcK\overset{Q\epsilon K}{\to}QK\to \textup{id}.\]


Denote by $X\hat{\ }$ the totalisation of $\calX^\bullet$, naming $X\hat{\ }$ the \emph{completion of $X$ with respect to Andr\'e-Quillen homology}. There is a natural zig-zag completion map $X\overset{\sim}{\from} cX\to X\hat{\ }$, and one says that $X$ is \emph{complete with respect to Andr\'e-Quillen homology} when the map $cX\to X\hat{\ }$ is an equivalence. We will prove that this occurs whenever $X$ is connected. In order to justify the name of the completion functor, note that upon taking Andr\'e-Quillen homology of $\calX^\bullet$, one has a cosimplicial vector space which is weakly equivalent to its coagmentation $H_*X$ (cf. \cite{BlumRiehlResolutions.pdf}).






%
%
%\section*{leftovers}
%
%
% $Q$ is the functor $QX=X/X^2$, and $KV$ is the zero-square algebra on the vector space $V$.
%
%a group object of  The group objects of $\calC$ are all abelian, and The simplicial model category $s\calC$
%
%
%Above, and in what follows, the functor $c$ denotes the cofibrant replacement functor produced by Quillen's small object argument \cite{QuillenHomAlg.pdf}. Radulescu-Banu's innovation was to mix the cofibrant replacement functor $c$ into the naive cosimplicial resolution of a simplicial algebra obtained from the abelianization adjunction $Q:s\calC\rightleftarrows s\mathsf{Vect}:K$, in which $QX=X/X^2$, and $KV$ is the zero-square algebra on the vector space $V$. The Andr\'e-Quillen homology of simplicial algebras is defined as the left derived functors of $Q$, which explains the title `completion with respect to Andr\'e-Quillen homology'. Goerss gives in \cite[\S4]{MR1089001} an excellent introduction to Andr\'e-Quillen homology.
%
%More precisely, he constructs a comonad diagonal $c\to cc$ (with counit the augmentation $\epsilon: c\to I$) in order to construct coface maps in the coaugmented cosimplicial object
%\[\makebox[0cm][r]{\,$\calX^\bullet:\qquad $}\vcenter{
%\def\labelstyle{\scriptstyle}
%\xymatrix@C=1.5cm@1{
%cX\,
%\ar[r]
%&
%\,cKQcX\,
%\ar[r];[]
%&
%\,c(KQc)^2X\,
%\ar@<-1ex>[l];[]
%\ar@<+1ex>[l];[]
%\ar@<+1ex>[r];[]
%\ar@<-1ex>[r];[]
%&
%\,c(KQc)^3X\,\makebox[0cm][l]{\,$\cdots. $}
%\ar[l];[]
%\ar@<-2ex>[l];[]
%\ar@<+2ex>[l];[]
%}}\]
%Blumberg and Riehl \cite{BlumRiehlResolutions.pdf} have extended this approach, building on Bousfield's general theory of completions \cite{BousCosimpResnHtpySS.pdf}.


\section{The Adams tower}\label{sec:derWRTab}
%In this paper we work in the category $\calC$ of nonunital commutative algebras over a field $k$. These are simply $k$-vector spaces $A$ equipped with a commutative and associative pairing $A\otimes A\to A$. The category $s\calC$ of simplicial algebras is a model category (cf.\ \cite{QuillenHomAlg.pdf}), and we let $c:s\calC\to s\calC$ be a functorial cofibrant replacement produced by Quillen's small object argument.




For any functor $F:s\calC\to s\calC$, we define the $r^\textup{th}$ derivation $\dupdown{r}{b}F$ of $F$ with respect to Andr\'e-Quillen homology. The definition is recursive:
\begin{alignat*}{2}
(\dupdown{0}{b}F)(X)
&:=
F(cX)%
\\
(\dupdown{s}{b}F)(X)
&:=
\hofib((\dupdown{s-1}{c}F)(cX)\xrightarrow{(\dupdown{s-1}{c}F)(\eta_{cX})} (\dupdown{s-1}{c}F)(KQcX))
\end{alignat*}
where $\eta$ is the unit of the adjunction $Q\dashv K$, i.e.\ the natural surjection onto indecomposables, and $\hofib$ is any fixed  functorial construction of the homotopy fiber. These functors fit into a tower via the following composite natural transformation:
\[\delta:\left((\dupdown{s}{c}F)(X)\to (\dupdown{s-1}{c}F)(cX)\overset{(\dupdown{s-1}{c}F)(\epsilon)}{\to} (\dupdown{s-1}{c}F)(X)\right).\]
We have thus constructed a tower
\[\xymatrix@R=4mm{
\cdots 
\ar[r]
&%r1c1
(\dupdown{2}{c}F)X
\ar[r]
&%r1c3
%r1c4
(\dupdown{1}{c}F)X
\ar[r]&%r1c5
(\dupdown{0}{c}F)X=FcX,
}\]
which is natural in the object $X$ and the functor $F$.
The functors $\dupdown{r}{c}F$ are homotopical as long as $F$ preserves weak equivalences between cofibrant objects. Employing the shorthand
\[\dupdown{s}{c}X:=(\dupdown{s}{c}I)X,\]
we define \emph{the Adams tower of $X$} to be the tower
\[\xymatrix@R=4mm{
\cdots 
\ar[r]
&%r1c1
\dupdown{2}{c}X
\ar[r]
&%r1c3
%r1c4
\dupdown{1}{c}X
\ar[r]&%r1c5
\dupdown{0}{c}X=cX.
}\]
%For clarity, we'll depict the construction of $(\dupdown{1}{c}F)(X)$, $(\dupdown{2}{c}F)(X)$ and the map between them in the following diagram. Here, every composable pair of parallel arrows is \emph{defined} to be a homotopy fiber sequence. The upper horizontal plane defines $(\dupdown{2}{c}F)(X)$.....
%%@=50pt
%\[\def\labelstyle{\scriptstyle}
%\xymatrix@!0@R=30pt@C=50pt{
%(\dupdown{2}{c}F)(X)\ar[dr]\ar[dd]_\delta\\
%&(\dupdown{1}{c}F)(cX) \ar[rr]\ar[rd]\ar[dd]^(.75){(\dupdown{1}{c}F)(\epsilon)}         &           &FcccX \ar[rr]\ar[rd]\ar'[d][dd]^{Fcc\epsilon}         &           &   FcKQccX \ar'[d][dd]^{FcKQc\epsilon}           \ar[dr]  &                  \\
%(\dupdown{1}{c}F)(X)\ar[dr]&        &  (\dupdown{1}{c}F)(KQcX) \ar[rr] \ar[dd]  &                     &  FccKQcX \ar[rr] \ar[dd]  &             & FcKQcKQcX
%         \ar[dd] \\
%&(\dupdown{1}{c}F)(X) \ar'[r][rr] \ar[dr] &        &   FccX \ar'[r][rr] \ar[dr] &        &   FcKQcX
%\ar[dr] &                   \\
%&        &   0 \ar[rr]      &               &   0 \ar[rr]      &                    &
%0
%}\]

For example, $(\dupdown{2}{c}F)(X)$ is constructed by the following diagram in which every composable pair of parallel arrows is \emph{defined} to be a homotopy fiber sequence.
\[\def\labelstyle{\scriptstyle}
\xymatrix@!0@R=30pt@C=43pt{
(\dupdown{2}{c}F)(X)\ar[d]\\
(\dupdown{1}{c}F)(cX) \ar[rr]\ar[d]         &           &FcccX \ar[rr]\ar[d]         &           &   FcKQccX            \ar[d]  &                  \\
(\dupdown{1}{c}F)(KQcX) \ar[rr] &                     &  FccKQcX \ar[rr] &             & FcKQcKQcX
}\]
In general, $(\dupdown{n+1}{c}F)(X)$ will be the homotopy total fiber of an $(n+1)$-cubical diagram (see \cite{GoodwillieCalcII}, \cite{LuisGoodwillie.pdf} or Munson-Voli\'c \textbf{(?)}). To set notation, for $n\geq0$ let $[n]=\{0,\ldots,n\}$ and define
\[\calP_0[n]
=
\left\{S\subset [n]\,\middle|\, S\neq\emptyset\right\}\quad \textup{and}\quad \calP[n]=\left\{S\subset [n]\right\}.\]
These sets are partially ordered by inclusion, and we think of them as poset categories. The category $\calP_0[n]$ indexes punctured $(n+1)$-cubical diagrams, and $\calP_0[n]$ indexes $(n+1)$-cubical diagrams.

We may observe that $(D_{n+1}F)X$ is simply (the iterative definition of) the homotopy total fiber of the following $(n+1)$-cubical diagram $({D}_{n+1}^{\smash{\square}}F)X:\calP[n]\to s\calC$:
\[(({D}_{n+1}^{\smash{\square}}F)X)(S):= Fc(KQ)^{\chi_{n}}c(KQ)^{\chi_{n-1}}c\cdots c(KQ)^{\chi_0}cX\quad \textup{where}\quad \chi_i:=\begin{cases}
1,&\textup{if }i\in S;\\
0,&\textup{if }i\notin S,
%\\,&\textup{if }
\end{cases}
\]
where for $S\subset S'$, the map $(({D}_{n+1}^{\smash{\square}}F)X)(S)\to (({D}_{n+1}^{\smash{\square}}F)X)(S')$ is given by applying the counit $\eta:1\to KQ$ in those locations indexed by $S'\setminus S$.

\section{Relationship between the Adams tower and Radulescu-Banu's resolution}\label{sec:relnWithRB}

Radulescu-Banu defines the completion of $X$ to be the totalization
\[X\hat{\ }:=\textup{Tot}(\calX^\bullet)=\holim (\textup{Tot}_n(\calX^\bullet)),\]
and the unstable Adams spectral sequence to be the spectral sequence of the tower
\[\cdots \to\textup{Tot}_n(\calX^\bullet)\to \textup{Tot}_{n-1}(\calX^\bullet)\to\cdots \]
under $cX$. Our goal in this section is to prove
\begin{prop}\label{towerIdentification}
There is a natural equivalence  $D_{n+1}X\sim\hofib(cX\to\textup{Tot}_{n}(\calX^\bullet))$ for all $n\geq-1$, compatible with the tower maps. That is, the $\textup{Tot}$ tower induces the Adams tower by taking fibers.
\end{prop}
Before giving the proof, we recall a useful relationship between cosimplicial objects and cubical diagrams, explained by Sinha in \cite[Theorem 6.5]{SinhaSpacesOfKnots.pdf}, and expanded on by Munson-Voli\'c\textbf{(?)}.
We will only present that part of the theory that we need. There is a diagram of inclusions of categories
\[\xymatrix@R=4mm{
\calP[-1]\ar[r]^-{\tau}\ar_-{h_{-1}}[drr]
&%r1c1
\calP[0]\ar[r]^-{\tau}\ar|-{h_{0}}[dr]
&%r1c2
\calP[1]\ar[r]^-{\tau}\ar|-{h_{1}}[d]
&%r1c3
\calP[2]\ar[r]^-{\tau}\ar^-{h_{2}}[dl]
&%r1c4
\cdots \\
&&\Delta_+
}\]
As the augmented cosimplicial object $\calX^\bullet$ is Reedy fibrant (cf.\ \cite[{X.4.9}]{YellowMonster}), there are natural equivalences $\hofib(\calX^{-1}\to\textup{Tot}_n\calX^\bullet)\sim \textup{hototfib}(h_n^*(\calX^\bullet))$ under which the tower map 
\[\hofib(\calX^{-1}\to\textup{Tot}_n\calX^\bullet)\to \hofib(\calX^{-1}\to\textup{Tot}_{n-1}\calX^\bullet)\]
is identified with the map
\[\textup{hototfib}(h_n^*(\calX^\bullet))\to \textup{hototfib}(\tau^*h_n^*(\calX^\bullet))=\textup{hototfib}(h_{n-1}^*(\calX^\bullet)).\]

\begin{proof}[Proof of Proposition \ref{towerIdentification}]
It will suffice to construct a weak equivalence $h_n^*\calX^\bullet\to (D_n^{\smash{\square}}I)(X)$ of $(n+1)$-cubes. The $(n+1)$-cubical diagram $h_n^*\calX^\bullet$ is defined by
\[(h_n^*\calX^\bullet)(S):= c(KQc)^{\chi_{n}}(KQc)^{\chi_{n-1}}\cdots (KQc)^{\chi_0}X\quad \textup{where}\quad \chi_i:=\begin{cases}
1,&\textup{if }i\in S;\\
0,&\textup{if }i\notin S,
%\\,&\textup{if }
\end{cases}
\]
where we describe the map $(h_n^*\calX^\bullet)(S)\to (h_n^*\calX^\bullet)(S\sqcup\{i\})$, for $i\notin S$, as follows. Let $j$ be the smallest element of $S\sqcup\{n+1\}$ exceeding $i$, so that
\[(h_n^*\calX^\bullet)(S):= \begin{cases}
c(KQc)^{\chi_{n}}\cdots (KQc)^{\chi_{j+1}}(KQ\underline{c})(KQc)^{\chi_{i-1}}\cdots (KQc)^{\chi_0}X,&\textup{if }j\leq n;\\
\underline{c}(KQc)^{\chi_{i-1}}\cdots (KQc)^{\chi_0}X,&\textup{if }j=n+1.
%\\,&\textup{if }
\end{cases}
\]
In the expression for either case, we have distinguished one of the applications of $c$ with an underline, and the map to $(h_n^*\calX^\bullet)(S\sqcup\{i\})$ is induced by the composite $\underline{c}\to cc\to cKQc$ of the diagonal of the comonad $c$ with the unit of the monad $KQ$. %This cube only commutes due to the coassociativity of the diagonal $\psi$.

%The reason that this cube commutes is (of course) the same as the reason that the cosimplicial relation $d^j\circ d^i=d^i\circ d^{j-1}$ ($i<j$) is satisfied in $\calX^\bullet$. That is, $c$ is a comonad, and as such its diagonal map is coassociative.

We now define maps $(h_n^*\calX^\bullet)(S)\to (({D}_{n+1}^{\smash{\square}}I)X)(S)$ for $S=\{j_0<j_1<\cdots<j_r\}\subset\{0,\ldots,n\}$. 
The only difference between the domain and codomain is that in $(({D}_{n+1}^{\smash{\square}}I)X)(S)$, all $n+2$ applications of $c$ are present, whereas in $(h_n^*\calX^\bullet)(S)$, only $r+2$ appear. The map is then
\[\psi^{n-j_r}KQ\psi^{j_r-j_{r-1}-1}KQ\psi^{j_{r-1}-j_{r-2}-1}KQ\cdots KQ\psi^{j_{1}-j_0-1}KQ\psi^{j_0}X\]
which is to say that we apply the iterated diagonal the appropriate number of times in each $c$ appearing in the domain. This is unambiguous as $\psi$ is coassociative. As $\psi$ is coassociative, these maps assemble to a weak equivalence of $(n+1)$-cubes. 
%For example, when $n=2$, we have defined $\dupdown{2}{c}X$ to be the homotopy total fiber of the 2-cube
%\[\xymatrix@R=4mm{
%cccX\ar[r]
%\ar[d]
%&%r1c1
%cKQccX\ar[d]
%\\%r1c2
%ccKQcX\ar[r]
%&%r2c1
%cKQcKQcX%r2c2
%}\]
%which receives a weak equivalence from the 2-cube
%\[\xymatrix@R=4mm{
%cX\ar[r]
%\ar[d]
%&%r1c1
%cKQcX\ar[d]
%\\%r1c2
%cKQcX\ar[r]
%&%r2c1
%cKQcKQcX%r2c2
%}\]
%which defines the homotopy total fiber of $\textup{hototfib}(h_1^*\calX^\bullet)\sim \hofib(cX\to\textup{Tot}_1(\calX^\bullet))$. The weak equivalence of cubes is given by applying the diagonal map $c\to cc$ as appropriate. The final result will commute, even for large $n$, due to the coassociativity of the diagonal.
\end{proof}
%\begin{shaded}\tiny
%\begin{itemise}
%\setlength{\parindent}{.25in}
%\item The spectral sequence may be defined to be either the spectral sequence of the bicomplex or the spectral sequence of the Tot tower, thanks to \cite[Lemma 2.2]{BousfieldHSSCS.pdf}.
%\item It converges to the totalization of the resolution, which we make the definition of $X\hat{\ }$.
%\end{itemise}
%\end{shaded}

\section{Connectivity estimates in the Adams tower}
\label{sec:connectivityAnalysis}
We wish to prove the following connectivity result for the Adams tower:
\begin{prop}\label{convergenceProp}
For integers $t\geq2$ and $q\geq0$, the map $\pi_q(\dupdown{2t+q-1}{c}X)\to\pi_q(\dupdown{t}{c}X)$ is zero.
\end{prop}
\noindent We defer the the proof until the end of this section. Note that proposition \ref{towerIdentification} and \ref{convergenceProp} together imply the completeness theorem:
\begin{proof}[Proof of completeness theorem]
The fiber sequences $\dupdown{n+1}{c}X\to cX\to \textup{Tot}_n\calX^\bullet$ fit together into a tower of fiber sequences. Taking homotopy limits, one obtains a fiber sequence
\[\holim (\dupdown{n}{c}X)\to cX\to X\hat{\ }.\]
We need to show that $\holim (\dupdown{n}{c}X)$ has zero homotopy groups.
We may replace the tower $\dupdown{n}{c}X$ with a weakly equivalent tower of fibrations in $s\calC$ whose set-theoretic inverse limit is the homotopy limit in question. Applying \cite[Proposition 6.14]{goerss-jardine.pdf} to the new tower, there is a short exact sequence
\[0\to \textup{lim}^1\pi_{q+1}(\dupdown{n}{c}X)\to \pi_q(\holim (\dupdown{n}{c}X))\to \textup{lim}\,\pi_{q}(\dupdown{n}{c}X)\to 0.\]
Proposition \ref{convergenceProp} implies that for each $q$, the tower $\{\pi_{q}(\dupdown{n}{c}X)\}$ has zero inverse limit, and satisfies the Mittag-Leffler condition (cf.\ \cite[p.264]{YellowMonster}), so that the $\textup{lim}^1$ groups appearing also vanish.
\end{proof}
With an eye to proving proposition \ref{convergenceProp} we define a somewhat less homotopical version $\caldup{s}F$ of the derivations $\dupdown{s}{b}F$ of $F$. Again, the definition is recursive:
\begin{alignat*}{2}
(\caldup{0}F)(X)
&:=
F(X)%
\\
(\caldup{s}F)(X)
&:=
\ker((\caldup{s-1}F)(bX)\xrightarrow{(\caldup{s-1}F)(\eta_{bX})} (\caldup{s-1}F)(KQbX))
\end{alignat*}
There are three differences between this definition and that of $D_sF$: here, there is one fewer cofibrant replacement applied, we use the standard simplicial bar construction $b$ instead of the SOA functor $c$, and we take \emph{strict} fibers, not \emph{homotopy} fibers.
While these functors are not generally homotopical, we define \emph{the modified Adams tower of $X$} to be the tower
\[\xymatrix@R=4mm{
\cdots 
\ar[r]
&%r1c1
\caldup{2}X
\ar[r]
&%r1c3
%r1c4
\caldup{1}X
\ar[r]&%r1c5
\caldup{0}X=X,
}\]
where $\caldup{s}X$ is again shorthand for $(\caldup{s}I)X$, and the tower maps $\delta$ are defined as before.
\begin{prop}\label{prop:modifiedAdamsTower}
There's a natural zig-zag of weak equivalences of towers between the Adams tower of $X$ and the modified Adams tower of $X$. In particular, the modified tower is homotopical.
\end{prop}
\begin{proof}
Let $\mathsf{CR(s\calC)}$ be the category of cofibrant replacement functors in $s\calC$. That is, an object of $\mathsf{CR(s\calC)}$ is a pair, $(f,\epsilon)$, such that $f:s\calC\to s\calC$ is a functor whose image consists only of cofibrant objects, and $\epsilon:f\Rightarrow I$ is a natural acyclic fibration. Morphisms in $\mathsf{CR(s\calC)}$ are natural transformations which commute with the augmentations. For any $(f,\epsilon)\in\mathsf{CR(s\calC)}$ we obtain an alternative definition of the derivations of a functor $F:s\calC\to s\calC$:
\begin{alignat*}{2}
(D_{\smash{0}}^{\smash{f}}F)(X)
:=
F(fX),\quad %
(D_{\smash{s}}^{\smash{f}}F)(X)
:=
\hofib((D_{\smash{s-1}}^{\smash{f}}F)(fX)\to (D_{\smash{s-1}}^{\smash{f}}F)(KQfX)).
\end{alignat*}
These functors are natural in $f$, so that a morphism in $\mathsf{CR(s\calC)}$ induces a weak equivalence of towers. Our proposed zig-zag of towers is: %Now there is a zig-zag in $\mathsf{CR(s\calC)}$ from the SOA functor $c$ to the bar construction $b$, as both receive a map from the cofibrant replacement $b\circ c$. Thus we need only produce a zig-zag between the towers $\{(D_{\smash{s}}^{\smash{b}}I)X\}$ and $\{(\caldup{s}I)X\}$
\[D_{\smash{s}}= D_{\smash{s}}^{\smash{c}}I\overset{}{\from}D_{\smash{s}}^{\smash{b\circ c}}\!I\overset{}{\to}D_{\smash{s}}^{\smash{b}}I\overset{\gamma_s}{\from}\caldup{s}b\overset{\caldup{s}\epsilon}{\to}\caldup{s}I=\caldup{s}\]
The maps with domain $D_{\smash{s}}^{\smash{b\circ c}}I$ are induced by the maps $\epsilon c:b\circ c\to c$ and $b\epsilon:b\circ c\to b$ and are evidently natural weak equivalences of towers. The map $\gamma_0:(\caldup{0}b)X\to (D_{\smash{0}}^{\smash{b}}I)X$ is the identity of $bX$, and the map $\caldup{0}\epsilon:(\caldup{0}b)X\to (D_{\smash{0}}^{\smash{b}}I)X$ is $\epsilon:bX\to X$. Thereafter, $\gamma_s$ and $\caldup{s}\epsilon$ are defined recursively:
\[\qquad \quad \xymatrix@R=4mm{
\makebox[0cm][r]{$(\caldup{s+1}I)X:=\,$}\makebox[5cm][l]{$\,\ker((\caldup{s}I)(bX)\to (\caldup{s}I)(KQbX))$}\\
\makebox[0cm][r]{$(\caldup{s+1}b)X:=\,$}\makebox[5cm][l]{$\,\ker((\caldup{s}b)(bX)\to (\caldup{s}b)(KQbX))$}\ar[d]^-{\textup{incl.}}_-{}%m_2}
\ar[u]_-{\textup{induced by }(\caldup{s}\epsilon,\caldup{s}\epsilon)}^-{}%m_1}
\\%r1c1
\makebox[5cm][l]{$\,\hofib((\caldup{s}b)(bX)\to(\caldup{s}b)(KQbX))$}\ar[d]^-{\textup{induced by }(\gamma_{s},\gamma_{s})}_-{}%m_3}
\\%r2c1
\makebox[0cm][r]{$(D_{\smash{s+1}}^{\smash{b}})X:=\,$}\makebox[5cm][l]{$\,\hofib((D_{\smash{s}}^{\smash{b}}I)(bX)\to (D_{\smash{s}}^{\smash{b}}I)(KQbX))$}%r3c1
}\]
\noindent Lemma \ref{towerWithPowers} shows that the kernels taken are actually kernels of surjective maps, and by induction on $s$, the maps $\gamma_s$ and $\caldup{s}\epsilon$ are weak equivalences.
% The base case is trivial, and for $s\geq0$, the maps are defined recursively as follows:
%
%\noindent where the kernels taken are actually kernels of surjective maps, by lemma \ref{towerWithPowers}. Thus $m_2$ is an equivalence. Assuming the inductive hypothesis, the 5-lemma confirms that $m_1$ is an equivalence, and $m_3$ is an equivalence as $\hofib$ is homotopical.
\end{proof}
The connectivity result will rely on the observation that any element in the $s^\textup{th}$ level of the modified tower maps down to an $(s+1)$-fold product in $X$. In order to formalise this, let $P^s:s\calC\to s\calC$ be the ``$s^\textup{th}$ power'' functor, the prolongation of the endofunctor $Y\mapsto Y^s$ of $\calC$, where $Y^s=\textup{im}(\textup{mult}:Y^{\otimes s}\to Y)$. Then we have:
\begin{lem}\label{towerWithPowers}
The functors $\caldup{r}$, $\caldup{r}b$ and $\caldup{r}P^{s}$ preserve surjective maps and there is a commuting diagram of functors:
\[\xymatrix@R=4mm{
\cdots 
\ar[r]
&%r1c1
\caldup{r}
\ar[r]
\ar[d]
&%r1c2
\cdots \ar[r]
&%r1c4
\caldup{2}
\ar[r]
\ar[d]
&%r1c4
\caldup{1}
\ar[r]
\ar[d]&%r1c5
\caldup{0}
\ar@{=}[d]
\\%r1c6
\cdots
\ar[r]
&%r2c1
P^{r+1}
\ar[r]
&%r2c3
\cdots 
\ar[r]&%r2c4
P^3
\ar[r]
&P^2
\ar[r]
&%r2c5
I%r2c6
}\]
\end{lem}
\begin{proof}

Suppose that $X$ is a simplicial algebra. Then $\caldup{r}X$
is constructed as the subalgebra
\[\caldup{r}X:= \bigcap_{i=1}^{r}\ker\!\left(b^{r-i}\eta b^{i}:b^{r}X\to b^{r-i}KQb^{i}X\right)\]
of $b^rX$. In dimension $n$, this is the following subset of $(b^rX)_n:=(T^{n+1})^rX_n$:
\[(\caldup{r}X)_n:=\bigcap_{i=1}^{r}\ker\!\left(T^{(r-i)(n+1)}\eta T^{i(n+1)}:(T^{n+1})^rX_n\to (T^{n+1})^{r-i}KQ(T^{n+1})^iX_n\right)\]
The $r$ conditions on elements of $(\caldup{r}X)_n$ ensure that their image in $(\caldup{0}X)_n=X_n$ under the iterated tower map is a sum of $(r+1)$-fold products. This completes the construction of the tower of functors.

In order to prove the surjectivity statements, we must describe the iterated free construction $T^{r(n+1)}X_n$. A basis of $TX_n$ may be given by the \emph{monomials} in a basis of $X_n$, and $T^{r(n+1)}X_n$ has basis given by taking monomials iteratively, $r(n+1)$ times. The subset $(\caldup{r}X)_n$ has basis those iterated monomials in which the monomials formed in the $((n+1)i)^\textup{th}$ iteration have degree at least two for each $1\leq i\leq r$. This simple description of a basis of $(\caldup{r}X)_n$ shows that $\caldup{r}$ preserves surjections. Similar analysis applies to $\caldup{r}b$ and $\caldup{r}P^s$.
\end{proof}
We are now able to state and prove the key connectivity result:
\begin{lem}\label{connectivityOfDerivedPowers}
For $A\in s\Comm$ connected, and any $t\geq1$ and $s\geq2$, $(\caldup{t}P^{s})(A)$ is $(s-t)$-connected.
\end{lem}
\begin{proof}
We will prove this by induction on $t$. When $t=1$:
\[(\caldup{1}P^{s})A:=\ker(P^{s}(bA)\to P^{s}(QbA))=P^{s}(bA),\]
so we must show that $P^s(bA)$ is $(s-1)$-connected. For this we use a truncation of Quillen's fundamental spectral sequence, as presented in \cite[thm 6.2]{MR1089001}. That is, filter $P^s(bA)$ by the ideals $P^{s+p}(bA)$ for $p\geq0$. Then there is a convergent spectral sequence:
\[E^1_{p,q}= \pi_q\bigl((QbA)^{\otimes (s+p)}_{\Sigma_{s+p}}\bigr)\implies \pi_q(P^s(bA)).\]
The simplicial vector space $QbA$ is connected, as $\pi_0(QbA)=Q(\pi_0bA)=Q(0)=0$. The $t=1$ case now follows from \cite[Satz 12.1]{DoldPuppeSuspension.pdf}: if $V$ is a connected simplicial vector space then $V^{\otimes n}_{\Sigma_n}$ is $(n-1)$-connected. 
%(Bousfield? Obvious? Whatever.) \textbf{You know the proof. You only need to do it for spheres, and then you end up getting $k\otimes ((S^1)^{\wedge n}_{\Sigma_n})$, which is the homology of something that's not too bad.}

Now let $t\geq2$ and suppose by induction that $(\caldup{t-1}P^{s})(B)$ is $(s-(t-1))$-connected for any connected $B$ and any $s\geq2$. Then by lemma \ref{towerWithPowers}, there's a short exact sequence:
\[\xymatrix{
0\ar[r]&
(\caldup{t}P^{s})(A)\ar[r]&
(\caldup{t-1}P^{s})(cA)\ar[r]&
(\caldup{t-1}P^{s})(QcA)\ar[r]&
0
}\]
in which the rightmost two objects are each $(s-t+1)$-connected. The associated long exact sequence shows that $(\caldup{t}P^{s})(A)$ is $(s-t)$-connected.
\end{proof}
Before we can give the proof of proposition \ref{convergenceProp}, we need the following \emph{twisting lemma}, analogous to that of \cite{BK_pairings.pdf}. Before stating it, we note that $\caldup{s}\caldup{t}(X)$ and $\caldup{s+t}X$ are equal by construction.
\begin{twistinglemma}
\label{DsDt=Dt+s}
The maps $\caldup{i}\delta:\caldup{n}X\to \caldup{n-1}X$ are homotopic for $0\leq i< n$.
\end{twistinglemma}
\begin{proof}
We may reindex the twisting lemma as follows: the maps 
\[\caldup{s}\delta,\caldup{s-1}\delta:\caldup{s+t}X\to \caldup{s+t-1}X\]
are homotopic whenever $s,t\geq1$. Now $\caldup{s+t}X$ is constructed as the subalgebra
\[\caldup{s+t}X:= \bigcap_{i=1}^{s+t}\ker\!\left(b^{s+t-i}\eta b^{i}:b^{s+t}X\to b^{s+t-i}KQb^{i}X\right)\]
of the iterated bar construction $b^{s+t}X$, and for $0\leq i<s+t$, $\caldup{i}\delta$ is the restriction of the map $b^i\epsilon b^{s+t-i-1}:b^{s+t}X\to b^{s+t-1}X$.
%
% there are commuting squares
%\[\xymatrix@R=7mm{
%b^{s+t}X
%\ar[r]_-{b^s\epsilon b^{t-1}}
%&%r1c1
%b^{s+t-1}X\\%r1c2
%\caldup{s+t}X
%\ar[r]^-{\caldup{s}\delta}
%\ar[u]
%&%r2c1
%\caldup{s+t-1}X
%\ar[u]
%%r2c2
%}\raisebox{-6mm}{\textup{\quad and\quad }}
%\xymatrix@R=7mm{
%b^{s+t}X
%\ar[r]_-{b^{s-1}\epsilon b^{t}}
%&%r1c1
%b^{s+t-1}X\\%r1c2
%\caldup{s+t}X
%\ar[r]^-{\caldup{s-1}\delta}
%\ar[u]
%&%r2c1
%\caldup{s+t-1}X
%\ar[u]
%%r2c2
%}\]
Proposition \ref{IteratedBarConstructionHomotopy} gives an explicit simplicial homotopy between the maps $b^s\epsilon b^{t-1}$ and $b^{s-1}\epsilon b^{t}$. Moreover, the naturality of the construction of proposition \ref{IteratedBarConstructionHomotopy} implies that this homotopy does indeed restrict to a homotopy of maps $\caldup{s+t}X\to \caldup{s+t-1}X$.%, and to simply note that the homotopy restricts to a homotopy between $\caldup{s}\delta$ and $\caldup{s-1}\delta$. Indeed, the homotopy between $b^s\epsilon b^{t-1}$ and $b^{s-1}\epsilon b^t$ is provided by proposition \ref{IteratedBarConstructionHomotopy}, and the naturality of its construction implies that it does indeed restrict as desired. 
\end{proof}


Now that we have the twisting lemma, Proposition \ref{convergenceProp} follows:
\begin{proof}[Proof of Proposition \ref{convergenceProp}]
%The first claim is \ref{towerIsHomotopical}. 
By proposition \ref{prop:modifiedAdamsTower}, it is enough to prove that $\pi_q(\caldup{2t+q-1}X)\to\pi_q(\caldup{t}X)$ is zero.
Apply $\caldup{t}\DASH$ to the diagram of functors constructed in \ref{towerWithPowers} and apply the result to $X$ to obtain a commuting diagram of functors
\[\xymatrix@R=4mm{
\caldup{2t+q-1}X
\ar[r]^-{\caldup{t}\delta}
\ar[d]
&
\cdots \ar[r]^-{\caldup{t}\delta}
&%r1c4
\caldup{t+1}X
\ar[r]^-{\caldup{t}\delta}
\ar[d]&%r1c5
\caldup{t}X
\ar@{=}[d]
\\%r1c6
\caldup{t}P^{t+q}X
\ar[r]
&%r2c3
\cdots 
\ar[r]&%r2c4
\caldup{t}P^2X
\ar[r]
&%r2c5
\caldup{t}P^1X%r2c6
}\]
By the twisting lemma, \ref{DsDt=Dt+s}, the composite along the top row is homotopic to the map of interest, and factors through $(\caldup{t}P^{t+q})(A)$, which is $q$-connected by lemma \ref{connectivityOfDerivedPowers}.
\end{proof}







\appendix
\section{Iterated simplicial bar constructions}\label{sec:ItSimpBar}
\newcommand{\algCat}{\calA}
\newcommand{\trip}[3]{{#1}_{\smash{#2}}^{\smash{#3}}}
In this section we will be concerned with iterated simplicial bar constructions. Before getting into the details, let's establish a little notation. Fix a category of universal algebras $\algCat$. For any simplicial object $X$ in $\algCat$, we'll write 
\[\trip{d}{i,q}{X}:X_q\to X_{q-1}\textup{\ and\ }\trip{s}{i,q}{X}:X_q\to X_{q+1}\]
for the $i^\textup{th}$ face and degeneracy maps out of $X_q$. Suppose $F,G\in\algCat^\algCat$ are endofunctors, $\Phi:F\to G$ is a natural transformation, and $A,B\in\algCat$ are objects. Write $[\Phi]:\algCat(A,B)\to\algCat(FA,GB)$ for the operator sending $m:A\to B$ to the diagonal composite in the commuting square
\[\xymatrix@R=4mm{
FA
\ar[r]^-{\Phi_A}
\ar[d]_-{Fm}
\ar[dr]|-{[\Phi]m}
&%r1c1
GA
\ar[d]^-{Gm}
\\%r1c2
FB
\ar[r]_-{\Phi_B}
&%r2c1
GB
%r2c2
}\]
Write $T:\algCat\to \algCat$ for the comonad of the adjunction $\textup{free}:\algCat\rightleftarrows\mathsf{Vect}:\textup{forget}$. Then there is an (augmented) simplicial endofunctor, $\frakb\in s\algCat^\algCat$, derived from the unit and counit of the adjunction:
\[\vcenter{
\def\labelstyle{\scriptstyle}
\xymatrix@C=1.65cm@1{
{\ I\,}
&
\,T^1\,
\ar@{..>}[l]|(.65){\frakd_{0,0}}
\ar[r]|(.65){\fraks_{0,0}}
&
\,T^2\,
\ar@<-1ex>[l]|(.65){\frakd_{0,1}}
\ar@<+1ex>[l]|(.65){\frakd_{1,1}}
\ar@<+1ex>[r]|(.65){\fraks_{0,1}}
\ar@<-1ex>[r]|(.65){\fraks_{1,1}}
&
\,T^3\,
\ar[l]|(.65){\frakd_{1,2}}
\ar@<-2ex>[l]|(.65){\frakd_{0,2}}
\ar@<+2ex>[l]|(.65){\frakd_{2,2}}
\ar[r]|(.65){\fraks_{1,2}}
\ar@<+2ex>[r]|(.65){\fraks_{0,2}}
\ar@<-2ex>[r]|(.65){\fraks_{2,2}}
&
\,T^4\,\makebox[0cm][l]{\,$\cdots $}
\ar@<-3ex>[l]|(.65){\frakd_{0,3}}
\ar@<-1ex>[l]|(.65){\frakd_{1,3}}
\ar@<+1ex>[l]|(.65){\frakd_{2,3}}
\ar@<+3ex>[l]|(.65){\frakd_{3,3}}
}}\]
The simplicial bar construction is the cofibrant replacement functor $(b,\epsilon)$ on $s\algCat$ which is the diagonal of the bisimplicial object obtained by application of $\frakb$ levelwise. That is, for $X\in s\algCat$, $bX$ is the simplicial object with $(bX)_q:=T^{q+1}X_q$, and with
\[\trip{d}{i,q}{bX}:=[\frakd_{i,q}]\trip{d}{i,q}{X}.\]
The augmentation $\epsilon:b\to I$ is defined on level $q$ by 
\[\epsilon_q=\frakd_{0,0}\frakd_{0,1}\cdots \frakd_{0,q}:T^{q+1}\to I\]
%\begin{prop}\label{IteratedBarConstructionHomotopy}
%For any integers $s,t\geq1$, there is a simplicial homotopy $b^{s-1}\epsilon_{(b^{t}X)}\simeq b^{s}\epsilon_{(b^{t-1}X)}$ of maps $b^{s+t}X\to b^{s+t-1}X$, natural in $X\in s\algCat$, such that whenever $0\leq i\leq s-1$ or $1\leq k<t$ there are commuting squares
%\[\xymatrix@R=4mm@C=15mm{
%(b^iKQb^{s+t-i}X)_q\ar@{..>}[d]
%&%r1c1
%(b^{s+t}X)_q\ar[d]^-{h_{jq}}
%\ar[l]_-{b^i\eta b^{s+t-i}}
%\ar[r]^-{b^{s+t-k}\eta b^{k}}
%&%r1c2
%(b^{s+t-k}KQb^kX)_q\ar@{..>}[d]\\%r1c3
%(b^iKQb^{s+t-i}X)_{q+1}&%r2c1
%(b^{s+t-1}X)_{q+1}
%\ar[l]_-{b^i\eta b^{s+t-1-i}}
%\ar[r]^-{b^{s+t-1-k}\eta b^{k}}&%r2c2
%(b^{s+t-1-k}KQb^kX)_{q+1}%r2c3
%}\]
%\end{prop}
We can now construct the simplicial homotopy needed for the twisting lemma, \ref{DsDt=Dt+s}.
\begin{prop}\label{IteratedBarConstructionHomotopy}
The natural transformations $\epsilon$ and $b\epsilon$ from $b^2:s\calA\to s\calA$ to $b:s\calA\to s\calA$ are naturally simplicially homotopic.
\end{prop}

\begin{proof}
Write $K=b^{2}X$ and $L=bX$ for the source and target of these maps respectively. Noting the formulae
\[[\frakd_{iq}]^2= [\frakd_{q+i,2q}\circ\frakd_{i,2q+1}]\ \textup{and}\  [\fraks_{iq}]^2= [\fraks_{q+i+2,2q+2}\circ\fraks_{i,2q+1}],\]
we can describe the simplicial structure maps in $K$ and $L$ as follows:
\begin{alignat*}{2}
\trip{d}{iq}{L}&=[\frakd_{iq}]\trip{d}{iq}{X}\\
\trip{s}{iq}{L}&=[\fraks_{iq}]\trip{s}{iq}{X}\\
\trip{d}{iq}{K}&=[\frakd_{q+i,2q}\circ\frakd_{i,2q+1}]\trip{d}{iq}{X}\\
\trip{s}{iq}{K}&=[\fraks_{q+i+2,2q+2}\circ\fraks_{i,2q+1}]\trip{s}{iq}{X}
\end{alignat*}
We can now state an explicit simplicial homotopy between the two maps of interest. Using precisely the notation of \cite[\S5]{MaySimpObj.pdf}, we define $\trip{h}{jq}{}:K_q\to L_{q+1}$, for $0\leq j\leq q$, by the formula
\[\trip{h}{jq}{}:=[\frakd_{j+1,q+2}\circ\cdots \circ\frakd_{j+1,2q+1}]\trip{s}{jq}{X}.\]
We first check that these maps satisfy the defining identities for the notion of simplicial homotopy, numbered (1)-(5) as in \cite[\S5]{MaySimpObj.pdf}. Each identity can be checked in two parts (a)-(b):
{\renewcommand{\circ}{\relax}
\begin{enumerate}\squishlist
\setlength{\parindent}{.25in}
\item We must check that $\trip{d}{i,q+1}{L}\circ \trip{h}{j,q}{}=\trip{h}{j-1,q-1}{}\circ \trip{d}{i,q}{K}$ whenever $0\leq i<j\leq q$. Equivalently:
\begin{enumerate}\squishlist
\setlength{\parindent}{.25in}
\item[({\makebox[.51em][c]{a}})] $\trip{d}{i,q+1}{X}\circ \trip{s}{j,q}{X}=\trip{s}{j-1,q-1}{X}\circ \trip{d}{i,q}{X}$,\textup{ and}%$\trip{\frakd}{i,q+1}{}\circ \trip{\fraks}{j,q}{}=\trip{\fraks}{j-1,q-1}{}\circ \trip{\frakd}{i,q}{}$
\item[({\makebox[.51em][c]{b}})]
$\frakd_{i,q+1}\circ
\frakd_{j+1,q+2}\circ\cdots \circ\frakd_{j+1,2q+1}=
\frakd_{j,q+1}\circ\cdots \circ\frakd_{j,2q-1}\circ
\frakd_{q+i,2q}\circ\frakd_{i,2q+1}$.
\end{enumerate}
\item We must check that $\trip{d}{j+1,q+1}{L}\circ \trip{h}{j,q}{}=\trip{d}{j+1,q+1}{L}\circ \trip{h}{j+1,q}{}$ whenever $0\leq j\leq q-1$. Equivalently:
\begin{enumerate}\squishlist
\setlength{\parindent}{.25in}
\item[({\makebox[.51em][c]{a}})] $\trip{d}{j+1,q+1}{X}\circ \trip{s}{j,q}{X}=\trip{d}{j+1,q+1}{X}\circ \trip{s}{j+1,q}{X}$,\textup{ and}%$\trip{\frakd}{j+1,q+1}{}\circ \trip{\fraks}{j,q}{}=\trip{\frakd}{j+1,q+1}{}\circ \trip{\fraks}{j+1,q}{}$
\item[({\makebox[.51em][c]{b}})]
$\frakd_{j+1,q+1}\circ
\frakd_{j+1,q+2}\circ\cdots \circ\frakd_{j+1,2q+1}=
\frakd_{j+1,q+1}\circ
\frakd_{j+2,q+2}\circ\cdots \circ\frakd_{j+2,2q+1}$.
\end{enumerate}
\item We must check that $\trip{d}{i,q+1}{L}\circ \trip{h}{j,q}{}=\trip{h}{j,q-1}{}\circ \trip{d}{i-1,q}{K}$ whenever $0\leq j<i-1\leq q$. Equivalently:
\begin{enumerate}\squishlist
\setlength{\parindent}{.25in}
\item[({\makebox[.51em][c]{a}})] $\trip{d}{i,q+1}{X}\circ \trip{s}{j,q}{X}=\trip{s}{j,q-1}{X}\circ \trip{d}{i-1,q}{X}$,\textup{ and}%$\trip{\frakd}{i,q+1}{}\circ \trip{\fraks}{j,q}{}=\trip{\fraks}{j,q-1}{}\circ \trip{\frakd}{i-1,q}{}$
\item[({\makebox[.51em][c]{b}})] 
$\frakd_{i,q+1}\circ
\frakd_{j+1,q+2}\circ\cdots \circ\frakd_{j+1,2q+1}=
\frakd_{j+1,q+1}\circ\cdots \circ\frakd_{j+1,2q-1}\circ
\frakd_{q+i-1,2q}\circ\frakd_{i-1,2q+1}$.
\end{enumerate}
\item We must check that $\trip{s}{i,q+1}{L}\circ \trip{h}{j,q}{}=\trip{h}{j+1,q+1}{}\circ \trip{s}{i,q}{K}$ whenever $0\leq i\leq j\leq q$. Equivalently:
\begin{enumerate}\squishlist
\setlength{\parindent}{.25in}
\item[({\makebox[.51em][c]{a}})] $\trip{s}{i,q+1}{X}\circ \trip{s}{j,q}{X}=\trip{s}{j+1,q+1}{X}\circ \trip{s}{i,q}{X}$,\textup{ and}%$\trip{\fraks}{i,q+1}{}\circ \trip{\fraks}{j,q}{}=\trip{\fraks}{j+1,q+1}{}\circ \trip{\fraks}{i,q}{}$
\item[({\makebox[.51em][c]{b}})] 
$\fraks_{i,q+1}\circ
\frakd_{j+1,q+2}\circ\cdots \circ\frakd_{j+1,2q+1}=
\frakd_{j+2,q+3}\circ\cdots \circ\frakd_{j+2,2q+3}\circ
\fraks_{q+i+2,2q+2}\circ\fraks_{i,2q+1}$.
\end{enumerate}
\item We must check that $\trip{s}{i,q+1}{L}\circ \trip{h}{j,q}{}=\trip{h}{j,q+1}{}\circ \trip{s}{i-1,q}{K}$ whenever $0\leq j<i\leq q+1$. Equivalently:
\begin{enumerate}\squishlist
\setlength{\parindent}{.25in}
\item[({\makebox[.51em][c]{a}})] $\trip{s}{i,q+1}{X}\circ \trip{s}{j,q}{X}=\trip{s}{j,q+1}{X}\circ \trip{s}{i-1,q}{X}$,\textup{ and}%$\trip{\fraks}{i,q++1}{}\circ \trip{\fraks}{j,q}{}=\trip{\fraks}{j,q+1}{}\circ \trip{\fraks}{i-1,q}{}$
\item[({\makebox[.51em][c]{b}})] 
$\fraks_{i,q+1}\circ
\frakd_{j+1,q+2}\circ\cdots \circ\frakd_{j+1,2q+1}=
\frakd_{j+1,q+3}\circ\cdots \circ\frakd_{j+1,2q+3}\circ
\fraks_{q+i+1,2q+2}\circ\fraks_{i-1,2q+1}$.
\end{enumerate}
\end{enumerate}
\noindent Each of these equations follows from the simplicial identities, proving that the $h_{jq}$ form a homotopy. 
Finally, we check that this homotopy is indeed a homotopy between the two maps of interest:
\begin{alignat*}{2}
\trip{d}{0,q+1}{L}\trip{h}{0,q}{}
&=
[\frakd_{0,q+1}\frakd_{1,q+2}\circ\cdots \circ\frakd_{1,2q+1}](\trip{d}{0,q+1}{X}\trip{s}{0q}{X})%
\\
&=
[\frakd_{0,q+1}\frakd_{0,q+2}\circ\cdots \circ\frakd_{0,2q+1}]\trip{\textup{id}}{X_q}{}%
\end{alignat*}
is the action of  $\epsilon_{(bX)}$ in level $q$, and similarly,
\begin{alignat*}{2}
\trip{d}{q+1,q+1}{L}\trip{h}{q,q}{}
&=
[\frakd_{q+1,q+1}\frakd_{q+1,q+2}\circ\cdots \circ\frakd_{q+1,2q+1}](\trip{d}{q+1,q+1}{X}\trip{s}{qq}{X})
\\
&=
[\frakd_{q+1,q+1}\frakd_{q+1,q+2}\circ\cdots \circ\frakd_{q+1,2q+1}]\trip{\textup{id}}{X_q}{}%
\end{alignat*}
is the action of $b\epsilon_{X}$ in level $q$.
}%end "don't print \circ"
\end{proof}



\printbibliography

\end{document}

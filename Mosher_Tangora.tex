% !TEX root = z_output/_Mosher_Tangora.tex
%%%%%%%%%%%%%%%%%%%%%%%%%%%%%%%%%%%%%%%%%%%%%%%%%%%%%%%%%%%%%%%%%%%%%%%%%%%%%%%%
%%%%%%%%%%%%%%%%%%%%%%%%%%% 80 characters %%%%%%%%%%%%%%%%%%%%%%%%%%%%%%%%%%%%%%
%%%%%%%%%%%%%%%%%%%%%%%%%%%%%%%%%%%%%%%%%%%%%%%%%%%%%%%%%%%%%%%%%%%%%%%%%%%%%%%%
\documentclass[11pt]{article}
\usepackage{fullpage}
\usepackage{amsmath,amsthm,amssymb}
\usepackage{mathrsfs,nicefrac}
\usepackage{amssymb}
\usepackage{epsfig}
\usepackage[all,2cell]{xy}
\usepackage{sseq}
\usepackage{tocloft}
\usepackage{cancel}
\usepackage[strict]{changepage}
\usepackage{color}
\usepackage{tikz}
\usepackage{extpfeil}
\usepackage{version}
\usepackage{framed}
\definecolor{shadecolor}{rgb}{.925,0.925,0.925}

%\usepackage{ifthen}
%Used for disabling hyperref
\ifx\dontloadhyperref\undefined
%\usepackage[pdftex,pdfborder={0 0 0 [1 1]}]{hyperref}
\usepackage[pdftex,pdfborder={0 0 .5 [1 1]}]{hyperref}
\else
\providecommand{\texorpdfstring}[2]{#1}
\fi
%>>>>>>>>>>>>>>>>>>>>>>>>>>>>>>
%<<<        Versions        <<<
%>>>>>>>>>>>>>>>>>>>>>>>>>>>>>>
%Add in the following line to include all the versions.
%\def\excludeversion#1{\includeversion{#1}}

%>>>>>>>>>>>>>>>>>>>>>>>>>>>>>>
%<<<       Better ToC       <<<
%>>>>>>>>>>>>>>>>>>>>>>>>>>>>>>
\setlength{\cftbeforesecskip}{0.5ex}

%>>>>>>>>>>>>>>>>>>>>>>>>>>>>>>
%<<<      Hyperref mod      <<<
%>>>>>>>>>>>>>>>>>>>>>>>>>>>>>>

%needs more testing
\newcounter{dummyforrefstepcounter}
\newcommand{\labelRIGHTHERE}[1]
{\refstepcounter{dummyforrefstepcounter}\label{#1}}


%>>>>>>>>>>>>>>>>>>>>>>>>>>>>>>
%<<<  Theorem Environments  <<<
%>>>>>>>>>>>>>>>>>>>>>>>>>>>>>>
\ifx\dontloaddefinitionsoftheoremenvironments\undefined
\theoremstyle{plain}
\newtheorem{thm}{Theorem}[section]
\newtheorem*{thm*}{Theorem}
\newtheorem{lem}[thm]{Lemma}
\newtheorem*{lem*}{Lemma}
\newtheorem{prop}[thm]{Proposition}
\newtheorem*{prop*}{Proposition}
\newtheorem{cor}[thm]{Corollary}
\newtheorem*{cor*}{Corollary}
\newtheorem{defprop}[thm]{Definition-Proposition}
\newtheorem*{punchline}{Punchline}
\newtheorem*{conjecture}{Conjecture}
\newtheorem*{claim}{Claim}

\theoremstyle{definition}
\newtheorem{defn}{Definition}[section]
\newtheorem*{defn*}{Definition}
\newtheorem{exmp}{Example}[section]
\newtheorem*{exmp*}{Example}
\newtheorem*{exmps*}{Examples}
\newtheorem*{nonexmp*}{Non-example}
\newtheorem{asspt}{Assumption}[section]
\newtheorem{notation}{Notation}[section]
\newtheorem{exercise}{Exercise}[section]
\newtheorem*{fact*}{Fact}
\newtheorem*{rmk*}{Remark}
\newtheorem{fact}{Fact}
\newtheorem*{aside}{Aside}
\newtheorem*{question}{Question}
\newtheorem*{answer}{Answer}

\else\relax\fi

%>>>>>>>>>>>>>>>>>>>>>>>>>>>>>>
%<<<      Fields, etc.      <<<
%>>>>>>>>>>>>>>>>>>>>>>>>>>>>>>
\DeclareSymbolFont{AMSb}{U}{msb}{m}{n}
\DeclareMathSymbol{\N}{\mathbin}{AMSb}{"4E}
\DeclareMathSymbol{\Octonions}{\mathbin}{AMSb}{"4F}
\DeclareMathSymbol{\Z}{\mathbin}{AMSb}{"5A}
\DeclareMathSymbol{\R}{\mathbin}{AMSb}{"52}
\DeclareMathSymbol{\Q}{\mathbin}{AMSb}{"51}
\DeclareMathSymbol{\PP}{\mathbin}{AMSb}{"50}
\DeclareMathSymbol{\I}{\mathbin}{AMSb}{"49}
\DeclareMathSymbol{\C}{\mathbin}{AMSb}{"43}
\DeclareMathSymbol{\A}{\mathbin}{AMSb}{"41}
\DeclareMathSymbol{\F}{\mathbin}{AMSb}{"46}
\DeclareMathSymbol{\G}{\mathbin}{AMSb}{"47}
\DeclareMathSymbol{\Quaternions}{\mathbin}{AMSb}{"48}


%>>>>>>>>>>>>>>>>>>>>>>>>>>>>>>
%<<<       Operators        <<<
%>>>>>>>>>>>>>>>>>>>>>>>>>>>>>>
\DeclareMathOperator{\ad}{\textbf{ad}}
\DeclareMathOperator{\coker}{coker}
\renewcommand{\ker}{\textup{ker}\,}
\DeclareMathOperator{\End}{End}
\DeclareMathOperator{\Aut}{Aut}
\DeclareMathOperator{\Hom}{Hom}
\DeclareMathOperator{\Maps}{Maps}
\DeclareMathOperator{\Mor}{Mor}
\DeclareMathOperator{\Gal}{Gal}
\DeclareMathOperator{\Ext}{Ext}
\DeclareMathOperator{\Tor}{Tor}
\DeclareMathOperator{\Map}{Map}
\DeclareMathOperator{\Der}{Der}
\DeclareMathOperator{\Rad}{Rad}
\DeclareMathOperator{\rank}{rank}
\DeclareMathOperator{\ArfInvariant}{Arf}
\DeclareMathOperator{\KervaireInvariant}{Ker}
\DeclareMathOperator{\im}{im}
\DeclareMathOperator{\coim}{coim}
\DeclareMathOperator{\trace}{tr}
\DeclareMathOperator{\supp}{supp}
\DeclareMathOperator{\ann}{ann}
\DeclareMathOperator{\spec}{Spec}
\DeclareMathOperator{\SPEC}{\textbf{Spec}}
\DeclareMathOperator{\proj}{Proj}
\DeclareMathOperator{\PROJ}{\textbf{Proj}}
\DeclareMathOperator{\fiber}{F}
\DeclareMathOperator{\cofiber}{C}
\DeclareMathOperator{\cone}{cone}
\DeclareMathOperator{\skel}{sk}
\DeclareMathOperator{\coskel}{cosk}
\DeclareMathOperator{\conn}{conn}
\DeclareMathOperator{\colim}{colim}
\DeclareMathOperator{\limit}{lim}
\DeclareMathOperator{\ch}{ch}
\DeclareMathOperator{\Vect}{Vect}
\DeclareMathOperator{\GrthGrp}{GrthGp}
\DeclareMathOperator{\Sym}{Sym}
\DeclareMathOperator{\Prob}{\mathbb{P}}
\DeclareMathOperator{\Exp}{\mathbb{E}}
\DeclareMathOperator{\GeomMean}{\mathbb{G}}
\DeclareMathOperator{\Var}{Var}
\DeclareMathOperator{\Cov}{Cov}
\DeclareMathOperator{\Sp}{Sp}
\DeclareMathOperator{\Seq}{Seq}
\DeclareMathOperator{\Cyl}{Cyl}
\DeclareMathOperator{\Ev}{Ev}
\DeclareMathOperator{\sh}{sh}
\DeclareMathOperator{\intHom}{\underline{Hom}}
\DeclareMathOperator{\Frac}{frac}



%>>>>>>>>>>>>>>>>>>>>>>>>>>>>>>
%<<<   Cohomology Theories  <<<
%>>>>>>>>>>>>>>>>>>>>>>>>>>>>>>
\DeclareMathOperator{\KR}{{K\R}}
\DeclareMathOperator{\KO}{{KO}}
\DeclareMathOperator{\K}{{K}}
\DeclareMathOperator{\OmegaO}{{\Omega_{\Octonions}}}

%>>>>>>>>>>>>>>>>>>>>>>>>>>>>>>
%<<<   Algebraic Geometry   <<<
%>>>>>>>>>>>>>>>>>>>>>>>>>>>>>>
\DeclareMathOperator{\Spec}{Spec}
\DeclareMathOperator{\Proj}{Proj}
\DeclareMathOperator{\Sing}{Sing}
\DeclareMathOperator{\shfHom}{\mathscr{H}\textit{\!\!om}}
\DeclareMathOperator{\WeilDivisors}{{Div}}
\DeclareMathOperator{\CartierDivisors}{{CaDiv}}
\DeclareMathOperator{\PrincipalWeilDivisors}{{PrDiv}}
\DeclareMathOperator{\LocallyPrincipalWeilDivisors}{{LPDiv}}
\DeclareMathOperator{\PrincipalCartierDivisors}{{PrCaDiv}}
\DeclareMathOperator{\DivisorClass}{{Cl}}
\DeclareMathOperator{\CartierClass}{{CaCl}}
\DeclareMathOperator{\Picard}{{Pic}}
\DeclareMathOperator{\Frob}{Frob}


%>>>>>>>>>>>>>>>>>>>>>>>>>>>>>>
%<<<  Mathematical Objects  <<<
%>>>>>>>>>>>>>>>>>>>>>>>>>>>>>>
\newcommand{\sll}{\mathfrak{sl}}
\newcommand{\gl}{\mathfrak{gl}}
\newcommand{\GL}{\mbox{GL}}
\newcommand{\PGL}{\mbox{PGL}}
\newcommand{\SL}{\mbox{SL}}
\newcommand{\Mat}{\mbox{Mat}}
\newcommand{\Gr}{\textup{Gr}}
\newcommand{\Squ}{\textup{Sq}}
\newcommand{\catSet}{\textit{Sets}}
\newcommand{\RP}{{\R\PP}}
\newcommand{\CP}{{\C\PP}}
\newcommand{\Steen}{\mathscr{A}}
\newcommand{\Orth}{\textup{\textbf{O}}}

%>>>>>>>>>>>>>>>>>>>>>>>>>>>>>>
%<<<  Mathematical Symbols  <<<
%>>>>>>>>>>>>>>>>>>>>>>>>>>>>>>
\newcommand{\DASH}{\textup{---}}
\newcommand{\op}{\textup{op}}
\newcommand{\CW}{\textup{CW}}
\newcommand{\ob}{\textup{ob}\,}
\newcommand{\ho}{\textup{ho}}
\newcommand{\st}{\textup{st}}
\newcommand{\id}{\textup{id}}
\newcommand{\Bullet}{\ensuremath{\bullet} }
\newcommand{\sprod}{\wedge}

%>>>>>>>>>>>>>>>>>>>>>>>>>>>>>>
%<<<      Some Arrows       <<<
%>>>>>>>>>>>>>>>>>>>>>>>>>>>>>>
\newcommand{\nt}{\Longrightarrow}
\let\shortmapsto\mapsto
\let\mapsto\longmapsto
\newcommand{\mapsfrom}{\,\reflectbox{$\mapsto$}\ }
\newcommand{\bigrightsquig}{\scalebox{2}{\ensuremath{\rightsquigarrow}}}
\newcommand{\bigleftsquig}{\reflectbox{\scalebox{2}{\ensuremath{\rightsquigarrow}}}}

%\newcommand{\cofibration}{\xhookrightarrow{\phantom{\ \,{\sim\!}\ \ }}}
%\newcommand{\fibration}{\xtwoheadrightarrow{\phantom{\sim\!}}}
%\newcommand{\acycliccofibration}{\xhookrightarrow{\ \,{\sim\!}\ \ }}
%\newcommand{\acyclicfibration}{\xtwoheadrightarrow{\sim\!}}
%\newcommand{\leftcofibration}{\xhookleftarrow{\phantom{\ \,{\sim\!}\ \ }}}
%\newcommand{\leftfibration}{\xtwoheadleftarrow{\phantom{\sim\!}}}
%\newcommand{\leftacycliccofibration}{\xhookleftarrow{\ \ {\sim\!}\,\ }}
%\newcommand{\leftacyclicfibration}{\xtwoheadleftarrow{\sim\!}}
%\newcommand{\weakequiv}{\xrightarrow{\ \,\sim\,\ }}
%\newcommand{\leftweakequiv}{\xleftarrow{\ \,\sim\,\ }}

\newcommand{\cofibration}
{\xhookrightarrow{\phantom{\ \,{\raisebox{-.3ex}[0ex][0ex]{\scriptsize$\sim$}\!}\ \ }}}
\newcommand{\fibration}
{\xtwoheadrightarrow{\phantom{\raisebox{-.3ex}[0ex][0ex]{\scriptsize$\sim$}\!}}}
\newcommand{\acycliccofibration}
{\xhookrightarrow{\ \,{\raisebox{-.55ex}[0ex][0ex]{\scriptsize$\sim$}\!}\ \ }}
\newcommand{\acyclicfibration}
{\xtwoheadrightarrow{\raisebox{-.6ex}[0ex][0ex]{\scriptsize$\sim$}\!}}
\newcommand{\leftcofibration}
{\xhookleftarrow{\phantom{\ \,{\raisebox{-.3ex}[0ex][0ex]{\scriptsize$\sim$}\!}\ \ }}}
\newcommand{\leftfibration}
{\xtwoheadleftarrow{\phantom{\raisebox{-.3ex}[0ex][0ex]{\scriptsize$\sim$}\!}}}
\newcommand{\leftacycliccofibration}
{\xhookleftarrow{\ \ {\raisebox{-.55ex}[0ex][0ex]{\scriptsize$\sim$}\!}\,\ }}
\newcommand{\leftacyclicfibration}
{\xtwoheadleftarrow{\raisebox{-.6ex}[0ex][0ex]{\scriptsize$\sim$}\!}}
\newcommand{\weakequiv}
{\xrightarrow{\ \,\raisebox{-.3ex}[0ex][0ex]{\scriptsize$\sim$}\,\ }}
\newcommand{\leftweakequiv}
{\xleftarrow{\ \,\raisebox{-.3ex}[0ex][0ex]{\scriptsize$\sim$}\,\ }}

%>>>>>>>>>>>>>>>>>>>>>>>>>>>>>>
%<<<    xymatrix Arrows     <<<
%>>>>>>>>>>>>>>>>>>>>>>>>>>>>>>
\newdir{ >}{{}*!/-5pt/@{>}}
\newcommand{\xycof}{\ar@{ >->}}
\newcommand{\xycofib}{\ar@{^{(}->}}
\newcommand{\xycofibdown}{\ar@{_{(}->}}
\newcommand{\xyfib}{\ar@{->>}}
\newcommand{\xymapsto}{\ar@{|->}}

%>>>>>>>>>>>>>>>>>>>>>>>>>>>>>>
%<<<     Greek Letters      <<<
%>>>>>>>>>>>>>>>>>>>>>>>>>>>>>>
%\newcommand{\oldphi}{\phi}
%\renewcommand{\phi}{\varphi}
\let\oldphi\phi
\let\phi\varphi
\renewcommand{\to}{\longrightarrow}
\newcommand{\from}{\longleftarrow}
\newcommand{\eps}{\varepsilon}

%>>>>>>>>>>>>>>>>>>>>>>>>>>>>>>
%<<<  1st-4th & parentheses <<<
%>>>>>>>>>>>>>>>>>>>>>>>>>>>>>>
\newcommand{\first}{^\text{st}}
\newcommand{\second}{^\text{nd}}
\newcommand{\third}{^\text{rd}}
\newcommand{\fourth}{^\text{th}}
\newcommand{\ZEROTH}{$0^\text{th}$ }
\newcommand{\FIRST}{$1^\text{st}$ }
\newcommand{\SECOND}{$2^\text{nd}$ }
\newcommand{\THIRD}{$3^\text{rd}$ }
\newcommand{\FOURTH}{$4^\text{th}$ }
\newcommand{\iTH}{$i^\text{th}$ }
\newcommand{\jTH}{$j^\text{th}$ }
\newcommand{\nTH}{$n^\text{th}$ }

%>>>>>>>>>>>>>>>>>>>>>>>>>>>>>>
%<<<    upright commands    <<<
%>>>>>>>>>>>>>>>>>>>>>>>>>>>>>>
\newcommand{\upcol}{\textup{:}}
\newcommand{\upsemi}{\textup{;}}
\providecommand{\lparen}{\textup{(}}
\providecommand{\rparen}{\textup{)}}
\renewcommand{\lparen}{\textup{(}}
\renewcommand{\rparen}{\textup{)}}
\newcommand{\Iff}{\emph{iff} }

%>>>>>>>>>>>>>>>>>>>>>>>>>>>>>>
%<<<     Environments       <<<
%>>>>>>>>>>>>>>>>>>>>>>>>>>>>>>
\newcommand{\squishlist}
{ %\setlength{\topsep}{100pt} doesn't seem to do anything.
  \setlength{\itemsep}{.5pt}
  \setlength{\parskip}{0pt}
  \setlength{\parsep}{0pt}}
\newenvironment{itemise}{
\begin{list}{\textup{$\rightsquigarrow$}}
   {  \setlength{\topsep}{1mm}
      \setlength{\itemsep}{1pt}
      \setlength{\parskip}{0pt}
      \setlength{\parsep}{0pt}
   }
}{\end{list}\vspace{-.1cm}}
\newcommand{\INDENT}{\textbf{}\phantom{space}}
\renewcommand{\INDENT}{\rule{.7cm}{0cm}}

\newcommand{\itm}[1][$\rightsquigarrow$]{\item[{\makebox[.5cm][c]{\textup{#1}}}]}


%\newcommand{\rednote}[1]{{\color{red}#1}\makebox[0cm][l]{\scalebox{.1}{rednote}}}
%\newcommand{\bluenote}[1]{{\color{blue}#1}\makebox[0cm][l]{\scalebox{.1}{rednote}}}

\newcommand{\rednote}[1]
{{\color{red}#1}\makebox[0cm][l]{\scalebox{.1}{\rotatebox{90}{?????}}}}
\newcommand{\bluenote}[1]
{{\color{blue}#1}\makebox[0cm][l]{\scalebox{.1}{\rotatebox{90}{?????}}}}


\newcommand{\funcdef}[4]{\begin{align*}
#1&\to #2\\
#3&\mapsto#4
\end{align*}}

%\newcommand{\comment}[1]{}

%>>>>>>>>>>>>>>>>>>>>>>>>>>>>>>
%<<<       Categories       <<<
%>>>>>>>>>>>>>>>>>>>>>>>>>>>>>>
\newcommand{\Ens}{{\mathscr{E}ns}}
\DeclareMathOperator{\Sheaves}{{\mathsf{Shf}}}
\DeclareMathOperator{\Presheaves}{{\mathsf{PreShf}}}
\DeclareMathOperator{\Psh}{{\mathsf{Psh}}}
\DeclareMathOperator{\Shf}{{\mathsf{Shf}}}
\DeclareMathOperator{\Varieties}{{\mathsf{Var}}}
\DeclareMathOperator{\Schemes}{{\mathsf{Sch}}}
\DeclareMathOperator{\Rings}{{\mathsf{Rings}}}
\DeclareMathOperator{\AbGp}{{\mathsf{AbGp}}}
\DeclareMathOperator{\Modules}{{\mathsf{\!-Mod}}}
\DeclareMathOperator{\fgModules}{{\mathsf{\!-Mod}^{\textup{fg}}}}
\DeclareMathOperator{\QuasiCoherent}{{\mathsf{QCoh}}}
\DeclareMathOperator{\Coherent}{{\mathsf{Coh}}}
\DeclareMathOperator{\GSW}{{\mathcal{SW}^G}}
\DeclareMathOperator{\Burnside}{{\mathsf{Burn}}}
\DeclareMathOperator{\GSet}{{G\mathsf{Set}}}
\DeclareMathOperator{\FinGSet}{{G\mathsf{Set}^\textup{fin}}}
\DeclareMathOperator{\HSet}{{H\mathsf{Set}}}
\DeclareMathOperator{\Cat}{{\mathsf{Cat}}}
\DeclareMathOperator{\Fun}{{\mathsf{Fun}}}
\DeclareMathOperator{\Orb}{{\mathsf{Orb}}}
\DeclareMathOperator{\Set}{{\mathsf{Set}}}
\DeclareMathOperator{\sSet}{{\mathsf{sSet}}}
\DeclareMathOperator{\Top}{{\mathsf{Top}}}
\DeclareMathOperator{\GSpectra}{{G-\mathsf{Spectra}}}
\DeclareMathOperator{\Lan}{Lan}
\DeclareMathOperator{\Ran}{Ran}

%>>>>>>>>>>>>>>>>>>>>>>>>>>>>>>
%<<<     Script Letters     <<<
%>>>>>>>>>>>>>>>>>>>>>>>>>>>>>>
\newcommand{\scrQ}{\mathscr{Q}}
\newcommand{\scrW}{\mathscr{W}}
\newcommand{\scrE}{\mathscr{E}}
\newcommand{\scrR}{\mathscr{R}}
\newcommand{\scrT}{\mathscr{T}}
\newcommand{\scrY}{\mathscr{Y}}
\newcommand{\scrU}{\mathscr{U}}
\newcommand{\scrI}{\mathscr{I}}
\newcommand{\scrO}{\mathscr{O}}
\newcommand{\scrP}{\mathscr{P}}
\newcommand{\scrA}{\mathscr{A}}
\newcommand{\scrS}{\mathscr{S}}
\newcommand{\scrD}{\mathscr{D}}
\newcommand{\scrF}{\mathscr{F}}
\newcommand{\scrG}{\mathscr{G}}
\newcommand{\scrH}{\mathscr{H}}
\newcommand{\scrJ}{\mathscr{J}}
\newcommand{\scrK}{\mathscr{K}}
\newcommand{\scrL}{\mathscr{L}}
\newcommand{\scrZ}{\mathscr{Z}}
\newcommand{\scrX}{\mathscr{X}}
\newcommand{\scrC}{\mathscr{C}}
\newcommand{\scrV}{\mathscr{V}}
\newcommand{\scrB}{\mathscr{B}}
\newcommand{\scrN}{\mathscr{N}}
\newcommand{\scrM}{\mathscr{M}}

%>>>>>>>>>>>>>>>>>>>>>>>>>>>>>>
%<<<     Fractur Letters    <<<
%>>>>>>>>>>>>>>>>>>>>>>>>>>>>>>
\newcommand{\frakQ}{\mathfrak{Q}}
\newcommand{\frakW}{\mathfrak{W}}
\newcommand{\frakE}{\mathfrak{E}}
\newcommand{\frakR}{\mathfrak{R}}
\newcommand{\frakT}{\mathfrak{T}}
\newcommand{\frakY}{\mathfrak{Y}}
\newcommand{\frakU}{\mathfrak{U}}
\newcommand{\frakI}{\mathfrak{I}}
\newcommand{\frakO}{\mathfrak{O}}
\newcommand{\frakP}{\mathfrak{P}}
\newcommand{\frakA}{\mathfrak{A}}
\newcommand{\frakS}{\mathfrak{S}}
\newcommand{\frakD}{\mathfrak{D}}
\newcommand{\frakF}{\mathfrak{F}}
\newcommand{\frakG}{\mathfrak{G}}
\newcommand{\frakH}{\mathfrak{H}}
\newcommand{\frakJ}{\mathfrak{J}}
\newcommand{\frakK}{\mathfrak{K}}
\newcommand{\frakL}{\mathfrak{L}}
\newcommand{\frakZ}{\mathfrak{Z}}
\newcommand{\frakX}{\mathfrak{X}}
\newcommand{\frakC}{\mathfrak{C}}
\newcommand{\frakV}{\mathfrak{V}}
\newcommand{\frakB}{\mathfrak{B}}
\newcommand{\frakN}{\mathfrak{N}}
\newcommand{\frakM}{\mathfrak{M}}

\newcommand{\frakq}{\mathfrak{q}}
\newcommand{\frakw}{\mathfrak{w}}
\newcommand{\frake}{\mathfrak{e}}
\newcommand{\frakr}{\mathfrak{r}}
\newcommand{\frakt}{\mathfrak{t}}
\newcommand{\fraky}{\mathfrak{y}}
\newcommand{\fraku}{\mathfrak{u}}
\newcommand{\fraki}{\mathfrak{i}}
\newcommand{\frako}{\mathfrak{o}}
\newcommand{\frakp}{\mathfrak{p}}
\newcommand{\fraka}{\mathfrak{a}}
\newcommand{\fraks}{\mathfrak{s}}
\newcommand{\frakd}{\mathfrak{d}}
\newcommand{\frakf}{\mathfrak{f}}
\newcommand{\frakg}{\mathfrak{g}}
\newcommand{\frakh}{\mathfrak{h}}
\newcommand{\frakj}{\mathfrak{j}}
\newcommand{\frakk}{\mathfrak{k}}
\newcommand{\frakl}{\mathfrak{l}}
\newcommand{\frakz}{\mathfrak{z}}
\newcommand{\frakx}{\mathfrak{x}}
\newcommand{\frakc}{\mathfrak{c}}
\newcommand{\frakv}{\mathfrak{v}}
\newcommand{\frakb}{\mathfrak{b}}
\newcommand{\frakn}{\mathfrak{n}}
\newcommand{\frakm}{\mathfrak{m}}

%>>>>>>>>>>>>>>>>>>>>>>>>>>>>>>
%<<<  Caligraphic Letters   <<<
%>>>>>>>>>>>>>>>>>>>>>>>>>>>>>>
\newcommand{\calQ}{\mathcal{Q}}
\newcommand{\calW}{\mathcal{W}}
\newcommand{\calE}{\mathcal{E}}
\newcommand{\calR}{\mathcal{R}}
\newcommand{\calT}{\mathcal{T}}
\newcommand{\calY}{\mathcal{Y}}
\newcommand{\calU}{\mathcal{U}}
\newcommand{\calI}{\mathcal{I}}
\newcommand{\calO}{\mathcal{O}}
\newcommand{\calP}{\mathcal{P}}
\newcommand{\calA}{\mathcal{A}}
\newcommand{\calS}{\mathcal{S}}
\newcommand{\calD}{\mathcal{D}}
\newcommand{\calF}{\mathcal{F}}
\newcommand{\calG}{\mathcal{G}}
\newcommand{\calH}{\mathcal{H}}
\newcommand{\calJ}{\mathcal{J}}
\newcommand{\calK}{\mathcal{K}}
\newcommand{\calL}{\mathcal{L}}
\newcommand{\calZ}{\mathcal{Z}}
\newcommand{\calX}{\mathcal{X}}
\newcommand{\calC}{\mathcal{C}}
\newcommand{\calV}{\mathcal{V}}
\newcommand{\calB}{\mathcal{B}}
\newcommand{\calN}{\mathcal{N}}
\newcommand{\calM}{\mathcal{M}}

%>>>>>>>>>>>>>>>>>>>>>>>>>>>>>>
%<<<<<<<<<DEPRECIATED<<<<<<<<<<
%>>>>>>>>>>>>>>>>>>>>>>>>>>>>>>

%%% From Kac's template
% 1-inch margins, from fullpage.sty by H.Partl, Version 2, Dec. 15, 1988.
%\topmargin 0pt
%\advance \topmargin by -\headheight
%\advance \topmargin by -\headsep
%\textheight 9.1in
%\oddsidemargin 0pt
%\evensidemargin \oddsidemargin
%\marginparwidth 0.5in
%\textwidth 6.5in
%
%\parindent 0in
%\parskip 1.5ex
%%\renewcommand{\baselinestretch}{1.25}

%%% From the net
%\newcommand{\pullbackcorner}[1][dr]{\save*!/#1+1.2pc/#1:(1,-1)@^{|-}\restore}
%\newcommand{\pushoutcorner}[1][dr]{\save*!/#1-1.2pc/#1:(-1,1)@^{|-}\restore}











\title{Cohomology Operations\small{ --- Mosher and Tangora}}
\author{}
\date{}
% >>>

%% >>> header

\begin{document}
%\tableofcontents
\comment{
\section{Introduction to Cohomology Operations}
\subsection{Cohomology Operations and \texorpdfstring{$K(\pi,n)$}{K(G,n)} spaces}
There is a natural equivalence of functors $[\DASH,K(\pi,n)]\simeq H^n(\DASH;\pi)$ given by $[f]\longleftrightarrow f^*(\imath_n)$. Here the \emph{fundamental class} $\imath_n\in H^n(K(\pi,n);\pi)$ is defined to be the cohomology class corresponding to $\text{id}_\pi$ under:
\[\xymatrix{
H^n(K(\pi,n);\pi)\ar[r]^{\simeq\ \ \ \ }_{\textup{UCT}\ \ \ \ }&\Hom(H_n(K(\pi,n)),\pi)\ar[r]_{\ \ \ \ \ \ \text{Hur.}}^{\ \ \ \ \ \ \simeq}&\Hom(\pi,\pi).
}\]
In summary:
\[H^m(K(\pi,n);\pi')\longleftrightarrow [K(\pi,n),K(\pi',m)]\longleftrightarrow \textup{Nat}( H^n(\DASH,\pi), H^{m}(\DASH,\pi')).\]

\subsection{The Machinery of obstruction theory \texorpdfstring{$\otimes$}{}}
\subsection{Applications of obstruction theory \texorpdfstring{$\otimes$}{}}


\section{Construction of the Steenrod squares}
\subsection{The complex \texorpdfstring{$K(\Z_2,1)$}{K(Z/2,1)}}
$\RP^\infty$ is a $K(\Z_2,1)$, and $S^\infty\downarrow\RP^\infty$ is a model for $E\Z_2\downarrow B\Z_2$. Moreover:
\[H_n(\RP^\infty)=\begin{cases}0&n\text{ even}\\\Z_2&n\text{ odd}\end{cases}\ \text{ and }H^n(\RP^\infty)=\begin{cases}\Z_2&n\text{ even}\\0&n\text{ odd}\end{cases}\ \text{ and }H^n(\RP^\infty;\Z_2)=\Z_2[\alpha_1].\]
%\subsection{Other Sections}
%The acyclic carrier theorem. Construction of the cup-$i$ products. The squaring operations. Compatibility with coboundary and suspension.

\section{Properties of the squares}
\begin{itemise}
\item[0.] $\Squ^i$ is a natural homomorphism $H^p(K,L;\Z_2)\to H^{p+i}(K,L;\Z_2)$
\item[1.] $\Squ^i(x)=0$ when $i>|x|$
\item[2.] $\Squ^i(x)=x^2$ when $i=|x|$
\item[3.] $\Squ^0$ is the identity
\item[4.] $\Squ^1$ is the Bockstein (associated with $0\to\Z_2\to\Z_4\to\Z_2\to0$)
\item[5.] $\Squ^i$ is compatible with the coboundary (and thus also with suspension)
\item[6.] $\Squ(xy)=\Squ(x)\Squ(y)$, where $\Squ$ denoted the total Steenrod square
\item[7.] For $a<2b$, $\Squ^a\Squ^b=\sum\binom{b-c-1}{a-2c}\Squ^{a+b-c}\Squ^c$. Note that in this sum only involves $0\leq c\leq a/2$. In particular, the terms on the right have $a+b-c\geq 2c$.
\end{itemise}
\setcounter{subsection}{2}
\subsection{Squares in the \texorpdfstring{$n$-fold}{n-fold} Cartesian product of \texorpdfstring{$K(\Z_2,1)$}{K(Z/2,1)}}
Let $K_n$ be the product of $n$ copies of $K(\Z_2,1)$. Then 
\[H^*(K_n,\Z_2)=\Z_2[x_1,\ldots,x_n]\supset\Z_2[\sigma_1,\ldots,\sigma_n]=S,\]
where the $\sigma_i$ are the elementary symmetric functions.

%\noindent \textbf{Proposition 3.} In $H^*(K_n;\Z_2)$, $\Squ^i(\sigma_n)=\sigma_n\sigma_i\neq0$.

\begin{prop*}
In $H^*(K_n;\Z_2)$, $\Squ^i(\sigma_n)=\sigma_n\sigma_i\neq0$.
\end{prop*}
A sequence $I=\{i_1,\ldots,i_r\}$ of (strictly positive) integers is \emph{admissible} if $i_j\geq 2i_{j+1}$, that is if it is decreasing faster that $2^{-j}$. The \emph{degree} $d(I)$ is the sum of the terms, and $\Squ^I$ raises dimension by $d(I)$. The \emph{excess}, defined only for admissible sequences, is:
\[e(I):=2i_1-d(I)=(i_1-2i_2)+(i_2-2i_3)+\cdots+(i_r).\]
Note that $\Squ^I(u)=0$ whenever $e(I)>|u|$, and if $e(I)=|u|$, then $\Squ^I(u)=(\Squ^J(u))^2$, where $J$ is obtained from $I$ by removing $i_1$.
\begin{thm*}
There exists an ordering on the monomials in $S=\Z_2[\sigma_1,\ldots,\sigma_n]$ such that whenever $I$ is admissible, with $d(I)\leq n$, we can write $\Squ^I(\sigma_n)=\sigma_n(\sigma_{i_1}\cdots\sigma_{i_r}+\textup{lower})$.
\end{thm*}
In particular, this means that the elements $\Squ^I(\imath_n)\in H^*(K(\Z_2,n);\Z_2)$ are linearly independent, where $I$ ranges over admmissible sequences with $d(I)\leq n$.

\subsection{The Adem Relations}
Some memorable Adem relations:
\begin{itemise}
\item $\Squ^1\Squ^{2n+1}=0$
\item $\Squ^1\Squ^{2n}=\Squ^{2n+1}$
\item $\Squ^2\Squ^2=\Squ^3\Squ^1$.
\item $\Squ^{2n-1}\Squ^{n}=0$
\end{itemise}
\section{Application: The Hopf invariant}
The hopf invariant is a homomorphism $H:\pi_{2n-1}(S^n)\to\Z$. It is zero whenever $n$ is odd, and surjective when $n=2,4,8$. When $n$ is even, there is always an element of Hopf invariant $2$, so that $\pi_{2n-1}(S^n)$ contains a free summand.
\setcounter{subsection}{1}
\subsection{Decomposable operations}
Say $\Squ^i$ is \emph{decomposable} if it can be written as $\sum_{t<i}a_t\Squ^t$, where $a_t$ is some combination of squaring operations.
\begin{thm*}
$\Squ^i$ is indecomposable iff $i$ is a power of $2$.
\end{thm*}
If $i$ is a power of two, consider the action of $\Squ$ on $\alpha^i\in H^i(\RP^\infty;\Z_2)$. We have $\Squ(\alpha^i)=(\alpha+\alpha^2)^i=\alpha^i+\alpha^{2i}$, so that $\Squ^t(\alpha^i)=0$ except when $t=0,i$. Thus $\Squ^i$ cannot be decomposable. The converse comes directly from the Adem relations.
\begin{lem*}
If $p$ is prime, and $a=\sum a_ip^i$ and $b=\sum b_ip^i$ are $p$-adic expressions for $a$ and $b$, then modulo $p$, we have $\binom{b}{a}=\prod\binom{b_i}{a_i}$.
\end{lem*}
\subsection{Non-existence of elements of Hopf invariant one}
If $\pi_{2n-1}S^n$ contains an element of odd Hopf invariant, then $n$ is a power of two, as otherwise $\Squ^n$ is decomposable, and so squaring the $n$ dimensional class in the relevant CW complex returns zero, modulo 2.

\section{Application: Vector fields on spheres}
Having $k$ linearly independent fields on $S^{n-1}$ is the same as having a section for the map $V_{n,k+1}\downarrow S^{n-1}$, where $V_{n,k+1}$ is the Steifel manifold, whose points are orthonormal $k+1$-frames in $\R^n$. 

Now let $P_{n,k}$ be the stunted projective space $\RP^n/\RP^{n-k}$. It is shown that when $2k\leq n$, $P_{n,k}$ is the $n$-skeleton of $V_{n,k}$. Moreover, $\widetilde H^q(P_{n,k};\Z_2)$ is $\Z_2$ for $(n-k)\leq q<n$, and otherwise zero. By direct calculation, $\Squ^{k-1}v_{n-k}=\binom{n-k}{k-1}v_{n-1}$, where $v_q$ is the generator of $\widetilde H^q$.
\begin{thm*}
If we write $n=(2s+1)2^m$, then $S^{n-1}$ does not admit a $2^m$-field.
\end{thm*}
\begin{proof}
This is trivial when $s=0$, or when $m=0$, so assume $s,m\geq1$. Then, writing $k=2^m+1$, we have $2k\leq n$, so that $P_{n,k}$ is the $n$-skeleton of $V_{n,k}$. In particular: \[\widetilde H^{n-k}(V_{n,k};\Z_2)=\Z_2\langle v_{n-k}\rangle,\ \ \widetilde H^{n-1}(V_{n,k};\Z_2)=\Z_2\langle v_{n-1}\rangle,\text{ and }\Squ^{k-1}v_{n-k}=v_{n-1}.\]
Note that the binomial coefficient involved is odd, by the above lemma.
Having a $2^m$-field would guarantee a section $f$ for $V_{n,k}\downarrow S^{n-1}$, inducing an  isomorphism $\widetilde H^{n-1}(V_{n,k};\Z_2)\to\widetilde H^{n-1}(S^{n-1};\Z_2)$. This would imply that the top cohomology class of $S^{n-1}$ is in the image of $\Squ^{k-1}$, which is false.
\end{proof}

\section{The Steenrod algebra}
\subsection{Graded modules and algebras}
Suppose $R$ is a commutative ring with unit. The \emph{tensor product} of graded $R$-modules $M,N$ has $(M\otimes N)_t=\oplus M_{i}\otimes N_{t-i}$. A \emph{graded $R$-algebra} is a graded $R$-module $A$ with a module homomorphism $\phi:A\otimes A\to A$. It's \emph{associative} if the obvious diagram commutes. It's \emph{commutative} if $\phi\circ T=\phi$, where $T:M\otimes N\to N\otimes M$ sends $m\otimes n\mapsto (-1)^{|m||n|}n\otimes m$.

An \emph{augmentation} is a graded $R$-algebra homomorphism $\epsilon:A\to R$ (where $R$ is a graded $R$-algebra concentrated in degree zero). An augmented $R$-algebra is \emph{connected} if $\epsilon:A_0\to R$ is an isomorphism.

The \emph{tensor product of algebras} $A$ and $B$ is formed using the multiplication $(\phi_A\otimes\phi_b)\circ(1\otimes T\otimes 1)$. That is, $(a_1\otimes b_1)(a_2\otimes b_2)=(-1)^{|b_1||a_2|}(a_1a_2)\otimes(b_1b_2)$. Given an $R$-module $M$, let $\Gamma(M)$ be the graded module with $\Gamma(M)_r=M^{\otimes r}$. This is an associative but \textbf{non}-commutative graded $R$-algebra, the \emph{tensor algebra}, with multiplication using the isomorphisms $M^{\otimes a}\otimes M^{\otimes b}=M^{\otimes (a+b)}$.

\subsection{The Steenrod algebra \texorpdfstring{$\Steen$}{A}}
Let $R=\Z_2$, so $M_i=\Z_2\langle\Squ^i\rangle$ for $i\geq0$ specifies a graded $\Z_2$-module $M$. The \emph{Steenrod algebra} $\Steen$ is the quotient of $\Gamma(M)$ be the Adem relations (and $\Squ^0=1$).
\begin{thm*}[Serre-Cartan basis]
The monomials $\Squ^I$ form a basis for $\Steen$ over $\Z_2$.
\end{thm*}
We have already seen, via the action of $\Steen$ on the cohomology of $K(\Z_2,1)^n$, that these monomials are linearly independent. That they span follows by use of the Adem relations, keeping track of the \emph{moment} $m(I):=\sum i_ss$ of a sequence $I$.

Note that $\Steen$ is generated as an algebra by the $\Squ^{2^i}$, but not freely.
\setcounter{subsection}{3}
\subsection{The diagonal map of \texorpdfstring{$\Steen$}{A}}
There's an algebra homomorphism $\psi:\Gamma(M)\to\Gamma(M)\otimes\Gamma(M)$ defined by $\psi(\Squ^i)=\sum_j \Squ^j\otimes\Squ^{i-j}$. That is, extended by the requirement that $\psi$ is an algebra homomorphism.
\begin{thm*}
$\psi$ descends to an algebra homomorphism $\Steen\to\Steen\otimes\Steen$.
\end{thm*}
\begin{proof}
We have maps $w_n:\Steen\to H^*(K_n;\Z_2)$ ($K_n=K(\Z_2,1)^n$) defined by $\theta\mapsto\theta(\sigma_n)$. Moreover, $w_n$ is injective on elements of degree at most $n$. It is shown  that the following commutes:
\[\xymatrix{
\Gamma(M)\ar[r]_p\ar[d]^\psi&\Steen\ar[d]_{w_n\times w_n}\ar[rd]^{w_{2n}}\\
\Steen\otimes\Steen\ar[r]^{w\otimes w\qquad \ }&H^*(K_n)\otimes H^*(K_n)\ar@{=}[r]&H^*(K_{2n})
}\]
Then, given an element $z\in\Gamma(M)$ of maximum degree $n$ such that $p(z)=0$, we must have $\psi(z)=0$ (by the diagram). This shows that the map $\psi$ descends.% Note that the commutativity of the diagram follows formally from the Cartan formula on elements $\Squ^i$, see the text for the extension to all of $\Gamma(M)$.
\end{proof}
In particular, $\Steen$ is a Hopf algebra. It is associative but not commutative, while it is both co-associative and co-commutative.
\subsection{The dual of the Steenrod algebra}
Suppose that $A$ is a (connected) Hopf algebra over a field $R$ of finite type (each graded part is finite dimensional). Then the \emph{dual Hopf algebra} $A^*$ is the componentwise dual of $A$, with multiplication the dual of the old comultiplication, etc. If the old multiplication is associative (commutative), then the new comultiplication is so, and vice versa.

In particular, $\Steen^*$ is a Hopf algebra which is associative, commutative, and coassociative.

Let $x$ be the generator of $H^1(K(\Z_2,1);\Z_2)$. For each $i\geq0$, let $\xi_i\in\Steen^*_{2^i-1}$ be characterised by:
\[\theta(x)=\xi_i(\theta)\cdot x^{2^i},\text{ for all }\theta\in\Steen_{2^i-1}.\]
\begin{prop*}
If $k\geq1$ and $I$ is admissible, then $\langle\xi_k,\Squ^I\rangle=\begin{cases}1,&I=I_k:=\{2^{k-1},2^{k-2},\ldots,1\};\\0,&\textup{otherwise.}\end{cases}$ Moreover, for $I$ inadmissible, $\langle\xi_k,\Squ^I\rangle=0$ unless $I$ is obtained from $I_k$ by interspersion of zeros.
\end{prop*}
As an algebra, $\Steen^*$ is the polynomial ring $\Z_2[\xi_1,\xi_2,\ldots]$.

\subsection{Algebras over a Hopf algebra}
If $A$ is a Hopf algebra and $M$ is a graded $A$-module (i.e.\ $M$ is a module of the algebra $A$), then $M\otimes M$ has a natural structure of graded $A$-module via
\[\xymatrix{
A\otimes M\otimes M\ar[r]^{\psi\otimes1\otimes1\ \ \ }&A\otimes A\otimes M\otimes M\ar[r]^T&A\otimes M\otimes A\otimes M\ar[r]^{\qquad\mu\otimes \mu\ }&M\otimes M
}\]
Now if $M$ is also an $R$-algebra, say that $M$ is an \emph{algebra over the Hopf algebra} $A$ is multiplication $M\otimes M\to M$ is an $A$-module homomorphism.

For example, $H^*(X;\Z_2)$ is a module over $\Steen$, and by the Cartan formula (which is reflected in the diagonal map), it is actually an algebra thereover.
\subsection{The diagonal map of \texorpdfstring{$\Steen^*$\ \ \ $\bigotimes$}{A}}
\subsection{The Milnor basis for \texorpdfstring{$\Steen$}{A}}
For any sequence $R$ of nonnegative integers (all but finitely many zero), we have an element $\xi^R$ of $\Steen^*$. These elements are a basis of $\Steen^*$, and the dual basis $\{\Squ^R\}$ of $\Steen$ is referred to as the Milnor basis. It is very different from the Serre-Cartan basis, but we do have $\Squ^{\{i,0,\ldots\}}=\Squ^i$.

\section{Exact couples and spectral sequences}

\section{Fiber Spaces}
\begin{itemise}
\item By fibration, we mean 'Serre fibration'.
\item In a Hurewicz fibration, all fibres are homotopic, while in a Serre fibration, we can only assert this when both fibres are finite complexes.
\item If $C\downarrow B$ is a covering space ($C$ connected), then the fiber is the discrete group $G$ of covering transformations. Moreover, the LES gives $\pi_n(C)=\pi_n(B)$ for $n\geq2$, and a short exact sequence $0\to\pi_1(C)\to\pi_1(B)\to G\to0$.
\item $p:\Orth(n+1)\to S^n$ given by evaluation at a fixed point of $S^n$, has fibre $\Orth(n)$. Then the LES gives $\pi_{i+1}(S^n)\to\pi_i(\Orth(n))\to\pi_i(\Orth(n+1))\to\pi_i(S^n)$, and the middle map is an isomorphism when $i<n-1$. Thus the groups $\pi_i(\Orth(n))$ are isomorphic for $n>i+1$. The stable values are known, and satisfy Bott periodicity.
\end{itemise}
\setcounter{subsection}{3}
\subsection{Serre's exact sequence}
Suppose that $F\rightarrow E\downarrow B$ is a fibre space, with $B$ simply connected. Suppose further that $H_i(B)=0$ for $0<i<p$ and $H_j(F)=0$ for $0<j<q$. Then there is a (finite) exact sequence
\[\xymatrix{
H_{p+q-1}(F)\ar[r]^{i_*}&H_{p+q-1}(E)\ar[r]^{p_*}&H_{p+q-1}(B)\ar[r]^{\tau}&H_{p+q-2}(F)\ar[r]&\cdots\ar[r]&H_1(E)\ar[r]&0
}\]
This can be used to prove the Hurewicz and relative Hurewicz theorems.
\begin{thm*}\label{Whthd}
Suppose $f:X\hookrightarrow Y$ is an inclusion, where $X$ is arc-connected, $Y$ is simply connected, and $\pi_2(Y,X)$ is abelian. Then the following are equivalent (for any $n\geq1$):
\begin{itemise}
\item[1.] $f_*:H_i(X)\to H_i(Y)$ is iso for $i<n$ and epi for $i=n$;
\item[2.] $f_\#:\pi_i(X)\to \pi_i(Y)$ is iso for $i<n$ and epi for $i=n$;
\item[3.] The pair $(Y,X)$ is $n$-connected; and
\item[4.] $H_i(Y,X)=0$ for $i\leq n$. \hfill (Compare with section \ref{WhthdModC})
\end{itemise}
\end{thm*}
\begin{proof}
2 and 3 are equivalent, as are 1 and 4, by exactness in the long exact sequences. Moreover, 3 and 4 are equivalent by the relative Hurewicz theorem, as long as $\pi_2(Y,X)$ is abelian.
\end{proof}
\begin{cor*} If $X$ and $Y$ are simply connected, and $f:X\to Y$, then $f$ is a homology isomorphism if and only if it is a homotopy isomorphism.
\begin{proof}
Replace $f$ by an inclusion. We need $\pi_2(Y,X)$ abelian, but have exact sequence $\pi_2(Y)\to\pi_2(Y,X)\to0$.
\end{proof}
\end{cor*}
\setcounter{subsection}{5}
\subsection{The cohomology spectral sequence of a fibre space}
\begin{itemise}
\squishlist
\item Each $E_r$ is a bigraded ring.
\item In $E_r$, $d_r$ is an anti-derivation (with respect to \emph{total} degree): $d(ab)=d(a)b+(-1)^{|a|}ad(b)$.
\item The product on $E_{r+1}$ is induced by that of $E_r$, and that on $E^\infty$ is induced by that in $H^*(E;R)$.
\item When $B$ and $F$ are $(p-1)$- and $(q-1)$-connected, respectively, we get:
\[\xymatrix{
\cdots\ar[r]&H^{p+q-2}(F)\ar[r]^\tau&H^{p+q-1}(B)\ar[r]^{p^*}&H^{p+q-1}(E)\ar[r]^{i^*}&H^{p+q-1}(F).
}\]
\item If $x$ is transgressive, then so is $\Squ^i(x)$, and if $\tau(x)=[y]$, $\tau(\Squ^i(x))=[\Squ^i(y)]$.
\end{itemise}
To see the last point, consider the following diagram which describes the transgression:
\[\xymatrix@R=5mm{
H^*(F)\ar[r]&H^{*+1}(E,F)&E_\infty^{*+1,0}\\
E_\infty^{0*}\ar@{^{(}->}[u]\ar@{.>}[rur]&H^{*+1}(B,*)\ar[u]\ar[r]&H^{*+1}(B)\ar@{->>}[u]
}\]
By the naturality of $\Squ^i$, and compatibility with the coboundary, we have:
\[\xymatrix@R=5mm{
x\ar@{|->}[r]&\delta x&[y]\\
x\ar@{|.>}[rur]\ar@{|->}[u]&z\ar@{|->}[u]\ar@{|->}[r]&y\ar@{|->}[u]
}\raisebox{-.6cm}{\qquad\bigrightsquig\qquad }
\xymatrix@R=5mm{
\Squ^ix\ar@{|->}[r]&\delta\,\Squ^ix&[\Squ^iy]\\
\Squ^ix\ar@{|.>}[rur]\ar@{|->}[u]&\Squ^iz\ar@{|->}[u]\ar@{|->}[r]&\Squ^iy\ar@{|->}[u]
}\]

\section{Cohomology of \texorpdfstring{$K(\pi,n)$}{K(G,n)}}
We can attempt to use the path-loop fibration and the fibration of Eilenberg-MacLane spaces associated with a short exact sequence of abelian groups to calculate $H^*(K(\pi,n);G)$.
\subsection{\texorpdfstring{$H^*(K(Z_2,q);Z_2)$}{H*(K(Z/2,q);Z/2)} and Borel's theorem}
A graded ring $R$ over $\Z_2$ has an order set $x_1,x_2,\ldots$ of homogeneous elements as a \emph{simple system of generators} if the monomials $\{x_{i_1}\cdots x_{i_r}:i_1<\cdots<i_r\}$ form a $\Z_2$-basis, and each graded part is finite dimensional.
\begin{thm*}
Suppose $F\rightarrow E\downarrow B$ is a fiber space with $E$ acyclic, and that $H^*(F;\Z_2)$ has a simple system $\{x_\alpha\}$ of transgressive generators. Then $H^*(B;\Z_2)$ is a polynomial ring in the $\{\tau(x_\alpha)\}$.
\end{thm*}
The following results then follow from Borel's theorem:
\begin{thm*}
$H^*(K(\Z_2,q);\Z_2)=\Z_2[\Squ^I(\imath_q)]$, over those admissible $I$ with excess less than $q$.
\end{thm*}
\begin{prop*}
$H^*(K(\Z,2);\Z_2)=\Z_2[i_2]$.
\end{prop*}
\begin{thm*}$H^*(K(\Z,q);\Z_2)=\Z_2[\Squ^I(\imath_q)]$, over those admissible $I$ with excess less than $q$ and last entry not equal to 1.
\end{thm*}
\begin{prop*}$H^*(K(\Z_{2^m},1);\Z_2)=\Z_2[d_m(\imath_q)]\otimes \Lambda(\imath_1)$ for $m\geq2$, where $d_m$ denotes the $m^\textup{th}$ differential in the Bockstein spectral sequence.
\end{prop*}
\begin{proof}
We calculate that $H^1(K(\Z_2,1);\Z)=\Z_{2^m}$. Then (using universal coefficients, if you will) $H^1(K(\Z_2,1);\Z_2)=\Z_{2}\langle\imath_1\rangle$. Moreover, as $\imath_1$ comes from a class of order $2^m$ and no further, we have $d_j(\imath_1)=0$ for $j<m$ and $d_m(\imath_1)\neq0$. Thus $d_m\imath_1$ is a nonzero element of $H^2$. Moreover, as $\imath_1^2=d_1\imath_1$, we have $\imath_1^2=0$. Everything else needed comes from the spectral sequence.
\end{proof}
\begin{thm*}
$H^*(K(\Z_{2^m},q);\Z_2)$ is a polynomial ring generated by
$\{\Squ^{I_m}\imath_q\}$, for $I$ admissible satisfying $e(I)<d$. Here,
$\Squ^{I_m}$ refers to $\Squ^I$ when the last entry $i_r$ is not 1, and to
$\Squ^{i^1}\cdots \Squ^{i_{r-1}}d_m$ when $i_r=1$. Moreover, $\imath_q$ corresponds
to reduction $H^q(\DASH;\Z_{2^m})\to H^q(\DASH;\Z_{2})$, and $\Squ^1\imath_q=0$.
\end{thm*}
\subsection{Proof of Borel's Theorem}
Here we consider only first quadrant spectral sequences of standard cohomological type such that $E_2^{pq}=E_2^{p0}\otimes E_2^{0p}$. A homomorphism of such is to be a collection of maps $f_r^{pq}$ which commute with the differentials, such that $f^{pq}_{r+1}$ is the map induced  on cohomology by $f^{pq}_r$, and such that $f_r^{pq}=f_r^{p0}\otimes f_r^{0q}$. We have a comparison theorem:
\begin{thm*}
Suppose that $F:E\to\overline E$ is such a homomorphism of such spectral sequences, and that both $E_\infty$ and $\overline E_\infty$ are concentrated at $(0,0)$. Then if all the maps $f_2^{0q}$ (or alternatively all the $f_2^{p0}$) are isomorphisms, all the $f_r^{pq}$ are isomorphisms for $r\geq2$.
\end{thm*}
\begin{proof}[Proof of Borel's theorem]
\newcommand{\shrt}[1]{\makebox[0cm]{\ensuremath{#1}}}
%It's enough to show that all the $f_2^{p0}$ are isomorphisms.
The idea is to let $\overline E$ be the spectral sequence for the fibration of Borel's theorem, and to construct another spectral sequence $E$ and map $f$ such that the comparison theorem applies. Let the simple system of transgressive generators be $\{x_\alpha\}$ (and assume for ease that all have positive degree).

Now we synthesise a filtered differential bigraded algebra whose associated spectral sequence mimics the behaviour of $\overline E$. Define:
\[P:=\Z_2[y_\alpha]\text{ and }\Lambda:=\Lambda_{\Z_2}(z_\alpha)\text{ where $|y_\alpha|=|x_\alpha|+1$, and $|z_\alpha|=|x_\alpha|.$}\]
Let $R$ be the bigraded algebra $P\otimes\Lambda$, with differential specified by $\delta(y\otimes z_\alpha)=yy_\alpha$ (extended by the requirement it's a derivation). Finally, we filter $R$ by $F^p=\sum_{i\geq 0}P^{p+i}\otimes\Lambda$. This decreasing filtration has subquotients denoted $\Gr^p:=F^p/F^{p+1}$ which additively is just $P^p\otimes\Lambda$.
%\[\xymatrix{
%0\ar[rr]|{\ \cdot\ \cdot\ \cdot\ }&&H^*(F^{p+2})\ar[rr]\ar[ld]&&H^*(F^{p+1})\ar[rr] \ar[ld]&&H^*(F^{p})\ar[ld]
%\ar[rr]|{\ \cdot\ \cdot\ \cdot\ }&&H^*(R)\\
%&\ \ \ \ \shrt{H^*(\Gr^{p+2})}\phantom{\binom{5}{3}}&&
%\shrt{H^*(\Gr^{p+1})}\phantom{\binom{5}{3}}\ar@{~>}[lu]_\delta&&
%\shrt{H^*(\Gr^{p})}\phantom{\binom{5}{3}}\ar@{~>}[lu]_\delta&&
%\shrt{H^*(\Gr^{p-1})}\phantom{\binom{5}{3}}\ar@{~>}[lu]_\delta
%}\]
\[\xymatrix{
0\ar[rr]|{\ \cdot\ \cdot\ \cdot\ }&&H^*(F^{p+1})\ar[rr] \ar[ld]&&H^*(F^{p})\ar[ld]
\ar[rr]|{\ \cdot\ \cdot\ \cdot\ }&&H^*(R)\\
&
\shrt{H^*(\Gr^{p+1})}\phantom{\binom{5}{3}}&&
\shrt{H^*(\Gr^{p})}\phantom{\binom{5}{3}}\ar@{~>}[lu]_\delta&&
\shrt{H^*(\Gr^{p-1})}\phantom{\binom{5}{3}}\ar@{~>}[lu]_\delta
}\]
Now in the corresponding spectral sequence $E$, the $E_1^{pq}$ term is then $H^{p+q}(\Gr^p)$. Moreover, the differential $d_r$ is defined as follows. Take $[x]\in H^{p+q}(\Gr^p)$ which has persited to $E_r$. Then $\delta x$  in fact lies in $\sum_{i\geq r}P^{p+i}\otimes \Lambda^{q-i+1}$, as all the previous differentials vanished. Then $d_r[x]$ is the projection of $\delta x$ onto the first factor $P^{p+r}\otimes \Lambda^{q-r+1}$.

Now note that $d_1=0$, as all the $x_\alpha$ have positive degree. Thus $E_2^{pq}=R^{pq}$, and that $E$ has the properties of the opening paragraph. The acyclicity is easily verified if $\{x_\alpha\}$ has only one element, and follows in general by the K\"unneth formula.

Now define a ring homomorphism $f_2^{*0}:P\to H^*(B;\Z_2)$ by $y\mapsto \tau(x_\alpha)$, and an additive (not multiplicative) homomorphism $f_2^{0*}:\Lambda\to H^*(F;\Z_2)$ by $z_{i_1}\cdots z_{i_r}\mapsto x_{i_1}\cdots x_{i_r}$. As then $f_2^{**}$ has image written in terms of the transgressive $x_\alpha$ and their images $\tau(x_\alpha)$, it is easy to check that this map is compatible with $d_2$ --- every differential vanishes on $x_\alpha$ but the transgression, and they all vanish on $\tau(x_\alpha)$. This allows us to define $f_3$, and a similar check shows that it is compatible with $d_3$. This argument continues ad infinitum, producing a morphism $E\to\overline E$, to which we may apply the comparison theorem. As the $f_2^{0*}$ are isomorphisms, so are the $f_2^{*0}$, completing the proof.
%
% Now we can only define $f_r$ when $f_{r+1}$
%
%Note to self: Next discuss the map as defined. Show that it intertwines the differentials, by noting that the $x_\alpha$ all transgress, and so are killed by any differential which doesn't make it to the opposite edge.
% which is a sum of terms of the form
\end{proof}

\section{Classes of abelian groups}
A \emph{class of abelian groups} is a collection $\scrC$ of abelian groups that is closed under formation of subgroups, quotient groups, and group extensions. That is, given a short exact sequence $0\to A'\to A\to A''\to0$, $A\in\scrC$ iff $A',A''\in\scrC$.

A homomorphism $f:A\to B$ is a \emph{$\scrC$-monomorphism} if its kernel lies in $\scrC$, a \emph{$\scrC$-epimorphism} if its cokernel lies in $\scrC$, and a \emph{$\scrC$-isomorphism} if both its kernel and cokernel lie in $\scrC$. Say two groups are $\scrC$-isomorphic if they are connected by a chain of $\scrC$-isomorphisms (which needn't be composable).

Further axioms that Serre classes may satisfy include:
\begin{itemise}
\item[2A.] being closed under formation of tensor products and $\Tor$;
\item[2B.] if  $A\in\scrC$ then $A\otimes B\in\C$ for all abelian groups $B$ (this implies 2A); and
\item[3.] if $A\in\scrC$ then $H_n(K(A,1);\Z)\in\scrC$ for all $n>0$.
\end{itemise}
Examples of Serre classes include:
\begin{itemise}
\item $\scrC_{FG}$, the finitely generated abelian groups, which satisfy 2A and 3 (but not 2B).
\item $\scrC_{p}$ for $p$ prime, the class of abelian torsion groups of finite exponent $n$ coprime to $p$. Here, the exponent of $A$ is the infimum of those $n$ such that $na=0$ for all $a\in A$. If $(n,p)=1$ then all elements have order coprime to $p$. This class satisfies 2B and 3.
\item The countable abelian groups. These satisfy 2A (but not 2B) and I haven't thought about 3.
\end{itemise}
\setcounter{subsection}{1}
\subsection{Topological theorems mod \texorpdfstring{$\scrC$}{C}}
\begin{thm*}[Hurewicz mod $\scrC$]
Suppose that $X$ is simply connected, and that a Serre class $\scrC$ satisfies \textup{2A} and \textup{3}. Then if $\pi_i(X)\in\scrC$ for $i<n$ then $H_i(X)\in\scrC$ for $i<n$ and $h:\pi_n(X)\to H_n(X)$ is a $\scrC$-isomorphism.
\end{thm*}
This has the immediate consequence that the homotopy groups of a $1$-connected complex with only finitely many cells in each dimension are finitely generated (using $\scrC_{FG}$).
\begin{thm*}[Relative Hurewicz mod $\scrC$]
Suppose that $A\subseteq X$, that both $A$ and $X$ are simply connected, and that $\pi_2(A)\to\pi_2(X)$ is an epimorphism. Suppose that $\scrC$ satisfies \textup{2B} and \textup{3}. Then if $\pi_i(X,A)\in\scrC$ for $i<n$ then $H_i(X,A)\in\scrC$ for $i<n$ and $h:\pi_n(X,A)\to H_n(X,A)$ is a $\scrC$-isomorphism.
\end{thm*}
Note that this is a little weaker than the standard Relative Hurewicz theorem (immediately below). Mod $\scrC$, we need both $A$ and $X$ simply connected.
\begin{thm*}[Standard Relative Hurewicz]
Suppose that $(X,A)$ is an $(n-1)$-connected pair of path-connected spaces with $n\geq2$ and $A\neq\emptyset$. Then $H_i(X,A)=0$ for $i<n$ and $h:\pi_n(X,A)\to H_n(X,A)$ is an isomorphism.
\end{thm*}
\begin{thm*}[Whitehead's theorem mod $\scrC$]\label{WhthdModC}
Suppose that $f:A\to X$ is a map of simply connected spaces that induces an isomorphism on $\pi_2$. Suppose that $\scrC$ satisfies \textup{2B} and \textup{3}. Then TFAE:
\begin{itemise}
\item[1.] $f_*:H_i(X)\to H_i(Y)$ is $\scrC$-iso for $i<n$ and $\scrC$-epi for $i=n$; and
\item[2.] $f_\#:\pi_i(X)\to \pi_i(Y)$ is $\scrC$-iso for $i<n$ and $\scrC$-epi for $i=n$. \hfill (Compare with section \ref{Whthd}).
%\item[3.] The pair $(Y,X)$ is $n$-connected; and
%\item[4.] $H_i(Y,X)=0$ for $i\leq n$.
\end{itemise}
\end{thm*}
Note that this is a corollary of the relative Hurewicz theorem mod $\scrC$.


\setcounter{subsection}{5}
\subsection{The \texorpdfstring{$\scrC_p$}{Cp} approximation theorem}
Let $X$ and $A$ be simply connected (nice) spaces with $H^i(X)$ and $H^i(A)$ finitely generated for each $i$. Suppose $f:A\to X$ is a map inducing an epimorphism on $\pi_2$ (and assume without loss of generality that $f$ is an inclusion). Then conditions 1--6 are equivalent, and imply 7:
\begin{itemise}
\item[1.] $f^*:H^i(X;\Z_p)\to H^i(A;\Z_p)$ is iso for $i<n$ and mono for $i=n$;
\item[2.] $f_*:H_i(A;\Z_p)\to H_i(X;\Z_p)$ is iso for $i<n$ and epi for $i=n$;
\item[3.] $H_i(X,A;\Z_p)=0$ for $i\leq n$;
\item[4.] $H_i(X,A;\Z)\in\scrC_p$ for $i\leq n$;
\item[5.] $\pi_i(X,A)\in\scrC_p$ for $i\leq n$;
\item[6.] $f_*:\pi_i(A)\to \pi_i(X)$ is $\scrC_p$-iso for $i<n$ and $\scrC_p$-epi for $i=n$;
\item[7.] $\pi_i(A)$ and $\pi_i(X)$ have the same $p$-components for $i<n$.
\end{itemise}
That is, to calculate the $p$-component of $\pi_i(X)$, it is enough to find a map $A\to X$ where inducing isomorphisms on $H^*(\DASH;\Z_p)$ (as long as both spaces are nice, etc).
\begin{proof}
$1\Leftrightarrow 2$ by duality, noting finite generation. $2\Leftrightarrow 3$ using the LES. $3\Rightarrow 4$ as $H_i(X,A)\otimes\Z_p\subset H_i(X,A;\Z_p)$ by the UCT. $3\Leftarrow 4$ using the UCT an axiom 2B. $4\Leftrightarrow 5$ by relative Hurewicz. $5\Leftrightarrow 6$ using the exact homotopy sequence. To see that these imply 7, note that $\pi_i(A)$ and $\pi_i(X)$ are finitely generated, by Hurewicz mod $\scrC_{FG}$. So there is a $\scrC_p$-iso between the groups in question, which are finitely generated. Thus the $p$-components are the same.
\end{proof}

\section{More about fiber spaces}
Homotopic maps induce fiber homotopy equivalent pullback fiber spaces.
\begin{prop*}
Suppose that $E\downarrow B$ is a Hurewicz fibration, and we have maps $Y\to X\to B$:
\[\xymatrix{
&f^*E\ar[r]\ar[d]&E\ar[d]\\
Y\ar[r]^g\ar@{-->}[ur]^h&X\ar[r]^f&B
}\]
Then if $fg$ is null, a lifting $h$ of $g$ exists. If $E$ is contractible, the converse holds. (This works given a Serre fibration $E\downarrow B$, as long as $Y$ is a finite complex).
\end{prop*}

\setcounter{subsection}{1}
\subsection{The transgression of the fundamental class}
Consider  the fibration $F\rightarrow E\downarrow B$, where $F=K(\pi,n)$, $E$ is contractible and $B=K(\pi,n+1)$. We have the following isomorphisms from Hurewicz and long exact sequences:%LESs:
\[\xymatrix{
\pi_n(F)\ar[d]_h^\simeq&\pi_{n+1}(E,F)\ar[d]_h^\simeq\ar[l]^\simeq\ar[r]_\simeq&\pi_{n+1}(B,*)\ar[d]_h^\simeq\\
H_n(F)&H_{n+1}(E,F)\ar[l]_\simeq\ar[r]^\simeq&H_{n+1}(B,*)
}\]
Then entries on the top row are all canonically isomorphic to $\pi$. Applying $\Hom(\DASH,\pi)$ to the bottom row, we obtain (using the UCT and the definition of $\imath_n$):
\[\xymatrix@C=.7cm{
\Hom(H_n(F),\pi)\ar[d]^\simeq\ar[r]_{\simeq\ \ }&
\Hom(H_{n+1}(E,F),\pi)\ar[d]&
\Hom(H_{n+1}(B,*),\pi)\ar[d]^\simeq\ar[l]^\simeq&
h^{-1}\ar@{|->}[r]\ar@{|->}[d]&
h^{-1}\ar@{|->}[d]&
h^{-1}\ar@{|->}[l]\ar@{|->}[d]\\
H^n(F;\pi)\ar[r]^{\simeq\ \ }&H^{n+1}(E,F;\pi)&H^{n+1}(B,*;\pi)\ar[l]_\simeq&
\imath_n\ar@{|->}[r]&\jmath&\imath_{n+1}\ar@{|->}[l]
}\]
This shows that in the SSS for this fibration, with $\pi$ coefficients, the fundamental class $\imath_n$ transgresses to the fundamental class $\imath_{n+1}$.

Next, suppose that we are given a map $f:X\to K(\pi,n+1)$, corresponding to $\psi\in H^{n+1}(X;\pi)$ (so that $\psi=f^*(\imath_n)$). Then we have a map of fiber sequences:
\[\xymatrix@R=5mm{
%K(\pi,n+1)\ar[r]&{*}\ar[r]&K(\pi,n)
K(\pi,n)\ar[r]\ar@{=}[d]&f^*(\star)\ar[d]\ar[r]&X\ar[d]^f\\
K(\pi,n)\ar[r]&\star\ar[r]&K(\pi,n+1)
}\]
This induces a map of cohomology Serre exact sequences, which show that:
\begin{equation*}
\tag{Formula 1}\psi=f^*(\imath_{n+1})=\tau(\imath_n)
\end{equation*}
That is, in the SSS for the top row fibre sequence, the fundamental class (in the cohomology of the fibre) transgresses to the element $\psi$ (in that of the base $X$).

\subsection{Bocksteins and the Bockstein lemma}
We have the Bockstein exact couple:
\[\xymatrix{
H^*(\DASH;\Z)\ar[rr]^{\times2}&&H^*(\DASH;\Z)\ar[dl]^\rho\\
&H^*(\DASH;\Z_2)\ar@{~>}[ul]^{\beta}
}\]
The operator $d_r$ is defined to be the $r^\textup{th}$ differential in the corresponding spectral sequence. In particular, one applies $\beta:H^*(\DASH;\Z_2)\to H^*(\DASH;\Z)$, divides the result by $2^{r-1}$ (which can be done iff the previous $d_i$ vanished), and reduces modulo $2$.

Now $d_i$ vanishes on any class in the image of $\rho:H^*(\DASH;\Z_{2^{i+1}})\to H^*(\DASH;\Z_{2})$. Thus we get maps:
\[d_i\rho:H^*(\DASH;\Z_{2^{i}})\to H^{*+1}(\DASH;\Z_{2})\text{ corresp.\ to }f_i:K(\Z_{2^i},n)\to K(\Z_{2},n+1).\]
\begin{lem*}
The pullback construction mentioned above gives
\[\xymatrix@R=5mm{
%K(\pi,n+1)\ar[r]&{*}\ar[r]&K(\pi,n)
K(\Z_2,n)\ar[r]\ar@{=}[d]&K(\Z_{2^{i+1}},n)\ar[d]\ar[r]&K(\Z_{2^i},n)\ar[d]^{f_i}\\
K(\Z_2,n)\ar[r]&\star\ar[r]&K(\Z_2,n+1)
}\]
\end{lem*}
\begin{proof}
The total space is the only thing in doubt. It's a $K(G,n)$ by the LES. Suppose that $G$ was $\Z_2\times\Z_{2^i}$. Then the fibration would be trivial -- a Cartesian product. In this case, the transgression is zero, as the SSS has zero differentials from page 2 onwards. However, the map $f_i$ is not null, so corresponds to a nonzero cohomology class $\psi$, so the transgression is nonzero.
\end{proof}
\begin{lem*}[Bockstein Lemma]
Suppose that $\xymatrix{F\ar[r]^j&E\ar[r]^p&B}$ is a fibration, and that $u\in H^n(F;\Z_2)$ is transgressive. Suppose also that $v\in H^n(B;\Z_2)$ satisfies $d_i(v)=\tau(u)$. Then  $d_1(u)=j^*d_{i+1}p^*(v)$.
\[\begin{sseq}[leak=0mm,grid=none,entrysize=1cm,labelstep=1,
xlabels={{}_0;;{}_n;{}_{n+1}},ylabels={{}_0;;{}_n;{}_{n+1}}]%]
{0...3}{0...3}
\ssmoveto 2 0 \ssdrop{v}\ssname{20}
\ssmoveto 0 2 \ssdrop{u}\ssname{02}
\ssmoveto 0 3 \ssdrop{d_1(u)\makebox[0cm][l]{\,\,$=j^*d_{i+1}p^*(v)$}}\ssname{03}
\ssmoveto 3 0 \ssdrop{d_i(v)\makebox[0cm][l]{$\,\,=\tau(u)$}}\ssname{30}
\ssgoto{02} \ssgoto{03} \ssstroke[arrowto]
\ssgoto{02} \ssgoto{30} \ssstroke[arrowto]
\ssgoto{20} \ssgoto{30} \ssstroke[arrowto]
%\ssmoveto 3 2 \ssdrop{asdfasdfsafds}
\end{sseq}
\qquad\qquad\qquad\raisebox{2.5cm}{\textup{\underline{i.e.}}}\qquad\qquad
\raisebox{3.6cm}{$\xymatrix@!0@R=12mm@C=2.5cm{
H^n(B)\ar@/^.81pc/[rr]&H^n(F)\ar[ld]&H^n(E)\ar[l]\\
H^{n+1}(B)\ar@/^.81pc/[rr]&H^{n+1}(F)&H^{n+1}(E)\ar[l]\\
v\ar@{|->}[d]_{d_i}\ar@{|->}@/^.81pc/[rr]^{p^*}&
u\ar@{|->}[ld]^\tau\ar@{|->}[d]^{d_1}&
p^*v\ar@{|->}[d]_{d_{i+1}}\\
\fbox{$d_i(v)$}\makebox[0cm][l]{\ \ $\implies$}&\fbox{$d_1(u)$}&d_{i+1}p^*v\ar@{|->}[l]_{j^*}
}$}
\]%\ar@{|->}
\end{lem*}
\begin{proof}Neglected.\end{proof}
\subsection{Principal fiber spaces}
These are a generalisation of principal $G$-bundles. For any fibration $p:E\to B$, define $E^*$ to be the fiberwise cartesian square: the pullback of $p$ along $p$. Then $F\rightarrow E\downarrow B$ is a \emph{principal fiber space} if it is equipped with maps 
\[\varphi:E\times F\to E,\ \ h:E^*\to F.\]
Here $\varphi$ should be thought of as a fiber preserving action $F$ on $E$, and $h$ is a pairing on each fiber taking values in $F$. If we were actually talking about a principal $G$-bundle, then identifying $F$ with $G$, $\varphi$ is just the action, and $h(x_1,x_2)$ returns the element $g\in G$ such that $gx_1=x_2$.
\begin{itemise}
\item The action $\varphi$ respects the fibres.
\item The action restricted to $F\times F$ gives $F$ an H-space structure with strict unit and homotopy inverses.
\item The composite $\xymatrix{E^*\ar[r]^{(\pi_1,h)}&E\times F\ar[r]^{\varphi}&E}$ is homotopic to the second projection $\pi_2$.
\end{itemise}
For example, for any space $B$ let $X$ be the space of Moore paths over $B$. The action $\varphi$ is given by composing a loop with a path, and the map $h$ is given by sending $(e_1,e_2)$ to the path which follows $e_2$ then $e_1^{-1}$.
\begin{prop*}
Suppose $F\rightarrow E\downarrow B$ is a principal fibre space. Let $v,v':X\to E$ be two maps. Then $pv\simeq pv'$ iff there exists $w:X\to F$ such that $\varphi\circ(v,w)\simeq v'$:
\[\xymatrix@C=1.5cm{
X\ar@<.25ex>[r]^v\ar@<-.25ex>[r]_{v'}\ar@<.25ex>[dr]^(.63){pv}\ar@<-.25ex>[rd]_(.63){pv'}&E\ar[d]^p\\
&B
}\raisebox{-.6cm}{ \ Here, $pv\simeq pv'$ iff $\exists\, w$ s.t.\ \ }
\xymatrix{
&E\times F\ar[d]^\varphi\\
X\ar[ur]^{(v,w)}\ar[r]^{v'}&E
}\]
\end{prop*}

\section{Applications: Some homotopy groups of spheres}
\setcounter{subsection}{1}
\subsection{A better approximation to \texorpdfstring{$S^n$}{Sn}
 (and finding \texorpdfstring{$\pi_1^s$}{the stable 1-stem})}
Take $n$ large, for stability. View $K(\Z,n)$ as an approximation to $S^n$, and improve it, as follows. We know that $H^{n+1}(K(\Z,n);\Z_2)=0$ and that $H^{n+1}(K(\Z,n);\Z_2)=\Z_2\langle\Squ^2\imath_n\rangle$. This generator $\Squ^2\imath_n$ corresponds to a map $\Squ^2$ as drawn below:
\[\xymatrix@R=5mm{
%K(\pi,n+1)\ar[r]&{*}\ar[r]&K(\pi,n)
F_1\ar[r]\ar@{=}[d]&X_1\ar[d]\ar[r]&K(\Z,n)\makebox[0cm][l]{\,\,$=X_0$}\ar[d]^{\Squ^2}\\
K(\Z_2,n+1)\ar[r]&\star\ar[r]&K(\Z_2,n+2)
}\]
By (formula 1), in the SSS for the top row, $\imath_{n+1}$ transgresses to $\Squ^2\imath_n$. Moreover, by the compatibility of transgression with Steenrod squares, the generator $\Squ^1\imath_{n+1}$ of $H^{n+2}(F_1)$ transgresses to $\Squ^1\Squ^2\imath_{n}=\Squ^3\imath_{n}\neq0$.
We write down a part of the Serre exact sequence (with $\Z_2$ coefficients):
\[\xymatrix{
H^{n+1}(F_1)\ar@{->>}[r]^{\tau}&H^{n+2}(X_0)\ar[r]^{p^*}&H^{n+2}(X_1)\ar[r]^{i^*}&H^{n+2}(F_1)\ar@{^{(}->}[r]^{\tau}&H^{n+3}(X_0)
}\]
Thus $H^{n+2}(X_1;\Z_2)=0$. Note that to deduce this we needed one transgression surjective and the other injective!

Now we need a map $f_1:S^n\to X_1$, which should lift the map $f_0:S^n\to X_0$ corresponding to the generator of $H^n(S^n)$. This is easy, as we have a morphism of homotopy long exact sequences:
\[\xymatrix@R=5mm{
\pi_n(X_1)\ar[r]\ar[d]&\pi_n(X_0)\ar[r]\ar[d]&\pi_{n-1}(F_1)\ar@{=}[d]\\
\pi_n(\star)\ar[r]&\pi_n(K(\Z_2,n+2))\ar[r]&\pi_{n-1}(F_1)\\
}\raisebox{-.6cm}{\ \ $\bigrightsquig$\ \ }
\xymatrix@R=5mm{
f_1\ar@{|->}[r]&f_0\ar@{|->}[r]\ar@{|->}[d]&0\ar@{|->}[d]\\
&0\ar@{|->}[r]&0
}\]
Now $f_1$ is an isomorphism on $H^*(\DASH;\Z_2)$ for $*\leq n+1$, and 
monomorphic for $*=n+2$. So it's a $\scrC_2$-iso on $\pi_*$ for 
$*\leq n+1$ (and epi for $*=n+2$). The LES for the top fibration gives 
$\pi_{n+1(X_1)}=\Z_2$, so that $\pi_1^s=\Z_2$.

\subsection{Calculation of \texorpdfstring{$\pi_k^s$ for $k\leq7$}{stable 
k-stem, k<8}}
The plan of attack is as follows. Choose $n$ very large, so that $H^*(K(\pi,n))$
does not contain cup products in the range of interest, simplifying everything.
Start with $X_0=K(\Z,n)$, and produce fibrations $F_i\to X_i\to X_{i-1}$ so that
$H^{n+*}(X_i;\Z_2)=0$ for $1\leq i\leq r+1$, for $r$ as large as possible. Then 
show that the map $f_0:S^n\to X_0$ lifts to a map $f_i:S^n\to X_i$, which is an
isomorphism on $H^{n+*}(\DASH;\Z_2)$ for $*\leq r+1$.

Then we have, by the $\scrC_2$ approximation theorem, that $f_i$ induces 
$\scrC_2$-isomorphisms on $\pi_{n+*}$ for $*\leq r$. The homotopy groups
of $X_i$ should be easy to work out from the long exact sequences, so that we
have found the $2$-components of $\pi_*^s$ for $*\leq r$.

\subsubsection{Discussion: Producing the fibre sequence}
Suppose that $X_{i-1}$ has $H^{n+*}(X_{i-1};\Z_2)=0$ for $1\leq i\leq r-1$, but
 $H^{n+r}(X_{i-1};\Z_2)\neq0$. We would like to kill this group. One attempt
would be to use the map $X_{i-1}\to K(n+r;\Z_2)$ to corresponding to an element
$\gamma\in H^{n+r}(X_{i-1};\Z_2)$. Then we would pull back the standard
path-loop fibration along this map:
\[\xymatrix@R=5mm{
%K(\pi,n+1)\ar[r]&{*}\ar[r]&K(\pi,n)
F_i\ar[r]\ar@{=}[d]&
X_i\ar[d]\ar[r]&
X_{i-1}\ar[d]^\gamma\\
K(\Z_2,n+r-1)\ar[r]&\star\ar[r]&K(\Z_2,n+r)
}\]
In the Serre LES, we then have ($\Z_2$ coeffs):
\[\xymatrix@R=1.5mm{
H^{n+r-1}(F_i)\ar@{->}[r]^{\tau}&
H^{n+r}(X_{i-1})\ar[r]^{p^*}&
H^{n+r}(X_i)\ar[r]^{i^*}&
H^{n+r}(F_i)\ar@{->}[r]^{\tau\ \ \ }&%^{(}
H^{n+r+1}(X_{i-1})\\
\Z_2\langle\imath_{n+r-1}\rangle\ar@{=}[u]&
&&
\Z_2\langle\Squ^1\imath_{n+r-1}\rangle\ar@{=}[u]\\
\imath_{n+r-1}\ar@{|->}[r]&\gamma&&
\Squ^1\imath_{n+r-1}\ar@{|->}[r]&\Squ^1\gamma
}\]
In particular, if $\Squ^1\gamma\neq0$, then we have removed $\gamma$ from the 
cohomology at no cost. However, if $\Squ^1\gamma=0$, we have gotten rid of 
$\gamma$ but are stuck with something new! Thus, when this plan doesn't work, 
we must do something smarter.

For the calculations of this chapter, it seems that it is good enough to use a
fibration as follows, where $\gamma\in H^{n+r}(X_{i-1};\Z_{2^m})$ reduces to that
which is to be killed in $H^{n+r}(X_{i-1};\Z_2)$:
\[\xymatrix@R=5mm{
%K(\pi,n+1)\ar[r]&{*}\ar[r]&K(\pi,n)
F_i\ar[r]\ar@{=}[d]&
X_i\ar[d]\ar[r]&
X_{i-1}\ar[d]^\gamma\\
K(\Z_{2^m},n+r-1)\ar[r]&\star\ar[r]&K(\Z_{2^m},n+r)
}\]
As $F_i$ is an Eilenberg-MacLane space, its cohomology has basis given by 
various $\Squ^I\imath_{n+r-1}$ through a large range.
Now $\imath_{n+r-1}$ transgresses to $\gamma$, so compatibility with squaring
exactly describes the transgression.
Thus, using the long exact sequence, we can calculate $H^*(X_i)$ from 
$H^*(X_{i-1})$.
Recall the important theorems:
\begin{thm*}$H^*(K(\Z,q);\Z_2)=\Z_2[\Squ^I(\imath_q)]$, over those admissible $I$
 with excess less than $q$ and last entry not equal to 1.
\end{thm*}
\begin{thm*}
$H^*(K(\Z_{2^m},q);\Z_2)$ is a polynomial ring generated by
$\{\Squ^{I_m}\imath_q\}$, for $I$ admissible satisfying $e(I)<d$. Here,
$\Squ^{I_m}$ refers to $\Squ^I$ when the last entry $i_r$ is not 1, and to
$\Squ^{i^1}\cdots \Squ^{i_{r-1}}d_m$ when $i_r=1$. Moreover, $\imath_q$ corresponds
to reduction $H^q(\DASH;\Z_{2^m})\to H^q(\DASH;\Z_{2})$, and $\Squ^1\imath_q=0$.
\end{thm*}
\subsection{The calculation (continues)}
%\newpage 
%\changetext{0cm}{0cm}{0cm}{0cm}{}
%\small{
%\newcommand{\MyRuLe}{\makebox[0cm][l]{\hspace{-.4cm}\underline{\phantom{q}\hspace{15cm}}}}
%\newcommand{\MyVRuLe}[1]{
%    \raisebox{0cm}[0cm][0cm]{
%         \makebox[0cm][l]{#1\rule[-24cm]{1pt}{24.5cm}}
%    }
%}
%\newcommand{\Splus}{\mbox{+}}
%\[\xymatrix@R=.1cm@C=.051cm{ 
%\MyRuLe\raisebox{-.5mm}{\MyRuLe} k\MyVRuLe{\ \ } & X_0 & K(\Z_2,n\Splus1)&X_1 & K(\Z_2,n\Splus2)&X_2 & K(\Z_8,n\Splus3)&X_3 & K(\Z_2,n\Splus6)\\
%\MyRuLe0  		&\imath_n 				&					&\imath_n				&					&\imath_n				&					&\imath_n				&\\
%\MyRuLe1  		&					&\imath_{n\Splus1}\ar[ld]	&					&					&					&					&					&\\
%\MyRuLe2  		&2					&1		\ar[ld]		&					&\imath_{n\Splus2}		&					&					&					&\\
%\MyRuLe3  		&3					&2					&\alpha(2)				&1					&					&\imath_{n\Splus3}		&					&\\
%4         		&4					&3					&\Squ^4\imath_n			&2					&\Squ^4\imath_n			&d_3\imath_{n\Splus3}		&				&\\
%\MyRuLe   		&					&21		\ar[ld]		&\beta(3)				&					&					&					&					&\\
%5         		&5					&31					&\gamma(31)			&3					&A(3)					&2					&					&\\
%\MyRuLe   		&					&4		\ar[ldd]		&					&21					&					&					&					&\\
%6         		&6					&5		\ar[lddd]		&\Squ^6\imath_n			&31					&\Squ^6\imath_n			&3					&					&\imath_{n\Splus6}\\
%\MyRuLe   		&42					&41		\ar[ldd]		&\delta(5\Splus41)		&4					&					&2d_3					&					&\\
%7         		&7					&6		\ar[ldddd]		&\Squ^7\imath_n			&5					&\Squ^7\imath_n			&4					&P(4)					&1\\
%          		&52					&51					&\varepsilon(51)		&41					&B(5\Splus41)			&3d_3					&					&\\
%\MyRuLe   		&					&42					&\zeta(42)				&					&					&					&					&\\
%8         		&8					&421 \ar[ldddd]\ar[lddddd]&\Squ^8\imath_n			&6					&\Squ^8\imath_n			&5					&\Squ^8\imath_n			&2\\
%          		&62					&7	\ar[ldddd]			&\eta(52)				&51					&C(51)				&4d_3					&Q(5)					&\\
%          		&					&61		\ar[ldddd]		&					&42					&					&					&					&\\
%\MyRuLe   		&					&52					&					&					&					&					&					&\\
%9         		&9					&8		\ar[lddddd]	&\theta(62)			&7					&D(52)				&6					&R(6)					&?\\
%          		&72					&71		\ar[lddddd]	&\kappa(521)			&421					&					&5d_3					&S(5d_3)				&\\
%          		&63					&62					&					&61					&					&42					&					&\\
%\MyRuLe   		&					&521					&					&52					&					&					&					&\\
%10        		&10					&9		\ar[ldddddd]	&\Squ^{10}\imath_n		&8					&\Squ^{10}\imath_n		&					&					&\\
%          		&82					&621\ar[lddddd]\ar[ldddddd]&\lambda(72)			&71					&					&					&					&\\
%          		&73					&81\ar[ldddd]\ar[lddddd]	&\nu(63)				&62			&					&					&					&\\
%          		&					&72					&\mu(9\Splus81\Splus621)	&521					&					&					&					&\\
%\MyRuLe   		&					&63					&					&					&					&					&					&\\
%11        		&11					&10	\ar[lddddddd]		&\Squ^{11}\imath_n		&					&					&					&					&\\
%          		&92					&91	\ar[lddddddd]		&\xi(82)				&					&					&					&					&\\
%          		&83					&82					&\pi(73)				&					&					&					&					&\\
%          		&					&73					&\rho(631)				&					&					&					&					&\\
%          		&					&721	\ar[ldddd]		&\sigma(91\Splus721)			&				&					&					&					&\\
%\MyRuLe   		&					&631					&					&					&					&					&					&\\
%12        		&12					&					&					&					&					&					&					&\\
%          		&102					&					&					&					&					&					&					&\\
%          		&93					&					&					&					&					&					&					&\\
%          		&84					&					&					&					&					&					&			&
%}\]}


\pagebreak
\vspace*{-1.5cm}
%\begin{figure}
\begin{adjustwidth}{-5em}{-5em}
%\changetext{0cm}{0cm}{0cm}{0cm}{}
\small{
\newcommand{\MyRuLe}{\makebox[0cm][l]{\hspace{-.4cm}\underline{\phantom{q}\hspace{16.3cm}}}}
\newcommand{\MyVRuLe}[1]{
    \makebox[0cm][l]{\raisebox{0cm}[0cm][0cm]{
         #1\rule[-24cm]{.051pt}{24.5cm}}%.1 works for width
    }
}
\newcommand{\jableft}[2]{\makebox[#1][r]{#2}}
\newcommand{\Splus}{\mbox{+}}
\newcommand{\KWhatWhat}[2]{\makebox[3cm][r]{$K(#1,#2)$}\!\!\!\MyVRuLe{}\!\!\!\!\!}
\renewcommand{\KWhatWhat}[2]{\jableft{1.7cm}{$K(#1,#2)$}\MyVRuLe{}\!\!\!\!\!}
\renewcommand{\KWhatWhat}[2]{\jableft{1.7cm}{$K(#1,#2)$}\MyVRuLe{}}
\[\xymatrix@R=.1cm@C=.051cm{ 
\MyRuLe\raisebox{-.5mm}{\MyRuLe} k\MyVRuLe{\ \ }
			& X_0 				&\KWhatWhat{\Z_2}{n\Splus1}&X_1				&\KWhatWhat{\Z_2}{n\Splus2}&X_2 				&\KWhatWhat{\Z_8}{n\Splus3}&X_3 			&\KWhatWhat{\Z_2}{n\Splus6}&X_4 			&K(\Z_{16},n\Splus7)\\%\KWhatWhat{\Z_{16}}{n\Splus6}\\
\MyRuLe0  		&\imath_n 				&					&\imath_n				&					&\imath_n				&					&\imath_n				&					&\imath_n				&\\
\MyRuLe1  		&					&\imath_{n\Splus1}\ar[ld]	&					&					&					&					&					&\\
\MyRuLe2  		&2					&1		\ar[ld]		&					&\imath_{n\Splus2}\ar[ld]	&					&					&					&\\
\MyRuLe3  		&3					&2					&\alpha(2)\ar[l]		&1\ar[ldd]	\ar@{.}[ld]	&					&\imath_{n\Splus3}\ar[ld]	&					&\\
4         		&4					&3					&\Squ^4\imath_n	\ar@{-->}@/_1.75pc/[dd]_{d_2}&2\ar[ldd]&\Squ^4\imath_n	\ar@{-->}[dd]_{d_3}&d_3\ar[ldd]&					&\\
\MyRuLe   		&					&21		\ar[ld]		&\beta(3)	\ar[lu]		&					&					&					&					&\\
5         		&5					&31					&\gamma(31)	\ar[l]	&3					&A(3)\ar[l]			&2	\ar[ldd]			&					&\\
\MyRuLe   		&					&4		\ar[ldd]		&					&21\ar@{.}[ld]\ar[ldd]	&					&					&					&\\
6         		&6					&5		\ar[lddd]		&\Squ^6\imath_n			&31\ar@{.}[ldd]\ar[lddd]&\Squ^6\imath_n			&3\ar[ldd]				&					&\imath_{n\Splus6}\ar[ldd]\\
\MyRuLe   		&42					&41		\ar[ldd]		&\delta(\cdots)\ar[lu]\ar[l]&4\ar@{.}[ld]\ar[lddd]&					&2d_3	\ar[ldd]\ar@{.}[ld]		&					&\\
7         		&7					&6		\ar[ldddd]		&\Squ^7\imath_n			&					&\Squ^7\imath_n			&4					&P(4)	\ar[l]			&		&&\imath_{n\Splus7}\ar[lddd]\\
          		&52					&51					&\varepsilon(51)\ar[l]	&5\ar@{~>}[ldd]\ar[lddd]	&B(\cdots)\ar[l]\ar[ld]&3d_3\ar[lddd]\ar@{.}[ldd]	&					&\\
\MyRuLe   		&					&42					&\zeta(42)	\ar[l]		&41\ar@{~>}[ld]\ar[ldd]	&					&					&					&1\ar[ldd]\ar@{.}[ld]\\
8         		&8					&421 \ar[ldddd]\ar[lddddd]&\Squ^8\imath_n	\ar@{-->}@/_2pc/[ddddddd]_{d_2}&	&\Squ^8\imath_n\ar@{-->}@/_2pc/[dddd]_{d_3}	&	&\Squ^8\imath_n			&&\Squ^8\imath_n&d_4\ar[ldddd]\\
          		&62					&52					&\eta(52)\ar[l]			&51					&C(51)\ar[l]			&5					&Q(5)\ar[l]			&&&\\
          		&					&7	\ar[lddd]			&					&6\ar[ddldd]			&					&4d_3	\ar[ldd]			&					&2\ar[ddl]\\
\MyRuLe   		&					&61		\ar[lddd]		&					&42\ar[ddldd]			&					&					&					&\\
9         		&9					&8		\ar[lddddd]	&					&52					&D(52)\ar[l]			&6					&R(6)	\ar[l]			&?&S'\ar[l]\\
          		&72					&71		\ar[lddddd]	&					&7\ar[ldddd]\ar@{.}[lddd]&					&5d_3					&S(5d_3)\ar[l]			&\\
          		&63					&62					&\theta(62)\ar[l]		&61\ar[ldddd]\ar@{.}[ldd]&					&42\ar[ldd]			&					&\\
\MyRuLe   		&					&521					&\kappa(521)\ar[l]		&421\ar[ldddd]\ar[ldd]\ar@{.}[ld]	&			&					&					&\\
10        		&10					&72					&\Squ^{10}\imath_n		&					&\Squ^{10}\imath_n		&					&					&\\
          		&82					&63					&\lambda(72)\ar[ul]		&8\ar[lddddd]\ar@{.}[ldddd]&					&					&					&\\
          		&73					&9		\ar[ldddd]		&\nu(63)\ar[ul]			&71\ar[lddddd]\ar@{.}[lddd]&					&					&					&\\
          		&					&621\ar[lddd]\ar[ldddd]	&\mu(\cdots)\ar[l]\ar[ul]\ar[dl]&62\ar[lddddd]\ar@{.}[ldd]	&			&					&					&\\
\MyRuLe   		&					&81\ar[ldd]\ar[lddd]		&					&521	\ar[lddddd]\ar@{.}[ld]&					&					&					&\\
11        		&11					&10	\ar[lddddddd]		&\Squ^{11}\imath_n		&					&					&					&					&\\
          		&92					&82					&\xi(82)\ar[l]			&					&					&					&					&\\
          		&83					&73					&\pi(73)\ar[l]			&					&					&					&					&\\
          		&					&631					&\rho(631)\ar[l]		&					&					&					&					&\\
          		&					&721	\ar[ldddd]			&\sigma(\cdots)\ar[l]\ar[ld]&					&					&					&					&\\
\MyRuLe   		&					&91	\ar[lddd]			&					&					&					&					&					&\\
12        		&12					&					&					&					&					&					&					&\\
          		&102					&					&					&					&					&					&					&\\
          		&93					&					&					&					&					&					&					&\\
          		&84					&					&					&					&					&					&
}\]}
\end{adjustwidth}
%\end{figure}

\subsubsection{\texorpdfstring{The cohomology of $X_1$}{The cohomology of X1}}
We define $X_1$ by the following pullback:
\[\xymatrix@R=5mm{
%K(\pi,n+1)\ar[r]&{*}\ar[r]&K(\pi,n)
F_1\ar[r]\ar@{=}[d]&
X_1\ar[d]\ar[r]&
X_{0}\ar[d]^{\Squ^2\imath_n}\\
K(\Z_2,n+1)\ar[r]&\star\ar[r]&K(\Z_2,n+2)
}\]
The cohomology of $X_1$ is easy to calculate from the Serre LES. We calculate
the transgression for $K(\Z_2,n+1)\rightarrow X_1\downarrow X_0$, simply using
the Adem relations. Then the cohomology is recorded under $X_1$ in the third
column.

It will be useful to know that $d_2(\Squ^4\imath_n)=\beta$ and that
$d_2(\Squ^8\imath_n)=\kappa$. For these, apply the Bockstein lemma, which looks
like
\[\xymatrix{
4\ar@{|->}[d]_{d_1}\ar@{|->}@/^.81pc/[rr]^{p^*}&
21\ar@{|->}[ld]^\tau\ar@{|->}[d]^{d_1}&
\Squ^4\imath_n\ar@{|->}[d]_{d_{2}}\\
5&
31&d_2(\Squ^4(\imath_n))\ar@{|->}[l]_{j^*}
}
\raisebox{-.8cm}{\qquad and\qquad\ }
\xymatrix{
8\ar@{|->}[d]_{d_1}\ar@{|->}@/^.81pc/[rr]^{p^*}&
421+7\ar@{|->}[ld]^\tau\ar@{|->}[d]^{d_1}&
\Squ^8\imath_n\ar@{|->}[d]_{d_{2}}\\
9&
521&d_2(\Squ^8(\imath_n))\ar@{|->}[l]_{j^*}
}
\]
We used ``$\Squ^{1,2n}=\Squ^{2n+1}$'', ``$\Squ^{121}=\Squ^{31}$'' and
``$\Squ^{1421}+\Squ^{17}=\Squ^{521}$''.
\subsubsection{\texorpdfstring{The cohomology of $X_2$}{The cohomology of X2}}
\[\xymatrix@R=5mm{
%K(\pi,n+1)\ar[r]&{*}\ar[r]&K(\pi,n)
F_2\ar[r]\ar@{=}[d]&
X_2\ar[d]\ar[r]&
X_{1}\ar[d]^{\alpha}\\
K(\Z_2,n+2)\ar[r]&\star\ar[r]&K(\Z_2,n+3)
}\]
This calculation is more difficult. As we calculate the transgression, we often
will not need to determine certain coefficients. These undetermined coefficients
are marked with dotted lines. We list the considerations when calculating the
transgression at each dimension:
\begin{itemise}
\item[2.]By construction of the fibration, the fundamental class
$\imath_{n+2}$ transgresses to $\alpha$.
\item[3.] The following method for calculating 
$\tau(1)=\tau(\Squ^1(\imath_{n+2}))$
is used throughout this list. In general, for $I$ an admissible sequence:
\[i^*\tau(\Squ^I(\imath_{n+2}))=i^*\Squ^I\alpha=\Squ^I\Squ^2i_{n+1}.\]
Thus, $\tau(I)$ maps to $\Squ^{I2}\imath_{n+1}$ under $i^*$. Now Adem relations
can be used to simplify $\Squ^{I2}$, and we can thus determing $\tau(I)$ up to
an element in the image of $p^*$. For example, as $\Squ^{12}=\Squ^3$, we have
$\tau(1)\equiv\beta\pmod{\Squ^4\imath_n}$.
\item[4.] Using (3), as $\Squ^{22}=\Squ^{31}$, we have $i^*\tau(2)=31$,
which determines $\tau(2)=\gamma$.
\item[5.] Now $\tau(\Squ^3)=\tau(\Squ^{12})=\Squ^1\beta=0$, as 
$\beta=d_2(\Squ^4\imath_{n+2})$. $\tau(21)$ is straightforward.
\item[6.] Straightforward.
\item[7.] We need to see that $\tau(5)=\tau(41)$. This follows since
$\Squ^5+\Squ^{41}=\Squ^2\Squ^3$, and $\tau(3)=0$.
\item[8+.] The rest is straightforward.
\end{itemise}
It will be useful to know that $d_3(\Squ^4\imath_n)=A$ and that
$d_3(\Squ^8\imath_n)=D$. For these, apply Bockstein:
\[\xymatrix{
\Squ^4\imath_n\ar@{|->}[d]_{d_2}\ar@{|->}@/^.81pc/[rr]^{p^*}&
2\ar@{|->}[ld]^\tau\ar@{|->}[d]^{d_1}&
\Squ^4\imath_n\ar@{|->}[d]_{d_{3}}\\
\gamma&
3&d_3(\Squ^4(\imath_n))\ar@{|->}[l]_{j^*}
}
\raisebox{-.8cm}{\qquad and\qquad\ }
\xymatrix{
\Squ^8\imath_n\ar@{|->}[d]_{d_2}\ar@{|->}@/^.81pc/[rr]^{p^*}&
42\ar@{|->}[ld]^\tau\ar@{|->}[d]^{d_1}&
\Squ^8\imath_n\ar@{|->}[d]_{d_{3}}\\
\kappa&
52&d_3(\Squ^8(\imath_n))\ar@{|->}[l]_{j^*}
}
\]
We used ``$\Squ^{12}=\Squ^{3}$'' and
``$\Squ^{142}=\Squ^{52}$''.
\subsubsection{\texorpdfstring{The cohomology of $X_3$}{The cohomology of X3}}
\[\xymatrix@R=5mm{
%K(\pi,n+1)\ar[r]&{*}\ar[r]&K(\pi,n)
F_3\ar[r]\ar@{=}[d]&
X_3\ar[d]\ar[r]&
X_{2}\ar[d]^{\text{``$\Squ^4\imath_n$''}}\\
K(\Z_8,n+3)\ar[r]&\star\ar[r]&K(\Z_8,n+4)
}\]
Here $``\Squ^4\imath_n$'' refers to a class in $H^{n+4}(X_2;\Z_8)$ which reduces
to $\Squ^4\imath_n$ under the reduction map to $H^{n+4}(X_2;\Z_2)$. We should
really see that such a class exists. To do so, it's enough to see that the real
$\Squ^4\imath_n\in H^n(X_2;\Z_2)$ maps to zero under the coboundary
$H^{n+3}(X_2;\Z_2)\to H^{n+4}(X_2;\Z_4)$ coming from the short exact sequence
$0\to \Z_4\to\Z_8\to\Z_2\to0$. Now this coboundary is defined by representing
$\Squ^4\imath_n$ as a $\Z_8$-cochain, taking its coboundary, and observing that
this is the image of a $\Z_4$ cocycle. However, we have seen that the \THIRD
Bockstein operation is defined on $\Squ^4\imath_n$. Thus, if we represent it as
an integral cochain, its coboundary is divisible by 8. Thus $\Squ^4\imath_n$
indeed maps to zero under the aforementioned coboundary.

Now noting that the fundamental class goes to the fundamental class under the
reduction, we apply the principle of the transgression of the fundamental class
to the SSS for $\Z_8$ coefficients, and use naturality to see that
$\imath_{n+3}$ transgresses as claimed. That $d_3(\imath_{n+3})$ transgresses to
$A$ follows from:
\begin{lem*}
Suppose that $d_r$ is defined on both $y$ and $\tau(y)$ (where $\tau$ is that
found in the Serre exact sequence, with everything highly connected). Then
$\tau(d_ry)=d_r\tau(y)$.
\end{lem*}
\begin{proof}
we have 
$\xymatrix{y\ar@{|->}[r]&x&\tau'(y)\ar@{|->}[r]^\simeq\ar@{|->}[l]&\tau(y)}$
in the standard diagram for the transgression. Now by naturality of $d_r$, the
result follows.

\end{proof}


Somehow this is set up so that $\imath_{n+3}$ and $d_3$ transgress as claimed.
Investigate this tomorrow. That being said, we can continue our analysis:
\begin{itemise}
\item[5.]$\tau(2)=\Squ^2\Squ^4\imath_n=(\Squ^{51}+\Squ^6)\imath_n=\Squ^6\imath_n$,
as $H^{n+1}(X_2;\Z_2)$ is zero!
\item[6.] %For $3$, note that $i^*(3)=\Squ^3i^*(\Squ^4\imath_n)=0$. The coeffic
Calculate directly that $\tau(3)=\Squ^{34}\imath_n=\Squ^7\imath_n$.
$2d_3$ is also easy.
\item[7,8.] $3d_3$ and $4d_3$ are easy, looking at $i^*$. $\tau(4)=0$ since
 $\Squ^{44}=\Squ^{62}+\Squ^{71}$, which both vanish 
$\imath_n$. $\tau(5)=0$ as $\Squ^{54}=\Squ^{72}$.
\item[...] $54=72$ and $64=73$. $\tau(5d_3)=0$ is the only hard one left:
$\Squ^5d_3\imath_{n+3}=\Squ^1\Squ^4d_3\imath_{n+3}$.
So $\tau(5d_3)=d_1\tau(4d_3)=d_1D=d_1d_3\Squ^8\imath_n=0$. 
(Note that $\Squ^{4 2 4} = \Squ^{7 2 1} + \Squ^{9 1} + \Squ^{8 2} + \Squ^{10}$).
\end{itemise}
\subsubsection{\texorpdfstring{The cohomology of $X_4$}{The cohomology of X4}}
\[\xymatrix@R=5mm{
%K(\pi,n+1)\ar[r]&{*}\ar[r]&K(\pi,n)
F_4\ar[r]\ar@{=}[d]&
X_4\ar[d]\ar[r]&
X_{3}\ar[d]^{P}\\
K(\Z_2,n+6)\ar[r]&\star\ar[r]&K(\Z_2,n+7)
}\]
The calculations are all easy.
\subsubsection{\texorpdfstring{The cohomology of $X_5$}{The cohomology of X5}}
Has a tricky choice of cohomology class for killing.

} %%%% end comment %%%%

\section{\texorpdfstring{$n$}{n}-type and Postnikov systems}
\setcounter{subsection}{1}
\subsection{\texorpdfstring{$n$}{n}-type}
Maps $X\to Y$ are \emph{$n$-homotopic} if for every map from an $n$-dimensional
cell complex $K$ to $X$, the composites $K\to Y$ are homotopic. By cellular
approximation:
\begin{prop*}
If $f,g:K\to X$ are maps from a cell complex, $f \simeq_n g$ iff 
$f|_{K^n}\simeq g|_{K^n}$.
\end{prop*}
Two cell complexes have the same \emph{$n$-type} if their $n$-skeleta have the
same $(n-1)$-homotopy type. By convention (not theorem), we say that
$\infty$-type is simply homotopy type. $n$-type is then a homotopy invariant, in
light of the following, which also uses cellular approximation:
\begin{prop*}
If two cell complexes have the same $n$-type (for any $n\leq\infty$), they have
the same $m$-type for any $m<n$.
\end{prop*}
\begin{lem*}
If $X$ and $Y$ are $n$-homotopy equivalent, and $K$ is an $n$-dimensional
complex, $[K,X]=[K,Y]$.
\end{lem*}
\begin{thm*}
If complexes $L_1$ and $L_2$ have the same $n$-type, and $K$ is an
$n-1$-dimensional complex, $[K,L_1]=[K,L_2]$.
\end{thm*}
\begin{cor*}
If $K$ and $L$ have the same $n$-type, $\pi_i(K)=\pi_i(L)$ for $i<n$.
\end{cor*}
\begin{thm*}[Whitehead]
Given complexes $K$ and $L$ and a map $f:K^n\to L^n$ inducing isomorphisms
$\pi_i(K^n)\to\pi_i(L^n)$ for $i<n$. Then $K$ and $L$ have the same $n$-type.
\end{thm*}

\subsection{Postnikov systems}
If $X$ is an $(n-1)$-connected complex, a Postnikov system for $X$ is a diagram
\[\xymatrix@C=1.2cm@R=.3cm{
&\vdots\ar[d]\\
X\ar[r]^{\rho_{m+1}}\ar[rd]_{\rho_m}&X_{m+1}\ar[d]\\
&X_m\ar[r]^{k_m(X)\ \ \ \ \ \ \ \ \ }\ar[d]&K(\pi_{m+1}(X),m+2)\\
&\vdots\ar[d]\\
&\makebox[0cm][r]{$K(\pi_n(X),n)=\,$}
X_n\ar[r]^{k_n(X)\ \ \ \ \ \ \ \ \ }\ar[d]&K(\pi_{n+1}(X),n+2)\\
&{*}\ar[r]^{k_{n-1}(X)=*\ \ \ \ \ \ \ \ \ \ \ }&K(\pi_{n}(X),n+1)
}\]
where
\begin{itemise}
\item $X_m$ as the same $(m+1)$-type as $X$, and $\rho_m$ induces the
isomorphisms on $\pi_i$ for $i\leq m$.
\item All higher homotopy groups of $\pi_iX_m$ are trivial ($i>m$).
\item $X_{m+1}$ is the induced fiber space over $X_m$ induced by $k_m(X)$.
\item The triangles are homotopy commutative.
\end{itemise}
\subsection{Existence of Postnikov systems \texorpdfstring{$\bigotimes$}{}}
They exist. Read later.
\subsection{Naturality and Uniqueness \texorpdfstring{$\bigotimes$}{}}
They are and they are. Read later.

\section{The fiber mapping sequence}
Given a Hurewicz fibration $F\rightarrow E\downarrow B$, we can continue the
fiber sequence ad infinitum. Applying $[K,\DASH]$ to the sequence, we get a LES,
ending with
\[\rightarrow[K,\Omega B]\rightarrow [K,F]\rightarrow [K,E]\rightarrow [K,B].\]
The last three terms are not obviously groups. However,
\begin{prop*}\label{StableRange}
If $K$ is a complex of dimension $2n-2$, and $X$ is $(n-1)$-connected, then
$[K,X]=[\Sigma K,\Sigma X]$, so that $[K,X]$ has a natural group structure.
\end{prop*}
\begin{proof}
$X\to\Omega\Sigma X$ is a map inducing isomorphisms $\pi_i$ for $i\leq2n-2$ by
the suspension theorem. Now we approximate this map by a cellular map, and
restrict it to a map on $2n-1$-skeleta. By cellular approximation, this does not
change the isomorphisms in the lower dimensions. Whitehead's theorem then shows
that these spaces have the same $(2n-1)$-type, implying the result, via
$[K,X]=[K,\Omega\Sigma X]=[\Sigma K,\Sigma X]$.
\end{proof}
Any $\theta\in H^{n+k}(\pi,n;G)$ may be viewed as a map $K(\pi,n)\to K(G,n+k)$.
Then applying $\Omega$, we get $\Omega\theta\in H^{n+k-1}(\pi,n-1;G)$. This
is called the \emph{cohomology suspension}, written $\theta\mapsto {^1\theta}$.
\begin{lem*}[Not in text]\label{cohSusp}
In the Serre long exact sequence for $K(\pi,n-1)\to*\to K(\pi,n)$ with
coefficients in $G$, the transgression is inverse to the cohomology suspension.
In particular:
\[^1(\ \ ):H^{n+k}(\pi,n;G)\to H^{n+k-1}(\pi,n-1;G)\]
is an isomorphism for $k\leq n-2$.
\end{lem*}
\begin{proof}
We use an alternative (but equivalent) definition of the transgression:
\[\xymatrix{
H^{*-1}(F;G)\ar[r]&
H^{*-1}(\Omega B;G)\ar@{=}[r]&
H^{*}(\Sigma\Omega B;G)&
\ar[l] H^{*}(B;G)
}\]
Suppose that $\theta:K(\pi,n)\to K(G,*)$ is any map. Its cohomology suspension
is $\Omega\theta:\Omega K(\pi,n)\to \Omega K(G,*)$, in $H^{*-1}(F,G)$. This is
identified exactly with the same map in $H^{*-1}(\Omega B;G)$. Under the
equality, this maps to its adjoint $\Sigma\Omega K(\pi,n)\to K(G,*)$.

It remains to see that $\theta\in H^*(B;G)$ maps to this adjoint under
$\epsilon^*$, where $\epsilon:\Sigma\Omega B\to B$ is the counit. However, this
is clear from basic principles of adjunctions. In particular, passing from
$\theta$ to $\Omega\theta$ is the reverse of performing the transgression, which
is essentially just taking an adjoint.
\end{proof}
The following commutative diagrams justify calling the $\Squ^i$ `stable':
\[\xymatrix{
K(\Z,n-1)\ar[r]^{\Squ^i\ \ \ }\ar@{=}[d]& K(\Z_2,n+i-1)\ar@{=}[d]\\
\Omega K(\Z,n)\ar[r]^{^1\Squ^i\ \ \ }& \Omega K(\Z_2,n+i)
}
\raisebox{-.6cm}{\qquad\text{and}\qquad}
\xymatrix{
K(\Z_2,n-1)\ar[r]^{\Squ^i\ \ \ }\ar@{=}[d]& K(\Z_2,n+i-1)\ar@{=}[d]\\
\Omega K(\Z_2,n)\ar[r]^{^1\Squ^i\ \ \ }& \Omega K(\Z_2,n+i)
}
\]
\subsection{Mappings of low dimensional complexes into a sphere}
It is shown that if $p$ is an odd prime, the least ineger $n$ such that
$\pi_n(S^3)$ has a nonzero element of order $p$ is $n=2p$. This shows that for
$k\leq2$, $\pi_{n+k}(S^n)$ has non elements of odd order.

Consider mapping a complex $K$ into the sphere $S^n$. If $\dim K\leq n$, then as
$S^n$ and $K(\Z,n)$ have the same $(n+1)$-type, $[K,S^n]=[K,K(\Z,n)]=H^n(K)$.
This classical result, the Hopf classification theorem, is dual to the Hurewicz
theorem.

Now suppose that $\dim K=n+1$. Then $[K,S^n]=[K,X_{n+1}]$, where $X_{n+1}$ is
the next space up in a Postnikov tower, and $X_n=K(\Z,n)$. We have a fiber
sequence, where $F_1=K(\Z_2,n+1)$:
\[\xymatrix{
\Omega X_n\ar[r]^{\Omega\Squ^2}&F_1\ar[r]^i&X_{n+1}\ar[d]^p\\
&&X_n\ar[r]^{\Squ^2}&K(\Z_2,n+2)
}\]
The gives a long exact sequence:
\[\xymatrix{
[K,\Omega B]\ar[r]^f&
[K,F_1]\ar[r]^g&
[K,X_{n+1}]\ar[r]^h&
[K,B]\ar[r]^{k\qquad}&
[K,K(\Z_2,n+2)]
}\]
This determines a short exact sequence: 
$0\to\coker f\to [K,X_{n+1}]\to\ker k\to0$, and here
\[f=\Squ^2:H^{n-1}(K;\Z)\to H^{n+1}(K;\Z_2)\text{ and }
k=\Squ^2:H^{n}(K;\Z)\to H^{n+2}(K;\Z_2).\]
The text does a further example when $\dim K=n+2$, but really, it's all headed
towards:

\subsection{The spectral sequence for \texorpdfstring{$[K,X]$}{[K,X]}}
Suppose that $X$ is $(n-1)$-connected. Then from a Postnikov tower we get:
\[\xymatrix{
{*}\ar@{~>}[rd]&
X_n\ar[l]\ar@{~>}[rd]&
X_{n+1}\ar[l]\ar@{~>}[rd]&
X_{n+2}\ar[l]\ar@{~>}[rd]&
X_{n+3}\ar[l]\ar@{~>}[rd]&
X_{n+4}\ar[l]
&X\ar@{.>}[l]\\
&F_0\ar[u]&F_1\ar[u]&F_2\ar[u]&F_3\ar[u]&F_4\ar[u]
}\]
Here, $F_m:=K(\pi_{m+n}(X),m+n)$, and wavy arrows refer to those whose source is
actually the loop space of what is written, more precisely
$\Omega k_{n+i}:\Omega X_{n+i}\to F_{i+1}$.
Applying $[K,\DASH]$ gives an exact couple, which we index as:
\[E_2^{-p,q}=[K,\Omega^p F_{q-p-n}]=H^{q-p}(\Sigma^pK;\pi_{q-p}(X)).\]
Moreover, the differential $d_2:E_2^{-p,q}\to E_2^{-p+1,q}$ is induced by the
composite:
\[\xymatrix{
\Omega^pF_{q-p-n}\ar[r]&\Omega^p X_{q-p}\ar@{~>}[r]&\Omega^{p-1}F_{q-(p-1)-n}.
}\]
In particular, the map $d_2$ corresponds to a primary cohomology operation,
determined by the $k$-invariants of $X$.
Moreover, the $d_r:E_r^{-p,q}\to E_r^{-p+1,q+r-2}$ are induced by maps
\[\xymatrix{
\Omega^pF_{q-p-n}\ar[r]&\Omega^p X_{q-p}&
\Omega^p X_{q-p+r-2}\ar[l]\ar@{~>}[r]
&\Omega^{p-1}F_{(q+r-2)-(p-1)-n}.
}\]
These will correspond to $(r-1)\first$ order cohomology operations.

Now if $\dim k=d<\infty$, $E_2^{-p,q}=0$ for $q>d$, and we can replace $X$ by
$X_d$, so that $F_i=*$ for $i>d-n$. In particular, each column $-p=r$ is finite,
and the elements $E_\infty^{r,q}$ are the subquotients in a filtration for
$[S^rK,X]$.
\subsection{The skeleton mapping and the inclusion mapping sequence}
There's a dual derivation of this using the CW structure on $K$, giving the same
spectral sequence.
\section{Properties of the stable range}
Recall that if $K$ is a complex of dimension at most $2n-2$, and $X$ is an
$(n-1)$-connected space, $[K,X]\to[\Sigma K,\Sigma X]$ is an isomorphism. See
page \pageref{StableRange}.
\subsection{Natural group structures}
$[SK,X]=[K,\Omega X]$ as groups under the natural group structures. These
structures are natural in $K$ and $X$ respectively, not in $SK$ or in $\Omega
X$.
\begin{prop*}
If $[K,X]$ is stable (as above), it has a natural abelian group structure. It's
natural with respect to maps $L\to K$ and $X\to Y$ where $L$ and $Y$ have the
same connectivity and dimension properties as $K$ and $X$.
\end{prop*}
\subsection{Examples: cohomology and homotopy groups}
It happens that $H^n(X;\pi)=[X,K(\pi,n)]=[X,\Omega K(\pi,n+1)]$ is a series of
group isomorphisms. Now if $\theta:K(\pi,n)\to K(G,q)$ is a map, the
corresponding operation may or may not be a homomorphism. We call $\theta$
additive if it is. Then we have seen:
\begin{prop*}
If $\theta$ is in the image of the cohomology suspension, it is additive.
\end{prop*}
For another example, let $G_k$ be the $k\fourth$ stable homotopy group of
spheres. These form a graded abelian group $G$. Now there is a composition
operation, which would be linear in only one variable outside the stable range.
However:
\begin{thm*}
Composition makes $G$ a graded commutative ring with unit.
\end{thm*}
\subsection{Consequences of Serre's exact sequence}
For any space $X$, applying the loops functor to maps classifying $H^{k+1}(X)$
gives a map $H^{k+1}(X)\to H^{k}(\Omega X)$. We already showed (page
\pageref{cohSusp}) that this is inverse to the transgression in Serre's exact
sequence, when this is defined. Thus
\begin{prop*}
If $X$ is $m$-connected, the cohomology suspension 
$H^{k+1}(X)\to H^{k}(\Omega X)$ is an isomorphism if $k\leq 2m-1$.
\end{prop*}
\begin{cor*}
If $\theta\in H^k(\pi,m;G)$ and $k\leq2m-1$, then $\theta$ is additive, as it is
in the image of the cohomology suspension, which is an isomorphism.
\end{cor*}
If $u\in H^k(Y;\pi)$, we define $E(u)$ by inducing a fiber space over $Y$ via
$u$:
\[\xymatrix{
K(\pi,k-1)\ar[r]\ar@{=}[d]&E(u)\ar[r]\ar[d]&Y\ar[d]^u\\
K(\pi,k-1)\ar[r]&{*}\ar[r]&K(\pi,k)
}\]
\begin{prop*}
Suppose $X$ is $m$-connected and $n\leq 2m-1$. Let $u\in H^n(\Omega X;\pi)$.
Then $E(u)$ has the homotopy type of a loop space.
\end{prop*}
\begin{proof}
By the proposition, $u={^1v}$ for some $v\in H^{n+1}(X;\pi)$. Then we have:
\[\xymatrix{
K(\pi,k-1)\ar[r]\ar@{=}[d]&E(u)\ar[r]\ar[d]&\Omega X\ar[d]^u\ar[r]&
K(\pi,k)\ar[r]\ar@{=}[d]&E(v)\ar[r]\ar[d]&X\ar[d]^v\\
K(\pi,k-1)\ar[r]&{*}\ar[r]&K(\pi,k)\ar[r]&
K(\pi,k)\ar[r]&{*}\ar[r]&K(\pi,k+1)
}\]
Here we have continued the fiber sequences and morphisms thereof, starting at
the right.
\end{proof}
\begin{cor*}
Suppose $X$ is $(n-1)$-connected, and $\pi_i(X)=0$ for $i\geq 2n-1$. Then $X$ is
of the homotopy type of a loop space.
\end{cor*}
\begin{proof}
In a Postnikov tower for $X$, we have $X=X_{2n-2}$ (by Whitehead). So it is
enough to prove by induction on $j$ that $X_{j}$ is a loopspace. This is clear
for $j=n$. Suppose that $X_{j}=\Omega Y$ for $j\leq 2n-3$. Then $X_{j+1}$ is
$E(u)$ for $u\in H^{j+2}(\Omega Y;\pi_{j+1}(X))$. Here the previous proposition
applies, as $Y$ is $n$-connected.
\end{proof}
Finally it is noted that in the stable range, cofibre and fibre agree up to a
dimension shift, in their low homotopy groups. Passing to spectra, this shows
that cofibre and fibre sequences coincide.

\section{Higher cohomology operations}
\subsection{Functional cohomology operations}
Suppose that $f:Y\to X$ is a map (which may be viewed as a cofibration), and 
$\theta:H^n(\DASH;G)\to H^q(\DASH;\pi)$ is a cohomology operation in the stable
range ($q\leq2n-2$), so that it is additive.
Then we have a diagram:
\[\xymatrix{
&H^{n-1}(Y;G)\ar[r]^\delta\ar@{~>}[d]^{^1\theta}&
H^n(Cf;G)\ar[r]^{j^*}\ar[d]^\theta&
H^n(X;G)\ar@{.>}[r]^{f^*}\ar@{.>}[d]^\theta&
H^n(Y;G)\\
H^{q-1}(X;\pi)\ar@{~>}[r]^{f^*}&
H^{q-1}(Y;\pi)\ar[r]^\delta&
H^q(Cf;\pi)\ar[r]^{j^*}&
H^q(X;\pi)
}\]
Now by a simple diagram chase, given $u\in H^n(X;G)$ in the kernel of the dotted
arrows ($f^*u=0,\,\theta(u)=0$), we can produce an element of $H^{q-1}(Y;G)$ up
to indeterminacy the image of the wavy arrows. Call it $\theta_f(u)$, defined up
to indeterminacy $f^*H^{q-1}(X;\pi)+{^1\theta}H^{n-1}(Y;G)$.

This association is natural in $f$, in the following sense. If the following
square commutes,
\[\xymatrix{
Y\ar[r]_f& X\\
Y'\ar[r]^{f'}\ar[u]_\chi&X'\ar[u]^\xi
}\]
and $\theta_f$ is defined on $u\in H^n(X;G)$, then $\theta_{f'}$ is defined on
$\xi^*u\in H^n(X';G)$, and moreover, $\chi^*\theta_f(u)=\theta_{f'}(\xi^*u)$.
More precisely, the left hand side of this equation is contained in the right
hand side, as cosets in $H^{q-1}(Y';\pi)$.
\begin{prop*}
If there is a cohomology class $u$ such that $\theta_f(u)\neq0$, then $f$ is
essential.
\end{prop*}
For example, let $f=\Sigma^{n-2}\eta:S^{n+1}\to S^n$. We wish to see that this
is not null. Let $\theta=\Squ^2$. Then the diagram shows not only that $f$ is
essential, but that there is no indeterminacy.
\[\xymatrix{
&\cancel{H^{n-1}(S^{n+1};G)}\ar[r]^\delta\ar@{~>}[d]^{\Squ^2}&
H^n(Cf;\Z_2)\ar[r]^{j^*}\ar[d]^{\Squ^2}&
H^n(S^n;\Z_2)\ar@{.>}[r]^{f^*}\ar@{.>}[d]^{\Squ^2}&
\cancel{H^n(S^{n+1};\Z_2)}\\
\cancel{H^{n+1}(S^n;\Z_2)}\ar@{~>}[r]^{f^*}&
H^{n+1}(S^{n+1};\Z_2)\ar[r]^\delta&
H^{n+2}(Cf;\Z_2)\ar[r]^{j^*}&
\cancel{H^{n+2}(S^n;\Z_2)}
}\]
\subsection{Another formulation of \texorpdfstring{$\theta_f$}{theta f}}
$\theta_f$ is defined on $u$ iff both length 2 composites are null in the
following diagram:
\[\xymatrix{
Y\ar[r]^f&X\ar[r]^{u\ \ \ \ \ }&K(G,n)\ar[r]^\theta&K(\pi,q)
}\]
Now use $\theta$ to define a fibre space $E$ over $K(G,n)$ by pulling back the
path-loop fibration over $K(\pi,q)$. That is, let $E$ be the homotopy fibre of
$\theta$. Then we have a diagram:
\[\xymatrix{
K(\pi,q-1)\ar[rr]^i&&E\ar[d]^p\\
Y\ar[r]^f\ar@{.>}[u]^{\theta_f(u)}&
X\ar[r]^{u\ \ \ }\ar@{-->}[ru]^{\widetilde u}&
K(G,n)\ar[r]^\theta&K(\pi,q)
}\]
We obtain the dashed arrow $\widetilde u$ since the composite $\theta u$ is
null. We obtain the dotted arrow, which we define to be $\theta_f(u)$, as
$\widetilde u f\simeq puf$ is null. This has the same indeterminacy as the old
definition. In particular, the difference between any two liftings
$\widetilde u$ arises from maps $X\to K(\pi,q-1)$. Moreover, the difference
between further liftings $\theta_f(u)$ arises from maps 
$Y\to K(G,n-1)$, via $^1\theta$. This all makes sense as the homsets are
additive, as we are in the stable range (?).
\subsection{Secondary cohomology operations}
Suppose that $\theta:K(G,n)\to K(\pi,q)$ is a cohomology operation, and that
$E=F(\theta)$ is the fibre space over $K(G,n)$ pulled back from the standard
contractible space over $K(\pi,q)$. Let $\phi:E\to K(H,m)$ be some cohomology
class in $H^m(E;H)$. This is enough data to specify a two-stage Postnikov
system:
\[\xymatrix{
K(\pi,q-1)\ar[r]^{\ \ \ \ \ i}&E\ar[r]^{\phi\ \ \ }\ar[d]^p&K(H,m)\\
X\ar@{-->}[ru]^{\widetilde u}\ar[r]^{u\ \ \ }&K(G,n)\ar[r]^\theta&K(\pi,q)
}\]
Now if $u\in H^n(X;G)$ is in the kernel of $\theta$, we can lift it to
$\widetilde u:X\to E$. This lifting is determined up to a map $X\to K(\pi,q-1)$
in some sense. If the operation $\phi i$ is stable, then we have constructed a
map $\Phi$  from $\ker\theta\subseteq H^n(X;G)$ to $H^m(X;H)$, defined up to
indeterminacy the subgroup $\im(\phi i)$.

To have everything additive, we should assume:
\begin{itemise}
\item $q\leq 2n-1$ so that $\theta$ is additive, and $\ker\theta$ is a subgroup.
\item $m\leq 2q-3$ so that $\phi i$ is additive, and the indeterminacy is a
subgroup.
\item $m\leq q+n-2$ so that $H^m(K(G,n);H)\to H^m(E;H)\to H^m(K(\pi,q-1);H)$ is
part of a Serre long exact sequence (for the following reason).
\end{itemise}
Suppose now that $\phi i$ is null, so that there can never be any indeterminacy.
Then by the Serre exact sequence just mentioned, $\phi=p^*\theta'$ for some
$\theta':K(G,n)\to K(H,m)$. Adding this map to the diagram above, we realise
that the secondary cohomology operation defined is precisely the primary
operation $\theta'$. Thus we'll always assume that $\phi i$ is not null.

The naturality formula for secondary operations, for $f:Y\to X$, is:
$f^*(\Phi(u))\subseteq\Phi(f^*u)$.
\subsection{Secondary operations and relations}
Suppose that $m\leq 2q-3$. Then the composite $\phi i$ is in the image of $\psi$
under the cohomology suspension, and we can extend the diagram to:
\[\xymatrix{
K(G,n-1)\ar[r]^{^1\theta}&
K(\pi,q-1)\ar[r]_{\ \ \ \ \ i}\ar@/^.8pc/[rr]^{^1\psi}&
E\ar[r]_{\phi\ \ \ }\ar[d]^p&K(H,m)\\
&&%X\ar@{-->}[ru]^{\widetilde u}\ar[r]^{u\ \ \ }&
K(G,n)\ar[r]^\theta&K(\pi,q)\ar[r]^{\psi\ \ \ \ }&
K(H,m+1)
}\]
Now if $m\leq 2n-3$, as ${^1\psi}{^1\theta}$ is null, $\psi\theta$ is also null,
which constitutes a realtion between primary cohomology operations.

On the other hand, suppose that we instead start only with operations $\theta$
and $\psi$ such that $\psi\theta=0$. In this case, we can recreate every map but
$\phi$ in the diagram. Yet then, as ${^1\psi}{^1\theta}$ is null, from the Serre
long exact sequence at $H^m(K(\pi,q-1);H)$ we see that ${^1\psi}$ factors into a
composite $\phi i$, as hoped. This $\phi$ is determined up to primary cohomology
operations, which are those that factor through $p^*$ (an insight from the LES).
This argument works when $m\leq q+n-3$, as then the long exact sequence has
enough terms.

For a fantastic example of the obtaining an operation from a relation, we'll get
an operation $\Phi$ (as shown) from the relation $\Squ^3\Squ^1+\Squ^2\Squ^2=0$:
\[\label{31is22}\xymatrix@C=2cm{
\ker(\alpha)\ar@{^{(}->}[r]\ar[d]^\Phi&
H^n(X;\Z_2)\ar[r]_{\alpha\qquad\qquad}^{(\Squ^1,\Squ2)\qquad\qquad}&
H^{n+1}(X;\Z_2)\oplus H^{n+2}(X;\Z_2)\\
\coker(\beta)&H^{n+3}(X;\Z_2)\ar@{->>}[l]&
\ar[l]_{(\Squ^3,\Squ^2)\qquad\qquad}^{\beta\qquad\qquad}
H^{n}(X;\Z_2)\oplus H^{n+1}(X;\Z_2)
}\]
The relevant diagram for making this definition is as follows. Note that we may
replace $\theta$ with a \emph{family} of primary cohomology operations.
\[\xymatrix@C=2.3cm{
*\txt{$K(\Z_2,n)\times$\\$K(\Z_2,n+1)$}\ar[r]
\ar@/^.8pc/[rr]^{^1\psi=\Squ^3\otimes1+1\otimes\Squ^2}&
E\ar@{.>}[r]_{\phi\ \ \ }\ar[d]^p&K(\Z_2,n+3)\\
&
K(\Z_2,n)\ar[r]^\theta_{(\Squ^1,\Squ^2)}&
*\txt{$K(\Z_2,n+1)\times$\\$K(\Z_2,n+2)$}
\ar[r]^\psi_{\Squ^3\otimes1+1\otimes\Squ^2}&
K(\Z_2,n+4)
}\]
We can fill in the map $\phi$ because of the fact that $\psi\theta=0$. Moreover,
in this case, we have that $p^*:H^{n+3}(K(\Z_2,n);\Z_2)\to H^{n+3}(E;\Z_2)$ is
zero, so that $\phi$ is uniquely determined. To see this, note that:
\[H^{n+3}(K(\Z_2,n);\Z_2)=\Z_2\langle\Squ^2\Squ^1\imath_B,\Squ^3\imath_B\rangle
=\Z_2\langle\tau(\Squ^2\imath_{n}\otimes1),
\tau(1\otimes\Squ^1\imath_{n+1})\rangle
\]

\subsection{The Peterson-Stein formulas}
We would like to relate the two constructions --- secondary operations and
functional operations. In particular, we want formulae which allow us to
calculate secondary operations using functional and primary operations. The
three theorems of this chapter do just that. A diagram that summarises our two
constructions is as follows:
\[\xymatrix{
&K(\pi,q-1)\ar[r]_{\ \ \ \ \ i}\ar@{.>}@/^.8pc/[rr]^{^1\psi}&
E\ar@{~>}[r]_{\phi\ \ \ \ \ \ \ \ \ }\ar[d]&K(H,m)\\
Y\ar@{~>}[r]^f\ar@{-->}[ru]^{\theta_f(u)}&
X\ar@{-->}[ru]^(.4){\widetilde u}
\ar@{-->}[r]_{u\ \ \ }\ar@{-->}[rru]_(.65){\Phi(u)}&
K(G,n)\ar[r]^\theta&K(\pi,q)\ar@{.>}[r]^{\psi\ \ \ \ }&
K(H,m+1)
}\]
\begin{itemise}
\item To discuss either construction, we need the specify $\theta$, which
specifies all the full arrows in the diagram. We also need to give a candidate
element $u:X\to K(G,n)$ such that $\theta u=0$. This is lifted to $\widetilde
u$.
\item To define a secondary operation, we need to give $\phi$, which induces the
dotted arrows in the diagram. Then composing with $\phi$ gives $\Phi(u)$.
\item To define a functional operation, we need to give $f$, and $u$ must
satisfy $u f=0$. Then we pull $\widetilde u$ back to $\theta_f(u)$.
\end{itemise}
The following theorem essentially states that the two paths from $Y$ to $K(H,m)$
in this diagram are the same, which we knew, as the diagram is commutative:
\begin{thm*}[\FIRST Peterson-Stein formula]\label{PSfirst}
$f^*(\Phi(u))={^1\psi}\theta_f(u)$ in $H^m(Y;H)/Q'$. Here the indeterminacy $Q'$
is $f^*(\im({^1\psi}))$, where we write $^1\psi:H^{q-1}(X;\pi)\to H^{m}(X;H)$.
\end{thm*}
Suppose that the composite
$\xymatrix{Y\ar[r]^f&X\ar[r]^{v\ \ \ \ } &K(G,n)\ar[r]^\theta&K(\pi,q)}$
is null, but neither $\theta v$ or $vf$ are necessarily null.
\begin{itemise}
\item We can calculate $\Phi(f^*v)$, a secondary operation, which lies in a
quotient of $H^m(Y;H)$.
\item We can calculate a functional operation, $\psi_f(\theta v)$, which lies in
a different quotient of $H^m(Y;H)$.
\end{itemise}
\begin{thm*}[\SECOND Peterson-Stein formula]
$\Phi(f^*v)=\psi_f(\theta v)\in H^m(Y;H)/Q$, where the indeterminacy 
$Q={^1\psi}(H^{q-1}(Y;\pi))+f^*(H^m(X;H))$.
\end{thm*}
Suppose that:
\begin{itemise}
\item $\xymatrix{F'\ar[r]^j&E'\ar[r]^r&B'}$ is a fiber space;
\item $\theta:H^n(\DASH;G)\to H^q(\DASH;\pi)$ is an operation as before, and the
class $\phi\in H^m(E;H)$ and operation $\psi:H^q(\DASH;\pi)\to H^{m+1}(\DASH;H)$
are as usual in the construction of the secondary operation $\Phi$;
\item there are classes $u\in H^n(B';G)$ and $v\in H^{n-1}(F';G)$ such that
$\theta(u)=\tau(v)$; and
\item the following diagram commutes:
\[\xymatrix{
E'\ar[r]_r\ar@/^1.2pc/[rrr]^0&
B'\ar[r]_{\ \ u\qquad}&
K(G,n)\ar[r]_\theta\ar@/^1.2pc/[rr]^0&
K(\pi,q)\ar[r]_{\psi\ \ \ \ }&
K(H,m+1)
}\]
\end{itemise}
\begin{thm*}[\THIRD formula]In these circumstances,
\[j^*(\psi_r(\theta u))={^1\psi}(v)=j^*(\Phi(r^*u))\text{ \ in \ }
H^m(F';H)/Q\text{ \ where \ }Q=j^*(\im({^1\psi})).\]
That is, $Q$ is the image of 
$\xymatrix{H^{q-1}(E';\pi)\ar[r]^{^1\psi}&H^m(E';H)\ar[r]^{j^*}&H^m(F';H)}.$
\end{thm*}
\section{Compositions in the stable homotopy of spheres}
\subsection{Secondary compositions: (co)-extensions and Toda brackets}
For this discussion, fix a map $\beta:X\to Y$. Let $i:Y\to K$ be the cofibre of
$\beta$, where we write $K=(C^+\!X)\cup Y$ for clarity. Always assume that we're
in the stable range, so that everything has a natural abelian group structure.
\subsubsection*{Extensions}
Suppose we have a map $\alpha:Y\to Z$ such that $\alpha\beta$ is null. Then we
can form extensions $\overline\alpha$ (the dashed map) of $\alpha$. The
extension has indeterminacy given by composition with any map $\Sigma X\to Z$
(drawn dotted).
\[\xymatrix{
X\ar[r]^\beta&Y\ar[r]^\alpha\ar[d]_i&Z\\
&\makebox[0cm][r]{$(C^+\!\ X)\cup Y=$\ }K\ar@{-->}[ru]^{\overline\alpha}\ar[d]\\
&\makebox[0cm][r]{$(C^-\!\ Y)\cup K=$\ }\Sigma X\ar@{.>}[uur]
}\]
\subsubsection*{Co-extensions}
Suppose insteak that $\gamma:W\to X$ is such that $\beta\gamma$ is null. Abusing
notation, there is a map ``$C^+\!\gamma$'' from $C^+\!W\to K$ which just acts as
$\gamma$ does at any given height on the cone. A co-extension of $\gamma$ is a
map $\widetilde\gamma:\Sigma W\to K$ which extends $C^+\!\gamma$ and sends
$C^-\!W$ into $Y$:
\[\xymatrix{
\ar[d]W\ar[r]^\gamma&X\ar[r]^\beta&Y\ar[d]^i\\
C^+\!W\ar[d]\ar[rr]^(.35){C^+\!\gamma}&&K\ar[d]^j\\
\makebox[0cm][r]{$(C^+\!W)\cup (C^-\!W)=$\ }\Sigma W
\ar@{-->}[rru]_(.45){\widetilde\gamma}
\ar@{.>}[rruu]\ar[rr]^(.7){\Sigma\gamma}&&\Sigma X
}\]
As the difference between two nullhomotopies from $W$ can be viewed as a map
from $\Sigma W$, the co-extension is defined up to addition of a composite with
the dotted arrow (in the stable range). Note also that the composite which
becomes the bottom row of the diagram is in fact homotopic to $\Sigma\gamma$.
\subsubsection{Toda brackets}
Suppose that we are in both of these situations at once, and form an extension
and a co-extension. That is, if both composites along the top row are null, we
have a diagram:
\[\xymatrix{
\ar[dd]W\ar[r]^\gamma&X\ar[r]^\beta&Y\ar[d]^i\ar[r]^\alpha&Z\\
%C^+\!W\ar[d]\ar[rr]^(.35){C^+\!\gamma}
&&K\ar[d]^j\ar@{-->}[ur]^{\overline\alpha}\\
\Sigma W\ar@{-->}[rru]_(.45){\widetilde\gamma}
\ar@{.>}[rruu]\ar[rr]^(.7){\Sigma\gamma}&&\Sigma X\ar@{.>}[ruu]
}\]
The composite $\overline\alpha\widetilde\gamma:\Sigma W\to Z$ is the Toda
bracket $\langle\alpha,\beta,\gamma\rangle$. It is defined when $\alpha\beta$
and $\beta\gamma$ are both null, up to an indeterminacy of $\alpha_\#[\Sigma
W,Y]+(\Sigma\gamma)^\#[\Sigma X,Z]$.

Now if $\alpha,\beta,\gamma$ are maps of spheres in the stable range, we would
hope $\langle\alpha,\beta,\gamma\rangle$ would depend only on
$\alpha,\beta,\gamma$ as stable homotopy classes (in $G$). This is true up to
sign:
\begin{prop*}
Let $\alpha\in G_h$, $\beta\in G_k$ and $\gamma\in G_j$ be such that
$\alpha\beta$ and $\beta\gamma$ are both zero. Let $a,b,c$ represent
$\alpha,\beta,\gamma$ as follows:
$\xymatrix{
S^{n+h+k+j}\ar[r]^{c}&S^{n+h+k}\ar[r]^{b}&S^{n+h}\ar[r]^{a}&S^{n}
}$. Then the class of 
\[
(-1)^{n-1}\langle a,b,c\rangle\in
G_{h+k+j+1}/\left(\alpha\circ G_{k+j+1}+G_{h+k+1}\circ \beta\right)
\]
is independent of $n$, and so defines a stable Toda bracket
$\langle\alpha,\beta,\gamma\rangle$.
\end{prop*}
\subsection{The 0-, 1- and 2-stems}
We will analyse the $2$-components of the stable $k$-stems, for $k\leq7$. In the
$0$-stem:
\begin{prop*}[1]
$2^r\imath$ is detected by the Bockstein operator $d_r$. That is, in the cell
complex $K=S^n\cup_{2^r\imath}e^{n+1}$, if $H^{n}(K;\Z_2)=\Z_2\langle x\rangle$
and $H^{n+1}(K;\Z_2)=\Z_2\langle y\rangle$, then $d_r$ is defined on $x$, and
equals $y$.
\end{prop*}
\begin{proof}
To see this, we can actually calculate the Bockstein exact couple. We have
$H^{n+1}(K;\Z)=\Z_{2^r}\langle z\rangle$, and find that $\beta(x)=2^{r-1}z$. The
first nonzero differential is indeed $d_r$, as hoped.
\end{proof}
The stable 1-stem is $\Z_2$ and is generated by $\eta$, which is of course
detected by $\Squ^2$.
\begin{prop*}[2]
$\eta^2\neq0$, and thus generates $G_2\simeq\Z_2$.
\end{prop*}
\begin{proof}[First proof]
All coefficients are $\Z_2$. Suppose that $\eta^2=0$. Then we can form a
co-extension diagram. In the right hand diagram, $p$ is the pinch map on to the
top cell of $K$ (i.e.\ the cofiber of $S^n\to K$).
\[\xymatrix{
S^{n+2}\ar[d]\ar[r]^\eta&S^{n+1}\ar[r]^\eta&S^n\ar[d]\\
\Sigma S^{n+2}\ar[rr]^{\widetilde \eta}&&K
}
\raisebox{-.6cm}{\text{\qquad and \qquad}}
\xymatrix{
S^{n+3}\ar[r]^\eta&S^{n+2}\\
S^{n+3}\ar[r]^{\widetilde\eta}\ar@{=}[u]&K\ar[u]^p
}\]
Now we consider the functional operation 
$\Squ^2_{\widetilde\eta}:H^{n+2}(S^{n+2})\to H^{n+3}(S^{n+3})$.
For the class $e^{n+2}\in H^{n+2}(S^{n+2})$, the naturality formula states
that (as cosets in $H^{n+3}(S^{n+3})$):
\[\Squ^2_\eta(e^{n+2})\supseteq\Squ^2_{\widetilde\eta}(p^*e^{n+2}).\]
However, we calculated that the left hand side is nonzero with zero 
indeterminacy. Thus we have $\Squ^2_{\widetilde\eta}(e^{n+2})=e^{n+3}$, also 
with no indeterminacy. The calculation of $\Squ^2_{\widetilde\eta}$ runs like:
\[\xymatrix{
&\cancel{H^{n+1}(S^{n+3})}\ar[r]^\delta\ar@{~>}[d]^{\Squ^2}&
H^{n+2}(C\widetilde\eta)\ar[r]^{j^*}\ar[d]^{\Squ^2}&
H^{n+2}(K)\ar@{.>}[r]^{f^*}\ar@{.>}[d]^{\Squ^2}&
\cancel{H^{n+2}(S^{n+3})}\\
\cancel{H^{q+3}(K)}\ar@{~>}[r]^{\widetilde\eta^*}&
H^{q+3}(S^{n+3})\ar[r]^\delta&
H^{q+4}(C\widetilde\eta)\ar[r]^{j^*}&
\cancel{H^{q+4}(K)}
}\]
Thus $\Squ^2(e^{n+2})=e^{n+4}$ in $H^*(C\widetilde\eta)$. Moreover,
$\Squ^2(e^{n})=e^{n+2}$ in $H^*(C\widetilde\eta)$, as $\Squ^2$ detects odd Hopf
invariant. So $C\widetilde\eta$ has cells in dimensions $n$, $n+2$, and $n+4$,
and $\Squ^2\Squ^2(e^n)=e^{n+4}$. This contradicts the Adem relation
$\Squ^2\Squ^2+\Squ^3\Squ^1=0$.
\end{proof}
\begin{proof}[Second proof]
We have a morphism of cofiber diagrams coming from the left hand square:
\[\xymatrix{
S^{n+2}\ar[r]^\eta\ar@{=}[d]&\ar[d]_\eta S^{n+1}\ar[r]_\chi&K\ar[d]^f\\
S^{n+2}\ar[r]^{\eta^2}&S^n\ar[r]^\xi&L
}\]
Now to differentiate between the two versions of $\Squ^2$, write
$\underline{\Squ}^2:K(\Z_2,n)\to K(\Z_2,n+2)$. We'll use the notation
$\sigma^n_{L;\Z}$ to denote the generator of $H^n(L;Z)$, and so on. Then by the
naturality formula for functional cohomology operations:
\[\chi^*\underline{\Squ}^2_f(\sigma^n_{L;\Z})
\subseteq\underline{\Squ}^2_\eta(\xi^*\sigma^n_{L;\Z})
=\underline{\Squ}^2_\eta(\sigma^n_{S^n;\Z})
=\bigl\{\sigma^{n+1}_{S^{n+1};\Z_2}\bigr\},
\text{\ \ so\ \ }
\underline{\Squ}^2_f(\sigma^n_{L;\Z})=\bigl\{\sigma^{n+1}_{S^{n+1};\Z_2}\bigr\}.
\]
Now $\Squ^2\underline{\Squ}^2=0$, and we form a secondary operation $\Phi$ from
this relation, via the diagram:
\[\xymatrix{
K(\Z_2,n+1)\ar[r]_{\ \ \ \ \ i}\ar@/^.8pc/[rr]^{^1\psi=\Squ^2}&
X_1\ar[r]_{\phi\,=\,\alpha\ \ \ \ \ \ \ \ }\ar[d]^p&K(\Z_2,n+3)\\
&%X\ar@{-->}[ru]^{\widetilde u}\ar[r]^{u\ \ \ }&
\makebox[0in][r]{$X_0=$\ }K(\Z,n)
\ar[r]^{\theta\ \ \ }_{\underline{\Squ}^2\ \ \ }&
K(\Z_2,n+2)\ar[r]^{\psi}_{\Squ^2}&
K(\Z_2,n+4)
}\]
Note that $\alpha$ (from chapter 12) was defined so that
$i^*\alpha=\Squ^2\imath_{n+1}$, so it makes the diagram commute. Now by the
first Peterson-Stein relation:
\[f^*\Phi(\sigma^n_{L;\Z})=\Squ^2\underline{\Squ}^2_f(\sigma^n_{L;\Z})
=\Squ^2(\sigma^{n+1}_{K;\Z_2})=\sigma^{n+3}_{K;\Z_2}
\text{ \ with indet.\ \ }f^*(\Squ^2(H^{n+1}(L;\Z_2)))=0.\]
Now the indeterminacy of $\Phi(\sigma^n_{L;\Z})$ is $\Squ^2(H^{n+1}(L;\Z_2))$,
which is also zero, so that $\Phi(\sigma^n_{L;\Z})$ is nonzero, and $\eta^2$
cannot be null.
\end{proof}

\begin{prop*}[3]
$\langle 2\imath,\eta,2\imath\rangle=\eta^2\in\pi_{n+2}S^n$.
\end{prop*}
\begin{proof}
The diagram for the calculation of the Toda bracket is everything below which is
not drawn with wavy arrows. Note that there is no indeterminacy, as the \SECOND
stem is 2-torsion, and both dotted arrows are composed with $2\imath$.
\[\xymatrix{
\ar[dd]S^{n+1}\ar[r]^{2\imath}&
S^{n+1}\ar[r]^\eta&
S^n\ar[d]^i\ar[r]^{2\imath}&
S^n\ar@{~>}[r]&
K\\
%C^+\!W\ar[d]\ar[rr]^(.35){C^+\!\gamma}
&&L\ar[d]^j\ar@{-->}[ur]^{\overline{2\imath}}\ar@{~>}[r]&M\ar@{~>}[ur]_g\\
S^{n+2}\ar@{-->}[rru]_(.45){\widetilde{2\imath}}
\ar@{.>}[rruu]\ar[rr]^(.7){2\imath}&&S^{n+2}\ar@{.>}[ruu]
}
\raisebox{-1.3cm}{\text{ \ where: \ }}
\raisebox{-.4cm}{$
\xymatrix@R=0cm{
K=C(\overline{2\imath}\widetilde{2\imath})\text{ has cells }n,n+3\\
L=C(\eta)\text{ has cells }n,n+2\\
M=C(\widetilde{2\imath})\text{ has cells }n,n+2,n+3\\
C(g)\text{ has cells }n,n+3; n+1,n+3,n+4
}$}
\]
As the top attaching map $M$ is multiplication by 2, $M$ has a nonzero
cohomology group in all three dimensions, and
$\Squ^1\sigma^{n+2}_{M,\Z_2}=\sigma^{n+3}_{M,\Z_2}$ As $\Squ^3=\Squ^1\Squ^2$,
this shows that $\Squ^3\sigma^{n}_{M,\Z_2}=\sigma^{n+3}_{M,\Z_2}$.

Next, to calculate $\Squ^1_g(\sigma^n_{K,\Z_2})$, we need to know the action of
the Bockstein on $H^n(C(g))$. Now as the map $g$ acts as $2\imath$ in the
$n$-cells of $M$ and $K$, the attaching map from $S^{n+1}\to S^n$ in $C(g)$ is
multiplication by two. Thus it is detected by $\Squ^1\neq0:H^n(C(g))\to
H^{n+1}(C(g))$. Moreover, the map $g^*:H^n(K)\to H^n(M)$ is zero. This is enough
to show that $\Squ^1_g(\sigma^n_{K,\Z_2})=\sigma^{n+1}_{M,\Z_2}$ with zero
indeterminacy. Whereas, $\Squ^1_g(\sigma^n_{K,\Z_2})$ is zero with zero
indeterminacy, as its target vanishes.

Now if $\Psi$ is the secondary operation corresponding to
$\Squ^2\Squ^2=\Squ^1\Squ^3$, from page \pageref{31is22}, we can apply the first
Peterson-Stein relation (page \pageref{PSfirst} with $f=g$,
$\psi=\Squ^3\otimes1+1\otimes\Squ^2$ and $\theta=(\Squ^1,\Squ^2)$), to obtain,
with zero indeterminacy:
\[g^*(\Psi(\sigma^n_{K,\Z_2}))=
\Squ^3\Squ_g^1\sigma^n_{K,\Z_2}+\Squ^2\Squ_g^2\sigma^n_{K,\Z_2}=
\Squ^3\sigma^{n}_{M,\Z_2}=\sigma^{n+3}_{M,\Z_2}.\]
Of course, if the bracket were null, this could not hold.
\end{proof}
\subsection{The 3-stem}
We know that $G_3=\pi_3^s\simeq\Z_8$...
\begin{prop*}[4]
The Hopf map $\nu:S^{n+3}\to S^n$ generates $G_3$.
\end{prop*}
\begin{proof}
By the suspension theorem, $E:\pi_7(S^4)\to\pi_8(S^5)$ is an epimorphism onto a
stable group. Thus it is enough to show that $E\nu$ is not divisible by $2$. If
$Ev=2\alpha$, then by surjectivity, $E\beta=\alpha$ for some $\beta$, and so
$E(v-2\beta)=0$. However, the Hopf invariant is a homomorphism, whose parity is
detected in this case by $\Squ^4$. Thus $v-2\beta$ has even Hopf invariant,
which contradicts the fact that $v$ has odd Hopf invariant.
\end{proof}
\end{document}
















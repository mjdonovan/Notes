\documentclass[11pt]{article}
%\usepackage{cancel}
\usepackage{fullpage}
\usepackage{amsmath,amsthm,amssymb}
\usepackage{mathrsfs,nicefrac}
\usepackage{amssymb}
\usepackage{epsfig}
\usepackage[all]{xy}
\usepackage{sseq}
\usepackage[pdftex]{hyperref}


\newcommand{\II}{\mathcal{I}}

\DeclareSymbolFont{AMSb}{U}{msb}{m}{n}
\DeclareMathSymbol{\N}{\mathbin}{AMSb}{"4E}
\DeclareMathSymbol{\Z}{\mathbin}{AMSb}{"5A}
\DeclareMathSymbol{\R}{\mathbin}{AMSb}{"52}
\DeclareMathSymbol{\Q}{\mathbin}{AMSb}{"51}
\DeclareMathSymbol{\PP}{\mathbin}{AMSb}{"50}
\DeclareMathSymbol{\I}{\mathbin}{AMSb}{"49}
\DeclareMathSymbol{\C}{\mathbin}{AMSb}{"43}
\DeclareMathSymbol{\A}{\mathbin}{AMSb}{"41}
\DeclareMathSymbol{\F}{\mathbin}{AMSb}{"46}

\newcommand{\ad}{\textup{\textbf{ad}}}
\newcommand{\sll}{\mathfrak{sl}}
\newcommand{\gl}{\mathfrak{gl}}
\newcommand{\GL}{\mbox{GL}}
\newcommand{\SL}{\mbox{SL}}
\newcommand{\tr}{\mbox{tr\ }}
\newcommand{\Mat}{\mbox{Mat}}
\newcommand{\Lie}{\mbox{\textbf{Lie }}}
\newcommand{\Der}{\textup{Der}}
\newcommand{\End}{\mbox{End\ }}
\newcommand{\im}{\mbox{im}}
\newcommand{\Gr}{\textup{Gr}}
\newcommand{\pim}{\mbox{pim}}
\newcommand{\supp}{\mbox{supp\,}}
\newcommand{\Ker}{\mbox{ker\ }}
\renewcommand{\ker}{\textup{ker}\,}
\newcommand{\coker}{\textup{coker}\,}

\newcommand{\PPPP}{\mathcal{P}}

\newcommand{\aaa}{\mathfrak{a}}
\newcommand{\mmm}{\mathfrak{m}}
\newcommand{\qqq}{\mathfrak{q}}
\newcommand{\ppp}{\mathfrak{p}}
\newcommand{\g}{\mathfrak{g}}
\newcommand{\h}{\mathfrak{h}}
\newcommand{\m}{\mathfrak{m}}
\newcommand{\He}{\mathfrak{H}}
\newcommand{\shfF}{\mathscr{F}}
\newcommand{\shfG}{\mathscr{G}}
\newcommand{\shfH}{\mathscr{H}}
\newcommand{\shfL}{\mathscr{L}}
\newcommand{\sHOMFG}{\mathscr{H}\textit{\!\!om}(\shfF,\shfG)}
\newcommand{\sHOM}{\mathscr{H}\textit{\!\!om}}
\newcommand{\Hom}{\textup{Hom}}
\newcommand{\Ext}{\textup{Ext}}
\newcommand{\Map}{\textup{Map}}
\newcommand{\sk}{\vspace*{1em}}
\renewcommand{\phi}{\varphi}
%\newcommand{\ann}{\textup{ann}}
\DeclareMathOperator{\ann}{ann}
\newcommand{\Squ}{\textup{Sq}}
\newcommand{\DASH}{\textup{---}}


\theoremstyle{plain}
\newtheorem{thm}{Theorem}[section]
\newtheorem*{thm*}{Theorem}
\newtheorem{lem}[thm]{Lemma}
\newtheorem*{lem*}{Lemma}
\newtheorem{prop}[thm]{Proposition}
\newtheorem*{prop*}{Proposition}
%\newtheorem*{prop}{Proposition}
\newtheorem{cor}[thm]{Corollary}
\newtheorem{defprop}[thm]{Definition-Proposition}


\theoremstyle{definition}
\newtheorem{defn}{Definition}[section]
\newtheorem{exmp}{Example}[section]
\newtheorem{asspt}{Assumption}[section]
\newtheorem{notation}{Notation}[section]
\newtheorem{exercise}{Exercise}[section]


% 1-inch margins, from fullpage.sty by H.Partl, Version 2, Dec. 15, 1988.
\topmargin 0pt
\advance \topmargin by -\headheight
\advance \topmargin by -\headsep
\textheight 9.1in
\oddsidemargin 0pt
\evensidemargin \oddsidemargin
\marginparwidth 0.5in
\textwidth 6.5in

\parindent 0in
\parskip 1.5ex
%\renewcommand{\baselinestretch}{1.25}

\newcommand{\RAD}[1]{\textup{rad}(#1)}
\newcommand{\SPC}[1]{\textup{sp}(#1)}
\newcommand{\RED}[1]{{#1}_\textup{red}}
\newcommand{\REDpsh}[1]{{#1}_\textup{red}^-}
\newcommand{\OO}{\mathcal{O}}
\newcommand{\spec}{\textup{spec}\,}
\newcommand{\spce}{\textup{sp}}
\newcommand{\proj}{\textup{proj}\,}
\newcommand{\HOMs}{\textup{Hom}_\mathfrak{Sch}}
\newcommand{\HOMr}{\textup{Hom}_\mathfrak{Rings}}

\renewcommand{\to}{\longrightarrow}
\renewcommand{\mapsto}{\longmapsto}
\newcommand{\eps}{\varepsilon}

\newcommand{\catC}{\mathcal{C}}
\newcommand{\catD}{\mathcal{D}}
\newcommand{\catSet}{\textit{Sets}}
\newcommand{\op}{\textup{op}}
\newcommand{\dash}{\textup{---}}
\newcommand{\id}{\textup{id}}
\newcommand{\mapsfrom}{\,\reflectbox{$\mapsto$}\ }
\newcommand{\cone}{\textup{cone}}
\newcommand{\COMMENT}[1]{}

\newcommand{\calL}{\mathcal{L}}
\newcommand{\RP}{{\R\PP}}
\newcommand{\CP}{{\C\PP}}
\newcommand{\Steen}{\mathscr{A}}

\title{Cohomology Operations}
\author{Mosher \& Tangora}
% >>>

%% >>> header

\begin{document}

\maketitle
\tableofcontents

\section{Introduction of Cohomology Operations}
\subsection{Cohomology Operations and $K(\pi,n)$ spaces}
There is a natural equivalence of functors $[\DASH,K(\pi,n)]\simeq H^n(\DASH,\pi)$ given by $[f]\longleftrightarrow f^*(\imath_n)$, so:
\[H^m(K(\pi,n);\pi')\longleftrightarrow [K(\pi,n),K(\pi',m)]\longleftrightarrow \textup{Nat}( H^n(\DASH,\pi), H^{m}(\DASH,\pi'))\]

%\subsection{The Machinery of obstruction theory $\otimes$}
%\subsection{Applications of obstruction theory $\otimes$}


\section{Construction of the Steenrod squares}
\subsection{The complex $K(\Z_2,1)$}
$\RP^\infty$ is a $K(\Z_2,1)$, and $S^\infty\downarrow\RP^\infty$ is a model for $E\Z_2\downarrow B\Z_2$. Moreover:
\[H_n(\RP^\infty)=\begin{cases}0&n\text{ even}\\\Z_2&n\text{ odd}\end{cases}\ \text{ and }H^n(\RP^\infty)=\begin{cases}\Z_2&n\text{ even}\\0&n\text{ odd}\end{cases}\ \text{ and }H^n(\RP^\infty;\Z_2)=\Z_2[\alpha_1].\]
%\subsection{Other Sections}
%The acyclic carrier theorem. Construction of the cup-$i$ products. The squaring operations. Compatibility with coboundary and suspension.

\section{Properties of the squares}
\begin{enumerate}
\item[0.] $\Squ^i$ is a natural homomorphism $H^p(K,L;\Z_2)\to H^{p+i}(K,L;\Z_2)$
\item[1.] $\Squ^i(x)=0$ when $i>|x|$
\item[2.] $\Squ^i(x)=x^2$ when $i=|x|$
\item[3.] $\Squ^0$ is the identity
\item[4.] $\Squ^1$ is the Bockstein (associated with $0\to\Z_2\to\Z_4\to\Z_2\to0$)
\item[5.] $\Squ^i$ is compatible with the coboundary (and thus also with suspension)
\item[6.] $\Squ(xy)=\Squ(x)\Squ(y)$, where $\Squ$ denoted the total Steenrod square
\item[7.] For $a<2b$, $\Squ^a\Squ^b=\sum{b-c-1\choose a-2c}\Squ^{a+b-c}\Squ^c$. Note that in this sum only involves $0\leq c\leq a/2$. In particular, the terms on the right have $a+b-c\geq 2c$.
\end{enumerate}
\setcounter{subsection}{2}
\subsection{Squares in the $n$-fold Cartesian product of $K(\Z_2,1)$}
Let $K_n$ be the product of $n$ copies of $K(\Z_2,1)$. Then 
\[H^*(K_n,\Z_2)=\Z_2[x_1,\ldots,x_n]\supset\Z_2[\sigma_1,\ldots,\sigma_n]=S,\]
where the $\sigma_i$ are the elementary symmetric functions.

%\noindent \textbf{Proposition 3.} In $H^*(K_n;\Z_2)$, $\Squ^i(\sigma_n)=\sigma_n\sigma_i\neq0$.

\begin{prop*}
In $H^*(K_n;\Z_2)$, $\Squ^i(\sigma_n)=\sigma_n\sigma_i\neq0$.
\end{prop*}
A sequence $I=\{i_1,\ldots,i_r\}$ of (strictly positive) integers is \emph{admissible} if $i_j\geq 2i_{j+1}$, that is if it is decreasing faster that $2^{-j}$. The \emph{degree} $d(I)$ is the sum of the terms, and $\Squ^I$ raises dimension by $d(I)$. The \emph{excess}, defined only for admissible sequences, is:
\[e(I):=2i_1-d(I)=(i_1-2i_2)+(i_2-2i_3)+\cdots+(i_r).\]
Note that $\Squ^I(u)=0$ whenever $e(I)>|u|$, and if $e(I)=|u|$, then $\Squ^I(u)=(\Squ^J(u))^2$, where $J$ is obtained from $I$ by removing $i_1$.
\begin{thm*}
There exists an ordering on the monomials in $S=\Z_2[\sigma_1,\ldots,\sigma_n]$) such that whenever $I$ is admissible, with $d(I)\leq n$, we can write $\Squ^I(\sigma_n)=\sigma_n(\sigma_{i_1}\cdots\sigma_{i_r}+\textup{lower})$.
\end{thm*}
In particular, this means that the elements $\Squ^I(\imath_n)\in H^*(K(\Z_2,n);\Z_2)$ are linearly independent, where $I$ ranges over admmissible sequences with $d(I)\leq n$.

\subsection{The Adem Relations}
Some memorable Adem relations:
\begin{itemize}
\item $\Squ^1\Squ^{2n+1}=0$
\item $\Squ^1\Squ^{2n}=\Squ^{2n+1}$
\item $\Squ^{2n-1}\Squ^{n}=0$
\end{itemize}

\section{Application: The Hopf invariant}
The hopf invariant is a homomorphism $H:\pi_{2n-1}(S^n)\to\Z$. It is zero whenever $n$ is odd, and surjective when $n=2,4,8$. When $n$ is even, there is always an element of Hopf invariant $2$, so that $\pi_{2n-1}(S^n)$ contains a free summand.
\setcounter{subsection}{1}
\subsection{Decomposable operations}
Say $\Squ^i$ is \emph{decomposable} if it can be written as $\sum_{t<i}a_t\Squ^t$, where $a_t$ is some combination of squaring operations.
\begin{thm*}
$\Squ^i$ is indecomposable iff $i$ is a power of $2$.
\end{thm*}
If $i$ is a power of two, consider the action of $\Squ$ on $\alpha^i\in H^i(\RP^\infty;\Z_2)$. We have $\Squ(\alpha^i)=(\alpha+\alpha^2)^i=\alpha^i+\alpha^{2i}$, so that $\Squ^t(\alpha^i)=0$ except when $t=0,i$. Thus $\Squ^i$ cannot be decomposable. The converse comes directly from the Adem relations.
\begin{lem*}
If $p$ is prime, and $a=\sum a_ip^i$ and $b=\sum b_ip^i$ are $p$-adic expressions for $a$ and $b$, then modulo $p$, we have ${b\choose a}=\prod{b_i\choose a_i}$.
\end{lem*}
\subsection{Non-existence of elements of Hopf invariant one}
If $\pi_{2n-1}S^n$ contains an element of odd Hopf invariant, then $n$ is a power of two, as otherwise $\Squ^n$ is decomposable, and so squaring the $n$ dimensional class in the relevant CW complex returns zero, modulo 2.

\section{Application: Vector fields on spheres}
Having $k$ linearly independent fields on $S^{n-1}$ is the same as having a section for the map $V_{n,k+1}\downarrow S^{n-1}$, where $V_{n,k+1}$ is the Steifel manifold, whose points are orthonormal $k+1$-frames in $\R^n$. 

Now let $P_{n,k}$ be the stunted projective space $\RP^n/\RP^{n-k}$. It is shown that when $2k\leq n$, $P_{n,k}$ is the $n$-skeleton of $V_{n,k}$. Moreover, $\widetilde H^q(P_{n,k};\Z_2)$ is $\Z_2$ for $(n-k)\leq q<n$, and otherwise zero. By direct calculation, $\Squ^{k-1}v_{n-k}={n-k\choose k-1}v_{n-1}$, where $v_q$ is the generator of $\widetilde H^q$.
\begin{thm*}
If we write $n=(2s+1)2^m$, then $S^{n-1}$ does not admit a $2^m$-field.
\end{thm*}
\begin{proof}
This is trivial when $s=0$, or when $m=0$, so assume $s,m\geq1$. Then, writing $k=2^m+1$, we have $2k\leq n$, so that $P_{n,k}$ is the $n$-skeleton of $V_{n,k}$. In particular: \[\widetilde H^{n-k}(V_{n,k};\Z_2)=\Z_2\langle v_{n-k}\rangle,\ \ \widetilde H^{n-1}(V_{n,k};\Z_2)=\Z_2\langle v_{n-1}\rangle,\text{ and }\Squ^{k-1}v_{n-k}=v_{n-1}.\]
Note that the binomial coefficient involved is odd, by the above lemma.
Having a $2^m$-field would guarantee a section $f$ for $V_{n,k}\downarrow S^{n-1}$, inducing an  isomorphism $\widetilde H^{n-1}(V_{n,k};\Z_2)\to\widetilde H^{n-1}(S^{n-1};\Z_2)$. This would imply that the top cohomology class of $S^{n-1}$ is in the image of $\Squ^{k-1}$, which is false.
\end{proof}

\section{The Steenrod algebra}
\subsection{Graded modules and algebras}
Suppose $R$ is a commutative ring with unit. The \emph{tensor product} of graded $R$-modules $M,N$ has $(M\otimes N)_t=\oplus M_{i}\otimes N_{t-i}$. A \emph{graded $R$-algebra} is a graded $R$-module $A$ with a module homomorphism $\phi:A\otimes A\to A$. It's \emph{associative} if the obvious diagram commutes. It's \emph{commutative} if $\phi\circ T=\phi$, where $T:M\otimes N\to N\otimes M$ sends $m\otimes n\mapsto (-1)^{|m||n|}n\otimes m$.

An \emph{augmentation} is a graded $R$-algebra homomorphism $\epsilon:A\to R$ (where $R$ is a graded $R$-algebra concentrated in degree zero). An augmented $R$-algebra is \emph{connected} if $\epsilon:A_0\to R$ is an isomorphism.

The \emph{tensor product of algebras} $A$ and $B$ is formed using the multiplication $(\phi_A\otimes\phi_b)\circ(1\otimes T\otimes 1)$. That is, $(a_1\otimes b_1)(a_2\otimes b_2)=(-1)^{|b_1||a_2|}(a_1a_2)\otimes(b_1b_2)$. Given an $R$-module $M$, let $\Gamma(M)$ be the graded module with $\Gamma(M)_r=M^{\otimes r}$. This is an associative but \textbf{non}-commutative graded $R$-algebra, the \emph{tensor algebra}, with multiplication using the isomorphisms $M^{\otimes a}\otimes M^{\otimes b}=M^{\otimes (a+b)}$.

\subsection{The Steenrod algebra $\Steen$}
Let $R=\Z_2$, so $M_i=\Z_2\langle\Squ^i\rangle$ for $i\geq0$ specifies a graded $\Z_2$-module $M$. The \emph{Steenrod algebra} $\Steen$ is the quotient of $\Gamma(M)$ be the Adem relations (and $\Squ^0=1$).
\begin{thm*}[Serre-Cartan basis]
The monomials $\Squ^I$ form a basis for $\Steen$ over $\Z_2$.
\end{thm*}
We have already seen, via the action of $\Steen$ on the cohomology of $K(\Z_2,1)^n$, that these monomials are linearly independent. That they span follows by use of the Adem relations, keeping track of the \emph{moment} $m(I):=\sum i_ss$ of a sequence $I$.

Note that $\Steen$ is generated as an algebra by the $\Squ^{2^i}$, but not freely.
\setcounter{subsection}{3}
\subsection{The diagonal map of $\Steen$}
There's an algebra homomorphism $\psi:\Gamma(M)\to\Gamma(M)\otimes\Gamma(M)$ defined by $\psi(\Squ^i)=\sum_j \Squ^j\otimes\Squ^{i-j}$. That is, extended by the requirement that $\psi$ is an algebra homomorphism.
\begin{thm*}
$\psi$ descends to an algebra homomorphism $\Steen\to\Steen\otimes\Steen$.
\end{thm*}
\begin{proof}
We have maps $w_n:\Steen\to H^*(K_n;\Z_2)$ ($K_n=K(\Z_2,1)^n$) defined by $\theta\mapsto\theta(\sigma_n)$. Moreover, $w_n$ is injective on elements of degree at most $n$. It is shown  that the following commutes:
\[\xymatrix{
\Gamma(M)\ar[r]_p\ar[d]^\psi&\Steen\ar[d]_{w_n\times w_n}\ar[rd]^{w_{2n}}\\
\Steen\otimes\Steen\ar[r]^{w\otimes w\qquad \ }&H^*(K_n)\otimes H^*(K_n)\ar@{=}[r]&H^*(K_{2n})
}\]
Then, given an element $z\in\Gamma(M)$ of maximum degree $n$ such that $p(z)=0$, we must have $\psi(z)=0$ (by the diagram). This shows that the map $\psi$ descends.% Note that the commutativity of the diagram follows formally from the Cartan formula on elements $\Squ^i$, see the text for the extension to all of $\Gamma(M)$.
\end{proof}
\end{document}


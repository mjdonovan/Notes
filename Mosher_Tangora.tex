% !TEX root = z_output/_Mosher_Tangora.tex
\documentclass[11pt]{article}
%\usepackage{cancel}
\usepackage{fullpage}
\usepackage{amsmath,amsthm,amssymb}
\usepackage{mathrsfs,nicefrac}
\usepackage{amssymb}
\usepackage{epsfig}
\usepackage[all]{xy}
\usepackage{sseq}
\usepackage[pdftex]{hyperref}
\usepackage{tocloft}


\newcommand{\II}{\mathcal{I}}

\DeclareSymbolFont{AMSb}{U}{msb}{m}{n}
\DeclareMathSymbol{\N}{\mathbin}{AMSb}{"4E}
\DeclareMathSymbol{\Z}{\mathbin}{AMSb}{"5A}
\DeclareMathSymbol{\R}{\mathbin}{AMSb}{"52}
\DeclareMathSymbol{\Q}{\mathbin}{AMSb}{"51}
\DeclareMathSymbol{\PP}{\mathbin}{AMSb}{"50}
\DeclareMathSymbol{\I}{\mathbin}{AMSb}{"49}
\DeclareMathSymbol{\C}{\mathbin}{AMSb}{"43}
\DeclareMathSymbol{\A}{\mathbin}{AMSb}{"41}
\DeclareMathSymbol{\F}{\mathbin}{AMSb}{"46}

\newcommand{\ad}{\textup{\textbf{ad}}}
\newcommand{\sll}{\mathfrak{sl}}
\newcommand{\gl}{\mathfrak{gl}}
\newcommand{\GL}{\mbox{GL}}
\newcommand{\SL}{\mbox{SL}}
\newcommand{\tr}{\mbox{tr\ }}
\newcommand{\Mat}{\mbox{Mat}}
\newcommand{\Lie}{\mbox{\textbf{Lie }}}
\newcommand{\Der}{\textup{Der}}
\newcommand{\End}{\mbox{End\ }}
\newcommand{\im}{\mbox{im}}
\newcommand{\Gr}{\textup{Gr}}
\newcommand{\pim}{\mbox{pim}}
\newcommand{\supp}{\mbox{supp\,}}
\newcommand{\Ker}{\mbox{ker\ }}
\renewcommand{\ker}{\textup{ker}\,}
\newcommand{\coker}{\textup{coker}\,}

\newcommand{\PPPP}{\mathcal{P}}

\newcommand{\aaa}{\mathfrak{a}}
\newcommand{\mmm}{\mathfrak{m}}
\newcommand{\qqq}{\mathfrak{q}}
\newcommand{\ppp}{\mathfrak{p}}
\newcommand{\g}{\mathfrak{g}}
\newcommand{\h}{\mathfrak{h}}
\newcommand{\m}{\mathfrak{m}}
\newcommand{\He}{\mathfrak{H}}
\newcommand{\shfF}{\mathscr{F}}
\newcommand{\shfG}{\mathscr{G}}
\newcommand{\shfH}{\mathscr{H}}
\newcommand{\shfL}{\mathscr{L}}
\newcommand{\sHOMFG}{\mathscr{H}\textit{\!\!om}(\shfF,\shfG)}
\newcommand{\sHOM}{\mathscr{H}\textit{\!\!om}}
\newcommand{\Hom}{\textup{Hom}}
\newcommand{\Ext}{\textup{Ext}}
\newcommand{\Tor}{\textup{Tor}}
\newcommand{\Map}{\textup{Map}}
\newcommand{\sk}{\vspace*{1em}}
\renewcommand{\phi}{\varphi}
%\newcommand{\ann}{\textup{ann}}
\DeclareMathOperator{\ann}{ann}
\newcommand{\Squ}{\textup{Sq}}
\newcommand{\DASH}{\textup{---}}


\theoremstyle{plain}
\newtheorem{thm}{Theorem}[section]
\newtheorem*{thm*}{Theorem}
\newtheorem{lem}[thm]{Lemma}
\newtheorem*{lem*}{Lemma}
\newtheorem{prop}[thm]{Proposition}
\newtheorem*{prop*}{Proposition}
%\newtheorem*{prop}{Proposition}
\newtheorem{cor}[thm]{Corollary}
\newtheorem*{cor*}{Corollary}
\newtheorem{defprop}[thm]{Definition-Proposition}


\theoremstyle{definition}
\newtheorem{defn}{Definition}[section]
\newtheorem{exmp}{Example}[section]
\newtheorem{asspt}{Assumption}[section]
\newtheorem{notation}{Notation}[section]
\newtheorem{exercise}{Exercise}[section]


% 1-inch margins, from fullpage.sty by H.Partl, Version 2, Dec. 15, 1988.
\topmargin 0pt
\advance \topmargin by -\headheight
\advance \topmargin by -\headsep
\textheight 9.1in
\oddsidemargin 0pt
\evensidemargin \oddsidemargin
\marginparwidth 0.5in
\textwidth 6.5in

\parindent 0in
\parskip 1.5ex
%\renewcommand{\baselinestretch}{1.25}

\newcommand{\RAD}[1]{\textup{rad}(#1)}
\newcommand{\SPC}[1]{\textup{sp}(#1)}
\newcommand{\RED}[1]{{#1}_\textup{red}}
\newcommand{\REDpsh}[1]{{#1}_\textup{red}^-}
\newcommand{\OO}{\mathcal{O}}
\newcommand{\spec}{\textup{spec}\,}
\newcommand{\spce}{\textup{sp}}
\newcommand{\proj}{\textup{proj}\,}
\newcommand{\HOMs}{\textup{Hom}_\mathfrak{Sch}}
\newcommand{\HOMr}{\textup{Hom}_\mathfrak{Rings}}

\renewcommand{\to}{\longrightarrow}
\renewcommand{\mapsto}{\longmapsto}
\newcommand{\eps}{\varepsilon}

\newcommand{\catC}{\mathcal{C}}
\newcommand{\catD}{\mathcal{D}}
\newcommand{\catSet}{\textit{Sets}}
\newcommand{\op}{\textup{op}}
\newcommand{\dash}{\textup{---}}
\newcommand{\id}{\textup{id}}
\newcommand{\mapsfrom}{\,\reflectbox{$\mapsto$}\ }
\newcommand{\cone}{\textup{cone}}
\newcommand{\COMMENT}[1]{}

\newcommand{\calL}{\mathcal{L}}
\newcommand{\RP}{{\R\PP}}
\newcommand{\CP}{{\C\PP}}
\newcommand{\Steen}{\mathscr{A}}
\newcommand{\Orth}{\textup{\textbf{O}}}
\newcommand{\bigrightsquig}{\scalebox{2}{\ensuremath{\rightsquigarrow}}}
\newcommand{\SerreClass}{\mathscr{C}}

\newcommand{\pullbackcorner}[1][dr]{\save*!/#1+1.2pc/#1:(1,-1)@^{|-}\restore}
\newcommand{\pushoutcorner}[1][dr]{\save*!/#1-1.2pc/#1:(-1,1)@^{|-}\restore}

\newcommand{\squishlist}{
  \setlength{\itemsep}{1pt}
  \setlength{\parskip}{0pt}
  \setlength{\parsep}{0pt}
}
\newenvironment{itemise}{
\begin{list}{\textup{$\rightsquigarrow$}}
   {
      \setlength{\topsep}{.1cm}
      \setlength{\itemsep}{1pt}
      \setlength{\parskip}{0pt}
      \setlength{\parsep}{0pt}
   }
}{\end{list}\vspace{-.2cm}}



\title{Cohomology Operations\small{ --- Mosher and Tangora}}
\author{}
\date{}
% >>>

%% >>> header

\begin{document}
%\maketitle
\tableofcontents

\section{Introduction of Cohomology Operations}
\subsection{Cohomology Operations and \texorpdfstring{$K(\pi,n)$}{K(G,n)} spaces}
There is a natural equivalence of functors $[\DASH,K(\pi,n)]\simeq H^n(\DASH;\pi)$ given by $[f]\longleftrightarrow f^*(\imath_n)$. Here the \emph{fundamental class} $\imath_n\in H^n(K(\pi,n);\pi)$ is defined to be the cohomology class corresponding to $\text{id}_\pi$ under:
\[\xymatrix{
H^n(K(\pi,n);\pi)\ar[r]^{\simeq\ \ \ \ }_{\textup{UCT}\ \ \ \ }&\Hom(H_n(K(\pi,n)),\pi)\ar[r]_{\ \ \ \ \ \ \text{Hur.}}^{\ \ \ \ \ \ \simeq}&\Hom(\pi,\pi).
}\]
In summary:
\[H^m(K(\pi,n);\pi')\longleftrightarrow [K(\pi,n),K(\pi',m)]\longleftrightarrow \textup{Nat}( H^n(\DASH,\pi), H^{m}(\DASH,\pi')).\]

%\subsection{The Machinery of obstruction theory $\otimes$}
%\subsection{Applications of obstruction theory $\otimes$}


\section{Construction of the Steenrod squares}
\subsection{The complex \texorpdfstring{$K(\Z_2,1)$}{K(Z/2,1)}}
$\RP^\infty$ is a $K(\Z_2,1)$, and $S^\infty\downarrow\RP^\infty$ is a model for $E\Z_2\downarrow B\Z_2$. Moreover:
\[H_n(\RP^\infty)=\begin{cases}0&n\text{ even}\\\Z_2&n\text{ odd}\end{cases}\ \text{ and }H^n(\RP^\infty)=\begin{cases}\Z_2&n\text{ even}\\0&n\text{ odd}\end{cases}\ \text{ and }H^n(\RP^\infty;\Z_2)=\Z_2[\alpha_1].\]
%\subsection{Other Sections}
%The acyclic carrier theorem. Construction of the cup-$i$ products. The squaring operations. Compatibility with coboundary and suspension.

\section{Properties of the squares}
\begin{itemise}
\item[0.] $\Squ^i$ is a natural homomorphism $H^p(K,L;\Z_2)\to H^{p+i}(K,L;\Z_2)$
\item[1.] $\Squ^i(x)=0$ when $i>|x|$
\item[2.] $\Squ^i(x)=x^2$ when $i=|x|$
\item[3.] $\Squ^0$ is the identity
\item[4.] $\Squ^1$ is the Bockstein (associated with $0\to\Z_2\to\Z_4\to\Z_2\to0$)
\item[5.] $\Squ^i$ is compatible with the coboundary (and thus also with suspension)
\item[6.] $\Squ(xy)=\Squ(x)\Squ(y)$, where $\Squ$ denoted the total Steenrod square
\item[7.] For $a<2b$, $\Squ^a\Squ^b=\sum{b-c-1\choose a-2c}\Squ^{a+b-c}\Squ^c$. Note that in this sum only involves $0\leq c\leq a/2$. In particular, the terms on the right have $a+b-c\geq 2c$.
\end{itemise}
\setcounter{subsection}{2}
\subsection{Squares in the \texorpdfstring{$n$-fold}{n-fold} Cartesian product of \texorpdfstring{$K(\Z_2,1)$}{K(Z/2,1)}}
Let $K_n$ be the product of $n$ copies of $K(\Z_2,1)$. Then 
\[H^*(K_n,\Z_2)=\Z_2[x_1,\ldots,x_n]\supset\Z_2[\sigma_1,\ldots,\sigma_n]=S,\]
where the $\sigma_i$ are the elementary symmetric functions.

%\noindent \textbf{Proposition 3.} In $H^*(K_n;\Z_2)$, $\Squ^i(\sigma_n)=\sigma_n\sigma_i\neq0$.

\begin{prop*}
In $H^*(K_n;\Z_2)$, $\Squ^i(\sigma_n)=\sigma_n\sigma_i\neq0$.
\end{prop*}
A sequence $I=\{i_1,\ldots,i_r\}$ of (strictly positive) integers is \emph{admissible} if $i_j\geq 2i_{j+1}$, that is if it is decreasing faster that $2^{-j}$. The \emph{degree} $d(I)$ is the sum of the terms, and $\Squ^I$ raises dimension by $d(I)$. The \emph{excess}, defined only for admissible sequences, is:
\[e(I):=2i_1-d(I)=(i_1-2i_2)+(i_2-2i_3)+\cdots+(i_r).\]
Note that $\Squ^I(u)=0$ whenever $e(I)>|u|$, and if $e(I)=|u|$, then $\Squ^I(u)=(\Squ^J(u))^2$, where $J$ is obtained from $I$ by removing $i_1$.
\begin{thm*}
There exists an ordering on the monomials in $S=\Z_2[\sigma_1,\ldots,\sigma_n]$) such that whenever $I$ is admissible, with $d(I)\leq n$, we can write $\Squ^I(\sigma_n)=\sigma_n(\sigma_{i_1}\cdots\sigma_{i_r}+\textup{lower})$.
\end{thm*}
In particular, this means that the elements $\Squ^I(\imath_n)\in H^*(K(\Z_2,n);\Z_2)$ are linearly independent, where $I$ ranges over admmissible sequences with $d(I)\leq n$.

\subsection{The Adem Relations}
Some memorable Adem relations:
\begin{itemise}
\item $\Squ^1\Squ^{2n+1}=0$
\item $\Squ^1\Squ^{2n}=\Squ^{2n+1}$
\item $\Squ^{2n-1}\Squ^{n}=0$
\end{itemise}
\section{Application: The Hopf invariant}
The hopf invariant is a homomorphism $H:\pi_{2n-1}(S^n)\to\Z$. It is zero whenever $n$ is odd, and surjective when $n=2,4,8$. When $n$ is even, there is always an element of Hopf invariant $2$, so that $\pi_{2n-1}(S^n)$ contains a free summand.
\setcounter{subsection}{1}
\subsection{Decomposable operations}
Say $\Squ^i$ is \emph{decomposable} if it can be written as $\sum_{t<i}a_t\Squ^t$, where $a_t$ is some combination of squaring operations.
\begin{thm*}
$\Squ^i$ is indecomposable iff $i$ is a power of $2$.
\end{thm*}
If $i$ is a power of two, consider the action of $\Squ$ on $\alpha^i\in H^i(\RP^\infty;\Z_2)$. We have $\Squ(\alpha^i)=(\alpha+\alpha^2)^i=\alpha^i+\alpha^{2i}$, so that $\Squ^t(\alpha^i)=0$ except when $t=0,i$. Thus $\Squ^i$ cannot be decomposable. The converse comes directly from the Adem relations.
\begin{lem*}
If $p$ is prime, and $a=\sum a_ip^i$ and $b=\sum b_ip^i$ are $p$-adic expressions for $a$ and $b$, then modulo $p$, we have ${b\choose a}=\prod{b_i\choose a_i}$.
\end{lem*}
\subsection{Non-existence of elements of Hopf invariant one}
If $\pi_{2n-1}S^n$ contains an element of odd Hopf invariant, then $n$ is a power of two, as otherwise $\Squ^n$ is decomposable, and so squaring the $n$ dimensional class in the relevant CW complex returns zero, modulo 2.

\section{Application: Vector fields on spheres}
Having $k$ linearly independent fields on $S^{n-1}$ is the same as having a section for the map $V_{n,k+1}\downarrow S^{n-1}$, where $V_{n,k+1}$ is the Steifel manifold, whose points are orthonormal $k+1$-frames in $\R^n$. 

Now let $P_{n,k}$ be the stunted projective space $\RP^n/\RP^{n-k}$. It is shown that when $2k\leq n$, $P_{n,k}$ is the $n$-skeleton of $V_{n,k}$. Moreover, $\widetilde H^q(P_{n,k};\Z_2)$ is $\Z_2$ for $(n-k)\leq q<n$, and otherwise zero. By direct calculation, $\Squ^{k-1}v_{n-k}={n-k\choose k-1}v_{n-1}$, where $v_q$ is the generator of $\widetilde H^q$.
\begin{thm*}
If we write $n=(2s+1)2^m$, then $S^{n-1}$ does not admit a $2^m$-field.
\end{thm*}
\begin{proof}
This is trivial when $s=0$, or when $m=0$, so assume $s,m\geq1$. Then, writing $k=2^m+1$, we have $2k\leq n$, so that $P_{n,k}$ is the $n$-skeleton of $V_{n,k}$. In particular: \[\widetilde H^{n-k}(V_{n,k};\Z_2)=\Z_2\langle v_{n-k}\rangle,\ \ \widetilde H^{n-1}(V_{n,k};\Z_2)=\Z_2\langle v_{n-1}\rangle,\text{ and }\Squ^{k-1}v_{n-k}=v_{n-1}.\]
Note that the binomial coefficient involved is odd, by the above lemma.
Having a $2^m$-field would guarantee a section $f$ for $V_{n,k}\downarrow S^{n-1}$, inducing an  isomorphism $\widetilde H^{n-1}(V_{n,k};\Z_2)\to\widetilde H^{n-1}(S^{n-1};\Z_2)$. This would imply that the top cohomology class of $S^{n-1}$ is in the image of $\Squ^{k-1}$, which is false.
\end{proof}

\section{The Steenrod algebra}
\subsection{Graded modules and algebras}
Suppose $R$ is a commutative ring with unit. The \emph{tensor product} of graded $R$-modules $M,N$ has $(M\otimes N)_t=\oplus M_{i}\otimes N_{t-i}$. A \emph{graded $R$-algebra} is a graded $R$-module $A$ with a module homomorphism $\phi:A\otimes A\to A$. It's \emph{associative} if the obvious diagram commutes. It's \emph{commutative} if $\phi\circ T=\phi$, where $T:M\otimes N\to N\otimes M$ sends $m\otimes n\mapsto (-1)^{|m||n|}n\otimes m$.

An \emph{augmentation} is a graded $R$-algebra homomorphism $\epsilon:A\to R$ (where $R$ is a graded $R$-algebra concentrated in degree zero). An augmented $R$-algebra is \emph{connected} if $\epsilon:A_0\to R$ is an isomorphism.

The \emph{tensor product of algebras} $A$ and $B$ is formed using the multiplication $(\phi_A\otimes\phi_b)\circ(1\otimes T\otimes 1)$. That is, $(a_1\otimes b_1)(a_2\otimes b_2)=(-1)^{|b_1||a_2|}(a_1a_2)\otimes(b_1b_2)$. Given an $R$-module $M$, let $\Gamma(M)$ be the graded module with $\Gamma(M)_r=M^{\otimes r}$. This is an associative but \textbf{non}-commutative graded $R$-algebra, the \emph{tensor algebra}, with multiplication using the isomorphisms $M^{\otimes a}\otimes M^{\otimes b}=M^{\otimes (a+b)}$.

\subsection{The Steenrod algebra \texorpdfstring{$\Steen$}{A}}
Let $R=\Z_2$, so $M_i=\Z_2\langle\Squ^i\rangle$ for $i\geq0$ specifies a graded $\Z_2$-module $M$. The \emph{Steenrod algebra} $\Steen$ is the quotient of $\Gamma(M)$ be the Adem relations (and $\Squ^0=1$).
\begin{thm*}[Serre-Cartan basis]
The monomials $\Squ^I$ form a basis for $\Steen$ over $\Z_2$.
\end{thm*}
We have already seen, via the action of $\Steen$ on the cohomology of $K(\Z_2,1)^n$, that these monomials are linearly independent. That they span follows by use of the Adem relations, keeping track of the \emph{moment} $m(I):=\sum i_ss$ of a sequence $I$.

Note that $\Steen$ is generated as an algebra by the $\Squ^{2^i}$, but not freely.
\setcounter{subsection}{3}
\subsection{The diagonal map of \texorpdfstring{$\Steen$}{A}}
There's an algebra homomorphism $\psi:\Gamma(M)\to\Gamma(M)\otimes\Gamma(M)$ defined by $\psi(\Squ^i)=\sum_j \Squ^j\otimes\Squ^{i-j}$. That is, extended by the requirement that $\psi$ is an algebra homomorphism.
\begin{thm*}
$\psi$ descends to an algebra homomorphism $\Steen\to\Steen\otimes\Steen$.
\end{thm*}
\begin{proof}
We have maps $w_n:\Steen\to H^*(K_n;\Z_2)$ ($K_n=K(\Z_2,1)^n$) defined by $\theta\mapsto\theta(\sigma_n)$. Moreover, $w_n$ is injective on elements of degree at most $n$. It is shown  that the following commutes:
\[\xymatrix{
\Gamma(M)\ar[r]_p\ar[d]^\psi&\Steen\ar[d]_{w_n\times w_n}\ar[rd]^{w_{2n}}\\
\Steen\otimes\Steen\ar[r]^{w\otimes w\qquad \ }&H^*(K_n)\otimes H^*(K_n)\ar@{=}[r]&H^*(K_{2n})
}\]
Then, given an element $z\in\Gamma(M)$ of maximum degree $n$ such that $p(z)=0$, we must have $\psi(z)=0$ (by the diagram). This shows that the map $\psi$ descends.% Note that the commutativity of the diagram follows formally from the Cartan formula on elements $\Squ^i$, see the text for the extension to all of $\Gamma(M)$.
\end{proof}
In particular, $\Steen$ is a Hopf algebra. It is associative but not commutative, while it is both co-associative and co-commutative.
\subsection{The dual of the Steenrod algebra}
Suppose that $A$ is a (connected) Hopf algebra over a field $R$ of finite type (each graded part is finite dimensional). Then the \emph{dual Hopf algebra} $A^*$ is the componentwise dual of $A$, with multiplication the dual of the old comultiplication, etc. If the old multiplication is associative (commutative), then the new comultiplication is so, and vice versa.

In particular, $\Steen^*$ is a Hopf algebra which is associative, commutative, and coassociative.

Let $x$ be the generator of $H^1(K(\Z_2,1);\Z_2)$. For each $i\geq0$, let $\xi_i\in\Steen^*_{2^i-1}$ be characterised by:
\[\theta(x)=\xi_i(\theta)\cdot x^{2^i},\text{ for all }\theta\in\Steen_{2^i-1}.\]
\begin{prop*}
If $k\geq1$ and $I$ is admissible, then $\langle\xi_k,\Squ^I\rangle=\begin{cases}1,&I=I_k:=\{2^{k-1},2^{k-2},\ldots,1\};\\0,&\textup{otherwise.}\end{cases}$ Moreover, for $I$ inadmissible, $\langle\xi_k,\Squ^I\rangle=0$ unless $I$ is obtained from $I_k$ by interspersion of zeros.
\end{prop*}
As an algebra, $\Steen^*$ is the polynomial ring $\Z_2[\xi_1,\xi_2,\ldots]$.

\subsection{Algebras over a Hopf algebra}
If $A$ is a Hopf algebra and $M$ is a graded $A$-module (i.e.\ $M$ is a module of the algebra $A$), then $M\otimes M$ has a natural structure of graded $A$-module via
\[\xymatrix{
A\otimes M\otimes M\ar[r]^{\psi\otimes1\otimes1\ \ \ }&A\otimes A\otimes M\otimes M\ar[r]^T&A\otimes M\otimes A\otimes M\ar[r]^{\qquad\mu\otimes \mu\ }&M\otimes M
}\]
Now if $M$ is also an $R$-algebra, say that $M$ is an \emph{algebra over the Hopf algebra} $A$ is multiplication $M\otimes M\to M$ is an $A$-module homomorphism.

For example, $H^*(X;\Z_2)$ is a module over $\Steen$, and by the Cartan formula (which is reflected in the diagonal map), it is actually an algebra thereover.
\subsection{The diagonal map of \texorpdfstring{$\Steen^*$\ \ \ $\bigotimes$}{A}}
\subsection{The Milnor basis for \texorpdfstring{$\Steen$}{A}}
For any sequence $R$ of nonnegative integers (all but finitely many zero), we have an element $\xi^R$ of $\Steen^*$. These elements are a basis of $\Steen^*$, and the dual basis $\{\Squ^R\}$ of $\Steen$ is referred to as the Milnor basis. It is very different from the Serre-Cartan basis, but we do have $\Squ^{\{i,0,\ldots\}}=\Squ^i$.

\section{Exact couples and spectral sequences}

\section{Fiber Spaces}
\begin{itemise}
\item By fibration, we mean 'Serre fibration'.
\item In a Hurewicz fibration, all fibres are homotopic, while in a Serre fibration, we can only assert this when both fibres are finite complexes.
\item If $C\downarrow B$ is a covering space ($C$ connected), then the fiber is the discrete group $G$ of covering transformations. Moreover, the LES gives $\pi_n(C)=\pi_n(B)$ for $n\geq2$, and a short exact sequence $0\to\pi_1(C)\to\pi_1(B)\to G\to0$.
\item $p:\Orth(n+1)\to S^n$ given by evaluation at a fixed point of $S^n$, has fibre $\Orth(n)$. Then the LES gives $\pi_{i+1}(S^n)\to\pi_i(\Orth(n))\to\pi_i(\Orth(n+1))\to\pi_i(S^n)$, and the middle map is an isomorphism when $i<n-1$. Thus the groups $\pi_i(\Orth(n))$ are isomorphic for $n>i+1$. The stable values are known, and satisfy Bott periodicity.
\end{itemise}
\setcounter{subsection}{3}
\subsection{Serre's exact sequence}
Suppose that $F\rightarrow E\downarrow B$ is a fibre space, with $B$ simply connected. Suppose further that $H_i(B)=0$ for $0<i<p$ and $H_j(F)=0$ for $0<j<q$. Then there is a (finite) exact sequence
\[\xymatrix{
H_{p+q-1}(F)\ar[r]^{i_*}&H_{p+q-1}(E)\ar[r]^{p_*}&H_{p+q-1}(B)\ar[r]^{\tau}&H_{p+q-2}(F)\ar[r]&\cdots\ar[r]&H_1(E)\ar[r]&0
}\]
This can be used to prove the Hurewicz and relative Hurewicz theorems.
\begin{thm*}\label{Whthd}
Suppose $f:X\hookrightarrow Y$ is an inclusion, where $X$ is arc-connected, $Y$ is simply connected, and $\pi_2(Y,X)$ is abelian. Then the following are equivalent (for any $n\geq1$):
\begin{itemise}
\item[1.] $f_*:H_i(X)\to H_i(Y)$ is iso for $i<n$ and epi for $i=n$;
\item[2.] $f_\#:\pi_i(X)\to \pi_i(Y)$ is iso for $i<n$ and epi for $i=n$;
\item[3.] The pair $(Y,X)$ is $n$-connected; and
\item[4.] $H_i(Y,X)=0$ for $i\leq n$. \hfill (Compare with section \ref{WhthdModC})
\end{itemise}
\end{thm*}
\begin{proof}
2 and 3 are equivalent, as are 1 and 4, by exactness in the long exact sequences. Moreover, 3 and 4 are equivalent by the relative Hurewicz theorem, as long as $\pi_2(Y,X)$ is abelian.
\end{proof}
\begin{cor*} If $X$ and $Y$ are simply connected, and $f:X\to Y$, then $f$ is a homology isomorphism if and only if it is a homotopy isomorphism.
\begin{proof}
Replace $f$ by an inclusion. We need $\pi_2(Y,X)$ abelian, but have exact sequence $\pi_2(Y)\to\pi_2(Y,X)\to0$.
\end{proof}
\end{cor*}
\setcounter{subsection}{5}
\subsection{The cohomology spectral sequence of a fibre space}
\begin{itemise}
\squishlist
\item Each $E_r$ is a bigraded ring.
\item In $E_r$, $d_r$ is an anti-derivation (with respect to \emph{total} degree): $d(ab)=d(a)b+(-1)^{|a|}ad(b)$.
\item The product on $E_{r+1}$ is induced by that of $E_r$, and that on $E^\infty$ is induced by that in $H^*(E;R)$.
\item When $B$ and $F$ are $(p-1)$- and $(q-1)$-connected, respectively, we get:
\[\xymatrix{
\cdots\ar[r]&H^{p+q-2}(F)\ar[r]^\tau&H^{p+q-1}(B)\ar[r]^{p^*}&H^{p+q-1}(E)\ar[r]^{i^*}&H^{p+q-1}(F).
}\]
\item If $x$ is transgressive, then so is $\Squ^i(x)$, and if $\tau(x)=[y]$, $\tau(\Squ^i(x))=[\Squ^i(y)]$.
\end{itemise}
To see the last point, consider the following diagram which describes the transgression:
\[\xymatrix{
H^*(F)\ar[r]&H^{*+1}(E,F)&E_\infty^{*+1,0}\\
E_\infty^{0*}\ar@{^{(}->}[u]\ar@{.>}[rur]&H^{*+1}(B,*)\ar[u]\ar[r]&H^{*+1}(B)\ar@{->>}[u]
}\]
By the naturality of $\Squ^i$, and compatibility with the coboundary, we have:
\[\xymatrix{
x\ar@{|->}[r]&\delta x&[y]\\
x\ar@{|.>}[rur]\ar@{|->}[u]&z\ar@{|->}[u]\ar@{|->}[r]&y\ar@{|->}[u]
}\raisebox{-.8cm}{\qquad\bigrightsquig\qquad }
\xymatrix{
\Squ^ix\ar@{|->}[r]&\delta\,\Squ^ix&[\Squ^iy]\\
\Squ^ix\ar@{|.>}[rur]\ar@{|->}[u]&\Squ^iz\ar@{|->}[u]\ar@{|->}[r]&\Squ^iy\ar@{|->}[u]
}\]

\section{Cohomology of \texorpdfstring{$K(\pi,n)$}{K(G,n)}}
We can attempt to use the path-loop fibration and the fibration of Eilenberg-MacLane spaces associated with a short exact sequence of abelian groups to calculate $H^*(K(\pi,n);G)$.
\subsection{\texorpdfstring{$H^*(K(Z_2,q);Z_2)$}{H*(K(Z/2,q);Z/2)} and Borel's theorem}
A graded ring $R$ over $\Z_2$ has an order set $x_1,x_2,\ldots$ of homogeneous elements as a \emph{simple system of generators} if the monomials $\{x_{i_1}\cdots x_{i_r}:i_1<\cdots<i_r\}$ form a $\Z_2$-basis, and each graded part is finite dimensional.
\begin{thm*}
Suppose $F\rightarrow E\downarrow B$ is a fiber space with $E$ acyclic, and that $H^*(F;\Z_2)$ has a simple system $\{x_\alpha\}$ of transgressive generators. Then $H^*(B;\Z_2)$ is a polynomial ring in the $\{\tau(x_\alpha)\}$.
\end{thm*}
The following results then follow from Borel's theorem:
\begin{thm*}
$H^*(K(\Z_2,q);\Z_2)=\Z_2[\Squ^I(\imath_q)]$, over those admissible $I$ with excess less than $q$.
\end{thm*}
\begin{prop*}
$H^*(K(\Z,2);\Z_2)=\Z_2[i_2]$.
\end{prop*}
\begin{thm*}$H^*(K(\Z,q);\Z_2)=\Z_2[\Squ^I(\imath_q)]$, over those admissible $I$ with excess less than $q$ and last entry not equal to 1.
\end{thm*}
\begin{prop*}$H^*(K(\Z_{2^m},1);\Z_2)=\Z_2[d_m(\imath_q)]\otimes \Lambda(\imath_1)$ for $m\geq2$, where $d_m$ denotes the $m^\textup{th}$ differential in the Bockstein spectral sequence.
\end{prop*}
\begin{proof}
We calculate that $H^1(K(\Z_2,1);\Z)=\Z_{2^m}$. Then (using universal coefficients, if you will) $H^1(K(\Z_2,1);\Z_2)=\Z_{2}\langle\imath_1\rangle$. Moreover, as $\imath_1$ comes from a class of order $2^m$ and no further, we have $d_j(\imath_1)=0$ for $j<m$ and $d_m(\imath_1)\neq0$. Thus $d_m\imath_1$ is a nonzero element of $H^2$. Moreover, as $\imath_1^2=d_1\imath_1$, we have $\imath_1^2=0$. Everything else needed comes from the spectral sequence.
\end{proof}
\begin{thm*}
$H^*(K(\Z_{2^m},q);\Z_2)$ is a polynomial ring generated by $\{\Squ^{I_m}\imath_1\}$, for $I$ admissible satisfying $e(I)<d$. Here, $\Squ^{I_m}$ refers to $\Squ^I$ when the last entry $i_r$ is not 1, and to $\Squ^{i^1}\cdots \Squ^{i_{r-1}}d_m$ when $i_r=1$.
\end{thm*}
\subsection{Proof of Borel's Theorem}
Here we consider only first quadrant spectral sequences of standard cohomological type such that $E_2^{pq}=E_2^{p0}\otimes E_2^{0p}$. A homomorphism of such is to be a collection of maps $f_r^{pq}$ which commute with the differentials, such that $f^{pq}_{r+1}$ is the map induced  on cohomology by $f^{pq}_r$, and such that $f_r^{pq}=f_r^{p0}\otimes f_r^{0q}$. We have a comparison theorem:
\begin{thm*}
Suppose that $F:E\to\overline E$ is such a homomorphism of such spectral sequences, and that both $E_\infty$ and $\overline E_\infty$ are concentrated at $(0,0)$. Then if all the maps $f_2^{0q}$ (or alternatively all the $f_2^{p0}$) are isomorphisms, all the $f_r^{pq}$ are isomorphisms for $r\geq2$.
\end{thm*}
\begin{proof}[Proof of Borel's theorem]
\newcommand{\shrt}[1]{\makebox[0cm]{\ensuremath{#1}}}
%It's enough to show that all the $f_2^{p0}$ are isomorphisms.
The idea is to let $\overline E$ be the spectral sequence for the fibration of Borel's theorem, and to construct another spectral sequence $E$ and map $f$ such that the comparison theorem applies. Let the simple system of transgressive generators be $\{x_\alpha\}$ (and assume for ease that all have positive degree).

Now we synthesise a filtered differential bigraded algebra whose associated spectral sequence mimics the behaviour of $\overline E$. Define:
\[P:=\Z_2[y_\alpha]\text{ and }\Lambda:=\Lambda_{\Z_2}(z_\alpha)\text{ where $|y_\alpha|=|x_\alpha|+1$, and $|z_\alpha|=|x_\alpha|.$}\]
Let $R$ be the bigraded algebra $P\otimes\Lambda$, with differential specified by $\delta(y\otimes z_\alpha)=yy_\alpha$ (extended by the requirement it's a derivation). Finally, we filter $R$ by $F^p=\sum_{i\geq 0}P^{p+i}\otimes\Lambda$. This decreasing filtration has subquotients denoted $\Gr^p:=F^p/F^{p+1}$ which additively is just $P^p\otimes\Lambda$.
%\[\xymatrix{
%0\ar[rr]|{\ \cdot\ \cdot\ \cdot\ }&&H^*(F^{p+2})\ar[rr]\ar[ld]&&H^*(F^{p+1})\ar[rr] \ar[ld]&&H^*(F^{p})\ar[ld]
%\ar[rr]|{\ \cdot\ \cdot\ \cdot\ }&&H^*(R)\\
%&\ \ \ \ \shrt{H^*(\Gr^{p+2})}\phantom{{5\choose3}}&&
%\shrt{H^*(\Gr^{p+1})}\phantom{{5\choose3}}\ar@{~>}[lu]_\delta&&
%\shrt{H^*(\Gr^{p})}\phantom{{5\choose3}}\ar@{~>}[lu]_\delta&&
%\shrt{H^*(\Gr^{p-1})}\phantom{{5\choose3}}\ar@{~>}[lu]_\delta
%}\]
\[\xymatrix{
0\ar[rr]|{\ \cdot\ \cdot\ \cdot\ }&&H^*(F^{p+1})\ar[rr] \ar[ld]&&H^*(F^{p})\ar[ld]
\ar[rr]|{\ \cdot\ \cdot\ \cdot\ }&&H^*(R)\\
&
\shrt{H^*(\Gr^{p+1})}\phantom{{5\choose3}}&&
\shrt{H^*(\Gr^{p})}\phantom{{5\choose3}}\ar@{~>}[lu]_\delta&&
\shrt{H^*(\Gr^{p-1})}\phantom{{5\choose3}}\ar@{~>}[lu]_\delta
}\]
Now in the corresponding spectral sequence $E$, the $E_1^{pq}$ term is then $H^{p+q}(\Gr^p)$. Moreover, the differential $d_r$ is defined as follows. Take $[x]\in H^{p+q}(\Gr^p)$ which has persited to $E_r$. Then $\delta x$  in fact lies in $\sum_{i\geq r}P^{p+i}\otimes \Lambda^{q-i+1}$, as all the previous differentials vanished. Then $d_r[x]$ is the projection of $\delta x$ onto the first factor $P^{p+r}\otimes \Lambda^{q-r+1}$.

Now note that $d_1=0$, as all the $x_\alpha$ have positive degree. Thus $E_2^{pq}=R^{pq}$, and that $E$ has the properties of the opening paragraph. The acyclicity is easily verified if $\{x_\alpha\}$ has only one element, and follows in general by the K\"unneth formula.

Now define a ring homomorphism $f_2^{*0}:P\to H^*(B;\Z_2)$ by $y\mapsto \tau(x_\alpha)$, and an additive (not multiplicative) homomorphism $f_2^{0*}:\Lambda\to H^*(F;\Z_2)$ by $z_{i_1}\cdots z_{i_r}\mapsto x_{i_1}\cdots x_{i_r}$. As then $f_2^{**}$ has image written in terms of the transgressive $x_\alpha$ and their images $\tau(x_\alpha)$, it is easy to check that this map is compatible with $d_2$ --- every differential vanishes on $x_\alpha$ but the transgression, and they all vanish on $\tau(x_\alpha)$. This allows us to define $f_3$, and a similar check shows that it is compatible with $d_3$. This argument continues ad infinitum, producing a morphism $E\to\overline E$, to which we may apply the comparison theorem. As the $f_2^{0*}$ are isomorphisms, so are the $f_2^{*0}$, completing the proof.
%
% Now we can only define $f_r$ when $f_{r+1}$
%
%Note to self: Next discuss the map as defined. Show that it intertwines the differentials, by noting that the $x_\alpha$ all transgress, and so are killed by any differential which doesn't make it to the opposite edge.
% which is a sum of terms of the form
\end{proof}

\section{Classes of abelian groups}
A \emph{class of abelian groups} is a collection $\SerreClass$ of abelian groups that is closed under formation of subgroups, quotient groups, and group extensions. That is, given a short exact sequence $0\to A'\to A\to A''\to0$, $A\in\SerreClass$ iff $A',A''\in\SerreClass$.

A homomorphism $f:A\to B$ is a \emph{$\SerreClass$-monomorphism} if its kernel lies in $\SerreClass$, a \emph{$\SerreClass$-epimorphism} if its cokernel lies in $\SerreClass$, and a \emph{$\SerreClass$-isomorphism} if both its kernel and cokernel lie in $\SerreClass$. Say two groups are $\SerreClass$-isomorphic if they are connected by a chain of $\SerreClass$-isomorphisms (which needn't be composable).

Further axioms that Serre classes may satisfy include:
\begin{itemise}
\item[2A.] being closed under formation of tensor products and $\Tor$;
\item[2B.] if  $A\in\SerreClass$ then $A\otimes B\in\C$ for all abelian groups $B$ (this implies 2A); and
\item[3.] if $A\in\SerreClass$ then $H_n(K(A,1);\Z)\in\SerreClass$ for all $n>0$.
\end{itemise}
Examples of Serre classes include:
\begin{itemise}
\item $\SerreClass_{FG}$, the finitely generated abelian groups, which satisfy 2A and 3 (but not 2B).
\item $\SerreClass_{p}$ for $p$ prime, the class of abelian torsion groups of finite exponent $n$ coprime to $p$. Here, the exponent of $A$ is the infimum of those $n$ such that $na=0$ for all $a\in A$. If $(n,p)=1$ then all elements have order coprime to $p$. This class satisfies 2B and 3.
\item The countable abelian groups. These satisfy 2A (but not 2B) and I haven't thought about 3.
\end{itemise}
\setcounter{subsection}{1}
\subsection{Topological theorems mod \texorpdfstring{$\SerreClass$}{C}}
\begin{thm*}[Hurewicz mod $\SerreClass$]
Suppose that $X$ is simply connected, and that a Serre class $\SerreClass$ satisfies \textup{2A} and \textup{3}. Then if $\pi_i(X)\in\SerreClass$ for $i<n$ then $H_i(X)\in\SerreClass$ for $i<n$ and $h:\pi_n(X)\to H_n(X)$ is a $\SerreClass$-isomorphism.
\end{thm*}
This has the immediate consequence that the homotopy groups of a $1$-connected complex with only finitely many cells in each dimension are finitely generated (using $\SerreClass_{FG}$).
\begin{thm*}[Relative Hurewicz mod $\SerreClass$]
Suppose that $A\subseteq X$, that both $A$ and $X$ are simply connected, and that $\pi_2(A)\to\pi_2(X)$ is an epimorphism. Suppose that $\SerreClass$ satisfies \textup{2B} and \textup{3}. Then if $\pi_i(X,A)\in\SerreClass$ for $i<n$ then $H_i(X,A)\in\SerreClass$ for $i<n$ and $h:\pi_n(X,A)\to H_n(X,A)$ is a $\SerreClass$-isomorphism.
\end{thm*}
Note that this is a little weaker than the standard Relative Hurewicz theorem (immediately below). Mod $\SerreClass$, we need both $A$ and $X$ simply connected.
\begin{thm*}[Standard Relative Hurewicz]
Suppose that $(X,A)$ is an $(n-1)$-connected pair of path-connected spaces with $n\geq2$ and $A\neq\emptyset$. Then $H_i(X,A)=0$ for $i<n$ and $h:\pi_n(X,A)\to H_n(X,A)$ is an isomorphism.
\end{thm*}
\begin{thm*}[Whitehead's theorem mod $\SerreClass$]\label{WhthdModC}
Suppose that $f:A\to X$ is a map of simply connected spaces that induces an isomorphism on $\pi_2$. Suppose that $\SerreClass$ satisfies \textup{2B} and \textup{3}. Then TFAE:
\begin{itemise}
\item[1.] $f_*:H_i(X)\to H_i(Y)$ is $\SerreClass$-iso for $i<n$ and $\SerreClass$-epi for $i=n$; and
\item[2.] $f_\#:\pi_i(X)\to \pi_i(Y)$ is $\SerreClass$-iso for $i<n$ and $\SerreClass$-epi for $i=n$. \hfill (Compare with section \ref{Whthd}).
%\item[3.] The pair $(Y,X)$ is $n$-connected; and
%\item[4.] $H_i(Y,X)=0$ for $i\leq n$.
\end{itemise}
\end{thm*}
Note that this is a corollary of the relative Hurewicz theorem mod $\SerreClass$.


\setcounter{subsection}{5}
\subsection{The \texorpdfstring{$\SerreClass_p$}{Cp} approximation theorem}
Let $X$ and $A$ be simply connected (nice) spaces with $H^i(X)$ and $H^i(A)$ finitely generated for each $i$. Suppose $f:A\to X$ is a map inducing an epimorphism on $\pi_2$ (and assume without loss of generality that $f$ is an inclusion). Then conditions 1--6 are equivalent, and imply 7:
\begin{itemise}
\item[1.] $f^*:H^i(X;\Z_p)\to H^i(A;\Z_p)$ is iso for $i<n$ and mono for $i=n$;
\item[2.] $f_*:H_i(A;\Z_p)\to H_i(X;\Z_p)$ is iso for $i<n$ and epi for $i=n$;
\item[3.] $H_i(X,A;\Z_p)=0$ for $i\leq n$;
\item[4.] $H_i(X,A;\Z)\in\SerreClass_p$ for $i\leq n$;
\item[5.] $\pi_i(X,A)\in\SerreClass_p$ for $i\leq n$;
\item[6.] $f_*:\pi_i(A)\to \pi_i(X)$ is $\SerreClass_p$-iso for $i<n$ and $\SerreClass_p$-epi for $i=n$;
\item[7.] $\pi_i(A)$ and $\pi_i(X)$ have the same $p$-components for $i<n$.
\end{itemise}
That is, to calculate the $p$-component of $\pi_i(X)$, it is enough to find a map $A\to X$ where inducing isomorphisms on $H^*(\DASH;\Z_p)$ (as long as both spaces are nice, etc).
\begin{proof}
$1\Leftrightarrow 2$ by duality, noting finite generation. $2\Leftrightarrow 3$ using the LES. $3\Rightarrow 4$ as $H_i(X,A)\otimes\Z_p\subset H_i(X,A;\Z_p)$ by the UCT. $3\Leftarrow 4$ using the UCT an axiom 2B. $4\Leftrightarrow 5$ by relative Hurewicz. $5\Leftrightarrow 6$ using the exact homotopy sequence. To see that these imply 7, note that $\pi_i(A)$ and $\pi_i(X)$ are finitely generated, by Hurewicz mod $\SerreClass_{FG}$. So there is a $\SerreClass_p$-iso between the groups in question, which are finitely generated. Thus the $p$-components are the same.
\end{proof}

\section{More about fiber spaces}
Homotopic maps induce fiber homotopy equivalent pullback fiber spaces.
\begin{prop*}
Suppose that $E\downarrow B$ is a Hurewicz fibration, and we have maps $Y\to X\to B$:
\[\xymatrix{
&f^*E\ar[r]\ar[d]&E\ar[d]\\
Y\ar[r]^g\ar@{-->}[ur]^h&X\ar[r]^f&B
}\]
Then if $fg$ is null, a lifting $h$ of $g$ exists. If $E$ is contractible, the converse holds. (This works given a Serre fibration $E\downarrow B$, as long as $Y$ is a finite complex).
\end{prop*}

\setcounter{subsection}{1}
\subsection{The transgression of the fundamental class}
Consider  the fibration $F\rightarrow E\downarrow B$, where $F=K(\pi,n)$, $E$ is contractible and $B=K(\pi,n+1)$. We have the following isomorphisms from Hurewicz and long exact sequences:%LESs:
\[\xymatrix{
\pi_n(F)\ar[d]_h^\simeq&\pi_{n+1}(E,F)\ar[d]_h^\simeq\ar[l]^\simeq\ar[r]_\simeq&\pi_{n+1}(B,*)\ar[d]_h^\simeq\\
H_n(F)&H_{n+1}(E,F)\ar[l]_\simeq\ar[r]^\simeq&H_{n+1}(B,*)
}\]
Then entries on the top row are all canonically isomorphic to $\pi$. Applying $\Hom(\DASH,\pi)$ to the bottom row, we obtain (using the UCT and the definition of $\imath_n$):
\[\xymatrix@C=.7cm{
\Hom(H_n(F),\pi)\ar[d]^\simeq\ar[r]_{\simeq\ \ }&
\Hom(H_{n+1}(E,F),\pi)\ar[d]&
\Hom(H_{n+1}(B,*),\pi)\ar[d]^\simeq\ar[l]^\simeq&
h^{-1}\ar@{|->}[r]\ar@{|->}[d]&
h^{-1}\ar@{|->}[d]&
h^{-1}\ar@{|->}[l]\ar@{|->}[d]\\
H^n(F;\pi)\ar[r]^{\simeq\ \ }&H^{n+1}(E,F;\pi)&H^{n+1}(B,*;\pi)\ar[l]_\simeq&
\imath_n\ar@{|->}[r]&\jmath&\imath_{n+1}\ar@{|->}[l]
}\]
This shows that in the SSS for this fibration, with $\pi$ coefficients, the fundamental class $\imath_n$ transgresses to the fundamental class $\imath_{n+1}$.

Next, suppose that we are given a map $f:X\to K(\pi,n+1)$, corresponding to $\psi\in H^{n+1}(X;\pi)$ (so that $\psi=f^*(\imath_n)$). Then we have a map of fiber sequences:
\[\xymatrix{
%K(\pi,n+1)\ar[r]&{*}\ar[r]&K(\pi,n)
K(\pi,n)\ar[r]\ar@{=}[d]&f^*(\star)\ar[d]\ar[r]&X\ar[d]^f\\
K(\pi,n)\ar[r]&\star\ar[r]&K(\pi,n+1)
}\]
This induces a map of cohomology Serre exact sequences, which show that:
\begin{equation*}
\tag{Formula 1}\psi=f^*(\imath_{n+1})=\tau(\imath_n)
\end{equation*}
That is, in the SSS for the top row fibre sequence, the fundamental class (in the cohomology of the fibre) transgresses to the element $\psi$ (in that of the base $X$).

\subsection{Bocksteins and the Bockstein lemma}
We have the Bockstein exact couple:
\[\xymatrix{
H^*(\DASH;\Z)\ar[rr]^{\times2}&&H^*(\DASH;\Z)\ar[dl]^\rho\\
&H^*(\DASH;\Z_2)\ar@{~>}[ul]^{\beta}
}\]
The operator $d_r$ is defined to be the $r^\textup{th}$ differential in the corresponding spectral sequence. In particular, one applies $\beta:H^*(\DASH;\Z_2)\to H^*(\DASH;\Z)$, divides the result by $2^{r-1}$ (which can be done iff the previous $d_i$ vanished), and reduces modulo $2$.

Now $d_i$ vanishes on any class in the image of $\rho:H^*(\DASH;\Z_{2^{i+1}})\to H^*(\DASH;\Z_{2})$. Thus we get maps:
\[d_i\rho:H^*(\DASH;\Z_{2^{i}})\to H^{*+1}(\DASH;\Z_{2})\text{ corresp.\ to }f_i:K(\Z_{2^i},n)\to K(\Z_{2},n+1).\]
\begin{lem*}
The pullback construction mentioned above gives
\[\xymatrix{
%K(\pi,n+1)\ar[r]&{*}\ar[r]&K(\pi,n)
K(\Z_2,n)\ar[r]\ar@{=}[d]&K(\Z_{2^{i+1}},n)\ar[d]\ar[r]&K(\Z_{2^i},n)\ar[d]^{f_i}\\
K(\Z_2,n)\ar[r]&\star\ar[r]&K(\Z_2,n+1)
}\]
\end{lem*}
\begin{proof}
The total space is the only thing in doubt. It's a $K(G,n)$ by the LES. Suppose that $G$ was $\Z_2\times\Z_{2^i}$. Then the fibration would be trivial -- a Cartesian product. In this case, the transgression is zero, as the SSS has zero differentials from page 2 onwards. However, the map $f_i$ is not null, so corresponds to a nonzero cohomology class $\psi$, so the transgression is nonzero.
\end{proof}
\begin{lem*}[Bockstein Lemma]
Suppose that $\xymatrix{F\ar[r]^j&E\ar[r]^p&B}$ is a fibration, and that $u\in H^n(F)$ is transgressive. Suppose also that $v\in H^n(B)$ satisfies $d_i(v)=\tau(u)$. Then  $d_1(u)=j^*d_{i+1}p^*(v)$.
\[\begin{sseq}[leak=0mm,grid=none,entrysize=1cm,labelstep=1,
xlabels={{}_0;;{}_n;{}_{n+1}},ylabels={{}_0;;{}_n;{}_{n+1}}]%]
{0...3}{0...3}
\ssmoveto 2 0 \ssdrop{v}\ssname{20}
\ssmoveto 0 2 \ssdrop{u}\ssname{02}
\ssmoveto 0 3 \ssdrop{d_1(u)\makebox[0cm][l]{\,\,$=j^*d_{i+1}p^*(v)$}}\ssname{03}
\ssmoveto 3 0 \ssdrop{d_i(v)\makebox[0cm][l]{$\,\,=\tau(u)$}}\ssname{30}
\ssgoto{02} \ssgoto{03} \ssstroke[arrowto]
\ssgoto{02} \ssgoto{30} \ssstroke[arrowto]
\ssgoto{20} \ssgoto{30} \ssstroke[arrowto]
%\ssmoveto 3 2 \ssdrop{asdfasdfsafds}
\end{sseq}\]
\end{lem*}
\begin{proof}Neglected.\end{proof}
\subsection{Principal fiber spaces}
These are a generalisation of principal $G$-bundles. For any fibration $p:E\to B$, define $E^*$ to be the fiberwise cartesian square: the pullback of $p$ along $p$. Then $F\rightarrow E\downarrow B$ is a \emph{principal fiber space} if it is equipped with maps 
\[\varphi:E\times F\to E,\ \ h:E^*\to F.\]
Here $\varphi$ should be thought of as a fiber preserving action $F$ on $E$, and $h$ is a pairing on each fiber taking values in $F$. If we were actually talking about a principal $G$-bundle, then identifying $F$ with $G$, $\varphi$ is just the action, and $h(x_1,x_2)$ returns the element $g\in G$ such that $gx_1=x_2$.
\begin{itemise}
\item The action $\varphi$ respects the fibres.
\item The action restricted to $F\times F$ gives $F$ an H-space structure with strict unit and homotopy inverses.
\item The composite $\xymatrix{E^*\ar[r]^{(\pi_1,h)}&E\times F\ar[r]^{\varphi}&E}$ is homotopic to the second projection $\pi_2$.
\end{itemise}
For example, for any space $B$ let $X$ be the space of Moore paths over $B$. The action $\varphi$ is given by composing a loop with a path, and the map $h$ is given by sending $(e_1,e_2)$ to the path which follows $e_2$ then $e_1^{-1}$.
\begin{prop*}
Suppose $F\rightarrow E\downarrow B$ is a principal fibre space. Let $v,v':X\to E$ be two maps. Then $pv\simeq pv'$ iff there exists $w:X\to F$ such that $\varphi\circ(v,w)\simeq v'$:
\[\xymatrix@C=1.5cm{
X\ar@<.25ex>[r]^v\ar@<-.25ex>[r]_{v'}\ar@<.25ex>[dr]^(.63){pv}\ar@<-.25ex>[rd]_(.63){pv'}&E\ar[d]^p\\
&B
}\raisebox{-.6cm}{ \ Here, $pv\simeq pv'$ iff $\exists\, w$ s.t.\ \ }
\xymatrix{
&E\times F\ar[d]^\varphi\\
X\ar[ur]^{(v,w)}\ar[r]^{v'}&E
}\]
\end{prop*}

\section{Applications: Some homotopy groups of spheres}
\setcounter{subsection}{1}
\subsection{A better approximation to \texorpdfstring{$S^n$}{Sn}
 (and finding \texorpdfstring{$\pi_1^s$}{the stable 1-stem})}
Take $n$ large, for stability. View $K(\Z,n)$ as an approximation to $S^n$, and improve it, as follows. We know that $H^{n+1}(K(\Z,n);\Z_2)=0$ and that $H^{n+1}(K(\Z,n);\Z_2)=\Z_2\langle\Squ^2\imath_n\rangle$. This generator $\Squ^2\imath_n$ corresponds to a map $\Squ^2$ as drawn below:
\[\xymatrix{
%K(\pi,n+1)\ar[r]&{*}\ar[r]&K(\pi,n)
F_1\ar[r]\ar@{=}[d]&X_1\ar[d]\ar[r]&K(\Z,n)\makebox[0cm][l]{\,\,$=X_0$}\ar[d]^{\Squ^2}\\
K(\Z_2,n+1)\ar[r]&\star\ar[r]&K(\Z_2,n+2)
}\]
By (formula 1), in the SSS for the top row, $\imath_{n+1}$ transgresses to $\Squ^2\imath_n$. Moreover, by the compatibility of transgression with Steenrod squares, the generator $\Squ^1\imath_{n+1}$ of $H^{n+2}(F_1)$ transgresses to $\Squ^1\Squ^2\imath_{n}=\Squ^3\imath_{n}\neq0$.
We write down a part of the Serre exact sequence (with $\Z_2$ coefficients):
\[\xymatrix{
H^{n+1}(F_1)\ar@{->>}[r]^{\tau}&H^{n+2}(X_0)\ar[r]^{p^*}&H^{n+2}(X_1)\ar[r]^{i^*}&H^{n+2}(F_1)\ar@{^{(}->}[r]^{\tau}&H^{n+3}(X_0)
}\]
Thus $H^{n+2}(X_1;\Z_2)=0$.
\end{document}

















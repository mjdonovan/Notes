% !TEX root = z_output/_EquivStableHtpyHW.tex
%%%%%%%%%%%%%%%%%%%%%%%%%%%%%%%%%%%%%%%%%%%%%%%%%%%%%%%%%%%%%%%%%%%%%%%%%%%%%%%%
%%%%%%%%%%%%%%%%%%%%%%%%%%% 80 characters %%%%%%%%%%%%%%%%%%%%%%%%%%%%%%%%%%%%%%
%%%%%%%%%%%%%%%%%%%%%%%%%%%%%%%%%%%%%%%%%%%%%%%%%%%%%%%%%%%%%%%%%%%%%%%%%%%%%%%%
\documentclass[11pt]{article}
\usepackage{fullpage}
\usepackage{amsmath,amsthm,amssymb}
\usepackage{mathrsfs,nicefrac}
\usepackage{amssymb}
\usepackage{epsfig}
\usepackage[all,2cell]{xy}
\usepackage{sseq}
\usepackage{tocloft}
\usepackage{cancel}
\usepackage[strict]{changepage}
\usepackage{color}
\usepackage{tikz}
\usepackage{extpfeil}
\usepackage{version}
\usepackage{framed}
\definecolor{shadecolor}{rgb}{.925,0.925,0.925}

%\usepackage{ifthen}
%Used for disabling hyperref
\ifx\dontloadhyperref\undefined
%\usepackage[pdftex,pdfborder={0 0 0 [1 1]}]{hyperref}
\usepackage[pdftex,pdfborder={0 0 .5 [1 1]}]{hyperref}
\else
\providecommand{\texorpdfstring}[2]{#1}
\fi
%>>>>>>>>>>>>>>>>>>>>>>>>>>>>>>
%<<<        Versions        <<<
%>>>>>>>>>>>>>>>>>>>>>>>>>>>>>>
%Add in the following line to include all the versions.
%\def\excludeversion#1{\includeversion{#1}}

%>>>>>>>>>>>>>>>>>>>>>>>>>>>>>>
%<<<       Better ToC       <<<
%>>>>>>>>>>>>>>>>>>>>>>>>>>>>>>
\setlength{\cftbeforesecskip}{0.5ex}

%>>>>>>>>>>>>>>>>>>>>>>>>>>>>>>
%<<<      Hyperref mod      <<<
%>>>>>>>>>>>>>>>>>>>>>>>>>>>>>>

%needs more testing
\newcounter{dummyforrefstepcounter}
\newcommand{\labelRIGHTHERE}[1]
{\refstepcounter{dummyforrefstepcounter}\label{#1}}


%>>>>>>>>>>>>>>>>>>>>>>>>>>>>>>
%<<<  Theorem Environments  <<<
%>>>>>>>>>>>>>>>>>>>>>>>>>>>>>>
\ifx\dontloaddefinitionsoftheoremenvironments\undefined
\theoremstyle{plain}
\newtheorem{thm}{Theorem}[section]
\newtheorem*{thm*}{Theorem}
\newtheorem{lem}[thm]{Lemma}
\newtheorem*{lem*}{Lemma}
\newtheorem{prop}[thm]{Proposition}
\newtheorem*{prop*}{Proposition}
\newtheorem{cor}[thm]{Corollary}
\newtheorem*{cor*}{Corollary}
\newtheorem{defprop}[thm]{Definition-Proposition}
\newtheorem*{punchline}{Punchline}
\newtheorem*{conjecture}{Conjecture}
\newtheorem*{claim}{Claim}

\theoremstyle{definition}
\newtheorem{defn}{Definition}[section]
\newtheorem*{defn*}{Definition}
\newtheorem{exmp}{Example}[section]
\newtheorem*{exmp*}{Example}
\newtheorem*{exmps*}{Examples}
\newtheorem*{nonexmp*}{Non-example}
\newtheorem{asspt}{Assumption}[section]
\newtheorem{notation}{Notation}[section]
\newtheorem{exercise}{Exercise}[section]
\newtheorem*{fact*}{Fact}
\newtheorem*{rmk*}{Remark}
\newtheorem{fact}{Fact}
\newtheorem*{aside}{Aside}
\newtheorem*{question}{Question}
\newtheorem*{answer}{Answer}

\else\relax\fi

%>>>>>>>>>>>>>>>>>>>>>>>>>>>>>>
%<<<      Fields, etc.      <<<
%>>>>>>>>>>>>>>>>>>>>>>>>>>>>>>
\DeclareSymbolFont{AMSb}{U}{msb}{m}{n}
\DeclareMathSymbol{\N}{\mathbin}{AMSb}{"4E}
\DeclareMathSymbol{\Octonions}{\mathbin}{AMSb}{"4F}
\DeclareMathSymbol{\Z}{\mathbin}{AMSb}{"5A}
\DeclareMathSymbol{\R}{\mathbin}{AMSb}{"52}
\DeclareMathSymbol{\Q}{\mathbin}{AMSb}{"51}
\DeclareMathSymbol{\PP}{\mathbin}{AMSb}{"50}
\DeclareMathSymbol{\I}{\mathbin}{AMSb}{"49}
\DeclareMathSymbol{\C}{\mathbin}{AMSb}{"43}
\DeclareMathSymbol{\A}{\mathbin}{AMSb}{"41}
\DeclareMathSymbol{\F}{\mathbin}{AMSb}{"46}
\DeclareMathSymbol{\G}{\mathbin}{AMSb}{"47}
\DeclareMathSymbol{\Quaternions}{\mathbin}{AMSb}{"48}


%>>>>>>>>>>>>>>>>>>>>>>>>>>>>>>
%<<<       Operators        <<<
%>>>>>>>>>>>>>>>>>>>>>>>>>>>>>>
\DeclareMathOperator{\ad}{\textbf{ad}}
\DeclareMathOperator{\coker}{coker}
\renewcommand{\ker}{\textup{ker}\,}
\DeclareMathOperator{\End}{End}
\DeclareMathOperator{\Aut}{Aut}
\DeclareMathOperator{\Hom}{Hom}
\DeclareMathOperator{\Maps}{Maps}
\DeclareMathOperator{\Mor}{Mor}
\DeclareMathOperator{\Gal}{Gal}
\DeclareMathOperator{\Ext}{Ext}
\DeclareMathOperator{\Tor}{Tor}
\DeclareMathOperator{\Map}{Map}
\DeclareMathOperator{\Der}{Der}
\DeclareMathOperator{\Rad}{Rad}
\DeclareMathOperator{\rank}{rank}
\DeclareMathOperator{\ArfInvariant}{Arf}
\DeclareMathOperator{\KervaireInvariant}{Ker}
\DeclareMathOperator{\im}{im}
\DeclareMathOperator{\coim}{coim}
\DeclareMathOperator{\trace}{tr}
\DeclareMathOperator{\supp}{supp}
\DeclareMathOperator{\ann}{ann}
\DeclareMathOperator{\spec}{Spec}
\DeclareMathOperator{\SPEC}{\textbf{Spec}}
\DeclareMathOperator{\proj}{Proj}
\DeclareMathOperator{\PROJ}{\textbf{Proj}}
\DeclareMathOperator{\fiber}{F}
\DeclareMathOperator{\cofiber}{C}
\DeclareMathOperator{\cone}{cone}
\DeclareMathOperator{\skel}{sk}
\DeclareMathOperator{\coskel}{cosk}
\DeclareMathOperator{\conn}{conn}
\DeclareMathOperator{\colim}{colim}
\DeclareMathOperator{\limit}{lim}
\DeclareMathOperator{\ch}{ch}
\DeclareMathOperator{\Vect}{Vect}
\DeclareMathOperator{\GrthGrp}{GrthGp}
\DeclareMathOperator{\Sym}{Sym}
\DeclareMathOperator{\Prob}{\mathbb{P}}
\DeclareMathOperator{\Exp}{\mathbb{E}}
\DeclareMathOperator{\GeomMean}{\mathbb{G}}
\DeclareMathOperator{\Var}{Var}
\DeclareMathOperator{\Cov}{Cov}
\DeclareMathOperator{\Sp}{Sp}
\DeclareMathOperator{\Seq}{Seq}
\DeclareMathOperator{\Cyl}{Cyl}
\DeclareMathOperator{\Ev}{Ev}
\DeclareMathOperator{\sh}{sh}
\DeclareMathOperator{\intHom}{\underline{Hom}}
\DeclareMathOperator{\Frac}{frac}



%>>>>>>>>>>>>>>>>>>>>>>>>>>>>>>
%<<<   Cohomology Theories  <<<
%>>>>>>>>>>>>>>>>>>>>>>>>>>>>>>
\DeclareMathOperator{\KR}{{K\R}}
\DeclareMathOperator{\KO}{{KO}}
\DeclareMathOperator{\K}{{K}}
\DeclareMathOperator{\OmegaO}{{\Omega_{\Octonions}}}

%>>>>>>>>>>>>>>>>>>>>>>>>>>>>>>
%<<<   Algebraic Geometry   <<<
%>>>>>>>>>>>>>>>>>>>>>>>>>>>>>>
\DeclareMathOperator{\Spec}{Spec}
\DeclareMathOperator{\Proj}{Proj}
\DeclareMathOperator{\Sing}{Sing}
\DeclareMathOperator{\shfHom}{\mathscr{H}\textit{\!\!om}}
\DeclareMathOperator{\WeilDivisors}{{Div}}
\DeclareMathOperator{\CartierDivisors}{{CaDiv}}
\DeclareMathOperator{\PrincipalWeilDivisors}{{PrDiv}}
\DeclareMathOperator{\LocallyPrincipalWeilDivisors}{{LPDiv}}
\DeclareMathOperator{\PrincipalCartierDivisors}{{PrCaDiv}}
\DeclareMathOperator{\DivisorClass}{{Cl}}
\DeclareMathOperator{\CartierClass}{{CaCl}}
\DeclareMathOperator{\Picard}{{Pic}}
\DeclareMathOperator{\Frob}{Frob}


%>>>>>>>>>>>>>>>>>>>>>>>>>>>>>>
%<<<  Mathematical Objects  <<<
%>>>>>>>>>>>>>>>>>>>>>>>>>>>>>>
\newcommand{\sll}{\mathfrak{sl}}
\newcommand{\gl}{\mathfrak{gl}}
\newcommand{\GL}{\mbox{GL}}
\newcommand{\PGL}{\mbox{PGL}}
\newcommand{\SL}{\mbox{SL}}
\newcommand{\Mat}{\mbox{Mat}}
\newcommand{\Gr}{\textup{Gr}}
\newcommand{\Squ}{\textup{Sq}}
\newcommand{\catSet}{\textit{Sets}}
\newcommand{\RP}{{\R\PP}}
\newcommand{\CP}{{\C\PP}}
\newcommand{\Steen}{\mathscr{A}}
\newcommand{\Orth}{\textup{\textbf{O}}}

%>>>>>>>>>>>>>>>>>>>>>>>>>>>>>>
%<<<  Mathematical Symbols  <<<
%>>>>>>>>>>>>>>>>>>>>>>>>>>>>>>
\newcommand{\DASH}{\textup{---}}
\newcommand{\op}{\textup{op}}
\newcommand{\CW}{\textup{CW}}
\newcommand{\ob}{\textup{ob}\,}
\newcommand{\ho}{\textup{ho}}
\newcommand{\st}{\textup{st}}
\newcommand{\id}{\textup{id}}
\newcommand{\Bullet}{\ensuremath{\bullet} }
\newcommand{\sprod}{\wedge}

%>>>>>>>>>>>>>>>>>>>>>>>>>>>>>>
%<<<      Some Arrows       <<<
%>>>>>>>>>>>>>>>>>>>>>>>>>>>>>>
\newcommand{\nt}{\Longrightarrow}
\let\shortmapsto\mapsto
\let\mapsto\longmapsto
\newcommand{\mapsfrom}{\,\reflectbox{$\mapsto$}\ }
\newcommand{\bigrightsquig}{\scalebox{2}{\ensuremath{\rightsquigarrow}}}
\newcommand{\bigleftsquig}{\reflectbox{\scalebox{2}{\ensuremath{\rightsquigarrow}}}}

%\newcommand{\cofibration}{\xhookrightarrow{\phantom{\ \,{\sim\!}\ \ }}}
%\newcommand{\fibration}{\xtwoheadrightarrow{\phantom{\sim\!}}}
%\newcommand{\acycliccofibration}{\xhookrightarrow{\ \,{\sim\!}\ \ }}
%\newcommand{\acyclicfibration}{\xtwoheadrightarrow{\sim\!}}
%\newcommand{\leftcofibration}{\xhookleftarrow{\phantom{\ \,{\sim\!}\ \ }}}
%\newcommand{\leftfibration}{\xtwoheadleftarrow{\phantom{\sim\!}}}
%\newcommand{\leftacycliccofibration}{\xhookleftarrow{\ \ {\sim\!}\,\ }}
%\newcommand{\leftacyclicfibration}{\xtwoheadleftarrow{\sim\!}}
%\newcommand{\weakequiv}{\xrightarrow{\ \,\sim\,\ }}
%\newcommand{\leftweakequiv}{\xleftarrow{\ \,\sim\,\ }}

\newcommand{\cofibration}
{\xhookrightarrow{\phantom{\ \,{\raisebox{-.3ex}[0ex][0ex]{\scriptsize$\sim$}\!}\ \ }}}
\newcommand{\fibration}
{\xtwoheadrightarrow{\phantom{\raisebox{-.3ex}[0ex][0ex]{\scriptsize$\sim$}\!}}}
\newcommand{\acycliccofibration}
{\xhookrightarrow{\ \,{\raisebox{-.55ex}[0ex][0ex]{\scriptsize$\sim$}\!}\ \ }}
\newcommand{\acyclicfibration}
{\xtwoheadrightarrow{\raisebox{-.6ex}[0ex][0ex]{\scriptsize$\sim$}\!}}
\newcommand{\leftcofibration}
{\xhookleftarrow{\phantom{\ \,{\raisebox{-.3ex}[0ex][0ex]{\scriptsize$\sim$}\!}\ \ }}}
\newcommand{\leftfibration}
{\xtwoheadleftarrow{\phantom{\raisebox{-.3ex}[0ex][0ex]{\scriptsize$\sim$}\!}}}
\newcommand{\leftacycliccofibration}
{\xhookleftarrow{\ \ {\raisebox{-.55ex}[0ex][0ex]{\scriptsize$\sim$}\!}\,\ }}
\newcommand{\leftacyclicfibration}
{\xtwoheadleftarrow{\raisebox{-.6ex}[0ex][0ex]{\scriptsize$\sim$}\!}}
\newcommand{\weakequiv}
{\xrightarrow{\ \,\raisebox{-.3ex}[0ex][0ex]{\scriptsize$\sim$}\,\ }}
\newcommand{\leftweakequiv}
{\xleftarrow{\ \,\raisebox{-.3ex}[0ex][0ex]{\scriptsize$\sim$}\,\ }}

%>>>>>>>>>>>>>>>>>>>>>>>>>>>>>>
%<<<    xymatrix Arrows     <<<
%>>>>>>>>>>>>>>>>>>>>>>>>>>>>>>
\newdir{ >}{{}*!/-5pt/@{>}}
\newcommand{\xycof}{\ar@{ >->}}
\newcommand{\xycofib}{\ar@{^{(}->}}
\newcommand{\xycofibdown}{\ar@{_{(}->}}
\newcommand{\xyfib}{\ar@{->>}}
\newcommand{\xymapsto}{\ar@{|->}}

%>>>>>>>>>>>>>>>>>>>>>>>>>>>>>>
%<<<     Greek Letters      <<<
%>>>>>>>>>>>>>>>>>>>>>>>>>>>>>>
%\newcommand{\oldphi}{\phi}
%\renewcommand{\phi}{\varphi}
\let\oldphi\phi
\let\phi\varphi
\renewcommand{\to}{\longrightarrow}
\newcommand{\from}{\longleftarrow}
\newcommand{\eps}{\varepsilon}

%>>>>>>>>>>>>>>>>>>>>>>>>>>>>>>
%<<<  1st-4th & parentheses <<<
%>>>>>>>>>>>>>>>>>>>>>>>>>>>>>>
\newcommand{\first}{^\text{st}}
\newcommand{\second}{^\text{nd}}
\newcommand{\third}{^\text{rd}}
\newcommand{\fourth}{^\text{th}}
\newcommand{\ZEROTH}{$0^\text{th}$ }
\newcommand{\FIRST}{$1^\text{st}$ }
\newcommand{\SECOND}{$2^\text{nd}$ }
\newcommand{\THIRD}{$3^\text{rd}$ }
\newcommand{\FOURTH}{$4^\text{th}$ }
\newcommand{\iTH}{$i^\text{th}$ }
\newcommand{\jTH}{$j^\text{th}$ }
\newcommand{\nTH}{$n^\text{th}$ }

%>>>>>>>>>>>>>>>>>>>>>>>>>>>>>>
%<<<    upright commands    <<<
%>>>>>>>>>>>>>>>>>>>>>>>>>>>>>>
\newcommand{\upcol}{\textup{:}}
\newcommand{\upsemi}{\textup{;}}
\providecommand{\lparen}{\textup{(}}
\providecommand{\rparen}{\textup{)}}
\renewcommand{\lparen}{\textup{(}}
\renewcommand{\rparen}{\textup{)}}
\newcommand{\Iff}{\emph{iff} }

%>>>>>>>>>>>>>>>>>>>>>>>>>>>>>>
%<<<     Environments       <<<
%>>>>>>>>>>>>>>>>>>>>>>>>>>>>>>
\newcommand{\squishlist}
{ %\setlength{\topsep}{100pt} doesn't seem to do anything.
  \setlength{\itemsep}{.5pt}
  \setlength{\parskip}{0pt}
  \setlength{\parsep}{0pt}}
\newenvironment{itemise}{
\begin{list}{\textup{$\rightsquigarrow$}}
   {  \setlength{\topsep}{1mm}
      \setlength{\itemsep}{1pt}
      \setlength{\parskip}{0pt}
      \setlength{\parsep}{0pt}
   }
}{\end{list}\vspace{-.1cm}}
\newcommand{\INDENT}{\textbf{}\phantom{space}}
\renewcommand{\INDENT}{\rule{.7cm}{0cm}}

\newcommand{\itm}[1][$\rightsquigarrow$]{\item[{\makebox[.5cm][c]{\textup{#1}}}]}


%\newcommand{\rednote}[1]{{\color{red}#1}\makebox[0cm][l]{\scalebox{.1}{rednote}}}
%\newcommand{\bluenote}[1]{{\color{blue}#1}\makebox[0cm][l]{\scalebox{.1}{rednote}}}

\newcommand{\rednote}[1]
{{\color{red}#1}\makebox[0cm][l]{\scalebox{.1}{\rotatebox{90}{?????}}}}
\newcommand{\bluenote}[1]
{{\color{blue}#1}\makebox[0cm][l]{\scalebox{.1}{\rotatebox{90}{?????}}}}


\newcommand{\funcdef}[4]{\begin{align*}
#1&\to #2\\
#3&\mapsto#4
\end{align*}}

%\newcommand{\comment}[1]{}

%>>>>>>>>>>>>>>>>>>>>>>>>>>>>>>
%<<<       Categories       <<<
%>>>>>>>>>>>>>>>>>>>>>>>>>>>>>>
\newcommand{\Ens}{{\mathscr{E}ns}}
\DeclareMathOperator{\Sheaves}{{\mathsf{Shf}}}
\DeclareMathOperator{\Presheaves}{{\mathsf{PreShf}}}
\DeclareMathOperator{\Psh}{{\mathsf{Psh}}}
\DeclareMathOperator{\Shf}{{\mathsf{Shf}}}
\DeclareMathOperator{\Varieties}{{\mathsf{Var}}}
\DeclareMathOperator{\Schemes}{{\mathsf{Sch}}}
\DeclareMathOperator{\Rings}{{\mathsf{Rings}}}
\DeclareMathOperator{\AbGp}{{\mathsf{AbGp}}}
\DeclareMathOperator{\Modules}{{\mathsf{\!-Mod}}}
\DeclareMathOperator{\fgModules}{{\mathsf{\!-Mod}^{\textup{fg}}}}
\DeclareMathOperator{\QuasiCoherent}{{\mathsf{QCoh}}}
\DeclareMathOperator{\Coherent}{{\mathsf{Coh}}}
\DeclareMathOperator{\GSW}{{\mathcal{SW}^G}}
\DeclareMathOperator{\Burnside}{{\mathsf{Burn}}}
\DeclareMathOperator{\GSet}{{G\mathsf{Set}}}
\DeclareMathOperator{\FinGSet}{{G\mathsf{Set}^\textup{fin}}}
\DeclareMathOperator{\HSet}{{H\mathsf{Set}}}
\DeclareMathOperator{\Cat}{{\mathsf{Cat}}}
\DeclareMathOperator{\Fun}{{\mathsf{Fun}}}
\DeclareMathOperator{\Orb}{{\mathsf{Orb}}}
\DeclareMathOperator{\Set}{{\mathsf{Set}}}
\DeclareMathOperator{\sSet}{{\mathsf{sSet}}}
\DeclareMathOperator{\Top}{{\mathsf{Top}}}
\DeclareMathOperator{\GSpectra}{{G-\mathsf{Spectra}}}
\DeclareMathOperator{\Lan}{Lan}
\DeclareMathOperator{\Ran}{Ran}

%>>>>>>>>>>>>>>>>>>>>>>>>>>>>>>
%<<<     Script Letters     <<<
%>>>>>>>>>>>>>>>>>>>>>>>>>>>>>>
\newcommand{\scrQ}{\mathscr{Q}}
\newcommand{\scrW}{\mathscr{W}}
\newcommand{\scrE}{\mathscr{E}}
\newcommand{\scrR}{\mathscr{R}}
\newcommand{\scrT}{\mathscr{T}}
\newcommand{\scrY}{\mathscr{Y}}
\newcommand{\scrU}{\mathscr{U}}
\newcommand{\scrI}{\mathscr{I}}
\newcommand{\scrO}{\mathscr{O}}
\newcommand{\scrP}{\mathscr{P}}
\newcommand{\scrA}{\mathscr{A}}
\newcommand{\scrS}{\mathscr{S}}
\newcommand{\scrD}{\mathscr{D}}
\newcommand{\scrF}{\mathscr{F}}
\newcommand{\scrG}{\mathscr{G}}
\newcommand{\scrH}{\mathscr{H}}
\newcommand{\scrJ}{\mathscr{J}}
\newcommand{\scrK}{\mathscr{K}}
\newcommand{\scrL}{\mathscr{L}}
\newcommand{\scrZ}{\mathscr{Z}}
\newcommand{\scrX}{\mathscr{X}}
\newcommand{\scrC}{\mathscr{C}}
\newcommand{\scrV}{\mathscr{V}}
\newcommand{\scrB}{\mathscr{B}}
\newcommand{\scrN}{\mathscr{N}}
\newcommand{\scrM}{\mathscr{M}}

%>>>>>>>>>>>>>>>>>>>>>>>>>>>>>>
%<<<     Fractur Letters    <<<
%>>>>>>>>>>>>>>>>>>>>>>>>>>>>>>
\newcommand{\frakQ}{\mathfrak{Q}}
\newcommand{\frakW}{\mathfrak{W}}
\newcommand{\frakE}{\mathfrak{E}}
\newcommand{\frakR}{\mathfrak{R}}
\newcommand{\frakT}{\mathfrak{T}}
\newcommand{\frakY}{\mathfrak{Y}}
\newcommand{\frakU}{\mathfrak{U}}
\newcommand{\frakI}{\mathfrak{I}}
\newcommand{\frakO}{\mathfrak{O}}
\newcommand{\frakP}{\mathfrak{P}}
\newcommand{\frakA}{\mathfrak{A}}
\newcommand{\frakS}{\mathfrak{S}}
\newcommand{\frakD}{\mathfrak{D}}
\newcommand{\frakF}{\mathfrak{F}}
\newcommand{\frakG}{\mathfrak{G}}
\newcommand{\frakH}{\mathfrak{H}}
\newcommand{\frakJ}{\mathfrak{J}}
\newcommand{\frakK}{\mathfrak{K}}
\newcommand{\frakL}{\mathfrak{L}}
\newcommand{\frakZ}{\mathfrak{Z}}
\newcommand{\frakX}{\mathfrak{X}}
\newcommand{\frakC}{\mathfrak{C}}
\newcommand{\frakV}{\mathfrak{V}}
\newcommand{\frakB}{\mathfrak{B}}
\newcommand{\frakN}{\mathfrak{N}}
\newcommand{\frakM}{\mathfrak{M}}

\newcommand{\frakq}{\mathfrak{q}}
\newcommand{\frakw}{\mathfrak{w}}
\newcommand{\frake}{\mathfrak{e}}
\newcommand{\frakr}{\mathfrak{r}}
\newcommand{\frakt}{\mathfrak{t}}
\newcommand{\fraky}{\mathfrak{y}}
\newcommand{\fraku}{\mathfrak{u}}
\newcommand{\fraki}{\mathfrak{i}}
\newcommand{\frako}{\mathfrak{o}}
\newcommand{\frakp}{\mathfrak{p}}
\newcommand{\fraka}{\mathfrak{a}}
\newcommand{\fraks}{\mathfrak{s}}
\newcommand{\frakd}{\mathfrak{d}}
\newcommand{\frakf}{\mathfrak{f}}
\newcommand{\frakg}{\mathfrak{g}}
\newcommand{\frakh}{\mathfrak{h}}
\newcommand{\frakj}{\mathfrak{j}}
\newcommand{\frakk}{\mathfrak{k}}
\newcommand{\frakl}{\mathfrak{l}}
\newcommand{\frakz}{\mathfrak{z}}
\newcommand{\frakx}{\mathfrak{x}}
\newcommand{\frakc}{\mathfrak{c}}
\newcommand{\frakv}{\mathfrak{v}}
\newcommand{\frakb}{\mathfrak{b}}
\newcommand{\frakn}{\mathfrak{n}}
\newcommand{\frakm}{\mathfrak{m}}

%>>>>>>>>>>>>>>>>>>>>>>>>>>>>>>
%<<<  Caligraphic Letters   <<<
%>>>>>>>>>>>>>>>>>>>>>>>>>>>>>>
\newcommand{\calQ}{\mathcal{Q}}
\newcommand{\calW}{\mathcal{W}}
\newcommand{\calE}{\mathcal{E}}
\newcommand{\calR}{\mathcal{R}}
\newcommand{\calT}{\mathcal{T}}
\newcommand{\calY}{\mathcal{Y}}
\newcommand{\calU}{\mathcal{U}}
\newcommand{\calI}{\mathcal{I}}
\newcommand{\calO}{\mathcal{O}}
\newcommand{\calP}{\mathcal{P}}
\newcommand{\calA}{\mathcal{A}}
\newcommand{\calS}{\mathcal{S}}
\newcommand{\calD}{\mathcal{D}}
\newcommand{\calF}{\mathcal{F}}
\newcommand{\calG}{\mathcal{G}}
\newcommand{\calH}{\mathcal{H}}
\newcommand{\calJ}{\mathcal{J}}
\newcommand{\calK}{\mathcal{K}}
\newcommand{\calL}{\mathcal{L}}
\newcommand{\calZ}{\mathcal{Z}}
\newcommand{\calX}{\mathcal{X}}
\newcommand{\calC}{\mathcal{C}}
\newcommand{\calV}{\mathcal{V}}
\newcommand{\calB}{\mathcal{B}}
\newcommand{\calN}{\mathcal{N}}
\newcommand{\calM}{\mathcal{M}}

%>>>>>>>>>>>>>>>>>>>>>>>>>>>>>>
%<<<<<<<<<DEPRECIATED<<<<<<<<<<
%>>>>>>>>>>>>>>>>>>>>>>>>>>>>>>

%%% From Kac's template
% 1-inch margins, from fullpage.sty by H.Partl, Version 2, Dec. 15, 1988.
%\topmargin 0pt
%\advance \topmargin by -\headheight
%\advance \topmargin by -\headsep
%\textheight 9.1in
%\oddsidemargin 0pt
%\evensidemargin \oddsidemargin
%\marginparwidth 0.5in
%\textwidth 6.5in
%
%\parindent 0in
%\parskip 1.5ex
%%\renewcommand{\baselinestretch}{1.25}

%%% From the net
%\newcommand{\pullbackcorner}[1][dr]{\save*!/#1+1.2pc/#1:(1,-1)@^{|-}\restore}
%\newcommand{\pushoutcorner}[1][dr]{\save*!/#1-1.2pc/#1:(-1,1)@^{|-}\restore}










\newcommand{\NewLecture}[2]{\section*{#1 {\small(due #2)}}}
\newcommand{\Extracurricular}[1]{
\section*{#1 {\small(extracurricular)}}
}
%\makeatletter
%\def\@seccntformat#1{Problem Set \csname the#1\endcsname.\quad}
%\makeatother
\title{Equivariant Stable Homotopy Theory --- Math 289x}
\author{Michael Donovan}
\date{}

\begin{document}
\maketitle

\NewLecture{Problem Set 1}{on the 3$^\text{rd}$ of October}
\setcounter{section}{1}
\subsection*{Problem 1} A standard proof that any map $f:X\to Y$ of CW-complexes is homotopic to a cellular map runs as follows:

It is enough to prove, for any CW-complex $X$, and any map $f_n:X\to Y$ whose restriction to the $(n-1)$-skeleton $X^{n-1}$ is cellular, that:
\[\text{ $f_{n-1}|_{X^n}$ is homotopic (rel $X^{n-1}$) to a cellular map $g:X^n\to Y$.} \tag{$\star$}\]
For then, by the homotopy extension property for $(X,X^n)$, this homotopy extends to one on all of $X$, so we can produce a map $f_n$ homotopic to $f_{n-1}$ (rel $X^{n-1}$) which is cellular on restriction to $X^n$. Starting with $f_{-1}=f$, and repeating this process, we could homotope $f$ repeatedly to make it cellular on higher and higher skeleta, with homotopies fixing higher and higher skeleta. As every point of $x$ lies in some skeleton, and thus eventually stabilises under this sequence of homotopies, we can fit all of these homotopies into the interval $[0,1]$, to homotope $f$ into a cellular map.

Now to show ($\star$),
%that $f|_{X^n}$ is homotopic (rel $X^{n-1}$) to a cellular map (as in the first sentence above), 
one needs only see that:
\[\tag{$\star\star$}
\text{any map $\gamma:D^n\to Y$ sending $\partial D^n$ into $Y^{n-1}$ is homotopic (rel $S^n$) to a map into $Y^n$}.
\]
% any map $\gamma:D^n\to Y$ sending $\partial D^n$ into $Y^{n-1}$ is homotopic (rel $S^n$) to a map into $Y^n$. 
This will provide a homotopy for each $n$-cell of $X$ which fixes $X^{n-1}$. Performing all these homotopies at once gives the homotopy needed for $(\star)$.

In the equivariant context, all cells $D^n$ are swapped for cells\footnote{This convention on cells is made to make what's coming seem simpler.} of the form $G/H\times D^n$ for $H$ a subgroup of $G$, and all attaching maps must be equivariant. The following observations are enough to adapt the proof to $G$-CW-complexes:
\begin{itemize}\squishlist
\item We need to see that there is a $G$-equivariant homotopy extension property for the pair $(X,X^n)$. In fact, to do this, it is enough to see that there is a $G$-equivariant HEP for a pair $(A\cup (G/H\times D^n),A)$, where $A$ is any $G$-space to which we are attaching an $n$-cell. This is the case, as the usual retraction of $D^n\times I$ onto $(D^n\times\{0\})\cup(S^{n-1}\times I)$, used in the proof of the nonequivariant HEP for such a pair (in Hatcher's book, p\. 14) adapts to the $G$-equivariant context simply by applying $G/H\times(\DASH)$.
\item We need to see that $(\star\star)$ still holds, when $D^n$ is replaced by an equivariant cell, and $Y$ is a $G$-CW-complex. That is, we should see that any $\gamma:G/H\times D^n\to Y$ such that $\gamma(G/H\times\partial D^n)\subseteq Y^{n-1}$ is homotopic (rel $S\times S^n$) to a map into $Y^n$. However, as equivariant maps $G/H\times D^n\to Y$ are equivalent to nonequivariant maps $D^n\to Y^H$, this follows from $(\star\star)$.
\end{itemize}
\subsection*{Problem 2(a)}
The solution given here was suggested by Michael Andrews. Let $G=\Z_2$, and let $A:\FinGSet\to \AbGp$ be defined on objects by $\Z_2\mapsto\Z$ and $*\mapsto0$. Define the maps on morphisms by sending both of the endomorphisms of $\Z_2$ to the identity. Extend these definitions by additivity. This does not extend to a Mackey functor.

To see this, suppose that it did extend to a Mackey functor. Then by definition, the pullback square (at left) induces a commuting diagram (at right) commutes:
\[\xymatrix@!0@R=12mm@C=1.5cm{
\ar[d]_{\id\sqcup\id}G\sqcup G\ar[r]^j&G\ar[d]^p\\
G\ar[r]^i&\ast
}\qquad\qquad\xymatrix@!0@R=12mm@C=2.05cm{
\ar[d];[]^{q^*}\Z^2\ar[r]^{j_*}&\Z\ar[d];[]_{p^*}\\
\Z\ar[r]^{i_*}&0
}\]
But $j_*$ is the sum $\Z^2\to\Z$, so that $q^*$ (which factors through the diagonal $\Z\to\Z^2$) must be zero. Thus the contravariant part of the Mackey functor sends every morphism to zero, and so the identity roof $\Z_2=\Z_2=\Z_2$ is sent to zero, not the identity on $\Z$, as it must be.
\subsubsection*{A note on the cohomology of a $G$-CW-complex (relevant to 2(b) and 3)}
Suppose that $X$ is a $G$-CW-complex built by attaching $n$-cells $\vartheta^n_+\wedge D^n$ along an equivariant map $\alpha^n$:
\[\xymatrix{
\vartheta^n_+\wedge S^{n-1}\ar[d]\ar[r]^{\quad\alpha^n}&X^{n-1}\ar[d]\\
\vartheta^n_+\wedge D^{n}\ar[r]&X^{n}
}\]
I'm going to assume that the $G$-sets $\vartheta^p$ are finite, at least for the time being --- I don't really know yet how this should extend to when they are not.
We have a sequence of $G$-CW-inclusions, and thus a sequence of cofibre sequences:
\[\xymatrix@!0@C=1.6cm@R=1.28cm{%@C=.5cm{
\makebox[0cm][r]{$\cdots$\
}\ast\ar[rr]&&X^0\ar[rr]\ar[ld]&&X^1\ar[rr]
\ar[ld]&&X^2\ar[ld]
\ar[rr]&&X^3\makebox[0cm][l]{\ $\cdots$}\ar[ld]\\
%&\cdots\ar[r]&h_*(E)\\
&X^0&&%\ar@{~>}[ul]&&
X^1/X^0&&%\ar@{~>}[lu]&&
X^2/X^1&&%\ar@{~>}[lu]&&
X^3/X^2%\ar@{~>}[lu]
}\]
If we apply the functor $\oplus_{*}\{\DASH,K(M,*)\}^G$ to this diagram, we obtain an exact couple:
\[\xymatrix@!0@C=1.6cm@R=1.28cm{%@C=.5cm{
\makebox[0cm][r]{$\cdots$\
}\widetilde H^*(\ast;M)\ar[rr];[]&&\widetilde H^*(X^0;M)\ar[rr];[]\ar[ld];[]&&\widetilde H^*(X^1;M)\ar[rr];[]
\ar[ld];[]&&\widetilde H^*(X^2;M)\ar[ld];[]
\ar[rr];[]&&\widetilde H^*(X^3;M)\makebox[0cm][l]{\ $\cdots$}\ar[ld];[]\\
%&\cdots\ar[r];[]&h_*(E)\\
&\widetilde H^*(X^0;M)\ar@{~>}[ul];[]&&
\widetilde H^*(X^1/X^0;M)\ar@{~>}[lu];[]&&
\widetilde H^*(X^2/X^1;M)\ar@{~>}[lu];[]&&
\widetilde H^*(X^3/X^2;M)\ar@{~>}[lu];[]
}\]
Here the wiggly arrows raise degree by one, and each triangle is exact at each vertex. Such a structure induces a spectral sequence:
\[E_1^{pq}:=\widetilde H^{p+q}(X^p/X^{p-1};M)\implies\widetilde H^*(X;M).\]
Now $X^p/X^{p-1}=\vartheta^p_+\wedge S^p$, and by definition (of $K(n,M)$), this has cohomology group $M(\vartheta^p)$ in dimension $p$, and zero elsewhere. Thus $E_1^{pq}=0$ when $q\neq0$, and the spectral sequence will collapse by $E_2$. The $d_1$ differential $M(\vartheta^p)\to M(\vartheta^{p+1})$, (corresponding to following the maps north-east then south-east), as in the standard non-equivariant calculation, is the same as the map $\widetilde H^p(\vartheta^{p}_+\wedge S^{p})\to\widetilde H^p(\vartheta^{p+1}_+\wedge S^{p})$ induced by the composite:
\[c_p:\left\{\vartheta_+^{p+1}\wedge S^p\overset{\alpha^{p+1}}{\to} X^p\to X^p/X^{p-1}\simeq\vartheta_+^{p}\wedge S^p\right\}.\]
This composite, $c_p$ is an unstable map, and as such, cannot be just any map in $\{\vartheta_+^{p+1}\wedge S^p,\vartheta_+^{p}\wedge S^p\}^G\cong\Burnside_G(\vartheta^{p+1},\vartheta^p)$. Writing $\vartheta^{p+1}=\coprod_{i\in I}G/H_i$, we have:
\begin{alignat*}{2}
[\vartheta^{p+1}_+\wedge S^p,\vartheta^p_+\wedge S^p]^G&=\prod_{i\in I}[(G/H_i)_+\wedge S^p,\vartheta^p_+\wedge S^p]^G\\
&=\prod_{i\in I}[S^p,(\vartheta^p)^{H_i}_+\wedge S^p]\\
&=\prod_{i\in I}\bigoplus_{t\in(\vartheta^p)^{H_i}}[S^p,\{t\}_+\wedge S^p]\\
&=\prod_{i\in I}\bigoplus_{t\in(\vartheta^p)^{H_i}}\Z
\end{alignat*}
In particular, the possible maps $c_p$, which can be viewed as elements of $\Burnside_G(\vartheta^{p+1},\vartheta^p)$, all lie in the $\Z$-span of the image of $\FinGSet(\vartheta^{p+1},\vartheta^p)\to\Burnside_G(\vartheta^{p+1},\vartheta^p)$. This is why we can define cohomology with coefficients in a coefficient system, not only with coefficients in a Mackey functor. Another way to say this is that all the roofs which appear when writing $c_p$ as a morphism in the Burnside category can be taken to have the form:
\[\xymatrix@R=5mm{&\vartheta^{p+1}\ar@{=}[dl]\ar[dr]^\delta\\
\vartheta^{p+1}&&\vartheta^p}\]
The morphisms involved are $\Z$-linear combinations of these.
\subsection*{Problem 2(b)}
We can calculate directly the morphisms $\Burnside_G(S,G)$. Given a $G$-equivariant roof
\[\xymatrix@R=5mm{&U'\ar[dl]_\gamma\ar[dr]^\delta\\S&&G}\]
one knows that $U'$ must be a disjoint union of free $G$-sets. Up to equivalence, the roof is of the form
\[\xymatrix@R=5mm{&\coprod_{i=1}^rG\ar[dl]\ar[dr]^{\coprod(\id)}\\S&&G}\]
Now an equivariant map $G\to S$ is just a choice of an element of $S$, yielding an isomorphism $\Burnside_G(S,G)\simeq\Z S$, the free abelian group on generators the elements of $S$. To make this more explicit, in the first diagram, let $\{u_1,\ldots, u_n\}=\delta^{-1}(1)$. Then the element of $\Z S$ corresponding to this roof is $\sum_{i=1}^n\gamma(u_i)$.

Now we should consider the map $\Z\vartheta^{p}\to\Z\vartheta^{p+1}$ induced by a roof: $\xymatrix{\vartheta^{p+1}&\ar[l]_=\vartheta^{p+1}\ar[r]^\delta&\vartheta^p}$\!. We note easily that $t\in\vartheta^p$ is mapped to the sum of the elements of $\delta^{-1}(t)$, and with this observation, the map is determined by $\Z$-linearity. This agrees exactly with the map in the non-equivariant integral cellular reduced cochain complex, concluding the proof.
%Next we should investigate the map $\Z S\to\Z T$ induced by a roof
%\[\xymatrix@R=5mm{&U\ar[dl]_\alpha\ar[dr]^\beta\\T&&S}\]
%To pull back a generator $s\in S$ of $\Z S$ along this morphism is to form a pullback $P$:
%\[\xymatrix@R=5mm{
%&&P\ar[dl]\ar[dr]\\
%&U\ar[dl]_\alpha\ar[dr]^\beta&&G\ar[dl]^{1\shortmapsto s}\ar[dr]\\
%T&&S&&G
%}\]
%The elements of the pullback $P$ which map to $\id\in G$ are in bijection with $\beta^{-1}(s)$.
\subsection*{Problem 3: A $G$-CW-complex structure on $S^{n\sigma}$}
We'll give a cell structure on $S^{n\sigma}$, being careful about getting the right convention with the $0$-cell. In fact, the best convention is to view the basepoint as a $(-\infty)$-dimensional cell, and that the $0$-skeleton of $S^{n\sigma}$, a pair of fixed points, is formed from the $(-1)$-skeleton by attaching $\ast_+\wedge D^0=D^0$ along the empty set, its boundary.
%$\ast_+\wedge \emptyset$, the basepoint of $D_0$. 
This convention best fits the fact that the $E_1^{00}$ term $\widetilde H^0(X^0;M)$ in the above spectral sequence is \emph{reduced}, and the fact that in the filtration of $X$ written above, the basepoint is in arbitrarily negative skeleta. For this insight, I thank Michael Andrews.

There is a cell structure on $S^{n\sigma}$ with
\[\vartheta^i=\begin{cases} *;&i=0;\\\Z_2;&1\leq i\leq n;\\\emptyset;&\text{otherwise}.\end{cases}\]
We name all the maps we'll need:
\begin{alignat*}{2}
\imath&:\Z_2\to\Z_2&\quad&\text{the identity}\\
\sigma&:\Z_2\to\Z_2&\quad&\text{interchanging the two points}\\
\pi&:\Z_2\to *&\quad&\text{the unique map}
\end{alignat*}
With this terminology, we can view these as maps in $\Burnside_G$ (or in $\Orb_G$), and then write down linear combinations thereof which induce the equivariant cellular cochain complex for $S^{n\sigma}$:
\[\xymatrix@C=2cm{
\Z_2\ar[r]^{\imath+(-1)^{n-1}\sigma}&
\Z_2\ar[r]^{\imath+(-1)^{n-2}\sigma}&
\Z_2\ar[r]|{\ \cdots\cdots\ }&
\Z_2\ar[r]^{\imath+\sigma}&
\Z_2\ar[r]^{\imath-\sigma}&
\Z_2\ar[r]^\pi&\ast
}\]
Thus, to calculate the reduced cohomology with coefficients in $M$, it is enough to calculate $M(\Z_2)$, $M(*)$, $\imath^*$, $\sigma^*$ and $\pi^*$. We tabulate:
\[\begin{array}{c|ccccc}
M&M(\Z_2)&M(*)&\imath^*&\sigma^*&\pi^*\\\hline
\underline\Z&\Z&\Z&\id&\id&\id\\
R&\Z&\Z^2&\id&\id&(a,b)\shortmapsto a+b\\
A&\Z&\Z^2&\id&\id&(a,b)\shortmapsto 2a+b
\end{array}\]
These claims are obvious for $\underline\Z$. For $R$, it's a bit trickier. A $\Z_2$-equivariant vector bundle over $\Z_2$ is determined by its dimension. On the other hand, there are two non-isomorphic $\Z_2$-equivariant vector bundles over $*$, which generate the Grothendieck group $\Z^2$. This explains the second line.

For $A$, we find that $A(\Z_2)=\Burnside_G(\Z_2,*)$ is infinite cyclic generated by $\Z_2\overset{\imath}{\longleftarrow}\Z_2\overset{\pi}{\longrightarrow}*$, while $A(*)=\Burnside_G(*,*)$ is free abelian generated by $*\overset{\pi}{\longleftarrow}\Z_2\overset{\pi}{\longrightarrow}*$ and $*\longleftarrow*\longrightarrow*$. Then $\imath^*$ are $\sigma^*$ are easy, while the formula for $\pi^*$ is justified by calculating the following composites:
\[\xymatrix@R=4mm@C=.5cm{
&&\Z_2^2\ar[dl]\ar[dr]\\
&\Z_2\ar@{=}[dl]\ar[dr]&&\Z_2\ar[dl]\ar[dr]\\
\Z_2&&\ast&&\ast
}\qquad\qquad
\xymatrix@R=4mm@C=.5cm{
&&\Z_2\ar[dl]\ar[dr]\\
&\Z_2\ar@{=}[dl]\ar[dr]&&\ast\ar[dl]\ar[dr]\\
\Z_2&&\ast&&\ast
}\]
In particular, the cochain complexes, with coefficients in $\underline\Z$, $R$ and $A$ respectively, are:
\[\underline\Z:\ \xymatrix@C=2cm{
\Z\ar[r];[]_{1+(-1)^{n-1}}&
\Z\ar[r];[]_{1+(-1)^{n-2}}&
\Z\ar[r];[]|{\ \cdots\cdots\ }&
\Z\ar[r];[]_{2}&
\Z\ar[r];[]_{0}&
\Z\ar[r];[]_1&
\Z
}\]
\[R:\ \xymatrix@C=2cm{
\Z\ar[r];[]_{1+(-1)^{n-1}}&
\Z\ar[r];[]_{1+(-1)^{n-2}}&
\Z\ar[r];[]|{\ \cdots\cdots\ }&
\Z\ar[r];[]_{2}&
\Z\ar[r];[]_{0}&
\Z\ar[r];[]_{\text{`$a+b$'}}&
\Z^2
}\]
\[A:\ \xymatrix@C=2cm{
\Z\ar[r];[]_{1+(-1)^{n-1}}&
\Z\ar[r];[]_{1+(-1)^{n-2}}&
\Z\ar[r];[]|{\ \cdots\cdots\ }&
\Z\ar[r];[]_{2}&
\Z\ar[r];[]_{0}&
\Z\ar[r];[]_{\text{`$2a+b$'}}&
\Z^2
}\]
\subsection*{Problem 3(a)}
The cohomology of $S^\sigma$ is concentrated in dimension 0 with each of the three Mackey functors.  When $M=\underline\Z$, $R$, and $A$, we have that $\widetilde H^0(S^\sigma;M)$ is $0$, $\Z$ and $\Z$ respectively.
\subsection*{Problem 3(b)}
When we pass to $S^{n\sigma}$ for $n>1$, we do not change the cohomology in dimensions zero and one. Moreover, the cochain complexes are the same for each of the three Mackey functors. When $n$ is odd, we obtain copies of $\Z_2$ in dimensions $3,5,\ldots,n$. When $n$ is even, we obtain copies of $\Z_2$ in dimensions $3,5,\ldots,n-1$, and a copy of $\Z$ in dimension $n$.
\subsection*{Problem 3(c)}
We use the cofibre sequence $S(n\sigma)_+\to S^0\to S^{n\sigma}$. As $S^0$ has cohomology concentrated in dimension zero, the LES shows that:
\[\widetilde H^i(S(n\sigma)_+;M)\cong \widetilde H^{i+1}(S^{n\sigma};M)\text{\quad for $i>0$}\]
so that we only need to determine $\widetilde H^0(S(n\sigma)_+;M)$, which is the cokernel in a short exact sequence:
\[\widetilde H^{1}(S^{n\sigma};M)=0\longleftarrow\widetilde H^{0}(S(n\sigma)_+;M)\longleftarrow\widetilde H^{0}(S^{0};M)\longleftarrow\widetilde H^{0}(S^{n\sigma};M)\longleftarrow0\]
In each case, this cokernel is isomorphic to $\Z$, so that $\widetilde H^0(S(n\sigma)_+;M)=\Z$.
\subsection*{Problem 3(d)}
The cofibre $C$ of the map $X^G\to X$ has the $G$-equivariant homotopy type of a $G$-CW-complex with no fixed points but the basepoint. As $G=\Z_2$, $C$ is built only of cells of the form $(\coprod\Z_2)_+\wedge S^n$. Thus it is enough to check (because of the LES of the cofibration) that $\widetilde H^*(C;I)=0$ for any $G$-CW-complex built only of such cells.

It is enough to show (the equivalent statement) that $I(\coprod\Z_2)=0$ for any number of copies of $\Z_2$ in the coproduct, which is equivalent to showing that the augmentation map $A(\coprod\Z_2)\to\underline\Z(\coprod\Z_2)$ is always injective. If one thinks about this augmentation map, one sees that it is an isomorphism between groups isomorphic to $\bigoplus\Z$.
\subsection*{Problem 4}
Not done --- sorry.
\subsection*{Problem 5}
Not done --- sorry.
\end{document}
















% !TEX root = z_output/_CobordismNotes.tex

\documentclass[11pt]{article}
\usepackage{fullpage}
\usepackage{amsmath,amsthm,amssymb}
\usepackage{mathrsfs,nicefrac}
\usepackage{amssymb}
\usepackage{epsfig}
\usepackage[all]{xy}
\usepackage{sseq}
\usepackage{tocloft}
\usepackage{cancel}
\usepackage[strict]{changepage}
\usepackage{color}
\usepackage{tikz}
\usepackage{extpfeil}
\usepackage{version}
%\usepackage{ifthen}
%Used for disabling hyperref
\ifx\dontloadhyperref\undefined
%\usepackage[pdftex,pdfborder={0 0 0 [1 1]}]{hyperref}
\usepackage[pdftex,pdfborder={0 0 .5 [1 1]}]{hyperref}
\else
\providecommand{\texorpdfstring}[2]{#1}
\fi

%>>>>>>>>>>>>>>>>>>>>>>>>>>>>>>
%<<<       Better ToC       <<<
%>>>>>>>>>>>>>>>>>>>>>>>>>>>>>>
\setlength{\cftbeforesecskip}{0.5ex}

%>>>>>>>>>>>>>>>>>>>>>>>>>>>>>>
%<<<      Hyperref mod      <<<
%>>>>>>>>>>>>>>>>>>>>>>>>>>>>>>

%needs more testing
\newcounter{dummyforrefstepcounter}
\newcommand{\labelRIGHTHERE}[1]
{\refstepcounter{dummyforrefstepcounter}\label{#1}}


%>>>>>>>>>>>>>>>>>>>>>>>>>>>>>>
%<<<  Theorem Environments  <<<
%>>>>>>>>>>>>>>>>>>>>>>>>>>>>>>
\ifx\dontloaddefinitionsoftheoremenvironments\undefined
\theoremstyle{plain}
\newtheorem{thm}{Theorem}[section]
\newtheorem*{thm*}{Theorem}
\newtheorem{lem}[thm]{Lemma}
\newtheorem*{lem*}{Lemma}
\newtheorem{prop}[thm]{Proposition}
\newtheorem*{prop*}{Proposition}
\newtheorem{cor}[thm]{Corollary}
\newtheorem*{cor*}{Corollary}
\newtheorem{defprop}[thm]{Definition-Proposition}
\newtheorem*{punchline}{Punchline}

\theoremstyle{definition}
\newtheorem{defn}{Definition}[section]
\newtheorem*{defn*}{Definition}
\newtheorem{exmp}{Example}[section]
\newtheorem*{exmp*}{Example}
\newtheorem{asspt}{Assumption}[section]
\newtheorem{notation}{Notation}[section]
\newtheorem{exercise}{Exercise}[section]
\newtheorem*{fact*}{Fact}
\newtheorem*{rmk*}{Remark}
\newtheorem{fact}{Fact}
\newtheorem*{aside}{Aside}
\newtheorem*{question}{Question}
\newtheorem*{answer}{Answer}

\else\relax\fi

%>>>>>>>>>>>>>>>>>>>>>>>>>>>>>>
%<<<      Fields, etc.      <<<
%>>>>>>>>>>>>>>>>>>>>>>>>>>>>>>
\DeclareSymbolFont{AMSb}{U}{msb}{m}{n}
\DeclareMathSymbol{\N}{\mathbin}{AMSb}{"4E}
\DeclareMathSymbol{\Octonions}{\mathbin}{AMSb}{"4F}
\DeclareMathSymbol{\Z}{\mathbin}{AMSb}{"5A}
\DeclareMathSymbol{\R}{\mathbin}{AMSb}{"52}
\DeclareMathSymbol{\Q}{\mathbin}{AMSb}{"51}
\DeclareMathSymbol{\PP}{\mathbin}{AMSb}{"50}
\DeclareMathSymbol{\I}{\mathbin}{AMSb}{"49}
\DeclareMathSymbol{\C}{\mathbin}{AMSb}{"43}
\DeclareMathSymbol{\A}{\mathbin}{AMSb}{"41}
\DeclareMathSymbol{\F}{\mathbin}{AMSb}{"46}
\DeclareMathSymbol{\Quaternions}{\mathbin}{AMSb}{"48}


%>>>>>>>>>>>>>>>>>>>>>>>>>>>>>>
%<<<       Operators        <<<
%>>>>>>>>>>>>>>>>>>>>>>>>>>>>>>
\DeclareMathOperator{\ad}{\textbf{ad}}
\DeclareMathOperator{\coker}{coker}
\renewcommand{\ker}{\textup{ker}\,}
\DeclareMathOperator{\End}{End}
\DeclareMathOperator{\Aut}{Aut}
\DeclareMathOperator{\Hom}{Hom}
\DeclareMathOperator{\Maps}{Maps}
\DeclareMathOperator{\Mor}{Mor}
\DeclareMathOperator{\Gal}{Gal}
\DeclareMathOperator{\Ext}{Ext}
\DeclareMathOperator{\Tor}{Tor}
\DeclareMathOperator{\Map}{Map}
\DeclareMathOperator{\Der}{Der}
\DeclareMathOperator{\Rad}{Rad}
\DeclareMathOperator{\rank}{rank}
\DeclareMathOperator{\ArfInvariant}{Arf}
\DeclareMathOperator{\KervaireInvariant}{Ker}
\DeclareMathOperator{\im}{im}
\DeclareMathOperator{\coim}{coim}
\DeclareMathOperator{\trace}{tr}
\DeclareMathOperator{\supp}{supp}
\DeclareMathOperator{\ann}{ann}
\DeclareMathOperator{\spec}{Spec}
\DeclareMathOperator{\proj}{Proj}
\DeclareMathOperator{\fiber}{F}
\DeclareMathOperator{\cofiber}{C}
\DeclareMathOperator{\cone}{cone}
\DeclareMathOperator{\Skel}{Sk}
\DeclareMathOperator{\conn}{conn}
\DeclareMathOperator{\colim}{colim}
\DeclareMathOperator{\limit}{lim}

%>>>>>>>>>>>>>>>>>>>>>>>>>>>>>>
%<<<   Cohomology Theories  <<<
%>>>>>>>>>>>>>>>>>>>>>>>>>>>>>>
\DeclareMathOperator{\KR}{{K\R}}
\DeclareMathOperator{\KO}{{KO}}
\DeclareMathOperator{\K}{{K}}
\DeclareMathOperator{\OmegaO}{{\Omega_{\Octonions}}}

%>>>>>>>>>>>>>>>>>>>>>>>>>>>>>>
%<<<   Algebraic Geometry   <<<
%>>>>>>>>>>>>>>>>>>>>>>>>>>>>>>
\DeclareMathOperator{\Spec}{Spec}
\DeclareMathOperator{\Proj}{Proj}
\DeclareMathOperator{\Sing}{Sing}
\DeclareMathOperator{\shfHom}{\mathscr{H}\textit{\!\!om}}
\newcommand{\WeilDivisors}{\textup{Div}}
\newcommand{\CartierDivisors}{\textup{CaDiv}}
\newcommand{\PrincipalWeilDivisors}{\textup{PrDiv}}
\newcommand{\LocallyPrincipalWeilDivisors}{\textup{LPDiv}}
\newcommand{\PrincipalCartierDivisors}{\textup{PrCaDiv}}
\newcommand{\DivisorClass}{\textup{Cl}}
\newcommand{\CartierClass}{\textup{CaCl}}
\newcommand{\Picard}{\textup{Pic}}
\DeclareMathOperator{\Frob}{Frob}


%>>>>>>>>>>>>>>>>>>>>>>>>>>>>>>
%<<<  Mathematical Objects  <<<
%>>>>>>>>>>>>>>>>>>>>>>>>>>>>>>
\newcommand{\sll}{\mathfrak{sl}}
\newcommand{\gl}{\mathfrak{gl}}
\newcommand{\GL}{\mbox{GL}}
\newcommand{\PGL}{\mbox{PGL}}
\newcommand{\SL}{\mbox{SL}}
\newcommand{\Mat}{\mbox{Mat}}
\newcommand{\Gr}{\textup{Gr}}
\newcommand{\Squ}{\textup{Sq}}
\newcommand{\catSet}{\textit{Sets}}
\newcommand{\RP}{{\R\PP}}
\newcommand{\CP}{{\C\PP}}
\newcommand{\Steen}{\mathscr{A}}
\newcommand{\Orth}{\textup{\textbf{O}}}

%>>>>>>>>>>>>>>>>>>>>>>>>>>>>>>
%<<<  Mathematical Symbols  <<<
%>>>>>>>>>>>>>>>>>>>>>>>>>>>>>>
\newcommand{\DASH}{\textup{---}}
\newcommand{\op}{\textup{op}}
\newcommand{\ob}{\textup{ob}\,}
\newcommand{\ho}{\textup{ho}}
\newcommand{\st}{\textup{st}}
\newcommand{\id}{\textup{id}}
\newcommand{\Bullet}{\ensuremath{\bullet} }

%>>>>>>>>>>>>>>>>>>>>>>>>>>>>>>
%<<<      Some Arrows       <<<
%>>>>>>>>>>>>>>>>>>>>>>>>>>>>>>
\let\shortmapsto\mapsto
\let\mapsto\longmapsto
\newcommand{\mapsfrom}{\,\reflectbox{$\mapsto$}\ }
\newcommand{\bigrightsquig}{\scalebox{2}{\ensuremath{\rightsquigarrow}}}
\newcommand{\bigleftsquig}{\reflectbox{\scalebox{2}{\ensuremath{\rightsquigarrow}}}}

%\newcommand{\cofibration}{\xhookrightarrow{\phantom{\ \,{\sim\!}\ \ }}}
%\newcommand{\fibration}{\xtwoheadrightarrow{\phantom{\sim\!}}}
%\newcommand{\acycliccofibration}{\xhookrightarrow{\ \,{\sim\!}\ \ }}
%\newcommand{\acyclicfibration}{\xtwoheadrightarrow{\sim\!}}
%\newcommand{\leftcofibration}{\xhookleftarrow{\phantom{\ \,{\sim\!}\ \ }}}
%\newcommand{\leftfibration}{\xtwoheadleftarrow{\phantom{\sim\!}}}
%\newcommand{\leftacycliccofibration}{\xhookleftarrow{\ \ {\sim\!}\,\ }}
%\newcommand{\leftacyclicfibration}{\xtwoheadleftarrow{\sim\!}}
%\newcommand{\weakequiv}{\xrightarrow{\ \,\sim\,\ }}
%\newcommand{\leftweakequiv}{\xleftarrow{\ \,\sim\,\ }}

\newcommand{\cofibration}
{\xhookrightarrow{\phantom{\ \,{\raisebox{-.3ex}[0ex][0ex]{\scriptsize$\sim$}\!}\ \ }}}
\newcommand{\fibration}
{\xtwoheadrightarrow{\phantom{\raisebox{-.3ex}[0ex][0ex]{\scriptsize$\sim$}\!}}}
\newcommand{\acycliccofibration}
{\xhookrightarrow{\ \,{\raisebox{-.55ex}[0ex][0ex]{\scriptsize$\sim$}\!}\ \ }}
\newcommand{\acyclicfibration}
{\xtwoheadrightarrow{\raisebox{-.6ex}[0ex][0ex]{\scriptsize$\sim$}\!}}
\newcommand{\leftcofibration}
{\xhookleftarrow{\phantom{\ \,{\raisebox{-.3ex}[0ex][0ex]{\scriptsize$\sim$}\!}\ \ }}}
\newcommand{\leftfibration}
{\xtwoheadleftarrow{\phantom{\raisebox{-.3ex}[0ex][0ex]{\scriptsize$\sim$}\!}}}
\newcommand{\leftacycliccofibration}
{\xhookleftarrow{\ \ {\raisebox{-.55ex}[0ex][0ex]{\scriptsize$\sim$}\!}\,\ }}
\newcommand{\leftacyclicfibration}
{\xtwoheadleftarrow{\raisebox{-.6ex}[0ex][0ex]{\scriptsize$\sim$}\!}}
\newcommand{\weakequiv}
{\xrightarrow{\ \,\raisebox{-.3ex}[0ex][0ex]{\scriptsize$\sim$}\,\ }}
\newcommand{\leftweakequiv}
{\xleftarrow{\ \,\raisebox{-.3ex}[0ex][0ex]{\scriptsize$\sim$}\,\ }}


%>>>>>>>>>>>>>>>>>>>>>>>>>>>>>>
%<<<     Greek Letters      <<<
%>>>>>>>>>>>>>>>>>>>>>>>>>>>>>>
%\newcommand{\oldphi}{\phi}
%\renewcommand{\phi}{\varphi}
\let\oldphi\phi
\let\phi\varphi
\renewcommand{\to}{\longrightarrow}
\newcommand{\eps}{\varepsilon}

%>>>>>>>>>>>>>>>>>>>>>>>>>>>>>>
%<<<  1st-4th & parentheses <<<
%>>>>>>>>>>>>>>>>>>>>>>>>>>>>>>
\newcommand{\first}{^\text{st}}
\newcommand{\second}{^\text{nd}}
\newcommand{\third}{^\text{rd}}
\newcommand{\fourth}{^\text{th}}
\newcommand{\ZEROTH}{$0^\text{th}$ }
\newcommand{\FIRST}{$1^\text{st}$ }
\newcommand{\SECOND}{$2^\text{nd}$ }
\newcommand{\THIRD}{$3^\text{rd}$ }
\newcommand{\FOURTH}{$4^\text{th}$ }
\newcommand{\iTH}{$i^\text{th}$ }
\newcommand{\jTH}{$j^\text{th}$ }
\newcommand{\nTH}{$n^\text{th}$ }

%>>>>>>>>>>>>>>>>>>>>>>>>>>>>>>
%<<<    upright commands    <<<
%>>>>>>>>>>>>>>>>>>>>>>>>>>>>>>
\newcommand{\upcol}{\textup{:}}
\newcommand{\upsemi}{\textup{;}}
\providecommand{\lparen}{\textup{(}}
\providecommand{\rparen}{\textup{)}}
\renewcommand{\lparen}{\textup{(}}
\renewcommand{\rparen}{\textup{)}}
\newcommand{\Iff}{\emph{iff} }

%>>>>>>>>>>>>>>>>>>>>>>>>>>>>>>
%<<<     Environments       <<<
%>>>>>>>>>>>>>>>>>>>>>>>>>>>>>>
\newcommand{\squishlist}
{ %\setlength{\topsep}{100pt} doesn't seem to do anything.
  \setlength{\itemsep}{.5pt}
  \setlength{\parskip}{0pt}
  \setlength{\parsep}{0pt}}
\newenvironment{itemise}{
\begin{list}{\textup{$\rightsquigarrow$}}
   {  \setlength{\topsep}{1mm}
      \setlength{\itemsep}{1pt}
      \setlength{\parskip}{0pt}
      \setlength{\parsep}{0pt}
   }
}{\end{list}\vspace{-.1cm}}
\newcommand{\INDENT}{\textbf{}\phantom{space}}
\renewcommand{\INDENT}{\rule{.7cm}{0cm}}

\newcommand{\itm}[1][$\rightsquigarrow$]{\item[{\makebox[.5cm][c]{\textup{#1}}}]}

\newcommand{\rednote}[1]{{\color{red}#1}\scalebox{.1}{rednote}}
\newcommand{\bluenote}[1]{{\color{blue}#1}\scalebox{.1}{rednote}}
\newcommand{\funcdef}[4]{\begin{align*}
#1&\to #2\\
#3&\mapsto#4
\end{align*}}

%\newcommand{\comment}[1]{}

%>>>>>>>>>>>>>>>>>>>>>>>>>>>>>>
%<<<       Categories       <<<
%>>>>>>>>>>>>>>>>>>>>>>>>>>>>>>
\newcommand{\Ens}{{\mathscr{E}ns}}
\DeclareMathOperator{\Sheaves}{{\mathsf{Shf}}}
\DeclareMathOperator{\Presheaves}{{\mathsf{PreShf}}}
\DeclareMathOperator{\Varieties}{{\mathsf{Var}}}
\DeclareMathOperator{\Schemes}{{\mathsf{Sch}}}
\DeclareMathOperator{\Rings}{{\mathsf{Rings}}}
\DeclareMathOperator{\AbGp}{{\mathsf{AbGp}}}
\DeclareMathOperator{\Modules}{{\mathsf{\!-Mod}}}
\DeclareMathOperator{\QuasiCoherent}{{\mathsf{QCoh}}}
\DeclareMathOperator{\Coherent}{{\mathsf{Coh}}}
\DeclareMathOperator{\GSW}{{\mathcal{SW}^G}}
\DeclareMathOperator{\Burnside}{{\mathsf{Burn}}}
\DeclareMathOperator{\GSet}{{G\mathsf{Set}}}
\DeclareMathOperator{\FinGSet}{{G\mathsf{Set}^\textup{fin}}}
\DeclareMathOperator{\HSet}{{H\mathsf{Set}}}
\DeclareMathOperator{\Cat}{{\mathsf{Cat}}}
\DeclareMathOperator{\Orb}{{\mathsf{Orb}}}
\DeclareMathOperator{\Set}{{\mathsf{Set}}}
\DeclareMathOperator{\sSet}{{\mathsf{sSet}}}
\DeclareMathOperator{\Top}{{\mathsf{Top}}}
\DeclareMathOperator{\GSpectra}{{G-\mathsf{Spectra}}}

%>>>>>>>>>>>>>>>>>>>>>>>>>>>>>>
%<<<     Script Letters     <<<
%>>>>>>>>>>>>>>>>>>>>>>>>>>>>>>
\newcommand{\scrQ}{\mathscr{Q}}
\newcommand{\scrW}{\mathscr{W}}
\newcommand{\scrE}{\mathscr{E}}
\newcommand{\scrR}{\mathscr{R}}
\newcommand{\scrT}{\mathscr{T}}
\newcommand{\scrY}{\mathscr{Y}}
\newcommand{\scrU}{\mathscr{U}}
\newcommand{\scrI}{\mathscr{I}}
\newcommand{\scrO}{\mathscr{O}}
\newcommand{\scrP}{\mathscr{P}}
\newcommand{\scrA}{\mathscr{A}}
\newcommand{\scrS}{\mathscr{S}}
\newcommand{\scrD}{\mathscr{D}}
\newcommand{\scrF}{\mathscr{F}}
\newcommand{\scrG}{\mathscr{G}}
\newcommand{\scrH}{\mathscr{H}}
\newcommand{\scrJ}{\mathscr{J}}
\newcommand{\scrK}{\mathscr{K}}
\newcommand{\scrL}{\mathscr{L}}
\newcommand{\scrZ}{\mathscr{Z}}
\newcommand{\scrX}{\mathscr{X}}
\newcommand{\scrC}{\mathscr{C}}
\newcommand{\scrV}{\mathscr{V}}
\newcommand{\scrB}{\mathscr{B}}
\newcommand{\scrN}{\mathscr{N}}
\newcommand{\scrM}{\mathscr{M}}

%>>>>>>>>>>>>>>>>>>>>>>>>>>>>>>
%<<<     Fractur Letters    <<<
%>>>>>>>>>>>>>>>>>>>>>>>>>>>>>>
\newcommand{\frakQ}{\mathfrak{Q}}
\newcommand{\frakW}{\mathfrak{W}}
\newcommand{\frakE}{\mathfrak{E}}
\newcommand{\frakR}{\mathfrak{R}}
\newcommand{\frakT}{\mathfrak{T}}
\newcommand{\frakY}{\mathfrak{Y}}
\newcommand{\frakU}{\mathfrak{U}}
\newcommand{\frakI}{\mathfrak{I}}
\newcommand{\frakO}{\mathfrak{O}}
\newcommand{\frakP}{\mathfrak{P}}
\newcommand{\frakA}{\mathfrak{A}}
\newcommand{\frakS}{\mathfrak{S}}
\newcommand{\frakD}{\mathfrak{D}}
\newcommand{\frakF}{\mathfrak{F}}
\newcommand{\frakG}{\mathfrak{G}}
\newcommand{\frakH}{\mathfrak{H}}
\newcommand{\frakJ}{\mathfrak{J}}
\newcommand{\frakK}{\mathfrak{K}}
\newcommand{\frakL}{\mathfrak{L}}
\newcommand{\frakZ}{\mathfrak{Z}}
\newcommand{\frakX}{\mathfrak{X}}
\newcommand{\frakC}{\mathfrak{C}}
\newcommand{\frakV}{\mathfrak{V}}
\newcommand{\frakB}{\mathfrak{B}}
\newcommand{\frakN}{\mathfrak{N}}
\newcommand{\frakM}{\mathfrak{M}}

\newcommand{\frakq}{\mathfrak{q}}
\newcommand{\frakw}{\mathfrak{w}}
\newcommand{\frake}{\mathfrak{e}}
\newcommand{\frakr}{\mathfrak{r}}
\newcommand{\frakt}{\mathfrak{t}}
\newcommand{\fraky}{\mathfrak{y}}
\newcommand{\fraku}{\mathfrak{u}}
\newcommand{\fraki}{\mathfrak{i}}
\newcommand{\frako}{\mathfrak{o}}
\newcommand{\frakp}{\mathfrak{p}}
\newcommand{\fraka}{\mathfrak{a}}
\newcommand{\fraks}{\mathfrak{s}}
\newcommand{\frakd}{\mathfrak{d}}
\newcommand{\frakf}{\mathfrak{f}}
\newcommand{\frakg}{\mathfrak{g}}
\newcommand{\frakh}{\mathfrak{h}}
\newcommand{\frakj}{\mathfrak{j}}
\newcommand{\frakk}{\mathfrak{k}}
\newcommand{\frakl}{\mathfrak{l}}
\newcommand{\frakz}{\mathfrak{z}}
\newcommand{\frakx}{\mathfrak{x}}
\newcommand{\frakc}{\mathfrak{c}}
\newcommand{\frakv}{\mathfrak{v}}
\newcommand{\frakb}{\mathfrak{b}}
\newcommand{\frakn}{\mathfrak{n}}
\newcommand{\frakm}{\mathfrak{m}}

%>>>>>>>>>>>>>>>>>>>>>>>>>>>>>>
%<<<  Caligraphic Letters   <<<
%>>>>>>>>>>>>>>>>>>>>>>>>>>>>>>
\newcommand{\calQ}{\mathcal{Q}}
\newcommand{\calW}{\mathcal{W}}
\newcommand{\calE}{\mathcal{E}}
\newcommand{\calR}{\mathcal{R}}
\newcommand{\calT}{\mathcal{T}}
\newcommand{\calY}{\mathcal{Y}}
\newcommand{\calU}{\mathcal{U}}
\newcommand{\calI}{\mathcal{I}}
\newcommand{\calO}{\mathcal{O}}
\newcommand{\calP}{\mathcal{P}}
\newcommand{\calA}{\mathcal{A}}
\newcommand{\calS}{\mathcal{S}}
\newcommand{\calD}{\mathcal{D}}
\newcommand{\calF}{\mathcal{F}}
\newcommand{\calG}{\mathcal{G}}
\newcommand{\calH}{\mathcal{H}}
\newcommand{\calJ}{\mathcal{J}}
\newcommand{\calK}{\mathcal{K}}
\newcommand{\calL}{\mathcal{L}}
\newcommand{\calZ}{\mathcal{Z}}
\newcommand{\calX}{\mathcal{X}}
\newcommand{\calC}{\mathcal{C}}
\newcommand{\calV}{\mathcal{V}}
\newcommand{\calB}{\mathcal{B}}
\newcommand{\calN}{\mathcal{N}}
\newcommand{\calM}{\mathcal{M}}

%>>>>>>>>>>>>>>>>>>>>>>>>>>>>>>
%<<<<<<<<<DEPRECIATED<<<<<<<<<<
%>>>>>>>>>>>>>>>>>>>>>>>>>>>>>>

%%% From Kac's template
% 1-inch margins, from fullpage.sty by H.Partl, Version 2, Dec. 15, 1988.
%\topmargin 0pt
%\advance \topmargin by -\headheight
%\advance \topmargin by -\headsep
%\textheight 9.1in
%\oddsidemargin 0pt
%\evensidemargin \oddsidemargin
%\marginparwidth 0.5in
%\textwidth 6.5in
%
%\parindent 0in
%\parskip 1.5ex
%%\renewcommand{\baselinestretch}{1.25}

%%% From the net
%\newcommand{\pullbackcorner}[1][dr]{\save*!/#1+1.2pc/#1:(1,-1)@^{|-}\restore}
%\newcommand{\pushoutcorner}[1][dr]{\save*!/#1-1.2pc/#1:(-1,1)@^{|-}\restore}










\renewcommand{\Steen}{\calA}

\begin{document}
\section{Unoriented Bordism}
\subsection{Steenrod's question}
\begin{shaded}
When can an element $\alpha\in {H}_{n}(X;\F_2)$ be represented by a singular manifold $f:M\to X$?
\end{shaded}
\begin{itemize}\squishlist
\item The obvious map from singular $n$-manifolds on $X$ to ${H}_{n}(X;\F_2)$ descends to the \textbf{unoriented bordism group} $\frakN_n(X)$, giving a natural homomorphism $\frakN_n(X)\to H_n(X)$.
\item $\frakN_n$ is a functor $\Top\to\mathsf{Mod}_{\F_2}$. There are product maps $\frakN_n(X)\times\frakN_m(Y)\to\frakN_{n+m}(X\times Y)$, so that $\frakN_*(*)$ is a graded-commutative ring, the \textbf{bordism ring}.
\item We can do the same with oriented manifolds, to get an \textbf{oriented bordism group} $\Omega_n(X)$, and an \textbf{oriented bordism ring}.
\end{itemize}
\subsection{Thom spaces and stable normal bundles}
\begin{itemize}\squishlist
\item An embedding of a manifold $M^n$ into Euclidean space yields a normal bundle on $M$.
\item Any two normal bundles are stably isomorphic, so $M$ has a well defined \textbf{stable normal bundle}.
\item Choose an embedding $i:M^n\cofibration \R^{n+k}$, and find a tubular neighbourhood $E(\nu)$ of $M$. On one point compactifications, we get an induced map
\[S^{n+k}\to E(\nu)_+=:\Thom(\nu).\]
\end{itemize}
\subsection{The Pontrjagin-Thom Construction}
We'll give an isomorphism $\frakN_n(X)\to\MO_n(X)$. Given a singular manifold $f:M^n\to X$:
\begin{itemise}
\item Choose an embedding $i:M\cofibration\R^{n+k}$, with tubular neighbourhood $E(\nu)$.
\item Let $\overline{\nu}:E(\nu)\to EO(k)$ be a bundle map. We then have a map
\[\vcenter{\xymatrix@1{
E(\nu)\ar[r]^-{\overline{\nu}\times\pi}
&%r1c1
EO(k)\times M
\ar[r]^-{\id\times f}
&%r1c2
EO(k)\times X%r1c3
}}\]
We can take Thom spaces, and use the collapse map from $S^{n+k}$, to get
\[S^{n+k}\to\Thom\nu\to MO(k)\wedge X_+\]
which is a homotopy class in $\pi_{n+k}(MO(k)\wedge X_+)$.
\item One can check that this class is well defined in
\[\MO_n(X):=\varinjlim_k\pi_{n+k}(MO(k)\wedge X_+),\]
and only depends on bordism class.
\end{itemise}
To see that this is an isomorphism, we construct an inverse directly.
\begin{itemise}
\item Given an element of $\MO_n(X)$, realise it as a map as in the top row, and compress this map onto the Thom space of the tautolgous bundle over a finite grassmannian
\[\xymatrix{
X&
(\MO(k)\wedge X_+)\setminus(\ast)
\ar@{^{(}->}[r]
\ar[l]_-{\textup{pr}_2}
&
\MO(k)\wedge X_+
\ar@{=}[d]
&
\\
M
\ar@{-->}[ur]
\ar@{^{(}-->}[r]
\ar@{-->}[rd]
&%r1c0
S^{n+k}
\ar@{}[d]|{\textup{p.b.}}
\ar[r]
\ar[rd]^\gamma
&%r1c1
\frac{MO(k)\times X}{\ast\times X}
\ar[r]^-{\textup{pr}_1}
&%r1c2
\Thom\xi_k
\\&%r1c3
\Gr_k(\R^{n+k})
\ar@{^{(}->}[r]
&%r2c1
\Thom \xi_{k,n}
\ar@{^{(}->}[ur]
&%r2c2
%r2c3
}\]
\item The Thom space $\Thom \xi_{k,n}$ is a manifold away from the origin, so we can deform the map $\gamma$ to be transverse to $\Gr_{k}(\R^{n+k})$. Then the pullback (drawn dashed above) is a manifold $M$.
\item Although there is no map $\MO(k)\wedge X_+\to X$, there is a map $(\MO(k)\wedge X_+)\setminus(*)\to X$. As the composite $M\to\MO(k)\wedge X_+\to X$ does not hit the basepoint, we thus obtain a map $M\to X$, giving a (well defined) element of $\frakN_n(X)$.
\end{itemise}
\subsection{Spectra}
\begin{itemise}
\item A definition of spectra is given, along with some examples. 
\item $\MO$ is an associative, commutative ring spectrum --- there are maps $MO(k)\wedge MO(j)\to MO(j+k)$ induced by the map classifying the product bundle $EO(k)\times EO(j)$.
\item There's a ring homomorphism $\frakN_*\to \pi_{*}(\MO)$.
\end{itemise}
\subsection{The Thom Isomorphism}
Given $\xi$ an $R$-orientable $n$-plane bundle over paracompact $X$, we can use a degenerate relative Serre spectral sequence to see that
\[{H}^{q}(X;R)\cong \widetilde{H}^{q+n}(\Thom\xi;R)={H}^{q+n}(E(\xi),E(\xi)\setminus X;R)={H}^{q+n}(D(\xi),S(\xi);R)\]
In fact, $\widetilde{H}^{*}(\Thom\xi)$ is a free module of rank one over ${H}^{*}(X)$. This module structure comes either from
\begin{itemise}
\item noting that a cohomological Serre spectral sequence is a spectral sequence of modules over the cohomology of the base $X$; or
\item the maps induced by the following diagram:
\[\xymatrix{
E(\xi)
\ar[r]
\ar[d]
&%r1c1
E(\textbf{0})\times E(\xi)
\ar[d]
\\%r1c2
X
\ar[r]^-{\Delta}
&%r2c1
X\times X
%r2c2
}\qquad 
\xymatrix{
\Thom \xi
\ar[r]
&%r1c1
X_+\wedge \Thom\xi
\\%r1c2
\widetilde{H}^{*}(\Thom\xi)
&%r2c1
{H^*(X)}\otimes{\widetilde{H}^{*}(\Thom\xi)}
\ar[l]
%r2c2
}\]
\end{itemise} 
The generator $u$ of $\widetilde{H}^{*}(\Thom\xi)$ is called the Thom class.
\begin{itemize}\squishlist
\item The restriction of $u$ to any $D(\xi)_x/S(\xi)_x$ is the distinguished generator given by the orientation.
\item The Thom class is multiplicative: $u_{\xi\times\eta}=u_\xi\wedge u_\eta$.
\end{itemize}
\subsection{Steenrod Operations}
A brief discussion thereof. In particular, they satisfy a Cartan formula.
\subsection{Stiefel-Whitney Classes}
We define the total Stiefel-Whitney class and Euler class of $\xi$ as follows:
\[\xymatrix@R=.805cm@C=3.5cm@!0{
X_+\wedge \Thom\xi&
\Thom\xi
\ar[l]
&
X
\ar[l]^-{\textup{0 section}}_-s
\\
H^*(X)
\ar[r]^-{\cdot u}_-{\cong}
&%r1c1
\widetilde{H}^{*}(\Thom\xi)
\ar[r]^-{s^*}
&
{H}^{*}(X)
\\%r1c2
1
\ar@{|->}[r]
&%r2c1
u
\ar@{|->}[d]^-{\Squ}
\ar@{|->}[r]
&
e(\xi)
\\%r2c2
w(\xi)
&%r3c1
\Squ u\ar@{|->}[l]%r3c2
}\]
They satisfy a Whitney sum formula. Some standard applications are given.
\subsection{The Euler Class}
\begin{itemise}
\item There is an Euler class defined as above whenever $\xi$ is $R$-orientable.
\item If $\xi$ has a nonvanishing section, then $e(\xi)=0$. Thus $e$ is unstable.
\item When $R=\F_2$, $e(\xi)=w_n(\xi)\in H^n(X)$.
\end{itemise}
\rednote{I don't really see what's being said in the last paragraph.}
%$s^*:\widetilde{H}^{*}(\Thom\xi)\to{H}^{*}(X)$ is an $H^*(X)$-module homomorphism.
\subsection{The Steenrod Algebra}
By Yoneda's lemma:
\begin{itemize}\squishlist
\item The operations $\Squ^i:{H}^{n}(X;\F_2)\to{H}^{n+i}(X;\F_2)$ correspond to maps $K_k\to K_n$.
\item Addition on $H^k$ corresponds to a map $\mu:K_k\times K_k\to K_k$, as the functor $X\mapsto H^k(X)\times H^k(X)$ is represented by $K_k\times K_k$.
\item As addition is a natural abelian group structure on $H^k$, $K_k$ gets a natural abelian grouplike H-space structure. \rednote{Right?... and for some reason, this is the loopspace structure... right?}
\item Let $\psi:H^*(K_k)\to H^*(K_k)\otimes H^*(K_k)$ be the cocommutative comultiplication induced by $\mu$.
\end{itemize}
Suppose we are given a map $x:K_k\to K_n$, i.e.\ a natural transformation $x:H^k\nt H^n$, i.e.\ an element $x\in H^n(K_k)$. One finds that the natural transformation is additive \Iff the following square commutes:
\[\xymatrix{
K_k\times K_k
\ar[r]^-{\mu}
\ar[d]^-{x\times x}
&%r1c1
K_k
\ar[d]^-{x}
\\%r1c2
K_n\times K_n
\ar[r]^-{\mu}
&%r2c1
K_n%r2c2
}\]
The higher composite represents $\psi(x)$. The lower composite is perhaps a little more tricky, but it can be rewritten as
\[\vcenter{\xymatrix@1@C=2cm{
K_k\times K_k
\ar[r]^-{(x\circ\pi_1)\times (x\circ\pi_2)}
&%r1c1
K_n\times K_n
\ar[r]^-{\mu}
&%r1c2
K_n%r1c3
}}\]
which represents the sum of what can only be $x\otimes 1$ and $1\otimes x$. Thus
\begin{prop*}
Additive natural transformations $H^k\to H^n$ are in bijection with primitive elements
\[P(H^n(K_k))=\left\{x\in H^n(K_k)\,:\,\psi(x)=x\otimes 1+1\otimes x\right\}.\]
\end{prop*}
\noindent Now the multiplication has a two-sided unit, so that we obtain commuting diagrams:
\[\xymatrix@R=.4cm{
K_k\times K_k
\ar[r]^-{\mu}
&%r1c1
K_k
\ar@{=}[rd]
\ar@{=}[ld]
&%r1c2
K_k\times K_k
\ar[l]_-{\mu}
\\%r1c3
K_k\times\{\textup{pt}\}\ar@{^{(}->}[u]
&%r2c1
&%r2c2
\{\textup{pt}\}\times K_k\ar@{^{(}->}[u]
%r2c3
}\qquad 
\xymatrix@R=.4cm{
H^n(K_k)\otimes H^n(K_k)
\ar[r];[]_-{\psi}
&%r1c1
H^n(K_k)
\ar@{=}[dr];[]\ar@{=}[dl];[]
&%r1c2
H^n(K_k)\otimes H^n(K_k)
\ar[l];[]^-{\psi}
\\%r1c3
H^n(K_k)\otimes\F_2\ar[u];[]
&%r2c1
&%r2c2
\F_2\otimes H^n(K_k)\ar[u];[]
%r2c3
}\]
This shows that $\psi(x)$ is always of the form $x\otimes 1+1\otimes x+\cdots$. If $2k>n$, there are no possible cross terms, so that every $x$ is primitive.

In the path-loop Serre exact sequence for $K_k$, the following transgression is an isomorphism, with inverse given by applying loops:
\[\vcenter{\xymatrix@R=6mm{
{H}^{q-1}(K_{k-1})
\ar[r]_-{\cong}^-{\tau}
\ar@{=}[d]
&%r1c1
{H}^{q-1}(K_{k-1})
\ar@{=}[d]
\\%r1c2
[K_{k-1},K_{q-1}]
&%r2c1
[K_{k},K_{q}]
\ar[l]_-{\Omega}
%r2c2
}}\textup{\qquad for $q<2k$}.\]
Now for any $n$, we have an inverse system
\[\xymatrix{
0&
\ar[l]\cdots&
[K_{q-1},K_{q+n-1}]\ar[l]&
[K_q,K_{q+n}]\ar[l]_-\Omega&
[K_{q+1},K_{q+n+1}]\ar[l]_-\Omega&
\cdots\ar[l]}\]
and the maps are eventually isomorphisms. Thus
\[\Steen^n:=\varprojlim H^{r+n}(K_r)\cong H^{q+n}(K_q)\textup{\quad for $q\geq n$.}\]
The $\Steen^n$ fit together to form the Steenrod algebra $\Steen$, a Hopf algebra with diagonal defined by:
\[\psi:\Squ^k\mapsto\sum_{i+j=k}\Squ^i\otimes\Squ^j.\]
\subsection{Hopf Algebras}
\begin{itemize}\squishlist
\item $H_*(X)$ is a cocommutative coalgebra (using the correctly signed switch map).
\item $H_*(X)$ is connected \Iff $X$ is path connected.
\item Hopf algebras are defined (and the definition is symmetric).
\item For $A$ a Hopf algebra, the tensor product of $A$-modules is again an $A$-module, via
\[a(x\otimes y):=\sum(-1)^{|a''||x|}a'x\otimes a''y\textup{\quad where $\psi(a)=\sum a'\otimes a''$.}\]
Using this $\Steen$-module structure on the tensor product, the cross product map
\[H^*(X)\otimes H^*(Y)\overset{\times }{\to}H^*(X\times Y)\]
is an $\Steen$-module map, and $H^*(X)$ is an $\Steen$-module algebra.
\end{itemize}
\begin{prop*}[Milnor-Moore]
Let $k$ be a field, $A$ a connected Hopf algebra thereover, and $M$ a connected $A$-module coalgebra such that $i$ is monic, where
\[i:A\to M\quad a\mapsto a\cdot 1.\]
 Then $M$ is a free $A$-module.
\end{prop*}
\subsection{Return of the Steenrod Algebra}
\begin{itemize}\squishlist
\item On $\RP^\infty$ with $H^1(\RP^\infty)$ generated by $x$ (the Euler class), one calculates:
\[\Squ(x^n)=x^n(1+x)^n,\textup{ so that }\Squ(x^{2^n})=x^{2^n}+x^{2^{n+1}}.\]
Thus we have an $\Steen$-submodule of $H^*(\RP^\infty)$:
\[F(1)=\langle x,x^2,x^4,\ldots\rangle=\Steen x.\]
\item Simmilarly, it's handy to note the following. {\small On $X=(\RP^\infty)^n$, with $H^*(X)=\F_2[x_1,\ldots,x_n]$, the Euler class is $e=x_1x_2\cdots x_n$. One finds that $\Steen$ acts freely on $e$ up to degree $n$, i.e.:
\[\Steen^q\to H^{n+q}(X),\quad \theta\mapsto \theta(e)\qquad \textup{is monic for $q\leq n$.}\]}
\item Write $\Steen^*$ for $\Steen$, and $\Steen_*$ for its dual Hopf algebra. We can use the representation of $\Steen$ on $H^*(\RP^\infty)$ to give algebra generators for $\Steen_*$. We have a linear map
\[\Steen\to H^*(\RP^\infty)\qquad \theta\mapsto \theta (x)\in F(1).\]
Now write $\theta (x)=\sum \xi_i(\theta)\,x^{2^i}$, we have defined linear functionals $\xi_i$ on $\theta$. Note that $\xi_i$ has degree $2^i-1$, as it is only nonzero on $\Steen^{2^i-1}$. In fact, $\Steen_*$ is a \textbf{polynomial} algebra on the $\xi_i$.
\item Before finding the coalgbra structure, one proves the formula:
\[\theta (x^{2^n})=\xi_k^{2^n}(\theta)x^{2^{n+k}}\textup{\quad for $\theta\in\Steen^{2^n(2^k-1)}$.}\]
\item The formula for the diagonal in $\Steen_*$ is 
\[\psi(\xi_n)=\sum_{i+j=n}\xi_i^{2^j}\otimes \xi_j.\]
Let's understand $\psi$ a little better first. Note that if $V$ is a vector space, there is a natural map \smash{$V^*\otimes V^*\overset{c}{\to}(V\otimes V)^*$}, which is an isomorphism when $V$ is finite dimensional. The inverse is not so easy to write down.

\INDENT Now $\psi(\xi_n)$ should be an element of $\Steen_*\otimes \Steen_*$. However, it is easy to define an element 
\[c(\psi(\xi_n))\in(\Steen\otimes \Steen)^*\qquad c(\psi(\xi_n))(\theta\otimes\phi)=\xi_n(\theta\phi),\]
and to give a formula for $\psi(\xi_n)$ as an element of $\Steen_*\otimes \Steen_*$, one feels that it may be easiest to first make a guess, and then verify the guess! \rednote{(Is this silly?)}
Thus, we'd like to show that 
\[c\left(\sum \xi_i^{2^j}\otimes \xi_j\right)=\left(\theta\otimes\phi\mapsto \xi_n(\theta\phi)\right):\Steen\otimes \Steen\to\F_2.\]
That is, for all $\phi,\theta\in\Steen$, that
\[\sum_{i+j=n} \xi_i^{2^j}(\theta)\xi_j(\phi)=\xi_n(\theta\phi).\]
\rednote{This is all a bit confusing for some reason.}

\end{itemize}
\subsection{The Answer to the Question}
\begin{thm*}
$\Steen$ acts freely through degree $n$ on the Thom class $u\in H^n(MO(n))$.
\end{thm*}
\begin{proof}
We saw that it acts freely on the Euler class on $(\RP^\infty)^n$ in this range. By definition of Euler classes, we have
\[\xymatrix@R=0mm{
MO(n)
&%r1c1
BO(n)
\ar[l]_-{s}
&%r1c2
(\RP^\infty)^n
\ar[l]
\\%r1c3
u
\ar@{|->}[r]
&%r2c1
e(\xi_n)
\ar@{|->}[r]
&%r2c2
e(\lambda\times \cdots \times \lambda)
%r2c3
}\] and $\Steen$ acts freely in this range on $e(\lambda\times \cdots \times \lambda)$.
\end{proof}
\noindent  Now from our understanding of the Thom isomorphism, we have that $H^0(\MO)=\F_2\langle u\rangle$, and that $H^{i}(\MO)=0$ for $i<0$. Moreover, $\Steen$ acts freely on $u\in H^0(\MO)$.
\begin{thm*}
$H^*(\MO)$ is a free $\Steen$-module.
\end{thm*}
\begin{proof}
By the proposition of Milnor and Moore, it's enough to see that $H^*(\MO)$ is an $\Steen$-module coalgebra --- we have just seen that $H^*(\MO)$ is connected and that `$i$' is monic.

The coalgebra structure comes from the fact that $\MO$ is a ring spectrum, and from the K\"unneth isomorphism \rednote{(which is define how?)}. Recall that the classifying map for the cross product bundle gives maps $MO(k)\wedge MO(k')\to MO(k+k')$, and there is always an inclusion $S^k\to MO(k)$ of the fibre at the basepoint. These maps give ring spectrum structure maps which along with the K\"unneth isomorphism give coalgebra structure maps:
\[\xymatrix@R=.4cm{
&%r1c1
\MO\wedge \MO
\ar[r]
&%r1c3
\MO&%r1c4
&
\mathbb{S}
\ar[r]
&%r1c5
\MO\\%r1c6
H^*(\MO)\otimes H^*(\MO)
&%r2c1
H^*(\MO\wedge \MO)
\ar[l]
&%r2c2
H^*(\MO)
\ar[l]
&%r2c3
&%r2c4
\F_2&%r2c5
H^*(\MO)
\ar[l]
%r2c6
\\
x\otimes y
\ar@{|->}[d]^-{\Squ^i}
&
x\wedge y\ar@{|->}[l]
\ar@{|->}[d]^-{\Squ^i}
\\
\sum \Squ^{i'}x\otimes \Squ^{i''}y&
\Squ^i(x\wedge y)\ar@{|->}[l]
}\]
The little box in the bottom right demonstrates that as the coalgebra structure on $\Steen$ is defined using the Cartan formula, the K\"unneth map is $\Steen$-linear as needed.
\end{proof}
Choose free $\Steen$-module generators $\{v_\alpha\}$ for $H^*(\MO)$ of degree $|\alpha|$. We will show that:
\begin{thm*}
The induced map
\[v:\MO\to \prod_{\alpha}\Sigma^{|\alpha|}H=\bigvee_{\alpha}\Sigma^{|\alpha|}H\]
is a weak homotopy equivalence, so that $\MO$ is a graded Eilenberg-MacLane spectrum.
\end{thm*}
\begin{proof}
We'll need to recall a little bit of theory:
\begin{itemize}\squishlist
\item Associated with each abelian group is a $(-1)$-connected spectrum $M_G$ with integral homology $G$ concentrated in degree 0, the \textbf{Moore spectrum}. Now we write $EG$ for $E\wedge M_G$, and in particular, use the notation $SG$ for $M_G$ itself from now on.

We now have a homology theory $\pi_{*}(\DASH ;G)$, with terms
\[\pi_{n}(X;G):=[S^n,X\wedge SG]=[S^n,XG],\]
and long exact sequences
\[\xymatrix{\cdots\ar[r]&
\pi_n(X;\Z/p^{k-1})\ar[r]&
\pi_n(X;\Z/p^{k})\ar[r]&
\pi_n(X;\Z/p)\ar[r]&
\cdots}\]
which shows inductively that if a map of spectra is a $\pi(\DASH ,\Z/p)$ isomorphism, then it's a $\pi(\DASH ,\Z/p^k)$ isomorphism for all $k\leq\infty$.
\item There's a Whitehead theorem for spectra, stating that a map of spectra which is a mod $p$ homology isomorphism is a mod $p$ homotopy isomorphism.
\end{itemize}
Now we can write the proof:
\begin{itemise}
\item As $\frakN_*\cong\pi_{*}(\MO)$, the latter is an $\F_2$-vector space. Thus $\pi_{*}(\MO;\Z/p)$ is trivial for $p$ an odd prime (by the long exact sequence coming from $0\to \Z\to \Z\to \Z/p\to 0$). Thus $v$ is an isomorphism in $\Z/p^\infty$ homotopy for odd primes $p$.
\item Also, $v$ is an isomorphism in $\Q$ homotopy.
\item We saw above that $v$ is an isomorphism in mod 2 cohomology, thus in mod 2 homology, thus in $\Z/2^\infty$ homotopy.
\item We can now use the long exact sequence associated with
\[0\to\Z\to \Q\to \bigoplus\Z/p^\infty\to0\]
to see that $v$ is a weak homotopy equivalence.\qedhere
\end{itemise}

\end{proof}
To finally answer the question, we can show that the diagram
\[\xymatrix{
\frakN_*(X)
\ar[r]^-{\cong }
\ar[dr]_-{(f:M\to X)\mapsto f_*[M]}
&%r1c1
\MO_*(X)
\ar[d]^-{u_*}
\\%r1c2
&%r2c1
H_*(X)%r2c2
}\]
commutes, and as $u_*$ is surjective (as $u$ may be taken to be a generator $v_\alpha$), we see that \textbf{every} element of $H_*(X;\F_2)$ can be realised by a singular manifold.



\end{document}











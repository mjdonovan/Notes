% !TEX root = z_output/ModThyShvs.tex
%%%%%%%%%%%%%%%%%%%%%%%%%%%%%%%%%%%%%%%%%%%%%%%%%%%%%%%%%%%%%%%%%%%%%%%%%%%%%%%%
%%%%%%%%%%%%%%%%%%%%%%%%%%% 80 characters %%%%%%%%%%%%%%%%%%%%%%%%%%%%%%%%%%%%%%
%%%%%%%%%%%%%%%%%%%%%%%%%%%%%%%%%%%%%%%%%%%%%%%%%%%%%%%%%%%%%%%%%%%%%%%%%%%%%%%%
\documentclass[11pt]{article}
\usepackage{fullpage}
\usepackage{amsmath,amsthm,amssymb}
\usepackage{mathrsfs,nicefrac}
\usepackage{amssymb}
\usepackage{epsfig}
\usepackage[all]{xy}
\usepackage{sseq}
\usepackage{tocloft}
\usepackage{cancel}
\usepackage[strict]{changepage}
\usepackage{color}
\usepackage{tikz}
\usepackage{extpfeil}
\usepackage{version}
%\usepackage{ifthen}
%Used for disabling hyperref
\ifx\dontloadhyperref\undefined
%\usepackage[pdftex,pdfborder={0 0 0 [1 1]}]{hyperref}
\usepackage[pdftex,pdfborder={0 0 .5 [1 1]}]{hyperref}
\else
\providecommand{\texorpdfstring}[2]{#1}
\fi

%>>>>>>>>>>>>>>>>>>>>>>>>>>>>>>
%<<<       Better ToC       <<<
%>>>>>>>>>>>>>>>>>>>>>>>>>>>>>>
\setlength{\cftbeforesecskip}{0.5ex}

%>>>>>>>>>>>>>>>>>>>>>>>>>>>>>>
%<<<      Hyperref mod      <<<
%>>>>>>>>>>>>>>>>>>>>>>>>>>>>>>

%needs more testing
\newcounter{dummyforrefstepcounter}
\newcommand{\labelRIGHTHERE}[1]
{\refstepcounter{dummyforrefstepcounter}\label{#1}}


%>>>>>>>>>>>>>>>>>>>>>>>>>>>>>>
%<<<  Theorem Environments  <<<
%>>>>>>>>>>>>>>>>>>>>>>>>>>>>>>
\ifx\dontloaddefinitionsoftheoremenvironments\undefined
\theoremstyle{plain}
\newtheorem{thm}{Theorem}[section]
\newtheorem*{thm*}{Theorem}
\newtheorem{lem}[thm]{Lemma}
\newtheorem*{lem*}{Lemma}
\newtheorem{prop}[thm]{Proposition}
\newtheorem*{prop*}{Proposition}
\newtheorem{cor}[thm]{Corollary}
\newtheorem*{cor*}{Corollary}
\newtheorem{defprop}[thm]{Definition-Proposition}
\newtheorem*{punchline}{Punchline}

\theoremstyle{definition}
\newtheorem{defn}{Definition}[section]
\newtheorem*{defn*}{Definition}
\newtheorem{exmp}{Example}[section]
\newtheorem*{exmp*}{Example}
\newtheorem{asspt}{Assumption}[section]
\newtheorem{notation}{Notation}[section]
\newtheorem{exercise}{Exercise}[section]
\newtheorem*{fact*}{Fact}
\newtheorem*{rmk*}{Remark}
\newtheorem{fact}{Fact}
\newtheorem*{aside}{Aside}
\newtheorem*{question}{Question}
\newtheorem*{answer}{Answer}

\else\relax\fi

%>>>>>>>>>>>>>>>>>>>>>>>>>>>>>>
%<<<      Fields, etc.      <<<
%>>>>>>>>>>>>>>>>>>>>>>>>>>>>>>
\DeclareSymbolFont{AMSb}{U}{msb}{m}{n}
\DeclareMathSymbol{\N}{\mathbin}{AMSb}{"4E}
\DeclareMathSymbol{\Octonions}{\mathbin}{AMSb}{"4F}
\DeclareMathSymbol{\Z}{\mathbin}{AMSb}{"5A}
\DeclareMathSymbol{\R}{\mathbin}{AMSb}{"52}
\DeclareMathSymbol{\Q}{\mathbin}{AMSb}{"51}
\DeclareMathSymbol{\PP}{\mathbin}{AMSb}{"50}
\DeclareMathSymbol{\I}{\mathbin}{AMSb}{"49}
\DeclareMathSymbol{\C}{\mathbin}{AMSb}{"43}
\DeclareMathSymbol{\A}{\mathbin}{AMSb}{"41}
\DeclareMathSymbol{\F}{\mathbin}{AMSb}{"46}
\DeclareMathSymbol{\Quaternions}{\mathbin}{AMSb}{"48}


%>>>>>>>>>>>>>>>>>>>>>>>>>>>>>>
%<<<       Operators        <<<
%>>>>>>>>>>>>>>>>>>>>>>>>>>>>>>
\DeclareMathOperator{\ad}{\textbf{ad}}
\DeclareMathOperator{\coker}{coker}
\renewcommand{\ker}{\textup{ker}\,}
\DeclareMathOperator{\End}{End}
\DeclareMathOperator{\Aut}{Aut}
\DeclareMathOperator{\Hom}{Hom}
\DeclareMathOperator{\Maps}{Maps}
\DeclareMathOperator{\Mor}{Mor}
\DeclareMathOperator{\Gal}{Gal}
\DeclareMathOperator{\Ext}{Ext}
\DeclareMathOperator{\Tor}{Tor}
\DeclareMathOperator{\Map}{Map}
\DeclareMathOperator{\Der}{Der}
\DeclareMathOperator{\Rad}{Rad}
\DeclareMathOperator{\rank}{rank}
\DeclareMathOperator{\ArfInvariant}{Arf}
\DeclareMathOperator{\KervaireInvariant}{Ker}
\DeclareMathOperator{\im}{im}
\DeclareMathOperator{\coim}{coim}
\DeclareMathOperator{\trace}{tr}
\DeclareMathOperator{\supp}{supp}
\DeclareMathOperator{\ann}{ann}
\DeclareMathOperator{\spec}{Spec}
\DeclareMathOperator{\proj}{Proj}
\DeclareMathOperator{\fiber}{F}
\DeclareMathOperator{\cofiber}{C}
\DeclareMathOperator{\cone}{cone}
\DeclareMathOperator{\Skel}{Sk}
\DeclareMathOperator{\conn}{conn}
\DeclareMathOperator{\colim}{colim}
\DeclareMathOperator{\limit}{lim}

%>>>>>>>>>>>>>>>>>>>>>>>>>>>>>>
%<<<   Cohomology Theories  <<<
%>>>>>>>>>>>>>>>>>>>>>>>>>>>>>>
\DeclareMathOperator{\KR}{{K\R}}
\DeclareMathOperator{\KO}{{KO}}
\DeclareMathOperator{\K}{{K}}
\DeclareMathOperator{\OmegaO}{{\Omega_{\Octonions}}}

%>>>>>>>>>>>>>>>>>>>>>>>>>>>>>>
%<<<   Algebraic Geometry   <<<
%>>>>>>>>>>>>>>>>>>>>>>>>>>>>>>
\DeclareMathOperator{\Spec}{Spec}
\DeclareMathOperator{\Proj}{Proj}
\DeclareMathOperator{\Sing}{Sing}
\DeclareMathOperator{\shfHom}{\mathscr{H}\textit{\!\!om}}
\newcommand{\WeilDivisors}{\textup{Div}}
\newcommand{\CartierDivisors}{\textup{CaDiv}}
\newcommand{\PrincipalWeilDivisors}{\textup{PrDiv}}
\newcommand{\LocallyPrincipalWeilDivisors}{\textup{LPDiv}}
\newcommand{\PrincipalCartierDivisors}{\textup{PrCaDiv}}
\newcommand{\DivisorClass}{\textup{Cl}}
\newcommand{\CartierClass}{\textup{CaCl}}
\newcommand{\Picard}{\textup{Pic}}
\DeclareMathOperator{\Frob}{Frob}


%>>>>>>>>>>>>>>>>>>>>>>>>>>>>>>
%<<<  Mathematical Objects  <<<
%>>>>>>>>>>>>>>>>>>>>>>>>>>>>>>
\newcommand{\sll}{\mathfrak{sl}}
\newcommand{\gl}{\mathfrak{gl}}
\newcommand{\GL}{\mbox{GL}}
\newcommand{\PGL}{\mbox{PGL}}
\newcommand{\SL}{\mbox{SL}}
\newcommand{\Mat}{\mbox{Mat}}
\newcommand{\Gr}{\textup{Gr}}
\newcommand{\Squ}{\textup{Sq}}
\newcommand{\catSet}{\textit{Sets}}
\newcommand{\RP}{{\R\PP}}
\newcommand{\CP}{{\C\PP}}
\newcommand{\Steen}{\mathscr{A}}
\newcommand{\Orth}{\textup{\textbf{O}}}

%>>>>>>>>>>>>>>>>>>>>>>>>>>>>>>
%<<<  Mathematical Symbols  <<<
%>>>>>>>>>>>>>>>>>>>>>>>>>>>>>>
\newcommand{\DASH}{\textup{---}}
\newcommand{\op}{\textup{op}}
\newcommand{\ob}{\textup{ob}\,}
\newcommand{\ho}{\textup{ho}}
\newcommand{\st}{\textup{st}}
\newcommand{\id}{\textup{id}}
\newcommand{\Bullet}{\ensuremath{\bullet} }

%>>>>>>>>>>>>>>>>>>>>>>>>>>>>>>
%<<<      Some Arrows       <<<
%>>>>>>>>>>>>>>>>>>>>>>>>>>>>>>
\let\shortmapsto\mapsto
\let\mapsto\longmapsto
\newcommand{\mapsfrom}{\,\reflectbox{$\mapsto$}\ }
\newcommand{\bigrightsquig}{\scalebox{2}{\ensuremath{\rightsquigarrow}}}
\newcommand{\bigleftsquig}{\reflectbox{\scalebox{2}{\ensuremath{\rightsquigarrow}}}}

%\newcommand{\cofibration}{\xhookrightarrow{\phantom{\ \,{\sim\!}\ \ }}}
%\newcommand{\fibration}{\xtwoheadrightarrow{\phantom{\sim\!}}}
%\newcommand{\acycliccofibration}{\xhookrightarrow{\ \,{\sim\!}\ \ }}
%\newcommand{\acyclicfibration}{\xtwoheadrightarrow{\sim\!}}
%\newcommand{\leftcofibration}{\xhookleftarrow{\phantom{\ \,{\sim\!}\ \ }}}
%\newcommand{\leftfibration}{\xtwoheadleftarrow{\phantom{\sim\!}}}
%\newcommand{\leftacycliccofibration}{\xhookleftarrow{\ \ {\sim\!}\,\ }}
%\newcommand{\leftacyclicfibration}{\xtwoheadleftarrow{\sim\!}}
%\newcommand{\weakequiv}{\xrightarrow{\ \,\sim\,\ }}
%\newcommand{\leftweakequiv}{\xleftarrow{\ \,\sim\,\ }}

\newcommand{\cofibration}
{\xhookrightarrow{\phantom{\ \,{\raisebox{-.3ex}[0ex][0ex]{\scriptsize$\sim$}\!}\ \ }}}
\newcommand{\fibration}
{\xtwoheadrightarrow{\phantom{\raisebox{-.3ex}[0ex][0ex]{\scriptsize$\sim$}\!}}}
\newcommand{\acycliccofibration}
{\xhookrightarrow{\ \,{\raisebox{-.55ex}[0ex][0ex]{\scriptsize$\sim$}\!}\ \ }}
\newcommand{\acyclicfibration}
{\xtwoheadrightarrow{\raisebox{-.6ex}[0ex][0ex]{\scriptsize$\sim$}\!}}
\newcommand{\leftcofibration}
{\xhookleftarrow{\phantom{\ \,{\raisebox{-.3ex}[0ex][0ex]{\scriptsize$\sim$}\!}\ \ }}}
\newcommand{\leftfibration}
{\xtwoheadleftarrow{\phantom{\raisebox{-.3ex}[0ex][0ex]{\scriptsize$\sim$}\!}}}
\newcommand{\leftacycliccofibration}
{\xhookleftarrow{\ \ {\raisebox{-.55ex}[0ex][0ex]{\scriptsize$\sim$}\!}\,\ }}
\newcommand{\leftacyclicfibration}
{\xtwoheadleftarrow{\raisebox{-.6ex}[0ex][0ex]{\scriptsize$\sim$}\!}}
\newcommand{\weakequiv}
{\xrightarrow{\ \,\raisebox{-.3ex}[0ex][0ex]{\scriptsize$\sim$}\,\ }}
\newcommand{\leftweakequiv}
{\xleftarrow{\ \,\raisebox{-.3ex}[0ex][0ex]{\scriptsize$\sim$}\,\ }}


%>>>>>>>>>>>>>>>>>>>>>>>>>>>>>>
%<<<     Greek Letters      <<<
%>>>>>>>>>>>>>>>>>>>>>>>>>>>>>>
%\newcommand{\oldphi}{\phi}
%\renewcommand{\phi}{\varphi}
\let\oldphi\phi
\let\phi\varphi
\renewcommand{\to}{\longrightarrow}
\newcommand{\eps}{\varepsilon}

%>>>>>>>>>>>>>>>>>>>>>>>>>>>>>>
%<<<  1st-4th & parentheses <<<
%>>>>>>>>>>>>>>>>>>>>>>>>>>>>>>
\newcommand{\first}{^\text{st}}
\newcommand{\second}{^\text{nd}}
\newcommand{\third}{^\text{rd}}
\newcommand{\fourth}{^\text{th}}
\newcommand{\ZEROTH}{$0^\text{th}$ }
\newcommand{\FIRST}{$1^\text{st}$ }
\newcommand{\SECOND}{$2^\text{nd}$ }
\newcommand{\THIRD}{$3^\text{rd}$ }
\newcommand{\FOURTH}{$4^\text{th}$ }
\newcommand{\iTH}{$i^\text{th}$ }
\newcommand{\jTH}{$j^\text{th}$ }
\newcommand{\nTH}{$n^\text{th}$ }

%>>>>>>>>>>>>>>>>>>>>>>>>>>>>>>
%<<<    upright commands    <<<
%>>>>>>>>>>>>>>>>>>>>>>>>>>>>>>
\newcommand{\upcol}{\textup{:}}
\newcommand{\upsemi}{\textup{;}}
\providecommand{\lparen}{\textup{(}}
\providecommand{\rparen}{\textup{)}}
\renewcommand{\lparen}{\textup{(}}
\renewcommand{\rparen}{\textup{)}}
\newcommand{\Iff}{\emph{iff} }

%>>>>>>>>>>>>>>>>>>>>>>>>>>>>>>
%<<<     Environments       <<<
%>>>>>>>>>>>>>>>>>>>>>>>>>>>>>>
\newcommand{\squishlist}
{ %\setlength{\topsep}{100pt} doesn't seem to do anything.
  \setlength{\itemsep}{.5pt}
  \setlength{\parskip}{0pt}
  \setlength{\parsep}{0pt}}
\newenvironment{itemise}{
\begin{list}{\textup{$\rightsquigarrow$}}
   {  \setlength{\topsep}{1mm}
      \setlength{\itemsep}{1pt}
      \setlength{\parskip}{0pt}
      \setlength{\parsep}{0pt}
   }
}{\end{list}\vspace{-.1cm}}
\newcommand{\INDENT}{\textbf{}\phantom{space}}
\renewcommand{\INDENT}{\rule{.7cm}{0cm}}

\newcommand{\itm}[1][$\rightsquigarrow$]{\item[{\makebox[.5cm][c]{\textup{#1}}}]}

\newcommand{\rednote}[1]{{\color{red}#1}\scalebox{.1}{rednote}}
\newcommand{\bluenote}[1]{{\color{blue}#1}\scalebox{.1}{rednote}}
\newcommand{\funcdef}[4]{\begin{align*}
#1&\to #2\\
#3&\mapsto#4
\end{align*}}

%\newcommand{\comment}[1]{}

%>>>>>>>>>>>>>>>>>>>>>>>>>>>>>>
%<<<       Categories       <<<
%>>>>>>>>>>>>>>>>>>>>>>>>>>>>>>
\newcommand{\Ens}{{\mathscr{E}ns}}
\DeclareMathOperator{\Sheaves}{{\mathsf{Shf}}}
\DeclareMathOperator{\Presheaves}{{\mathsf{PreShf}}}
\DeclareMathOperator{\Varieties}{{\mathsf{Var}}}
\DeclareMathOperator{\Schemes}{{\mathsf{Sch}}}
\DeclareMathOperator{\Rings}{{\mathsf{Rings}}}
\DeclareMathOperator{\AbGp}{{\mathsf{AbGp}}}
\DeclareMathOperator{\Modules}{{\mathsf{\!-Mod}}}
\DeclareMathOperator{\QuasiCoherent}{{\mathsf{QCoh}}}
\DeclareMathOperator{\Coherent}{{\mathsf{Coh}}}
\DeclareMathOperator{\GSW}{{\mathcal{SW}^G}}
\DeclareMathOperator{\Burnside}{{\mathsf{Burn}}}
\DeclareMathOperator{\GSet}{{G\mathsf{Set}}}
\DeclareMathOperator{\FinGSet}{{G\mathsf{Set}^\textup{fin}}}
\DeclareMathOperator{\HSet}{{H\mathsf{Set}}}
\DeclareMathOperator{\Cat}{{\mathsf{Cat}}}
\DeclareMathOperator{\Orb}{{\mathsf{Orb}}}
\DeclareMathOperator{\Set}{{\mathsf{Set}}}
\DeclareMathOperator{\sSet}{{\mathsf{sSet}}}
\DeclareMathOperator{\Top}{{\mathsf{Top}}}
\DeclareMathOperator{\GSpectra}{{G-\mathsf{Spectra}}}

%>>>>>>>>>>>>>>>>>>>>>>>>>>>>>>
%<<<     Script Letters     <<<
%>>>>>>>>>>>>>>>>>>>>>>>>>>>>>>
\newcommand{\scrQ}{\mathscr{Q}}
\newcommand{\scrW}{\mathscr{W}}
\newcommand{\scrE}{\mathscr{E}}
\newcommand{\scrR}{\mathscr{R}}
\newcommand{\scrT}{\mathscr{T}}
\newcommand{\scrY}{\mathscr{Y}}
\newcommand{\scrU}{\mathscr{U}}
\newcommand{\scrI}{\mathscr{I}}
\newcommand{\scrO}{\mathscr{O}}
\newcommand{\scrP}{\mathscr{P}}
\newcommand{\scrA}{\mathscr{A}}
\newcommand{\scrS}{\mathscr{S}}
\newcommand{\scrD}{\mathscr{D}}
\newcommand{\scrF}{\mathscr{F}}
\newcommand{\scrG}{\mathscr{G}}
\newcommand{\scrH}{\mathscr{H}}
\newcommand{\scrJ}{\mathscr{J}}
\newcommand{\scrK}{\mathscr{K}}
\newcommand{\scrL}{\mathscr{L}}
\newcommand{\scrZ}{\mathscr{Z}}
\newcommand{\scrX}{\mathscr{X}}
\newcommand{\scrC}{\mathscr{C}}
\newcommand{\scrV}{\mathscr{V}}
\newcommand{\scrB}{\mathscr{B}}
\newcommand{\scrN}{\mathscr{N}}
\newcommand{\scrM}{\mathscr{M}}

%>>>>>>>>>>>>>>>>>>>>>>>>>>>>>>
%<<<     Fractur Letters    <<<
%>>>>>>>>>>>>>>>>>>>>>>>>>>>>>>
\newcommand{\frakQ}{\mathfrak{Q}}
\newcommand{\frakW}{\mathfrak{W}}
\newcommand{\frakE}{\mathfrak{E}}
\newcommand{\frakR}{\mathfrak{R}}
\newcommand{\frakT}{\mathfrak{T}}
\newcommand{\frakY}{\mathfrak{Y}}
\newcommand{\frakU}{\mathfrak{U}}
\newcommand{\frakI}{\mathfrak{I}}
\newcommand{\frakO}{\mathfrak{O}}
\newcommand{\frakP}{\mathfrak{P}}
\newcommand{\frakA}{\mathfrak{A}}
\newcommand{\frakS}{\mathfrak{S}}
\newcommand{\frakD}{\mathfrak{D}}
\newcommand{\frakF}{\mathfrak{F}}
\newcommand{\frakG}{\mathfrak{G}}
\newcommand{\frakH}{\mathfrak{H}}
\newcommand{\frakJ}{\mathfrak{J}}
\newcommand{\frakK}{\mathfrak{K}}
\newcommand{\frakL}{\mathfrak{L}}
\newcommand{\frakZ}{\mathfrak{Z}}
\newcommand{\frakX}{\mathfrak{X}}
\newcommand{\frakC}{\mathfrak{C}}
\newcommand{\frakV}{\mathfrak{V}}
\newcommand{\frakB}{\mathfrak{B}}
\newcommand{\frakN}{\mathfrak{N}}
\newcommand{\frakM}{\mathfrak{M}}

\newcommand{\frakq}{\mathfrak{q}}
\newcommand{\frakw}{\mathfrak{w}}
\newcommand{\frake}{\mathfrak{e}}
\newcommand{\frakr}{\mathfrak{r}}
\newcommand{\frakt}{\mathfrak{t}}
\newcommand{\fraky}{\mathfrak{y}}
\newcommand{\fraku}{\mathfrak{u}}
\newcommand{\fraki}{\mathfrak{i}}
\newcommand{\frako}{\mathfrak{o}}
\newcommand{\frakp}{\mathfrak{p}}
\newcommand{\fraka}{\mathfrak{a}}
\newcommand{\fraks}{\mathfrak{s}}
\newcommand{\frakd}{\mathfrak{d}}
\newcommand{\frakf}{\mathfrak{f}}
\newcommand{\frakg}{\mathfrak{g}}
\newcommand{\frakh}{\mathfrak{h}}
\newcommand{\frakj}{\mathfrak{j}}
\newcommand{\frakk}{\mathfrak{k}}
\newcommand{\frakl}{\mathfrak{l}}
\newcommand{\frakz}{\mathfrak{z}}
\newcommand{\frakx}{\mathfrak{x}}
\newcommand{\frakc}{\mathfrak{c}}
\newcommand{\frakv}{\mathfrak{v}}
\newcommand{\frakb}{\mathfrak{b}}
\newcommand{\frakn}{\mathfrak{n}}
\newcommand{\frakm}{\mathfrak{m}}

%>>>>>>>>>>>>>>>>>>>>>>>>>>>>>>
%<<<  Caligraphic Letters   <<<
%>>>>>>>>>>>>>>>>>>>>>>>>>>>>>>
\newcommand{\calQ}{\mathcal{Q}}
\newcommand{\calW}{\mathcal{W}}
\newcommand{\calE}{\mathcal{E}}
\newcommand{\calR}{\mathcal{R}}
\newcommand{\calT}{\mathcal{T}}
\newcommand{\calY}{\mathcal{Y}}
\newcommand{\calU}{\mathcal{U}}
\newcommand{\calI}{\mathcal{I}}
\newcommand{\calO}{\mathcal{O}}
\newcommand{\calP}{\mathcal{P}}
\newcommand{\calA}{\mathcal{A}}
\newcommand{\calS}{\mathcal{S}}
\newcommand{\calD}{\mathcal{D}}
\newcommand{\calF}{\mathcal{F}}
\newcommand{\calG}{\mathcal{G}}
\newcommand{\calH}{\mathcal{H}}
\newcommand{\calJ}{\mathcal{J}}
\newcommand{\calK}{\mathcal{K}}
\newcommand{\calL}{\mathcal{L}}
\newcommand{\calZ}{\mathcal{Z}}
\newcommand{\calX}{\mathcal{X}}
\newcommand{\calC}{\mathcal{C}}
\newcommand{\calV}{\mathcal{V}}
\newcommand{\calB}{\mathcal{B}}
\newcommand{\calN}{\mathcal{N}}
\newcommand{\calM}{\mathcal{M}}

%>>>>>>>>>>>>>>>>>>>>>>>>>>>>>>
%<<<<<<<<<DEPRECIATED<<<<<<<<<<
%>>>>>>>>>>>>>>>>>>>>>>>>>>>>>>

%%% From Kac's template
% 1-inch margins, from fullpage.sty by H.Partl, Version 2, Dec. 15, 1988.
%\topmargin 0pt
%\advance \topmargin by -\headheight
%\advance \topmargin by -\headsep
%\textheight 9.1in
%\oddsidemargin 0pt
%\evensidemargin \oddsidemargin
%\marginparwidth 0.5in
%\textwidth 6.5in
%
%\parindent 0in
%\parskip 1.5ex
%%\renewcommand{\baselinestretch}{1.25}

%%% From the net
%\newcommand{\pullbackcorner}[1][dr]{\save*!/#1+1.2pc/#1:(1,-1)@^{|-}\restore}
%\newcommand{\pushoutcorner}[1][dr]{\save*!/#1-1.2pc/#1:(-1,1)@^{|-}\restore}










\newcommand{\NewLecture}[3]{\section{#1 {\small(#2/#3/2011)}}}
\makeatletter
\def\@seccntformat#1{Lecture \csname the#1\endcsname.\quad}
\makeatother
\title{Topics in the Moduli Theory of Sheaves}
\author{Baosen Wu}
\date{}

\begin{document}
\maketitle
\NewLecture{}{31}{8}
Throughout this class, we'll take $k=\C$. (Much of the material will work for
other algebraically closed fields, however).
\begin{exmp*}
Projective spaces $\PP(V)$. Choose homogeneous coordinates $x_1,\ldots,x_n$ on
$V$. There is a tautological bundle $\calO_{\PP(V)}(-1)\downarrow \PP(V)$. Its
dual, $\calO(1)$, has $H^0(\calO(1))=k\{x_1,\ldots,x_n\}$.
\end{exmp*}
\begin{exmp*}
The grassmannian $\Gr(r,n)$. This is the set of $r$-dimensional subspaces of
$V$, where $\dim V=n$. It too has a tautological locally free sheaf.
\end{exmp*}
\begin{exmp*}
The space of projective lines in $\PP^n$. Equal to $\Gr(2,n+1)$.
\end{exmp*}
\begin{thm*}[classical]
A smooth cubic surface in $\PP^3$ contains 27 lines.
\end{thm*}
\begin{proof}[Proof approach]
Fix the cubic $S$, let $M$ be the set of lines in $S$. Then $M\subset
\Gr(2,4)=X$. The idea is then to construct a vector bundle $E\downarrow X$ and a
section of $E$ such that $M=s^{-1}(0)$. Now we'll have $\rank E=\dim X=4$, and
$M=s^{-1}(0)$ corresponds to the Euler class (when $s$ is `generic enough' --- I
don't know what this means). The Euler class is the top Chern class, $c_4(E)$,
so we just need to show that $c_4(E)=27$ {\small(which we will not do in this
class, at least right now)}.

For a fiberwise indiction of how we'll construct $E$, fix homogeneous
coordinates $x_0,x_1,x_2,x_3$ of $\PP^3$. Then the cubic equation for $S$ can be
viewed as a section $f\in\Gamma(\PP^3,\calO(3))$. Moreover, $\ell\subset S$ \Iff
$f|_\ell=0$. Now any line $\ell$ is isomorphic to $\PP^1$, and we can assign to
$\ell$ the global sections $H^0(\ell,\calO_\ell(3))$. \textbf{I'm lost.}

Another way to describe $E$ is as follows. Let $U$ be the tautological bundle on
$\Gr(2,4)$:
\[\xymatrix{
\ar[d]^{p_1}\ar[dr]^(.7){p_2}U\ar@{^{(}->}[r]&\Gr(2,4)\times
\PP^3\ar[d]\ar[dl]\\
\Gr(2,4)&\PP^3
}\]
The bundle $\calO_{\PP^3}(3)$ can be pulled back to $U$ and then pushed down to
$\Gr(2,4)$. That is, we set $E=p_{1*}p_2^*\calO(3)\downarrow\Gr(2,4)$. To see
that this is a vector bundle, we compute $(R^1p_{1*})p_2^*\calO(3)=0$
(\textbf{how? What would this mean?}). (\textbf{I should figure out why this $E$
works.})
\end{proof}
\begin{exmp*}[Moduli space of curves]
Define:
\[M_g:=\{\text{connected smooth curves with genus $g$}\}/\sim,\]
where $\sim$ denotes isomorphism of curves. There is an object $\overline{M_g}$,
the modular compactification, which is a proper Deligne-Mumford stack.
\end{exmp*}
\begin{exmp*}[Moduli of stable maps]
Fix a projective, smooth variety $X$ and fix $\beta\in H_2(X;\Z)$ (singular
homology). Define:
\[M_g(X;\beta):=\{f:C\to X\ |\ \text{$C$ is connected, smooth genus $g$ curve
with $f_*[C]=\beta$}\}/\sim.\]
Here $\sim$ means that there exists an isomorphism $C\to C'$ making the evident
triangle commute. Similarly, we get a proper Deligne-Mumford stack
$\overline{M_g(X;\beta)}$. We can define Gromov-Witten invariants as integrals
over $\overline{M_g(X;\beta)}$.
\end{exmp*}
\begin{exmp*}[Moduli of stable bundles over $C$]
\[M(r,d):=\{\text{rank $r$ degree $d$ ``stable bundles'' $E$ on $C$}\}/\sim.\]
Here, $\sim$ stands for isomorphism of bundles. Firstly, the degree is defined
to be:
\[\deg(E)=\chi(E)-r\chi(\calO)=\chi(E)-r(1-g).\]
The degree can also be defined as the degree\footnote{Degree may only be defined
for line bundles on curves, in which case, $\deg(\scrL)=c_1(\scrL)([c])$, the
first Chern class evaluated on the fundamental class. If the curve is smooth,
then it is also the degree of the corresponding Weil divisor (each prime divisor
has degree one).} of the determinant bundle (which is the top exterior power of
the bundle, a line bundle). A vector bundle $E\downarrow C$ is called semistable
if for every proper subbudle $F\subset E$:
\[\frac{\deg(F)}{\rank(F)}\leq\frac{\deg(E)}{\rank(E)}=:\mu(E).\]
We call $\mu(E)$ the slope of $E$. If this inequality is strict, then it is
called stable.

Note that $M(1,0)=JC$, the moduli space of degree zero line bundles. According
to wikipedia, it is the connected component of the identity in the Picard group
of C, hence an abelian variety.

\textbf{At this point I had to leave early. How rude!}
\end{exmp*}
\NewLecture{Coherent Sheaves}{2}{9}
Eventually we'll be interested in both the Harder-Narasimhan filtration and the
Jordan-H\"older filtration of a coherent sheaf. To follow, in due course.

For today, let $X$ be a smooth projective variety, and let $\scrE$ be a coherent
sheaf on $X$.
\begin{defn*}
$\scrE$ is \emph{torsion free} if $\scrE_x$ is a torsion free $\calO_x$-module
for all $x\in X$.
\end{defn*}
\begin{defn*}
We define the \emph{dual} $\check\scrE:=\shfHom(\scrE,\calO)$. It is also
coherent, and there is a natural map $f:\scrE\to\check{\check\scrE}$. $\scrE$ is
called \emph{reflexive} if $f$ is an isomorphism. $\scrT:=\ker(f)$ is called the
\emph{torsion subsheaf} of $\scrE$. Of course, $\scrE/\scrT$ is torsion free.
\end{defn*}
\begin{defn*}
$\supp(\scrE):=\{x\in X\,|\,\scrE_x\neq0\}$ is called the \emph{support} of
$\scrE$. When $\scrE$ is coherent, it is a \textbf{closed} subset of $X$. Note
that $\supp(\scrT)\neq X$.
\end{defn*}
\begin{defn*}
$\Sing(\scrE):=\{x\in X\,|\, \scrE_x\text{ is not a free
}\calO_x\text{-module}\}$. This is a proper closed subset of $X$. (Note that
$\calO_\eta$ is a the function \emph{field}, where $\eta$ is the generic point,
so that $\eta\notin\Sing(\scrE)$). I suppose that we call $\scrE$ \emph{smooth}
when $\Sing(\scrE)=\emptyset$. This is a synonym for `locally free' for coherent
sheaves.
\end{defn*}
\begin{thm*}
If $\scrE$ is torsion free, then $\Sing(\scrE)$ has codimension at least
two.\footnote{See R. Friedman, `Algebraic Surfaces and Holomorphic Vector
Bundles'.} In particular, on a curve, a torsion free coherent sheaf is locally
free.
\end{thm*}
\begin{thm*}
If $\scrE$ is reflexive, then $\Sing(\scrE)$ has codimension at least three. In
particular, a reflexive coherent sheaf on a curve or surface is locally free.
\end{thm*}
\begin{defn*}
The \emph{rank} $\rank(\scrE)$ of $\scrE$ is defined to be the dimension of
$\scrE_\eta$ over the function field $\calO_\eta$.
\end{defn*}
\begin{thm*}
Suppose that $\scrE$ is a coherent reflexive sheaf of rank one. Then $\scrE$ is
invertible (i.e.\ locally free of rank one).
\end{thm*}
\begin{proof}
For some $n\gg0$, we have an injection $\calO\to\scrE(n)$ [\textbf{I don't
understand why this is an injection. What if the section you use vanishes
somewhere? Then the map on stalks is zero where it vanishes, right?} like... NO,
it would need to be zero on an open neighbourhood! Actually, you get something
like... ``a module of rk one over a local ring on a curve has no 0 divisors'' or
something, according to saul (This might have been relevant somewhere else)].
Take the kernel, to obtain a short exact sequence
\[0\to\calO\to\scrE(n)\to T\to 0.\]
Then $T$ is a torsion sheaf, so $\check T=0$. Thus, dualising, we have an exact
sequence
\[0\to\check T\to\check\scrE(-n)\to\calO\to\Ext^1(T,\calO)\to\cdots
\textbf{ \ (is this the right Ext?)}\]
Thus $\check \scrE(-n)$ is a coherent sheaf of ideals, say $\scrI_Z$ for some
closed subscheme $Z\subset X$. It is claimed that the dual of any nonzero
coherent sheaf of ideals is locally free, so that as $\scrE$ is reflexive, it is
locally free.
\end{proof}
\subsection*{Coherent sheaves on smooth curves}
Let $C$ be a smooth curve.
\begin{lem*}
Every torsion free sheaf on $C$ is locally free.
\end{lem*}
\begin{proof}
$\calO_x$ is always a PID, and a torsion free, finitely generated module over a
PID is free.
\end{proof}
\noindent
For a counterexample when $C$ is not smooth, try a nodal curve, with node $p$.
Then $\scrI_p$ is torsion free but not locally free.
\begin{lem*}
Let $\scrE$ be coherent on $C$. Then $\scrE=\scrG\oplus\scrT$, where $\scrG$ is
locally free, and $\scrT$ is a torsion sheaf (and thus supported at only
finitely many points).
\end{lem*}
\begin{proof}
There is a short exact sequence $0\to \scrT\to\scrE\to\scrG\to0$. Now $\scrG$ is
torsion free, and so locally free, by the previous lemma. We need to see that
the sequence splits. The obstruction to its splitting lies in
$\Ext^1(\scrG,\scrT)=H^1(\check\scrG\otimes\scrT)$ (simply because
$\Hom(\scrG,\DASH)$ and $\Gamma(\check\scrG\otimes\DASH)$ are the same functor).
This group is zero, as $\check\scrG\otimes\scrT$ is a torsion sheaf, and a
torsion sheaf on a curve is flasque (\textbf{any better reason?}).
\end{proof}
\noindent For a counterexample when $C$ is not smooth, let $C$ be a nodal curve
in $S=\A^2$. Let $\scrI=\scrI_{p\in S}$ be the ideal sheaf of the node, as a
point of $S$. Consider $\scrI|_C:=\scrI\otimes_{\calO_S}\calO_C$ an
$\calO_C$-module. The second lemma should fail for this sheaf. To see why,
locally at $p$, $C$ is defined by $xy=0$. Then $\scrI_p=(x,y)$, and
$\scrI_p|_C=(x,y)\otimes_{k[x,y]}k[x,y]/(xy)$, which is gereated by $x\otimes1$
and $y\otimes1$ as a $k[x,y]/(xy)$-module. Now let $s=xy\otimes1$, a nonzero
element. Then $x\cdot s=y\cdot s=0$, so that $s$ is a torsion element. But the
module $\scrI_p|_C$ cannot be written as the direct sum it needs to in order to
split off this torsion element.
\subsection*{Vector bundles over curves}
\begin{thm*}
Every vector bundle on $\PP^1$ can be written as a direct sum of line bundles
$\calO(a_i)$.
\end{thm*}
\begin{proof}
By induction on rank. Take a vector bundle $E$. Now $E(-i)$ has no section for
$i\gg0$, by the following argument. For $i\gg0$, $H^1(\check E(i-2))=0$, by a
result of Serre\footnote{Hartshorne II.5.2}. However, by Serre duality,
\[0=H^1(\check E(i-2))=\left[H^{1-\dim\PP^1}(({\check E(i-2)})^*\otimes
K_{\PP^1})\right]^*=\left[H^0(E(-i))\right]^*,\]
as the dualising sheaf is $K_{\PP^1}=\calO(-2)$.\footnote{In general, the
dualising sheaf for $\PP^r$ is $K_{\PP^r}=\calO(-r-1)$, and the theorem is that
for any vector bundle $E$, $H^n(E)\cong [H^{r-n}(E^*\otimes K_{\PP^r})]^*$}
Choose the largest $i$ such that $E(-i)$ has a nonzero section. Then there is a
short exact sequence
\[0\to\calO\to E(-i)\to F\to 0,\]
where $F$ is a torsion free sheaf (\textbf{why torsion free?}). By induction,
$F=\bigoplus \calO(a_i)$. Consider:
\[0\to\calO(-1)\to E(-i-1)\to\bigoplus\calO(a_i-1)\to 0.\]
As $H^1(\calO(-1))=[H^0(\calO(1)\otimes\calO(-2))]^*=0$, the short exact
sequence gives a short exact sequence on $H^0$, and as the first two terms have
no global sections, neither does the third. Thus each $a_i\leq0$, and
$\Ext^1(F,\calO)=H^1(\check F\otimes\calO)=\bigoplus H^1(\calO(-a_i))=0$, so the
sequence splits.
\end{proof}
\subsubsection*{Vector bundles over curves of positive genus}
Suppose that $C$ is an elliptic curve.


\end{document}
















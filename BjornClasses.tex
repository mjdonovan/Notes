% !TEX root = z_output/_BjornClasses.tex
%%%%%%%%%%%%%%%%%%%%%%%%%%%%%%%%%%%%%%%%%%%%%%%%%%%%%%%%%%%%%%%%%%%%%%%%%%%%%%%%
%%%%%%%%%%%%%%%%%%%%%%%%%%% 80 characters %%%%%%%%%%%%%%%%%%%%%%%%%%%%%%%%%%%%%%
%%%%%%%%%%%%%%%%%%%%%%%%%%%%%%%%%%%%%%%%%%%%%%%%%%%%%%%%%%%%%%%%%%%%%%%%%%%%%%%%
\documentclass[11pt]{article}
\usepackage{fullpage}
\usepackage{amsmath,amsthm,amssymb}
\usepackage{mathrsfs,nicefrac}
\usepackage{amssymb}
\usepackage{epsfig}
\usepackage[all,2cell]{xy}
\usepackage{sseq}
\usepackage{tocloft}
\usepackage{cancel}
\usepackage[strict]{changepage}
\usepackage{color}
\usepackage{tikz}
\usepackage{extpfeil}
\usepackage{version}
\usepackage{framed}
\definecolor{shadecolor}{rgb}{.925,0.925,0.925}

%\usepackage{ifthen}
%Used for disabling hyperref
\ifx\dontloadhyperref\undefined
%\usepackage[pdftex,pdfborder={0 0 0 [1 1]}]{hyperref}
\usepackage[pdftex,pdfborder={0 0 .5 [1 1]}]{hyperref}
\else
\providecommand{\texorpdfstring}[2]{#1}
\fi
%>>>>>>>>>>>>>>>>>>>>>>>>>>>>>>
%<<<        Versions        <<<
%>>>>>>>>>>>>>>>>>>>>>>>>>>>>>>
%Add in the following line to include all the versions.
%\def\excludeversion#1{\includeversion{#1}}

%>>>>>>>>>>>>>>>>>>>>>>>>>>>>>>
%<<<       Better ToC       <<<
%>>>>>>>>>>>>>>>>>>>>>>>>>>>>>>
\setlength{\cftbeforesecskip}{0.5ex}

%>>>>>>>>>>>>>>>>>>>>>>>>>>>>>>
%<<<      Hyperref mod      <<<
%>>>>>>>>>>>>>>>>>>>>>>>>>>>>>>

%needs more testing
\newcounter{dummyforrefstepcounter}
\newcommand{\labelRIGHTHERE}[1]
{\refstepcounter{dummyforrefstepcounter}\label{#1}}


%>>>>>>>>>>>>>>>>>>>>>>>>>>>>>>
%<<<  Theorem Environments  <<<
%>>>>>>>>>>>>>>>>>>>>>>>>>>>>>>
\ifx\dontloaddefinitionsoftheoremenvironments\undefined
\theoremstyle{plain}
\newtheorem{thm}{Theorem}[section]
\newtheorem*{thm*}{Theorem}
\newtheorem{lem}[thm]{Lemma}
\newtheorem*{lem*}{Lemma}
\newtheorem{prop}[thm]{Proposition}
\newtheorem*{prop*}{Proposition}
\newtheorem{cor}[thm]{Corollary}
\newtheorem*{cor*}{Corollary}
\newtheorem{defprop}[thm]{Definition-Proposition}
\newtheorem*{punchline}{Punchline}
\newtheorem*{conjecture}{Conjecture}
\newtheorem*{claim}{Claim}

\theoremstyle{definition}
\newtheorem{defn}{Definition}[section]
\newtheorem*{defn*}{Definition}
\newtheorem{exmp}{Example}[section]
\newtheorem*{exmp*}{Example}
\newtheorem*{exmps*}{Examples}
\newtheorem*{nonexmp*}{Non-example}
\newtheorem{asspt}{Assumption}[section]
\newtheorem{notation}{Notation}[section]
\newtheorem{exercise}{Exercise}[section]
\newtheorem*{fact*}{Fact}
\newtheorem*{rmk*}{Remark}
\newtheorem{fact}{Fact}
\newtheorem*{aside}{Aside}
\newtheorem*{question}{Question}
\newtheorem*{answer}{Answer}

\else\relax\fi

%>>>>>>>>>>>>>>>>>>>>>>>>>>>>>>
%<<<      Fields, etc.      <<<
%>>>>>>>>>>>>>>>>>>>>>>>>>>>>>>
\DeclareSymbolFont{AMSb}{U}{msb}{m}{n}
\DeclareMathSymbol{\N}{\mathbin}{AMSb}{"4E}
\DeclareMathSymbol{\Octonions}{\mathbin}{AMSb}{"4F}
\DeclareMathSymbol{\Z}{\mathbin}{AMSb}{"5A}
\DeclareMathSymbol{\R}{\mathbin}{AMSb}{"52}
\DeclareMathSymbol{\Q}{\mathbin}{AMSb}{"51}
\DeclareMathSymbol{\PP}{\mathbin}{AMSb}{"50}
\DeclareMathSymbol{\I}{\mathbin}{AMSb}{"49}
\DeclareMathSymbol{\C}{\mathbin}{AMSb}{"43}
\DeclareMathSymbol{\A}{\mathbin}{AMSb}{"41}
\DeclareMathSymbol{\F}{\mathbin}{AMSb}{"46}
\DeclareMathSymbol{\G}{\mathbin}{AMSb}{"47}
\DeclareMathSymbol{\Quaternions}{\mathbin}{AMSb}{"48}


%>>>>>>>>>>>>>>>>>>>>>>>>>>>>>>
%<<<       Operators        <<<
%>>>>>>>>>>>>>>>>>>>>>>>>>>>>>>
\DeclareMathOperator{\ad}{\textbf{ad}}
\DeclareMathOperator{\coker}{coker}
\renewcommand{\ker}{\textup{ker}\,}
\DeclareMathOperator{\End}{End}
\DeclareMathOperator{\Aut}{Aut}
\DeclareMathOperator{\Hom}{Hom}
\DeclareMathOperator{\Maps}{Maps}
\DeclareMathOperator{\Mor}{Mor}
\DeclareMathOperator{\Gal}{Gal}
\DeclareMathOperator{\Ext}{Ext}
\DeclareMathOperator{\Tor}{Tor}
\DeclareMathOperator{\Map}{Map}
\DeclareMathOperator{\Der}{Der}
\DeclareMathOperator{\Rad}{Rad}
\DeclareMathOperator{\rank}{rank}
\DeclareMathOperator{\ArfInvariant}{Arf}
\DeclareMathOperator{\KervaireInvariant}{Ker}
\DeclareMathOperator{\im}{im}
\DeclareMathOperator{\coim}{coim}
\DeclareMathOperator{\trace}{tr}
\DeclareMathOperator{\supp}{supp}
\DeclareMathOperator{\ann}{ann}
\DeclareMathOperator{\spec}{Spec}
\DeclareMathOperator{\SPEC}{\textbf{Spec}}
\DeclareMathOperator{\proj}{Proj}
\DeclareMathOperator{\PROJ}{\textbf{Proj}}
\DeclareMathOperator{\fiber}{F}
\DeclareMathOperator{\cofiber}{C}
\DeclareMathOperator{\cone}{cone}
\DeclareMathOperator{\skel}{sk}
\DeclareMathOperator{\coskel}{cosk}
\DeclareMathOperator{\conn}{conn}
\DeclareMathOperator{\colim}{colim}
\DeclareMathOperator{\limit}{lim}
\DeclareMathOperator{\ch}{ch}
\DeclareMathOperator{\Vect}{Vect}
\DeclareMathOperator{\GrthGrp}{GrthGp}
\DeclareMathOperator{\Sym}{Sym}
\DeclareMathOperator{\Prob}{\mathbb{P}}
\DeclareMathOperator{\Exp}{\mathbb{E}}
\DeclareMathOperator{\GeomMean}{\mathbb{G}}
\DeclareMathOperator{\Var}{Var}
\DeclareMathOperator{\Cov}{Cov}
\DeclareMathOperator{\Sp}{Sp}
\DeclareMathOperator{\Seq}{Seq}
\DeclareMathOperator{\Cyl}{Cyl}
\DeclareMathOperator{\Ev}{Ev}
\DeclareMathOperator{\sh}{sh}
\DeclareMathOperator{\intHom}{\underline{Hom}}
\DeclareMathOperator{\Frac}{frac}



%>>>>>>>>>>>>>>>>>>>>>>>>>>>>>>
%<<<   Cohomology Theories  <<<
%>>>>>>>>>>>>>>>>>>>>>>>>>>>>>>
\DeclareMathOperator{\KR}{{K\R}}
\DeclareMathOperator{\KO}{{KO}}
\DeclareMathOperator{\K}{{K}}
\DeclareMathOperator{\OmegaO}{{\Omega_{\Octonions}}}

%>>>>>>>>>>>>>>>>>>>>>>>>>>>>>>
%<<<   Algebraic Geometry   <<<
%>>>>>>>>>>>>>>>>>>>>>>>>>>>>>>
\DeclareMathOperator{\Spec}{Spec}
\DeclareMathOperator{\Proj}{Proj}
\DeclareMathOperator{\Sing}{Sing}
\DeclareMathOperator{\shfHom}{\mathscr{H}\textit{\!\!om}}
\DeclareMathOperator{\WeilDivisors}{{Div}}
\DeclareMathOperator{\CartierDivisors}{{CaDiv}}
\DeclareMathOperator{\PrincipalWeilDivisors}{{PrDiv}}
\DeclareMathOperator{\LocallyPrincipalWeilDivisors}{{LPDiv}}
\DeclareMathOperator{\PrincipalCartierDivisors}{{PrCaDiv}}
\DeclareMathOperator{\DivisorClass}{{Cl}}
\DeclareMathOperator{\CartierClass}{{CaCl}}
\DeclareMathOperator{\Picard}{{Pic}}
\DeclareMathOperator{\Frob}{Frob}


%>>>>>>>>>>>>>>>>>>>>>>>>>>>>>>
%<<<  Mathematical Objects  <<<
%>>>>>>>>>>>>>>>>>>>>>>>>>>>>>>
\newcommand{\sll}{\mathfrak{sl}}
\newcommand{\gl}{\mathfrak{gl}}
\newcommand{\GL}{\mbox{GL}}
\newcommand{\PGL}{\mbox{PGL}}
\newcommand{\SL}{\mbox{SL}}
\newcommand{\Mat}{\mbox{Mat}}
\newcommand{\Gr}{\textup{Gr}}
\newcommand{\Squ}{\textup{Sq}}
\newcommand{\catSet}{\textit{Sets}}
\newcommand{\RP}{{\R\PP}}
\newcommand{\CP}{{\C\PP}}
\newcommand{\Steen}{\mathscr{A}}
\newcommand{\Orth}{\textup{\textbf{O}}}

%>>>>>>>>>>>>>>>>>>>>>>>>>>>>>>
%<<<  Mathematical Symbols  <<<
%>>>>>>>>>>>>>>>>>>>>>>>>>>>>>>
\newcommand{\DASH}{\textup{---}}
\newcommand{\op}{\textup{op}}
\newcommand{\CW}{\textup{CW}}
\newcommand{\ob}{\textup{ob}\,}
\newcommand{\ho}{\textup{ho}}
\newcommand{\st}{\textup{st}}
\newcommand{\id}{\textup{id}}
\newcommand{\Bullet}{\ensuremath{\bullet} }
\newcommand{\sprod}{\wedge}

%>>>>>>>>>>>>>>>>>>>>>>>>>>>>>>
%<<<      Some Arrows       <<<
%>>>>>>>>>>>>>>>>>>>>>>>>>>>>>>
\newcommand{\nt}{\Longrightarrow}
\let\shortmapsto\mapsto
\let\mapsto\longmapsto
\newcommand{\mapsfrom}{\,\reflectbox{$\mapsto$}\ }
\newcommand{\bigrightsquig}{\scalebox{2}{\ensuremath{\rightsquigarrow}}}
\newcommand{\bigleftsquig}{\reflectbox{\scalebox{2}{\ensuremath{\rightsquigarrow}}}}

%\newcommand{\cofibration}{\xhookrightarrow{\phantom{\ \,{\sim\!}\ \ }}}
%\newcommand{\fibration}{\xtwoheadrightarrow{\phantom{\sim\!}}}
%\newcommand{\acycliccofibration}{\xhookrightarrow{\ \,{\sim\!}\ \ }}
%\newcommand{\acyclicfibration}{\xtwoheadrightarrow{\sim\!}}
%\newcommand{\leftcofibration}{\xhookleftarrow{\phantom{\ \,{\sim\!}\ \ }}}
%\newcommand{\leftfibration}{\xtwoheadleftarrow{\phantom{\sim\!}}}
%\newcommand{\leftacycliccofibration}{\xhookleftarrow{\ \ {\sim\!}\,\ }}
%\newcommand{\leftacyclicfibration}{\xtwoheadleftarrow{\sim\!}}
%\newcommand{\weakequiv}{\xrightarrow{\ \,\sim\,\ }}
%\newcommand{\leftweakequiv}{\xleftarrow{\ \,\sim\,\ }}

\newcommand{\cofibration}
{\xhookrightarrow{\phantom{\ \,{\raisebox{-.3ex}[0ex][0ex]{\scriptsize$\sim$}\!}\ \ }}}
\newcommand{\fibration}
{\xtwoheadrightarrow{\phantom{\raisebox{-.3ex}[0ex][0ex]{\scriptsize$\sim$}\!}}}
\newcommand{\acycliccofibration}
{\xhookrightarrow{\ \,{\raisebox{-.55ex}[0ex][0ex]{\scriptsize$\sim$}\!}\ \ }}
\newcommand{\acyclicfibration}
{\xtwoheadrightarrow{\raisebox{-.6ex}[0ex][0ex]{\scriptsize$\sim$}\!}}
\newcommand{\leftcofibration}
{\xhookleftarrow{\phantom{\ \,{\raisebox{-.3ex}[0ex][0ex]{\scriptsize$\sim$}\!}\ \ }}}
\newcommand{\leftfibration}
{\xtwoheadleftarrow{\phantom{\raisebox{-.3ex}[0ex][0ex]{\scriptsize$\sim$}\!}}}
\newcommand{\leftacycliccofibration}
{\xhookleftarrow{\ \ {\raisebox{-.55ex}[0ex][0ex]{\scriptsize$\sim$}\!}\,\ }}
\newcommand{\leftacyclicfibration}
{\xtwoheadleftarrow{\raisebox{-.6ex}[0ex][0ex]{\scriptsize$\sim$}\!}}
\newcommand{\weakequiv}
{\xrightarrow{\ \,\raisebox{-.3ex}[0ex][0ex]{\scriptsize$\sim$}\,\ }}
\newcommand{\leftweakequiv}
{\xleftarrow{\ \,\raisebox{-.3ex}[0ex][0ex]{\scriptsize$\sim$}\,\ }}

%>>>>>>>>>>>>>>>>>>>>>>>>>>>>>>
%<<<    xymatrix Arrows     <<<
%>>>>>>>>>>>>>>>>>>>>>>>>>>>>>>
\newdir{ >}{{}*!/-5pt/@{>}}
\newcommand{\xycof}{\ar@{ >->}}
\newcommand{\xycofib}{\ar@{^{(}->}}
\newcommand{\xycofibdown}{\ar@{_{(}->}}
\newcommand{\xyfib}{\ar@{->>}}
\newcommand{\xymapsto}{\ar@{|->}}

%>>>>>>>>>>>>>>>>>>>>>>>>>>>>>>
%<<<     Greek Letters      <<<
%>>>>>>>>>>>>>>>>>>>>>>>>>>>>>>
%\newcommand{\oldphi}{\phi}
%\renewcommand{\phi}{\varphi}
\let\oldphi\phi
\let\phi\varphi
\renewcommand{\to}{\longrightarrow}
\newcommand{\from}{\longleftarrow}
\newcommand{\eps}{\varepsilon}

%>>>>>>>>>>>>>>>>>>>>>>>>>>>>>>
%<<<  1st-4th & parentheses <<<
%>>>>>>>>>>>>>>>>>>>>>>>>>>>>>>
\newcommand{\first}{^\text{st}}
\newcommand{\second}{^\text{nd}}
\newcommand{\third}{^\text{rd}}
\newcommand{\fourth}{^\text{th}}
\newcommand{\ZEROTH}{$0^\text{th}$ }
\newcommand{\FIRST}{$1^\text{st}$ }
\newcommand{\SECOND}{$2^\text{nd}$ }
\newcommand{\THIRD}{$3^\text{rd}$ }
\newcommand{\FOURTH}{$4^\text{th}$ }
\newcommand{\iTH}{$i^\text{th}$ }
\newcommand{\jTH}{$j^\text{th}$ }
\newcommand{\nTH}{$n^\text{th}$ }

%>>>>>>>>>>>>>>>>>>>>>>>>>>>>>>
%<<<    upright commands    <<<
%>>>>>>>>>>>>>>>>>>>>>>>>>>>>>>
\newcommand{\upcol}{\textup{:}}
\newcommand{\upsemi}{\textup{;}}
\providecommand{\lparen}{\textup{(}}
\providecommand{\rparen}{\textup{)}}
\renewcommand{\lparen}{\textup{(}}
\renewcommand{\rparen}{\textup{)}}
\newcommand{\Iff}{\emph{iff} }

%>>>>>>>>>>>>>>>>>>>>>>>>>>>>>>
%<<<     Environments       <<<
%>>>>>>>>>>>>>>>>>>>>>>>>>>>>>>
\newcommand{\squishlist}
{ %\setlength{\topsep}{100pt} doesn't seem to do anything.
  \setlength{\itemsep}{.5pt}
  \setlength{\parskip}{0pt}
  \setlength{\parsep}{0pt}}
\newenvironment{itemise}{
\begin{list}{\textup{$\rightsquigarrow$}}
   {  \setlength{\topsep}{1mm}
      \setlength{\itemsep}{1pt}
      \setlength{\parskip}{0pt}
      \setlength{\parsep}{0pt}
   }
}{\end{list}\vspace{-.1cm}}
\newcommand{\INDENT}{\textbf{}\phantom{space}}
\renewcommand{\INDENT}{\rule{.7cm}{0cm}}

\newcommand{\itm}[1][$\rightsquigarrow$]{\item[{\makebox[.5cm][c]{\textup{#1}}}]}


%\newcommand{\rednote}[1]{{\color{red}#1}\makebox[0cm][l]{\scalebox{.1}{rednote}}}
%\newcommand{\bluenote}[1]{{\color{blue}#1}\makebox[0cm][l]{\scalebox{.1}{rednote}}}

\newcommand{\rednote}[1]
{{\color{red}#1}\makebox[0cm][l]{\scalebox{.1}{\rotatebox{90}{?????}}}}
\newcommand{\bluenote}[1]
{{\color{blue}#1}\makebox[0cm][l]{\scalebox{.1}{\rotatebox{90}{?????}}}}


\newcommand{\funcdef}[4]{\begin{align*}
#1&\to #2\\
#3&\mapsto#4
\end{align*}}

%\newcommand{\comment}[1]{}

%>>>>>>>>>>>>>>>>>>>>>>>>>>>>>>
%<<<       Categories       <<<
%>>>>>>>>>>>>>>>>>>>>>>>>>>>>>>
\newcommand{\Ens}{{\mathscr{E}ns}}
\DeclareMathOperator{\Sheaves}{{\mathsf{Shf}}}
\DeclareMathOperator{\Presheaves}{{\mathsf{PreShf}}}
\DeclareMathOperator{\Psh}{{\mathsf{Psh}}}
\DeclareMathOperator{\Shf}{{\mathsf{Shf}}}
\DeclareMathOperator{\Varieties}{{\mathsf{Var}}}
\DeclareMathOperator{\Schemes}{{\mathsf{Sch}}}
\DeclareMathOperator{\Rings}{{\mathsf{Rings}}}
\DeclareMathOperator{\AbGp}{{\mathsf{AbGp}}}
\DeclareMathOperator{\Modules}{{\mathsf{\!-Mod}}}
\DeclareMathOperator{\fgModules}{{\mathsf{\!-Mod}^{\textup{fg}}}}
\DeclareMathOperator{\QuasiCoherent}{{\mathsf{QCoh}}}
\DeclareMathOperator{\Coherent}{{\mathsf{Coh}}}
\DeclareMathOperator{\GSW}{{\mathcal{SW}^G}}
\DeclareMathOperator{\Burnside}{{\mathsf{Burn}}}
\DeclareMathOperator{\GSet}{{G\mathsf{Set}}}
\DeclareMathOperator{\FinGSet}{{G\mathsf{Set}^\textup{fin}}}
\DeclareMathOperator{\HSet}{{H\mathsf{Set}}}
\DeclareMathOperator{\Cat}{{\mathsf{Cat}}}
\DeclareMathOperator{\Fun}{{\mathsf{Fun}}}
\DeclareMathOperator{\Orb}{{\mathsf{Orb}}}
\DeclareMathOperator{\Set}{{\mathsf{Set}}}
\DeclareMathOperator{\sSet}{{\mathsf{sSet}}}
\DeclareMathOperator{\Top}{{\mathsf{Top}}}
\DeclareMathOperator{\GSpectra}{{G-\mathsf{Spectra}}}
\DeclareMathOperator{\Lan}{Lan}
\DeclareMathOperator{\Ran}{Ran}

%>>>>>>>>>>>>>>>>>>>>>>>>>>>>>>
%<<<     Script Letters     <<<
%>>>>>>>>>>>>>>>>>>>>>>>>>>>>>>
\newcommand{\scrQ}{\mathscr{Q}}
\newcommand{\scrW}{\mathscr{W}}
\newcommand{\scrE}{\mathscr{E}}
\newcommand{\scrR}{\mathscr{R}}
\newcommand{\scrT}{\mathscr{T}}
\newcommand{\scrY}{\mathscr{Y}}
\newcommand{\scrU}{\mathscr{U}}
\newcommand{\scrI}{\mathscr{I}}
\newcommand{\scrO}{\mathscr{O}}
\newcommand{\scrP}{\mathscr{P}}
\newcommand{\scrA}{\mathscr{A}}
\newcommand{\scrS}{\mathscr{S}}
\newcommand{\scrD}{\mathscr{D}}
\newcommand{\scrF}{\mathscr{F}}
\newcommand{\scrG}{\mathscr{G}}
\newcommand{\scrH}{\mathscr{H}}
\newcommand{\scrJ}{\mathscr{J}}
\newcommand{\scrK}{\mathscr{K}}
\newcommand{\scrL}{\mathscr{L}}
\newcommand{\scrZ}{\mathscr{Z}}
\newcommand{\scrX}{\mathscr{X}}
\newcommand{\scrC}{\mathscr{C}}
\newcommand{\scrV}{\mathscr{V}}
\newcommand{\scrB}{\mathscr{B}}
\newcommand{\scrN}{\mathscr{N}}
\newcommand{\scrM}{\mathscr{M}}

%>>>>>>>>>>>>>>>>>>>>>>>>>>>>>>
%<<<     Fractur Letters    <<<
%>>>>>>>>>>>>>>>>>>>>>>>>>>>>>>
\newcommand{\frakQ}{\mathfrak{Q}}
\newcommand{\frakW}{\mathfrak{W}}
\newcommand{\frakE}{\mathfrak{E}}
\newcommand{\frakR}{\mathfrak{R}}
\newcommand{\frakT}{\mathfrak{T}}
\newcommand{\frakY}{\mathfrak{Y}}
\newcommand{\frakU}{\mathfrak{U}}
\newcommand{\frakI}{\mathfrak{I}}
\newcommand{\frakO}{\mathfrak{O}}
\newcommand{\frakP}{\mathfrak{P}}
\newcommand{\frakA}{\mathfrak{A}}
\newcommand{\frakS}{\mathfrak{S}}
\newcommand{\frakD}{\mathfrak{D}}
\newcommand{\frakF}{\mathfrak{F}}
\newcommand{\frakG}{\mathfrak{G}}
\newcommand{\frakH}{\mathfrak{H}}
\newcommand{\frakJ}{\mathfrak{J}}
\newcommand{\frakK}{\mathfrak{K}}
\newcommand{\frakL}{\mathfrak{L}}
\newcommand{\frakZ}{\mathfrak{Z}}
\newcommand{\frakX}{\mathfrak{X}}
\newcommand{\frakC}{\mathfrak{C}}
\newcommand{\frakV}{\mathfrak{V}}
\newcommand{\frakB}{\mathfrak{B}}
\newcommand{\frakN}{\mathfrak{N}}
\newcommand{\frakM}{\mathfrak{M}}

\newcommand{\frakq}{\mathfrak{q}}
\newcommand{\frakw}{\mathfrak{w}}
\newcommand{\frake}{\mathfrak{e}}
\newcommand{\frakr}{\mathfrak{r}}
\newcommand{\frakt}{\mathfrak{t}}
\newcommand{\fraky}{\mathfrak{y}}
\newcommand{\fraku}{\mathfrak{u}}
\newcommand{\fraki}{\mathfrak{i}}
\newcommand{\frako}{\mathfrak{o}}
\newcommand{\frakp}{\mathfrak{p}}
\newcommand{\fraka}{\mathfrak{a}}
\newcommand{\fraks}{\mathfrak{s}}
\newcommand{\frakd}{\mathfrak{d}}
\newcommand{\frakf}{\mathfrak{f}}
\newcommand{\frakg}{\mathfrak{g}}
\newcommand{\frakh}{\mathfrak{h}}
\newcommand{\frakj}{\mathfrak{j}}
\newcommand{\frakk}{\mathfrak{k}}
\newcommand{\frakl}{\mathfrak{l}}
\newcommand{\frakz}{\mathfrak{z}}
\newcommand{\frakx}{\mathfrak{x}}
\newcommand{\frakc}{\mathfrak{c}}
\newcommand{\frakv}{\mathfrak{v}}
\newcommand{\frakb}{\mathfrak{b}}
\newcommand{\frakn}{\mathfrak{n}}
\newcommand{\frakm}{\mathfrak{m}}

%>>>>>>>>>>>>>>>>>>>>>>>>>>>>>>
%<<<  Caligraphic Letters   <<<
%>>>>>>>>>>>>>>>>>>>>>>>>>>>>>>
\newcommand{\calQ}{\mathcal{Q}}
\newcommand{\calW}{\mathcal{W}}
\newcommand{\calE}{\mathcal{E}}
\newcommand{\calR}{\mathcal{R}}
\newcommand{\calT}{\mathcal{T}}
\newcommand{\calY}{\mathcal{Y}}
\newcommand{\calU}{\mathcal{U}}
\newcommand{\calI}{\mathcal{I}}
\newcommand{\calO}{\mathcal{O}}
\newcommand{\calP}{\mathcal{P}}
\newcommand{\calA}{\mathcal{A}}
\newcommand{\calS}{\mathcal{S}}
\newcommand{\calD}{\mathcal{D}}
\newcommand{\calF}{\mathcal{F}}
\newcommand{\calG}{\mathcal{G}}
\newcommand{\calH}{\mathcal{H}}
\newcommand{\calJ}{\mathcal{J}}
\newcommand{\calK}{\mathcal{K}}
\newcommand{\calL}{\mathcal{L}}
\newcommand{\calZ}{\mathcal{Z}}
\newcommand{\calX}{\mathcal{X}}
\newcommand{\calC}{\mathcal{C}}
\newcommand{\calV}{\mathcal{V}}
\newcommand{\calB}{\mathcal{B}}
\newcommand{\calN}{\mathcal{N}}
\newcommand{\calM}{\mathcal{M}}

%>>>>>>>>>>>>>>>>>>>>>>>>>>>>>>
%<<<<<<<<<DEPRECIATED<<<<<<<<<<
%>>>>>>>>>>>>>>>>>>>>>>>>>>>>>>

%%% From Kac's template
% 1-inch margins, from fullpage.sty by H.Partl, Version 2, Dec. 15, 1988.
%\topmargin 0pt
%\advance \topmargin by -\headheight
%\advance \topmargin by -\headsep
%\textheight 9.1in
%\oddsidemargin 0pt
%\evensidemargin \oddsidemargin
%\marginparwidth 0.5in
%\textwidth 6.5in
%
%\parindent 0in
%\parskip 1.5ex
%%\renewcommand{\baselinestretch}{1.25}

%%% From the net
%\newcommand{\pullbackcorner}[1][dr]{\save*!/#1+1.2pc/#1:(1,-1)@^{|-}\restore}
%\newcommand{\pushoutcorner}[1][dr]{\save*!/#1-1.2pc/#1:(-1,1)@^{|-}\restore}










\def\excludeversion#1{\includeversion{#1}}

\excludeversion{Oct18}
\excludeversion{Oct20}
\excludeversion{Oct25}
\excludeversion{Oct27}
\excludeversion{Nov1}
\excludeversion{Nov3}
\includeversion{Nov8}
\includeversion{Nov10}


\begin{document}
\begin{Oct18}
\section*{18 October}
\subsubsection*{Valuative criterion for Separatedness}
$K[x]_{(x)}$ is a DVR --- the local ring of a point on a curve. Thinking of it this way, a DVR should always have two points, as you take out all the points but the generic and the point zero. The local ring of a point on a curve should always be a DVR, as you can measure the vanishing at zero.

To obtain a criterion for the separatedness of $X$, we'd like to map $C-\{x\}$, for $C$ a curve, into $X$, and see if it extends to $C$ in more than one way. The problem is that $C$ might be too big to fit inside $X$, say if $X$ is missing a bunch of points. Taking the limit of curves with less and less points, you get DVRs. It'll suffice to use DVRs if it's of finite type, but in general we need all valuation rings.

\begin{thm*}
Let $f:X\to Y$ be
any morphism where $X$ is noetherian. The $f$ is separated \Iff
\begin{enumerate}\squishlist
\item $X\to X\times_Y X$ is quasicompact
\item whenever $R$ is a valuation ring over $S$, if $K=\text{frac}(R)$, there is at most one lifting (dotted) in any commuting
diagram of the following form:
\[\xymatrix@C=0cm{
(0)\,\makebox[-.05cm][l]{$\in$}\ar@{|->}[d]&&\Spec K\ar[rr]\ar[d]&
\qquad \qquad &X\ar[d]^f\\
\xi\,\makebox[-.05cm][l]{\,\,$\in$}&&\Spec R\ar[rr]\ar@{-->}[rru]&&Y
}\]
i.e.\ the map $X(R)\to X(K)$ is injective.
\end{enumerate}
\end{thm*}
\begin{defn*}
A $k$-variety is a separated scheme of finite type over a field $k$.
\end{defn*}
\noindent Hartshorne wants $k=\overline k$ and integrality. ``If you want a variety to act like the set of its points, you really want it to be reduced and $k=\overline k$.'' ``In going from a perfect field to its algebraic closure, why not look at the variety over the closure taken with the action of the absolute Galois group?''
\subsubsection*{Rational Functions}
\begin{defn*} If $X$ is an integral scheme, a rational function on $X$ is an element of the function field $\calO_{X,\eta}$. Alternatively, it's an equivalence class of functions on open subsets on $X$.
\end{defn*}
\noindent Note that in this situation, as $X$ is integral, these are rational maps to something separated, and these can be sewed...
\begin{defn*} Let $X$, $Y$ be $S$-schemes. A rational map $X\to Y$ is an equivalence class of $S$-maps from a dense open subscheme of $X$ to $Y$.
\end{defn*}
Thus usual situation is where $X$ is integral and $Y$ is separated over $S$. In this case, if $(U,\phi)\sim(V,\psi)$, then $\phi$ and $\psi$ agree on $U\cup V$, so we can sew representatives together --- the the pairs in an equivalence class glue. [This fails if $Y$ is not separated --- try mapping $\A^1$ into the bug-eyed line.]
\begin{exmp*}
$\A^2\to\PP^1$ given by $(x,y)\mapsto [x:y]$. [Doesn't extend to a morphism, as it takes two different constant values on the two axes.]
\end{exmp*}
\begin{prop*}
Let $k$ be any field, $X$ be an integral $k$-variety and $Y$ a $k$-variety. Let $K=\kappa(X)=\calO_{X,\eta}$ be the function field. Then there is a bijection
\[\{\text{rational maps $X\to Y$}\}\longleftrightarrow\{\text{$k$-morphisms $\Spec K\to Y$}\}=Y(K)\]
given by $\phi\mapsto (\spec K\to U\to Y)$.
\end{prop*}
\begin{proof}[Sketch]
Let $(\spec A_i)$ be the inverse system of all nonempty affine open subschemes of $X$, so that $A_i$ is a directed system of rings, with $\varinjlim A_i=K$. Then if $U$ runs over all dense opens in $X$
\[\{\text{rational maps $X\to Y$}\}:=\varinjlim_U Y(U)=\varinjlim Y(A_i)= Y(\varinjlim A_i)=Y(K)\]
The second last thing holds in general when $Y$ is locally of finite presentation.

In fact, suppose that $Z$ is an $S$-scheme locally of finite presentation, and suppose for simplicity that both $Z$ and $S$ are affine. Then we want to show that $\varinjlim [\spec A_i,Z]=[\spec K,Z]$. The idea is that given $\phi:\Spec A_n\to Z$, we obtain a map $\Spec K\to Z$ via the composite
\[\xymatrix{
\spec A_n\ar[r]^\phi&Z\\
\spec K\ar[u]
}\]
Now $\phi$ as an element of $\varinjlim [\spec A_i,Z]$ determines and is determined by this composite, as to specify $\phi$ is to specify a finite collection of elements of $A_n$ satisfying some relations, allowing ourselves to stabilise by passing deeper into the direct limit, and this is the same as to give the same type of collection of elements of $K$.
\end{proof}
\begin{defn*}
Locally of finite presentation --- you get finitely presented things, as in definition of locally of finite type.
\end{defn*}
\begin{defn*}
A rational map $X\to Y$ is \emph{dominant} if the image of some (hence all) representative is closed in $Y$.
\end{defn*}
[There's no way to compose rational maps, so we use dominant ones in the following.]
\begin{thm*}
Fix a field $k$. There is a contravariant equivalence of categories, between the category of integral $k$-varieties with DRMs, and the category of finitely generated field extensions of $k$, with $k$-algebra homomorphisms.
\end{thm*}
\begin{proof}
In a dominant rational map, the generic point maps to the generic point! (Otherwise, it couldn't be dominant, as the image would like in the closure of the image of the generic point, which would be closed and proper.) Thus we get a map on local rings at the generic points, i.e.\ a map on function fields.

We should now check that the functor is full, faithful and essentially surjective (everything in the target is isomorphic to something in the image). For full and faithfulness, recall the bijection:
\[\{\text{rational maps $X\to Y$}\}\longleftrightarrow\{\text{$k$-morphisms $\Spec K\to Y$}\}=Y(K)\]
If we restrict to \emph{dominant} rational maps, get a bijection
\begin{alignat*}{2}
\{\text{dominant rational maps $X\to Y$}\}&\longleftrightarrow\{\text{$k$-morphisms $\Spec K\to Y$ mapping $\eta_K\to \eta_Y$}\}\\
&\ \qquad=\Hom_k(\kappa(Y),\kappa(X))
\end{alignat*}
For essential surjectivity, given $K$ a finitely generated field over $k$, let $t_1,\ldots,t_n$ be the generators, and let $A$ be the $k$-subalgebra generated by $t_1,\ldots,t_n$. Use $\Spec A$.
\end{proof}
\begin{defn*}
Integral $k$-varieties are birational \Iff they are isomorphic in the category of dominant rational maps.
\end{defn*}
[The image of a constructible subset is constructible (Chevalley), and a dense constructible subset contains a dense open subset.]
\begin{prop*} TFAE:
\begin{enumerate}\squishlist
\item $X$ and $Y$ are birational.
\item $X$ and $Y$ contain dense open subschemes which admit a $k$-isomorphism between them.
\item $\kappa(X)$ and $\kappa(Y)$ are isomorphic as $k$-algebras.
\end{enumerate}
\end{prop*}
\begin{proof}
That $2\implies 3$ is easy. $3\iff 1$ is done. Need $1\implies 2$. By definition, we have $X_1\subset X$ and $Y_1\subset Y$ and morphisms $f:X_1\to Y$ and $g:Y_1\to X$ such that $fg$ and $gf$ are the identity where defined. How does one find open subsets that match up? It's sneaky.

Let $X_n=\{x\in X\ :\ \text{$x$ can be mapped $n$ times by $f$ and $g$ alternately without getting stuck}\}$ --- this is an $n$-fold inverse image, and is thus open in $X$. Similarly define $Y_n$. Define $X_\infty$ ($Y_\infty$) to be the intersection of the $X_n$ ($Y_n)$. Then $X_\infty=X_2$! So $f:X_\infty\longleftrightarrow Y_\infty:g$ (and can be viewed as maps to $Y$ and $X$ respectively).
\end{proof}
\begin{exmp*}
$xz(x+z)=yw(y+w)$ is rational --- birational to $\PP^2$.
\end{exmp*}
\end{Oct18}
\begin{Oct20}
\section*{20 October}
\begin{exmp*}
$xz(x+z)=yw(y+w)$ is rational --- birational to $\PP^2$.
\end{exmp*}
\begin{proof}
Let $L$ be the line $x=y=0$ in $\PP^3$, and let $L'$ be $w=z=0$. $L,L'\subset X$, and $L\cap L'=\emptyset$. We'll define a rational map $L\times L'\to X$, taking $(p,p')$ to the third intersection of the line through them with $X$. This works a charm, except when $pp'$ lies entirely in $X$ (problems with double points can be resolved).

For the inverse, any $(a:b:c:d)\in\PP^3$ not on $L\cup L'$ lies on a unique line $pp'$ with $p\in L$, $p'\in L'$, defining the rational map inverse to that above. So is is birational to $P^1\times P^1$, which is birational to $\PP^2$.
\end{proof}
\subsubsection*{Projective morphisms}
\begin{defn*}
$\PP^n_A:=\proj A[x_0,\ldots,x_n]\simeq \PP^n_\Z\times A$. For $S$ a scheme, $\PP^n_S:= \PP^n_\Z\times S$.
\end{defn*}
\begin{lem*}
If $(X_i)$ is an open cover of an $S$-scheme $X$, and that $R$ is a local ring over $S$ (i.e.\ it's an $S$-algebra). Then $X(R)=\bigcup X_i(R)$.
\end{lem*}
[This is obviously not true when $R$ is not local.]
\begin{proof}
Let $t\in\spec R$ be the closed point, corresponding to the maximal ideal. Any morphism $f:\spec R\to X$ sends $t$ into some $X_i$. We need to see that all of $\spec R$ maps into $X_i$. Well $f^{-1}X_i$ is open, containing the point $t$, but the only open subset containing the unique maximal ideal is $\spec R$.
\end{proof}
\begin{cor*}
If $R$ is a local ring, then we can describe the $R$-points of projective space:
\[\PP^n_\Z(R)\simeq\{(a_0,\ldots,a_n)\in R^{n+1}\,|\,\text{some $a_i\in R^\times$}\}/R^\times\]
\end{cor*}
[You can make this work for a PID if you demand that the $a_i$ generate the unit ideal, but even for other Dedekind rings, this doesn't work.]
\begin{proof}
$\PP^n(R)=\bigcup D_+(x_i)$, and $D_+(x_i)=\A^n_\Z$, and points of $\A^n_\Z$ are $n$-tuples of elements of $R$. Thus 
\[\PP^n(R)=\bigcup \{a_0,\ldots,a_{i-1},1,a_{i+1},\ldots,a_n\}\qedhere\]
\end{proof}
\begin{cor*}
If $R$ is a valuation ring, and $K$ is its fraction field, then $\PP^n(R)\weakequiv \PP^n(K)$.
\end{cor*}
\begin{proof}
Given $(a_0,\ldots,a_n)\in K^{n+1}\setminus\{0\}$, we can scale up just enough to get all in the valuation ring, but not so far that they all fail to be units. This expression is uniquely determined up to a unit of $R^\times$.
\end{proof}
\begin{defn*}
$X\to S$ is projective \Iff it factors as a composite $X\cofibration \PP^n_S\to S$. That is, the composite of a closed immersion into projective space with the structure map.
\end{defn*}
[The definition in EGA is different. Here, $\PP V=\proj\Sym V^*$. To do this relatively, you use a vector bundle on $S$. These are equivalent under mild hypothesis, for example if $S$ is of finite type over a field.]
\begin{defn*}
$X\to S$ is quasi-projective if it factors as the composite of an open immersion $X\cofibration Y$ and a projective morphism $Y\to S$.
\end{defn*}
\subsubsection*{Proper morphisms}
\begin{defn*}
A morphism $f:X\to Y$ is closed if the image of an closed subset of $X$ is closed in $Y$. This property is not preserved under base extension. A morphism $f:X\to Y$ is universally closed if any base extension thereof is closed.
\end{defn*}
\begin{nonexmp*} The map $\A^1_k\to \spec k$ is not universally closed. Consider:
\[\xymatrix{
\ar[d]\A^2_k\ar[r]&\A^1_k\ar[d]\\
\A^1_k\ar[r]&\spec k
}\]
The hyperbola $xy=1$ maps to a non-closed set.
\end{nonexmp*}
\begin{defn*}
$f:X\to Y$ is proper \Iff it is separated, of finite type, and universally closed.
\end{defn*}
\begin{exmp*}
Closed immersions, compositions of proper maps and base changes of proper maps are all proper. Properness is local on the base.
\end{exmp*}
\begin{thm*}[Valuative criterion for properness]
Suppose $f:X\to S$ is of finite type, where $S$ is noetherian. Then $f$ is proper \Iff for every (discrete)\footnote{It's a pain to go from the VR version to the DVR version.} valuation ring $R$ over $S$, if $K:=\text{frac}(R)$, then the natural restriction map $X(R)\to X(K)$ is bijective.
\end{thm*}
\noindent [This is good both for checking properness, and for extending morphisms!]
\begin{thm*} For a morphism of noetherian schemes, projective $\implies$ proper.
\end{thm*}
\begin{proof}
First, we'll prove that $\PP^n_Z\to\spec\Z$ is proper. By the valuative criterion for properness, this comes from a recent corollary. In general, we have to check this for composites of a closed immersion into $\PP^n_S$ and the structure map $\PP^n_S\to S$. The structure map, being a base extension of $\PP^n_Z\to\spec\Z$, is proper. But closed immersions are proper --- they're locally affine, thus separated, clearly finite type, and universally closed, as closed immersions are closed under base change. As compositions of proper morphisms are proper, we're done.
\end{proof}
\begin{exmp*}
Let $A=k[a,b,c,d]$, and $X=\proj \frac{A[x,y]}{(ax+by,cx+dy)}\subseteq\PP^1_A$. There's a map $f:X\to Y=\spec A=\A^4$. Now $f$ is projective, hence proper, hence universally closed, hence closed. Then $f(X)$ is the closed subset defined by $ad-bc=0$ in $\A^4$. 

In general, you get the closed subset defined by the resultant. \emph{This is really a proof of the existence of the resultant, a polynomial characterising when two homogeneous forms have a common zero --- to see that there's only one polynomial, one counts dimensions, and sees that you get a hypersurface.}\footnote{It's Krull's Hauptidealsatz which shows that a closed codimension one subset of affine space is a hypersurface.}
\end{exmp*}
\begin{thm*}
Suppose that $f:X\to Y$ is a morphism of $k$-varieties, and $X$ is projective (over $k$). Then the image $f(X)$ is closed in $Y$.
\end{thm*}
\begin{proof}
\[\xymatrix{
X\ar[d]_f\ar@/^1cm/[dd]^{\text{proper}}\\%bend me.
Y\ar[d]_{\text{separated}}\\
\spec k
}\]
Thus $f$ is proper. In particular, $f$ is universally closed, and $f(X)$ is closed in $Y$.
\end{proof}
[This is part of elimination thy --- if $f$ is the elimination of a variable, what you get is still defined by equations.....]
\begin{defn*}
A constructible subset is a Boolean combination of closed sets.
\end{defn*}
\begin{thm*}[Chevalley]
Any morphism of varieties $X\to Y$ sends constructible sets to constructible sets. More generally, this holds for any morphism of finite type of noetherian schemes.
\end{thm*}
There's a reinterpretation of this in terms of logic, called elimination of quantifiers, when $k$ is algebraically closed.
\begin{thm*}[elimination of quantifiers]
Any first order formula in the language $(+,\cdot,\text{a symbol for each element of $k$})$ is equivalent to a quantifier-free formula.
\end{thm*}
\begin{exmp*}
$(\exists y)\ xy=1\iff \text{not}(x=0)$.
\end{exmp*}
\noindent Over the reals, you also need to throw in the symbol $\leq$, and call things ``semi-algebraic''.
\end{Oct20}
\begin{Oct27}
\section*{27 October}
Smooth was defined by local obvious smoothness.
\begin{fact*}
If $(A,m)$ is a regular local ring of dimension $n$, and $x_1,\ldots, x_r\in m$, then $A/(x_1,\ldots,x_r)$ is a regular local ring of dimension $n-r$ \Iff the images of $x_1,\ldots,x_r$ in $m/m^2$ are linearly independent.
\end{fact*}
\begin{defn*}
Given $x\in\Spec\frac{ A[t_1,\ldots,t_n]}{(g_{r+1},\ldots,g_n)}\overset{f}{\to} \Spec A$, call $f$ obviously smooth of relative dimension $r$ at $x$ \Iff the matrix $(\partial g_i/\partial t_j(x))\in M_{n-r,n}(k(x))$ has rank $n-r$.
\end{defn*}
\begin{defn*}
Given any morphism of schemes $X\to Y$ and $x\in X$, call $f$ smooth of relative dimension $r$ at $x$ \Iff there are open neighbourhoods mapping into  each other in an obviously smooth way.
\end{defn*}
\begin{defn*}
A morphism is \'etale if it is smooth of relative dimension zero.
\end{defn*}
\noindent For $\C$-varieties, $X\to Y$ is \'etale \Iff the corresponding map of analytic spaces is locally biholomorphic. Finite \'etale maps correspond to finite-sheeted covering spaces in algebraic topology.

\begin{prop*}
For $k$ any field, suppose that $X=\Spec \frac{k[t_1,\ldots,t_n]}{(g_{r+1},\ldots,g_n)}$, and $x\in X(k)$. Then $X$ is obviously smooth (over $\spec k$) of relative dimension $r$ at $x$ \Iff $\calO_{X,x}$ is a regular local ring.
\end{prop*}
\begin{proof}
We may assume that $x=(0,0,\ldots,0)$. $X$ is obviously smooth of relative dimension $r$ at $x$ \Iff $(\partial g_i/\partial t_j(x))\in M_{n-r,n}(k(x))$ has rank $n-r$, \Iff $\overline g_{r+1},\ldots,\overline g_n\in m_{\A^n,x}/m_{\A^n,x}^2=\langle \overline t_1,\ldots,\overline t_n\rangle $ are linearly independent (as these are the rows of the matrix).

Using the above fact, this happens \Iff $\calO_{\A^n,x}/(g_{r+1},\ldots,g_n)$ is a regular local ring of dimension $r$. But this ring is exactly $\calO_{X,x}$.
\end{proof}
Fortunately, anything which is smooth is in fact obviously smooth:
\begin{prop*}
Suppose $x\in U:=\spec\frac{A[t_1,\ldots,t_n]}{(g_{r+1},\ldots,g_n)}\subset X$, and $X\downarrow\spec A$.
Then $X\downarrow \spec A$ is smooth of relative dimension $r$ at $x$ \Iff $U\downarrow\spec A$ is obviously smooth at $x$.
\end{prop*}
\begin{proof}
If it's obviously smooth, then obviously, it's smooth, by definition. For the other direction, let $k=k(x)$, giving a map $\spec k\cofibration X$. If we're given that $X\downarrow \spec A$ is smooth of relative dimension $r$ at $x$, then $U\downarrow\spec A$ is smooth of relative dimension $r$ at $x$. Now as smoothness is preserved by base extension, we have that $U_k\downarrow \spec k$ is smooth at $x$. This implies that $\calO_{U_x,x}$ is a regular local ring of dimension $r$. But now, $U_k\downarrow \spec k$ is obviously smooth of relative dimension $r$, by the above proposition. This implies that $U\downarrow\spec A$ is obviously smooth, as it doesn't make much difference whether we're evaluating at $k$ or what, the matrices are the same.
\end{proof}
\begin{fact*}
Suppose that $X$ is of finite type over any field, of pure dimension $r$. The $X$ is smooth over $\spec k$ \Iff $X$ is geometrically regular. Either of these conditions imply that $X$ is regular. [By regular, we mean that all the local rings are regular local rings. By geometrically regular, we mean regular after extending to an algebraic closure.]

There's a converse if $k$ is perfect --- then $X$ is smooth \Iff it is regular.
\end{fact*}
\begin{fact*}
If you localise a regular local ring at a prime, you get another regular local ring.
\end{fact*}

\begin{defn*}
The word `nonsingular' means either smooth or regular, depending on context.
\end{defn*}
\begin{defn*}
The \emph{singular locus} of a scheme $X$ of pure dimension $r$ over $k$ is
\[X_{\text{sing}}:=\left\{x\in X\,\middle|\,X\text{ is not smooth of relative dimension $r$ at $x$}\right\}\]
\end{defn*}
\begin{prop*}
$X_{\text{sing}}$ is a closed subset of $X$.
\end{prop*}
\begin{proof}
Obvious smoothness is an open condition --- locally it's the common zero locus of the minors of the matrix in the local smoothness condition.
\end{proof}
Note that smooth is a relative notion, while regular is absolute.
\begin{exmp*}
Consider $X\to Y$ given by $\spec k[x,t,y]/(xy-t)\to k[t]$. Then $X$ is regular, as it's isomorphic to $\A^2$ as a $k$-scheme. However, the map is not smooth at $(0,0,0)$. [To check smoothness, we can check that it's obviously smooth. The Jacobian matrix is $[y\ \ x]$. This has maximal rank when $(x,y)\neq(0,0)$ (recall that one evaluates into the residue field in order to discuss rank). Thus this morphism is smooth away from $(0,0,0)$.]
\end{exmp*}
\subsection*{Sheaves of modules}
\begin{defn*} A sheaf of $\calO_X$-modules is a sheaf of abelian groups $\calF$ on $X$ such that for any open subset $U$ of $X$, $\calF(U)$ is an $\calO(U)$-module, compatibly with restriction. A morphism thereof is the obvious.
\end{defn*}
The category $\calO_X\Modules$ of these is an abelian category, which is monoidal and self-enriched. For the tensor product, one needs to sheafify, and a section of the tensor is locally a tensor of sections.

There are the obvious notions of freeness and local freeness. A vector bundle is a locally free sheaf of finite rank. A line bundle (or invertible sheaf) is a locally free sheaf of dimension one.

Recall that given a ring homomorphism $B\from A$, there's an adjunction
\[B\Modules\longleftrightarrow A\Modules.\]
Similarly, given a morphism of schemes $f:X\to Y$, we have
\[f_*:\calO_X\Modules\longleftrightarrow\calO_Y\Modules:f^*.\]
Given $\calF$ an $\calO_X$-module, take $f_*\calF$, with structure given by the map $\calO_Y\to f_*\calO_X$. Given $\calG$ an $\calO_Y$-module, $f^{-1}\scrG$ is an $f^{-1}\calO_Y$-module, so we form
\[f^*\calG:=f^{-1}\calG\otimes_{f^{-1}{\calO_Y}}\calO_X.\]
\end{Oct27}
\begin{Nov1}
\section*{1 November}
Given an affine scheme $\spec A$, there's a functor
\funcdef{A\Modules}{\calO_X\Modules}{M}{\text{$\widetilde M$, where $\widetilde M(D(f))=M_f$}}
This functor is fully faithful, exact, and respects tensor product and arbitrary direct sums. Moreover, given a ring hom $B\from A$ and $M\in A\Modules$ and $N\in B\Modules$. We have a map $f:\spec B\to\spec A$. Then $f_*\widetilde N=\widetilde{_AN}$, and $f^*\widetilde M=\widetilde{M\otimes_AB}$.
\begin{exmp*}
Suppose $A$ is a Dedekind domain (e.g.\ the integral closure of $\Z$ in a finite extension of $\Q$), and $I$ is a fractional ideal (a nonzero f.g.\ $A$-submodule of $\Frac A$). Then $\widetilde I$ is locally free of rank one on $\spec A$. [As a fractional ideal of a DVR is always free.]  Moreover, $\widetilde I$ is free \Iff $I$ is principal.
\end{exmp*}
\begin{defn*}
An $\calO_x$-module $\calF$ is \emph{quasicoherent} \Iff there's an affine open cover $(X_i)$ with $X_i=\spec A_i$ such that $\calF|_{X_i}$ is isomorphic to $\widetilde {M_i}$ for some $A_i$-module $M_i$. $\calF$ is \emph{coherent} if all of the modules involved are finitely generated (we'll only use this term when the underlying scheme is noetherian).
\end{defn*}
\begin{prop*}
The following are equivalent:
\begin{itemise}
\item $\calF$ is quasicoherent
\item this condition on every affine open in $X$.
\item For every $\Spec A_f\subset\spec A\subset X$, the natural map
\[(\calF(\spec A))_f\to\calF(\spec A_f)\]
is an isomorphism.
\end{itemise}
\end{prop*}
\begin{exmp*}
If $X=\spec A$, then there are equivalences of categories:
\[A\Modules\longleftrightarrow\QuasiCoherent_X\ \ \ A\fgModules\longleftrightarrow\Coherent_X\]
For any scheme, $\QuasiCoherent_X$ is an abelian subcategory of $\calO_X\Modules$, closed under taking extensions. So is $\Coherent_X$ if $X$ is noetherian.
\end{exmp*}
\begin{exmp*}
The sheaf of 1-forms on a variety, whose sections are locally of the form $\sum a_i df_i$, with $a_i,f_i$ in the structure sheaf, is quasi-coherent.
\end{exmp*}
\begin{prop*}
Suppose that $f:X\to Y$,  $\calF\in\calO_X\Modules$, and $\calG\in\calO_Y\Modules$.
Then
\begin{itemise}
\item if $\calG\in\QuasiCoherent_Y$ then $f^*\calG\in\QuasiCoherent_X$.
\item if $\calG\in\Coherent_Y$ then $f^*\calG\in\Coherent_X$.
\item Suppose that $X$ is quasi-compact and quasi-separated\footnote{the intersection of two affine opens in $X$ is a finite union of affine opens} (e.g.\ if $X$ is noetherian). If $\calF\in\QuasiCoherent_X$, then $f_*\calF\in\Coherent_Y$.
\item If $X$ is proper, and if $\calF\in\Coherent_X$, then $f_*\calF\in\Coherent_Y$.
\end{itemise}
\end{prop*}
\begin{fact*}
It's hard to come up with examples of schemes which aren't quasi-separated. An example is the bug-eyed infinite-dimensional affine space.
\end{fact*}
\subsection*{Closed subschemes}
Recall that closed subschemes of $\spec A$ correspond to ideals of $A$. Given an ideal, one takes $\spec A/I$, and given a closed subscheme $Z\cofibration \spec A$, take $I=\ker\{\Gamma(\calO_{\spec A})\to\Gamma(\calO_Z)\}$.

We seek to generalise this to a bijection between closed subschemes of $X$ and quasi-coherent sheaves of ideals of $\calO_X$. The bijection is just about the same:
\[Z\mapsto\calI_Z:=\ker\{\calO_X\to i_*\calO_Z\}.\]
That is, the ideal sheaf $\calI_Z$ of $Z$ in $X$ is the ideal of functions which vanish upon restriction to $\Z$. For this inverse to this map, we send a quasi-coherent sheaf of ideals $\calI$ to $\SPEC\,\calO_X/\calI$. [Here, $\calO_x/\calI$ is a quasiocoherent sheaf ($\calF$) of $\calO_X$-algebras. Over each affine open $\spec A\subset X$, $\calF(\spec A)$ is an $A$-algebra, so that $\spec(\calF(\spec A))$ is a scheme over $\spec A$, and these can be glued to form $\SPEC \calF$.]
\begin{defn*}
The \emph{scheme-theoretic image}\footnote{The scheme-theoretic image is like `the closure of the image'.} of a morphism of schemes $f:X\to Y$ is the smallest closed subscheme $Z\cofibration Y$ through which $f$ factors. This corresponds to the largest quasi-coherent sheaf $\calI$ of ideals contained in $\calK=\ker\{\calO_Y\to f_*\calO_X\}$.
\end{defn*}
\begin{fact*}
If $f$ is quasi-compact, then:
\begin{itemise}
\item $\scrK$ is quasi-coherent, so that $\calI=\calK$;
\item the formation of $\calI$ and $Z$ is local on $Y$. That is, for $U$ open in $Y$, $\im(f^{-1}U\to U)=\Z\cap U$.
\item The underlying topological space of $Z$ is the closure of $f(X)$.
\end{itemise}
\end{fact*}
\subsection*{Quasicoherent sheaves on $\proj S$}
Given a graded ring $S=\bigoplus _{n\geq 0}S_n$ we can form a scheme $X=\Proj S$. Given a graded $S$-module $M=\bigoplus _{n\in \Z}M_n$, we obtain a quasi-coherent sheaf \reflectbox{$\widetilde {\reflectbox{$M$}}$} on $\proj S$ characterised by $\widetilde M(D_+(f)):=M_{(f)}$ (the degree zero localisation). This is quasi-coherent, as $\widetilde M|_{D_+(f)}\simeq \widetilde{M_{(f)}}$ on $\spec S_{(f)}$. In particular, $\widetilde M$ is coherent if $M$ is finitely generated.

Given a graded $S$-module $M$, we can \emph{twist} it, defining $M(n)$ to be the graded $S$-module with $M(n)_i=M_{n+i}$. The key example is $\calO_X(n):=\widetilde {S(n)}$. From now on, we'll assume that:
\[\text{ $S$ is generated by $S_1$ as an $S_0$-algebra,}\]
e.g.\ $S$ is a quotient of a polynomial ring in dimension one.
\begin{prop*}
$\calO_X(n)$ is a line bundle.
\end{prop*}
\begin{proof}
$X$ is covered by the $D_+(f)$ for $f\in S_1$. Fix $f\in S_1$. Then $\calO_X(n)|_{D_+(F)}\simeq \widetilde{S(n)_{(f)}}$, and it's enough to see that this module is a free $S_{(f)}$-module of rank one. However, there's an obvious $S_{(f)}$-module isomorphism between $S(n)_{(f)}$ and $S_{(f)}$ given by multiplication by $f^{\pm n}$.
\end{proof}
\begin{prop}
$\widetilde{M\otimes_SN}=\widetilde M\otimes_X\widetilde N$.
\end{prop}
\begin{proof}
On $D_+(f)$ for $f\in S_1$, we have $(M\otimes_S N)_{(f)}\simeq M_{(f)}\otimes_{S_{(f)}}N_{(f)}$.
\end{proof}
\begin{cor*}
$\calO_X(m)\otimes\calO_X(n)=\calO_X(m+n)$.
\end{cor*}
\end{Nov1}
\begin{Nov3}
\section*{3 November}
\begin{claim}
Take $S=k[x_0,\ldots,x_n]$, so that $X=\PP^r$. Then $\Gamma(\PP^r,\calO(n))=S_n$, the module of homogeneous polynomials of degree $n$.
\begin{proof}
$\Gamma(D_+(x_i),\calO(n))=S(n)_{(x_i)}$, which is the $k$-span of the monomials $\prod x_j^{a_j}$ where $\sum a_j=n$ and $a_j\geq0$ for $j\neq i$. The result follows.
\end{proof}
\end{claim}
\begin{cor*}
$\bigoplus_{n\in\Z}\Gamma(\PP^r,\calO(n))$ recovers $S$ as a graded ring.
\end{cor*}
\begin{defn*}
For $\calF$ on $X:=\proj S$, define $\Gamma_*(\calF):=\bigoplus_{n\in\Z}\Gamma(X,\calF(n))$, a graded $S$-module.
\end{defn*}
Then there are functors
\[\widetilde{\DASH}:\text{graded S-modules}\longleftrightarrow\text{quasicoherent $\calO_X$-modules}:\Gamma_*(\DASH)\]
\begin{prop*}
$\widetilde{\Gamma_*(\calF)}\simeq\calF$, but $M\to \Gamma_*(\widetilde M)$ need not be an isomorphism. [If $M$ is a sheaf of ideals, then $\Gamma_*(\widetilde M)$ is its saturation.]
\end{prop*}
Recall that $\PP^n_Y$ is equipped with a projection to $\PP^n_Z$:
\[\PP^n_Y:=\PP^n_Z\times Y\overset{\pi}{\to}\PP^n_Z.\]
We define $\calO(1)$ on $\PP^n_Y$ to be $\pi^*(\calO(1)\text{ on $\PP^n_\Z$})$.
\begin{defn*}
Given an immersion $i$, (a composite $X\overset{open}{\cofibration} U\overset{closed}{\cofibration} \PP^n_Y$), we define $\calO_{X}(1):=i^*\calO(1)$. Any line bundle on $X$ arising in this way is called \textit{very ample relative to $Y$}.
\end{defn*}
\noindent [Somehow, understanding all of the very ample line bundles on $X$ is tantamount to understanding all of the embeddings of $X$ in projective space.]
\begin{exmp*}
Recall the 2-uple embedding of $\PP^2$, a.k.a.\ the Veronese embedding, a closed immersion:
\[\PP^2\overset{i}{\cofibration} \PP^5\text{\qquad given by\qquad}(x:y:z:\mapsto(x^2:y^2:z^2:xy:xz:yz).\]
This satisfies $i^*\calO_{\PP^5}(1)=\calO_{\PP^2}(2)$. This shows that $\calO_{\PP^2}(@)$ is very ample relative to $k$.
\end{exmp*}
\begin{defn*}
Suppose that $X$ is a scheme, and $\calF$ is an $\calO_X$-module> Suppose that we are given some global sections $s_i\in\Gamma(X,\calF)$. We say that the $s_i$ \emph{generate} $\calF$ \Iff at each point $x\in X$, the images of the $s_i$ generate the stalk $\calF_{x,X}$ as an $\calO_{x,X}$-module.

This is equivalent to the statement that the  $\calO_X$-module homomorphism $\bigoplus_i\calO_X\to\calF$ corresponding to these sections is surjective.

We say that $\calF$ is \emph{generated by global sections} if such a collection of sections exists.
\end{defn*}
\begin{exmp*}
$x_1,\ldots,x_n\in\Gamma(\PP^n,\calO(1))$ generate $\calO(1)$. $\calO(-1)$ cannot be generated by global sections.
\end{exmp*}
\noindent In the special case that $\calF$ is very ample, generating global sections correspond to the coordinates of an embedding, for this reason.
\begin{exmp*}
Suppose that $r\geq1$. Then the sheaf $\calO(d)$ on $\PP^r$ is g.b.g.s.\ \Iff $d\geq0$. It is vey ample \Iff $d>0$. Note that we have the $d$-uple embedding, and if $d\leq0$, there aren't enough global sections for it to be very ample.
\end{exmp*}
\begin{exmp*}
Let $X:=\PP^1\times\PP^1$. This has two projections to $\PP^1$. For any $m,n\in\Z$, we form $\calO(m,n):=\pi_1^*\calO(m)\otimes\pi_2^*\calO(n)$, a line bundle. Now
\[\Gamma(\PP^1\times\PP^1,\calO(m,n))=\{\text{bihomogeneous polynomials of degree $m,n$.}\}.\]
This has dimension $(m+1)(n+1)$ over $k$.  $\calO(m,n)$ is g.b.g.s.\ \Iff $m,n\geq0$. It is very ample \Iff $m,n\geq1$.
\end{exmp*}
\begin{exmp*}
Under the Segre embedding $\PP^1\times\PP^1\cofibration \PP^3$, we have $\phi^*\calO_{\PP^3}(1)=\calO(1,1)$, and we can get the other very ample ones by mixing in some $m$-uple and $n$-uple embeddings.
\end{exmp*}
\begin{thm*}[Serre]
Suppose $A$ is a noethrian ring, and we have a closed immersion $X\cofibration \PP^n_A$, so that $X$ is projective over $A$, and we have a very ample sheaf $\calO(1)$ on $X$. Suppose that $\calF$ is a coherent $\calO_X$-module. Then for $n\gg1$ (depending on $\calF$), then $\calF(n)$ is g.b.g.s.

Another way to say this is that $\calO(1)$ is \textbf{ample}.
\end{thm*}
\begin{proof}
Over the affine patches, you choose some generators. All this is finite. These have some denominators, which can be cleared by twisting!
\end{proof}
\begin{defn*}
A line bundle $\calL$ on (any scheme) $X$ is \emph{ample} \Iff for any coherent scheme $\calF$, for $n\gg0$, $\calF\otimes\calL^{\otimes n}$ is g.b.g.s.\ for $n\gg1$. 
\end{defn*}
\begin{thm*}[II.7.6]
If $X$ is of finite type over a noetherian ring $A$, $\calF$ is ample \Iff for some $n>0$, $\calL^{\otimes n}$ is very ample over $A$.
\end{thm*}
\begin{cor*}
Suppose that $A$ is a noetherian ring and $X\cofibration \PP^r_A$ is a closed embedding, and that $\calF$ is an $\calO_X$-module. Then $\calF$ is  coherent \Iff there is a surjection
\[\bigoplus_{i=1}^s\calO(n_i)\fibration\calF.\]
\end{cor*}
\begin{proof}
By the theorem we have a surjection $\bigoplus \calO_X\fibration \calF(n)$. Untwisting is exact!
\end{proof}
Recall that if $X$ is a projective variety over an algebraically closed field, then $\Gamma(X,\calO_X)=k$. This `algebraic Liouville theorem' generalises.
\begin{prop*}
Suppose that $X$ is any field, $X$ is a projective scheme over $k$, and $\calF$ is a coherent $\calO_X$-module. Then $\dim_k\Gamma(X,\calF)$ is always finite.
\end{prop*}
\begin{exmp*}
$\dim_k\Gamma(\PP^n_K,\calO(d))={n+d\choose d}$.
\end{exmp*}
\begin{exmp*}
$\dim_k(\Gamma(\A^n_k,\calO)=\infty$.
\end{exmp*}
An even wider generalisation, in which we view $f_*$ as a relative version of global sections, is:
\begin{prop*}[Grothendieck]
Suppose that $Y$ is a noetherian scheme, $f:X\to Y$ is a proper morphism, and $\calF$ is a coherent $\calO_X$-module. Then $f_*\calF$ is a coherent $\calO_Y$-module.
\end{prop*}
\end{Nov3}
\begin{Nov8}
\section*{8 November}
\subsection*{Some commutative algebra}
\begin{prop*}
For a one-dimensional noetheriana local domain, TFAE:
\begin{enumerate}\squishlist
\item[1.] $A$ is a DVR;
\item[2.] $A$ is regular;
\item[2'.] $\frakm$ is a principal ideal;
\item[3] $A$ is integrally closed (in its field of fractions).
\end{enumerate}
\end{prop*}
\noindent Note that 2 and 2' are equivalent by Nakayama's lemma.

Suppose that $X$ is an integral $k$-variety\footnote{separated and of finite type} of dimension $n$.
\begin{prop*}
For $y\in X$, TFAE
\begin{enumerate}\squishlist
\item $y$ corresponds to a prime ideal of height $r$ in some open affine subscheme;
\item $\dim\calO_{X,y}=r$;
\item $y$ is the generic point of an integral subvariety $Y$ of dimension $n-r$.
\end{enumerate}
\end{prop*}
\begin{defn*}
If any of these equivalent conditions hold, then $y$ is called a point of codimension $r$ in $X$.
\end{defn*}
\begin{cor*}
If $y$ is a point of codimension one in $X$, TFAE:
\begin{enumerate}\squishlist
\item $\calO_{X,y}$ is a DVR;
\item $X$ is regular at $y$;
\item $X$ is normal at $y$.
\end{enumerate}
\end{cor*}
\noindent In this case, one can think of the valuation $v(f)$, for $f$ a rational function on $X$, as being ``the order of vanishing of $f$ along $Y$''.
\begin{exmp*}
Let $X=\A^2=\spec k[x,y]$. Let $Y$ be the $x$-axis, and $f=\frac{x^5}{y^3(x+y)}$. Then $\nu_Y(f)=-3$.\footnote{One should really note that $y$ generates the maximal ideal of the local ring.}
\end{exmp*}
\begin{defn*} The following conditions on an itegral $k$-variety increase in strength:
\begin{itemise}
\item $X$ is regular in codimension one if $X$ is regular at every point of codimension one.\footnote{ For an example, try $x^2_1+x_2^2+x_3^3=0$. This is normal, but not factorial.}
\item $X$ is factorial \Iff for all $x\in X$, $\calO_{X,x}$ is a UFD.\footnote{For rings, being factorial is \textbf{not} a local property. Suppose you have the affine part of an elliptic curve. Then this is locally factorial, but the affine coordinate ring is \textbf{not} a UFD --- the class group is not trivial, and relates to the rational points of the curve.}
\item $X$ is normal \Iff for all $x\in X$, $\calO_{X,x}$ is integrally closed \Iff $\calO_X(U)$ is integrally closed
\end{itemise}
Smooth implies regular implies factorial implies normal implies regular in codimension one.
\end{defn*}
\begin{thm*}[Normalisation; c.f.\ Ex.\ I.3.17, II.3.17]
Let $X$ be an integral scheme. Then
\begin{enumerate}\squishlist
\item There is a `normalisation' $f:\widetilde{X}\to X$ with $\widetilde{X}$ normal, $f$ dominant, with a universal property for dominant maps to $X$ from a normal scheme.
\item $\widetilde{X}$ is unique up to isomorphism over $X$.
\item $f$ is surjective
\item If $X$ is of finite type over $k$, then $f$ is a finite birational morphism.
\item If $U\subset X$ is a nonempty affine open subscheme, then 
\[\xymatrix{
%a1& a2& a3& a4& a5& a6\\
%b1& b2& b3& b4& b5& b6\\
 \ar[r]^\simeq\ar[d]f^{-1}U& \spec\widetilde{A}\ar[d]\\
 \ar[r]^\simeq U& \spec A\\
%e1& e2& e3& e4& e5& e6\\
%f1& f2& f3& f4& f5& f6\\
}\]
In fact, this can be done with the $\SPEC$ construction.
\end{enumerate}
\end{thm*}
\subsection*{Resolution of singularities}
Given an integral $k$-variety $X$, can one find a birational proper morphism $X'\to X$ such that $X'$ is regular?
\begin{exmp*}
Let $X=\spec\frac{k[x,y]}{(y^2-x^3)}\simeq\spec k[t^2,t^3]$ be the cuspidal cubic. Using the second presentation, one sees that the ring is not integrally closed, as $t$ misses out. Thus $X$ is not normal, and so $X$ is not regular. To resolve, one takes the map from $\spec k[x]$, which is the normalisation. Note that this is proper, as it is a finite morphism!
\end{exmp*}
\subsubsection*{Status}
\newcommand{\tick}{yes}
\begin{tabular}{c||c|c|c}
$\dim X$&char 0& char $p>0$&method\\\hline\hline
0&\tick&\tick\\\hline
1&Newton&\tick&(use the normalisation)\\\hline
2&\tick&Abhyankar 1956&(just blow everything up)\\\hline
3&Hironaka 1964&Abhyankar (char$>6)$\\
&&Cossart \& Piltant 2009 (all chars)\\\hline
4&Hironaka 1964&?
\end{tabular}
\subsection*{``The amazing synthesis''}
\begin{thm*}
Fix a field $k$. Define categories
%\[\calP:=\{\textup{regular projective integral $k$-varieties of dimension one with nonconstant morphisms}\}\]
%\[\calI:=\{\textup{integral $k$-varieties of dimension one, with dominant rational maps}\}\]
%\[\calF:=\{\textup{f.g.\ field extensions $K$ of $k$ of transcendene degree one, with $k$-algebra homomorphisms}\}\]
%\[\calR:=\{\textup{compact Riemann surfaces with nonconstant homolmorphic maps}\}\]
\begin{alignat*}{2}
\calP&:=\{\textup{regular projective integral $k$-varieties of dimension one with nonconstant morphisms}\}%&\qquad&\text{()}
\\\calI&:=\{\textup{integral $k$-varieties of dimension one, with dominant rational maps}\}%&\qquad&\text{()}
\\\calF&:=\{\textup{f.g.\ field extensions $K$ of $k$ of transcendene degree one, with $k$-algebra homomorphisms}\}%&\qquad&\text{()}
\\\calR&:=\{\textup{compact Riemann surfaces with nonconstant homolmorphic maps}\}%&\qquad&\text{()}
%\\&=%&\qquad&\text{()}
\end{alignat*}
Note that $\calR$ is only defined when $k=\C$. Then there are equivalences
\[\xymatrix{
%a1& a2& a3& a4& a5& a6\\
%b1& b2& b3& b4& b5& b6\\
 \calP\ar[rd]_{\textup{analytic space}}\ar[r]^{(*)}& \calI\ar[r]& \calF\\
&\calR\ar[ur]_{\textup{field of meromorphic functions}}
%e1& e2& e3& e4& e5& e6\\
%f1& f2& f3& f4& f5& f6\\
}\]
\end{thm*}
\begin{proof}
Lets start with $(*)$. It's a functor, as every nonconstant morphism between curves must contain an open subset (by Chevalley's theorem on constructible sets, as there are not so many constructible sets on a curve: they're all finite or cofinite).

Now let's check fullness. Suppose that $X,Y$ are regular projective curves (today meaning integral $k$-varieties of dimension one). The question is: ``does every dominant rational map $X\to Y$ extend to a morphism?'' The answer is yes (\emph{using the valuative criterion for properness!!!})!

For faithfulness, note that if $f,g:X\to Y$ are two nonconstant morphisms which are equal as rational maps, then $f=g$, and $Y$ is separated and $X$ is reduced.

For essential surjectivity, given an integral curve $X$, we need a regular projective curve $X'$ birational to $X$. First, replace $X$ by an affine open subscheme to assume that $X\subset\A^n$ is affine. Replace $X$ by its projective closure to assume that $X$ is projective. Take the normalisation $\widetilde{X}\to X$, which is finite and thus proper, so that $\widetilde{X}$ is proper over $k$. Thus $\widetilde{X}$ is normal, hence regular in codimension one, hence regular. Now we can apply Chow's lemma, which says that proper maps are almost projective, and in this case \emph{already} projective.

In fact we have that \textbf{any regular proper integral curve is projective}. To see this, we use Chow's lemma.
\end{proof}

\end{Nov8}
\begin{Nov10}
\section*{10 November}
\subsection*{Divisors}
Let $\calO$ be a Dedekind domain, e.g.\ $\Z[\sqrt3]$. How do we specify a nonzero fractional ideal\footnote{A fractional ideal is a finitely generated $\calO$-submodule of $\Frac\calO$} of $\calO$? 
\begin{enumerate}\squishlist
\item By a factorisation:
\[\prod_{\frakp}\frakp^{n_\frakp}\]
taken over nonzero prime ideals of $\calO$, with only finitely many $n_\frakp$ nonzero.
\item By giving generators locally. For example, if $I=(2,1+\sqrt{-5})$, then $I[1/2]=\calO[1/2]\cdot2$ and $I[1/3]=\calO[1/3]\cdot(1+\sqrt{-5})$.
\item By forming the line bundle $\widetilde{I}$ on $\spec\calO$.
\end{enumerate}
These correspond to:
\begin{enumerate}\squishlist
\item Weil divisors;
\item Cartier divisors;
\item Invertible sheaves.
\end{enumerate}
Now assume the following:
\begin{equation}\tag{$*$}
X\textup{ is a noeth integral separated scheme that is regular in codimension 1}
\end{equation}
Let $K$ be the function field of $X$.
\begin{defn*}
A prime divisor on $X$ is an integral closed subscheme $Y$ of codimension one. This corresponds to a point $y\in X$ of codimension one, so that $\calO_{X,y}$ is a DVR of $K$. This corresponds to a valuation $\nu_Y:K\to\Z\cup\{\infty\}$.

A Weil divisor is a formal linear combination of prime divisors. We define the support of a Weil divisor as the union of the prime divisors which appear therein. A Weil divisor is effective \Iff all the coefficients are nonnegative.

There is a homomorphism
\[K^\times\overset{\textup{div}}{\to}\WeilDivisors X,\qquad f\mapsto\sum \nu_Y(f)Y.\]
\end{defn*}
\begin{exmp*}
Let $X=\A^2$, $K=k(x,y)$, and $f=\frac{y-x^2}{y^2}$. Then $\textup{div}(f)=Y_2-2Y_1$, where $Y_2$ is the parabola, and $Y_1$ is the $x$-axis.
\end{exmp*}
\begin{lem*}
$\textup{div}(f)$ is well-defined, i.e.\ $\nu_Y(f)=0$ for all but finitely many $Y$.
\end{lem*}
\begin{proof}
Fix a nonempty open $\spec A\subset X$, for $A$ an integral domain. $f$ can be written as a fraction, say $a_1/a_2$, of nonzero elements of $A$. Let $U=D(a_1a_2)$, the open set where these are invertible. Then $f$ is a unit at any codimension one point contained in $U$. Those $Y$ such that $v_Y(f)\neq0$ are contained in $X\setminus U$. This is closed of codimension one, so that the $Y$ that could appear are irreducible components. There are only finitely many of these, as $X$ is noetherian.
\end{proof}
\begin{defn*}
The principal divisors are those in the image of $\textup{div}$. The quotient is the divisor class group $\DivisorClass(X)$, and the equivalence relation is called `linear equivalence'.
\end{defn*}
\begin{exmp*}
If $\calO$ is a Dedekind domain, then the prime divisors on $\spec\calO$ correspond to the nonzero prime ideals of $\calO$. Divisor correspond to linear combinations of these, i.e.\ fractional ideals, and principal divisors correspond to principal ideals, so that one obtains the usual class group.
\end{exmp*}
\begin{exmp*}
Let $k=\overline k$. The prime divisors of $\A^1=\spec k[t]$ are the closed points, each of which is $\textup{div}(t-a)$ for $a\in k$. Thus $\DivisorClass(\A^1)=0$.
\end{exmp*}
\begin{prop*}
Suppose that $A$ is an integral domain. Then $A$ is a UFD \Iff $\spec A$ is normal ($A$ is integrally closed) and $\DivisorClass(\spec A)=0$.
\end{prop*}
\begin{prop*}
Suppose that $k$ is a field. Then
\begin{enumerate}\squishlist
\item $\DivisorClass(\A^r)=0$ (by the previous proposition).
\item $\DivisorClass(\PP^r)=\Z$, generated by the class of a hyperplane.
\end{enumerate}
\end{prop*}
\begin{proof}
We only need 2. The prime divisors are to the zero sets of homogeneous irreducible poynomials. Hrm:
\[\{\textup{homog.\ irred $g\in k[x_0,\ldots,x_r]$}\}\leftrightarrow\{\textup{prime divisors}\}\]
Any $f\in K^*$ factors as
$\prod g_i^{n_i}$
where the $g_i$ are homogeneous irreducible polynomials, the $n_i\in\Z$ are almost all zero, and $\sum n_i\textup{deg}g_i=0$.

Now we can define $\deg:\WeilDivisors\PP^r\to\Z$ mapping a prime divisor to its degree. Then the principal divisors are exactly the divisors of degree zero, and as $\deg$ is surjective, it descends to an isomorphism $\DivisorClass\to\Z$.
\end{proof}
\begin{prop*}
Suppose that $X$ satisfies ($*$), and $Z\subset X$ is a proper closed subset. Let $U=X\setminus Z$, a dense open subset of $X$. Then there is an isomorphism
\[\frac{\DivisorClass X}{\langle \textup{classes of prime divisors of $X$ contained in $Z$}\rangle}\weakequiv\DivisorClass U.\]
\end{prop*}
\begin{proof}
Prime divisors of $X$ not contained in $Z$ are in bijection with prime divisors of $U$ (by taking intersection with $U$/closure in $X$). Let $S$ be the set of prime divisors of $X$ contained in $Z$. Then there's a morphism of short exact sequences:
\[\xymatrix{
0\ar[r]&0\ar[d]\ar[r]&K^\times\ar[r]\ar[d]&K^\times\ar[d]\ar[r]&0\\
0\ar[r]&
\Z^S\ar[r]&
\WeilDivisors X\ar[r]&
\WeilDivisors U\ar[r]&
0
}\]
The induced exact sequence on cokernels is our result.
\end{proof}
\begin{cor*}
Let $k$ be a number field, let $S$ be a set of places (a bunch of `absolute value' functions on the field), let $\calO_k$ be the ring of integers, and $\calO_{k,S}=\{x\in k\ :\ v(x)\geq0\ \ \forall\ v\notin S\}$ be the ring of $S$-integers.\footnote{For example, $k=\Q$, $S=\{2,3\}$, $\calO_k=\Z$, $\calO_{k,S}=\Z[1/2,1/3]$.} Then $\spec\calO_{n,S}=\spec\calO_k\setminus S$, and
\[\DivisorClass\calO_{k,S}\cong\frac{\DivisorClass\calO_k}{\langle \textup{classes of primes in $S$}\rangle}\]
In particular, as the class group is finite, we can choose finitely many primes at which to invert in order to obtain a UFD!
\end{cor*}
\end{Nov10}

\end{document}

























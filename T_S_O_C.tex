% !TEX root = z_output/_T_S_O_C.tex
%%%%%%%%%%%%%%%%%%%%%%%%%%%%%%%%%%%%%%%%%%%%%%%%%%%%%%%%%%%%%%%%%%%%%%%%%%%%%%%%
%%%%%%%%%%%%%%%%%%%%%%%%%%% 80 characters %%%%%%%%%%%%%%%%%%%%%%%%%%%%%%%%%%%%%%
%%%%%%%%%%%%%%%%%%%%%%%%%%%%%%%%%%%%%%%%%%%%%%%%%%%%%%%%%%%%%%%%%%%%%%%%%%%%%%%%
\documentclass[11pt]{article}
\newcommand{\dontloaddefinitionsoftheoremenvironments}{}
\newcommand{\dontloadhyperref}{}
\usepackage{fullpage}
\usepackage{amsmath,amsthm,amssymb}
\usepackage{mathrsfs,nicefrac}
\usepackage{amssymb}
\usepackage{epsfig}
\usepackage[all,2cell]{xy}
\usepackage{sseq}
\usepackage{tocloft}
\usepackage{cancel}
\usepackage[strict]{changepage}
\usepackage{color}
\usepackage{tikz}
\usepackage{extpfeil}
\usepackage{version}
\usepackage{framed}
\definecolor{shadecolor}{rgb}{.925,0.925,0.925}

%\usepackage{ifthen}
%Used for disabling hyperref
\ifx\dontloadhyperref\undefined
%\usepackage[pdftex,pdfborder={0 0 0 [1 1]}]{hyperref}
\usepackage[pdftex,pdfborder={0 0 .5 [1 1]}]{hyperref}
\else
\providecommand{\texorpdfstring}[2]{#1}
\fi
%>>>>>>>>>>>>>>>>>>>>>>>>>>>>>>
%<<<        Versions        <<<
%>>>>>>>>>>>>>>>>>>>>>>>>>>>>>>
%Add in the following line to include all the versions.
%\def\excludeversion#1{\includeversion{#1}}

%>>>>>>>>>>>>>>>>>>>>>>>>>>>>>>
%<<<       Better ToC       <<<
%>>>>>>>>>>>>>>>>>>>>>>>>>>>>>>
\setlength{\cftbeforesecskip}{0.5ex}

%>>>>>>>>>>>>>>>>>>>>>>>>>>>>>>
%<<<      Hyperref mod      <<<
%>>>>>>>>>>>>>>>>>>>>>>>>>>>>>>

%needs more testing
\newcounter{dummyforrefstepcounter}
\newcommand{\labelRIGHTHERE}[1]
{\refstepcounter{dummyforrefstepcounter}\label{#1}}


%>>>>>>>>>>>>>>>>>>>>>>>>>>>>>>
%<<<  Theorem Environments  <<<
%>>>>>>>>>>>>>>>>>>>>>>>>>>>>>>
\ifx\dontloaddefinitionsoftheoremenvironments\undefined
\theoremstyle{plain}
\newtheorem{thm}{Theorem}[section]
\newtheorem*{thm*}{Theorem}
\newtheorem{lem}[thm]{Lemma}
\newtheorem*{lem*}{Lemma}
\newtheorem{prop}[thm]{Proposition}
\newtheorem*{prop*}{Proposition}
\newtheorem{cor}[thm]{Corollary}
\newtheorem*{cor*}{Corollary}
\newtheorem{defprop}[thm]{Definition-Proposition}
\newtheorem*{punchline}{Punchline}
\newtheorem*{conjecture}{Conjecture}
\newtheorem*{claim}{Claim}

\theoremstyle{definition}
\newtheorem{defn}{Definition}[section]
\newtheorem*{defn*}{Definition}
\newtheorem{exmp}{Example}[section]
\newtheorem*{exmp*}{Example}
\newtheorem*{exmps*}{Examples}
\newtheorem*{nonexmp*}{Non-example}
\newtheorem{asspt}{Assumption}[section]
\newtheorem{notation}{Notation}[section]
\newtheorem{exercise}{Exercise}[section]
\newtheorem*{fact*}{Fact}
\newtheorem*{rmk*}{Remark}
\newtheorem{fact}{Fact}
\newtheorem*{aside}{Aside}
\newtheorem*{question}{Question}
\newtheorem*{answer}{Answer}

\else\relax\fi

%>>>>>>>>>>>>>>>>>>>>>>>>>>>>>>
%<<<      Fields, etc.      <<<
%>>>>>>>>>>>>>>>>>>>>>>>>>>>>>>
\DeclareSymbolFont{AMSb}{U}{msb}{m}{n}
\DeclareMathSymbol{\N}{\mathbin}{AMSb}{"4E}
\DeclareMathSymbol{\Octonions}{\mathbin}{AMSb}{"4F}
\DeclareMathSymbol{\Z}{\mathbin}{AMSb}{"5A}
\DeclareMathSymbol{\R}{\mathbin}{AMSb}{"52}
\DeclareMathSymbol{\Q}{\mathbin}{AMSb}{"51}
\DeclareMathSymbol{\PP}{\mathbin}{AMSb}{"50}
\DeclareMathSymbol{\I}{\mathbin}{AMSb}{"49}
\DeclareMathSymbol{\C}{\mathbin}{AMSb}{"43}
\DeclareMathSymbol{\A}{\mathbin}{AMSb}{"41}
\DeclareMathSymbol{\F}{\mathbin}{AMSb}{"46}
\DeclareMathSymbol{\G}{\mathbin}{AMSb}{"47}
\DeclareMathSymbol{\Quaternions}{\mathbin}{AMSb}{"48}


%>>>>>>>>>>>>>>>>>>>>>>>>>>>>>>
%<<<       Operators        <<<
%>>>>>>>>>>>>>>>>>>>>>>>>>>>>>>
\DeclareMathOperator{\ad}{\textbf{ad}}
\DeclareMathOperator{\coker}{coker}
\renewcommand{\ker}{\textup{ker}\,}
\DeclareMathOperator{\End}{End}
\DeclareMathOperator{\Aut}{Aut}
\DeclareMathOperator{\Hom}{Hom}
\DeclareMathOperator{\Maps}{Maps}
\DeclareMathOperator{\Mor}{Mor}
\DeclareMathOperator{\Gal}{Gal}
\DeclareMathOperator{\Ext}{Ext}
\DeclareMathOperator{\Tor}{Tor}
\DeclareMathOperator{\Map}{Map}
\DeclareMathOperator{\Der}{Der}
\DeclareMathOperator{\Rad}{Rad}
\DeclareMathOperator{\rank}{rank}
\DeclareMathOperator{\ArfInvariant}{Arf}
\DeclareMathOperator{\KervaireInvariant}{Ker}
\DeclareMathOperator{\im}{im}
\DeclareMathOperator{\coim}{coim}
\DeclareMathOperator{\trace}{tr}
\DeclareMathOperator{\supp}{supp}
\DeclareMathOperator{\ann}{ann}
\DeclareMathOperator{\spec}{Spec}
\DeclareMathOperator{\SPEC}{\textbf{Spec}}
\DeclareMathOperator{\proj}{Proj}
\DeclareMathOperator{\PROJ}{\textbf{Proj}}
\DeclareMathOperator{\fiber}{F}
\DeclareMathOperator{\cofiber}{C}
\DeclareMathOperator{\cone}{cone}
\DeclareMathOperator{\skel}{sk}
\DeclareMathOperator{\coskel}{cosk}
\DeclareMathOperator{\conn}{conn}
\DeclareMathOperator{\colim}{colim}
\DeclareMathOperator{\limit}{lim}
\DeclareMathOperator{\ch}{ch}
\DeclareMathOperator{\Vect}{Vect}
\DeclareMathOperator{\GrthGrp}{GrthGp}
\DeclareMathOperator{\Sym}{Sym}
\DeclareMathOperator{\Prob}{\mathbb{P}}
\DeclareMathOperator{\Exp}{\mathbb{E}}
\DeclareMathOperator{\GeomMean}{\mathbb{G}}
\DeclareMathOperator{\Var}{Var}
\DeclareMathOperator{\Cov}{Cov}
\DeclareMathOperator{\Sp}{Sp}
\DeclareMathOperator{\Seq}{Seq}
\DeclareMathOperator{\Cyl}{Cyl}
\DeclareMathOperator{\Ev}{Ev}
\DeclareMathOperator{\sh}{sh}
\DeclareMathOperator{\intHom}{\underline{Hom}}
\DeclareMathOperator{\Frac}{frac}



%>>>>>>>>>>>>>>>>>>>>>>>>>>>>>>
%<<<   Cohomology Theories  <<<
%>>>>>>>>>>>>>>>>>>>>>>>>>>>>>>
\DeclareMathOperator{\KR}{{K\R}}
\DeclareMathOperator{\KO}{{KO}}
\DeclareMathOperator{\K}{{K}}
\DeclareMathOperator{\OmegaO}{{\Omega_{\Octonions}}}

%>>>>>>>>>>>>>>>>>>>>>>>>>>>>>>
%<<<   Algebraic Geometry   <<<
%>>>>>>>>>>>>>>>>>>>>>>>>>>>>>>
\DeclareMathOperator{\Spec}{Spec}
\DeclareMathOperator{\Proj}{Proj}
\DeclareMathOperator{\Sing}{Sing}
\DeclareMathOperator{\shfHom}{\mathscr{H}\textit{\!\!om}}
\DeclareMathOperator{\WeilDivisors}{{Div}}
\DeclareMathOperator{\CartierDivisors}{{CaDiv}}
\DeclareMathOperator{\PrincipalWeilDivisors}{{PrDiv}}
\DeclareMathOperator{\LocallyPrincipalWeilDivisors}{{LPDiv}}
\DeclareMathOperator{\PrincipalCartierDivisors}{{PrCaDiv}}
\DeclareMathOperator{\DivisorClass}{{Cl}}
\DeclareMathOperator{\CartierClass}{{CaCl}}
\DeclareMathOperator{\Picard}{{Pic}}
\DeclareMathOperator{\Frob}{Frob}


%>>>>>>>>>>>>>>>>>>>>>>>>>>>>>>
%<<<  Mathematical Objects  <<<
%>>>>>>>>>>>>>>>>>>>>>>>>>>>>>>
\newcommand{\sll}{\mathfrak{sl}}
\newcommand{\gl}{\mathfrak{gl}}
\newcommand{\GL}{\mbox{GL}}
\newcommand{\PGL}{\mbox{PGL}}
\newcommand{\SL}{\mbox{SL}}
\newcommand{\Mat}{\mbox{Mat}}
\newcommand{\Gr}{\textup{Gr}}
\newcommand{\Squ}{\textup{Sq}}
\newcommand{\catSet}{\textit{Sets}}
\newcommand{\RP}{{\R\PP}}
\newcommand{\CP}{{\C\PP}}
\newcommand{\Steen}{\mathscr{A}}
\newcommand{\Orth}{\textup{\textbf{O}}}

%>>>>>>>>>>>>>>>>>>>>>>>>>>>>>>
%<<<  Mathematical Symbols  <<<
%>>>>>>>>>>>>>>>>>>>>>>>>>>>>>>
\newcommand{\DASH}{\textup{---}}
\newcommand{\op}{\textup{op}}
\newcommand{\CW}{\textup{CW}}
\newcommand{\ob}{\textup{ob}\,}
\newcommand{\ho}{\textup{ho}}
\newcommand{\st}{\textup{st}}
\newcommand{\id}{\textup{id}}
\newcommand{\Bullet}{\ensuremath{\bullet} }
\newcommand{\sprod}{\wedge}

%>>>>>>>>>>>>>>>>>>>>>>>>>>>>>>
%<<<      Some Arrows       <<<
%>>>>>>>>>>>>>>>>>>>>>>>>>>>>>>
\newcommand{\nt}{\Longrightarrow}
\let\shortmapsto\mapsto
\let\mapsto\longmapsto
\newcommand{\mapsfrom}{\,\reflectbox{$\mapsto$}\ }
\newcommand{\bigrightsquig}{\scalebox{2}{\ensuremath{\rightsquigarrow}}}
\newcommand{\bigleftsquig}{\reflectbox{\scalebox{2}{\ensuremath{\rightsquigarrow}}}}

%\newcommand{\cofibration}{\xhookrightarrow{\phantom{\ \,{\sim\!}\ \ }}}
%\newcommand{\fibration}{\xtwoheadrightarrow{\phantom{\sim\!}}}
%\newcommand{\acycliccofibration}{\xhookrightarrow{\ \,{\sim\!}\ \ }}
%\newcommand{\acyclicfibration}{\xtwoheadrightarrow{\sim\!}}
%\newcommand{\leftcofibration}{\xhookleftarrow{\phantom{\ \,{\sim\!}\ \ }}}
%\newcommand{\leftfibration}{\xtwoheadleftarrow{\phantom{\sim\!}}}
%\newcommand{\leftacycliccofibration}{\xhookleftarrow{\ \ {\sim\!}\,\ }}
%\newcommand{\leftacyclicfibration}{\xtwoheadleftarrow{\sim\!}}
%\newcommand{\weakequiv}{\xrightarrow{\ \,\sim\,\ }}
%\newcommand{\leftweakequiv}{\xleftarrow{\ \,\sim\,\ }}

\newcommand{\cofibration}
{\xhookrightarrow{\phantom{\ \,{\raisebox{-.3ex}[0ex][0ex]{\scriptsize$\sim$}\!}\ \ }}}
\newcommand{\fibration}
{\xtwoheadrightarrow{\phantom{\raisebox{-.3ex}[0ex][0ex]{\scriptsize$\sim$}\!}}}
\newcommand{\acycliccofibration}
{\xhookrightarrow{\ \,{\raisebox{-.55ex}[0ex][0ex]{\scriptsize$\sim$}\!}\ \ }}
\newcommand{\acyclicfibration}
{\xtwoheadrightarrow{\raisebox{-.6ex}[0ex][0ex]{\scriptsize$\sim$}\!}}
\newcommand{\leftcofibration}
{\xhookleftarrow{\phantom{\ \,{\raisebox{-.3ex}[0ex][0ex]{\scriptsize$\sim$}\!}\ \ }}}
\newcommand{\leftfibration}
{\xtwoheadleftarrow{\phantom{\raisebox{-.3ex}[0ex][0ex]{\scriptsize$\sim$}\!}}}
\newcommand{\leftacycliccofibration}
{\xhookleftarrow{\ \ {\raisebox{-.55ex}[0ex][0ex]{\scriptsize$\sim$}\!}\,\ }}
\newcommand{\leftacyclicfibration}
{\xtwoheadleftarrow{\raisebox{-.6ex}[0ex][0ex]{\scriptsize$\sim$}\!}}
\newcommand{\weakequiv}
{\xrightarrow{\ \,\raisebox{-.3ex}[0ex][0ex]{\scriptsize$\sim$}\,\ }}
\newcommand{\leftweakequiv}
{\xleftarrow{\ \,\raisebox{-.3ex}[0ex][0ex]{\scriptsize$\sim$}\,\ }}

%>>>>>>>>>>>>>>>>>>>>>>>>>>>>>>
%<<<    xymatrix Arrows     <<<
%>>>>>>>>>>>>>>>>>>>>>>>>>>>>>>
\newdir{ >}{{}*!/-5pt/@{>}}
\newcommand{\xycof}{\ar@{ >->}}
\newcommand{\xycofib}{\ar@{^{(}->}}
\newcommand{\xycofibdown}{\ar@{_{(}->}}
\newcommand{\xyfib}{\ar@{->>}}
\newcommand{\xymapsto}{\ar@{|->}}

%>>>>>>>>>>>>>>>>>>>>>>>>>>>>>>
%<<<     Greek Letters      <<<
%>>>>>>>>>>>>>>>>>>>>>>>>>>>>>>
%\newcommand{\oldphi}{\phi}
%\renewcommand{\phi}{\varphi}
\let\oldphi\phi
\let\phi\varphi
\renewcommand{\to}{\longrightarrow}
\newcommand{\from}{\longleftarrow}
\newcommand{\eps}{\varepsilon}

%>>>>>>>>>>>>>>>>>>>>>>>>>>>>>>
%<<<  1st-4th & parentheses <<<
%>>>>>>>>>>>>>>>>>>>>>>>>>>>>>>
\newcommand{\first}{^\text{st}}
\newcommand{\second}{^\text{nd}}
\newcommand{\third}{^\text{rd}}
\newcommand{\fourth}{^\text{th}}
\newcommand{\ZEROTH}{$0^\text{th}$ }
\newcommand{\FIRST}{$1^\text{st}$ }
\newcommand{\SECOND}{$2^\text{nd}$ }
\newcommand{\THIRD}{$3^\text{rd}$ }
\newcommand{\FOURTH}{$4^\text{th}$ }
\newcommand{\iTH}{$i^\text{th}$ }
\newcommand{\jTH}{$j^\text{th}$ }
\newcommand{\nTH}{$n^\text{th}$ }

%>>>>>>>>>>>>>>>>>>>>>>>>>>>>>>
%<<<    upright commands    <<<
%>>>>>>>>>>>>>>>>>>>>>>>>>>>>>>
\newcommand{\upcol}{\textup{:}}
\newcommand{\upsemi}{\textup{;}}
\providecommand{\lparen}{\textup{(}}
\providecommand{\rparen}{\textup{)}}
\renewcommand{\lparen}{\textup{(}}
\renewcommand{\rparen}{\textup{)}}
\newcommand{\Iff}{\emph{iff} }

%>>>>>>>>>>>>>>>>>>>>>>>>>>>>>>
%<<<     Environments       <<<
%>>>>>>>>>>>>>>>>>>>>>>>>>>>>>>
\newcommand{\squishlist}
{ %\setlength{\topsep}{100pt} doesn't seem to do anything.
  \setlength{\itemsep}{.5pt}
  \setlength{\parskip}{0pt}
  \setlength{\parsep}{0pt}}
\newenvironment{itemise}{
\begin{list}{\textup{$\rightsquigarrow$}}
   {  \setlength{\topsep}{1mm}
      \setlength{\itemsep}{1pt}
      \setlength{\parskip}{0pt}
      \setlength{\parsep}{0pt}
   }
}{\end{list}\vspace{-.1cm}}
\newcommand{\INDENT}{\textbf{}\phantom{space}}
\renewcommand{\INDENT}{\rule{.7cm}{0cm}}

\newcommand{\itm}[1][$\rightsquigarrow$]{\item[{\makebox[.5cm][c]{\textup{#1}}}]}


%\newcommand{\rednote}[1]{{\color{red}#1}\makebox[0cm][l]{\scalebox{.1}{rednote}}}
%\newcommand{\bluenote}[1]{{\color{blue}#1}\makebox[0cm][l]{\scalebox{.1}{rednote}}}

\newcommand{\rednote}[1]
{{\color{red}#1}\makebox[0cm][l]{\scalebox{.1}{\rotatebox{90}{?????}}}}
\newcommand{\bluenote}[1]
{{\color{blue}#1}\makebox[0cm][l]{\scalebox{.1}{\rotatebox{90}{?????}}}}


\newcommand{\funcdef}[4]{\begin{align*}
#1&\to #2\\
#3&\mapsto#4
\end{align*}}

%\newcommand{\comment}[1]{}

%>>>>>>>>>>>>>>>>>>>>>>>>>>>>>>
%<<<       Categories       <<<
%>>>>>>>>>>>>>>>>>>>>>>>>>>>>>>
\newcommand{\Ens}{{\mathscr{E}ns}}
\DeclareMathOperator{\Sheaves}{{\mathsf{Shf}}}
\DeclareMathOperator{\Presheaves}{{\mathsf{PreShf}}}
\DeclareMathOperator{\Psh}{{\mathsf{Psh}}}
\DeclareMathOperator{\Shf}{{\mathsf{Shf}}}
\DeclareMathOperator{\Varieties}{{\mathsf{Var}}}
\DeclareMathOperator{\Schemes}{{\mathsf{Sch}}}
\DeclareMathOperator{\Rings}{{\mathsf{Rings}}}
\DeclareMathOperator{\AbGp}{{\mathsf{AbGp}}}
\DeclareMathOperator{\Modules}{{\mathsf{\!-Mod}}}
\DeclareMathOperator{\fgModules}{{\mathsf{\!-Mod}^{\textup{fg}}}}
\DeclareMathOperator{\QuasiCoherent}{{\mathsf{QCoh}}}
\DeclareMathOperator{\Coherent}{{\mathsf{Coh}}}
\DeclareMathOperator{\GSW}{{\mathcal{SW}^G}}
\DeclareMathOperator{\Burnside}{{\mathsf{Burn}}}
\DeclareMathOperator{\GSet}{{G\mathsf{Set}}}
\DeclareMathOperator{\FinGSet}{{G\mathsf{Set}^\textup{fin}}}
\DeclareMathOperator{\HSet}{{H\mathsf{Set}}}
\DeclareMathOperator{\Cat}{{\mathsf{Cat}}}
\DeclareMathOperator{\Fun}{{\mathsf{Fun}}}
\DeclareMathOperator{\Orb}{{\mathsf{Orb}}}
\DeclareMathOperator{\Set}{{\mathsf{Set}}}
\DeclareMathOperator{\sSet}{{\mathsf{sSet}}}
\DeclareMathOperator{\Top}{{\mathsf{Top}}}
\DeclareMathOperator{\GSpectra}{{G-\mathsf{Spectra}}}
\DeclareMathOperator{\Lan}{Lan}
\DeclareMathOperator{\Ran}{Ran}

%>>>>>>>>>>>>>>>>>>>>>>>>>>>>>>
%<<<     Script Letters     <<<
%>>>>>>>>>>>>>>>>>>>>>>>>>>>>>>
\newcommand{\scrQ}{\mathscr{Q}}
\newcommand{\scrW}{\mathscr{W}}
\newcommand{\scrE}{\mathscr{E}}
\newcommand{\scrR}{\mathscr{R}}
\newcommand{\scrT}{\mathscr{T}}
\newcommand{\scrY}{\mathscr{Y}}
\newcommand{\scrU}{\mathscr{U}}
\newcommand{\scrI}{\mathscr{I}}
\newcommand{\scrO}{\mathscr{O}}
\newcommand{\scrP}{\mathscr{P}}
\newcommand{\scrA}{\mathscr{A}}
\newcommand{\scrS}{\mathscr{S}}
\newcommand{\scrD}{\mathscr{D}}
\newcommand{\scrF}{\mathscr{F}}
\newcommand{\scrG}{\mathscr{G}}
\newcommand{\scrH}{\mathscr{H}}
\newcommand{\scrJ}{\mathscr{J}}
\newcommand{\scrK}{\mathscr{K}}
\newcommand{\scrL}{\mathscr{L}}
\newcommand{\scrZ}{\mathscr{Z}}
\newcommand{\scrX}{\mathscr{X}}
\newcommand{\scrC}{\mathscr{C}}
\newcommand{\scrV}{\mathscr{V}}
\newcommand{\scrB}{\mathscr{B}}
\newcommand{\scrN}{\mathscr{N}}
\newcommand{\scrM}{\mathscr{M}}

%>>>>>>>>>>>>>>>>>>>>>>>>>>>>>>
%<<<     Fractur Letters    <<<
%>>>>>>>>>>>>>>>>>>>>>>>>>>>>>>
\newcommand{\frakQ}{\mathfrak{Q}}
\newcommand{\frakW}{\mathfrak{W}}
\newcommand{\frakE}{\mathfrak{E}}
\newcommand{\frakR}{\mathfrak{R}}
\newcommand{\frakT}{\mathfrak{T}}
\newcommand{\frakY}{\mathfrak{Y}}
\newcommand{\frakU}{\mathfrak{U}}
\newcommand{\frakI}{\mathfrak{I}}
\newcommand{\frakO}{\mathfrak{O}}
\newcommand{\frakP}{\mathfrak{P}}
\newcommand{\frakA}{\mathfrak{A}}
\newcommand{\frakS}{\mathfrak{S}}
\newcommand{\frakD}{\mathfrak{D}}
\newcommand{\frakF}{\mathfrak{F}}
\newcommand{\frakG}{\mathfrak{G}}
\newcommand{\frakH}{\mathfrak{H}}
\newcommand{\frakJ}{\mathfrak{J}}
\newcommand{\frakK}{\mathfrak{K}}
\newcommand{\frakL}{\mathfrak{L}}
\newcommand{\frakZ}{\mathfrak{Z}}
\newcommand{\frakX}{\mathfrak{X}}
\newcommand{\frakC}{\mathfrak{C}}
\newcommand{\frakV}{\mathfrak{V}}
\newcommand{\frakB}{\mathfrak{B}}
\newcommand{\frakN}{\mathfrak{N}}
\newcommand{\frakM}{\mathfrak{M}}

\newcommand{\frakq}{\mathfrak{q}}
\newcommand{\frakw}{\mathfrak{w}}
\newcommand{\frake}{\mathfrak{e}}
\newcommand{\frakr}{\mathfrak{r}}
\newcommand{\frakt}{\mathfrak{t}}
\newcommand{\fraky}{\mathfrak{y}}
\newcommand{\fraku}{\mathfrak{u}}
\newcommand{\fraki}{\mathfrak{i}}
\newcommand{\frako}{\mathfrak{o}}
\newcommand{\frakp}{\mathfrak{p}}
\newcommand{\fraka}{\mathfrak{a}}
\newcommand{\fraks}{\mathfrak{s}}
\newcommand{\frakd}{\mathfrak{d}}
\newcommand{\frakf}{\mathfrak{f}}
\newcommand{\frakg}{\mathfrak{g}}
\newcommand{\frakh}{\mathfrak{h}}
\newcommand{\frakj}{\mathfrak{j}}
\newcommand{\frakk}{\mathfrak{k}}
\newcommand{\frakl}{\mathfrak{l}}
\newcommand{\frakz}{\mathfrak{z}}
\newcommand{\frakx}{\mathfrak{x}}
\newcommand{\frakc}{\mathfrak{c}}
\newcommand{\frakv}{\mathfrak{v}}
\newcommand{\frakb}{\mathfrak{b}}
\newcommand{\frakn}{\mathfrak{n}}
\newcommand{\frakm}{\mathfrak{m}}

%>>>>>>>>>>>>>>>>>>>>>>>>>>>>>>
%<<<  Caligraphic Letters   <<<
%>>>>>>>>>>>>>>>>>>>>>>>>>>>>>>
\newcommand{\calQ}{\mathcal{Q}}
\newcommand{\calW}{\mathcal{W}}
\newcommand{\calE}{\mathcal{E}}
\newcommand{\calR}{\mathcal{R}}
\newcommand{\calT}{\mathcal{T}}
\newcommand{\calY}{\mathcal{Y}}
\newcommand{\calU}{\mathcal{U}}
\newcommand{\calI}{\mathcal{I}}
\newcommand{\calO}{\mathcal{O}}
\newcommand{\calP}{\mathcal{P}}
\newcommand{\calA}{\mathcal{A}}
\newcommand{\calS}{\mathcal{S}}
\newcommand{\calD}{\mathcal{D}}
\newcommand{\calF}{\mathcal{F}}
\newcommand{\calG}{\mathcal{G}}
\newcommand{\calH}{\mathcal{H}}
\newcommand{\calJ}{\mathcal{J}}
\newcommand{\calK}{\mathcal{K}}
\newcommand{\calL}{\mathcal{L}}
\newcommand{\calZ}{\mathcal{Z}}
\newcommand{\calX}{\mathcal{X}}
\newcommand{\calC}{\mathcal{C}}
\newcommand{\calV}{\mathcal{V}}
\newcommand{\calB}{\mathcal{B}}
\newcommand{\calN}{\mathcal{N}}
\newcommand{\calM}{\mathcal{M}}

%>>>>>>>>>>>>>>>>>>>>>>>>>>>>>>
%<<<<<<<<<DEPRECIATED<<<<<<<<<<
%>>>>>>>>>>>>>>>>>>>>>>>>>>>>>>

%%% From Kac's template
% 1-inch margins, from fullpage.sty by H.Partl, Version 2, Dec. 15, 1988.
%\topmargin 0pt
%\advance \topmargin by -\headheight
%\advance \topmargin by -\headsep
%\textheight 9.1in
%\oddsidemargin 0pt
%\evensidemargin \oddsidemargin
%\marginparwidth 0.5in
%\textwidth 6.5in
%
%\parindent 0in
%\parskip 1.5ex
%%\renewcommand{\baselinestretch}{1.25}

%%% From the net
%\newcommand{\pullbackcorner}[1][dr]{\save*!/#1+1.2pc/#1:(1,-1)@^{|-}\restore}
%\newcommand{\pushoutcorner}[1][dr]{\save*!/#1-1.2pc/#1:(-1,1)@^{|-}\restore}









%\renewcommand{\comment}{} % comment to suppress previous chapters

\swapnumbers
\theoremstyle{plain}
\newtheorem{thm}{Theorem}[subsection]
\newtheorem*{thm*}{Theorem}
\newtheorem{lem}[thm]{Lemma}
\newtheorem*{lem*}{Lemma}
\newtheorem{prop}[thm]{Proposition}
\newtheorem*{prop*}{Proposition}
\newtheorem{cor}[thm]{Corollary}
\newtheorem*{cor*}{Corollary}
\newtheorem*{statement*}{Statement}
\newtheorem{lemdef}[thm]{Lemma-Definition}
\newtheorem{thmdef}[thm]{Theorem-Definition}
\newtheorem{propdef}[thm]{Proposition-Definition}
\theoremstyle{definition}
\newtheorem{condef}[thm]{Construction-Definition}
\newtheorem{const}[thm]{Construction}
\newtheorem{defn}[thm]{Definition}
\newtheorem*{defn*}{Definition}
\newtheorem{exmp}[thm]{Example}
\newtheorem{rmk}[thm]{Remark}
\newtheorem{nonexample}[thm]{Non-example}
\renewcommand{\thesection}{\Roman{section}}


\newcommand{\HS}{\scrH\!\scrS}
\newcommand{\HC}{\scrH\!\scrC}
\newcommand{\CW}{\scrC}
\newcommand{\AG}{\scrA\!\scrG}
\newcommand{\CWf}{\scrC_\textup{f}}
\newcommand{\CWfd}{\scrC_\textup{fd}}
\newcommand{\Specf}{\scrC_\textup{f}}
\newcommand{\Specfd}{\scrC_\textup{fd}}
\newcommand{\Specs}{\scrC_\textup{s}}
\newcommand{\Specsfd}{\scrC_\textup{sfd}}

\title{On Thom Spectra, Orientability, and Cobordism\small{ --- Rudyak}}

\begin{document}
%\tableofcontents
\setcounter{section}{1}
\section{Spectra and \lparen co\rparen homology theories}
\subsection{Preliminaries on spectra}
\comment{
\begin{defn}[objects of $\scrS$]\hfil
\begin{itemise}
\itm[(a)]A spectrum is a sequence of pointed CW-complexes $E_n$ and pointed
CW-inclusions $s_n:SE_n\to E_{n+1}$.
\itm[(b)]A subspectrum is a sequence of sub-CW-complexes $E'_n$ such that $s_n$
maps $SE'_n$ into $E'_{n+1}$.
\itm[(c)]A filtration of $E$ is a sequence 
$\cdots\subset E(i)\subset E(i+1)\subset\cdots$ of subspectra such that the
union (levelwise) of the $E(i)$ equals $E$
\itm[(d)]$(\Sigma^kE)_n:=E_{k+n}$.
\itm[(e)]$(\Sigma^\infty X)_n:=S^nX$ if $n\geq0$, and $*$ is $n<0$.
\end{itemise}
\end{defn}
\begin{defn}[cells, skeleta, \ldots]\hfil
\begin{itemise}
\itm[(a)]
A cell of $E$ is $\{e,Se,S^2e,\ldots\}$, where $e$ is a $d$-cell of $E_n$ which
is not a suspension of a cell of $E_{n-1}$. It then has dimension $d-n$.
\itm[(b)]A subspectrum $E'\subset E$ is cofinal if all cells of $E$ are
eventually in $E'$.
\itm[(c)]The $n$-skeleton $E^{(n)}$ consists of all cells of dimension at most
$n$.
\itm[(d)]$E$ is finite if it has finitely many cells.
\itm[(e)]$E$ is of finite type if it has finitely many cells in each dimension.
\itm[(f)]$E$ is finite dimensional if $E=E^{(n)}$ for some $n$.
\itm[(g)]$E$ is a suspension spectrum if it equals $\Sigma^k\Sigma^\infty X$.
\end{itemise}
\end{defn}
\begin{defn}[morphisms in $\scrS$]\hfil
\begin{itemise}
\itm[(a)]A map of spectra is just a family of compatible pointed cellular maps.
\itm[(b)]A morphism of spectra $E\to F$ is an equivalence class of maps out of
cofinal subspectra $E'$. Two such maps are equivalent if they agree on some
subspectrum of the (levelwise) intersection $E'\cap E''$ which is cofinal in
$E$.
\end{itemise}
\end{defn}
Given a spectrum $E$ there are maps $S^kE_n\to E_{n+k}$ by composition. For any
$n$, there is a map $i_n:\Sigma^{-n}\Sigma^\infty E_n\to E$, as the proposed
domain $(n+k)\fourth$ space $S^kE_n$ (when $k\geq0$). Thus $i_n$ is the
inclusion of a subspectrum.
\setcounter{thm}{4}
\begin{prop}\hfil
\begin{itemise}
\itm[(i)] For any fixed $k$, the map $s_n$ sends $S((E_n)^{(n+k)})$ into
$(E_{n+1})^{(n+k+1)}$, as the maps $s_n$ are cellular. Thus $\{E_n^{n+k}\}$ is a
subspectrum of $E^{(k)}$. This subspectrum is in fact cofinal in $E^{(k)}$.
\itm[(ii)]If $E$ is finite, the spaces $E_n$ are all finite.
\itm[(iii)]If $E$ is finite, there exists $N$ such that 
$\Sigma^{-N}\Sigma^\infty E_N$ is cofinal in $E$.
\itm[NB:]Note also that if $E$ is finite, then every cofinal $E'$ contains
$\Sigma^{-N}\Sigma^\infty E_N$ for $N\gg0$.
\end{itemise}
\end{prop}
\begin{prop}[greatest element of a morphism]\hfil
\begin{itemise}
\itm[(i)] If maps $f,g:E\to F$ agree on cofinal $B\subset E$ then $f=g$.
\itm[(ii)] If $E',E''\subset E$ are cofinal and $f':E'\to F$ and $f'':E''\to F$
are equivalent, then $f'=f''$ on restriction to $E'\cap E''$.
\itm[(iii)] Any two maps representing the same morphism agree on the
intersection of their domains. In particular there is a unique map in every
morphism with support containing the support of all the others, and all the
others are restrictions thereof. Call this the greatest element of the morphism.
\end{itemise}
\end{prop}
\begin{defn}[cofiber]\hfil
\begin{itemise}
\itm[(a)] Given a map $f:E\to F$ define the cone $Cf=\{C(f_n)\}$. This is a
spectrum as $S(C(f_n))$ decomposes as $SF_n\cup C(SE_n)$. Note that $C(f_n)$ is
the unreduced mapping cone.
\itm[(b)] If $\phi:E\to F$ is a morphism with greatest element $f':E'\to F$,
the cone $C\phi$ is defined to be the cone $Cf'$, and called the cofiber.
\end{itemise}
\end{defn}
\begin{prop}
The cone of a morphism of finite spectra is a finite spectrum. The cone of a
morphism of spectra of finite type is of finite type.
\end{prop}
Given a CW-complex $X$, define the smash product $E\wedge X$ levelwise. Then
\begin{defn}[homotopy]\hfil
\begin{itemise}
\itm[(a)] Given maps $g_0,g_1:E\to F$, a homotopy $g_0\simeq g_1$ is a map
$G:E\wedge I^+\to F$ such that $G|_{E\wedge\{i,*\}}=g_i$. This is just a 
levelwise homotopy compatible with the suspension structure.
\itm[(b)] We say that morphisms $\phi_0,\phi_1:E\to F$ are homotopic if for some
cofinal spectrum $E'$, there are homotopic maps $g_i:E'\to F$ such that
$g_i\in\phi_i$.
\end{itemise}
\end{defn}
\noindent Thus we can form the homotopy category $\HS$ with objects spectra and
morphisms homotopy classes of morphisms. The isomorphisms here are called
equivalences. The cones of homotopic morphisms are equivalences.
\begin{thm}[Whitehead]
A morphism is an equivalence iff it induces isomorphisms on homotopy groups 
$\pi_*(\DASH):=[\Sigma^kS,\DASH]$ for all $k\in\Z$.
\end{thm}
\begin{prop}
The spectra $S^1\wedge E$ and $\Sigma E$ are equivalent.
\end{prop}
\begin{defn} 
A sequence $X\to Y\to Z$ in $\scrS$ is a \emph{cofiber sequence} if it admits a
homotopy commuting diagram as follows, with vertical arrows equivalences:
\[\xymatrix{
X\ar[r]\ar[d]&Y\ar[r]\ar[d]&Z\ar[d]\\
E\ar[r]^\phi&F\ar[r]&C\phi
}\]
\end{defn}
\begin{lem}\hfil
\begin{itemise}
\itm[(i)]If $X\to Y\to Z$ is a cofiber sequence, there is a map $Z\to \Sigma X$
such that $X\to Y\to Z\to \Sigma X$ is a long cofiber sequence.
\itm[(ii)]The construction of this map is somehow natural: given two cofiber
sequences and the full arrows, such that the left square is homotopy
commutative, we can find dotted arrows such that everything is homotopy
commutative \lparen with rows\ long cofiber sequences\rparen:
\[\xymatrix{
X\ar[r]\ar[d]^\alpha&
Y\ar[r]\ar[d]&Z\ar@{.>}[d]\ar@{.>}[r]&
\Sigma X\ar[d]^{\Sigma\alpha}\\
X\ar[r]&Y\ar[r]&Z\ar@{.>}[r]&\Sigma X'
}\]
\itm[(iii)]$\Sigma^\infty$ maps cofiber sequences of CW-complexes to cofiber
sequences of spectra.
\end{itemise}
\end{lem}
As suspension distributes over wedge, the levelwise wedge of spectra is a
spectrum, and
\begin{prop}
The wedge of spectra is the coproduct in $\HS$.
\end{prop}
As $[E,F]=[S^2\wedge E,\Sigma^2F]$ all the homsets have an abelian group
structure from the comulitplication on $S^2$. Moreover, composition is
biadditive, and $\HS$ is an additive category (see 1.16). As we can desuspend we
can form long cofiber sequences in both directions, and:
\begin{thm}
For any spectrum $E$, applying $\Hom(\DASH,E)$ to the long cofiber sequence
\[\xymatrix{
\cdots\ar[r]&\Sigma^{-1}Y\ar[r]^{\Sigma^{-1}g}&
\Sigma^{-1}Z\ar[r]&X\ar[r]^f&
Y\ar[r]^g&Z\ar[r]&\Sigma X\ar[r]^{\Sigma f}&\Sigma Y\ar[r]&\cdots
}\]
gives a long exact sequence, \textbf{and so does applying} $\Hom(E,\DASH)$.
\end{thm}
\begin{prop}\hfil
\begin{itemise}
\itm[(i)]For all $F$ there is an isomorphism of abelian groups:
\[\{i^*_\lambda\}:[\vee_\lambda E(\lambda)]\to\prod_\lambda[E(\lambda),F],
\qquad\{i^*_\lambda\}(f)=\{fi_\lambda\}.\]
\itm[(ii)]For all $F$ there is an isomorphism of abelian groups:
\[\langle(i_k)_*\rangle:\bigoplus_{k=1}^m\lambda[F,E(k)]\to
[F,\vee_{k=1}^m E(k)]\]
\itm[(iii)]For all finite $F$ there is an isomorphism of abelian groups:
\[\langle(i_k)_*\rangle:\oplus_\lambda[F,E(\lambda)]\to
[F,\vee_\lambda E(\lambda)]\]
\end{itemise}
\end{prop}
\noindent By 1.16(ii), $[X,X\vee X]\simeq[X,X]\oplus[X,X]$. Under this
isomorphism, $(\id_X,\id_X)$ corresponds to a homotopy class $\nabla:X\to X\vee
X$, called \emph{coaddition}. Moreover, addition in $[X,E]$ is given by
\[\xymatrix{[X,E]\oplus[X,E]\ar[r]^{\ \ \simeq}&
[X\vee X,E]\ar[r]^{\ \ \ \nabla^*}&[X,E].}\]
\begin{prop}
Let $\xymatrix{X\ar[r]^f&Y\ar[r]^g&Z}$ be a cofiber sequence. Then TFAE:
\begin{itemise}
\itm[(i)]The morphisms $g$ us innessential; and
\itm[(ii)]There is a splitting $s:Y\to X$ such that $fs\simeq\id_Y$.
\end{itemise}
If these hold then $X\simeq \Sigma^{-1}Z\vee Y$.
\end{prop}
\begin{defn}\hfil
\begin{itemise}
\itm A \emph{prespectrum} is simply a sequence of pointed spaces and pointed
maps.
\itm A CW-\emph{prespectrum} is one in which the spaces are pointed
CW-complexes, and the maps are cellular, but not CW-inclusions.
\end{itemise}
\end{defn}
\begin{lemdef}[spectral substitute]
For every prespectrum $X=\{X_n,t_n\}$ there is a spectrum $\{E_n,s_n\}$ and
pointed homotopy equivalences $f_n:E_n\to X_n$ forming a map of prespectra. Any
such $E$ is called a \emph{spectral substitute} for the prespectrum $X$.

If $X$ is a CW-spectrum in which for some $k$, $(X_n)^{(n+k)}=*$ for all $n$,
then we can choose $E$ such that $E^{(k)}=*$.
\end{lemdef}
\begin{defn}
A prespectrum $X$ is called an $\Omega$-\emph{prespectrum} if each adjoint
$X_n\to \Omega X_{n+1}$ is a homotopy equivalence. A spectrum is called an
$\Omega$-\emph{spectrum} if it is an $\Omega$-prespectrum.
\end{defn}
\begin{prop}[Adams]
Every spectrum is equivalent to an $\Omega$-spectrum.
\end{prop}
\begin{prop}
If $A,B\subset E$ are subspectra such that $A\cup B=E$, and $h:A\vee B\to E$ is
the wedge of inclusions, then $Ch\simeq\Sigma(A\cap B)$.
\end{prop}
\noindent A useful mapping telescope construction is given for spectra. To
appear.
}   %%% End comment %%%
\subsection{The smash product, duality, ring and module spectra}
\comment{
\begin{thm}
There is a construction $E,F\mapsto E\wedge F$ of smash product such that
\begin{itemise}
\itm[(i)]It is a covariant functor in each argument.
\itm[(ii)]The are the expected natural equivalences ``$a,\tau,l,r,\Sigma$'':
\[(E\wedge F)\wedge G\simeq E\wedge(F\wedge G),\ E\wedge F\simeq F\wedge E,\
S\wedge E\simeq E\simeq E\wedge S,\ \Sigma E\wedge F\simeq \Sigma(E\wedge F).\]
\itm[(iii)]There are natural equivalences $E\wedge X\simeq E\wedge \Sigma^\infty
X$ for CW-complexes $X$.
\itm[(iv)]If $f:E\to F$ is an equivalence, so is $f\wedge \id_G$.
\itm[(v)]The natural morphism $\vee_\lambda(E_\lambda\wedge
F)\to(\vee_\lambda(E_\lambda))\wedge F$ is an equivalence.
\itm[(vi)]If $A\to B\to C$ is a cofiber sequence, so is $A\wedge E\to B\wedge
E\to C\wedge E$.
\end{itemise}
\end{thm}
\begin{thm}
There are compatibilities up to homotopy between:
\begin{itemise}
\itm[(i)] Different ways to associate a quadruple smash;
\itm[(ii)] $\tau^2$ and the identity \lparen swapping order twice does
nothing\rparen;
\itm[(iii)] Associating and commuting terms;
\itm[(iv)] Smashing with $S$ and associating in various ways;
\itm[(vii)] Smashing with $S$ and commuting;
\itm[(viii)] $\tau:S\wedge S\to S\wedge S$ and $\id_s$.
\end{itemise}
\end{thm}
\begin{defn}
A map $u:S\to A\wedge A^\perp$ is called a duality if for all $E$, the maps
%\[[A,E]\to[S,E\wedge A^\perp],\qquad (A\to E)\mapsto (S\to A\wedge A^\perp\to 
\begin{alignat*}{2}
u_E:[A,E]\to[S,E\wedge A^\perp],&\qquad &
(A\to E)\mapsto (S\to A\wedge A^\perp\to E\wedge A^\perp)\\
u^E:[A^\perp,E]\to[S,A\wedge E],&\qquad &
(A^\perp\to E)\mapsto (S\to A\wedge A^\perp\to A\wedge E)
\end{alignat*}
are isomorphisms. Duality is symmetric, as smash product is homotopy
commutative. Moreover, given two dualities, we have isomorphisms
$[A,B]\longrightarrow[S,B\wedge A^\perp]\longleftarrow[B^\perp,A^\perp]$, so
that we can define the adjoint $f^\perp:B^\perp\to A^\perp$ of a map $f:A\to B$.
\end{defn}
\begin{lem}
Given a map $f:A\to B$ there is a compatibility
\[\xymatrix{
\ar[d]^{f^*}[B,E]\ar[r]^{\cong\ \ \ }&
[S,E\wedge B^\perp]\ar[d]^{(\id_E\wedge f^\perp)_*}\\
[A,E]\ar[r]^{\cong\ \ \ }&[S,E\wedge A^\perp]
}\]
which shows that the dual is unique up to equivalence when it exists, and that
$(A^\perp)^\perp\simeq A$.
\end{lem}
\begin{lem}
Let $A$ and $A^\perp$ be two finite spectra.
\begin{itemise}
\itm[(i)]Suppose $u:S\to A\wedge A^\perp$ is such that $u_E$ and $u^E$ are
isomorphisms for $E=\Sigma^k S$. Then $u$ is a duality.
\itm[(ii)]If $u$ is a duality, then for any $E,F$ the homomorphisms
\begin{alignat*}{2}
_Fu_E:[F\wedge A,E]\to[F,E\wedge A^\perp],&\qquad &
(F\wedge A\to E)\mapsto (F\to F\wedge A\wedge A^\perp\to E\wedge A^\perp)\\
^Fu^E:[A^\perp\wedge F,E]\to[F,A\wedge E],&\qquad &
(A^\perp\wedge F\to E)\mapsto (F\to A\wedge A^\perp\wedge F\to A\wedge E)
\end{alignat*}
are isomorphisms, so that internal hom objects exist in this setting.
\itm[(iii)]Given a dual for $A$ and $B$, a dual for $A\wedge B$ is
$B^\perp\!\wedge A^\perp$, via:
\[S\to A\wedge A^\perp\to A\wedge S\wedge A^\perp\to 
A\wedge B\wedge B^\perp\!\wedge A^\perp.\]
\end{itemise}
\end{lem}
\setcounter{thm}{6}
\begin{defn}
Spectra $A$ and $B$ are called \emph{$n$-dual} if $A$ and $\Sigma^{-n}B$ are
dual (a symmetric relation). Spaces are called \emph{stably $n$-dual} if their
suspension spectra are $n$-dual.
\end{defn}
\begin{exmp}
Let $X$ be a finite cellular subspace of $\R^n$, and let $U$ be a regular
neighbourhood (one that admits a deformation retract onto $X$). Then $X$ is
stably $n$-dual to $\R^n\setminus U$. This is Spanier-Whitehead duality, and the
map is explicitly constructed here.

Let a finite CW-complex $X$ be cellularly embedded in a sphere $S^n$ and let $U$
be a regular neighbourhood of $X$ therein. They construct explicity a duality
morphism showing that $X^+$ is stably $n$-dual to $\overline U/\partial U$.
\end{exmp}
\begin{cor}
Every finite CW-space admits an $n$-dual finite CW-space for $n\gg0$, and every
finite spectrum admits a finite dual.
\end{cor}
\begin{proof}
The first claim follows as every finite CW-space can be embedded in some large
sphere. The next follows as a finite spectrum $A$ has $A\simeq
\Sigma^{-m}\Sigma^\infty A_m$ for $m\gg0$, then finding some $n$-dual space to
$A_m$, we're done.
\end{proof}
\begin{prop}
The dual of a cofiber sequence is a cofiber sequence.
\end{prop}
\begin{statement*}
If $X$ and $Y$ are finite CW-complexes such that 
$\Sigma^\infty X\simeq\Sigma^\infty Y$ then $S^NX\simeq S^NY$ for $N\gg0$.
If $f,g:X\to Y$ are maps of finite CW-complexes and $\Sigma^\infty
f\simeq\Sigma^\infty g$, then $S^Nf\simeq S^Ng$ for $N\gg0$.
\end{statement*}
\begin{proof}
Fix homotopy inverse morphisms $\Sigma^\infty X\longleftrightarrow\Sigma^\infty
Y$. Consider the composite of these, a self map of $\Sigma^\infty X$ homotopic
to $\id_{\Sigma^{\infty}X}$. The homotopy is a \emph{map}
$\Sigma^r\Sigma^{\infty}S^rX\wedge I^+\to\Sigma^\infty X$ for some $r$, by
1.5(iii) and 1.5(NB). In particular, the homotopy to the identity is defined
eventually.

The second statement follows from a similar argument, or, according to the text,
from the Freudenthal suspension theorem.
\end{proof}
\begin{thm}[duals of finite complexes {\small(follows from above statement)}]
Fix $n$.
\begin{itemise}
\itm[(i)]Suppose $X$ is a finite CW-space with a finite CW $n$-dual $X'$. Then
the homotopy type of $S^NX'$ is determined by that of $X$ for $N\gg0$.
\itm[(ii)] Let $f:X\to Y$ be a map of finite CW-spaces with $n$-duals as above.
Then for some $N$ there is a map $f':S^NY'\to S^NX'$ such that
$\Sigma^{-N}\Sigma^\infty \Sigma^Nf$ and $\Sigma^{-N}\Sigma^\infty f'$ are dual
morphisms. Moreover, $f'$ is unique up to homotopy for $N\gg0$.
\itm[(iii)]If we perform the process in \textup{(ii)} on both maps in a cofiber
sequence $X\to Y\to Z$ of finite CW-spaces, then $S^NZ'\to S^NY'\to S^NX'$ is a
cofiber sequence for $n\gg0$.
\end{itemise}
\end{thm}
\begin{defn}
A \emph{ring spectrum} is a triple $(E,\mu,\imath)$ where $\mu:E\wedge E\to E$
and $\imath:S\to E$ satisfy the ring axioms up to homotopy. It may be or may not
be commutative. A \emph{module spectrum} over a ring spectrum is a pair $(F,m)$
satisfying the module axioms up to homotopy (with a unitarity condition).
\end{defn}
\setcounter{thm}{13}
\begin{condef}
Given a module $F$ over a ring $E$, and a morphism $a:S^d\to E$, the morphism
\[a_\#:S^d\wedge F\xrightarrow{\ a\wedge 1\ }
E\wedge F\xrightarrow{\ \ m\ \ } F\]
is called multiplication by $a$. This induces a map $F_n(X)\to F_{n+d}(X)$,
which is multiplication by $a\in\pi_*(E)$ on the $\pi_*(E)$-module $\pi_*(F)$.
See page \pageref{modulestructures}.
\end{condef}
\begin{prop}
Let $E\to E'$ be a ring morphism. If $\xymatrix{S^d\ar[r]^a&E\ar[r]&E'}$ is
inessential then so is the morphism $a_\#:S^d\wedge E'\to E'$.
\end{prop}
}   %%% End comment %%%
\subsection{\lparen Co\rparen homology theories and their connection with
spectra}
\comment{
Write $\CW$ for the category of CW-spaces (spaces homeomorphic to CW-complexes),
and $\CWf$ and $\CWfd$ for the full subcategories of finite and finite
dimensional CW-complexes. With a `2' superscript, we mean the category of pairs,
and $\scrK$ denotes any one of these categories. We have $R:\scrK^2\to\scrK^2$
defined by $R(X,A)=(A,\emptyset)$.
\begin{defn}
An unreduced homology theory on $\scrK^2$ is a family of functors
$h_n:\scrK^2\to\AG$ and natural transformations $\partial_n:h_n\to h_{n-1}\circ
R$:
\begin{itemise}
\itm Satisfying homotopy invariance;
\itm Giving a long exact sequence;
\itm Satisfying the collapse axiom (as $\scrK$ is a nice category of spaces):
the collapse map $c:(X,A)\to (X/A,\{*\})$ induces an isomorphism $h_n(X,A)\to
h_n(X/A,\{*\})$.
\end{itemise}
\end{defn}
\noindent Any such with a dimension axiom ($h_n(\textup{pt},\emptyset)=0$ for
$n\neq0$) is a classical homology theory.
\begin{prop}For any unreduced homology theory:
\begin{itemise}
\itm The inclusion $t:(X\cup CA,*)\xrightarrow{\ \simeq\ }(X\cup CA,CA)$ induces
an isomorphism $h_n(t)$.
%\itm If $i:A\subset X$ is a homotopy equivalence, then $h_*(X,A)=0$. In
particular, if $X$ is contractible, $h_*(X,*)=0$.
%\itm
\itm For every CW-triple $A\subset X\subset Y$, we have a long exact sequence:
\[\xymatrix{
\cdots\ar[r]&
h_{n+1}(Y,X)\ar[r]^d&h_{n}(X,A)\ar[r]&
h_{n}(Y,A)\ar[r]&h_{n}(Y,X)\ar[r]^{\ \ d}&\cdots
}\]
where $d$ is the composition $\xymatrix{h_{n+1}(Y,X)\ar[r]^{\ \ \partial_{i+1}}&
h_n(X,\emptyset)\ar[r]&h_n(X,A)}$.
\itm For every CW-triad $(X;A,B)$ \lparen i.e.\ $A\cup B=X$\rparen\ with $A\cap
B=C$ we have a Mayer-Vietoris sequence:
\[\xymatrix@C=1.3cm{
\ar@{.>}[r]&
h_{n}(C,\emptyset)\ar[r]_{(i_1)_*\oplus (i_2)_*\ \ \ \ \ }&
h_{n}(A,\emptyset)\oplus h_n(B,\emptyset)\ar[r]_{\qquad \ \ (j_1)_*-(j_2)_*}&
h_{n}(X,\emptyset)\ar[r]_{\Delta\ \ }&h_{n-1}(C,\emptyset)\ar@{.>}[r]&
}\]
defining $\Delta:h_n(X,\emptyset)\to h_n(X,A)\cong
h_n(X/A,*)=h_n(B/C,*)\cong h_n(B,C)\to h_{n-1}(C,\emptyset)$.
\end{itemise}
\end{prop}
\noindent The LES of a triple to $*\subset X\subset CX$ gives the
suspension isomorphism: $t_n:h_{n+1}S\to h_n$:
\[0\to h_n(CX,X)\to h_{n-1}(X,*)\to0,\text{ \ yet \ }h_{n}(CX,X)=h_{n}(SX,*).\]
So letting $\scrK^\bullet$ stand for the pointed category:
\setcounter{thm}{3}
\begin{condef}
Given an unreduced cohomology theory on $\scrK^2$, the functors $\widetilde h_n$
induced on $\scrK^\bullet$ taken with the natural equivalences $t_n:\widetilde
h_{n+1}S\to\widetilde h_n$ are the \emph{reduced cohomology theory}
corresponding to $h_n$. The following properties of the $\widetilde h_n$ are the
axioms for a reduced cohomology theory:
\begin{itemise}
\itm The functors $\widetilde h_n$ are pointed homotopy invariant.
\itm For any pointed pair, 
$\widetilde h_n(A,x_0)\to\widetilde h_n(X,x_0)\to\widetilde h_n(X/A,*)$
is exact.
\end{itemise}
\end{condef}
\setcounter{thm}{5}
\begin{prop}
Every reduced cohomology theory is the reduced cohomology theory corresponding
to an unreduced cohomology theory which is unique up to equivalence.
\end{prop}
\begin{prop} For any reduced cohomology theory:
\begin{itemise}
\itm Cofiber sequences map to exact sequences under each $\widetilde h_n$.
\itm For every pointed CW-pair $A\subset X$, we have a long exact sequence:
\[\xymatrix{
\ar@{.>}[r]&
\widetilde h_n(A)\ar[r]&\widetilde h_{n}(X)\ar[r]&\widetilde h_{n}(X/A)
\ar[r]^{\widetilde\partial_n}&\widetilde h_{n-1}(A)\ar@{.>}[r]&
}\]
defining: $\widetilde\partial_n:\xymatrix{\widetilde h_n(X/A)\ar[r]&
\widetilde h_n(X\cup CA)\ar[r]&\widetilde h_n(SA)\ar[r]&\widetilde h_{n-1}(A)}$.
\itm For every CW-triad $(X;A,B)$ \lparen i.e.\ $A\cup B=X$\rparen\ with $A\cap
B=C$ we have a Mayer-Vietoris sequence:
\[\xymatrix@C=1.3cm{
\ar@{.>}[r]&
\widetilde h_{n}(C)\ar[r]_{(i_1)_*\oplus (i_2)_*\ \ \ \ \ }&
\widetilde h_{n}(A)\oplus \widetilde h_n(B)\ar[r]_{\qquad \ \ (j_1)_*-(j_2)_*}&
\widetilde h_{n}(X)\ar[r]_{\Delta\ \ }&\widetilde h_{n-1}(C)\ar@{.>}[r]&
}\]
defining $\Delta:\widetilde h_n(X)\to \widetilde h_n(X/A)=\widetilde
h_n(B/C)\xrightarrow{\ \widetilde\partial_n\ }\widetilde h_{n-1}(C)$. 
\itm $\widetilde h_n(X\vee Y)=\widetilde h_n(X)\oplus\widetilde h_n(Y)$.
\end{itemise}
\end{prop}
\noindent Exactly the same story holds for (un)reduced cohomology theories. Now
we have some full subcategories of $\scrS$: $\Specfd$ (finite dimensional
spectra), $\Specs$ (suspension spectra), $\Specsfd$ (suspensions of finite
dimensional CW complexes), and $\Specf$ (finite spectra). Let $\scrL$ be $\scrS$
or one of these subcategories.
\setcounter{thm}{9}
\begin{defn}
A homology theory on $\scrL$ is a family $h_n:\scrL\to\AG$ of functors and a
family $\hat\fraks:h_n\to h_{n+1}\Sigma$ of natural equivalences. The $h_n$ must
satisfy a homotopy axiom, and must send cofiber sequences to exact sequences.
Cohomology theories are defined analogously.
\end{defn}
\begin{prop}
From these axioms we can derive that $h_n(X\vee Y)=h_n(X)\oplus h_n(Y)$, that
cofiber sequences give long exact sequences, and a Mayer-Vietoris sequence. The
same goes for cohomology.
\end{prop}
\setcounter{thm}{12}
\begin{const}
Given a \lparen co\rparen homology theory on $\scrS$, we obtain a reduced
\lparen co\rparen homology theory on $\scrK^\bullet$ by composing with
$\Sigma^\infty$.
\end{const}
\begin{lem}
Let $X(1)\to X(2)\to \cdots\to X(n)$ be a sequence of maps of spectra. Then
there exists a set $\Omega$ and finite subspectra $X(i)_\omega\subset X(i)$ for
all $\omega\in\Omega$ such that:
\begin{itemise}
\itm For each $i\leq n$, every finite subspectrum of $X(i)$ is contained in some
$X(i)_\omega$.
\itm For each $i<n$ and $\omega\in\Omega$, $X(i)_\omega$ maps into
$X(i+1)_\omega$ under the map $X(i)\to X(i+1)$.
\end{itemise}
\end{lem}
\begin{lem}
Given a strict cofiber sequence $X\to Y\to Z$, the family as above can be chosen
so that $X_\omega\to Y_\omega\to Z_\omega$ are strict cofiber sequences.
\end{lem}
\begin{defn}
Let $\scrL$ be one of the above categories of spectra. A \lparen co\rparen
homology theory on $\scrL$ is called \emph{additive} if whenever
$\{X_\lambda\}\subset\scrL$ is a family whose wedge is still inside $\scrL$:
\[\oplus_\lambda h_*(X_\lambda)\xrightarrow{\ \cong\ }h_*(\vee X_\lambda)
\text{ \ for homology, and \ }h^*(\vee X_\lambda)
\xrightarrow{\ \cong\ }\prod h^*(X_\lambda) \text{\  for cohomology}.\]
We make the same definitions for reduced \lparen co\rparen homology theories on
$\scrK^\bullet$. An unreduced \lparen co\rparen homology theory is additive when
the corresponding reduced theory is, which is to say that the isomorphisms occur
with disjoint union in place of wedge. (As these isomorphisms always hold for
finite wedges, additivity is no restriction on theories on
$\scrC_\textup{f}^\bullet$ or $\Specf$.)
\end{defn}
\begin{nonexample}
Set $\widetilde h_k(X):=\prod_{n=0}^\infty\widetilde H_n(X)
/\bigoplus_{n=0}^\infty\widetilde H_n(X)$, and apply to $\vee_{n=1}^\infty S^n$.
\end{nonexample}
\begin{prop}
Every reduced \lparen co\rparen homology theory $\widetilde h$ on
$\scrC^\bullet$, $\scrC_\textup{fd}^\bullet$ or $\scrC_\textup{f}^\bullet$ is
obtained from a \lparen co\rparen homology theory $h$ on $\scrS$, $\Specf$ or
$\Specfd$ via construction 3.13 \lparen composition with $\Sigma^\infty$\rparen.
$h$ is unique up to equivalence, and it additive iff $\widetilde h$ is.
\end{prop}
\begin{prop}\hfil
\begin{itemise}
\itm[(i)] Let $\phi$ be a morphism of reduced \lparen co\rparen homology
theories on $\scrC_\textup{f}^\bullet$. If $\phi$ is an isomorphism on $S^0$
then it is an equivalence.
\itm[(ii)] Let $\phi$ be a morphism of reduced \textbf{additive} \lparen
co\rparen homology theories on any of the above categories $\scrK^\bullet$ or
$\scrL$ of spaces or spectra. If $\phi$ is an isomorphism on $S^0$ then it is an
equivalence.
\end{itemise}
\end{prop}
\begin{prop}Let $\{X_\lambda\}$ be the set of all finite subspectra of $X$.
\begin{itemise}
\itm[(i)] Let $h_*$ be a homology theory on $\Specf$. Set $k_*(X):=\varinjlim
h_*(X\lambda)$ for any $X\in\scrS$. Then $k_*$ is an additive homology theory on
$\scrS$.
\itm[(ii)] If $h_*$ is already defined on $\scrS$, the natural map $h_*\to k_*$
is an equivalence.
\end{itemise}
Thus there is no difference between homology theories on $\Specf$ and additive
homology theories on $\scrS$. This does \textbf{not} hold for cohomology
theories.
\end{prop}
\begin{condef}[dual theories]
Given a cohomology theory $h^*$ on $\Specf$, we can form the \emph{dual homology
theory} thereupon via $h_i(X):=h^{-i}(X^\perp)$. (The minus sign appears so that
the suspension isomorphisms in $h_*$ are as required. This is well defined as
dual spectra are unique, and exist by 2.4 and 2.9.) Thus we have a bijection
(duality) between homology and cohomology theories.
\end{condef}
\begin{const}%[(co)homology theories from spectra]
Any spectrum $E$ yields a (co)homology theory on $\scrS$ via:
\[E_n(X):=\pi_n(E\wedge X)\text{\qquad and\qquad}E^n(X):=[X,\Sigma^nE].\]
\end{const}
\begin{prop}
$E^*$ is dual to $E_*$ for any $E$.
\end{prop}
\begin{exmp}[Eilenberg-MacLane spectra]
Let $\pi$ be an abelian group, and let $H\pi$ be a spectrum such that
$\pi_*(H\pi)$ is $\pi$ concentrated in degree zero. Letting $(H\pi)_n=K(\pi,n)$
for each $n\geq0$, and using the homotopy equivalences $K(\pi,n)\to\Omega
K(\pi,n+1)$ gives an $\Omega$-prespectrum (defining $(H\pi)_{r}$ to be
$\Omega^{-r}K(\pi,0)$ for $r\leq0$). Replace it with an $\Omega$-spectrum to get
$H\pi$. (As $\pi_i(H\pi)=\varinjlim \pi_{i+N}H\pi_N$, this works). $H\pi$ is
unique up to equivalence, as we may always take it to be an $\Omega$-spectrum
$F$, and $\pi_i(F)=\pi_{i+j}(F_j)$ for any $j$. Thus any two...COME BACK LATER, having thought about the next few propositions.

By looking at coefficients, we see that $H\pi$ represents ordinary (co)homology
with coefficients in $\pi$. We also obtain:
$H_i(X;\pi)=\varinjlim\pi_{i+N}(K(\pi,N)\wedge X^+)$.
\end{exmp}
\begin{prop}
If $E$ is an $\Omega$-spectrum, then for every space $X$,
$\widetilde E^i(X)\simeq[X,E_i]^\bullet$.
\end{prop}
\begin{cor}
For every spectrum $E$, the functors $\widetilde E^i:\HC^\bullet\to\AG$ are
representable.
\end{cor}
\begin{propdef}
Given a spectrum $E$, let $\Omega^\infty E$ denote a representing space for
$\widetilde E^0$, or equivalently the \ZEROTH space of an $\Omega$ replacement.
This is called the \emph{infinite delooping}. By Yoneda's lemma, it is unique up
to equivalence, and functorial once an infinite delooping is chosen for each
spectrum.
\end{propdef}
\setcounter{thm}{28}
\begin{cor}
Let $F$ be an $\Omega$-spectrum equivalent to $E$. Then $\Omega^\infty E\simeq
F_0$. Moreover, the functors $\Sigma^\infty$ and $\Omega^\infty$ are adjoint.
\end{cor}
\noindent On homotopy groups: $\pi_k(E)=\pi_K(\Omega^\infty E)$. However:
\[i_*:\pi_k(X)\to\pi_k(\Omega^\infty\Sigma^\infty X)=
\pi_k(\Sigma^\infty X)=\pi^s_k(X)\text{ \ is stabilisation.}\]
Moreover, if $X$ is an infinite loop space, then by standard adjoint functor
formalism, the map $X\to\Omega^\infty\Sigma^\infty X$ has a homotopy left
inverse, so that
\begin{prop}
If $X$ is an infinite loop space, $\pi_k(X)$ is a direct summand of
$\pi_k^s(X)$.
\end{prop}
Given a morphism $\alpha:S\to E$ we obtain a \emph{Hurewicz homomorphism} with
respect to $\alpha$:
\[h_\alpha:\pi_*(X)\to E_*(X),\ \ \ f\mapsto \alpha\wedge f=
E_*(f)(\hat\fraks^d[\alpha]).\]
Given a ring spectrum $E=(E,\mu,\imath)$ write $h$ for $h_\imath$. We have
a cohomology pairing:
\[\mu^{X,Y}:E^m(X)\otimes E^n(Y)\to E^{m+n}(X\wedge Y),\ \ 
[f]\otimes[g]\mapsto \left[\Sigma^{-m}X\wedge\Sigma^{-n}Y\xrightarrow{f\wedge
g}E\wedge E\xrightarrow{\mu}E\right].\]
Similarly, we have a homoology pairing 
$\mu_{X,Y}:E_m(X)\otimes E_n(Y)\to E_{m+n}(X\wedge Y)$:
\[[f]\otimes[g]\mapsto\left[\Sigma^mS\wedge\Sigma^nS\xrightarrow{f\wedge
g}(E\wedge X)\wedge(E\wedge X)\to(E\wedge E)\wedge(X\wedge
Y)\xrightarrow{\mu\wedge\id}E\wedge X\wedge Y\right].\]
These pairings are associative and compatible with suspension. Moreover, if $E$
is a commutative ring spectrum, then $\mu^{X,Y}$ and $\mu_{X,Y}$ are both
commutative, in the sense that $\tau_*\mu(a\otimes b)=(-1)^{|a||b|}\mu(b\otimes
a)$.

There is a pairing 
$\mu^X_{\bullet,Y}:E^m(X)\otimes E_n(X\wedge Y)\to E_{n-m}(Y)$:
\[[f]\otimes[g]\mapsto\left[\Sigma^nS\xrightarrow{g}(E\wedge X\wedge
Y)\xrightarrow{\id\wedge f\wedge\id}(E\wedge \Sigma^mE\wedge
Y)\xrightarrow{\Sigma^m\mu\wedge\id}\Sigma^mE\wedge Y\right].\]
Similarly, there is a pairing 
$\mu_X^{\bullet,Y}:E^m(X\wedge Y)\otimes E_n(X)\to E^{m-n}(Y)$:
\[[f]\otimes[g]\mapsto\left[\Sigma^nY\xrightarrow{g\wedge\id_Y}(X\wedge E\wedge
Y)\to(X\wedge Y\wedge E)\xrightarrow{f\wedge\id}(\Sigma^mE\wedge E
)\xrightarrow{\Sigma^m\mu}\Sigma^mE\right].\]
Finally, putting $Y=S$ in either of these, we obtain the \emph{Kronecker
pairing}:
\[\langle\DASH,\DASH\rangle:E^m(X)\otimes E_n(X)\to \pi_{n-m}(E).\]
This all gives extra structure. For example, $E_*(Y)$ is a graded left
$\pi_*(E)$-module, and $E^*(Y)$ is a graded left $E^*(S)$-module. If $X$ is a
ring spectrum with multiplication $\nu$ then there is a pairing: $E_*(X)\otimes
E_*(X)\xrightarrow{\mu_{X,X}}E_*(X\wedge X)\xrightarrow{\ \nu_* \ }E_*(X)$,
making $E_*(X)$ a ring. Thus $E_*(E)$ is a ring.

\refstepcounter{dummyforrefstepcounter}\label{modulestructures}
In fact as long as $(F,m)$ is a module spectrum over $E$, we have pairings:
\begin{alignat*}{2}
m^{X,Y}&:\,\,&E^m(X)\otimes F^n(Y)&\to F^{m+n}(X\wedge Y)\\
m_{X,Y}&:&E_m(X)\otimes F_n(Y)&\to F_{m+n}(X\wedge Y)\\
m^X_{\bullet,Y}&:&E^m(X)\otimes F_n(X\wedge Y)&\to F_{n-m}(Y)\\
m_X^{\bullet,Y}&:&E^m(X\wedge Y)\otimes F_n(X)&\to F^{m-n}(Y)\\
'm_X^{\bullet,Y}&:&E^m(X\wedge Y)\otimes F_n(X)&\to F^{m-n}(Y)\\
\langle\DASH,\DASH\rangle&:&E^m(X)\otimes F_n(X)&\to \pi_{n-m}(F)\\
\langle\DASH,\DASH\rangle&:&F^m(X)\otimes E_n(X)&\to \pi_{n-m}(F)
\end{alignat*}
It is easy to replace $X$ and $Y$ by spaces, to get formulae in reduced
(co)homology. Adapting them for unreduced cohomology, we obtain a \emph{cup
product}:
\[\cup:E^m(X,A)\otimes E^n(X,B)\xrightarrow{\mu}E^{m+n}(X\times X,X\times B\cup
A\times X)\xrightarrow{d^*}E^{m+n}(X,A\cup B),\]
and when $A=B$ this gives a graded ring structure, which is graded commutative
if $E$ is commutative. Moreover we have a \emph{cap product}:
\[\cap:E^m(X,A)\otimes E_n(X,A\cup B)\xrightarrow{1\otimes d_*}E^m(X,A)\otimes
E_n(X\times X,X\times B\cup A\times X)\xrightarrow{\mu}E_{n-m}(X,B).\]

Now let $Y$ be a module spectrum over the ring spectrum $X$. The pairing:
\[E_*(X)\otimes F_*(Y)\xrightarrow{m_{X,Y}}F_*(X\wedge Y)\to F_*(Y)\]
makes $F_*(Y)$ a graded $E_*(X)$-module. Similarly, $F_*(E)$ is an
$E_*(E)$-module.

A morphism $a:S^d\to X$ induces $a_\#:\Sigma^dY\to Y$, which induces
$a_*:F_{i-d}(Y)\to F_i(Y)$.
\setcounter{thm}{43}
\begin{lem}
$a_*$ is multiplication by $h(a)\in E_*(X)$.
\end{lem}
Let $X$ be a spectrum, $E$ a ring spectrum and $F$ a module spectrum over $E$.
Then the Kronecker pairing gives an evaluation homomorphism:
\[\textup{ev}:F^n(X)\to\Hom^n_{\pi_*(E)}(E_*(X),\pi_*(F))\text{\ \
(homomorphisms reducing degree by $n$)}.\]
\begin{thm}
Suppose that for some $N$, $\pi_i(X)=0$ for $i<N$. Then if the Atiyah-Hirzebruch
spectral sequence for $E_*(X)$ collapses ($d_r=0$ for $r\geq2$), and if the
$\pi_*(E)$-module $E^2_{**}$ is free, then the above evaluation homomorphism is
an isomorphism.
\end{thm}
\begin{thm}
Let $X$ be a ring spectrum, and let $E$ be a commutative ring spectrum. Consider
the evaluation
\[\textup{ev}:E^n(X)\to\Hom^n_{\pi_*(E)}(E_*(X),\pi_*(E)),\]
and suppose that all the conditions of 3.54 hold for $X$ and $X\wedge X$. Then a
morphism $f:X\to E$ is a ring morphism iff the homomorphism $\textup{ev}(f)$ is
a homomorphism of $\pi_*(E)$-algebras.
\end{thm}
\begin{condef}
If $E$ is any spectrum, $E^*(E)$ is a ring under composition of morphisms $E\to
\Sigma^?E$. Moreover, $E^*(X)$ is naturally an $E^*(E)$-module via
postcomposition. By Yoneda's lemma, $E^*(E)$ is exactly the ring of
$E$-operations.
\end{condef}
}   %%% End comment %%%
\subsection{Homotopy properties of spectra}
\comment{
\setcounter{thm}{3}
\begin{defn}
A spectrum or space is $n$-\emph{connected} is $\pi_i(E)=0$ for $i\leq n$. A
spectrum is called \emph{connected} if it is $(-1)$-connected. A spectrum is
\emph{bounded below} if it is $n$-connected for some $n\in\Z$. A morphism is
called $n$-connected if its cone is $(n+1)$-connected or an
$n$-\emph{equivalence} (i.e.\ it is an isomorphism on $\pi_{\leq n}$ and an
epimorphism on $\pi_{n+1}$).
\end{defn}
\setcounter{thm}{0}
\begin{lem}Let $E,X$ be spectra, and let $Y$ be a pointed CW-space.
\begin{itemise}
\itm[(i)] Suppose $\widetilde E^k(X_k)$ and $\widetilde E^{k-1}(X_k)$ both
vanish for all $k$. Then $E^0(X)=0$.
\itm[(ii)] Suppose $E$ is $(n+1)$-connected, and $X^{(n)}=X$. Then $E^0(Y)=0$.
\itm[(iii)] Suppose $\widetilde E^0(Y^{(r)})$ and $\widetilde E^{-1}(Y^{(r)})$
both vanish for all $k$. Then $\widetilde E^0(Y)=0$.
\itm[(iv)] Suppose $\pi_j(E)=0$ for $j\geq n$, and $X$ is $n$-connected. Then
$E^0(X)=0$.
\end{itemise}
\end{lem}
\begin{lem}
A spectrum is equivalent to a spectrum $F$ with $F^{(n)}=*$ iff it is $n$
connected.
\end{lem}
\begin{prop}
Let $h_*$ be an additive homology theory on $\scrS$ with $h_i(S)=0$ for $i\leq
m$. Let $X\in\scrS$ be $n$-connected. Then $h_i(X)=0$ for $i\leq m+n+1$.
\end{prop}
\setcounter{thm}{4}
\begin{prop}\hfil
\begin{itemise}
\itm[(i)] If $E$ is $m$-connected and $F$ is $n$ connected, $E\wedge F$ is
$m+n+1$-connected.
\itm[(ii)] If $E$ is $m$-connected and $\phi:F\to G$ is an $n$-equivalence, than
$\id_E\wedge \phi$ is an $(m+n+1)$-equivalence.
\itm[(iii)] Suppose that $f:E\to F$ is a \textbf{map} of spectra such that
$f_n:E_n\to F_n$ is an $(n+k)$-equivalence for all sufficiently large $n$. Then
$f$ is a $k$-equivalence.
\itm[(iv)] The inclusion $E^{(k)}\subset E$ is always a $k$-equivalence.
\end{itemise}
\end{prop}
\begin{cor}[Hurewicz theorem]
Let $\alpha:S\to E$ be a 0-equivalence. If $X$ is $(n-1)$-connected then
$E_i(X)=0$ for $i<n$, and $\alpha_*:\pi_k(X)\to E_k(X)$ is an isomorphism for
$k=n$ and an epimorphism for $k=n+1$.
\end{cor}
\begin{cor}The Hurewicz theorem holds for the usual case, $E=H\Z$, and:
\begin{itemise}
\itm[(ii)] If $X$ is \textbf{bounded below} and such that $H_i(X)=0$ for $i\leq
n$, then $X$ is $n$-connected.
\itm[(iii)] An $H_*$-isomorphism between spectra which are \textbf{bounded
below} is an equivalence.
\end{itemise}
\end{cor}
\begin{rmk}
The boundedness is essential: if $K(n)$ is a Morava K-theory, $H_*(K(n))=0$.
\end{rmk}
\begin{thm}[the Universal Coefficient Theorem]
There are exact sequences:
\[0\to\Ext(H_{n-1}(E),G)\to H^n(E;G)\to\Hom(H_n(E),G)\to0\text{ \ and \
}H_0(H(A),B)=A\otimes B,\]
\[0\to H_n(E)\otimes G\to H_n(E;g)\to\Tor(H_{n-1}(E),G)\to0\text{ \ and \
}H^0(H(A),B)=\Hom(A,B).\]
\end{thm}
\begin{prop}
For every ring $R$, the spectrum $HR$ admits a ring structure so that the map
$\mu^{\textup{pt$,$pt}}:H^0(\textup{pt})\otimes H^0(\textup{pt})\to
H^0(\textup{pt})$ coincides with multiplication $R\otimes R\to R$. \lparen Here
\textup{pt} is the one point space, but could be replaced with the sphere
spectrum $S$ if desired\rparen.
\end{prop}
\noindent Thus, for spectra $E,F$, there a K\"unneth maps
$\mu_{E,F}:H_*(E;R)\otimes H_*(F;R)\to H_*(E\wedge F;R)$ and
$\mu^{E,F}:H^*(E;R)\otimes H^*(F;R)\to H^*(E\wedge F;R)$.
\begin{thm}[K\"unneth]
If $R=k$ is a field:
\begin{itemise}
\itm $\mu_{E,F}$ is an isomorphism.
\itm $\mu^{E,F}$ is an isomorphism as long as $E$ is bounded below and $F$ has
finite type.
\end{itemise}
\end{thm}
%\[\mu_{E,F}:H_*(E;R)\otimes H_*(F;R)\to H_*(E\wedge F;R)
%\text{ and }\mu_{E,F}:H_*(E;R)\otimes H_*(F;R)\to H_*(E\wedge F;R)\]
\begin{defn}
A \emph{Postnikov tower} of a spectrum $E$ is a homotopy commutative diagram:
\[\xymatrix{
\cdots\ar@{=}[r]&
E\ar[d]^{\tau_{n+1}}\ar@{=}[r]&
E\ar[d]^{\tau_{n}}\ar@{=}[r]&
E\ar[d]^{\tau_{n-1}}\ar@{=}[r]&\cdots\\
\cdots\ar[r]&
E_{(n+1)}\ar[r]^{\rho_{n+1}}&
E_{(n)}\ar[r]^{\rho_{n}}&
E_{(n-1)}\ar[r]&\cdots
}\]
where $\pi_i(E_{(n)})=0$ for $i>n$, and $\pi_{i}(\tau_n)$ is an isomorphism for
$i\leq n$. $E_{(n)}$ is called the $n$-\emph{coskeleton} or the \nTH
\emph{Postnikov stage}. It is unique up to equivalence.
\end{defn}
\begin{thm}Every spectrum has a Postnikov tower.
\end{thm}
\begin{defn}
A morphism $q=q^E_n:F\to E$ is called an $(n-1)$-\emph{connective covering} if
$F$ is $(n-1)$-connected, and $\pi_{\geq n}(q)$ is isomorphic. Any such $F$ is
called an $(n-1)$-\emph{killing spectrum}, denoted $E|n$. It is unique up to
equivalence.
\end{defn}
\begin{thm}
Every spectrum has $n$-connective coverings.
\end{thm}
\begin{thm}
The $(n-1)$-connected covering $F|n\to F$ is homotopy universal as a morphism
$D\to F$ where $D$ is $(n-1)$-connected.
\end{thm}
\begin{thm}
The \nTH Postnikov stage $F\to F_{(n)}$ is homotopy universal as a morphism
$F\to G$ where $G$ has $\pi_{>n}(G)=0$.
\end{thm}
\begin{cor}
A morphism of spectra induces a morphism of \lparen any chosen\rparen\ Postnikov
towers.
\end{cor}
\begin{defn}
There is a cofiber sequence $E_{(n)}\xrightarrow{p_n}
E_{(n-1)}\xrightarrow{\kappa_n}\Sigma^{n+1}H(\pi_n(E))$, and the element
$\kappa_n\in H^{n+1}(E_{(n-1)};\pi_n(E))$ is called the \emph{\nTH Postnikov
invariant} of $E$. Note that $\kappa_n$ is only truly defined up to the action
of self-equivalences of $E_{(n-1)}$ and $\Sigma^{n+1}H(\pi_n(E))$.
\end{defn}
\begin{prop}
The Postnikov invariant $\kappa_n$ is trivial iff $p_n$ admits a homotopy right
inverse \lparen i.e.\ a homotopy section\rparen\ $s:E_{(n-1)}\to E_{(n)}$ such
that $p_ns=\id_{E_{(n-1)}}$.
\end{prop}
\begin{defn}
A \emph{Serre class} is a family $\calC$ of abelian groups closed under taking
subgroups, quotient groups and extensions. A Serre class is called \emph{stable}
if for $A\in\calC$, $H_i(H(A))\in\calC$ for all $i$.
\end{defn}
\begin{prop}
Let $\calC$ be a Serre class such that
\begin{itemise}
\itm[(i)]If $A,B\in\calC$ then $A\otimes B\in\calC$ and $\Tor(A,B)\in\calC$;
\itm[(ii)]If $A\in\calC$ then $H_i(K(A,1))\in\calC$ for all $i>0$.
\end{itemise}
Then $\calC$ is a stable Serre class.
\end{prop}
\begin{prop}
The following are stable Serre classes: finite abelian groups; finitely
generated abelian groups; abelian groups of $p$-primary exponent for any $p$
\lparen i.e.\ for some $k$, $p^kA=0$\rparen; finite abelian groups of
$p$-primary exponent.
\end{prop}
\begin{thm}[Hurewicz mod $\calC$]
Let $\calC$ be a stable Serre class. Let $E$
be a spectrum \textbf{bounded below} such that $\pi_i(E)\in\calC$ for $i<n$.
Then $H_i(E)\in\calC$ for $i<n$, and the Hurewicz homomorphism $h:\pi_n(E)\to
H_n(E)$ is a $\calC$-isomorphism. Thus $\pi_i(E)\in \calC$ for all $i$ iff
$H_i(E)\in \calC$ for all $i$ as long as $E$ is \textbf{bounded below}.
\end{thm}
\begin{prop}\hfil
\begin{itemise}
\itm[(i)]If $\calC$ is any Serre class and $\pi_i(E)\in\calC$ for all $i$, then
$E_i(X)\in\calC$ and $E^i(X)\in\calC$ for every finite spectrum $X$ and all $i$.
\itm[(iii)]Let $R$ be a commutative Noetherian ring, and let $E$ be a spectrum
such that $E_i(X)$ and $E^i(X)$ are $R$-modules naturally in $X$. Suppose that
each group $\pi_i(E)$ is a finitely generated $R$-module. Then the $E_i(X)$ and
$E^i(X)$ are finitely generated $R$-modules for finite spectra $X$.
\end{itemise}
\end{prop}
\begin{prop}\hfil
\begin{itemise}
\itm[(i)]If $X$ has finite type then all the $\pi_i(X)$ are finitely generated.
\itm[(ii)]If $X$ is bounded below and all the $\pi_i(X)$ are finitely generated,
then $X$ is equivalenct to a spectrum of finite type.
\end{itemise}
\end{prop}
\noindent\textbf{Section not completed.} More material follows on equipping
connective coverings and Postnikov towers of ring spectra with ring structures.
}   %%% End comment %%%
\subsection{Localization}
\comment{
If $p$ is a prime let $\Z[p]=\Z$ be the subring of $\Q$ whose irreducible
fractions have denominator coprime to $p$. Let $\Z[1/p]$ be the subring of $\Q$
whose irreducible fractions have denominator a power of $p$. In general, let
$\Lambda$ be any subring of $\Q$, and write $\Lambda$ for its additive group as
well. Let $\pi$ be any abelian group.
\begin{defn}
The map $l:\pi\to\pi\otimes \Lambda$ is called the $\Lambda$-\emph{localization}
of $\pi$. $\pi$ is called $\Lambda$-\emph{local} if it is an isomorphism. Any
map $\pi\to\tau$ is said to $\Lambda$-\emph{localize} $\pi$ if $\tau\cong
\pi\otimes\Lambda$ compatibly with $l$.
\end{defn}
Let $\imath:S\to M(\Lambda)$ be the morphism given by the unit
$1\in\Lambda=\pi_0(M(\Lambda))$, where $M(\Lambda)$ is the Moore spectrum. Set
$E_\Lambda:=E\wedge M(\Lambda)$ for any spectrum $E$.
\begin{defn}
The morphism $j:E\wedge S\xrightarrow{\id\wedge \imath}E\wedge
M(\Lambda)=E_\Lambda$ is called the $\Lambda$-\emph{localization} of $E$.
\end{defn}
\begin{prop}
$\Lambda$-localization is a functor which preserves cofiber sequences.
\end{prop}
\begin{thm}
For all $X,E$ there is an isomorphism $t:(E\Lambda)_*(X)\simeq E_*(X)\otimes
\Lambda$, and it can be chosen so that the following commutes, so that $j_*$
$\Lambda$-localizes $E_*(X)$:
\[\xymatrix{
E_*(X)\ar[r]_{j_*}\ar[d]^l&(E_\Lambda)_*(X)\ar[d]_t^\cong\\
E_*(X)\otimes\Lambda&E_*(X)\otimes\Lambda\ar@{=}[l]
}\]
Analogously, $j_*:E^*(X)\to E_\Lambda^*(X)$ $\Lambda$-localizes $E^*(X)$.
We fix such $t$ from now on without mention.
\end{thm}
\begin{cor}
There are natural isomorphisms $\pi_i(E_\Lambda)\cong \pi_i(E)\otimes\Lambda$
and $H_i(E_\Lambda)\cong H_i(E)\otimes\Lambda$ such that the maps
$\pi_i(E)\xrightarrow{j_*}\pi_i(E_\Lambda)\cong\pi_i(E)\otimes\Lambda$ and
$H_i(E)\xrightarrow{j_*}H_i(E_\Lambda)\cong H_i(E)\otimes\Lambda$ are of the
form $x\mapsto x\otimes1$. Thus a $\Lambda$-local spectrum has $\Lambda$-local
homotopy and homology.
\end{cor}
\begin{lem}Let $\tau$ be a flat abelian group \lparen e.g.\ an additive subgroup
of $\Q$\rparen. Then:
\begin{itemise}
\itm[(i)]$H\tau\wedge M(\pi)\simeq H(\tau\otimes\pi)\simeq H\pi\wedge M(\tau)$
for any $\pi\in\AG$. In particular, $H\Z\wedge M(\pi)\simeq H\pi$.
\itm[(ii)] $M(\tau)\wedge M(\pi)\simeq M(\tau\otimes\pi)$. In particular,
$M(\Lambda)\wedge M(\Lambda)\simeq M(\Lambda)$.
\itm[(iii)] If $\pi$ is a $\Lambda$-local group, $H\pi$ and $M(\pi)$ are
$\Lambda$-local spectra, in that the $\Lambda$-localizations $j^H:H\pi\to
(H\pi)_\Lambda$ and $j^M:M(\pi)\to M(\pi)_\Lambda$ are equivalences.
\end{itemise}
\end{lem}
\begin{lem}
Let $Cj$ be the cone of the localization $j:E\to E_\Lambda$. Then
$H^*(Cj;\pi)=0$ for every $\Lambda$-local abelian group $\pi$.
\end{lem}
\begin{thm}Let $E$ be an arbitrary spectrum, and let $F$ be a $\Lambda$-local
spectrum.
\begin{itemise}
\itm[(i)]$E_\Lambda$ is $\Lambda$-local, as localization
$E_\Lambda\to(E_\Lambda)_\Lambda$ is an equivalence
\itm[(ii)] For every morphism $f:E\to F$ there is a morphism $g:E_\lambda\to F$
with $gj_\Lambda^E=f$. Thus $j^*:[E_\Lambda,F]\to[E,F]$ is epic, and $E_\Lambda$
is `versal' for maps from $E$ to a $\Lambda$-local spectrum.
\itm[(iii)] If in (ii) either $E$ or $F$ is bounded below, $j^*$ is an
isomorphism. In particular, if $E$ is bounded below, $E_\Lambda$ is universal
for maps from $E$ to a $\Lambda$-local spectrum.
\end{itemise}
\end{thm}
\begin{cor}Let $E$ be any spectrum, and let $F$ be a $\Lambda$-local spectrum.
Fix $f:E\to F$.
\begin{itemise}
\itm[(i)]$f$ $\Lambda$-localizes $E$ iff $\pi_*(f)$ $\Lambda$-localizes homotopy
groups.
\itm[(ii)]If $E$ and $F$ are \textbf{bounded below}, $f$ $\Lambda$-localizes $E$
iff $H_*(f)$ $\Lambda$-localizes homology groups.
\end{itemise}
\end{cor}
\begin{cor}
If either all the $\pi_i(E)$ are $\Lambda$-local, or $E$ is bounded below and
all the $H_i(E)$ are $\Lambda$-local, then $E$ is a $\Lambda$-local spectrum. In
particular, if $F$ is $\Lambda$-local, so are all the $F_{(n)}$ and $F|n$.
\end{cor}
\begin{prop}
The Postnikov tower of $E_\Lambda$ is the localization of the Postnikov tower of
$E$.
\end{prop}
\begin{thmdef}
For every simple space $X$ there is a simple space $X_\Lambda$ and $j:X\to
X_\Lambda$ such that $\pi_i(X)\xrightarrow{j_*}\pi_i(X_\Lambda)\cong
\pi_i(X)\otimes\Lambda$ and $H_i(X)\xrightarrow{j_*}H_i(X_\Lambda)\cong
H_i(X)\otimes\Lambda$ both have the form $x\mapsto x\otimes1$, so that $j$
localizes homotopy and homology groups, and is called a \emph{localization} of
$X$.
\end{thmdef}
\begin{thmdef}
For every map $f:X\to Y$ of simple spaces, TFAE:
\begin{itemise}
\itm[(i)]$f$ $\Lambda$-localizes homotopy groups.
\itm[(ii)]$f$ $\Lambda$-localizes homology groups.
\itm[(iii)]For every $\Lambda$-local space $Z$, $f^*:[Y,Z]\to[X,Z]$ is a
bijection.
\end{itemise}
If these equivalent conditions hold there is a homotopy equivalence
$h:X_\Lambda\to Y$ with $f=hj$, so we say that $f$ \emph{localizes} $X$.
\end{thmdef}
If $f:X\to Y$ localizes $X$ then $\Sigma^\infty f$ localizes $\Sigma^\infty X$,
by 5.9. Thus $j^*:E^*(X_\Lambda)\to E^*(X)$ is an isomorphism for any
$\Lambda$-local spectrum $E$. Similarly, if $\phi:E\to F$ localizes the spectrum
$E$, $\Omega^\infty\phi$ localizes $\Omega^\infty E$.
\begin{lem}
There is an equivalence $(E\wedge F)_\Lambda\to E_\Lambda\wedge F_\Lambda$
compatible with localisation maps.
\end{lem}
\begin{thm} Let $E$ be a ring spectrum and $F$ be a module spectrum over $E$.
\begin{itemise}
\itm $E_\Lambda$ admits a ring structure such that $E\to E_\Lambda$ is a ring
morphism. This structure is unique up to ring equivalence.
\itm If $E$ and $F$ are both \textbf{bounded below}, there is a pairing
$\overline m:E_\Lambda\wedge F_\Lambda\to F_\Lambda$ turning $F_\Lambda$ into an
$E_\Lambda$-module such that $\overline m(j^E\wedge j^F)=j^Fm$, and this
$\overline m$ is unique up to homotopy.
\end{itemise}
\end{thm}
We write $X[0]$, $X[p]$ and $X[1/p]$ for $X_\Lambda$ when $\Lambda$ is $\Z[0]$,
$\Z[p]$ and $\Z[1/p]$ respectively.
\begin{defn}
A spectrum $E$ has \emph{finite $\Lambda$-type} if it is bounded below and every
group $\pi_i(E)$ is a finitely generated $\Lambda$-module.
\end{defn}
\begin{rmk}[on $\Lambda$-modules]\hfil
\begin{itemise}
\itm A spectrum has finite $\Z$-type iff it is equivalent to one of finite type
(by 4.26). A spectrum of finite $\Lambda$-type is $\Lambda$-local.
\itm A submodule of a finitely generated $\Lambda$-module is finitely generated
(as this property characterises noetherian rings), so a $\Lambda$-module is
finitely generated iff it is finitely presented. As $\Lambda$ is a PID, every
finitely presented $\Lambda$-module splits into a direct sum of cyclic
submodules. Thus, every finitely generated $\Z[p]$-module splits into copies of
$\Z[p]$ and $\Z/p^k$.
\end{itemise}
\end{rmk}
\begin{lem}Let $p$ be a prime, and let $E,F$ be spectra of finite $\Z[p]$-type.
\begin{itemise}
\itm[(i)] If $H^i(E;\Z/p)=0$ for $i\leq n$, then $E$ is $n$-connected. Thus $E$
is contractible iff $H^i(E;\Z_p)=0$ for all $i$.
\itm[(ii)] $f:E\to F$ is an equivalence iff it is a $\Z/p$-cohomology
isomorphism.
\end{itemise}
\end{lem}
\begin{prop}Let $E,F$ be spectra such that all the $\pi_i(E),\pi_i(F)$ are
finitely generated.
\begin{itemise}
\itm[(i)] If $E[p]$ is contractible for all primes $p$ then $E$ is contractible.
\itm[(ii)] $f:E\to F$ is such that $f[p]$ is an equivalence for all $p$, then
$f$ is an equivalence.
\end{itemise}
\begin{prop}
Suppose that $E$ is a ring spectrum such that $1\in\pi_0(E)$ has order $p$ a
prime. Then $E$ is a $\Z[p]$-local spectrum.
\end{prop}
\begin{proof}
As the $\pi_i(E)$ are $\pi_0(E)$-modules, they are modules for
$\Z/p\subseteq\pi_0(E)$. Apply 5.10.
\end{proof}
\end{prop}
}   %%% End comment %%%
\subsection{Algebras, coalgebras, and Hopf algebras
\texorpdfstring{$\bigotimes$}{}}
\subsection{Graded Eilenberg-MacLane spectra \texorpdfstring{$\bigotimes$}{}}
\section{Phantoms}
\subsection{Phantoms and the inverse limit functor}
\comment{
Let $\scrK$ be a family $\{X_\alpha\}$ of subobjects of a space or spectrum $X$.
Maps $X\to Y$ are $\scrK$-homotopic if their restrictions to each $X_\alpha$ are
homotopic. Elements of $E^*(X)$ are $\scrK$-equivalent if their restrictions to
each $X_\alpha$ are equal. The classes of $\scrK$-homotopic maps (or elements)
for a pointed set $[X,Y]_\scrK$ (or a group $E^*_\scrK(X)$), and there are
quotient maps $\sigma:[X,Y]\to[X,Y]_\scrK$ and $\sigma:E^*(X)\to E^*_\scrK(X)$.
\begin{defn}
A map $f:X\to Y$ or class $f\in E^*(X)$ is a \emph{phantom} if $f$ is nontrivial
yet $\sigma(f)$ is trivial.
\end{defn}
\begin{defn}
Given a CW-complex (spectrum) $X$, let $\scrK$ be the family of all skeleta of
$X$, and let $\scrK'$ be the family of all finite subcomplexes (subspectra) of
$X$. Then a $\scrK$-phantom is just called a \emph{phantom}, and a
$\scrK'$-phantom is called a \emph{weak phantom}.
\end{defn}
\begin{prop}
Equivalences of spaces (spectra) map phantoms to phantoms and weak phantoms to
weak phantoms.
\end{prop}
\begin{proof}
Exercise using II.3.14.
\end{proof}
\begin{prop}
Let $X,E$ be spectra. Then every phantom in $E^*(X)$ is a weak phantom. If $X$
has finite $\Z$-type then every weak phantom is a phantom.
\end{prop}
\setcounter{thm}{11}
\begin{exmp}
Let $E$ be a spectrum and $X$ be a CW-complex (resp.\ spectrum). Let
$\scrK=\{X_\lambda:\lambda\in\Lambda\}$ be a family of CW-subcomplexes (resp.\
subspectra) ordered by inclusion, and let $i_\lambda$ be the inclusions. Assume
$\scrK$ is closed under finite intersections and finite unions. From the maps
$i^*\lambda:E^*(X)\to E^*(X_\lambda)$ we obtain a map $\rho:E^*(X)\to\varprojlim
E^*(X_\lambda)$.
\end{exmp}
\begin{prop}
The epio-mono factorisation of $\rho$ is
$\xymatrix@C=.7cm{E^*(X)\ar@{->>}[r]^\sigma&E^*_\scrK(X)
\ar@{^{(}->}[r]^{\chi\ \ }&\varprojlim E^*(X_\lambda)}$. Thus $\ker
\sigma=\ker\rho$, and $\scrK$-phantoms are just nontrivial elements of
$\ker\rho$.
\end{prop}
\begin{const}
Suppose $f:X\to Y$ is a \textbf{map} of spectra. Let $\{X_\lambda\}$ and
$\{Y_\mu\}$ be the families of finite subspectra. We construct a morphism
$f^*:\varprojlim E^*(Y_\mu)\to\varprojlim E^*(X_\lambda)$ as follows. By II.3.14
there are cofinal subfamilies families $\{X_\omega\}\subseteq \{X_\lambda\}$ and
$\{Y_\omega\subseteq\{Y_\mu\}\}$ such that $f(X_\omega)\subseteq Y_\omega$.
These maps induce a map of direct limits over $\omega$, and cofinality gives us
what we need. If instead $f$ is a morphism of spectra, we can represent it by a
map from a cofinal subspectrum, and the perform this construction to the same
effect.
\end{const}
\begin{prop}
The morphism $f^*$ defined above does not depend on the choice of cofinal
subfamilies, and so the group $\varprojlim E^*(X_\lambda)$ is natural in $X$.
Moreover, homotopic maps of spectra induce equal maps of groups.
\end{prop}
\begin{prop}
Let $X_n$ be the $n$-skeleton of a CW-complex $X$. Then for every space $Y$,
$\rho:[X,Y]\to \varprojlim[X_n,Y]$ is surjective. The same holds for pointed
spaces and maps.
\end{prop}
\begin{lem}
If $\cdots\rightarrow K_n\rightarrow K_{n-1}\rightarrow\cdots$ is an inverse
system of finite sets, $\varprojlim K_n\neq\emptyset$.
\end{lem}
\begin{thm}Let $Y$ be a pointed, connected simple topological space.
\begin{itemise}
\itm[(i)]Let $(Z,A)$ be a CW-pair such that $H^k(Z,A;\pi_k(Y))$ is finite for
$k>0$. Then a map $u:A\to Y$ extends to a map $Z\to Y$ iff it extends to a map
$Z^{(n)}\to Y$ for all $n$.
\itm[(ii)]Let $X$ be a pointed CW-complex such that $H^{k-1}(X;\pi_k(Y))$ is
finite for all $k>0$. Then both functions
$\rho:[X,Y]^\bullet\to\varprojlim[X^{(n)},Y]^\bullet$ and
$\rho:[X,Y]\to\varprojlim[X^{(n)},Y]$ are bijections.
\end{itemise}
\end{thm}
\begin{lem}
If $\Lambda'\subset\Lambda$ is a subset of a poset given the induced poset
structure, then there is a natural map
$\scrM_{\Lambda'}^\Lambda:\varprojlim_\Lambda M_\lambda\to\varprojlim_{\Lambda'}
M_\lambda$. It is an isomorphism if $\Lambda'$ is cofinal.
\end{lem}
\begin{lem}
Direct limits commute, in the sense that if $\Lambda,\Lambda'$ are two posets:
\[\varprojlim_{\Lambda\times\Lambda'}A_{\lambda,\lambda'}
=\varprojlim_{\Lambda}\biggl\{\varprojlim_{\Lambda'}A_{\lambda,\lambda'}\biggr\}
=\varprojlim_{\Lambda'}\biggl\{\varprojlim_{\Lambda}A_{\lambda,\lambda'}\biggr\}
.\]
In particular, direct limits commute with products. Here, the poset
$\Lambda\times\Lambda'$ has $(\lambda,\lambda')\leq(\mu,\mu')$ iff
$\lambda\leq\mu$ and $\lambda'\leq\mu'$.
\end{lem}
\begin{defn}
Given a commutative ring $R$ and inverse systems $\{M_\lambda\}$,
$\{N_{\lambda'}\}$ of $R$-modules, we can form the
$\Lambda\DASH\Lambda'$-\emph{completed tensor product}
$M\otimes^{\Lambda\DASH\Lambda'}N:=\varprojlim M_\lambda\otimes N_{\lambda'}$.
\end{defn}
}   %%% End comment %%%
\subsection{Derived functors of the inverse limit functor}
The inverse limit is a left exact covariant functor from the category of inverse
systems (over a fixed poset) to abelian groups. It is not right exact. We hope
to construct right derived functors.

\end{document}





ehdi 8579194209








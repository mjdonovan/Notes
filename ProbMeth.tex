% !TEX root = z_output/_ProbMeth.tex
%%%%%%%%%%%%%%%%%%%%%%%%%%%%%%%%%%%%%%%%%%%%%%%%%%%%%%%%%%%%%%%%%%%%%%%%%%%%%%%%
%%%%%%%%%%%%%%%%%%%%%%%%%%% 80 characters %%%%%%%%%%%%%%%%%%%%%%%%%%%%%%%%%%%%%%
%%%%%%%%%%%%%%%%%%%%%%%%%%%%%%%%%%%%%%%%%%%%%%%%%%%%%%%%%%%%%%%%%%%%%%%%%%%%%%%%
\documentclass[11pt]{article}
\usepackage{fullpage}
\usepackage{amsmath,amsthm,amssymb}
\usepackage{mathrsfs,nicefrac}
\usepackage{amssymb}
\usepackage{epsfig}
\usepackage[all]{xy}
\usepackage{sseq}
\usepackage{tocloft}
\usepackage{cancel}
\usepackage[strict]{changepage}
\usepackage{color}
\usepackage{tikz}
\usepackage{extpfeil}
\usepackage{version}
%\usepackage{ifthen}
%Used for disabling hyperref
\ifx\dontloadhyperref\undefined
%\usepackage[pdftex,pdfborder={0 0 0 [1 1]}]{hyperref}
\usepackage[pdftex,pdfborder={0 0 .5 [1 1]}]{hyperref}
\else
\providecommand{\texorpdfstring}[2]{#1}
\fi

%>>>>>>>>>>>>>>>>>>>>>>>>>>>>>>
%<<<       Better ToC       <<<
%>>>>>>>>>>>>>>>>>>>>>>>>>>>>>>
\setlength{\cftbeforesecskip}{0.5ex}

%>>>>>>>>>>>>>>>>>>>>>>>>>>>>>>
%<<<      Hyperref mod      <<<
%>>>>>>>>>>>>>>>>>>>>>>>>>>>>>>

%needs more testing
\newcounter{dummyforrefstepcounter}
\newcommand{\labelRIGHTHERE}[1]
{\refstepcounter{dummyforrefstepcounter}\label{#1}}


%>>>>>>>>>>>>>>>>>>>>>>>>>>>>>>
%<<<  Theorem Environments  <<<
%>>>>>>>>>>>>>>>>>>>>>>>>>>>>>>
\ifx\dontloaddefinitionsoftheoremenvironments\undefined
\theoremstyle{plain}
\newtheorem{thm}{Theorem}[section]
\newtheorem*{thm*}{Theorem}
\newtheorem{lem}[thm]{Lemma}
\newtheorem*{lem*}{Lemma}
\newtheorem{prop}[thm]{Proposition}
\newtheorem*{prop*}{Proposition}
\newtheorem{cor}[thm]{Corollary}
\newtheorem*{cor*}{Corollary}
\newtheorem{defprop}[thm]{Definition-Proposition}
\newtheorem*{punchline}{Punchline}

\theoremstyle{definition}
\newtheorem{defn}{Definition}[section]
\newtheorem*{defn*}{Definition}
\newtheorem{exmp}{Example}[section]
\newtheorem*{exmp*}{Example}
\newtheorem{asspt}{Assumption}[section]
\newtheorem{notation}{Notation}[section]
\newtheorem{exercise}{Exercise}[section]
\newtheorem*{fact*}{Fact}
\newtheorem*{rmk*}{Remark}
\newtheorem{fact}{Fact}
\newtheorem*{aside}{Aside}
\newtheorem*{question}{Question}
\newtheorem*{answer}{Answer}

\else\relax\fi

%>>>>>>>>>>>>>>>>>>>>>>>>>>>>>>
%<<<      Fields, etc.      <<<
%>>>>>>>>>>>>>>>>>>>>>>>>>>>>>>
\DeclareSymbolFont{AMSb}{U}{msb}{m}{n}
\DeclareMathSymbol{\N}{\mathbin}{AMSb}{"4E}
\DeclareMathSymbol{\Octonions}{\mathbin}{AMSb}{"4F}
\DeclareMathSymbol{\Z}{\mathbin}{AMSb}{"5A}
\DeclareMathSymbol{\R}{\mathbin}{AMSb}{"52}
\DeclareMathSymbol{\Q}{\mathbin}{AMSb}{"51}
\DeclareMathSymbol{\PP}{\mathbin}{AMSb}{"50}
\DeclareMathSymbol{\I}{\mathbin}{AMSb}{"49}
\DeclareMathSymbol{\C}{\mathbin}{AMSb}{"43}
\DeclareMathSymbol{\A}{\mathbin}{AMSb}{"41}
\DeclareMathSymbol{\F}{\mathbin}{AMSb}{"46}
\DeclareMathSymbol{\Quaternions}{\mathbin}{AMSb}{"48}


%>>>>>>>>>>>>>>>>>>>>>>>>>>>>>>
%<<<       Operators        <<<
%>>>>>>>>>>>>>>>>>>>>>>>>>>>>>>
\DeclareMathOperator{\ad}{\textbf{ad}}
\DeclareMathOperator{\coker}{coker}
\renewcommand{\ker}{\textup{ker}\,}
\DeclareMathOperator{\End}{End}
\DeclareMathOperator{\Aut}{Aut}
\DeclareMathOperator{\Hom}{Hom}
\DeclareMathOperator{\Maps}{Maps}
\DeclareMathOperator{\Mor}{Mor}
\DeclareMathOperator{\Gal}{Gal}
\DeclareMathOperator{\Ext}{Ext}
\DeclareMathOperator{\Tor}{Tor}
\DeclareMathOperator{\Map}{Map}
\DeclareMathOperator{\Der}{Der}
\DeclareMathOperator{\Rad}{Rad}
\DeclareMathOperator{\rank}{rank}
\DeclareMathOperator{\ArfInvariant}{Arf}
\DeclareMathOperator{\KervaireInvariant}{Ker}
\DeclareMathOperator{\im}{im}
\DeclareMathOperator{\coim}{coim}
\DeclareMathOperator{\trace}{tr}
\DeclareMathOperator{\supp}{supp}
\DeclareMathOperator{\ann}{ann}
\DeclareMathOperator{\spec}{Spec}
\DeclareMathOperator{\proj}{Proj}
\DeclareMathOperator{\fiber}{F}
\DeclareMathOperator{\cofiber}{C}
\DeclareMathOperator{\cone}{cone}
\DeclareMathOperator{\Skel}{Sk}
\DeclareMathOperator{\conn}{conn}
\DeclareMathOperator{\colim}{colim}
\DeclareMathOperator{\limit}{lim}

%>>>>>>>>>>>>>>>>>>>>>>>>>>>>>>
%<<<   Cohomology Theories  <<<
%>>>>>>>>>>>>>>>>>>>>>>>>>>>>>>
\DeclareMathOperator{\KR}{{K\R}}
\DeclareMathOperator{\KO}{{KO}}
\DeclareMathOperator{\K}{{K}}
\DeclareMathOperator{\OmegaO}{{\Omega_{\Octonions}}}

%>>>>>>>>>>>>>>>>>>>>>>>>>>>>>>
%<<<   Algebraic Geometry   <<<
%>>>>>>>>>>>>>>>>>>>>>>>>>>>>>>
\DeclareMathOperator{\Spec}{Spec}
\DeclareMathOperator{\Proj}{Proj}
\DeclareMathOperator{\Sing}{Sing}
\DeclareMathOperator{\shfHom}{\mathscr{H}\textit{\!\!om}}
\newcommand{\WeilDivisors}{\textup{Div}}
\newcommand{\CartierDivisors}{\textup{CaDiv}}
\newcommand{\PrincipalWeilDivisors}{\textup{PrDiv}}
\newcommand{\LocallyPrincipalWeilDivisors}{\textup{LPDiv}}
\newcommand{\PrincipalCartierDivisors}{\textup{PrCaDiv}}
\newcommand{\DivisorClass}{\textup{Cl}}
\newcommand{\CartierClass}{\textup{CaCl}}
\newcommand{\Picard}{\textup{Pic}}
\DeclareMathOperator{\Frob}{Frob}


%>>>>>>>>>>>>>>>>>>>>>>>>>>>>>>
%<<<  Mathematical Objects  <<<
%>>>>>>>>>>>>>>>>>>>>>>>>>>>>>>
\newcommand{\sll}{\mathfrak{sl}}
\newcommand{\gl}{\mathfrak{gl}}
\newcommand{\GL}{\mbox{GL}}
\newcommand{\PGL}{\mbox{PGL}}
\newcommand{\SL}{\mbox{SL}}
\newcommand{\Mat}{\mbox{Mat}}
\newcommand{\Gr}{\textup{Gr}}
\newcommand{\Squ}{\textup{Sq}}
\newcommand{\catSet}{\textit{Sets}}
\newcommand{\RP}{{\R\PP}}
\newcommand{\CP}{{\C\PP}}
\newcommand{\Steen}{\mathscr{A}}
\newcommand{\Orth}{\textup{\textbf{O}}}

%>>>>>>>>>>>>>>>>>>>>>>>>>>>>>>
%<<<  Mathematical Symbols  <<<
%>>>>>>>>>>>>>>>>>>>>>>>>>>>>>>
\newcommand{\DASH}{\textup{---}}
\newcommand{\op}{\textup{op}}
\newcommand{\ob}{\textup{ob}\,}
\newcommand{\ho}{\textup{ho}}
\newcommand{\st}{\textup{st}}
\newcommand{\id}{\textup{id}}
\newcommand{\Bullet}{\ensuremath{\bullet} }

%>>>>>>>>>>>>>>>>>>>>>>>>>>>>>>
%<<<      Some Arrows       <<<
%>>>>>>>>>>>>>>>>>>>>>>>>>>>>>>
\let\shortmapsto\mapsto
\let\mapsto\longmapsto
\newcommand{\mapsfrom}{\,\reflectbox{$\mapsto$}\ }
\newcommand{\bigrightsquig}{\scalebox{2}{\ensuremath{\rightsquigarrow}}}
\newcommand{\bigleftsquig}{\reflectbox{\scalebox{2}{\ensuremath{\rightsquigarrow}}}}

%\newcommand{\cofibration}{\xhookrightarrow{\phantom{\ \,{\sim\!}\ \ }}}
%\newcommand{\fibration}{\xtwoheadrightarrow{\phantom{\sim\!}}}
%\newcommand{\acycliccofibration}{\xhookrightarrow{\ \,{\sim\!}\ \ }}
%\newcommand{\acyclicfibration}{\xtwoheadrightarrow{\sim\!}}
%\newcommand{\leftcofibration}{\xhookleftarrow{\phantom{\ \,{\sim\!}\ \ }}}
%\newcommand{\leftfibration}{\xtwoheadleftarrow{\phantom{\sim\!}}}
%\newcommand{\leftacycliccofibration}{\xhookleftarrow{\ \ {\sim\!}\,\ }}
%\newcommand{\leftacyclicfibration}{\xtwoheadleftarrow{\sim\!}}
%\newcommand{\weakequiv}{\xrightarrow{\ \,\sim\,\ }}
%\newcommand{\leftweakequiv}{\xleftarrow{\ \,\sim\,\ }}

\newcommand{\cofibration}
{\xhookrightarrow{\phantom{\ \,{\raisebox{-.3ex}[0ex][0ex]{\scriptsize$\sim$}\!}\ \ }}}
\newcommand{\fibration}
{\xtwoheadrightarrow{\phantom{\raisebox{-.3ex}[0ex][0ex]{\scriptsize$\sim$}\!}}}
\newcommand{\acycliccofibration}
{\xhookrightarrow{\ \,{\raisebox{-.55ex}[0ex][0ex]{\scriptsize$\sim$}\!}\ \ }}
\newcommand{\acyclicfibration}
{\xtwoheadrightarrow{\raisebox{-.6ex}[0ex][0ex]{\scriptsize$\sim$}\!}}
\newcommand{\leftcofibration}
{\xhookleftarrow{\phantom{\ \,{\raisebox{-.3ex}[0ex][0ex]{\scriptsize$\sim$}\!}\ \ }}}
\newcommand{\leftfibration}
{\xtwoheadleftarrow{\phantom{\raisebox{-.3ex}[0ex][0ex]{\scriptsize$\sim$}\!}}}
\newcommand{\leftacycliccofibration}
{\xhookleftarrow{\ \ {\raisebox{-.55ex}[0ex][0ex]{\scriptsize$\sim$}\!}\,\ }}
\newcommand{\leftacyclicfibration}
{\xtwoheadleftarrow{\raisebox{-.6ex}[0ex][0ex]{\scriptsize$\sim$}\!}}
\newcommand{\weakequiv}
{\xrightarrow{\ \,\raisebox{-.3ex}[0ex][0ex]{\scriptsize$\sim$}\,\ }}
\newcommand{\leftweakequiv}
{\xleftarrow{\ \,\raisebox{-.3ex}[0ex][0ex]{\scriptsize$\sim$}\,\ }}


%>>>>>>>>>>>>>>>>>>>>>>>>>>>>>>
%<<<     Greek Letters      <<<
%>>>>>>>>>>>>>>>>>>>>>>>>>>>>>>
%\newcommand{\oldphi}{\phi}
%\renewcommand{\phi}{\varphi}
\let\oldphi\phi
\let\phi\varphi
\renewcommand{\to}{\longrightarrow}
\newcommand{\eps}{\varepsilon}

%>>>>>>>>>>>>>>>>>>>>>>>>>>>>>>
%<<<  1st-4th & parentheses <<<
%>>>>>>>>>>>>>>>>>>>>>>>>>>>>>>
\newcommand{\first}{^\text{st}}
\newcommand{\second}{^\text{nd}}
\newcommand{\third}{^\text{rd}}
\newcommand{\fourth}{^\text{th}}
\newcommand{\ZEROTH}{$0^\text{th}$ }
\newcommand{\FIRST}{$1^\text{st}$ }
\newcommand{\SECOND}{$2^\text{nd}$ }
\newcommand{\THIRD}{$3^\text{rd}$ }
\newcommand{\FOURTH}{$4^\text{th}$ }
\newcommand{\iTH}{$i^\text{th}$ }
\newcommand{\jTH}{$j^\text{th}$ }
\newcommand{\nTH}{$n^\text{th}$ }

%>>>>>>>>>>>>>>>>>>>>>>>>>>>>>>
%<<<    upright commands    <<<
%>>>>>>>>>>>>>>>>>>>>>>>>>>>>>>
\newcommand{\upcol}{\textup{:}}
\newcommand{\upsemi}{\textup{;}}
\providecommand{\lparen}{\textup{(}}
\providecommand{\rparen}{\textup{)}}
\renewcommand{\lparen}{\textup{(}}
\renewcommand{\rparen}{\textup{)}}
\newcommand{\Iff}{\emph{iff} }

%>>>>>>>>>>>>>>>>>>>>>>>>>>>>>>
%<<<     Environments       <<<
%>>>>>>>>>>>>>>>>>>>>>>>>>>>>>>
\newcommand{\squishlist}
{ %\setlength{\topsep}{100pt} doesn't seem to do anything.
  \setlength{\itemsep}{.5pt}
  \setlength{\parskip}{0pt}
  \setlength{\parsep}{0pt}}
\newenvironment{itemise}{
\begin{list}{\textup{$\rightsquigarrow$}}
   {  \setlength{\topsep}{1mm}
      \setlength{\itemsep}{1pt}
      \setlength{\parskip}{0pt}
      \setlength{\parsep}{0pt}
   }
}{\end{list}\vspace{-.1cm}}
\newcommand{\INDENT}{\textbf{}\phantom{space}}
\renewcommand{\INDENT}{\rule{.7cm}{0cm}}

\newcommand{\itm}[1][$\rightsquigarrow$]{\item[{\makebox[.5cm][c]{\textup{#1}}}]}

\newcommand{\rednote}[1]{{\color{red}#1}\scalebox{.1}{rednote}}
\newcommand{\bluenote}[1]{{\color{blue}#1}\scalebox{.1}{rednote}}
\newcommand{\funcdef}[4]{\begin{align*}
#1&\to #2\\
#3&\mapsto#4
\end{align*}}

%\newcommand{\comment}[1]{}

%>>>>>>>>>>>>>>>>>>>>>>>>>>>>>>
%<<<       Categories       <<<
%>>>>>>>>>>>>>>>>>>>>>>>>>>>>>>
\newcommand{\Ens}{{\mathscr{E}ns}}
\DeclareMathOperator{\Sheaves}{{\mathsf{Shf}}}
\DeclareMathOperator{\Presheaves}{{\mathsf{PreShf}}}
\DeclareMathOperator{\Varieties}{{\mathsf{Var}}}
\DeclareMathOperator{\Schemes}{{\mathsf{Sch}}}
\DeclareMathOperator{\Rings}{{\mathsf{Rings}}}
\DeclareMathOperator{\AbGp}{{\mathsf{AbGp}}}
\DeclareMathOperator{\Modules}{{\mathsf{\!-Mod}}}
\DeclareMathOperator{\QuasiCoherent}{{\mathsf{QCoh}}}
\DeclareMathOperator{\Coherent}{{\mathsf{Coh}}}
\DeclareMathOperator{\GSW}{{\mathcal{SW}^G}}
\DeclareMathOperator{\Burnside}{{\mathsf{Burn}}}
\DeclareMathOperator{\GSet}{{G\mathsf{Set}}}
\DeclareMathOperator{\FinGSet}{{G\mathsf{Set}^\textup{fin}}}
\DeclareMathOperator{\HSet}{{H\mathsf{Set}}}
\DeclareMathOperator{\Cat}{{\mathsf{Cat}}}
\DeclareMathOperator{\Orb}{{\mathsf{Orb}}}
\DeclareMathOperator{\Set}{{\mathsf{Set}}}
\DeclareMathOperator{\sSet}{{\mathsf{sSet}}}
\DeclareMathOperator{\Top}{{\mathsf{Top}}}
\DeclareMathOperator{\GSpectra}{{G-\mathsf{Spectra}}}

%>>>>>>>>>>>>>>>>>>>>>>>>>>>>>>
%<<<     Script Letters     <<<
%>>>>>>>>>>>>>>>>>>>>>>>>>>>>>>
\newcommand{\scrQ}{\mathscr{Q}}
\newcommand{\scrW}{\mathscr{W}}
\newcommand{\scrE}{\mathscr{E}}
\newcommand{\scrR}{\mathscr{R}}
\newcommand{\scrT}{\mathscr{T}}
\newcommand{\scrY}{\mathscr{Y}}
\newcommand{\scrU}{\mathscr{U}}
\newcommand{\scrI}{\mathscr{I}}
\newcommand{\scrO}{\mathscr{O}}
\newcommand{\scrP}{\mathscr{P}}
\newcommand{\scrA}{\mathscr{A}}
\newcommand{\scrS}{\mathscr{S}}
\newcommand{\scrD}{\mathscr{D}}
\newcommand{\scrF}{\mathscr{F}}
\newcommand{\scrG}{\mathscr{G}}
\newcommand{\scrH}{\mathscr{H}}
\newcommand{\scrJ}{\mathscr{J}}
\newcommand{\scrK}{\mathscr{K}}
\newcommand{\scrL}{\mathscr{L}}
\newcommand{\scrZ}{\mathscr{Z}}
\newcommand{\scrX}{\mathscr{X}}
\newcommand{\scrC}{\mathscr{C}}
\newcommand{\scrV}{\mathscr{V}}
\newcommand{\scrB}{\mathscr{B}}
\newcommand{\scrN}{\mathscr{N}}
\newcommand{\scrM}{\mathscr{M}}

%>>>>>>>>>>>>>>>>>>>>>>>>>>>>>>
%<<<     Fractur Letters    <<<
%>>>>>>>>>>>>>>>>>>>>>>>>>>>>>>
\newcommand{\frakQ}{\mathfrak{Q}}
\newcommand{\frakW}{\mathfrak{W}}
\newcommand{\frakE}{\mathfrak{E}}
\newcommand{\frakR}{\mathfrak{R}}
\newcommand{\frakT}{\mathfrak{T}}
\newcommand{\frakY}{\mathfrak{Y}}
\newcommand{\frakU}{\mathfrak{U}}
\newcommand{\frakI}{\mathfrak{I}}
\newcommand{\frakO}{\mathfrak{O}}
\newcommand{\frakP}{\mathfrak{P}}
\newcommand{\frakA}{\mathfrak{A}}
\newcommand{\frakS}{\mathfrak{S}}
\newcommand{\frakD}{\mathfrak{D}}
\newcommand{\frakF}{\mathfrak{F}}
\newcommand{\frakG}{\mathfrak{G}}
\newcommand{\frakH}{\mathfrak{H}}
\newcommand{\frakJ}{\mathfrak{J}}
\newcommand{\frakK}{\mathfrak{K}}
\newcommand{\frakL}{\mathfrak{L}}
\newcommand{\frakZ}{\mathfrak{Z}}
\newcommand{\frakX}{\mathfrak{X}}
\newcommand{\frakC}{\mathfrak{C}}
\newcommand{\frakV}{\mathfrak{V}}
\newcommand{\frakB}{\mathfrak{B}}
\newcommand{\frakN}{\mathfrak{N}}
\newcommand{\frakM}{\mathfrak{M}}

\newcommand{\frakq}{\mathfrak{q}}
\newcommand{\frakw}{\mathfrak{w}}
\newcommand{\frake}{\mathfrak{e}}
\newcommand{\frakr}{\mathfrak{r}}
\newcommand{\frakt}{\mathfrak{t}}
\newcommand{\fraky}{\mathfrak{y}}
\newcommand{\fraku}{\mathfrak{u}}
\newcommand{\fraki}{\mathfrak{i}}
\newcommand{\frako}{\mathfrak{o}}
\newcommand{\frakp}{\mathfrak{p}}
\newcommand{\fraka}{\mathfrak{a}}
\newcommand{\fraks}{\mathfrak{s}}
\newcommand{\frakd}{\mathfrak{d}}
\newcommand{\frakf}{\mathfrak{f}}
\newcommand{\frakg}{\mathfrak{g}}
\newcommand{\frakh}{\mathfrak{h}}
\newcommand{\frakj}{\mathfrak{j}}
\newcommand{\frakk}{\mathfrak{k}}
\newcommand{\frakl}{\mathfrak{l}}
\newcommand{\frakz}{\mathfrak{z}}
\newcommand{\frakx}{\mathfrak{x}}
\newcommand{\frakc}{\mathfrak{c}}
\newcommand{\frakv}{\mathfrak{v}}
\newcommand{\frakb}{\mathfrak{b}}
\newcommand{\frakn}{\mathfrak{n}}
\newcommand{\frakm}{\mathfrak{m}}

%>>>>>>>>>>>>>>>>>>>>>>>>>>>>>>
%<<<  Caligraphic Letters   <<<
%>>>>>>>>>>>>>>>>>>>>>>>>>>>>>>
\newcommand{\calQ}{\mathcal{Q}}
\newcommand{\calW}{\mathcal{W}}
\newcommand{\calE}{\mathcal{E}}
\newcommand{\calR}{\mathcal{R}}
\newcommand{\calT}{\mathcal{T}}
\newcommand{\calY}{\mathcal{Y}}
\newcommand{\calU}{\mathcal{U}}
\newcommand{\calI}{\mathcal{I}}
\newcommand{\calO}{\mathcal{O}}
\newcommand{\calP}{\mathcal{P}}
\newcommand{\calA}{\mathcal{A}}
\newcommand{\calS}{\mathcal{S}}
\newcommand{\calD}{\mathcal{D}}
\newcommand{\calF}{\mathcal{F}}
\newcommand{\calG}{\mathcal{G}}
\newcommand{\calH}{\mathcal{H}}
\newcommand{\calJ}{\mathcal{J}}
\newcommand{\calK}{\mathcal{K}}
\newcommand{\calL}{\mathcal{L}}
\newcommand{\calZ}{\mathcal{Z}}
\newcommand{\calX}{\mathcal{X}}
\newcommand{\calC}{\mathcal{C}}
\newcommand{\calV}{\mathcal{V}}
\newcommand{\calB}{\mathcal{B}}
\newcommand{\calN}{\mathcal{N}}
\newcommand{\calM}{\mathcal{M}}

%>>>>>>>>>>>>>>>>>>>>>>>>>>>>>>
%<<<<<<<<<DEPRECIATED<<<<<<<<<<
%>>>>>>>>>>>>>>>>>>>>>>>>>>>>>>

%%% From Kac's template
% 1-inch margins, from fullpage.sty by H.Partl, Version 2, Dec. 15, 1988.
%\topmargin 0pt
%\advance \topmargin by -\headheight
%\advance \topmargin by -\headsep
%\textheight 9.1in
%\oddsidemargin 0pt
%\evensidemargin \oddsidemargin
%\marginparwidth 0.5in
%\textwidth 6.5in
%
%\parindent 0in
%\parskip 1.5ex
%%\renewcommand{\baselinestretch}{1.25}

%%% From the net
%\newcommand{\pullbackcorner}[1][dr]{\save*!/#1+1.2pc/#1:(1,-1)@^{|-}\restore}
%\newcommand{\pushoutcorner}[1][dr]{\save*!/#1-1.2pc/#1:(-1,1)@^{|-}\restore}









\usepackage{makeidx}


\includeversion{prelims}
\includeversion{chapter1}
\excludeversion{chapter2}
\includeversion{chapter3}




%not yet:
\includeversion{chapter4}
\includeversion{chapter5}
\includeversion{chapter6}
\includeversion{chapter7}
\includeversion{chapter8}
\includeversion{chapter9}



\usepackage{multicol}
\makeatletter
\renewenvironment{theindex}
  {\if@twocolumn
      \@restonecolfalse
   \else
      \@restonecoltrue
   \fi
   \setlength{\columnseprule}{0pt}
   \setlength{\columnsep}{10pt}% I CHANGED THIS from 35pt
   \begin{multicols}{3}[\section*{\indexname}]
   \markboth{\MakeUppercase\indexname}%
            {\MakeUppercase\indexname}%
   \thispagestyle{plain}
   \setlength{\parindent}{0pt}
   \setlength{\parskip}{0pt plus 0.3pt}
   \relax
   \let\item\@idxitem
   \small}% I ADDED THIS
  {\end{multicols}\if@restonecol\onecolumn\else\clearpage\fi}
\makeatother


\makeindex
\newcommand{\indexThmCode}{}
\newcommand{\Index}[1]{\index{#1}#1}
\newcommand{\indexThm}[1]{\index{Hartshorne!Theorems, etc.!#1}}
\newcommand{\Entry}[3][notcustom]{\Bullet
\ifthenelse{\equal{#1}{}}{}
{
\ifthenelse{\equal{#1}{notcustom}}{\index{#2}}{\index{#1}}
}
 \textbf{#2:}&#3}

%\newenvironment{INT}{\begin{itemize}\small\item}{\end{itemize}}

\newenvironment{INT}[1][]{\begin{itemize}\small\item\textbf{#1}}{\end{itemize}}
\newcommand{\moreINT}[1][]{\item\textbf{#1}}



\begin{document}
\begin{prelims}
\section*{\S$-\infty$: Preliminaries}
\begin{itemise}
\item Asymptotic notations
\begin{itemize}\squishlist
\item \makebox[4.5em][l]{$f= O(g)$} means ``$f\leq cg$''.
\item \makebox[4.5em][l]{$f=o(g)$} means ``$f/g\to0$''.
\item \makebox[4.5em][l]{$f=\Omega(g)$} means ``$f\geq cg$''.
\item \makebox[4.5em][l]{$f=\Theta(g)$} means ``$c_1 g\leq f\leq c_2 g$''.
\item \makebox[4.5em][l]{$f\sim g$} means ``$f/g\to1$''.
\end{itemize}
\item $1-p\leq e^{-p}$ for all nonnegative $p$, close for small $p$.
\item The Stirling approximation: $k!\sim \sqrt{2\pi k}(k/e)^k$.
\item Some bounds on ${n\choose k}$:
\[\left(\frac{n}{k}\right)^k\leq {n\choose k}\leq\frac{n^k}{k!}\leq\left(\frac{n\cdot e}{k}\right)^k.\]
\item On central binomial coefficients:
\[\frac{2^{2n-1}}{\sqrt n}\leq {2n\choose n}\sim\frac{4^n}{\sqrt{\pi n}}.\]
\item Let $S_n$ be the sum of $n$ iid uniform $\{\pm1\}$ random variables. Then:
\[\Exp[|S_n|]=n2^{1-n}{n-1\choose\lfloor(n-1)/2\rfloor}=(\sqrt{2/\pi}+o(1))\sqrt n.\]
\end{itemise}
\end{prelims}
\begin{chapter1}
\section*{\S1: The basic method}
\begin{itemise}
\item The \textbf{Ramsey number} $R(k,l)$ is the least $n$ such that for any 2-colouring of the vertices of the complete graph $K_n$ on $n$ vertices, there is always either a blue $K_k$ or a red $K_l$.
\begin{INT}[Prop 1.1.1:]
If ${n\choose k}2^{1-{k\choose 2}}<1$ then $R(k,k)>n$.
\begin{proof}
Want to show that there's a chance you fail to get a monochromatic subgraph from a random colouring. Add up the probabilities of getting a monochromatic subgraph at each opportunity, to get ${n\choose k}2^{1-{k\choose 2}}$. It is possible that none of these occur.
\end{proof}
\end{INT}
\item A \textbf{tournament} on a set $V$ of $n$ players is an orientation of the complete graph $K_V$. That is, every pair of players has a unique winner.
A tournament could have \textbf{property $S_k$}: for every group of $k$ players, there is one that beats them all.
\begin{INT}[Thm 1.2.1:]
If ${n\choose k}(1-2^{-k})^{n-k}<1$ then there is a tournament with $S_k$ of size $n$.
\begin{proof}
Choose a random tournament. The chance that it fails to have $S_k$ is bounded above by this probability.
\end{proof}
\end{INT}
\item A \textbf{dominating set} in a graph $(V,E)$ is a set $U$ of vertices to which every vertex $v\in V-U$ is connected by an edge.
\begin{INT}[Thm 1.2.2:]
A graph on $n$ vertices with minimum degree $\delta>1$ has a dominating set of at most $n[1+\log(\delta+1)]/(\delta+1)$ vertices.
\begin{proof}
Fix a probability $p$. Choose vertices of with probability $p$. The expected size of the set $\{\text{vertices chosen}\}\cup\{\text{vertices not neighbouring the chosen set}\}$ is at most $pn+(1-p)^{\delta+1}n$, by linearity of expectation (also note: alteration). Thus there is a dominating set of size at most this amount. Using $1-p\leq e^{-p}$ to simplify, and then minimizing, we get the result.
\end{proof}
\end{INT}
\item A \textbf{cut} in a graph is simply a partition of the edges into two nonempty disjoint subsets. The \textbf{size} of a cut is the number of edges whose endpoints lie in opposite sides. The \textbf{edge connectivity} is the minimum size of a cut.
\item A \textbf{hypergraph} $(V,E)$ is a finite set $V$ of vertices, and a subset $E\subset 2^V$. It is \textbf{$n$-uniform} if each edge contains $n$ vertices. It has \textbf{property B}, or is \textbf{two-colourable}, if there is a two-colouring of the vertices so that no edge is monochromatic. Let $m(n)$ be the minimum number of edges of an $n$-uniform hypergraph which is not two-colourable.
\begin{INT}[Prop 1.3.1:]
Every $n$-uniform hypergraph with less that $2^{n-1}$ edges is two-colourable, so that $m(n)\geq 2^{n-1}$.
\begin{proof}
Randomly colour the vertices. The probability of failing to give a valid 2-colouring is less than $2^{n-1}\cdot 2^{1-n}$.
\end{proof}
\item \textbf{Thm 1.3.2:} $m(n)<(1+o(1))\frac{e\log2}{4}n^22^n$.
\begin{proof}
We need to construct a small $n$-uniform hypergraph which is not 2-colourable. Let $v$ and $m$ be to be determined. Then choose a set $V$ of size $v$ for the vertices. A random subset $S$ of size $n$ has probability $({a\choose n}+{b\choose n})/{v\choose n}$ of being monochromatic (in a colouring with $a$ and $b$ of each colour), which is at least $p=2{v/2\choose n}/{v\choose n}$, by convexity (if $v$ is even).

\INDENT Now choose $m$ random sets of size $n$ independently. The probability that a given colouring is a 2-colouring is at most $(1-p)^m$, so the probability that there exists a 2-colouring is at most $2^v(1-p)^m$. Thus, if this quantity is less than 1, we have $m(n)\leq m$. Now it's asymptotics.
\end{proof}
\end{INT}
\item A \textbf{$(k,l)$-system} is a family $\calF=\{(A_i,B_i)\}_{i=1}^h$ pairs of sets\footnote{Really, pairs of subsets of an arbitrary set.}, such that $|A_i|=k$, $|B_i|=l$, and $A_i\cap B_j$ is empty \Iff $i=j$, for $1\leq i,j\leq h$.
\begin{INT}[Thm 1.3.3:]
If $\calF=\{(A_i,B_i)\}_{i=1}^h$ is a $(k,l)$-system, then $h\leq {k+l\choose k}$.
\begin{proof}
Let $X=\bigcup_i (A_i\cup B_i)$, and randomly order $X$. Let $X_i$ be the event that the elements of $A_i$ precede those of $B_i$. If these events are pairwise disjoint, we have $1\geq \sum \PP[X_i]=h/{k+l\choose k}$.

\INDENT Suppose then that both $X_i$ and $X_j$ can occur simultaneously. Then there's an ordering such that $A_i$ precedes $B_i$ and $A_j$ precedes $B_j$. WLOG the last element of $A_i$ precedes that of $A_j$, but then $A_i$ precedes $B_j$, contradicting that $A_i\cap B_j$ is nonempty.
\end{proof}
\item This is sharp, writing $\calF=\{(A,X\setminus A) : A\subset X, |A|=k\}$, where $|X|=k+l$.
\end{INT}
\item A subset $A$ of an abelian group is \textbf{sum-free} if $(A+A)\cap A$ is empty.
\begin{INT}[Thm 1.4.1:]
Every set $A$ of $n$ nonzero integers contains a sum-free subset of size at least $n/3$.
\begin{proof}
Choose a prime $p=3k+2$ far larger than the largest absolute value of any element of $A$. Embed the question in the field $\F=\Z/p\Z$, where it is unchanged. In this field, there is a large sum-free subset $C=\{k+1,\ldots,2k+1\}$, which has at least $1/3$ of the elements. Choose a random $x\in\F^\times$, and consider $xA\cap C$. The average size of this set is at least $|A|/3$, and it is always sum-free!
\end{proof}
\end{INT}
\item For $\calF$ a family of subsets of $\{1,\ldots,n\}$, let $d(\calF)$ be the number of unordered \textbf{pairs of disjoint elements} of $\calF$.
\begin{INT}[Thm 1.5.1:]
Let $\calF$ be a family of $m=2^{(1/2+\delta)n}$ subsets of $\{1,\ldots,n\}$. Then $d(\calF)<m^{2-\delta^2/2}$.
\begin{proof}
Suppose for contradiction that $d(\calF)\geq m^{2-\delta^2/2}$. Choose an integer $t$, to be determined. Choose $A_1,\ldots,A_t$ independently and with repetition from $\calF$. The point is that it will be likely that $U=\bigcup A_i$ has at least $n/2$ elements, and that there are at least $2^{n/2}$ distinct subsets of $X$ disjoint from $U$, a contradiction. For the first, it's really unlikely that $U$ will be contained in any given subset $S$ of size $n/2$, so:
\[\Prob[|U|\leq n/2]\leq \sum_{|S|=n/2}\Prob[\text{all $A_i\subset S$}]\leq 2^n\left(\frac{2^{n/2}}{m}\right)^t=2^{n(1-\delta t)}.\]
Note: One has that $A_i$ could be any of $m$ options, and at most $2^{n/2}$ of these lie in each $S$. We used a very clumsy approximation $2^n\geq |\{|S|=n/2\}|$.

\INDENT Now to the other side. We have a function $v:\calF\to\N$ which maps $B$ to the number of disjoint pairs its a member of. Then $v$ takes average value $2d(\calF)/m\geq2m^{1-\delta^2/2}$. Let $Y$ be the r.v.\ returning the number of $B\in \calF$ disjoint to $U$. Then we want to see that $Y$ often exceeds $2^{n/2}$. Applying convexity:
\[E[Y]=
%\sum\left(\frac{v(b)}{m}\right)^t=
m^{1-t}\left(\frac{\sum v(b)^t}{m}\right)\geq m^{1-t}\left(2m^{1-\delta^2/2}\right)^t=2^tm^{1-t\delta^2/2}=2^{t+(1/2+\delta)n(1-t\delta^2/2)}.\]
We'd like to bound $p=\Prob[Y\geq 2^n/2]$. As $Y<m$, we've got an inequality
\[pm+(1-p)2^{n/2}\geq E[Y].\]
This shows that $p$ is pretty big, and in fact, for some good $t$, bigger than $2^{n(1-\delta t)}$.
\end{proof}
\end{INT}
\end{itemise}
\subsection*{The Probabilistic Lens: The Erd\H{o}s-Ko-Rado theorem}
A family $\calF$ of sets is \textbf{intersecting} if any $A,B\in\calF$ satisfy $A\cap B\neq\emptyset$. We are interested in large intersecting families of $k$-element subsets of $\{0,\ldots, n-1\}$ (where $k\leq n/2$). One option is the set of $k$-element subsets containing a given element.
\begin{INT}[Erd\H{o}s-Ko-Rado:]
This is best possible --- any intersecting family of $k$-element subsets of an $n$-element set has size at most ${n-1\choose k-1}$.
\begin{proof}
Suppose $\calF$ is such. View $\{0,\ldots, n-1\}$ as $\Z/n\Z$. Let $A_0=\{0,\ldots,k-1\}$ and define translates $A_s=A_0+s$. Then $\calF$ contains at most $k$ of the $A_s$. [Suppose $A_0\in\calF$, the other $A_t$ intersecting can be arranged in $k-1$ pairs which are disjoint.]

Now for any $k$-element subset $A$ of $\Z/n\Z$ and a random $i\in \Z/n\Z$, the probability that $A+i\in\calF$ is at most $k/n$. Thus, if we randomise $A$ uniformly amoungst the $k$-sets, the probability that $A+i\in\calF$ is still at most $k/n$. Now $A+i$ is uniformly chosen from amoungst the $k$-sets, so that the probability a random $k$-element subset is in $\calF$ is at most $k/n$, giving the result.
\end{proof}
\end{INT}
%  Suppose $\calF$ is an intersecting family of 
\end{chapter1}

\begin{chapter2}
\section*{\S2: Linearity of expectation}
\begin{itemise}
\item A \textbf{hamiltonian path} is a path in a graph which visits each vertex exactly once. A \textbf{hamiltonian cycle} is such which returns to its starting vertex.
\begin{INT}[Thm 2.1.1:]
There's a tournament $T$ with $n$ players and at least $n!2^{-(n-1)}$ Hamiltonian paths.
\begin{proof}
For this is the expected number of Hamiltonian paths in a random Tournament.
\end{proof}
\end{INT}
\item A \textbf{bipartite graph} is one in which the vertices can be partitioned into two disjoint subsets, such that the induced subgraph on each has no edges.
\begin{INT}[Thm 2.2.1:]
Any graph with $e$ edges contains a bipartite subgraph (not necessarily complete) with at least $e/2$ edges.
\begin{proof}
For each vertex, randomly assign it to either $A$ or $B$, to give a random partition. On average, there should be $e/2$ crossing edges.
\end{proof}
\moreINT[Thm 2.2.2:]
If $G$ has $2n$ vertices and $e$ edges, then it contains a bipartite subgraph with at least $en/(2n-1)$ edges. [Similarly, get $e(n+1)/(2n+1)$ edges given $2n+1$ vertices.]
\begin{proof}
Choose $A$ uniformly from all $n$-element subsets of $V$. Same deal.
\end{proof}
\end{INT}
\item \rednote{There's something nasty which I don't get about 2.2.3 --- what's a $k$-set?}
\item Let $K_{a,b}$ be the graph $(A\sqcup B,\text{``$A\times B$''})$ where $|A|=a$ and $|B|=b$.
\begin{INT}[Thm 2.3.1:]
There's a 2-colouring of $K_n$ with at most ${n\choose a}2^{1-{a\choose 2}}$ monochromatic $K_a$.
\begin{proof}
This is the expected number of monochromatics in a random 2-colouring.
\end{proof}
\moreINT[Thm 2.3.2:] Similarly, there's a 2-colouring of $K_n$ with at most ${m\choose a}{n\choose b}2^{1-ab}$ monochromatic $K_{a,b}$.
\end{INT}
\item \textbf{Balancing vectors:}
\begin{INT}[Thm 2.4.1:]
Suppose that $v_1,\ldots,v_n\in\R^n$ all have length one. Then there is a signed sum $\pm\epsilon_1 v_1\pm\epsilon_2v_2\pm\cdots\pm\epsilon_n v_n$ of length at least (and one with length at most) $\sqrt n$.
\begin{proof}
For one can calculate that the average norm-squared random signed sum is $n$.
\end{proof}
\moreINT[Thm 2.4.2:]
Let $v_1,\ldots,v_n\in\R^n$ have length at most one. Fix weights $p_1,\ldots,p_n\in[0,1]$, and let $w=\sum p_iv_i$ be the weighted sum. Then there exists $S\subset \{1,\ldots,n\}$ with $|w-\sum_{i\in S} v_i|\leq\sqrt n/2$.
\begin{proof}
Let $\epsilon_i$ be one with probability $p_i$, else zero. Let $v=\sum \epsilon_iv_i$, and let $X=|w-v|^2$. The expectation of $X$ is calculated just like the variance of $\sum\epsilon_i$, with a bunch of inner products multiplied in, but all these inner products are less than one in magnitude, so $\Exp[X]\leq n/4$.
\end{proof}
\end{INT}
\item \textbf{Unbalancing lights:} Given an $n\times n$ array of lights, some on and some off, and $2n$ switches, one for each row and column, that toggle each light therein. Then it's possible to flick switches in order to ensure that there are $(\sqrt{2/\pi}+o(1))n^{3/2}$ more lights on than off.
\begin{INT}
%Let $a_{ij}=\pm1$ depending whether the corresponding light is on or off. We want to maximise $\sum_{ij}a_{ij}x_iy_j$ for $x_i,y_j$ all $\pm1$, and claim that this quantity can be at least $(\sqrt{2/\pi}+o(1))n^{3/2}$.
\begin{proof}
Randomly flick the switches for each column. Let $R_i$ be the difference of the number on and off in the $i^\text{th}$ row. This is has distribution $S_n$ --- the sum of $n$ iid uniform $\{\pm1\}$ random variables. Now $\Exp[|S_n|]=(\sqrt{2/\pi}+o(1))\sqrt n$. Thus $\Exp[\sum|R_i|]$ is the quantity we desire.
\end{proof}
\end{INT}
\end{itemise}
\subsection*{The Probabilistic Lens: Br\'egman's theorem}
The \textbf{permanent} of a matrix is the same sum as the determinant, taken without the signs. We want to bound the permanent of an $n\times n$ $\{0,1\}$-matrix $A$ with $r_i$ ones in the $i^\text{th}$ row. Note that one might be interested in doing this as the permanent of the matrix is the number of cycle coverings in the digraph associated with $A$.
\begin{INT}[Br\'egman's theorem:]
$\text{per}(A)\leq \prod (r_i!)^{1/r_i}$.
\begin{proof}
To solve this problem, we use the \textbf{geometric mean}. That is, if $Y$ is a random variable, we define $\GeomMean(Y):=\exp(\Exp(\log Y))$. Note that $\GeomMean(XY)=\GeomMean(X)\GeomMean(Y)$ even when $X$ and $Y$ are not independent. Note that if $Y$ takes values $a_i$ with probabilities $p_i$, we have $\GeomMean(Y)=\prod a_i^{p_i}$. Just as $\Exp(X)$ is the mean of the conditional expectations, $\GeomMean(Y)$ is the geometric mean of the conditional geometric means.

\INDENT Define an ordered transversal of $A$ to be a pair of permutations $(\alpha,\beta)$ such that $A_{\alpha(i),\beta(i)}=1$ for all $i$. That is, a sequence of ones contributing one to the permanent, taken with an order. Define a transversal to be a permutation $\gamma$ such that $A_{i,\gamma(i)}=1$. Let $\calO$ be the set of ordered transversals, and let $\calT$ be the set of transversals. There's a  function $\pi:\calO\to\calT$ whose fibres each have order $n!$.


\INDENT Choose $o=(\alpha,\beta)$ randomly from $\calO$. For $i=1,\ldots, n$, remove the $\alpha(i)^\text{th}$ row and the $\beta(i)^\text{th}$ column (containing the $i^\text{th}$ one in the ordering), and let $R_i$ be the number of ones remaining in the $\alpha(i)^\text{th}$ row just before it is removed. Define $L=\prod R_i$. The random variable $L$ should overestimate $\text{per}(A)$, in the sense that $\text{per}(A)\leq\GeomMean(L)$. \textbf{Come back and prove this.} We calculate:
\begin{alignat*}{2}
\GeomMean[L|\pi(o)=\sigma]&=\prod\GeomMean[R_i|\pi(o)=\sigma]%&\qquad&\text{()}
\\&=\prod(r_i)^{1/r_i}(r_i-1)^{1/r_i}\cdots(1)^{1/r_i}&\qquad&\text{($R_i$ is uniform on $\{1,\ldots, r_i\}$)}
\\&=\prod(r_i!)^{1/r_i}&\qquad&\text{which is independent of $\sigma$, so}
\\\GeomMean[L]&=\prod(r_i!)^{1/r_i}&\qquad&\text{(each $\sigma$ is equally likely)}\qedhere
\end{alignat*}
\end{proof}
\end{INT}
\end{chapter2}
\begin{chapter3}
\section*{\S3: Alterations}
\begin{itemise}
\item Recall that $R(k,l)>n$ \Iff there is a two-colouring of the edges of $K_n$ with neither a red $K_k$ nor blue $K_l$.
\begin{INT}[Thm 3.1.1:]
For any integer $n$, $R(k,k)>n-{n\choose k}2^{1-{k\choose2}}$.
\begin{proof}
For choosing a random two-colouring of $K_n$, ${n\choose k}2^{1-{k\choose2}}$ is the average number of monochromatic $K_k$. There's some colouring with at most this many; remove a vertex from each.
\end{proof}
There are a couple of analogous results on off-diagonal Ramsey numbers.
\end{INT}
\item The \textbf{independence number} $\alpha(G)$ of a graph $G$ is the maximal size of a collection of vertices of $G$ sharing no edges.
\begin{INT}[Thm 3.2.1:]
Let $G$ have $n$ vertices and $nd/2$ edges ($d\geq1$). Then $\alpha(G)\geq n/(2d)$.
\moreINT[Restatement:] In $G$ has $v$ vertices and $e$ edges with $e\geq n/2$, then $\alpha(G)\geq (n/2)^2/e$.
\begin{proof}
Assign each vertex independently a probability of $p$ on being put in a set $S$. Then the average number of elements of $S$ is $pn$, and the average number of edges in the induced subgraph is $p^2e$. We thus obtain a randomised method of returning an independent subset of $G$ with an average of $pn-p^2e$ elements. This quantity is maximised at $p=n/(2e)\in[0,1]$.
\end{proof}
\end{INT}
\item \textbf{Combinatorial geometry:} We are interested in finding ways to distribute $n$ points in a unit square such that all of the triangles that can be formed from the $n$ points are large in area. For $S\subset I^2$, let $T(S)$ be the area of the smallest triangle with corners in $S$, and let $T(n)$ be the largest value of $T(S)$ as $S$ ranges over subsets of $I^2$ of size $n$.

\begin{INT}[Thm 3.3.1:]
There is a set of $n$ points is the unit square such that $T(S)\geq1/(10n)^2$, so that $T(n)=\Omega(n^{-2})$.
\begin{proof}
For any $\epsilon>0$, suppose that $P,Q,R$ are chosen uniformly from $I^2$. Then it is an easy exercise to bound $\Prob[\text{Area}(PQR)\leq\epsilon]\leq16\pi\epsilon$.

\INDENT Now choose at random $2n$ points in $I^2$. Then expected number of triangles with area less that $1/(10n)^2$ is at most ${2n\choose3}\cdot16\pi/(10n)^2<n$. There is a configuration with less than $n$ small triangles, and we delete a vertex from each of these.
\end{proof}
\moreINT[A construction of Erd\H{o}s:] $T(n)\geq1/(2(n-1)^2)$ when $n$ is prime.

\begin{proof} Graph the parabola $y=x^2$ in $(\Z/n\Z)^2$ after embedding $(\Z/n\Z)^2$ in $[0,n-1]^2$ as the lattice points. As $n$ is prime, no three points are collinear, and each triangle has integer coordinates, so has area a multiple of $1/2$.
\end{proof}
\end{INT}
Note that in fact, $T(n)=\Omega(\log n/n^2)$.
\item Given a bounded measurable subset $C\subset \R^d$, the \textbf{packing number} of $X$ is 
\[\delta(C):=\mu(C)\lim_{x\rightarrow\infty} f(x)/x^d\]
where $f(x)$ is the largest number of copies of $C$ that fit inside $[0,x]^d$.
\begin{INT}[Thm 3.4.1:]
Let $C$ be convex and centrally symmetric around the origin. Then $\delta(C)\geq2^{-d-1}$.
\begin{proof}
We'd like to choose the centres of the copies of $C$ randomly inside the box $B=[0,x]^d$. Choosing $p,q\in B$ at random, $(C+p)$ intersects $(C+q)$ \Iff $p-q\in2C$, by central symmetry. Conditioning on fixed $q$, this occurs with probability at most $\mu(2C)/x^{d}$, and thus the unconditional probability of an intersection is at most $(2/x)^d\mu(C)$.

\INDENT Now choose $n$ points in $B$ at random, and put a copy of $C$ at each. Then average number of intersections is at most $(n^2/2)(2/x)^d\mu(C)$, so removing some, we obtain $n-(n^2/2)(2/x)^d\mu(C)$ points. Choose $n$ to maximise this quantity, and note that the fact that some of the copies may hang over the side of the box is insignificant in the limit.
\end{proof}
\moreINT[A greedy algorithm:] By simply adding new copies of $C$ until we can no longer go on, we see that $\delta(C)\geq 2^{-d}$.
\begin{proof}
Note that if we choose $p_1,\ldots,p_m$ a maximal subset of $B$ such that the $C+p_i$ are disjoint. Then by maximality, there can be no point $q$ such that $q-p_i\notin 2c$ for all $i$ (else we can add $q$ to the collection). Thus $n$ copies of $2C$ cover $B$, from which we can derive the result.
\end{proof}
\end{INT}
\item \textbf{Recolouring:}  recall that we wrote $m(n)>m$ \Iff every $n$-uniform hypergraph with $m$ edges is \textbf{2-colourable}, i.e.\ has \textbf{property B}.
\begin{INT}[Thm 3.5.1:]
If there is a probability $p$ such that $k(1-p)^n+k^2p<1$, then $m(n)>2^{n-1}k$.
\moreINT[Cor 3.5.2:] $m(n)=\Omega(2^n\sqrt{n/\log n})$.
\begin{proof}[Proof of 3.5.2:]
It's good enough to find the largest $p$ such that $k^2p<1/2$ and then plug it into $k(1-p)^n<1/2$, to see that $k$ satisfying the hypotheses of 3.5.1 can be chosen fairly large.
\end{proof}
\begin{proof}[Proof of 3.5.1:]
It's enough to fix a hypergraph with $2^{n-k}$ edges, and prove that its vertices can be 2-coloured.
We give three sources of randomness. For each vertex, we flip a fair `red'/`blue' coin and a $(p,1-p)$ `recolour'/`don't recolour' coin. We also randomly order the vertices.

\INDENT The idea is as follows. Colour the vertices according to their red/blue coins. Then go through the vertices in their (random) order and recolour the coin in question \Iff it lies in a monochromatic edge and its second coin displays `recolour'. The point is that we can bound the failure probability by $k(1-p)^n+k^2p$.
\end{proof}





\end{INT}
\end{itemise}
\end{chapter3}
\end{document}























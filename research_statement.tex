% !TEX root = z_output/_research_statement.tex
\documentclass[11pt]{article}
\usepackage{fullpage}
\usepackage{amsmath,amsthm,amssymb}
\usepackage{mathrsfs,nicefrac}
\usepackage{amssymb}
\usepackage{epsfig}
\usepackage[all,2cell]{xy}
\usepackage{sseq}
\usepackage{tocloft}
\usepackage{cancel}
\usepackage[strict]{changepage}
\usepackage{color}
\usepackage{tikz}
\usepackage{extpfeil}
\usepackage{version}
\usepackage{framed}
\definecolor{shadecolor}{rgb}{.925,0.925,0.925}

%\usepackage{ifthen}
%Used for disabling hyperref
\ifx\dontloadhyperref\undefined
%\usepackage[pdftex,pdfborder={0 0 0 [1 1]}]{hyperref}
\usepackage[pdftex,pdfborder={0 0 .5 [1 1]}]{hyperref}
\else
\providecommand{\texorpdfstring}[2]{#1}
\fi
%>>>>>>>>>>>>>>>>>>>>>>>>>>>>>>
%<<<        Versions        <<<
%>>>>>>>>>>>>>>>>>>>>>>>>>>>>>>
%Add in the following line to include all the versions.
%\def\excludeversion#1{\includeversion{#1}}

%>>>>>>>>>>>>>>>>>>>>>>>>>>>>>>
%<<<       Better ToC       <<<
%>>>>>>>>>>>>>>>>>>>>>>>>>>>>>>
\setlength{\cftbeforesecskip}{0.5ex}

%>>>>>>>>>>>>>>>>>>>>>>>>>>>>>>
%<<<      Hyperref mod      <<<
%>>>>>>>>>>>>>>>>>>>>>>>>>>>>>>

%needs more testing
\newcounter{dummyforrefstepcounter}
\newcommand{\labelRIGHTHERE}[1]
{\refstepcounter{dummyforrefstepcounter}\label{#1}}


%>>>>>>>>>>>>>>>>>>>>>>>>>>>>>>
%<<<  Theorem Environments  <<<
%>>>>>>>>>>>>>>>>>>>>>>>>>>>>>>
\ifx\dontloaddefinitionsoftheoremenvironments\undefined
\theoremstyle{plain}
\newtheorem{thm}{Theorem}[section]
\newtheorem*{thm*}{Theorem}
\newtheorem{lem}[thm]{Lemma}
\newtheorem*{lem*}{Lemma}
\newtheorem{prop}[thm]{Proposition}
\newtheorem*{prop*}{Proposition}
\newtheorem{cor}[thm]{Corollary}
\newtheorem*{cor*}{Corollary}
\newtheorem{defprop}[thm]{Definition-Proposition}
\newtheorem*{punchline}{Punchline}
\newtheorem*{conjecture}{Conjecture}
\newtheorem*{claim}{Claim}

\theoremstyle{definition}
\newtheorem{defn}{Definition}[section]
\newtheorem*{defn*}{Definition}
\newtheorem{exmp}{Example}[section]
\newtheorem*{exmp*}{Example}
\newtheorem*{exmps*}{Examples}
\newtheorem*{nonexmp*}{Non-example}
\newtheorem{asspt}{Assumption}[section]
\newtheorem{notation}{Notation}[section]
\newtheorem{exercise}{Exercise}[section]
\newtheorem*{fact*}{Fact}
\newtheorem*{rmk*}{Remark}
\newtheorem{fact}{Fact}
\newtheorem*{aside}{Aside}
\newtheorem*{question}{Question}
\newtheorem*{answer}{Answer}

\else\relax\fi

%>>>>>>>>>>>>>>>>>>>>>>>>>>>>>>
%<<<      Fields, etc.      <<<
%>>>>>>>>>>>>>>>>>>>>>>>>>>>>>>
\DeclareSymbolFont{AMSb}{U}{msb}{m}{n}
\DeclareMathSymbol{\N}{\mathbin}{AMSb}{"4E}
\DeclareMathSymbol{\Octonions}{\mathbin}{AMSb}{"4F}
\DeclareMathSymbol{\Z}{\mathbin}{AMSb}{"5A}
\DeclareMathSymbol{\R}{\mathbin}{AMSb}{"52}
\DeclareMathSymbol{\Q}{\mathbin}{AMSb}{"51}
\DeclareMathSymbol{\PP}{\mathbin}{AMSb}{"50}
\DeclareMathSymbol{\I}{\mathbin}{AMSb}{"49}
\DeclareMathSymbol{\C}{\mathbin}{AMSb}{"43}
\DeclareMathSymbol{\A}{\mathbin}{AMSb}{"41}
\DeclareMathSymbol{\F}{\mathbin}{AMSb}{"46}
\DeclareMathSymbol{\G}{\mathbin}{AMSb}{"47}
\DeclareMathSymbol{\Quaternions}{\mathbin}{AMSb}{"48}


%>>>>>>>>>>>>>>>>>>>>>>>>>>>>>>
%<<<       Operators        <<<
%>>>>>>>>>>>>>>>>>>>>>>>>>>>>>>
\DeclareMathOperator{\ad}{\textbf{ad}}
\DeclareMathOperator{\coker}{coker}
\renewcommand{\ker}{\textup{ker}\,}
\DeclareMathOperator{\End}{End}
\DeclareMathOperator{\Aut}{Aut}
\DeclareMathOperator{\Hom}{Hom}
\DeclareMathOperator{\Maps}{Maps}
\DeclareMathOperator{\Mor}{Mor}
\DeclareMathOperator{\Gal}{Gal}
\DeclareMathOperator{\Ext}{Ext}
\DeclareMathOperator{\Tor}{Tor}
\DeclareMathOperator{\Map}{Map}
\DeclareMathOperator{\Der}{Der}
\DeclareMathOperator{\Rad}{Rad}
\DeclareMathOperator{\rank}{rank}
\DeclareMathOperator{\ArfInvariant}{Arf}
\DeclareMathOperator{\KervaireInvariant}{Ker}
\DeclareMathOperator{\im}{im}
\DeclareMathOperator{\coim}{coim}
\DeclareMathOperator{\trace}{tr}
\DeclareMathOperator{\supp}{supp}
\DeclareMathOperator{\ann}{ann}
\DeclareMathOperator{\spec}{Spec}
\DeclareMathOperator{\SPEC}{\textbf{Spec}}
\DeclareMathOperator{\proj}{Proj}
\DeclareMathOperator{\PROJ}{\textbf{Proj}}
\DeclareMathOperator{\fiber}{F}
\DeclareMathOperator{\cofiber}{C}
\DeclareMathOperator{\cone}{cone}
\DeclareMathOperator{\skel}{sk}
\DeclareMathOperator{\coskel}{cosk}
\DeclareMathOperator{\conn}{conn}
\DeclareMathOperator{\colim}{colim}
\DeclareMathOperator{\limit}{lim}
\DeclareMathOperator{\ch}{ch}
\DeclareMathOperator{\Vect}{Vect}
\DeclareMathOperator{\GrthGrp}{GrthGp}
\DeclareMathOperator{\Sym}{Sym}
\DeclareMathOperator{\Prob}{\mathbb{P}}
\DeclareMathOperator{\Exp}{\mathbb{E}}
\DeclareMathOperator{\GeomMean}{\mathbb{G}}
\DeclareMathOperator{\Var}{Var}
\DeclareMathOperator{\Cov}{Cov}
\DeclareMathOperator{\Sp}{Sp}
\DeclareMathOperator{\Seq}{Seq}
\DeclareMathOperator{\Cyl}{Cyl}
\DeclareMathOperator{\Ev}{Ev}
\DeclareMathOperator{\sh}{sh}
\DeclareMathOperator{\intHom}{\underline{Hom}}
\DeclareMathOperator{\Frac}{frac}



%>>>>>>>>>>>>>>>>>>>>>>>>>>>>>>
%<<<   Cohomology Theories  <<<
%>>>>>>>>>>>>>>>>>>>>>>>>>>>>>>
\DeclareMathOperator{\KR}{{K\R}}
\DeclareMathOperator{\KO}{{KO}}
\DeclareMathOperator{\K}{{K}}
\DeclareMathOperator{\OmegaO}{{\Omega_{\Octonions}}}

%>>>>>>>>>>>>>>>>>>>>>>>>>>>>>>
%<<<   Algebraic Geometry   <<<
%>>>>>>>>>>>>>>>>>>>>>>>>>>>>>>
\DeclareMathOperator{\Spec}{Spec}
\DeclareMathOperator{\Proj}{Proj}
\DeclareMathOperator{\Sing}{Sing}
\DeclareMathOperator{\shfHom}{\mathscr{H}\textit{\!\!om}}
\DeclareMathOperator{\WeilDivisors}{{Div}}
\DeclareMathOperator{\CartierDivisors}{{CaDiv}}
\DeclareMathOperator{\PrincipalWeilDivisors}{{PrDiv}}
\DeclareMathOperator{\LocallyPrincipalWeilDivisors}{{LPDiv}}
\DeclareMathOperator{\PrincipalCartierDivisors}{{PrCaDiv}}
\DeclareMathOperator{\DivisorClass}{{Cl}}
\DeclareMathOperator{\CartierClass}{{CaCl}}
\DeclareMathOperator{\Picard}{{Pic}}
\DeclareMathOperator{\Frob}{Frob}


%>>>>>>>>>>>>>>>>>>>>>>>>>>>>>>
%<<<  Mathematical Objects  <<<
%>>>>>>>>>>>>>>>>>>>>>>>>>>>>>>
\newcommand{\sll}{\mathfrak{sl}}
\newcommand{\gl}{\mathfrak{gl}}
\newcommand{\GL}{\mbox{GL}}
\newcommand{\PGL}{\mbox{PGL}}
\newcommand{\SL}{\mbox{SL}}
\newcommand{\Mat}{\mbox{Mat}}
\newcommand{\Gr}{\textup{Gr}}
\newcommand{\Squ}{\textup{Sq}}
\newcommand{\catSet}{\textit{Sets}}
\newcommand{\RP}{{\R\PP}}
\newcommand{\CP}{{\C\PP}}
\newcommand{\Steen}{\mathscr{A}}
\newcommand{\Orth}{\textup{\textbf{O}}}

%>>>>>>>>>>>>>>>>>>>>>>>>>>>>>>
%<<<  Mathematical Symbols  <<<
%>>>>>>>>>>>>>>>>>>>>>>>>>>>>>>
\newcommand{\DASH}{\textup{---}}
\newcommand{\op}{\textup{op}}
\newcommand{\CW}{\textup{CW}}
\newcommand{\ob}{\textup{ob}\,}
\newcommand{\ho}{\textup{ho}}
\newcommand{\st}{\textup{st}}
\newcommand{\id}{\textup{id}}
\newcommand{\Bullet}{\ensuremath{\bullet} }
\newcommand{\sprod}{\wedge}

%>>>>>>>>>>>>>>>>>>>>>>>>>>>>>>
%<<<      Some Arrows       <<<
%>>>>>>>>>>>>>>>>>>>>>>>>>>>>>>
\newcommand{\nt}{\Longrightarrow}
\let\shortmapsto\mapsto
\let\mapsto\longmapsto
\newcommand{\mapsfrom}{\,\reflectbox{$\mapsto$}\ }
\newcommand{\bigrightsquig}{\scalebox{2}{\ensuremath{\rightsquigarrow}}}
\newcommand{\bigleftsquig}{\reflectbox{\scalebox{2}{\ensuremath{\rightsquigarrow}}}}

%\newcommand{\cofibration}{\xhookrightarrow{\phantom{\ \,{\sim\!}\ \ }}}
%\newcommand{\fibration}{\xtwoheadrightarrow{\phantom{\sim\!}}}
%\newcommand{\acycliccofibration}{\xhookrightarrow{\ \,{\sim\!}\ \ }}
%\newcommand{\acyclicfibration}{\xtwoheadrightarrow{\sim\!}}
%\newcommand{\leftcofibration}{\xhookleftarrow{\phantom{\ \,{\sim\!}\ \ }}}
%\newcommand{\leftfibration}{\xtwoheadleftarrow{\phantom{\sim\!}}}
%\newcommand{\leftacycliccofibration}{\xhookleftarrow{\ \ {\sim\!}\,\ }}
%\newcommand{\leftacyclicfibration}{\xtwoheadleftarrow{\sim\!}}
%\newcommand{\weakequiv}{\xrightarrow{\ \,\sim\,\ }}
%\newcommand{\leftweakequiv}{\xleftarrow{\ \,\sim\,\ }}

\newcommand{\cofibration}
{\xhookrightarrow{\phantom{\ \,{\raisebox{-.3ex}[0ex][0ex]{\scriptsize$\sim$}\!}\ \ }}}
\newcommand{\fibration}
{\xtwoheadrightarrow{\phantom{\raisebox{-.3ex}[0ex][0ex]{\scriptsize$\sim$}\!}}}
\newcommand{\acycliccofibration}
{\xhookrightarrow{\ \,{\raisebox{-.55ex}[0ex][0ex]{\scriptsize$\sim$}\!}\ \ }}
\newcommand{\acyclicfibration}
{\xtwoheadrightarrow{\raisebox{-.6ex}[0ex][0ex]{\scriptsize$\sim$}\!}}
\newcommand{\leftcofibration}
{\xhookleftarrow{\phantom{\ \,{\raisebox{-.3ex}[0ex][0ex]{\scriptsize$\sim$}\!}\ \ }}}
\newcommand{\leftfibration}
{\xtwoheadleftarrow{\phantom{\raisebox{-.3ex}[0ex][0ex]{\scriptsize$\sim$}\!}}}
\newcommand{\leftacycliccofibration}
{\xhookleftarrow{\ \ {\raisebox{-.55ex}[0ex][0ex]{\scriptsize$\sim$}\!}\,\ }}
\newcommand{\leftacyclicfibration}
{\xtwoheadleftarrow{\raisebox{-.6ex}[0ex][0ex]{\scriptsize$\sim$}\!}}
\newcommand{\weakequiv}
{\xrightarrow{\ \,\raisebox{-.3ex}[0ex][0ex]{\scriptsize$\sim$}\,\ }}
\newcommand{\leftweakequiv}
{\xleftarrow{\ \,\raisebox{-.3ex}[0ex][0ex]{\scriptsize$\sim$}\,\ }}

%>>>>>>>>>>>>>>>>>>>>>>>>>>>>>>
%<<<    xymatrix Arrows     <<<
%>>>>>>>>>>>>>>>>>>>>>>>>>>>>>>
\newdir{ >}{{}*!/-5pt/@{>}}
\newcommand{\xycof}{\ar@{ >->}}
\newcommand{\xycofib}{\ar@{^{(}->}}
\newcommand{\xycofibdown}{\ar@{_{(}->}}
\newcommand{\xyfib}{\ar@{->>}}
\newcommand{\xymapsto}{\ar@{|->}}

%>>>>>>>>>>>>>>>>>>>>>>>>>>>>>>
%<<<     Greek Letters      <<<
%>>>>>>>>>>>>>>>>>>>>>>>>>>>>>>
%\newcommand{\oldphi}{\phi}
%\renewcommand{\phi}{\varphi}
\let\oldphi\phi
\let\phi\varphi
\renewcommand{\to}{\longrightarrow}
\newcommand{\from}{\longleftarrow}
\newcommand{\eps}{\varepsilon}

%>>>>>>>>>>>>>>>>>>>>>>>>>>>>>>
%<<<  1st-4th & parentheses <<<
%>>>>>>>>>>>>>>>>>>>>>>>>>>>>>>
\newcommand{\first}{^\text{st}}
\newcommand{\second}{^\text{nd}}
\newcommand{\third}{^\text{rd}}
\newcommand{\fourth}{^\text{th}}
\newcommand{\ZEROTH}{$0^\text{th}$ }
\newcommand{\FIRST}{$1^\text{st}$ }
\newcommand{\SECOND}{$2^\text{nd}$ }
\newcommand{\THIRD}{$3^\text{rd}$ }
\newcommand{\FOURTH}{$4^\text{th}$ }
\newcommand{\iTH}{$i^\text{th}$ }
\newcommand{\jTH}{$j^\text{th}$ }
\newcommand{\nTH}{$n^\text{th}$ }

%>>>>>>>>>>>>>>>>>>>>>>>>>>>>>>
%<<<    upright commands    <<<
%>>>>>>>>>>>>>>>>>>>>>>>>>>>>>>
\newcommand{\upcol}{\textup{:}}
\newcommand{\upsemi}{\textup{;}}
\providecommand{\lparen}{\textup{(}}
\providecommand{\rparen}{\textup{)}}
\renewcommand{\lparen}{\textup{(}}
\renewcommand{\rparen}{\textup{)}}
\newcommand{\Iff}{\emph{iff} }

%>>>>>>>>>>>>>>>>>>>>>>>>>>>>>>
%<<<     Environments       <<<
%>>>>>>>>>>>>>>>>>>>>>>>>>>>>>>
\newcommand{\squishlist}
{ %\setlength{\topsep}{100pt} doesn't seem to do anything.
  \setlength{\itemsep}{.5pt}
  \setlength{\parskip}{0pt}
  \setlength{\parsep}{0pt}}
\newenvironment{itemise}{
\begin{list}{\textup{$\rightsquigarrow$}}
   {  \setlength{\topsep}{1mm}
      \setlength{\itemsep}{1pt}
      \setlength{\parskip}{0pt}
      \setlength{\parsep}{0pt}
   }
}{\end{list}\vspace{-.1cm}}
\newcommand{\INDENT}{\textbf{}\phantom{space}}
\renewcommand{\INDENT}{\rule{.7cm}{0cm}}

\newcommand{\itm}[1][$\rightsquigarrow$]{\item[{\makebox[.5cm][c]{\textup{#1}}}]}


%\newcommand{\rednote}[1]{{\color{red}#1}\makebox[0cm][l]{\scalebox{.1}{rednote}}}
%\newcommand{\bluenote}[1]{{\color{blue}#1}\makebox[0cm][l]{\scalebox{.1}{rednote}}}

\newcommand{\rednote}[1]
{{\color{red}#1}\makebox[0cm][l]{\scalebox{.1}{\rotatebox{90}{?????}}}}
\newcommand{\bluenote}[1]
{{\color{blue}#1}\makebox[0cm][l]{\scalebox{.1}{\rotatebox{90}{?????}}}}


\newcommand{\funcdef}[4]{\begin{align*}
#1&\to #2\\
#3&\mapsto#4
\end{align*}}

%\newcommand{\comment}[1]{}

%>>>>>>>>>>>>>>>>>>>>>>>>>>>>>>
%<<<       Categories       <<<
%>>>>>>>>>>>>>>>>>>>>>>>>>>>>>>
\newcommand{\Ens}{{\mathscr{E}ns}}
\DeclareMathOperator{\Sheaves}{{\mathsf{Shf}}}
\DeclareMathOperator{\Presheaves}{{\mathsf{PreShf}}}
\DeclareMathOperator{\Psh}{{\mathsf{Psh}}}
\DeclareMathOperator{\Shf}{{\mathsf{Shf}}}
\DeclareMathOperator{\Varieties}{{\mathsf{Var}}}
\DeclareMathOperator{\Schemes}{{\mathsf{Sch}}}
\DeclareMathOperator{\Rings}{{\mathsf{Rings}}}
\DeclareMathOperator{\AbGp}{{\mathsf{AbGp}}}
\DeclareMathOperator{\Modules}{{\mathsf{\!-Mod}}}
\DeclareMathOperator{\fgModules}{{\mathsf{\!-Mod}^{\textup{fg}}}}
\DeclareMathOperator{\QuasiCoherent}{{\mathsf{QCoh}}}
\DeclareMathOperator{\Coherent}{{\mathsf{Coh}}}
\DeclareMathOperator{\GSW}{{\mathcal{SW}^G}}
\DeclareMathOperator{\Burnside}{{\mathsf{Burn}}}
\DeclareMathOperator{\GSet}{{G\mathsf{Set}}}
\DeclareMathOperator{\FinGSet}{{G\mathsf{Set}^\textup{fin}}}
\DeclareMathOperator{\HSet}{{H\mathsf{Set}}}
\DeclareMathOperator{\Cat}{{\mathsf{Cat}}}
\DeclareMathOperator{\Fun}{{\mathsf{Fun}}}
\DeclareMathOperator{\Orb}{{\mathsf{Orb}}}
\DeclareMathOperator{\Set}{{\mathsf{Set}}}
\DeclareMathOperator{\sSet}{{\mathsf{sSet}}}
\DeclareMathOperator{\Top}{{\mathsf{Top}}}
\DeclareMathOperator{\GSpectra}{{G-\mathsf{Spectra}}}
\DeclareMathOperator{\Lan}{Lan}
\DeclareMathOperator{\Ran}{Ran}

%>>>>>>>>>>>>>>>>>>>>>>>>>>>>>>
%<<<     Script Letters     <<<
%>>>>>>>>>>>>>>>>>>>>>>>>>>>>>>
\newcommand{\scrQ}{\mathscr{Q}}
\newcommand{\scrW}{\mathscr{W}}
\newcommand{\scrE}{\mathscr{E}}
\newcommand{\scrR}{\mathscr{R}}
\newcommand{\scrT}{\mathscr{T}}
\newcommand{\scrY}{\mathscr{Y}}
\newcommand{\scrU}{\mathscr{U}}
\newcommand{\scrI}{\mathscr{I}}
\newcommand{\scrO}{\mathscr{O}}
\newcommand{\scrP}{\mathscr{P}}
\newcommand{\scrA}{\mathscr{A}}
\newcommand{\scrS}{\mathscr{S}}
\newcommand{\scrD}{\mathscr{D}}
\newcommand{\scrF}{\mathscr{F}}
\newcommand{\scrG}{\mathscr{G}}
\newcommand{\scrH}{\mathscr{H}}
\newcommand{\scrJ}{\mathscr{J}}
\newcommand{\scrK}{\mathscr{K}}
\newcommand{\scrL}{\mathscr{L}}
\newcommand{\scrZ}{\mathscr{Z}}
\newcommand{\scrX}{\mathscr{X}}
\newcommand{\scrC}{\mathscr{C}}
\newcommand{\scrV}{\mathscr{V}}
\newcommand{\scrB}{\mathscr{B}}
\newcommand{\scrN}{\mathscr{N}}
\newcommand{\scrM}{\mathscr{M}}

%>>>>>>>>>>>>>>>>>>>>>>>>>>>>>>
%<<<     Fractur Letters    <<<
%>>>>>>>>>>>>>>>>>>>>>>>>>>>>>>
\newcommand{\frakQ}{\mathfrak{Q}}
\newcommand{\frakW}{\mathfrak{W}}
\newcommand{\frakE}{\mathfrak{E}}
\newcommand{\frakR}{\mathfrak{R}}
\newcommand{\frakT}{\mathfrak{T}}
\newcommand{\frakY}{\mathfrak{Y}}
\newcommand{\frakU}{\mathfrak{U}}
\newcommand{\frakI}{\mathfrak{I}}
\newcommand{\frakO}{\mathfrak{O}}
\newcommand{\frakP}{\mathfrak{P}}
\newcommand{\frakA}{\mathfrak{A}}
\newcommand{\frakS}{\mathfrak{S}}
\newcommand{\frakD}{\mathfrak{D}}
\newcommand{\frakF}{\mathfrak{F}}
\newcommand{\frakG}{\mathfrak{G}}
\newcommand{\frakH}{\mathfrak{H}}
\newcommand{\frakJ}{\mathfrak{J}}
\newcommand{\frakK}{\mathfrak{K}}
\newcommand{\frakL}{\mathfrak{L}}
\newcommand{\frakZ}{\mathfrak{Z}}
\newcommand{\frakX}{\mathfrak{X}}
\newcommand{\frakC}{\mathfrak{C}}
\newcommand{\frakV}{\mathfrak{V}}
\newcommand{\frakB}{\mathfrak{B}}
\newcommand{\frakN}{\mathfrak{N}}
\newcommand{\frakM}{\mathfrak{M}}

\newcommand{\frakq}{\mathfrak{q}}
\newcommand{\frakw}{\mathfrak{w}}
\newcommand{\frake}{\mathfrak{e}}
\newcommand{\frakr}{\mathfrak{r}}
\newcommand{\frakt}{\mathfrak{t}}
\newcommand{\fraky}{\mathfrak{y}}
\newcommand{\fraku}{\mathfrak{u}}
\newcommand{\fraki}{\mathfrak{i}}
\newcommand{\frako}{\mathfrak{o}}
\newcommand{\frakp}{\mathfrak{p}}
\newcommand{\fraka}{\mathfrak{a}}
\newcommand{\fraks}{\mathfrak{s}}
\newcommand{\frakd}{\mathfrak{d}}
\newcommand{\frakf}{\mathfrak{f}}
\newcommand{\frakg}{\mathfrak{g}}
\newcommand{\frakh}{\mathfrak{h}}
\newcommand{\frakj}{\mathfrak{j}}
\newcommand{\frakk}{\mathfrak{k}}
\newcommand{\frakl}{\mathfrak{l}}
\newcommand{\frakz}{\mathfrak{z}}
\newcommand{\frakx}{\mathfrak{x}}
\newcommand{\frakc}{\mathfrak{c}}
\newcommand{\frakv}{\mathfrak{v}}
\newcommand{\frakb}{\mathfrak{b}}
\newcommand{\frakn}{\mathfrak{n}}
\newcommand{\frakm}{\mathfrak{m}}

%>>>>>>>>>>>>>>>>>>>>>>>>>>>>>>
%<<<  Caligraphic Letters   <<<
%>>>>>>>>>>>>>>>>>>>>>>>>>>>>>>
\newcommand{\calQ}{\mathcal{Q}}
\newcommand{\calW}{\mathcal{W}}
\newcommand{\calE}{\mathcal{E}}
\newcommand{\calR}{\mathcal{R}}
\newcommand{\calT}{\mathcal{T}}
\newcommand{\calY}{\mathcal{Y}}
\newcommand{\calU}{\mathcal{U}}
\newcommand{\calI}{\mathcal{I}}
\newcommand{\calO}{\mathcal{O}}
\newcommand{\calP}{\mathcal{P}}
\newcommand{\calA}{\mathcal{A}}
\newcommand{\calS}{\mathcal{S}}
\newcommand{\calD}{\mathcal{D}}
\newcommand{\calF}{\mathcal{F}}
\newcommand{\calG}{\mathcal{G}}
\newcommand{\calH}{\mathcal{H}}
\newcommand{\calJ}{\mathcal{J}}
\newcommand{\calK}{\mathcal{K}}
\newcommand{\calL}{\mathcal{L}}
\newcommand{\calZ}{\mathcal{Z}}
\newcommand{\calX}{\mathcal{X}}
\newcommand{\calC}{\mathcal{C}}
\newcommand{\calV}{\mathcal{V}}
\newcommand{\calB}{\mathcal{B}}
\newcommand{\calN}{\mathcal{N}}
\newcommand{\calM}{\mathcal{M}}

%>>>>>>>>>>>>>>>>>>>>>>>>>>>>>>
%<<<<<<<<<DEPRECIATED<<<<<<<<<<
%>>>>>>>>>>>>>>>>>>>>>>>>>>>>>>

%%% From Kac's template
% 1-inch margins, from fullpage.sty by H.Partl, Version 2, Dec. 15, 1988.
%\topmargin 0pt
%\advance \topmargin by -\headheight
%\advance \topmargin by -\headsep
%\textheight 9.1in
%\oddsidemargin 0pt
%\evensidemargin \oddsidemargin
%\marginparwidth 0.5in
%\textwidth 6.5in
%
%\parindent 0in
%\parskip 1.5ex
%%\renewcommand{\baselinestretch}{1.25}

%%% From the net
%\newcommand{\pullbackcorner}[1][dr]{\save*!/#1+1.2pc/#1:(1,-1)@^{|-}\restore}
%\newcommand{\pushoutcorner}[1][dr]{\save*!/#1-1.2pc/#1:(-1,1)@^{|-}\restore}









\usepackage{framed}
%\usepackage{biblatex} 
\usepackage[style=alphabetic,citestyle=alphabetic,
url=false,doi=false,isbn=false,eprint=false]{biblatex}%
\newcommand{\NOFULLPAGE}{\relax}

\newcommand{\Sq}{\mathrm{Sq}}

\bibliography{../../Dropbox/logbook/_LOGBOOK/papers}
\begin{document}

\section*{\Huge Research statement\hfill\normalsize Michael Donovan} 

\section{The unstable Adams spectral sequence for simplicial $k$-algebras}
In \cite{radelescuBanu.pdf}, Radelescu-Banu shows how to form a homotopically correct Adams resolution for a simplicial commutative algebra $X$. There results a homotopical functor $X\mapsto \hat X$, \emph{completion with respect to abelianization}, and an Adams spectral sequence converging to the homotopy of the completion $\pi_*\hat X$.
\subsection{Past work}
We study the completion of a connected simplicial $k$-algebra, proving a conjecture of Radelescu-Banu:
\begin{thm}
If $X$ is a connected simplicial algebra, then the completion map $X\to \hat X$ is an equivalence.
\end{thm}
This is the analogue of the topological fact that simply-connected spaces are $R$-complete. Combining this with theorem [reference vanishing line result] (to follow), one has:
\begin{cor}
If $X$ is a simplicial $\F_2$-algebra, the Adams spectral sequence converges strongly to the homotopy $\pi_*(X)$ of $X$.
\end{cor}
\section{Structure on the homology of unstable Lie coalgebras over $\F_2$}
We will soon be faced with the task of calculating $E^2_{-s,t}(X)=\Ext^s(\F_2[t],H_*^Q(X))$, the $E^2$ page of the unstable Adams spectral sequence for $X$. Before attempting such a calculation one should be equipped with certain homotopical tools.

\subsection{Past work}
For simplicity, I restrict to the case where $X$ is connected. Then, omitting certain details, work of Goerss [] shows that $H:=H^Q_*X$ is an object in a certain comonadic category $\calC^1$ of Lie coalgebras which are compatibly coalgebras for the algebra of $P$-operations [\textbf{or something}]. It's in this category that we're calculating $\Ext$. Write $\calV^1$ for the category of graded vector spaces. Write $\textup{Pr}:\calC^1\to \calV^1$ for the functor whose value in dimension $t$ is $\Hom(\F_2[t],\DASH)$, so that $E^2_{-s,t}(X)=\mathbb{R}^s\textup{Pr}_t(H_*X)$.

In general, let $W$ be any object of $\calC^1$. In the style of the construction of Goerss, one defines a function $\xi_\textup{alg}:(C^{s+1}W)^{\wedge2}\to C^{s+2}W$, writing $C$ for the cofree construction in $\calC^1$, and $\wedge $ for the cokernel of the natural map from coproduct to product. One may combine this with a natural map $j:(\textup{Pr}(C^{s+1}W))^{\otimes2}\to \textup{Pr}((C^{s+1}W)^{\wedge 2})$, to obtain a pairing
\[\mathbb{R}^s\textup{Pr}(W)\otimes \mathbb{R}^{s'}\textup{Pr}(W)\to \mathbb{R}^{s+s'+1}\textup{Pr}(W)\textup{ given by }[x]\otimes [y]\mapsto [\xi_{\textup{alg}}j(D^0(x\otimes y))]\]
along with `Steenrod operations'
\[\Sq^{i+1}:\mathbb{R}^s\textup{Pr}(W)\to \mathbb{R}^{s+i+1}\textup{Pr}(W)\textup{ given by }[x]\mapsto [\xi_{\textup{alg}}j(D^{s-i}(x\otimes x))]\]
where $\{D^k\}$ is a \emph{special cosimplicial Eilenberg-Zilber map} (c.f.\ \cite[5.2]{turner_opns_and_sseqs_I.pdf}).

Moreover, by virtue of the algebra of $\delta$-operations being Koszul dual to the algebra of $P$-operations, [\textbf{approximately}], the derived functors admit a left action of the $\delta$-operations:
\[\delta_i:\mathbb{R}^s\textup{Pr}(W)_t\to \mathbb{R}^{s+1}\textup{Pr}(W)_{t+i+1},\textup{ for $2\leq i<t$}.\]
When attempting to calculate $\mathbb{R}^*\textup{Pr}_*(W)$, one should also attempt to describe the action of these operations.


\subsection{Future progress}
In the future I hope to show that the `Steenrod operations' in fact satisfy the Adem relations --- a claim analogous to results \cite[5.3]{PriddySimplicialLie.pdf} of Priddy, but a little more difficult, since in this context the Eilenberg-Mac Lane functor $\overline{W}$ is rather less useful.


\section{Pairings and operations in the Adams spectral sequence for $\F_2$-algebras}
If $X$ is a simplicial $\F_2$-algebra, then $\pi_*(X)$ is itself a commutative algebra, of which $\oplus_{t>0}\pi_{t}X$ is a divided power ideal. Moreover, $\pi_*(X)$ exhibits higher divided square operations $\delta_i:\pi_t(X)\to \pi_{t+i}(X)$, for $2\leq i \leq t$, which satisfy an Adem relation, and the unstability relation $\delta_ix=\gamma_2(x)$. One would hope to be able to find products and $\delta$-operations on the whole spectral sequence, compatible at $E^\infty$ with those on $\pi_*(X)$.

Now we'll use the canonical double complex, $E^0_{-s,t}=c(KQc)^{s+1}X$, in which case we have 
\[E^1_{-s,t}=\pi_t(c(KQc)^{s+1}X)\cong \textup{Pr}(H_t^Q(c(KQc)^{s+1}X))\cong \textup{Pr}(C^{s+1}H_*^Q(X))\cong \Hom(H_*^QS^t,C^{s+1}(H^Q_*(X))).\]
We are then deriving the Adams spectral sequence as the spectral sequence of a cosimplicial simplicial algebra, filtered by cosimplicial degree. Dwyer \cite{DwyerHigherDividedSquares.pdf} has investigated the products and unary operations available in the spectral sequence of a cosimplicial simplicial coalgebra, filtered by cosimplicial degree. This is not the linear dual of the situation of interest for the Adams spectral sequence I investigate, and so, using analogous constructions, I prove:
\begin{thm}
Denote by $E^r_{-s,t}$ the spectral sequence of a cosimplicial simplicial nonuital $\F_2$-algebra $X$. Then one can construct a product pairing, Steenrod operations $\Sq^i:E^r_{-s,t}\to E^r_{-s-i,2t}$ (for $i\geq0$), and $\delta$ operations $\delta_i:E^r_{-s,t}\to E^r_{-s,t+i}$ (for $i$ satisfying $2\leq i<t-(r-2)$). The Steenrod operations have indeterminacy vanishing by $E^{2r-2}$, survive to $E^{2r-1}$, and commute with differentials, in the sense that $d_{2r-1}(\Sq^ix)=\Sq^{i+r-1}(d_rx)$ modulo indeterminacy. The $\delta$-operations may be multivalued when $i\geq \textup{max}(t-2(r-2),n+1)$ (so that at $E^2$, all the operations are well defined except for $\delta_t$, which is not defined at all).
\begin{enumerate}\squishlist
\setlength{\parindent}{.25in}
\item When $i<t-s$, $d_r\delta_i(x)=\delta_i(d_rx)$.
\item When $i=t-s$, $d_r\delta_i(x)=\begin{cases}
\delta_i(d_rx),&\textup{if }s>0;\\
\delta_i(d_rx)+x\times d_rx,&\textup{if }s=0.
%\\,&\textup{if }
\end{cases}$
\item When $i>t-s$, $d_r\delta_i(x)=\begin{cases}
\delta_i(d_rx),&\textup{if }r<t-i+1;\\
\delta_i(d_rx)+\Sq^{t-i+1}(x),&\textup{if }r=t-i+1;\\
\Sq^{t-i+1}(x),&\textup{if }r>t-i+1.\\
%\\,&\textup{if }
\end{cases}$
\end{enumerate}
Broadly speaking then, the $\delta_i$ operations for $i>t-s$ support a differential hitting a Steenrod operation. If $X$ admits a coaugmentation from a simplicial algebra $X$, then the $\delta$-operations 
\end{thm}
\section{Iterative calculation of the homology of unstable Lie coalgebras}
\begin{prop}
The functor $\textup{Pr}$ factors through the category $\calL^1$ of Lie coalgebras:
\[\xymatrix@R=4mm@1{
\calC^1\ar[r]^-{\textup{Pr}_P}
&%r1c1
\calL^1\ar[r]^-{\textup{Pr}_\calL}
&%r1c2
\calV^1%r1c3
}\]
\end{prop}
At this point, we may apply the machinery of 

\printbibliography

\end{document}

Quillen \cite{QuillenHomAlg.pdf} has formulated the notion of homotopy theory in a variety of algebraic contexts. In particular, there is a homotopy theory of any given type of universal algebra. The homotopy theory of augmented \ensuremath{\F_2}-algebras is an important and interesting example. It is of interest both to topologists at least insofar as it appears in the proofs of such interesting results as the Sullivan conjecture \cite{MillerSullivanConjecture.pdf}. This homotopy theory is of interest in commutative algebra, as Andr\'e and Quillen's definition of the cohomology of an augmented $\F_2$-algebra coincides with the Quillen homology of said algebra using the homotopical definitions of \cite{QuillenHomAlg.pdf}.

We are thus amply justified in considering the homotopy theory of augmented \ensuremath{\F_2}-algebras worthy of study. Interestingly, however, it is not well understood at present --- there is much progress to be made, including a number of unproven potential theorems analogous to long-standing results in the homotopy theory of spaces. I thus propose an investigation into a number of elements of this homotopy theory, with scope for either broadening or specialisation.
Present research has focused on the definition and study of an analogue of the Adams spectral sequence, while the hope is to turn to applications as soon as possible.
%Recently, it was suggested by my advisor that I should develop and study an Adams spectral sequence in the category $s\calA$ of simplicial augmented $\F_2$-algebras. Since that time, I have enjoyed studying the work relevant to this research, and have found the various challenges associated with understanding the resulting spectral sequence to be of interest. Thus, I make the following proposal.

\section*{Background}
Let $\calA$ be any category of universal algebras. Quillen developed in \cite{QuillenHomAlg.pdf} a model structure on the category $s\calA$ of simplicial objects in $\calA$, and introduced the notion of the Quillen homology of objects of $s\calA$.
An \emph{abelian object of $s\calA$} is an object $A\in s\calA$ along with a natural abelian group structure on $s\calA(\DASH,A)$. There is an evident category $\mathsf{ab}(s\calA)$ of abelian objects, and a left adjoint to the forgetful functor yielding an adjunction $s\calA\rightleftarrows\mathsf{ab}(s\calA)$. Quillen homology is then defined to be the left derived functors of \emph{abelianization}, this left adjoint.

At this point we specialise, letting $\calA$ be the category of augmented commutative \ensuremath{\F_2}-algebras.
%Let $\calA$ be the category of augmented commutative \ensuremath{\F_2}-algebras (or more generally any category of universal algebras). Quillen developed in \cite{QuillenHomAlg.pdf} a model structure on the category $s\calA$ of simplicial objects in $\calA$, and introduced the notion of the Andre-Quillen (co)homology of objects of $s\calA$. Quillen (co)homology appears to be of interest to the algebraists.
%An \emph{abelian object of $s\calA$} is an object $A\in s\calA$ along with a natural abelian group structure on $s\calA(\DASH,A)$. There is an evident category $\mathsf{ab}(s\calA)$ of abelian objects, and a left adjoint to the forgetful functor yielding an adjunction $s\calA\rightleftarrows\mathsf{ab}(s\calA)$. Andre-Quillen homology is supposed (in some generality) to be the left derived functors of \emph{abelianisation}, this left adjoint.
There, an analysis \cite[\S4]{MR1089001} of $\mathsf{ab}(s\calA)$ reveals that the abelian objects of $s\calA$ are precisely those simplicial algebras with levelwise trivial multiplication, and we can model the above adjunction as 
\[Q:s\calA\rightleftarrows s(v\ensuremath{\F_2}):K\]
where $v\ensuremath{\F_2}$ is the category of $\F_2$-vector spaces, $Q$ prolongs the indecomposables functor $Q:\calA\to v\ensuremath{\F_2}$, and $K$ prolongs the `square-zero' functor $K:v\ensuremath{\F_2}\to \calA$, under which $V$ maps to $\ensuremath{\F_2}\oplus V$ with trivial multiplication. Thus in our context, the Quillen homology $H_*^Q(A)$ of an object $A\in s\calA$ is $\mathbb{L}_*Q(A)=\pi_*Q(cA)$, where $c$ denotes a cofibrant replacement in $s\calA$. This coincides with those older definitions of Andr\'e-Quillen homology of interest in commutative algebra, and will be central in the following exposition.

\subsection*{Structure on cohomology and homotopy}
Let $S$ be the 
`symmetric algebra' monad on $v\ensuremath{\F_2}$, arising from the adjunction $\calA\rightleftarrows v\ensuremath{\F_2}$. Then objects of $\calA$ are precisely algebras for this monad. Now Dold's theorem \cite{DoldHomologySPs.pdf} shows that there is an endofunctor $\frakS$ of $n\ensuremath{\F_2}$ creating a commuting diagram
\[\xymatrix@R=4mm{
s(v\ensuremath{\F_2})\ar[r]^-{S}
\ar[d]^-{\pi_*}
&%r1c1
s(v\ensuremath{\F_2})\ar[d]^-{\pi_*}
\\%r1c2
n\ensuremath{\F_2}\ar[r]^-{\frakS}
&%r2c1
n\ensuremath{\F_2}%r2c2
}\]
Moreover, the naturality of Dold's result shows that $\frakS$ is also a monad, and $\pi_*$ sends (simplicial) $S$-algebras to $\frakS$-algebras. Thus the homotopy of an object of $\ensuremath{s\calA}$ is naturally an $\frakS$-algebra, and the category of $\frakS$-algebras is the richest value category for $\pi_*$ on $\ensuremath{s\calA}$. To justify this last claim, one calculates the homotopy of the `spheres' in $s\calA$, which represent homotopy --- the sphere `$S(n)\in\ensuremath{s\calA}$' (representing $I\pi_n:\ensuremath{s\calA}\to v\ensuremath{\F_2}$) is the image under $S$ of a $K(\ensuremath{\F_2},n)$: the free simplicial $F_2$-vector space on the standard sphere $S^n\in s\mathsf{Set}_*$. One finds that an $\frakS$-algebra is precisely a graded vector space with an action of the divided square operations of Bousfield \cite{BousOpnsDerFun.pdf,BousHomogFunctors.pdf}, Cartan \cite{CartanDivSquares} and Dwyer \cite{DwyerHtpyOpsSimpComAlg.pdf}.

One may next wonder what natural algebraic structure exists on $H^Q_*(A)$. It is more convenient to discuss the AQ cohomology $H_Q^*(A):=(H^Q_*(A))^*$. There are Eilenberg-Mac Lane objects in $ s\calA$ which represent AQ cohomology, and an analysis of the cohomology of products thereof, as performed in \cite{MR1089001}, reveals that the richest value category for $H_Q^*$ is $\calW$, whose definition we now recall.
An object of $\calW$ is a non-negatively graded \ensuremath{\F_2}-vector space $W^*$, along with:
\begin{enumerate}\squishlist
\setlength{\parindent}{.25in}
\item a Lie bracket $[\DASH,\DASH]:{W^n}\otimes{W^m}\to W^{n+m+1}$;
\item linear operations $P^i:W^n\to W^{n+i+1}$, satisfying certain `unstableness' axioms and `Adem relations'; and
\item a quadratic operation $\beta:W^0\to W^1$, which is a (partially defined) restriction for $[\DASH,\DASH]$; it satisfies those axioms which `$x\mapsto x^2 = \frac{1}{2}[x,x]$' might, were any of this to make sense.
\end{enumerate}

\subsection*{Quillen's spectral sequence}
In \cite{QuillenHCommRings.pdf} Quillen introduced a spectral sequence whose $E^1$ page depends functorially on the homology $H_*^QA$, converging to the homotopy groups $\pi_*A$.  More precisely:
\[E^1_{st}=\frakS_s(H_*^QA)\Rightarrow\pi_tA,\]
where we use the decomposition $\frakS=\bigoplus_{s\geq0}{\frakS_s}$ corresponding to the decomposition $S=\bigoplus_{s\geq0}{S_s}$ of the free symmetric algebra into lengths $s$.

This spectral sequence is straightforward to define. Take $A$ to be cofibrant, and filter by powers of the augmentation ideal, obtaining a filtered simplicial vector space. The associated graded simplicial vector space can be identified with $S_*(QA)$, and the definitions of $\frakS_*$ and $H_*^Q$ show that the associated spectral sequence has the desired $E^1$-term.

This spectral sequence is convergent when $A$ is connected (i.e.\ $\pi_0A\simeq\ensuremath{\F_2}$), and supports divided square operations (modulo indeterminacy) which reflect both those on $\pi_*A$ and those on $\frakS_s(H^Q_*A)_t$. It is the direct analogue of Curtis' unstable lower central series spectral sequence \cite{MR0184231}, which leaves us the task of defining an Adams spectral sequence.

\section*{An Adams spectral sequence}
One can form a classical unstable Adams spectral sequence using the Bousfield-Kan completion tower. This is the homotopy spectral sequence (cf.\ \cite{Bousfield-htpySS.pdf})   of the cosimplicial space $\F_2^\bullet X$ \cite{MR0365573} whose totalization is the \ensuremath{\F_2}-completion of $X$.

One might first ask how to replace $A\in \ensuremath{s\calA}$ with an appropriate cosimplicial object in $\ensuremath{s\calA}$, in order to imitate this process.  One could use the monad of the adjunction $Q:\ensuremath{s\calA}\rightleftarrows s(v\ensuremath{\F_2}):K$, but as the image of $K$ does not land within the cofibrant objects of \ensuremath{s\calA}, we would not be able to identify any resulting $E_2$-page with a functor of the (co)homology. [This problem was not present for the BK tower, as every simplicial set is cofibrant.]

In order to produce a homotopically correct resolution, one must mix intelligent cofibrant replacements into the iteration of this monad. This process is the subject of recent work of Blumberg and Riehl \cite{BlumRiehlResolutions.pdf}. They show that if one can choose a functorial cofibrant replacement $c$ on $\ensuremath{s\calA}$ which has the additional structure of a comonad%footnote:
\footnote{As every object of $s(v\ensuremath{\F_2})$ is fibrant, we need not mention a monadic fibrant replacement on that category, although there is no such asymmetry in the treatment of \cite{BlumRiehlResolutions.pdf}.}, then one can give a homotopically correct cosimplicial resolution of the type we desire. Precisely, they produce an augmented cosimplicial endofunctor of $\ensuremath{s\calA}$ (writing $\overline{Q}$ for $Qc:\ensuremath{s\calA}\to s(v\ensuremath{\F_2})$):
\[
\vcenter{
\def\labelstyle{\scriptstyle}
\xymatrix@C=1.5cm@1{
c\,
\ar[r]
&
\,cK\overline{Q}\,
\ar[r];[]
&
\,c(K\overline{Q})^2\,
\ar@<-1ex>[l];[]
\ar@<+1ex>[l];[]
\ar@<+1ex>[r];[]
\ar@<-1ex>[r];[]
&
\,c(K\overline{Q})^3\,
\ar[l];[]
\ar@<-2ex>[l];[]
\ar@<+2ex>[l];[]
\ar[r];[]
\ar@<+2ex>[r];[]
\ar@<-2ex>[r];[]
&
\,c(K\overline{Q})^4\,\makebox[0cm][l]{\,$\cdots $}
\ar@<-3ex>[l];[]
\ar@<-1ex>[l];[]
\ar@<+1ex>[l];[]
\ar@<+3ex>[l];[]
}}\]
For $A\in\ensuremath{s\calA}$, write $\underline{A}^\bullet$ for the cosimplicial object $c(K\overline{Q})^\bullet A$.
The Blumberg-Riehl approach augments Bousfield's work on cosimplicial resolutions \cite{BousCosimpResnHtpySS.pdf},  allowing us to define a homotopical `derived completion' endofunctor on $\ensuremath{s\calA}$ by $A\mapsto\hat A:=\textup{Tot}((\underline{A}^\bullet)^{\textup{rf}})$, where `$\textup{rf}$' denotes a Reedy fibrant replacement.

Bousfield shows how to obtain a spectral sequence $\pi^s\pi_t(\underline{A}^\bullet;S^0)\Rightarrow\pi_{t-s}(\hat A;S^0)$ by applying the well established Bousfield-Kan spectral sequence to the evident cosimplicial space of Dwyer-Kan mapping complexes. It serves our purposes best to dualize, and rewrite:
\[\pi_s\pi^t((Ic(K\overline{Q})^\bullet A)^*)\Rightarrow \pi^{t-s}((I\hat A)^*)\]
A complete derivation would at this point include a detailed examination of the dual Hurewicz homomorphism $H^*_Q(X)\to \pi^*((IX)^*)$, in particular when $X$ is an abelian object. One shows that:
\begin{prop*}
When $H^*_Q(A)$ is of finite type (for example when $A$ is connected and of finite type) the spectral sequence has $E_2^{st}=(\mathbb{L}_s\textup{Ind}_\calW)(H_Q^*A)^t=(\Ext^s_{\calW}(H_Q^*A,H_Q^*S^t))^*$.
\end{prop*}
\noindent I have not yet considered whether or not this spectral sequence converges.

One should be clear what is meant by the derived functors $\mathbb{L}_s\textup{Ind}_\calW$. $\calW$ is an example of what Blanc-Stover call a \emph{category of universal graded commutative algebras}, or `CUGA' \cite{Blanc_Stover-Groth_SS.pdf}. One notes that there is a free-forgetful adjunction $n\mathsf{Set}_*\rightleftarrows \calW$ ($n\mathsf{Set}_*$ being graded pointed sets), and the free objects provide enough projectives, so that $s\calW$ is a Quillen model category. Moreover, there is an evident `indecomposables' functor
\[\textup{Ind}_\calW:\calW\to n\ensuremath{\F_2}\]
which sends an object of $\calW$ to its vector space quotient by the image therein of all nontrivial operations. We mean the derived functors%footnote:
\footnote{Note that as every object in a CUGA is fibrant, any prolongation will preserve weak equivalence between cofibrant objects (which are just simplicial homotopy equivalences), so that any functor from a CUGA is left derivable.} of this functor.


\section*{A Grothendieck spectral sequence}
It was first noted by Haynes Miller in \cite{MillerSullivanConjecture.pdf} that composite functor spectral sequences can be of some utility in contexts such as ours. Thus one is led to consider a factorisation of the functor $\textup{Ind}_\calW$ through a third category, $\calL$, of graded Lie algebras.%footnote:
\footnote{An object of $\calL$ is a non-negatively graded \ensuremath{\F_2}-vector space $W^*$, with a Lie bracket ${W^n}\otimes{W^m}\to W^{n+m+1}$ satisfying the (non-redundant) axiom `$[x,x]=0$'.}

It is simply a matter of examining the definition (cf.\ \cite[p.17]{MR1089001}) of $\calW$ to see that $\textup{Ind}_\calW$ factors as
\[\calW\overset{\textup{Ind}_\textbf{P$\beta$}}{\to}\calL\overset{\textup{Ind}_{[]}}{\to}n\F_2\]
where $\textup{Ind}_\textbf{P$\beta$}$ and $\textup{Ind}_{[]}$ are the evident `indecomposables' functors killing $P^i$ and $\beta$ operations and Lie brackets respectively.%footnote:
\footnote{The factorisation $\textup{Ind}_{[]}\circ\textup{Ind}_{\textbf{P$\beta$}}$ represents a choice that I made some time ago, which may not be the most natural or most useful. Another alternative is to factor as $\textup{Ind}_{[]\beta}\circ\textup{Ind}_{\textbf{P}}$ through a category of Lie algebras with partially defined restriction.}

The category $\calL$ is itself a CUGA, and $\textup{Ind}_\textbf{P$\beta$}$ sends free objects in $\calW$ to free objects in $\calL$. In this context, Blanc and Stover have developed in \cite{Blanc_Stover-Groth_SS.pdf} a Grothendieck spectral sequence calculating $\mathbb{L}_*\textup{Ind}_\calW$. Before stating this spectral sequence, we must describe a little more of the Blanc-Stover theory.

\subsection*{Categories of $\Pi$-algebras}
Given a CUGA $\calC$, Blanc-Stover define a category $\calC'$ (which they denote $\calC$-$\Pi$-$\textup{Alg}$), which is the natural value category for the functor $\pi_*$ on $s\calC$. We may be more explicit as follows.

Observe that the homotopy groups of $X\in s\calC$ are in fact \textbf{bi}graded --- write $G_k\pi_nX$ for the $k^{\textup{th}}$ internally graded part of the $n^{\textup{th}}$ homotopy group of $X\in s\calC$. Then $G_k\pi_n$ is represented (in $\ho(s\calC)$) by $F(S^n(k))$, where the `sphere' $S^n(k)\in n(s\mathsf{Set}_*)$ is the graded pointed simplicial set which has a standard $n$-sphere in grading $k$ and a point in all other gradings.

As ever, the homotopy groups of $r$-fold wedges of spheres dictate the natural $r$-ary operations on the homotopy groups of objects of $s\calC$. These operations are subject to relations dictated by the data of various equations of maps under various compositions. Now an object of $\calC'$ is a bigraded vector space with an $r$-ary operation for every homotopy class in an $r$-fold wedge of spheres. Of course, we demand that the operations satisfy the appropriate relations.

Now if $T: \calC\to\calB$ is a functor between CUGAs, one can define a functor $\overline{T}:\calC'\to\calB'$. This is, of course, analogous to Dold's theorem. In the case where $\calC=\calB=v\ensuremath{\F_2}$, $\calC'=\calB'=n\ensuremath{\F_2}$, and $\overline{T}$ is precisely the functor associated with $T$ by Dold's theorem. We are now in position to present the promised Grothendieck spectral sequence.
\begin{prop*}
For any $W\in \calW$ there is a convergent spectral sequence:
\[(\mathbb{L}_p\overline{\textup{Ind}_{[]}}_q)(\mathbb{L}_*\textup{Ind}_\textbf{P$\beta$})(W)\Rightarrow (\mathbb{L}_{p+q}\textup{Ind}_\calW)(W).\]
\end{prop*}
\noindent 
The RHS is bigraded, while the LHS is trigraded.
The indices may seem a little confusing, but they needn't be. When interpreting the LHS, one thinks ``we're deriving a functor $\overline{\textup{Ind}_{[]}}:\calL'\to nn\ensuremath{\F_2}$ between categories of bigraded objects''.

\subsection*{The functor $\overline{\textup{Ind}_{[]}}:\calL'\to nn\ensuremath{\F_2}$}
\newcommand{\uLLa}{\mathsf{u}\Lambda\!^{+\!}\mathsf{La}}
Before making sense of the above spectral sequence, we must understand $\calL'$, and the functor $\overline{\textup{Ind}_{[]}}$. For this we introduce the category $\uLLa$ of `unstable $\Lambda^+$-Lie algebras'. An object of this category is a graded Lie algebra $L_*$ over \ensuremath{\F_2} (with axiom $[[x,x]]=0$, and no $+1$ shift in grading), along with right operations $\lambda_i:L_n\to L_{n+i}$ for $1\leq i\leq n$ (written $x\mapsto x\lambda_i$), such that:
\begin{enumerate}\squishlist
\setlength{\parindent}{.25in}
\item $\lambda_{\textup{top}}:=\lambda_n:L_n\to L_{2n}$ is a restriction for ($n>0$):
\begin{enumerate}\squishlist
\setlength{\parindent}{.25in}
\item $[[x\lambda_{\textup{top}},z]]=[[x,[[x,z]]]]$ for $|x|>0$, $|z|\geq0$
\item $\lambda_{\textup{top}}(x+y)=\lambda_{\textup{top}}(x)+\lambda_{\textup{top}}(y)+[[x,y]]$ for $|x|=|y|>0$
\end{enumerate}
\item the $\lambda_i:L_n\to L_{n+i}$ for $1\leq i< n$ are linear;
\item the $[[x\lambda_i,z]]=0$ for $1\leq i<|x|$.
\item the $\lambda_i$ operations satisfy the Adem relations of \cite{6Author.pdf}.
\end{enumerate}
One is able to see that in fact there is a Hilton-Milnor theorem%footnote:
\footnote{This comes from an essentially formal Hilton-Milnor decomposition for the coproduct of a finite number of free Lie algebras, where by `Lie algebra' we mean `Lie algebra over $\ensuremath{\F_2}$ subject to the extra axiom $[x,x]=0$'.} for homotopy groups of wedges of spheres in $s\calL$. This, along with the description of the homotopy groups of the free unrestricted Lie algebra given in \cite{6Author.pdf} can be used to demonstrate that $\calL'$ is a category of $+1$-shifted graded objects in $\uLLa$ (thus bigraded \ensuremath{\F_2}-Lie algebras with one of the gradings yielding a $+1$ shift, etc.).

%Moreover, the homotopy groups of a given sphere are essentially calculated by the six authors of \cite{6Author.pdf}. In particular, the elements of the homotopy groups of wedges of spheres are of the form ``$\lambda_{i_1}\lambda_{i_2}\cdots\lambda_{i_r}\ell$'' for $\ell$ an iterated Lie bracket of fundamental classes. Here the Lie bracket is $[[\DASH,\DASH]]$, the Eilenberg-Zilber Lie bracket on the homotopy of a simplicial Lie algebra.
%
%The details here haven't been worked out precisely. In particular, I don't know what an admissible sequence is supposed to be yet, and I don't know how the $\lambda$ operations squeeze outside the Lie brackets. However, one knows enough already to say that $\calL'$ is a category of bigraded objects subject to a bigraded Lie bracket%footnote:
%\footnote{(with a $+1$ shift in the internal grading)} and certain $\lambda_i$ operations.

At this point, it is not completely obvious what $\overline{\textup{Ind}_{[]}}$ will turn out to be, as demanded by the definitions of \cite{Blanc_Stover-Groth_SS.pdf}. However, it can be calculated that it simply kills the image of $[[\DASH,\DASH]]$ and of each $\lambda_i$ operation.

\subsection*{A third spectral sequence computing $\mathbb{L}_*\overline{\textup{Ind}_{[]}}$}
We have seen that $\calL'$ is the category of unstable $\Lambda^+$-Lie algebras with an internal grading, and $\overline{\textup{Ind}_{[]}}$ is the functor $\calL'\to nn\ensuremath{\F_2}$ which kills all $\lambda$ operations and brackets. In particular, we can factor this functor as
\[\calL'\overset{\textup{Ind}_{\lambda}}{\to}\textup{gr}{\calL}\overset{\textup{Ind}_{[[,]]}}{\to}nn\F_2\]
where (for present lack of notation) $\textup{gr}{\calL}$ is the category of bigraded Lie algebras with one of the gradings $+1$-shifted. Again, since $\textup{Ind}_\lambda$ preserves free objects, we obtain a convergent spectral sequence
\[(\mathbb{L}_p\overline{\textup{Ind}_{[[,]]}}_q)(\mathbb{L}_*\textup{Ind}_\lambda)(\Omega)\Rightarrow (\mathbb{L}_{p+q}\overline{\textup{Ind}_{[]}})(\Omega),\]
for any $\Omega\in\calL'$, for example $\Omega=(\mathbb{L}_*\textup{Ind}_\textbf{P$\beta$})(W)$. Now $\overline{\textup{Ind}_{[[,]]}}$ is no more difficult than $\overline{\textup{Ind}_{[]}}$, and this process can be iterated. One wonders if this has any potential use.


%It seems just possible at this moment that there is another Blanc-Stover-Grothendieck spectral sequence calculating the derived functors of $\overline{\textup{Ind}_{[]}}$. After all, given that the $\lambda_i$ can move out the the left, $\overline{\textup{Ind}_{[]}}$ \emph{may} factor as a composite in which the $\Lambda$ operations are killed first, and then the Lie brackets. If so, there \emph{may} be another Grothendieck spectral sequence. One of the terms in this \emph{third} spectral sequence would be Quillen homology of Lie algebras (I suspect), and thus conceptually easy to calculate. I suppose that if this worked out, this could be iterated literally forever. It's a strange thought.
\section*{Future directions}
\subsection*{A Koszul approach to the Grothendieck spectral sequence}
There may be a second approach to defining the above composite functor spectral sequence.
%
%
%Another question to be investigated was raised by Haynes some weeks ago, but hasn't been allocated much processor time up to this point. The question is whether or not the following can be made reality.
There is a commuting diagram:
\[\xymatrix@R=4mm{
\calW
\ar[r]^-{\textup{Ind}_{\textbf{P$\beta$}}}
\ar[d]
&%r1c1
\calL
\ar[r]
\ar[d]
&%r1c2
n\ensuremath{\F_2}
\\%r1c3
\calP
\ar[r]^-{\textup{Ind}_{\calP}}
&%r2c1
n\ensuremath{\F_2}
&%r2c2
%r2c3
}\]
Here, the category $\calP$ is obtained from $\calW$ by forgetting the Lie structure, and supports $P$ operations and the $\beta$ operation.

Now the derived functors of $\textup{Ind}_\calP$ might be thought of as $\Tor_\calP$ with a trivial object in $\calP$, and although some structure operations in $\calP$ are quadratic, not linear, one may hope that there is a Koszul calculation of these derived functors. Further, that some Koszul resolution can be naturally enriched with a Lie bracket and filtered in such a way that that the required spectral sequence is just that associated with said filtration.


\subsection*{Operations in the Adams spectral sequence}
The Adams spectral sequence defined here converges to $(\pi_*(I\hat A))^*$, which is (modulo augmentations) dual to the homotopy of $\hat A$. As such, at least the abutment contains operations dual to the divided square operations, and a coalgebra structure. One wonders whether this structure can be extended to the $E^r$ pages for $r\geq2$. If so, it would be rather aesthetic if the structure on $E^2$ could be induced by some natural structure on $\mathbb{L}_*\textup{Ind}_{\calW}$ groups. We have yet to perform this investigation.
%The question comes up of how the coalgebra structure and dual $\delta$ operations to be found on the abutment of the Adams spectral sequence might be reflected on the Adams $E_2$, and moreover how this reflection might be reflected in the Grothendieck spectral sequence! 
%
%Question of how the resulting $\Lambda$ operations act on the $E_2$ page --- they should exhibit some duality with the $\delta$ operations. Hrrrrrmmmmmm... That's funny... Don't we kill the $\Lambda$ operations by applying $IndP$? Maybe this works just fine if you use the $\Lambda$-algebra approach ------ that's probably the point: use a filtration of the $\Lambda$ thing which gives what you want in homotopy AND whose filtered parts also do the right thing.

\subsection*{Possible applications}
As indicated above, there are a number of possible directions for this research, many of which I am at present unaware. Here I'll indicate a few possibilities.

In \cite{MR0413089}, Kan and Thurston show that every connected simplicial complex admits a homology isomorphism from a $K(G,1)$. An analogue in our context would replace $K(G,1)$ with `constant object of \ensuremath{s\calA}', and is yet to be investigated.

In \cite{PiAlgModuli.pdf}, Blanc, Dwyer and Goerss give an obstruction theory for the realisation of a $\Pi$-algebra as the homotopy $\Pi$-algebra of a topological space. One may search in this context for a similar theory, either for the realisation of a $\frakS$-algebra as the homotopy, or of an element of $\calW$ as the cohomology, of an element of $\ensuremath{s\calA}$. Again, this is work yet to be performed.

Interest (c.f.\ \cite{MR896094,MR733698}) has been shown in non-vanishing results for the cohomology of objects of \ensuremath{s\calA} with homotopy in only finitely many simplicial degrees, but various questions remain open for investigation.

It is possible to form a stable model category of `spectra in \ensuremath{s\calA}'. Having done this, one may be able to define a stable Adams spectral sequence, and a stable version of Quillen's fundamental spectral sequence. The topological analogues coincide up to a grading shift in the stable case, although the relationship between the unstable spectral sequences is not well understood. The corresponding investigation in the \ensuremath{\F_2}-algebra context to be more straightforward, but said analysis is yet to be performed.

Recent work of Haugseng and Miller \cite{RuneMillerCohomLoopSpace.pdf} contains a spectral for the cohomology of an infinite loop space, starting with certain derived functors applied to the stable cohomology. This spectral sequence may be related to the Adams spectral sequence via a map from the Dyer-Lashov algebra to the $\Lambda$ algebra. It would be interesting to see if this analysis has analogues in our context.

\subsection*{To do}
\begin{enumerate}\squishlist
\setlength{\parindent}{.25in}
\item Do all of Goerss' book in a more general context, including:
\item Check out Quillen's fundamental spectral sequence --- its $(E_1,d_1)$ depends on the homology, and maybe can be dualised to get something more comparable to the Adams SS I'm working on. In particular, I need to understand the $\frakS$ construction's interactions with dualisation in order to rewrite Goerss' formula for $d_1$, and then think a little about what in $E_1$ is hit by $d_1$, especially in the case of a sphere, where the two abutments are so well known. \textbf{Question:} I would think that the Quillen SS is degenerate for a sphere, from $E_1$. What about the Adams SS? It would be AMAZING if it had no differentials.
\end{enumerate}

\subsection*{Simplicial Lie algebras}
Do all this for simplicial Lie algebras, of the type I like. There should be an Adams SS defined in basically the same way, and a Quillen SS to compare it to. Note that I'll need to calculate (like Goerss on simplical commutative algebras) the cohomology of Eilenberg-Mac Lane objects, in order to talk meaningfully about factorisations of the Hurewicz.



\printbibliography

\end{document}















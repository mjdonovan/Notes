% !TEX root = z_output/_michael_donovan_resume.tex
% @brief    LaTeX2e Resume for Kamil K Wojcicki
\documentclass[margin,line]{resume}
\usepackage{mathrsfs}
\usepackage{amsmath, amsthm, amssymb, graphicx}
\usepackage[bookmarks=false,pdftex,pdfborder={0 0 0 [1 1]}]{hyperref}
\usepackage{cancel}

\headheight=-2pt
\topmargin=8pt
\topmargin=2pt
\topmargin=-27pt
\topmargin=-34pt
\textheight=634pt
\textheight=654pt
\textheight=695pt
\textheight=740pt
%\textwidth=440pt
%\oddsidemargin=-12pt
%\evensidemargin=8pt


\newcommand{\dashtab}{\makebox[1cm][r]{$-$ }}
\newcommand{\whitetab}{\makebox[1cm][r]{ }}

\newcommand{\CVsection}[1]{\section{\mysidestyle #1}}
\newcommand{\entry}[3]{\textbf{#1} #2 \hfill {#3}
           
\vspace{-3.4mm}}
\newcommand{\twolineentry}[4]{\textbf{#1} #2 \hfill {#4}\\%
#3
           
\vspace{-3.4mm}}



\newcommand{\FINALentry}[3]{\textbf{#1} #2 \hfill {#3}\\\vspace{-.55cm}}
\newcommand{\FINALtwolineentry}[4]{\textbf{#1} #2 \hfill {#4}\\%
#3\\\vspace{-.55cm}}



%- change the font on your email & web address to something less nerdy, it's best to use just one font for everything & use one size except for the header
%- there's a lot of bold text. Maybe the years don't have to be bold? Maybe something else can go. It's hard on the eyes if everything is bold
%- only show programming experience if you think it's plausible the employer would need it. For a business application, you might do a section on skills and include 1 line on programming amongst a couple of fluffier things like mentoring, facilitation etc
%- if you need more space, perhaps you don't need to list where the AMSI summer schools were, that could save 3 lines
%- a lot of Aus cvs have one line on recreation at the end. It goes something like 'in my free time I enjoy soccer, squash, baking fresh bread, socialising with uni club XYZ and travel' or similar. I don't know if that's the go in the US but we sometimes like to see it on cvs that are otherwise pretty academic only 


\renewcommand{\namefont}{}

\begin{document}
\name{\raisebox{.1cm}{\LARGE\sc Michael Jack Donovan}\qquad \qquad \qquad \qquad \qquad \qquad \qquad \hspace{-1.2pt}\ \  857-891-2805 \ $\cdot$ \  mdono@math.mit.edu}
\begin{resume}
%\CVsection{Overview}
%\entry{\hspace{-.275em}}{$-$ Mathematics Ph.D.\ student at MIT with actuarial science background}{}
%\entry{\hspace{-.275em}}{$-$ Proven ability to reach deep into a complicated and uncharted research problem}{}
%\entry{\hspace{-.275em}}{$-$ Experienced in programming computational solutions to higher mathematical problems}{}
%\entry{\hspace{-.275em}}{$-$ Comfortable public speaker and communicator of mathematics with extensive teaching experience}{}
%\FINALentry{\hspace{-.275em}}{$-$ Enjoys working collaboratively to find the solutions to problems}{}

\CVsection{education}

\twolineentry{Ph.D.\ in Mathematics}{Massachusetts Institute of Technology, USA}{%
\dashtab GPA: 5.0/5.0\\%
\dashtab Topic: \emph{The Adams spectral sequence in the homotopy theory of simplicial algebras}\\%
\dashtab Developed Maple program to assist in conjecturing the outcome of a complex,\\
\whitetab unexplored calculation; improved program iteratively for massive time and storage\\
\whitetab economies; proved conjectures by developing extensive new mathematical theory \\
%%%
%%%
%%%\dashtab built partial calculational techniques for unexplored mathematical question; wrote\\
%%%\whitetab Maple program to execute complex calculations; iteratively modified program to\\
%%%\whitetab create massive storage and time economies; used output to conjecture general\\
%%%\whitetab   solution; constructed extensive mathematical machinery to prove conjecture\\
%\dashtab four upcoming invited talks on completed research\\
\dashtab Four preprints in preparation detailing research
}{expected June 2015}
\twolineentry{B.Sc in Pure Mathematics}{University of Sydney, Aust.}{%
\dashtab Awarded First Class Honours and a University Medal\\
\dashtab Thesis: \emph{The Cellularity of Certain Hecke Quotients}
%\dashtab built Gap 3 program to analyse idempotents in the Hecke algebra; used resulting\\
%\whitetab data to conjecture and prove a structural algebraic result.
}{2010}
\FINALtwolineentry{B.Com (Actuarial Studies) with B.Sc (Mathematics)}{Macquarie University, Aust.}{
\dashtab GPA: 3.96/4.00\\
\dashtab Selected actuarial coursework: probability; statistical data analysis; mathematics\\
\whitetab of finance; financial accounting and management; quantitative methods for asset\\
\whitetab liability management; survival models; general insurance pricing and reserving
}{2009}
%\FINALtwolineentry{B.Com (Actuarial Studies) with B.Sc (Mathematics)}{Macquarie University, Australia}{
%\dashtab GPA: 3.96/4.00\\
%\dashtab Actuarial Coursework Summary: combinatorial probability; financial accounting and\\
%\whitetab management; statistical data analysis; mathematics of finance; insurance and superannuation\\
%\whitetab practice; quantitative methods for asset liability management; survival models; contingent\\
%\whitetab payments; general insurance pricing and reserving
%}{2009}
\CVsection{experience\\and skills}
\twolineentry{MSRI Algebraic Topology program}{associate member}{\dashtab Collaborated in a synergistic forum for a broad range of expert topologists}{2014}
\twolineentry{Recitation Instructor}{(MqU, USyd, MIT)}{
\dashtab Actuarial courses (`06-`08): combinatorial probability; mathematics of finance;\\
\whitetab contingent payments I \& II; mathematical theory of risk \\
\dashtab Math courses (`09-`14): integral calculus and modelling; linear algebra; ODEs
}{2006-2014}
\twolineentry{Casual Academic Staff}{Department of Actuarial Studies (MqU)}{%
\dashtab Temporary lecturer, ACST356 Mathematical Theory of Risk (2009)\\
\dashtab Assisted with grading and preparation of teaching material for numerous courses\\
\dashtab PR representative for the department of EFS, promoting the Macquarie\\\whitetab University actuarial studies program at career fairs%
}{2008-2009}
\twolineentry{Summer scholar}{Department of Mathematics (USyd)}{%
\dashtab Researched combinatorial problem on permutation groups
}{2006}
\FINALentry{Programming}{\,\texttt{Maple}; \LaTeX; \texttt{C++} (recreational)}{}

\CVsection{leadership}
\entry{MIT Algebraic Topology Seminar}{organiser}{2014}
\entry{MIT Math Department Retreat}{organizing committee}{2012-2013}
\FINALentry{Mathematics Directed Reading Program}{mentor}{2011,\,2013}

\CVsection{selected\\research\\talks}
\entry{University of Chicago Topology Seminar}{}{Dec 2014}%{12/2/2014}
\entry{Northwestern University Topology Seminar}{}{Dec 2014}%{12/1/2014}
\twolineentry{Ohio State University Topology Seminar}{}{\whitetab \emph{Koszul duality and unstable operations at the prime 2}}{Oct 2014}%{10/14/2014}
\twolineentry{Johns Hopkins University Topology Seminar}{}{\whitetab \emph{Calculating the Adams spectral sequence for a simplicial algebra sphere}}{Sep 2014}%{9/15/2014}
\FINALtwolineentry{Union College Mathematics Conference}{}{\whitetab \emph{The Adams spectral sequence for simplicial algebras}}{Oct 2013}%{10/20/2013}
%\twolineentry{MIT Juvitop topology seminar}{}{\emph{The construction of the EHP sequence}}{9/18/2013}
%\twolineentry{MIT Babytop topology seminar}{}{\emph{The EHP spectral sequence}}{8/10/2012}
%\FINALtwolineentry{MIT Pure Mathematics Graduate Student Seminar}{}{\emph{The Robinson-Schensted correspondence and the Kazhdan-Lusztig Basis}}{2/25/2011}

\CVsection{competitions}
\entry{CiSRA team puzzle competition}{third prize}{2008}
\entry{SUMS problem competition}{prizewinner}{2006-2008}%{`06,`07,`08}
\entry{COMP225.5 Macquarie University Programming competition}{runner up}{2005}
\FINALentry{ACM South Pacific Programming Contest}{contestant}{2005}

\CVsection{selected\\honors and\\awards} 
\entry{Charles and Holly Housman Award for Excellence in Teaching}{(MIT)}{2014}
\entry{MJ and M Ashby Prize}{for Mathematics in Science (USyd)}{2009}
\entry{Invited Speaker}{Dean's Excellence Awards, Macquarie University}{2008}
\FINALtwolineentry{Alf and Pearl Pollard Memorial Prize}{for the most outstanding undergraduate}{student in the division of Economic and Financial Studies (MqU)}{2008}
%\FINALentry{Assoc Prize}{for proficiency in ACST211: Combinatorial Probability (MqU)}{2005}
%\entry{Vice-Chancellor's Research Scholarship}{(declined, USyd)}{2010}
%\entry{University of Sydney Medal}{}{2009}
%\entry{University of Sydney Honours Scholarship}{}{2009}
%\entry{Barker Prize}{for Mathematics Honours (USyd)}{2009}
%\entry{Doris Wallent Scholarship}{for 300 level mathematics, division of ICS (MqU)}{2008}
%\entry{Frederic Chong Mathematics Prize}{for proficiency in 300 level math (MqU)}{2008}
%\entry{Alan McIntosh Analysis Prize}{for proficiency in 200 and 300 level analysis (MqU)}{2008}
%\entry{Hubert-Vaughn Prize}{for proficiency in ACST354: Survival Models (MqU)}{2007}
%\entry{2006-2007 Vacation Scholarship}{at USyd (AMSI/ICE-EM)}{2006}
%\entry{SUMS problem competition}{prizewinner (USyd)}{2006-2008}
%\entry{Frederic Chong Mathematics Prize}{for proficiency in 200 level math (MqU)}{2006}
%\entry{Merit Certificate}{Division of EFS (MqU)}{2005-2008}



%\twolineentry{Recitation Instructor}{Department of Mathematics (USyd)}{%
%\dashtab MATH1903: Integral Calculus and Modelling (Adv)\\
%\dashtab MATH1002: Linear Algebra\\
%\dashtab MATH1014: Introduction to Linear Algebra%
%}{2009}
%\twolineentry{Casual Academic Staff}{Department of Actuarial Studies (MqU)}{%
%\dashtab Temporary lecturer, ACST356: Mathematical Theory of Risk (2009)\\
%\dashtab exam grading and preparation of teaching material\\
%\dashtab PR representative for the department of EFS, promoting the Macquarie\\\whitetab University actuarial studies progam at career fairs%
%}{2008-2009}
%\FINALtwolineentry{Recitation Instructor}{Department of Actuarial Studies (MqU)}{%
%\dashtab ACST211: Combinatorial Probability (2006-2008)\\
%\dashtab ACST200: Mathematics of Finance (2007-2008)\\
%\dashtab ACST255: Contingent Payments I (2008)\\
%\dashtab ACST355: Contingent Payments II (2008)\\
%\dashtab ACST356: Mathematical Theory of Risk (2008)%
%}{2006-2008}



\end{resume}
\end{document}

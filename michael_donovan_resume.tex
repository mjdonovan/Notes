% !TEX root = z_output/_michael_donovan_resume.tex
% @brief    LaTeX2e Resume for Kamil K Wojcicki
\documentclass[margin,line]{resume}
\usepackage{mathrsfs}
\usepackage{amsmath, amsthm, amssymb, graphicx}
\usepackage[bookmarks=false,pdftex,pdfborder={0 0 0 [1 1]}]{hyperref}
\usepackage{cancel}

\headheight=-2pt
\topmargin=8pt
\topmargin=2pt
\topmargin=-3pt
\textheight=634pt
\textheight=654pt
\textheight=670pt
%\textwidth=440pt
%\oddsidemargin=-12pt
%\evensidemargin=8pt




\newcommand{\CVsection}[1]{\section{\mysidestyle #1}}
\newcommand{\entry}[3]{\textbf{#1} #2 \hfill {#3}
           
\vspace{-3.1mm}}
\newcommand{\twolineentry}[4]{\textbf{#1} #2 \hfill {#4}\\%
#3
           
\vspace{-3.1mm}}



\newcommand{\FINALentry}[3]{\textbf{#1} #2 \hfill {#3}}
\newcommand{\FINALtwolineentry}[4]{\textbf{#1} #2 \hfill {#4}\\%
#3}



%- change the font on your email & web address to something less nerdy, it's best to use just one font for everything & use one size except for the header
%- there's a lot of bold text. Maybe the years don't have to be bold? Maybe something else can go. It's hard on the eyes if everything is bold
%- only show programming experience if you think it's plausible the employer would need it. For a business application, you might do a section on skills and include 1 line on programming amongst a couple of fluffier things like mentoring, facilitation etc
%- if you need more space, perhaps you don't need to list where the AMSI summer schools were, that could save 3 lines
%- a lot of Aus cvs have one line on recreation at the end. It goes something like 'in my free time I enjoy soccer, squash, baking fresh bread, socialising with uni club XYZ and travel' or similar. I don't know if that's the go in the US but we sometimes like to see it on cvs that are otherwise pretty academic only 


\renewcommand{\namefont}{}

\begin{document}
\name{{\Large\bf Michael Donovan}\qquad \qquad \qquad \qquad \qquad \qquad \qquad \ \  \hfil 857-891-2805 \ $\cdot$ \  mdono@math.mit.edu}
\begin{resume}
%\CVsection{Overview}
%\entry{\hspace{-.275em}}{$-$ Mathematics Ph.D.\ student at MIT with actuarial science background}{}
%\entry{\hspace{-.275em}}{$-$ Proven ability to reach deep into a complicated and uncharted research problem}{}
%\entry{\hspace{-.275em}}{$-$ Experienced in programming computational solutions to higher mathematical problems}{}
%\entry{\hspace{-.275em}}{$-$ Comfortable public speaker and communicator of mathematics with extensive teaching experience}{}
%\FINALentry{\hspace{-.275em}}{$-$ Enjoys working collaboratively to find the solutions to problems}{}

\CVsection{Education}

\twolineentry{Ph.D.\ in Mathematics}{Massachusetts Institute of Technology, USA}{%
\phantom{space}$-$ GPA: 5.0/5.0\\%
\phantom{space}$-$ Topic: \emph{The Adams spectral sequence in the homotopy theory of simplicial algebras}\\%
\phantom{space}$-$ built partial calculational techniques for unexplored mathematical question; built Maple\\
\phantom{space$-$ }program to execute complex calculations; iteratively modified program to create massive\\
\phantom{space$-$ }storage and time economies; used program output to conjecture general solution; built\\
\phantom{space$-$ }extensive mathematical machinery to prove conjectured solution\\
%\phantom{space}$-$ four upcoming invited talks on completed research\\
\phantom{space}$-$ four preprints in preparation detailing research
}{June 2015 (expected)}
\twolineentry{B.Sc in Pure Mathematics}{University of Sydney, Australia}{%
\phantom{space}$-$ Awarded First Class Honours and a University Medal\\
\phantom{space}$-$ Thesis: \emph{The Cellularity of Certain Hecke Quotients}\\
\phantom{space}$-$ built Gap 3 program to analyse idempotents in the Hecke algebra; used resulting data to\\
\phantom{space$-$ }conjecture and prove a structural algebraic result.
}{2010}
\FINALtwolineentry{B.Com (Actuarial Studies) with B.Sc (Mathematics)}{Macquarie University, Australia}{
\phantom{space}$-$ GPA: 3.96/4.00\\
\phantom{space}$-$ Selected actuarial coursework: probability; statistical data analysis; mathematics of finance;\\
\phantom{space$-$ }financial accounting and management; quantitative methods for asset liability management;\\
\phantom{space$-$ }survival models; general insurance pricing and reserving
}{2009}
%\FINALtwolineentry{B.Com (Actuarial Studies) with B.Sc (Mathematics)}{Macquarie University, Australia}{
%\phantom{space}$-$ GPA: 3.96/4.00\\
%\phantom{space}$-$ Actuarial Coursework Summary: combinatorial probability; financial accounting and\\
%\phantom{space$-$ }management; statistical data analysis; mathematics of finance; insurance and superannuation\\
%\phantom{space$-$ }practice; quantitative methods for asset liability management; survival models; contingent\\
%\phantom{space$-$ }payments; general insurance pricing and reserving
%}{2009}
\CVsection{Experience\\and skills}

\twolineentry{Recitation Instructor}{(MqU, USyd, MIT)}{
\phantom{space}$-$ Actuarial courses (`06-`08): combinatorial probability; mathematics of finance;\\
\phantom{space$-$ }contingent payments I \& II; mathematical theory of risk \\
\phantom{space}$-$ Math courses (`09-`14): integral calculus and modelling; linear algebra; ODEs
}{2006-present}
\twolineentry{Casual Academic Staff}{Department of Actuarial Studies (MqU)}{%
\phantom{space}$-$ Temporary lecturer, ACST356 Mathematical Theory of Risk (2009)\\
\phantom{space}$-$ assisted with grading and preparation of teaching material for numerous courses\\
\phantom{space}$-$ PR representative for the department of EFS, promoting the Macquarie\\\phantom{space$-$ }University actuarial studies progam at career fairs%
}{2008-2009}
\twolineentry{Vacation scholar}{Department of Mathematics (USyd)}{%
\phantom{space}$-$ Researched combinatorial problem on permutation groups
}{Summer `06-`07}
\FINALentry{Programming}{\texttt{Maple} (adv.); \LaTeX\ (adv.); \texttt{C++} (recreational)}{}

\CVsection{Selected\\Conference\\and seminar\\talks}
\entry{University of Chicago Topology Seminar}{}{12/2/2014}
\entry{Northwestern University Topology Seminar}{}{12/1/2014}
\twolineentry{Ohio State University Topology Seminar}{}{\emph{Koszul duality and unstable operations at the prime 2}}{10/14/2014}
\twolineentry{Johns Hopkins University Topology Seminar}{}{\emph{Calculating the Adams spectral sequence for a simplicial algebra sphere}}{9/15/2014}
\FINALtwolineentry{Union College Mathematics Conference}{}{\emph{The Adams spectral sequence for simplicial algebras}}{10/20/2013}
%\twolineentry{MIT Juvitop topology seminar}{}{\emph{The construction of the EHP sequence}}{9/18/2013}
%\twolineentry{MIT Babytop topology seminar}{}{\emph{The EHP spectral sequence}}{8/10/2012}
%\FINALtwolineentry{MIT Pure Mathematics Graduate Student Seminar}{}{\emph{The Robinson-Schensted correspondence and the Kazhdan-Lusztig Basis}}{2/25/2011}

\CVsection{Selected\\honors,\\awards and\\scholarships} 
\entry{Charles and Holly Housman Award for Excellence in Teaching}{(MIT)}{2014}
\entry{MJ and M Ashby Prize}{for Mathematics in Science (USyd)}{2009}
\entry{Invited Speaker}{Dean's Excellence Awards, Macquarie University}{2008}
\twolineentry{Alf and Pearl Pollard Memorial Prize}{for the most outstanding undergraduate}{student in the division of Economic and Financial Studies (MqU)}{2008}
\entry{Assoc Prize}{for proficiency in ACST211: Combinatorial Probability (MqU)}{2005}
%\entry{Vice-Chancellor's Research Scholarship}{(declined, USyd)}{2010}
%\entry{University of Sydney Medal}{}{2009}
%\entry{University of Sydney Honours Scholarship}{}{2009}
%\entry{Barker Prize}{for Mathematics Honours (USyd)}{2009}
%\entry{Doris Wallent Scholarship}{for 300 level mathematics, division of ICS (MqU)}{2008}
%\entry{Frederic Chong Mathematics Prize}{for proficiency in 300 level math (MqU)}{2008}
%\entry{Alan McIntosh Analysis Prize}{for proficiency in 200 and 300 level analysis (MqU)}{2008}
%\entry{Hubert-Vaughn Prize}{for proficiency in ACST354: Survival Models (MqU)}{2007}
%\entry{2006-2007 Vacation Scholarship}{at USyd (AMSI/ICE-EM)}{2006}
%\entry{SUMS problem competition}{prizewinner (USyd)}{2006-2008}
%\entry{Frederic Chong Mathematics Prize}{for proficiency in 200 level math (MqU)}{2006}
%\entry{Merit Certificate}{Division of EFS (MqU)}{2005-2008}



%\twolineentry{Recitation Instructor}{Department of Mathematics (USyd)}{%
%\phantom{space}$-$ MATH1903: Integral Calculus and Modelling (Adv)\\
%\phantom{space}$-$ MATH1002: Linear Algebra\\
%\phantom{space}$-$ MATH1014: Introduction to Linear Algebra%
%}{2009}
%\twolineentry{Casual Academic Staff}{Department of Actuarial Studies (MqU)}{%
%\phantom{space}$-$ Temporary lecturer, ACST356: Mathematical Theory of Risk (2009)\\
%\phantom{space}$-$ exam grading and preparation of teaching material\\
%\phantom{space}$-$ PR representative for the department of EFS, promoting the Macquarie\\\phantom{space$-$ }University actuarial studies progam at career fairs%
%}{2008-2009}
%\FINALtwolineentry{Recitation Instructor}{Department of Actuarial Studies (MqU)}{%
%\phantom{space}$-$ ACST211: Combinatorial Probability (2006-2008)\\
%\phantom{space}$-$ ACST200: Mathematics of Finance (2007-2008)\\
%\phantom{space}$-$ ACST255: Contingent Payments I (2008)\\
%\phantom{space}$-$ ACST355: Contingent Payments II (2008)\\
%\phantom{space}$-$ ACST356: Mathematical Theory of Risk (2008)%
%}{2006-2008}

\CVsection{Competitions (provisional)}
\entry{CiSRA team puzzle competition}{third prize}{2008}
\entry{SUMS problem competition}{prizewinner}{06,07,08}
\entry{COMP225.5 Macquarie University Programming competition}{Runner Up}{2005}
\FINALentry{ACM South Pacific Programming Contest}{contestant}{2005}

\end{resume}
\end{document}
\pagebreak


\CVsection{Contact\\Information}

75 Pleasant St, Apt \#2 \hfill {mdono@math.mit.edu}\\
Cambridge, MA  02139 \hfill {math.mit.edu/$\sim$mdono/}\\
USA \hfill +1 857 891 2805





% Speech processing, speech enhancement...\\ 
% machine learning and pattern recognition.


\CVsection{Education}

\twolineentry{Ph.D.\ in Mathematics}{Massachusetts Institute of Technology, USA}{%
\phantom{space}$-$ Advisor: Prof. Haynes Miller\\
\phantom{space}$-$ Thesis topic: The Adams spectral sequence in the homotopy theory of simplicial algebras%
}{May 2015 (expected)}
\twolineentry{B.Sc in Pure Mathematics}{University of Sydney, Australia}{%
\phantom{space}$-$ with First Class Honours and a University Medal\\
\phantom{space}$-$ Advisor: Prof. Gustav Lehrer\\
\phantom{space}$-$ Thesis title: \emph{The Cellularity of Certain Hecke Quotients}
}{2010}
\FINALentry{B.Com (Actuarial Studies) with B.Sc (Mathematics)}{Macquarie University, Australia}{2009}


\CVsection{Preprints in preparation}

\entry{\hspace{-.275em}}{\emph{On the Adams spectral sequence for commutative algebras}}{}
\entry{\hspace{-.275em}}{\emph{Operations in the spectral sequence of a mixed simplicial $\mathbb{F}_2$-algebra}}{}


\CVsection{Work in \\progress}

\entry{\hspace{-.275em}}{\emph{The homology of unstable Lie coalgebras}}{}
\FINALentry{\hspace{-.275em}}{\emph{The Adams spectral sequence for a simplicial commutative algebra
sphere}}{}


\CVsection{Upcoming \\talks}
\entry{University of Chicago Topology Seminar}{}{12/2/2014}
\entry{Northwestern University Topology Seminar}{}{12/1/2014}
\entry{Ohio State University Topology Seminar}{}{10/14/2014}
\twolineentry{Johns Hopkins University Topology Seminar}{}{\emph{Calculating the Adams spectral sequence for a simplicial algebra sphere}}{9/15/2014}
\vspace{-2mm}
\CVsection{Past talks}
\twolineentry{Union College Mathematics Conference}{}{\emph{The Adams spectral sequence for simplicial algebras}}{10/20/2013}
\twolineentry{MIT Juvitop topology seminar}{}{\emph{The construction of the EHP sequence}}{9/18/2013}
\twolineentry{MIT Babytop topology seminar}{}{\emph{The EHP spectral sequence}}{8/10/2012}
\FINALtwolineentry{MIT Pure Mathematics Graduate Student Seminar}{}{\emph{The Robinson-Schensted correspondence and the Kazhdan-Lusztig Basis}}{2/25/2011}


\CVsection{Conferences \\ \& Workshops}

\entry{Algebraic Topology semester long program}{MSRI, Berkeley}{2014}
\entry{MIT Talbot Workshop --- Calculus of Functors}{Garden City, Utah}{2012}
\entry{Young Topologists Meeting}{University of Copenhagen, Denmark}{2012}
\entry{Graduate Student Topology Conference}{Indiana University}{2012}
\entry{Union College Mathematics Conference}{Algebraic Topology session}{2011,2013}
\FINALentry{AMSI/ICE-EM Mathematics Summer Schools}{}%{%
%\phantom{space}$-$ University of Sydney (2007)\\
%\phantom{space}$-$ Monash University, Melbourne (2008)\\
%\phantom{space}$-$ University of Wollongong (2009)}
{2007-2009}


\CVsection{Awards and\\Scholarships} 

\entry{Charles and Holly Housman Award for Excellence in Teaching}{(MIT)}{2014}
\entry{Vice-Chancellor's Research Scholarship}{(declined, USyd)}{2010}
\entry{University of Sydney Medal}{}{2009}
\entry{University of Sydney Honours Scholarship}{}{2009}
\entry{MJ and M Ashby Prize}{for Mathematics in Science (USyd)}{2009}
\entry{Barker Prize}{for Mathematics Honours (USyd)}{2009}
\twolineentry{Alf and Pearl Pollard Memorial Prize}{for the most outstanding undergraduate}{student in the division of Economic and Financial Studies (MqU)}{2008}
\entry{Doris Wallent Scholarship}{for 300 level mathematics, division of ICS (MqU)}{2008}
\entry{Frederic Chong Mathematics Prize}{for proficiency in 300 level math (MqU)}{2008}
\entry{Alan McIntosh Analysis Prize}{for proficiency in 200 and 300 level analysis (MqU)}{2008}
\entry{Hubert-Vaughn Prize}{for proficiency in ACST354: Survival Models (MqU)}{2007}
\entry{2006-2007 Vacation Scholarship}{at USyd (AMSI/ICE-EM)}{2006}
\entry{SUMS problem competition}{prizewinner (USyd)}{2006-2008}
\entry{Frederic Chong Mathematics Prize}{for proficiency in 200 level math (MqU)}{2006}
\entry{Merit Certificate}{Division of EFS (MqU)}{2005-2008}
\FINALentry{Assoc Prize}{for proficiency in ACST211: Combinatorial Probability (MqU)}{2005}


\CVsection{Teaching and Work Experience}

\twolineentry{Recitation Instructor}{Department of Mathematics (MIT)}{\phantom{space}$-$ 18.03: Ordinary Differential Equations}{2012-present}
\twolineentry{Grader}{Department of Mathematics (MIT)}{%
\phantom{space}$-$ 18.725: Algebraic Geometry I\\
%\phantom{space}$-$ 18.726: Algebraic Geometry II\\
\phantom{space}$-$ 18.906: Algebraic Topology II%
}{2011-2012}
\twolineentry{Recitation Instructor}{Department of Mathematics (USyd)}{%
\phantom{space}$-$ MATH1903: Integral Calculus and Modelling (Adv)\\
\phantom{space}$-$ MATH1002: Linear Algebra\\
\phantom{space}$-$ MATH1014: Introduction to Linear Algebra%
}{2009}
\twolineentry{Casual Academic Staff}{Department of Actuarial Studies (MqU)}{%
\phantom{space}$-$ exam grading and preparation of teaching material\\
\phantom{space}$-$ PR representative for the department of EFS%
}{2008-2009}
\FINALtwolineentry{Recitation Instructor}{Department of Actuarial Studies (MqU)}{%
\phantom{space}$-$ ACST211: Combinatorial Probability (2006-2008)\\
\phantom{space}$-$ ACST200: Mathematics of Finance (2007-2008)\\
\phantom{space}$-$ ACST255: Contingent Payments I (2008)\\
\phantom{space}$-$ ACST355: Contingent Payments II (2008)\\
\phantom{space}$-$ ACST356: Mathematical Theory of Risk (2008)%
}{2006-2008}


\CVsection{Undergraduate \\ Research}

\entry{The cellularity of certain Hecke quotients}{Supervisor: G.\ I.\ Lehrer, (USyd)}{2009}
\FINALentry{The permuation degrees of finite groups}{Supervisor: A.\ Henderson, (USyd)}{2006-2007}


\CVsection{Service}

\entry{MIT Algebraic Topology Seminar}{Organiser}{2014}
\entry{Directed Reading Program}{Mentor (MIT)}{2011,2013}
\entry{MIT Pure Math Graduate Student Seminar}{Organiser}{2011-2012}
\entry{MIT Math Department Retreat}{Planning Committee Member}{2012-2013}
\FINALentry{Macquarie University Soccer Club}{Team Manager \& Club Secretary}{2008}


\CVsection{Programming}

\entry{Languages:}{\texttt{Maple} (adv.), \LaTeX\ (adv.), \texttt{C++} (int.).}{}
\entry{COMP225.5 Macquarie University Programming competition}{Runner Up}{2005}
\FINALentry{ACM South Pacific Programming Contest}{Contestant}{2005}


\CVsection{Recreation}

In my free time I enjoy playing soccer, squash and street hockey, cycling, baking \\fresh bread and cooking with friends.

%\CVsection{Referees} 
%    {\sl Available on request.}

%\CVsection{References} 
%
%\textbf{Scott Sheffield}, Professor of Mathematics, Department of Mathematics, MIT. \vspace{2mm}\\%
%\textsl{e-mail:} \texttt{sheffield@math.mit.edu} \vspace{1mm}\\%
%\textsl{phone:} +1 617 253 4350
%
%\textbf{Vojkan Jaksic}, Professor of Mathematics, Department of Mathematics and Statistics, McGill University. \vspace{2mm}\\%
%\textsl{e-mail:} \texttt{jaksic@math.mcgill.ca} \vspace{1mm}\\%
%\textsl{phone:} +1 514 398 3827

% \begin{tabular}{@{}p{6cm}p{6cm}}
% \textbf{Professor Kuldip Paliwal}       &  \textbf{Dr Stephen So}                   \\
% Professor                               &  Associate Lecturer                       \\
% Griffith University                     &  Griffith University                      \\
% Nathan, Queensland, Australia           &  Gold Coast, Queensland, Australia        \\
% phone: \textsl{available on request}    &  phone: \textsl{available on request}     \\
% e-mail: \textsl{available on request}   &  e-mail: \textsl{available on request}    \\
% \end{tabular}

% \begin{tabular}{@{}p{6cm}p{6cm}}
% \textbf{Dr Conrad Sanderson}            &  \textbf{Mr Sean Loye}                    \\
% Researcher                              &  Systems Engineer                         \\
% National ICT Australia                  &  Hewlett Packard                          \\
% St Lucia, Queensland, Australia         &  Milton, Queensland, Australia            \\
% phone: \textsl{available on request}    &  phone: \textsl{available on request}     \\
% e-mail: \textsl{available on request}   &  e-mail: \textsl{available on request}    \\
% \end{tabular}




\end{resume}
\end{document}



% EOF


% !TEX root = z_output/Ravenel.tex

\documentclass[11pt]{article}
\usepackage{fullpage}
\usepackage{amsmath,amsthm,amssymb}
\usepackage{mathrsfs,nicefrac}
\usepackage{amssymb}
\usepackage{epsfig}
\usepackage[all,2cell]{xy}
\usepackage{sseq}
\usepackage{tocloft}
\usepackage{cancel}
\usepackage[strict]{changepage}
\usepackage{color}
\usepackage{tikz}
\usepackage{extpfeil}
\usepackage{version}
\usepackage{framed}
\definecolor{shadecolor}{rgb}{.925,0.925,0.925}

%\usepackage{ifthen}
%Used for disabling hyperref
\ifx\dontloadhyperref\undefined
%\usepackage[pdftex,pdfborder={0 0 0 [1 1]}]{hyperref}
\usepackage[pdftex,pdfborder={0 0 .5 [1 1]}]{hyperref}
\else
\providecommand{\texorpdfstring}[2]{#1}
\fi
%>>>>>>>>>>>>>>>>>>>>>>>>>>>>>>
%<<<        Versions        <<<
%>>>>>>>>>>>>>>>>>>>>>>>>>>>>>>
%Add in the following line to include all the versions.
%\def\excludeversion#1{\includeversion{#1}}

%>>>>>>>>>>>>>>>>>>>>>>>>>>>>>>
%<<<       Better ToC       <<<
%>>>>>>>>>>>>>>>>>>>>>>>>>>>>>>
\setlength{\cftbeforesecskip}{0.5ex}

%>>>>>>>>>>>>>>>>>>>>>>>>>>>>>>
%<<<      Hyperref mod      <<<
%>>>>>>>>>>>>>>>>>>>>>>>>>>>>>>

%needs more testing
\newcounter{dummyforrefstepcounter}
\newcommand{\labelRIGHTHERE}[1]
{\refstepcounter{dummyforrefstepcounter}\label{#1}}


%>>>>>>>>>>>>>>>>>>>>>>>>>>>>>>
%<<<  Theorem Environments  <<<
%>>>>>>>>>>>>>>>>>>>>>>>>>>>>>>
\ifx\dontloaddefinitionsoftheoremenvironments\undefined
\theoremstyle{plain}
\newtheorem{thm}{Theorem}[section]
\newtheorem*{thm*}{Theorem}
\newtheorem{lem}[thm]{Lemma}
\newtheorem*{lem*}{Lemma}
\newtheorem{prop}[thm]{Proposition}
\newtheorem*{prop*}{Proposition}
\newtheorem{cor}[thm]{Corollary}
\newtheorem*{cor*}{Corollary}
\newtheorem{defprop}[thm]{Definition-Proposition}
\newtheorem*{punchline}{Punchline}
\newtheorem*{conjecture}{Conjecture}
\newtheorem*{claim}{Claim}

\theoremstyle{definition}
\newtheorem{defn}{Definition}[section]
\newtheorem*{defn*}{Definition}
\newtheorem{exmp}{Example}[section]
\newtheorem*{exmp*}{Example}
\newtheorem*{exmps*}{Examples}
\newtheorem*{nonexmp*}{Non-example}
\newtheorem{asspt}{Assumption}[section]
\newtheorem{notation}{Notation}[section]
\newtheorem{exercise}{Exercise}[section]
\newtheorem*{fact*}{Fact}
\newtheorem*{rmk*}{Remark}
\newtheorem{fact}{Fact}
\newtheorem*{aside}{Aside}
\newtheorem*{question}{Question}
\newtheorem*{answer}{Answer}

\else\relax\fi

%>>>>>>>>>>>>>>>>>>>>>>>>>>>>>>
%<<<      Fields, etc.      <<<
%>>>>>>>>>>>>>>>>>>>>>>>>>>>>>>
\DeclareSymbolFont{AMSb}{U}{msb}{m}{n}
\DeclareMathSymbol{\N}{\mathbin}{AMSb}{"4E}
\DeclareMathSymbol{\Octonions}{\mathbin}{AMSb}{"4F}
\DeclareMathSymbol{\Z}{\mathbin}{AMSb}{"5A}
\DeclareMathSymbol{\R}{\mathbin}{AMSb}{"52}
\DeclareMathSymbol{\Q}{\mathbin}{AMSb}{"51}
\DeclareMathSymbol{\PP}{\mathbin}{AMSb}{"50}
\DeclareMathSymbol{\I}{\mathbin}{AMSb}{"49}
\DeclareMathSymbol{\C}{\mathbin}{AMSb}{"43}
\DeclareMathSymbol{\A}{\mathbin}{AMSb}{"41}
\DeclareMathSymbol{\F}{\mathbin}{AMSb}{"46}
\DeclareMathSymbol{\G}{\mathbin}{AMSb}{"47}
\DeclareMathSymbol{\Quaternions}{\mathbin}{AMSb}{"48}


%>>>>>>>>>>>>>>>>>>>>>>>>>>>>>>
%<<<       Operators        <<<
%>>>>>>>>>>>>>>>>>>>>>>>>>>>>>>
\DeclareMathOperator{\ad}{\textbf{ad}}
\DeclareMathOperator{\coker}{coker}
\renewcommand{\ker}{\textup{ker}\,}
\DeclareMathOperator{\End}{End}
\DeclareMathOperator{\Aut}{Aut}
\DeclareMathOperator{\Hom}{Hom}
\DeclareMathOperator{\Maps}{Maps}
\DeclareMathOperator{\Mor}{Mor}
\DeclareMathOperator{\Gal}{Gal}
\DeclareMathOperator{\Ext}{Ext}
\DeclareMathOperator{\Tor}{Tor}
\DeclareMathOperator{\Map}{Map}
\DeclareMathOperator{\Der}{Der}
\DeclareMathOperator{\Rad}{Rad}
\DeclareMathOperator{\rank}{rank}
\DeclareMathOperator{\ArfInvariant}{Arf}
\DeclareMathOperator{\KervaireInvariant}{Ker}
\DeclareMathOperator{\im}{im}
\DeclareMathOperator{\coim}{coim}
\DeclareMathOperator{\trace}{tr}
\DeclareMathOperator{\supp}{supp}
\DeclareMathOperator{\ann}{ann}
\DeclareMathOperator{\spec}{Spec}
\DeclareMathOperator{\SPEC}{\textbf{Spec}}
\DeclareMathOperator{\proj}{Proj}
\DeclareMathOperator{\PROJ}{\textbf{Proj}}
\DeclareMathOperator{\fiber}{F}
\DeclareMathOperator{\cofiber}{C}
\DeclareMathOperator{\cone}{cone}
\DeclareMathOperator{\skel}{sk}
\DeclareMathOperator{\coskel}{cosk}
\DeclareMathOperator{\conn}{conn}
\DeclareMathOperator{\colim}{colim}
\DeclareMathOperator{\limit}{lim}
\DeclareMathOperator{\ch}{ch}
\DeclareMathOperator{\Vect}{Vect}
\DeclareMathOperator{\GrthGrp}{GrthGp}
\DeclareMathOperator{\Sym}{Sym}
\DeclareMathOperator{\Prob}{\mathbb{P}}
\DeclareMathOperator{\Exp}{\mathbb{E}}
\DeclareMathOperator{\GeomMean}{\mathbb{G}}
\DeclareMathOperator{\Var}{Var}
\DeclareMathOperator{\Cov}{Cov}
\DeclareMathOperator{\Sp}{Sp}
\DeclareMathOperator{\Seq}{Seq}
\DeclareMathOperator{\Cyl}{Cyl}
\DeclareMathOperator{\Ev}{Ev}
\DeclareMathOperator{\sh}{sh}
\DeclareMathOperator{\intHom}{\underline{Hom}}
\DeclareMathOperator{\Frac}{frac}



%>>>>>>>>>>>>>>>>>>>>>>>>>>>>>>
%<<<   Cohomology Theories  <<<
%>>>>>>>>>>>>>>>>>>>>>>>>>>>>>>
\DeclareMathOperator{\KR}{{K\R}}
\DeclareMathOperator{\KO}{{KO}}
\DeclareMathOperator{\K}{{K}}
\DeclareMathOperator{\OmegaO}{{\Omega_{\Octonions}}}

%>>>>>>>>>>>>>>>>>>>>>>>>>>>>>>
%<<<   Algebraic Geometry   <<<
%>>>>>>>>>>>>>>>>>>>>>>>>>>>>>>
\DeclareMathOperator{\Spec}{Spec}
\DeclareMathOperator{\Proj}{Proj}
\DeclareMathOperator{\Sing}{Sing}
\DeclareMathOperator{\shfHom}{\mathscr{H}\textit{\!\!om}}
\DeclareMathOperator{\WeilDivisors}{{Div}}
\DeclareMathOperator{\CartierDivisors}{{CaDiv}}
\DeclareMathOperator{\PrincipalWeilDivisors}{{PrDiv}}
\DeclareMathOperator{\LocallyPrincipalWeilDivisors}{{LPDiv}}
\DeclareMathOperator{\PrincipalCartierDivisors}{{PrCaDiv}}
\DeclareMathOperator{\DivisorClass}{{Cl}}
\DeclareMathOperator{\CartierClass}{{CaCl}}
\DeclareMathOperator{\Picard}{{Pic}}
\DeclareMathOperator{\Frob}{Frob}


%>>>>>>>>>>>>>>>>>>>>>>>>>>>>>>
%<<<  Mathematical Objects  <<<
%>>>>>>>>>>>>>>>>>>>>>>>>>>>>>>
\newcommand{\sll}{\mathfrak{sl}}
\newcommand{\gl}{\mathfrak{gl}}
\newcommand{\GL}{\mbox{GL}}
\newcommand{\PGL}{\mbox{PGL}}
\newcommand{\SL}{\mbox{SL}}
\newcommand{\Mat}{\mbox{Mat}}
\newcommand{\Gr}{\textup{Gr}}
\newcommand{\Squ}{\textup{Sq}}
\newcommand{\catSet}{\textit{Sets}}
\newcommand{\RP}{{\R\PP}}
\newcommand{\CP}{{\C\PP}}
\newcommand{\Steen}{\mathscr{A}}
\newcommand{\Orth}{\textup{\textbf{O}}}

%>>>>>>>>>>>>>>>>>>>>>>>>>>>>>>
%<<<  Mathematical Symbols  <<<
%>>>>>>>>>>>>>>>>>>>>>>>>>>>>>>
\newcommand{\DASH}{\textup{---}}
\newcommand{\op}{\textup{op}}
\newcommand{\CW}{\textup{CW}}
\newcommand{\ob}{\textup{ob}\,}
\newcommand{\ho}{\textup{ho}}
\newcommand{\st}{\textup{st}}
\newcommand{\id}{\textup{id}}
\newcommand{\Bullet}{\ensuremath{\bullet} }
\newcommand{\sprod}{\wedge}

%>>>>>>>>>>>>>>>>>>>>>>>>>>>>>>
%<<<      Some Arrows       <<<
%>>>>>>>>>>>>>>>>>>>>>>>>>>>>>>
\newcommand{\nt}{\Longrightarrow}
\let\shortmapsto\mapsto
\let\mapsto\longmapsto
\newcommand{\mapsfrom}{\,\reflectbox{$\mapsto$}\ }
\newcommand{\bigrightsquig}{\scalebox{2}{\ensuremath{\rightsquigarrow}}}
\newcommand{\bigleftsquig}{\reflectbox{\scalebox{2}{\ensuremath{\rightsquigarrow}}}}

%\newcommand{\cofibration}{\xhookrightarrow{\phantom{\ \,{\sim\!}\ \ }}}
%\newcommand{\fibration}{\xtwoheadrightarrow{\phantom{\sim\!}}}
%\newcommand{\acycliccofibration}{\xhookrightarrow{\ \,{\sim\!}\ \ }}
%\newcommand{\acyclicfibration}{\xtwoheadrightarrow{\sim\!}}
%\newcommand{\leftcofibration}{\xhookleftarrow{\phantom{\ \,{\sim\!}\ \ }}}
%\newcommand{\leftfibration}{\xtwoheadleftarrow{\phantom{\sim\!}}}
%\newcommand{\leftacycliccofibration}{\xhookleftarrow{\ \ {\sim\!}\,\ }}
%\newcommand{\leftacyclicfibration}{\xtwoheadleftarrow{\sim\!}}
%\newcommand{\weakequiv}{\xrightarrow{\ \,\sim\,\ }}
%\newcommand{\leftweakequiv}{\xleftarrow{\ \,\sim\,\ }}

\newcommand{\cofibration}
{\xhookrightarrow{\phantom{\ \,{\raisebox{-.3ex}[0ex][0ex]{\scriptsize$\sim$}\!}\ \ }}}
\newcommand{\fibration}
{\xtwoheadrightarrow{\phantom{\raisebox{-.3ex}[0ex][0ex]{\scriptsize$\sim$}\!}}}
\newcommand{\acycliccofibration}
{\xhookrightarrow{\ \,{\raisebox{-.55ex}[0ex][0ex]{\scriptsize$\sim$}\!}\ \ }}
\newcommand{\acyclicfibration}
{\xtwoheadrightarrow{\raisebox{-.6ex}[0ex][0ex]{\scriptsize$\sim$}\!}}
\newcommand{\leftcofibration}
{\xhookleftarrow{\phantom{\ \,{\raisebox{-.3ex}[0ex][0ex]{\scriptsize$\sim$}\!}\ \ }}}
\newcommand{\leftfibration}
{\xtwoheadleftarrow{\phantom{\raisebox{-.3ex}[0ex][0ex]{\scriptsize$\sim$}\!}}}
\newcommand{\leftacycliccofibration}
{\xhookleftarrow{\ \ {\raisebox{-.55ex}[0ex][0ex]{\scriptsize$\sim$}\!}\,\ }}
\newcommand{\leftacyclicfibration}
{\xtwoheadleftarrow{\raisebox{-.6ex}[0ex][0ex]{\scriptsize$\sim$}\!}}
\newcommand{\weakequiv}
{\xrightarrow{\ \,\raisebox{-.3ex}[0ex][0ex]{\scriptsize$\sim$}\,\ }}
\newcommand{\leftweakequiv}
{\xleftarrow{\ \,\raisebox{-.3ex}[0ex][0ex]{\scriptsize$\sim$}\,\ }}

%>>>>>>>>>>>>>>>>>>>>>>>>>>>>>>
%<<<    xymatrix Arrows     <<<
%>>>>>>>>>>>>>>>>>>>>>>>>>>>>>>
\newdir{ >}{{}*!/-5pt/@{>}}
\newcommand{\xycof}{\ar@{ >->}}
\newcommand{\xycofib}{\ar@{^{(}->}}
\newcommand{\xycofibdown}{\ar@{_{(}->}}
\newcommand{\xyfib}{\ar@{->>}}
\newcommand{\xymapsto}{\ar@{|->}}

%>>>>>>>>>>>>>>>>>>>>>>>>>>>>>>
%<<<     Greek Letters      <<<
%>>>>>>>>>>>>>>>>>>>>>>>>>>>>>>
%\newcommand{\oldphi}{\phi}
%\renewcommand{\phi}{\varphi}
\let\oldphi\phi
\let\phi\varphi
\renewcommand{\to}{\longrightarrow}
\newcommand{\from}{\longleftarrow}
\newcommand{\eps}{\varepsilon}

%>>>>>>>>>>>>>>>>>>>>>>>>>>>>>>
%<<<  1st-4th & parentheses <<<
%>>>>>>>>>>>>>>>>>>>>>>>>>>>>>>
\newcommand{\first}{^\text{st}}
\newcommand{\second}{^\text{nd}}
\newcommand{\third}{^\text{rd}}
\newcommand{\fourth}{^\text{th}}
\newcommand{\ZEROTH}{$0^\text{th}$ }
\newcommand{\FIRST}{$1^\text{st}$ }
\newcommand{\SECOND}{$2^\text{nd}$ }
\newcommand{\THIRD}{$3^\text{rd}$ }
\newcommand{\FOURTH}{$4^\text{th}$ }
\newcommand{\iTH}{$i^\text{th}$ }
\newcommand{\jTH}{$j^\text{th}$ }
\newcommand{\nTH}{$n^\text{th}$ }

%>>>>>>>>>>>>>>>>>>>>>>>>>>>>>>
%<<<    upright commands    <<<
%>>>>>>>>>>>>>>>>>>>>>>>>>>>>>>
\newcommand{\upcol}{\textup{:}}
\newcommand{\upsemi}{\textup{;}}
\providecommand{\lparen}{\textup{(}}
\providecommand{\rparen}{\textup{)}}
\renewcommand{\lparen}{\textup{(}}
\renewcommand{\rparen}{\textup{)}}
\newcommand{\Iff}{\emph{iff} }

%>>>>>>>>>>>>>>>>>>>>>>>>>>>>>>
%<<<     Environments       <<<
%>>>>>>>>>>>>>>>>>>>>>>>>>>>>>>
\newcommand{\squishlist}
{ %\setlength{\topsep}{100pt} doesn't seem to do anything.
  \setlength{\itemsep}{.5pt}
  \setlength{\parskip}{0pt}
  \setlength{\parsep}{0pt}}
\newenvironment{itemise}{
\begin{list}{\textup{$\rightsquigarrow$}}
   {  \setlength{\topsep}{1mm}
      \setlength{\itemsep}{1pt}
      \setlength{\parskip}{0pt}
      \setlength{\parsep}{0pt}
   }
}{\end{list}\vspace{-.1cm}}
\newcommand{\INDENT}{\textbf{}\phantom{space}}
\renewcommand{\INDENT}{\rule{.7cm}{0cm}}

\newcommand{\itm}[1][$\rightsquigarrow$]{\item[{\makebox[.5cm][c]{\textup{#1}}}]}


%\newcommand{\rednote}[1]{{\color{red}#1}\makebox[0cm][l]{\scalebox{.1}{rednote}}}
%\newcommand{\bluenote}[1]{{\color{blue}#1}\makebox[0cm][l]{\scalebox{.1}{rednote}}}

\newcommand{\rednote}[1]
{{\color{red}#1}\makebox[0cm][l]{\scalebox{.1}{\rotatebox{90}{?????}}}}
\newcommand{\bluenote}[1]
{{\color{blue}#1}\makebox[0cm][l]{\scalebox{.1}{\rotatebox{90}{?????}}}}


\newcommand{\funcdef}[4]{\begin{align*}
#1&\to #2\\
#3&\mapsto#4
\end{align*}}

%\newcommand{\comment}[1]{}

%>>>>>>>>>>>>>>>>>>>>>>>>>>>>>>
%<<<       Categories       <<<
%>>>>>>>>>>>>>>>>>>>>>>>>>>>>>>
\newcommand{\Ens}{{\mathscr{E}ns}}
\DeclareMathOperator{\Sheaves}{{\mathsf{Shf}}}
\DeclareMathOperator{\Presheaves}{{\mathsf{PreShf}}}
\DeclareMathOperator{\Psh}{{\mathsf{Psh}}}
\DeclareMathOperator{\Shf}{{\mathsf{Shf}}}
\DeclareMathOperator{\Varieties}{{\mathsf{Var}}}
\DeclareMathOperator{\Schemes}{{\mathsf{Sch}}}
\DeclareMathOperator{\Rings}{{\mathsf{Rings}}}
\DeclareMathOperator{\AbGp}{{\mathsf{AbGp}}}
\DeclareMathOperator{\Modules}{{\mathsf{\!-Mod}}}
\DeclareMathOperator{\fgModules}{{\mathsf{\!-Mod}^{\textup{fg}}}}
\DeclareMathOperator{\QuasiCoherent}{{\mathsf{QCoh}}}
\DeclareMathOperator{\Coherent}{{\mathsf{Coh}}}
\DeclareMathOperator{\GSW}{{\mathcal{SW}^G}}
\DeclareMathOperator{\Burnside}{{\mathsf{Burn}}}
\DeclareMathOperator{\GSet}{{G\mathsf{Set}}}
\DeclareMathOperator{\FinGSet}{{G\mathsf{Set}^\textup{fin}}}
\DeclareMathOperator{\HSet}{{H\mathsf{Set}}}
\DeclareMathOperator{\Cat}{{\mathsf{Cat}}}
\DeclareMathOperator{\Fun}{{\mathsf{Fun}}}
\DeclareMathOperator{\Orb}{{\mathsf{Orb}}}
\DeclareMathOperator{\Set}{{\mathsf{Set}}}
\DeclareMathOperator{\sSet}{{\mathsf{sSet}}}
\DeclareMathOperator{\Top}{{\mathsf{Top}}}
\DeclareMathOperator{\GSpectra}{{G-\mathsf{Spectra}}}
\DeclareMathOperator{\Lan}{Lan}
\DeclareMathOperator{\Ran}{Ran}

%>>>>>>>>>>>>>>>>>>>>>>>>>>>>>>
%<<<     Script Letters     <<<
%>>>>>>>>>>>>>>>>>>>>>>>>>>>>>>
\newcommand{\scrQ}{\mathscr{Q}}
\newcommand{\scrW}{\mathscr{W}}
\newcommand{\scrE}{\mathscr{E}}
\newcommand{\scrR}{\mathscr{R}}
\newcommand{\scrT}{\mathscr{T}}
\newcommand{\scrY}{\mathscr{Y}}
\newcommand{\scrU}{\mathscr{U}}
\newcommand{\scrI}{\mathscr{I}}
\newcommand{\scrO}{\mathscr{O}}
\newcommand{\scrP}{\mathscr{P}}
\newcommand{\scrA}{\mathscr{A}}
\newcommand{\scrS}{\mathscr{S}}
\newcommand{\scrD}{\mathscr{D}}
\newcommand{\scrF}{\mathscr{F}}
\newcommand{\scrG}{\mathscr{G}}
\newcommand{\scrH}{\mathscr{H}}
\newcommand{\scrJ}{\mathscr{J}}
\newcommand{\scrK}{\mathscr{K}}
\newcommand{\scrL}{\mathscr{L}}
\newcommand{\scrZ}{\mathscr{Z}}
\newcommand{\scrX}{\mathscr{X}}
\newcommand{\scrC}{\mathscr{C}}
\newcommand{\scrV}{\mathscr{V}}
\newcommand{\scrB}{\mathscr{B}}
\newcommand{\scrN}{\mathscr{N}}
\newcommand{\scrM}{\mathscr{M}}

%>>>>>>>>>>>>>>>>>>>>>>>>>>>>>>
%<<<     Fractur Letters    <<<
%>>>>>>>>>>>>>>>>>>>>>>>>>>>>>>
\newcommand{\frakQ}{\mathfrak{Q}}
\newcommand{\frakW}{\mathfrak{W}}
\newcommand{\frakE}{\mathfrak{E}}
\newcommand{\frakR}{\mathfrak{R}}
\newcommand{\frakT}{\mathfrak{T}}
\newcommand{\frakY}{\mathfrak{Y}}
\newcommand{\frakU}{\mathfrak{U}}
\newcommand{\frakI}{\mathfrak{I}}
\newcommand{\frakO}{\mathfrak{O}}
\newcommand{\frakP}{\mathfrak{P}}
\newcommand{\frakA}{\mathfrak{A}}
\newcommand{\frakS}{\mathfrak{S}}
\newcommand{\frakD}{\mathfrak{D}}
\newcommand{\frakF}{\mathfrak{F}}
\newcommand{\frakG}{\mathfrak{G}}
\newcommand{\frakH}{\mathfrak{H}}
\newcommand{\frakJ}{\mathfrak{J}}
\newcommand{\frakK}{\mathfrak{K}}
\newcommand{\frakL}{\mathfrak{L}}
\newcommand{\frakZ}{\mathfrak{Z}}
\newcommand{\frakX}{\mathfrak{X}}
\newcommand{\frakC}{\mathfrak{C}}
\newcommand{\frakV}{\mathfrak{V}}
\newcommand{\frakB}{\mathfrak{B}}
\newcommand{\frakN}{\mathfrak{N}}
\newcommand{\frakM}{\mathfrak{M}}

\newcommand{\frakq}{\mathfrak{q}}
\newcommand{\frakw}{\mathfrak{w}}
\newcommand{\frake}{\mathfrak{e}}
\newcommand{\frakr}{\mathfrak{r}}
\newcommand{\frakt}{\mathfrak{t}}
\newcommand{\fraky}{\mathfrak{y}}
\newcommand{\fraku}{\mathfrak{u}}
\newcommand{\fraki}{\mathfrak{i}}
\newcommand{\frako}{\mathfrak{o}}
\newcommand{\frakp}{\mathfrak{p}}
\newcommand{\fraka}{\mathfrak{a}}
\newcommand{\fraks}{\mathfrak{s}}
\newcommand{\frakd}{\mathfrak{d}}
\newcommand{\frakf}{\mathfrak{f}}
\newcommand{\frakg}{\mathfrak{g}}
\newcommand{\frakh}{\mathfrak{h}}
\newcommand{\frakj}{\mathfrak{j}}
\newcommand{\frakk}{\mathfrak{k}}
\newcommand{\frakl}{\mathfrak{l}}
\newcommand{\frakz}{\mathfrak{z}}
\newcommand{\frakx}{\mathfrak{x}}
\newcommand{\frakc}{\mathfrak{c}}
\newcommand{\frakv}{\mathfrak{v}}
\newcommand{\frakb}{\mathfrak{b}}
\newcommand{\frakn}{\mathfrak{n}}
\newcommand{\frakm}{\mathfrak{m}}

%>>>>>>>>>>>>>>>>>>>>>>>>>>>>>>
%<<<  Caligraphic Letters   <<<
%>>>>>>>>>>>>>>>>>>>>>>>>>>>>>>
\newcommand{\calQ}{\mathcal{Q}}
\newcommand{\calW}{\mathcal{W}}
\newcommand{\calE}{\mathcal{E}}
\newcommand{\calR}{\mathcal{R}}
\newcommand{\calT}{\mathcal{T}}
\newcommand{\calY}{\mathcal{Y}}
\newcommand{\calU}{\mathcal{U}}
\newcommand{\calI}{\mathcal{I}}
\newcommand{\calO}{\mathcal{O}}
\newcommand{\calP}{\mathcal{P}}
\newcommand{\calA}{\mathcal{A}}
\newcommand{\calS}{\mathcal{S}}
\newcommand{\calD}{\mathcal{D}}
\newcommand{\calF}{\mathcal{F}}
\newcommand{\calG}{\mathcal{G}}
\newcommand{\calH}{\mathcal{H}}
\newcommand{\calJ}{\mathcal{J}}
\newcommand{\calK}{\mathcal{K}}
\newcommand{\calL}{\mathcal{L}}
\newcommand{\calZ}{\mathcal{Z}}
\newcommand{\calX}{\mathcal{X}}
\newcommand{\calC}{\mathcal{C}}
\newcommand{\calV}{\mathcal{V}}
\newcommand{\calB}{\mathcal{B}}
\newcommand{\calN}{\mathcal{N}}
\newcommand{\calM}{\mathcal{M}}

%>>>>>>>>>>>>>>>>>>>>>>>>>>>>>>
%<<<<<<<<<DEPRECIATED<<<<<<<<<<
%>>>>>>>>>>>>>>>>>>>>>>>>>>>>>>

%%% From Kac's template
% 1-inch margins, from fullpage.sty by H.Partl, Version 2, Dec. 15, 1988.
%\topmargin 0pt
%\advance \topmargin by -\headheight
%\advance \topmargin by -\headsep
%\textheight 9.1in
%\oddsidemargin 0pt
%\evensidemargin \oddsidemargin
%\marginparwidth 0.5in
%\textwidth 6.5in
%
%\parindent 0in
%\parskip 1.5ex
%%\renewcommand{\baselinestretch}{1.25}

%%% From the net
%\newcommand{\pullbackcorner}[1][dr]{\save*!/#1+1.2pc/#1:(1,-1)@^{|-}\restore}
%\newcommand{\pushoutcorner}[1][dr]{\save*!/#1-1.2pc/#1:(-1,1)@^{|-}\restore}










\includeversion{ASSp7-8}
\includeversion{Algebroid Definition}
\includeversion{Gamma-comodules}
\includeversion{Maps of algebroids}
\includeversion{Formal Group Laws}


\newcommand\OB{\textup{OB}}
\newcommand\ARR{\textup{ARR}}
\newcommand\COMP{\textup{COMP}}
\newcommand\Fsum{\sideset{}{^F}\sum}
\newcommand{\FGL}{\textup{FGL}}
\newcommand{\FGLp}{\textup{FGL}^{p}}
\newcommand{\SI}{\textup{SI}}
\newcommand{\SIp}{\textup{SI}^p}
\newcommand{\Ring}{\textsf{Rings}}
\newcommand{\ZpAlg}{\textsf{Alg}_{\Z_{(p)}}}

\begin{document}
\begin{ASSp7-8}

\section*{The Adams Spectral Sequence (pages 7-8)}
Choose a space $X=X_0$. Work only in the stable range \rednote{(?)} and on the $p$-component. We always use $\F_p$ coefficients, and write $A$ for the mod $p$ Steenrod algebra.

Define spaces $X_i$ and $K_i$ inductively by writing $K_i=\prod_{j>0}K(H^j(X_i),j)$, and letting $X_{i+1}$ be the fibre of the obvious map $X_i\to K_i$. We have a system of fibrations, with wavy arrows denoting maps $\phi:\Omega K_i\to X_{i+1}$:
\[\xymatrix{
\cdots\ar[r]&
X_2\ar[r]\ar[d]&
X_1\ar[r]\ar[d]&
X_0\ar@{=}[r]\ar[d]&
X_0\ar@{=}[r]\ar[d]&
\cdots\\
&
K_2\ar@{~>}[ul]&
K_1\ar@{~>}[ul]&
K_0\ar@{~>}[ul]&
\ast\ar@{~>}[ul]&
}\]
This yields an exact couple, after applying $\pi_*$, whose wavy arrows lower degree by one:
\[\xymatrix{
\cdots\ar[r]&
\pi_*X_2\ar[r]\ar[d]&
\pi_*X_1\ar[r]\ar[d]&
\pi_*X_0\ar@{=}[r]\ar[d]&
\pi_*X_0\ar@{=}[r]\ar[d]&
\cdots\\
&
\pi_*K_2\ar@{~>}[ul]&
\pi_*K_1\ar@{~>}[ul]&
\pi_*K_0\ar@{~>}[ul]&
0\ar@{~>}[ul]&
}
\]
From this we obtain a spectral sequence:
\[E_1^{st}=\pi_{s+t}(K_s)\implies \pi_{s+t}(X_0)\textup{\qquad (when there are \textit{no ancients}).}\]
Now consider the map $\psi:K_i\to\Sigma X_{i+1}$ adjoint to $\phi:\Omega K_i\to X_{i+1}$. Then we have a commuting diagram:
\[\xymatrix{
X_{i+1}\ar[d]\\
K_{i+1}\ar[d]^\eta&\Omega K_i\ar[l]_\phi\ar[ld]^{\Omega\psi}\ar@{~>}[lu]\\
\Omega\Sigma K_{i+1}
}
\qquad 
\xymatrix{
\pi_{*-1}(X_{i+1})\ar[d]\\
\pi_{*-1}(K_{i+1})\ar[d]^\Sigma&\pi_{*} (K_i)\ar[l]\ar[ld]^{\psi_*}\ar@{~>}[lu]\\
\pi_{*}(\Sigma K_{i+1})
}
\]
In the stable range, suspension is an isomorphism, so we only need to consider the map $\psi_*$ in order to calculate the $E_2$ page. Here's where it gets interesting. We consider instead the map $\psi^*$ on cohomology. It is the long diagonal in the following commuting diagram:
\[\xymatrix{
H^{*-1}(X_{i+1})\ar@{=}[d]\ar[r]^{\phi^*}&H^{*-1}(\Omega K_i)\ar@{=}[r]&H^*(\Sigma\Omega K_i)& H^*(K_i)\ar[l]_{\quad  \eta^*}\\
H^{*}(\Sigma X_{i+1})\ar[rru]_{(\Sigma\phi)^*}&\\
H^*(\Sigma K_{i+1})\ar[uurrr]^{\psi^*}\xyfib[u]
}\]
Now the top row of this diagram represents the transgression map in the cohomology Serre spectral sequence for the fibration $X_{i+1}\to X_i\to K_i$. By design, every third map in the Serre short exact sequence is surjective, so we obtain short exact sequences:
\[\xymatrix{
0&
H^*(X_i)\ar[l]&
H^*(K_i)\ar[l]&
H^{*-1}(X_{i+1})\ar[l]_\tau&
0\ar[l]
}\]
Splicing these short exact sequences together, we obtain an exact sequence:
\[\xymatrix@!0@R=12mm@C=1.2cm{
0&&H^*(X_0)\ar[ll]&&
H^*(K_0)\ar[ll]&&
H^*(\Sigma K_1)\ar[ll]_{\psi^*}\xyfib[ld]&&
H^*(\Sigma^2 K_2)\ar[ll]_{\psi^*}\xyfib[ld]&&
\cdots\ar[ll]\\
&&&&&H^*(\Sigma X_1)\xycofibdown[lu]_\tau&&
H^*(\Sigma^2 X_2)\xycofibdown[lu]_\tau&&
}\]
We have that the differential is $\psi^*$ because of the previous commuting diagram.

Now the cohomology groups $H^*(K_*)$ are all free $A$-modules \rednote{(in a range, or something)}. Thus, the above exact sequence gives a free resolution of $H^*(X_0)$. The idea is that after applying $\Hom_A(\DASH,\F_p)$ to this exact sequence {(with $H^*(X)$ removed)} we should get a cochain complex isomorphic to the $E_1$ page of the spectral sequence, so that the $E_2$-page is $\Ext_A^{**}(H^*(X),\F_p)$, as desired.

There is a map \[\pi_*(K(\F_p,i))\to\Hom_{\F_p}(H^*(K(\F_p,i)),\F_p)\textup{\qquad sending\qquad }\sigma\mapsto\langle \DASH,h(\sigma)\rangle.\]
Now we note that this map factors as
\[\xymatrix{
\ar[r]\ar[dr]^\theta_\exists\pi_*(K(\F_p,i))&
\Hom_{\F_p}(H^*(K(\F_p,i)),\F_p)\\
&\Hom_{A}(H^*(K(\F_p,i)),\F_p)\xycofib[u]
}\]
Moreover, the map $\theta$ is an isomorphism \rednote{(some stable stuff happens here?)}. So there is such an isomorphism $\theta$ whenever the EM space is replaced by a product of EM spaces. This gives an isomorphism of cochain complexes between $(\pi_*K_*,\psi_*)$ and $(\Hom_A(H^*(K_*),\F_p),(\psi^*)^*)$, which shows that the $E_2$-page of the spectral sequence is $\Ext_A^{**}(H^*(X),\F_p)$.
\end{ASSp7-8}
\begin{Algebroid Definition}
\section*{Hopf Algebroids}
From the text: A Hopf algebroid over a commutative ring $K$ is a cogroupoid object in the category of (graded or bigraded) commutative $K$-algebras,
i.e., a pair $(A, \Gamma)$ of commutative $K$-algebras with structure maps such that for
any other commutative $K$-algebra $B$, the sets $\Hom(A, B)$ and $\Hom(\Gamma, B)$ are the
objects and morphisms of a groupoid.

We use the following abbreviations for certain co-representable functors:
\begin{itemize}\squishlist
\item $\OB:=\Hom(A,\DASH)$
\item $\ARR:=\Hom(\Gamma,\DASH)$
\item $\COMP:=\Hom(\Gamma\otimes_A\Gamma,\DASH)=\ARR\times_{\OB}\ARR$
\end{itemize}
We tabulate the structure maps, and what they correspond to under the co-Yoneda embedding.
\[\xymatrix@R=.1cm{
A\ar[r]^{\eta_L}&\Gamma&& \ARR\ar[r]^{\textup{src}}&\OB&& \textit{left unit or source} \\
A\ar[r]^{\eta_R}&\Gamma&& \ARR\ar[r]^{\textup{targ}}&\OB&& \textit{right unit or target} \\
\Gamma\ar[r]^-{\Delta}&\Gamma\otimes_A\Gamma
&&\COMP\ar[r]^\circ&\ARR&&\textit{coproduct or composition} \\
\Gamma\ar[r]^{\epsilon}&A&&
\OB\ar[r]^{\textup{iden}}&\ARR&&\textit{counit or identity} \\
\Gamma\ar[r]^{c}&\Gamma&&
\ARR\ar[r]^{\textup{inv}}&\ARR&&\textit{conjugation or inverse} 
&
}\]
We demand that:
\begin{itemize}\squishlist
\item $\Gamma$ is a left $A$-module via $\eta_L$;
\item $\Gamma$ is a right $A$-module via $\eta_R$;
\item $\Delta$ and $\epsilon$ are $A$-bimodule maps;
\item The source (likewise, target) of an
identity morphism are the object on which it is defined:
\[\xymatrix{
A\ar@{=}[d]\ar[r]^{\eta_{L}}& \Gamma\ar[ld]^\epsilon&&
%
\OB\ar@{=}[d];[] \ar[r];[]_{\textup{src}} &\ARR\ar[ld];[]_{\textup{iden}}
\\
A&&&
\OB
}\]
\item Pre-composition (likewise, post-composition) with the identity leaves a morphism
unchanged:
\[\xymatrix{
\Gamma\ar@{=}[dr]\ar[r]^-{\Delta}& \Gamma\otimes_A\Gamma
\ar[d]^{\epsilon\otimes 1}&&
%
\ARR\ar@{=}[dr]\ar[r];[]_-{\circ}& \COMP\ar[d];[]_{((\textup{iden}\circ\textup{src}),\id_{\ARR})}&&\\
&\Gamma&&
&\ARR
}\]
\item Composition of morphisms is associative:
\[\xymatrix{
\Gamma\ar[r]^-{\Delta}\ar[d]^{\Delta}&
\Gamma\otimes\Gamma\ar[d]^{1\otimes\Delta}&&
\ARR\ar[r];[]_-{\circ}\ar[d];[]_{\circ}&
\ARR\times_{\OB}\ARR\ar[d];[]_{\id\times\circ}
\\
\Gamma\otimes\Gamma\ar[r]^-{\Delta\otimes1}&
\Gamma\otimes\Gamma\otimes\Gamma&&
\ARR\times_{\OB}\ARR\ar[r];[]_-{\circ\times\id}&
\ARR\times_{\OB}\ARR\times_{\OB}\ARR
}\]

\item The inverse of the inverse is the original morphism:
\[\textup{The composites }
\xymatrix{\Gamma\ar[r]^c& \Gamma\ar[r]^c&\Gamma}\textup{ and }\xymatrix{\ARR\ar[r];[]_{\textup{inv}}& \ARR\ar[r];[]_{\textup{inv}}&\ARR}\textup{ are the identity.}\]
\item Inverting a morphism interchanges source and
target:
\[\xymatrix{
A\ar[r]^{\eta_R}\ar[rd]_{\eta_L}&\Gamma\ar[d]^c&&
\OB\ar[r];[]_{\textup{targ}}\ar[rd];[]^{\textup{src}}& \ARR\ar[d];[]_{\textup{inv}}\\
&\Gamma&&
&\ARR
}\]


\item Composition of a
morphism with its inverse on either side gives an identity morphism:
\[\xymatrix@C=2.5cm{
\Gamma&\Gamma\otimes_K\Gamma\ar@{->>}[d]
\ar[l]_-%
{c(\gamma_1)\gamma_2\,\reflectbox{\tiny$\mapsto$}\, \gamma_1\otimes\gamma_2}&
\ARR&\ARR\times\ARR\xycofib[d];[]
\ar[l];[]^-%
{\phi\,\mbox{\tiny$\mapsto$}\,(\phi^{-1}\!,\,\phi)}
\\
%\ar[l];[]^-{(\textup{inv,id})}\\
%
&\Gamma\otimes_A\Gamma\ar@{-->}[ul]^{\exists!}&
&\ARR\times_{\OB}\ARR\ar@{-->}[ul];[]_{\exists!}\\
%
A\ar[uu]^{\eta_R}&\Gamma\ar[u]^\Delta \ar[l]_\epsilon&
\OB\ar[uu];[]_{\textup{targ}}& \ARR\ar[u];[]_{\circ}\ar[l];[]^{\textup{iden}}
}\]
\end{itemize}
\end{Algebroid Definition}
\begin{Gamma-comodules}
\section*{$\Gamma$-comodules}

\begin{itemize}\squishlist
\item A \textbf{left $\Gamma$-comodule} $M$ is:
\begin{itemize}\squishlist
\item a left $A$-module $M$;
\item a left $A$-module map $\psi:M\to\Gamma\otimes_A M$ which is counitary and coassociative:
\[\xymatrix{
M\ar[r]^-{\psi}&\Gamma\otimes M\ar[d]^{\epsilon\otimes1}&&
M\ar[d]^{\psi}\ar[r]^-{\psi}&\Gamma\otimes M\ar[d]^{\Delta\otimes1}\\
&A\otimes M\ar@{=}[ul]&&
\Gamma\otimes M\ar[r]^-{1\otimes\psi}& \Gamma\otimes\Gamma\otimes M
}\]
\begin{shaded}
I think that if $M=\Gamma$, $\eta_L:A\to \Gamma$ is one of these --- you have the right hand diagram corresponding to $\Delta$ being a module map, and the left hand diagram is replicated above.
\begin{itemize}\squishlist
\item $\Gamma$ is a left $\Gamma$-comodule via $\Delta:\Gamma\to\Gamma\otimes_A\Gamma$.
\item I think that: $A$ is a right $\Gamma$-comodule via $\eta_L:A\to\Gamma$.
\end{itemize}

\end{shaded}

\end{itemize}
\item A \textbf{comodule algebra} $M$ is:
\begin{itemize}\squishlist
\item a commutative associative $A$-algebra;
\item a left $\Gamma$-comodule via an \emph{algebra map} $\psi:M\to\Gamma\otimes_A M$.
\end{itemize}
Note that we have an algebra map $A\to M$, giving a map $h^M\to\OB$. Thus, applying the co-Yoneda lemma, we get what looks like an action of $\ARR$ on $h^M:=[M,\DASH]$:
\[\xymatrix{
h^M\ar[r];[]_-{\psi^*}&\ARR\times_\OB h^M\ar[d];[]_{\textup{iden}\times1}&&
h^M\ar[d];[]_{\psi^*}\ar[r];[]_-{\psi^*}& \ARR\times_\OB h^M\ar[d];[]_{\circ\times1}\\
&\OB\times_\OB h^M\ar@{=}[ul];[]&&
\ARR\times_\OB h^M\ar[r];[]_-{1\times\psi^*}& \COMP\times_\OB h^M
}\]
\item The \textbf{tensor product} of left $\Gamma$-comodules has an obvious structure of left $\Gamma$-comodule.
\item If $\Gamma$ is flat as a (right) $A$-module, then the category of left $\Gamma$-comodules is abelian. Thus we'll always assume that $\Gamma$ is flat over $A$.
\item We can define a \textbf{cotensor product} $M\Box_\Gamma N$ whenever $M$ (resp.\ $N$) is a right (resp.\ left) $\Gamma$-comodule. This fails to be a comodule, or even an $A$-module. It's defined by:
\[\xymatrix{
0\ar[r]&
M\square_\Gamma N\ar[r]&
M\otimes_AN\ar[r]&
M\otimes_A\Gamma\otimes_AN
}\]

\item We can turn left comodules into right comodules using the inverse map $c:\Gamma\to\Gamma$, so it makes sense to claim that $M\square_\Gamma N=N\Box_\Gamma M$.

\begin{shaded}
\item For any $A$-modules $M$ and $N$, we have a natural map $\gamma_M:\Hom_A(M,A)\otimes_AN\to\Hom_A(M,N)$. This is an isomorphism when $M$ is projective.

\INDENT To see this, write \smash{$F=M\oplus M'$} with $F$ free. Then $\gamma_F=\gamma_M\oplus\gamma_{M'}$, and $\gamma_F$ is obviously an isomorphism, so that $\gamma_M$ must be an isomorphism.
\end{shaded}
\item In light of the previous, if $M$ is a left $\Gamma$-comodule that is projective over $A$, then $\Hom_A(M,A)$ is a left comodule, via (I suppose):
\[\Hom_A(M,A)\overset{(\eta_L)_*}{\to}\Hom_A(M,\Gamma)\overset{\cong}{\from} \Hom_A(M,A)\otimes\Gamma\]
In particular, we can form the cotensor product in the following isomorphism, for left $\Gamma$-modules $M$ and $N$ with $M$ projective:
\[\Hom_\Gamma(M,N)=\Hom_{A}(M,A)\square_\Gamma N.\]
\end{itemize}
\end{Gamma-comodules}
\begin{Maps of algebroids}
\subsection*{Maps of algebroids}
\begin{itemize}\squishlist
\item A map of Hopf algebroids is the obvious.
\item Suppose that $f=(f_1,f_2):(A,\Gamma)\to(B,\Sigma)$ is a map of Hopf algebroids. Then
\begin{itemize}\squishlist
\item $B$ is a left $A$-module via $f_1$.
\item $\Sigma$ is a left $A$-module by restriction along $f_1$.
\item
The following diagram displays $\Gamma\otimes_A B$ as a right $\Sigma$-comodule.
\[\xymatrix{
A\ar[r]^-{\eta_R}\ar[d]^-{f_1}&%r1c1
\Gamma\ar[d]\ar[r]^-{(1\otimes f_2)\circ\Delta}&%r1c2
\Gamma\otimes_A\Sigma\ar@{=}[d]\\%r2c1
B\ar@/_1em/[rr]\ar[r]&%r3c1
\Gamma\otimes_AB\ar@{-->}[r]&%r3c2
\Gamma\otimes_A B\otimes_B\Sigma
}\]
\item For any left $\Sigma$-comodule $N$:
\begin{itemize}\squishlist
\item  $\Gamma\otimes_AN$ is a left $\Gamma$-comodule, via: %by co-restricting along $f_2$:
\smash{$\Gamma\otimes_AN\overset{\Delta\otimes N}{\to}\Gamma\otimes_A\Gamma\otimes_AN$}.
\item There is an inclusion $(\Gamma\otimes_AB)\square_\Sigma N\to\Gamma\otimes_AN$ of left $\Gamma$-comodules.
\end{itemize}

\end{itemize}
\item The Hopf algebra associated with an algebroid $(A,\Gamma)$ is $\Gamma'=\Gamma/(\eta_L(a)-\eta_R(a))$.
\item A map $f=(f_1,f_2):(A,\Gamma)\to(A,\Sigma)$ is \textbf{normal} if:
\begin{itemize}\squishlist
\item $f_1:A\to A$ is the identity;
\item $f_2:\Gamma\to\Sigma$ is surjective;
\item $\Gamma\square_{\Sigma'}A=A\square_{\Sigma'}\Gamma$ in $\Gamma$.
\begin{shaded}
I should understand the last requirement.
\end{shaded}
\end{itemize}

\end{itemize}

\end{Maps of algebroids}
\begin{Formal Group Laws}
\section*{Formal group laws}
Let $\Ring$ be the category of commutative unital rings. Let $\FGL:\Ring\to\Set$ be the functor sending a ring $R$ to the set of FGLs over $R$. Let $\SI:\Ring\to\Set$ be the functor sending a ring to the set of triples $(F,f,G)$ where $R,G\in\FGL(R)$ and $f:F\to G$ is a strict isomorphism.

We have natural transformations:
\begin{itemise}
\item $\textup{src},\textup{targ}:\SI\to\FGL$;
\item composition $\circ:\SI\times_\FGL\SI\to \SI$;
\item $\textup{iden}:\FGL\to\SI$; and
\item $\textup{inv}:\SI\to\SI$.
\end{itemise}
Thus, these two functors give a functor $\Ring\to\mathsf{Groupoids}$. If we could only see that $\FGL$ and $\SI$ were co-representable, say by $L$ and $LB$, we would have shown that $(L,LB)$ is a Hopf algebroid.

\subsection*{The setup}
\begin{itemise}
\renewcommand{\labelitemii}{$\rightarrow$}
\item A \textbf{formal group law} over a ring $R$ is a power series $F(x,y)\in R[[x,y]]$ which is `unital', `commutative' and `associative'.
\begin{itemize}\squishlist
\item Once $F$ has these properties, there's a power series $i(x)$, the \textbf{formal inverse}, such that $F(x,i(x))=0$.
\item $F(x,y)\equiv x+y\mod{(x,y)^2}$.
\item One can obtain a FGL from a one-dimensional abelian Lie group, by parameterising a neighbourhood of the identity.
\end{itemize}
There are a number of obvious categories whose objects are the formal group laws over $R$. The arrows from $F$ to $G$ could be:
\begin{itemize}
\item \textbf{homomorphisms}: power series $f(x)\in r[[x]]$ such that:
\begin{itemize}\squishlist
\item$f(F(x,y))=G(f(x),f(y))$; and
\item $f(0)=0$.
\end{itemize}
\item \textbf{isomorphisms}: homomorphisms which are invertible (under composition);
\begin{itemize}\squishlist
\item This is equivalent to the condition that $f'(0)$ be a unit.
\end{itemize}
\item \textbf{strict isomorphisms}: homomorphisms such that $f'(0)=1$.
\end{itemize}
\item A \textbf{logarithm} for a FGL $F$ is a \textbf{strict} isomorphism $\log_F$ from $F$ to the additive formal group law, so that
\[\log_F(F(x,y))=\log_F(x)+\log_F(y).\]
\begin{itemize}\squishlist
\item For example, suppose that $F(x,y)=x+y+uxy$. This FGL comes from the parameterisation $t\mapsto 1+ut$ at the origin of the multiplicative group $R^\times$. In this case, at least when $R$ is a $\Q$-algebra, we have a homomorphism 
\[R\to R^\times\qquad x\mapsto e^{ux}.\]
This has an inverse near the identity:
\[1+ut\mapsto\frac{1}{u}\log(1+ut)=t-\frac{ut^2}{2}+\frac{u^2t^3}{3}+\cdots,\]
and this power series is the logarithm.
\end{itemize}
Any formal group law over a $\Q$-algebra has a logarithm --- in fact is is easy to write down a formula. We can derive an ODE for the logarithm $f$ of $F$:
\begin{alignat*}{2}
f(F(x,y))&=
f(x)+f(y)%
&\qquad&\text{(by definition)}\\
F_{2}(x,y)f'(F(x,y))&=
f'(y)%
&\qquad&\text{(differentiating w.r.t.\ $y$)} \\
F_2(x,0)f'(x)&=
1%
&\qquad&\text{(setting $y=0$)} \\
f(x)&=
\int_0^x\frac{dt}{F_2(t,0)}%
 \end{alignat*}
which makes sense, as we can form the power series $1/F_2(t,0)$ using rational coefficients.

\end{itemise}
\subsection*{The Lazard Ring}
\begin{itemise}
\renewcommand{\labelitemii}{$\rightarrow$}
\item The functor $\FGL:\Ring\to\Set$ sending $R$ to the set of FGLs over $R$ is corepresentable. This is, for some $L\in\Ring$, and some $F\in\FGL(L)$:% That is, there is a ring $L$ (the Lazard ring) and a FGL $F(x,y)=\sum a_{ij}x^iy^j$ over $L$ such that every
%\[\textup{For some }L\in\Ring,\textup{ and some }F\in\FGL(L)\textup{:}\]
\[\textup{For all $G\in\FGL(R)$ there is a unique }\theta:L\to R\textup{ such that }G=\theta_*(F).\]
\begin{itemize}\squishlist
\item The construction of the Lazard ring $L$ is easy. Planning to let $F=\sum a_{ij}x^iy^j$, let $L=\Z[a_{ij}]/I$, where $I$ is the ideal generated by all the relations needed to make $F$ a formal group law:
\begin{itemize}\squishlist
\item $a_{10}=a_{01}=1$;%, and each has degree 0;
\item $a_{i0}=a_{0i}=0$ for $i>1$;
\item $a_{ij}=a_{ji}$;% (also homogeneous);
\item $b_{ijk}=0$, where the $b_{ijk}$ are defined by the formal expansion:
\[F(F(x,y),z)-F(x,F(y,z))=\sum b_{ijk}x^iy^jz^k.\]
\end{itemize}
\item Give $L$ a grading, by defining $|a_{ij}|=2(i+j)-2$. The topologist's thinking here is that we should have $F(x,y)$ homogeneous of degree $-2$ when $|x|=|y|=-2$.
\end{itemize}
\end{itemise}
\subsection*{Finding $L$ rationally}

\begin{itemise}
\item
As $L\otimes\Q$ is the product of $L$ and $Q$ in $\Rings$, we have:
%We have a pushout square defining $L\otimes \Q$:
%\[\vcenter{\xymatrix{
%\Z \ar[d] \ar[r] \ar@{}[rd]|{\textup{p.o.}}&
%\Q \ar[d]\\
%L \ar[r]&
%L\otimes \Q
%}}\]
%Thus
\[\Hom_{\Rings}(L\otimes\Q,R)=\begin{cases}
\emptyset,&\textup{if $R$ is not a $\Q$-algebra};\\
\Hom_{\Rings}(L,R),&\textup{if $R$ is a $\Q$-algebra}.
\end{cases}\]
In particular, $L\otimes\Q$ is the representing object for the functor on the right. Now if $R$ is a $\Q$-algebra, all FGLs $G$ on $R$ have a logarithm $g(x)=x+a_1x^2+a_2x^3+\cdots$, and $G$ is determined by $g(x)$.

\INDENT Any object representing the functor on the right should be isomorphic to $\Q[m_1,m_2,\ldots]$, so that $L\otimes\Q=\Q[m_1,m_2,\ldots]$. Of course, the universal FGL on $L\otimes\Q$ is then $F(x,y)=f^{-1}(f(x)+f(y))$, where $f$ is the universal logarithm $f(x)=x+m_1x^2+m_2x^3+\cdots$.

\INDENT Let $M=\Z[\{m_i\}]\subset L\otimes\Q$. Then the image of $L$ in $L\otimes\Q$ lies in $M$. The degree of $m_i$ is $2i$.
\end{itemise}
\subsection*{Lazard's theorem}
\begin{itemise}
\item We define the group $QR$ of \textbf{indecomposables} of a connected graded ring $R$ to be $I/I^2$, where $I$ is the ideal of elements of positive degree.
\item \textbf{Lazard's theorem} (\textbf{A2.1.10}):
\begin{enumerate}\squishlist
\item[(a)] $L=\Z[x_q,x_2,\ldots]$ with $|x_i|=2i$.
\item[(b)] The $x_i$ can be chosen so that the image of $x_i$ in $QL\otimes\Q$ is
\[\begin{cases}
pm_i,&\textup{if }i=p^k-1;\\
m_i,&\textup{otherwise}.
%\\,&\textup{if }
\end{cases}
\]
\item[(c)] $L$ is a subring of $M=\Z[\{m_i\}]$.
\end{enumerate}
\end{itemise}
\subsection*{Proof of Lazard's theorem}
\begin{itemise}

\item We will need the following tricky lemma (\textbf{A2.1.12}):
\begin{itemize}\squishlist
\item Let $F,G\in\FGL(R)$ be such that $F\equiv G\mod{(x,y)^n}$. Then $F\equiv G+\gamma\cdot C_n\mod{(x,y)^{n+1}}$ for some $\gamma\in R$, where we define $C_n$ as follows.

\INDENT For $n\in\N$, define:
\begin{alignat*}{2}
\textup{fac}(n)&:=
\begin{cases}
p,&\textup{if }n=p^k\textup{ for some prime $p$};\\
1,&\textup{otherwise}.
%\\,&\textup{if }
\end{cases}%
\\
v_{ij}&:=
\frac{{i+j\choose i}}{\textup{fac}(i+j)}%
 \\
C_n&:=\frac{\left((x+y)^n-x^n-y^n\right)}{\textup{fac}(n)}=\sum v_{ij}x^iy^j
\end{alignat*}
It is an exercise to see that the $v_{ij}$ are all integers, and generate the unit ideal of $\Z$, so that we can fix a choice of integers $b_{ij}$ such that
\[\sum b_{ij}v_{ij}=1.\]
\end{itemize}
\item Finally, we have (\textbf{A2.1.13}):
\begin{enumerate}\squishlist
\item[(a)] In ${QL}\otimes{\Q}$, $a_{ij}=-{i+j\choose i} m_{i+j-1}$.
\item[(b)] $QL$ is torsion free.
\end{enumerate}
\begin{proof}
For (a), we use the fact that we have a logarithm for ${L}\otimes{\Q}$. This gives:
\begin{alignat*}{2}
\sum_{n\geq1}m_{n-1}\left(\sum a_{ij}x^iy^j\right)^n&=\sum_{n\geq1}m_{n-1} \left(x^n+y^n\right)&\qquad&\text{(in ${L}\otimes{\Q}[[x,y]]$)}
\\\sum a_{ij}x^iy^j+\sum_{n\geq2}m_{n-1} \left(x+y+\cdots\right)^n&=\sum_{n\geq1}m_{n-1} \left(x^n+y^n\right)
\\\sum a_{ij}x^iy^j+\sum_{n\geq2}m_{n-1} \left(x+y\right)^n&=\sum_{n\geq1}m_{n-1} \left(x^n+y^n\right)&\qquad&\text{(in ${QL}\otimes{\Q}[[x,y]]$)}
\\\sum a_{ij}x^iy^j-(x+y)&=\sum_{n\geq2}m_{n-1} \left(x^n+y^n-\left(x+y\right)^n\right).&& \textup{This gives (a).}
\end{alignat*}
For (b):
\begin{itemize}\squishlist
\item let $\theta:L\to R:=\Z\oplus Q_{2n}L$ be the obvious ring map, where $Q_{2n}L:=(QL)_{2n}$. 
\item let $F\in\FGL(R)$ correspond to the map $\theta$. Note that $F$ is of the form 
\[F(x,y)=x+y+\sum_{i+j=n+1}\theta(a_{ij})x^iy^j.\]
 To see this, observe that $\theta$ annihilates homogeneous elements of all degrees but 0 and $2n$, and $|a_{ij}|=2(i+j)-2=2n$ implies that $i+j=n+1$.
\end{itemize}
In particular, $F$ is congruent mod $(x,y)^{n+1}$ to the additive group law on $R$, and thus
\[F\equiv x+y+\gamma\cdot C_{n+1}(x,y)\mod{(x,y)^{n+2}},\textup{\quad for some $\gamma\in Q_{2n}L$}\]
Moreover, this congruence can be replaced by an equation. This shows that:
\[a_{ij}\mapsto v_{ij}\cdot \gamma,\textup{\quad for $i+j=n+1$.}\]
We define
\[x_{n}:=\sum_{i+j=n+1} b_{ij}a_{ij}\in L_{2n}\textup{\quad so that\quad }\theta(x_{n})=\gamma.\]
Thus $a$ is in the image of $\theta$, and generates $Q_{2n}L$ as an abelian group.

\INDENT Finally, we note that $Q_{2n}L\otimes\Q=\Q\{m_{n-1}\}$, which is generated by $\gamma$, using part (a). Thus the cyclic group $Q_{2n}L$ must be infinite cyclic, proving (b).
\end{proof}
\item From this proof, we have seen that each $Q_{2n}L$ is infinite cyclic, generated by $\theta(x_{n})$. In particular, the $x_n$ generate $L$. Moreover,

\begin{alignat*}{2}
x_n&=
\sum_{i+j=n+1} b_{ij}a_{ij}%
\\
&=
-\left(\sum{i+j\choose j}b_{ij}\right)m_{n}%
&\qquad&\text{(in $QL\otimes L$, by (a))} \\
&=
\textup{fac}(n+1)\cdot m_n%
 \end{alignat*}
Thus the image of $x_n$ in ${L}\otimes{\Q}$ is the correct multiple of $m_i$, as desired.
\item Finally, we have $L=\Z[\{x_i\}]/I$ for some ideal $I$. Moreover, passing to ${L}\otimes{\Q}$, a relation between the $x_i$ will give a relation between the $m_i$, but they are algebraically independent. Thus $I=0$, completing the proof.
\end{itemise}
\begin{shaded}
\subsection*{Haynes' exposition on this topic (19/1/2012)}
We are interested in $(QL)^{2n-2}$, and this can be studied by considering formal group laws over rings $R=\Z\oplus A[2n-2]$, where $A$ is an abelian group. Here we'll actually be considering \emph{graded} FGLs over graded rings. I haven't thought about what difference this makes. Anyway:
\[\FGL^{\textup{gr}}(R)=\Rings^\textup{gr}(L,R)=\Hom_{\mathsf{Ab}}((QL)^{2n-2},A).\]
As $L$ is finitely generated in each degree, we'll then understand $(QL)^{2n-2}$ if we can calculate $\FGL^{\textup{gr}}(R)$ for $A$ any finitely generated abelian group.

Now any $F\in\FGL(R)$ will be of the form 
\[F(x,y)=x+y+\sum_{i+j=n}a_{ij}x^iy^j=:x+y+h(x,y).\]
This polynomial $h(x,y)$ must be symmetric in $x$ and $y$, and we must have $a_{n0}=a_{0n}=0$. Finally, we must have associativity:
\begin{alignat*}{2}
F(x,F(y,z))
&=
F(F(x,y),z)%
\\
x+(y+z+h(y,z))+h(x,y+z+h(x,y))
&=
(x+y+h(x,y))+z+h(x+y+h(x,y),z)%
\\
x+(y+z+h(y,z))+h(x,y+z)
&=
(x+y+h(x,y))+z+h(x+y,z)%
\\
h(y,z)+h(x,y+z)
&=
h(x,y)+h(x+y,z)%
\\
% Left hand side
h(y,z)-h(x+y,z)+h(x,y+z)-h(x,y)
% Relation
&=
% Right hand side
0
% Comment
\end{alignat*}
Now there is a cochain complex:
\[0\to A\to A[x_1]\to A[x_1,x_2]\to A[x_1,x_2,x_3]\to\cdots\]
with differential $\delta$ sending $f(x_1,\ldots,x_{n-1})$ to
\[f(x_2,\ldots,x_n)-f(x_1+x_2,x_3,\ldots)+\cdots+(-1)^{n} f(\ldots,x_{n-2},x_{n-1}+x_n)+(-1)^{n+1} f(x_1,\ldots,x_{n-1})\]
So, we are asking that $h$ is a symmetric 2-cocycle of degree $2n$ (perhaps it would be more normal to say `of degree $n$' here). That is, $h\in Z_{\textup{sym}}^{2,2n}$.

\INDENT Anyway, the ``best way in the world'' to get a cocycle is to take a coboundary and divide it by something! The obvious candidate is the coboundary $\delta(-x^n)=(x+y)^n-x^n-y^n$. In fact, 
\[\textup{fac}(n):=\gcd\left\{{n\choose i}\,:\,0<i<n\right\}=
\begin{cases}
p,&\textup{if }n=p^k\textup{ for some prime $p$};\\
1,&\textup{otherwise};
%\\,&\textup{if }
\end{cases}\]
so we write $C_n(x,y)=\delta(-x^n)/\textup{fac}(n)$. The point then is to see that we have an isomorphism
\[A\cong Z^{2,2n}_{\textup{sym}}(A)\qquad a\mapsto aC_n.\]
Using this calculation, we'll be able to show that $(QL)^{2n-2}\simeq \Z$.

Now we can associate a group $\Gamma(R)$ to each ring $R$, and this group acts naturally on $\FGL(R)$. Define:
\[\Gamma(R):=\left\{x+b_1x^2+b_2x^3+\cdots\right\}\]
acting on $\FGL(R)$ by conjugation.
 In fact, the group $\Gamma$ is co-represented by the ring $S=\Z[b_1,b_2,\ldots]$, so that $S$ is a hopf algebra.

From now on, \textbf{fix a prime} $p$. Then consider the following idempotent on the set of logarithms:
\[\left(\log_F(x)=\sum_{i\geq1}m_{i-1}x^{i}\right) \mapsto
\left(\widetilde{\log}_F(x):= \sum_{k\geq0}m_{p^k-1}x^{p^k}\right)\]
\begin{thm*}
Over a $\Z_{(p)}$-algebra $R$ there is a natural idempotent\ $\widetilde{\,\,}:\FGL(R)\to\FGL(R)$ and a natural strict isomorphism $F\to\widetilde{F}$ such that if $R$ is in fact a $\Q$-algebra,
\[\log_{\widetilde{F}}=\widetilde{\log}_F.\]
\end{thm*}
We'll write $\FGLp(R)$ for the image of this idempotent in $\FGL(R)$. Fortunately, we have:
\[\FGLp(R)=\Rings(L^{(p)},R)\textup{\qquad where }L^{(p)}=\Z_{(p)}[v_1,v_2,\ldots]\textup{ with }|v_i|=2(p^i-1).\]
One question then is whether there are canonical choices for the $v_i$. The answer is \emph{yes}. In fact, any $p$-typical FGL gives an expression
\[[p]_F(x)=\Fsum_{i=0}^{\infty}w_ix^{p^i},\]
(with $w_0=p$), and we can choose these $w_i$ as the correct generators.
\subsection*{On Hopf algebroids}
Suppose one is given a set with a group action. Then one can obtain a groupoid, the \emph{translation groupoid}, whose objects are the elements of the set, and with an arrow $x\to y$ for each $g$ such that $gx=y$. Such a groupoid is called ``split''.

The groupoid consisting of all FGLs on a ring $R$ is a split groupoid --- the group $\SI(R)$ acts on $\FGL(R)$. In particular, there is a coaction map
\[L\to S\otimes L\]
which gives the natural action of $\SI(R)$ on $\FGL(R)$.

In general, if $S$ is a Hopf algebra, and $L$ is an algebra, all over a ring $R$, a coaction of $S$ on $L$ gives a Hopf algebroid structure on $(L,S\otimes L)$.

Suppose now that we take the pair $(L,S\otimes L)$, tensor everything with $\Z_{(p)}$, and then take an appropriate quotient, to obtain $(L^{(p)},W^{(p)})$, representing $p$-typical FGLs and strict isomorphisms.

Taking the quotient corresponds to taking sub-groupoids. (Note that we can also view $\FGLp$ as a termwise quotient of $\FGL$, via the idempotent.) This Hopf algebroid is \emph{not} split, as the strict isomorphisms that can act on a $p$-typical FGL to give another $p$-typical one depend on the group law you started with. This is where Hopf algebroids are really needed.
\subsection*{Didn't discuss in depth}
Haynes seems to suggest that there was a natural way to define a coaction of $S$ and an action of $L$ on $MU_*(X)$ for any spectrum $X$, which seems to provide the co-algebra side of the story... More to be revealed later.


\end{shaded}

\subsection*{The Hopf algebroid}
Recall that $\SI:\Ring\to\Set$ is the functor sending a ring to the set of triples $(F,f,G)$ where $R,G\in\FGL(R)$ and $f:F\to G$ is a strict isomorphism.

Now an element of $(F,f,G)\in\SI(R)$ is determined by $F$ and $f$, and the choices of $F$ and $f$ are independent. Thus, $\SI$ is corepresented by $LB:=L\otimes\Z[b_1,b_2,\ldots]$, where $b_i$ corresponds to the choice of the coefficient of $x^{i+1}$ in the strict isomorphism $f$. For some reason, we set $|b_i|=2i$.

Because the pair $(\FGL(R),\SI(R))$ gives a groupoid, we have that $(L,LB)$ is a cogroupoid object in $\Rings$, i.e.\ a Hopf algebroid.
\begin{itemise}
\item The formulae for the Hopf algebroid structure on $(L,LB)$ are given in (\textbf{A2.1.16}). It is noted that these formulae are the same as those for $(MU_*,MU_*MU)$, and that this Hopf algebroid is split.
\end{itemise}
\subsection*{$p$-typical formal group laws}
From now on, we'll be thinking about formal group laws over $\Z_{(p)}$-algebras, so we'll write $\ZpAlg$ for the category of unital $\Z_{(p)}$-algebras.
\begin{itemise}
\item We wish to give a definition of $p$-typicality, which we imagine to mean `the logarithm only has terms $x^{p^i}$', when we are over a general $\Z_{(p)}$-algebra $R$.
\begin{itemize}\squishlist
\item When $R$ is torsion free, we lose no information tensoring with $\Q$, so that we could say that $F\in\FGL(R)$ is $p$-typical when $\log_{F}=\sum\ell_ix^{p^i}$ (with $\ell_0=1$).
\item When $R$ has torsion, tensoring with $\Q$ does lose information.
\item Suppose that $q\in\N$ is invertible in $R$, and define
\[f_q(x):=[1/q]\left(\Fsum_{i=1}^q\zeta^ix\right)\]
where $\zeta$ is a primitive $q^\textup{th}$ root of unity. This is defined over $R$, and when $R$ is torsion free,
\[\log(x)=\sum_{i\geq0}m_ix^{i+1}\implies \log(f_q(x))=\sum_{j>0}m_{qj-1}x^{qj}.\]
\item It is then natural to define:
\begin{defn*}
If $R$ is a $\Z_{(p)}$-algebra, $F\in\FGL(R)$ is \textbf{$p$-typical} if $f_{q}(x)=0$ for all primes $q\neq p$.
\end{defn*}
\end{itemize}
\item  We'd like to prove the following theorem of Cartier:
\begin{thm*}[A2.1.18]
Every formal group law over a $\Z_{(p)}$-algebra is canonically strictly isomorphic to a $p$-typical one.
\end{thm*}
To prove this, it'll be enough to give a strict isomorphism from the universal FGL $F$ over $L\otimes \Z_{(p)}$, to $p$-typical FGL $\widetilde{F}$. To see this, suppose that $G\in\FGL(R)$, for \smash{$R\in\ZpAlg$}. Then there is a unique map $\theta:L\to R$ such that $\theta(F)=R$. Now given a strict isomorphism $f:F\to\widetilde{F}$, we obtain a strict isomorphism $\theta(f):G\to\widetilde{G}$.
\begin{shaded}
We'll need the M\"obius function $\mu:\N\to\{-1,0,1\}$.\footnote{Here $\N=\{1,2,\ldots\}$.} Given an arithmetic function $\alpha$, we can form another arithmetic function $\beta$ by the formula below, which determines $\alpha$ using the M\"obius function $\mu$:
\[\alpha\quad \rightsquigarrow \quad \beta(n):=\sum_{q|n}\alpha(q)\quad \rightsquigarrow\quad  \alpha(n)=\sum_{q|n}\mu(q)\beta(n/q).\]
This is the \emph{M\"obius inversion formula}. The M\"obius function $\mu$ is the function $\alpha$ needed so that $\beta(n)=\delta_{n1}$. Explicitly:
\[\mu(n)=\begin{cases}
0,&\textup{if $n$ is divisible by a square};\\
(-1)^r,&\textup{if $n$ is the product of $r$ distinct primes.}
\end{cases}
\]
It is not hard to see by direct inspection of $\mu$ that 
\[\sum_{p\nmid q\mid n}\mu(q)=\begin{cases}
1,&\textup{if }n=p^k;\\
0,&\textup{otherwise}.
%\\,&\textup{if }
\end{cases}
\]
\end{shaded}
Now we are looking for a strict isomorphism $f$ (dotted) such that we have
\[\textup{mog}(x):=\log_{F'}=\log_{F}(f^{-1}(x))=\sum_{i\geq0}m_{p^i-1}x^{p^{i}}\textup{\qquad in\qquad }
\vcenter{\xymatrix@R=.3cm@C=.4cm{
\ar[rd]_-(.0){\textup{log}_F}F\ar@{..>}[rr]^-{f}&%r1c1
&%r1c2
\widetilde{F}\ar[ld]^-(.0){\textup{mog}=\textup{log}\circ f^{-1}}\\%r1c3
&%r2c1
\textup{Add}&%r2c2
%r2c3
}}\]
We define $f$ by the following {formal} sum, whose logarithm we understand:
\begin{alignat*}{2}
f^{-1}(x)
&:=
\Fsum_{p\nmid q}[\mu(q)]_F(f_q(x))%
\\
\log_F(f^{-1}(x))
&=
\sum_{p\nmid q}\mu(q)\cdot\log_F(f_q(x))%
&\qquad&\text{(applying $\log_F$)}\\
&=
\sum_{p\nmid q}\mu(q)\cdot\sum_{j>0}m_{qj-1}x^{qj}%
&\qquad&\text{(by definition of $f_q$)}\\
&=
\sum_{n>0}\left(\sum_{p\nmid q\mid n}\mu(q)\right)m_{n-1}x^{n}%
&\qquad&\text{(collecting like terms)}\\
% Left hand side
% Relation
&=
% Right hand side
\sum_{i\geq0}m_{p^i-1}x^{p^{i}}%
% Comment
&\qquad&\text{(by the formula above)}
\end{alignat*}
This proves A2.1.18, as $L\otimes\Z_{(p)}$ is a \emph{torsion free} $\Z_{(p)}$-algebra.
\item 
At this point, we need to be clear:
\begin{itemize}\squishlist
\item $L\otimes\Z_{(p)}$ corepresents $\FGL:\ZpAlg\to\Set$.
\item $\FGL$ has a sub-functor $\FGLp:\ZpAlg\to\Set$, which sends a $\Z_{(p)}$-algebra to the set of $p$-typical FGLs over it.
\item We have chosen a strict isomorphism $f:F\to\widetilde{F}$ of FGLs on $L\otimes\Z_{(p)}$. \item $f$ induces a natural assignment of a strict isomorphism $G\to\widetilde{G}$, whenever $G\in\FGL(R)$, for $R\in\ZpAlg$.
\item Sending $G$ to $\widetilde{G}$ gives a natural transformation $\sim:\FGL\to\FGL$, which must be induced by an endomorphism $\phi$ of $L\otimes\Z_{(p)}$.
\item $\phi$ is determined by its action on the generators $m_i$ of ${L}\otimes{\Q}$, and as we know the logarithm of $\widetilde{F}$, we see that
\[\phi(m_i)=\begin{cases}
m_i,&\textup{if }i=p^k-1;\\
0,&\textup{otherwise}.
%\\,&\textup{if }
\end{cases}
\]
\item This $\phi$ is idempotent, so that we have an idempotent natural endomorphism of $\FGL$, whose image lies in $\FGLp$.
\item $\sim$ acts as the identity on the set of $p$-typical FGLs, as it is defined by the same formula on a general FGL, and we have the correct definition of $p$-typical. Thus $\FGLp$ is the image of $\sim$.
%\item The $p$-typical FGL $\widetilde{F}$ on $L\otimes\Z_{(p)}$ is 
\end{itemize}
We have almost proven
\begin{thm*}
Let $V=\Z_{(p)}[v_1,v_2,\ldots]$ with $|v_n|=2(p^n-1)$. Then:
\begin{itemize}\squishlist
\item  $V$ corepresents $\FGLp$ --- there is a universal $p$-typical formal group law $\widetilde{F}\in\FGLp(V)$.
\item $V$ is the image of the idempotent $\phi:{L}\otimes{\Z_{(p)}}\to{L}\otimes{\Z_{(p)}}$.
\item As such, the map ${L}\otimes{\Z_{(p)}}\to V$ corresponding to $\widetilde{F}\in\FGL(V)$ is the projection onto a direct summand.
\end{itemize}
\end{thm*}
\begin{proof}
Suppose that $G\in\FGLp(R)$. Then $G$ is also an element of $\FGL(R)$, represented by a map $g:L\otimes\Z_{(p)}\to R$. As $G=\widetilde{G}$, we have a commuting diagram, the full arrows below:
\[\xymatrix@R=1cm{
L\otimes\Z_{(p)}\ar[r]^-{\phi}\ar[dr]_-{g}&%r1c1
L\otimes\Z_{(p)}\ar[d]^-{g}&
\im(\phi)\ar@{_{(}->}[l]\ar@{-->}[ld]^-{\exists !}\\
&%r2c1
R%r2c2
}\]
As $\phi$ is an idempotent, there is a unique map completing the right hand triangle. Thus, $V:=\im(\phi)$ corepresents $\FGLp(R)$. The calculation of the structure of $V$ is similar to that for $L$, apparently.
\end{proof}
\end{itemise}
\subsection*{$p$-typical matched pairs}
\begin{itemise}
\item We're going to need a ring $VT$ which corepresents the functor $\SIp$ of $p$-typical matched pairs:
\[\SIp:\ZpAlg\to\Set\qquad \SIp(R):=\left\{(F,f,G)\in\SI(R)\,\middle|\,F,G\in\FGLp(R)\right\}.\]
Now in the $\FGL$ story, it happened that one could choose $f$ independently of $F$, but this is no longer true --- in order to ensure that $G$ is $p$-typical, one must choose $f$ in a way that depends on $F$. For this reason, the resulting Hopf algebroid $(V,VT)$ will not be split.
\begin{lem*}[A2.1.26]
Suppose $F\in\FGLp(R)$, $G\in\FGL(R)$ and $f:F\to G$ is an isomorphism. Then
\[G\textup{\ is $p$-typical\ }\iff f^{-1}(x)=\Fsum_{i\geq0}t_ix^{p^i}\textup{\ for $t_i\in R$, $t_0$ invertible.}\]
Of course, if $f$ is a strict isomorphism, then $t_0=1$.
\end{lem*}
That is, the power series that we can use in a $p$-typical matched pair $(F,f,G)$ are the $F$-formal sums of power series $t_ix^{p^{i}}$.
\item From this, it follows that $VT=V\otimes\Z_{(p)}[t_1,\ldots]$, with $|t_n|=|v_n|=2(p^n-1)$. One can determine the structure maps of the Hopf algebroid $(V,VT)$, as is described in A2.1.27.

\INDENT The structure maps are described in terms of the elements $\ell_i\in V\otimes\Q$, which are defined to be the image of $m_{p^i-1}\in {L}\otimes{\Q}$ under $\phi:L\to V$.

\INDENT The resulting Hopf algebroid is isomorphic to $(BP_*,BP_*(BP))$.
\end{itemise}
\subsection*{Choice of generators for $V$}
\begin{itemise}
\item 
Hazewinkel's generators $v_n$ of $V$ are defined recursively by:
\[p\ell_n=\sum_{0\leq i<n}\ell_iv_{n-i}^{p^i}\textup{,\quad i.e.\quad }v_n=p\ell_n-\sum_{0< i<n}\ell_iv_{n-i}^{p^i}\]
One must prove that $v_n\in V$, and that the $v_n$ generate.
\item 
Araki's generators $v_n$ of $V$ are defined recursively, starting with $v_0:=p$, by:
\[p\ell_n=\sum_{0\leq i\leq n}\ell_iv_{n-i}^{p^i}\textup{,\quad i.e.\quad }v_n=p\ell_n-\sum_{0< i\leq n}\ell_iv_{n-i}^{p^i}\]
These give worse formulae for $\ell_n$, but better ones for $\eta_R$. We also have a defining formula:
\[[p]_F(x)=\Fsum_{i\geq0}v_ix^{p^i}.\]
\item From now on, $v_n$ will mean the corresponding Araki generator. Note, however:
 \begin{thm*}[A2.2.3]
The Hazewinkel and Araki generators are congruent mod $(p)$.
\end{thm*}
\item A formula is given A2.2.5 for $\eta_R$ in terms of the Araki generators.
\end{itemise}
\subsection*{The height of a FGL over an $\F_p$-algebra}
\begin{itemise}
\item The following lemma will help us to define the height of a FGL over an $\F_p$-algebra.
\begin{lem*}[A2.2.6]
Let $F$ be an FGL over a commutative $\F_p$-algebra, and let $f:F\to F$ be a nontrivial endomorphism. Then for some $n$, $f(x)$ is a power series in $x^{p^n}$ with leading term $ax^{p^n}$.
\end{lem*}
\item We have the endomorphism $[p]_F:F\to F$, which then has leading coefficient $ax^{p^n}$, and we say that $F$ has \textbf{height} $n$. If $[p]_F=0$, we say that the height is $\infty$.
\item \rednote{Over any $\F_p$-algebra R} we can define a formal group law $F_n$ of each height $n\leq\infty$:
\begin{itemize}\squishlist
\item Let $F_{\infty}(x,y)=x+y$.
\item Let $F_n$ be the FGL induced by the map
\[\theta:V\to R,\qquad v_i\mapsto\begin{cases}
1,&\textup{if }i=n;\\
0,&\textup{otherwise}.
%\\,&\textup{if }
\end{cases}
\]
\end{itemize}
\item When $K$ is a separably closed field of characteristic $p>0$, any FGL over $K$ is isomorphic to one of these.
\end{itemise}
\subsection*{Endomorphism rings}
\begin{itemise}
\item If $F\in\FGL(K)$, where $K$ is a field of characteristic $p$, the set $E$ of endomorphisms of $F$ is an integral domain under composition and formal sum.
\item It also happens that $E$ is a $\Z_p$-algebra, where $\Z_p$ is the $p$-adic integers. To give the structure, note that a $p$-adic integer $a$ can be written as $a=\sum a_ip^i$ for $a_i\in\Z$. Then we may define the image of $a$ in $E$ to be $[a]_F$ where:
\[[a]_F(x):=\Fsum [a_i]_F([p^i]_F(x)).\]
Note that $[p^i]_F(x)=[p]_F([p]_F([p]_F( \cdots([p]_F(x))\cdots)))$, and $[p]_F(x)$ is a power series with leading term $ax^{p^n}$, where $n>0$ is the height of $F$. Thus, this definition makes sense.

\INDENT When $F$ has finite height, $E$ is free as a $\Z_p$-module \rednote{(I don't get why...)}

\INDENT When $F$ has height $\infty$, $[p]_F(x)=0$, so that $F$ is an $\F_p$-vector space.
\item It remains for me to read pp357--358.
\end{itemise}
\end{Formal Group Laws}

\end{document}






A-adams resolution:
\[\xymatrix{
X_0\ar[d]^-{f_0}&%r1c1
X_1\ar[l]_-{g_0}\ar[d]^-{f_1}&%r1c2
X_2\ar[d]^-{f_2}\ar[l]_-{g_1}&%r1c3
X_3\ar[l]_-{g_2}\\%r1c4
K_0&%r2c1
K_1&%r2c2
K_2&%r2c3
%r2c4
}\]

Jandr Claims:
\[\xymatrix{
\pi_n(X)\ar[rr]\ar[d]^-{\sim}&%r1c1
&%r1c2
H_n(X)\\%r1c3
\pi_{n-1}(\Omega X)\ar[r]&%r2c1
H_{n-1}{(\Omega X)}\ar[r]^-{\sim}&%r2c2
H_n(\Sigma\Omega X)\ar[u]%r2c3
}\]


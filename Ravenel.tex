% !TEX root = z_output/_Ravenel.tex

%%%%%%%%%%%%%%%%%%%%%%%%%%% 80 characters %%%%%%%%%%%%%%%%%%%%%%%%%%%%%%%%%%%%%%

\documentclass[11pt]{article}
\usepackage{fullpage}
\usepackage{amsmath,amsthm,amssymb}
\usepackage{mathrsfs,nicefrac}
\usepackage{amssymb}
\usepackage{epsfig}
\usepackage[all,2cell]{xy}
\usepackage{sseq}
\usepackage{tocloft}
\usepackage{cancel}
\usepackage[strict]{changepage}
\usepackage{color}
\usepackage{tikz}
\usepackage{extpfeil}
\usepackage{version}
\usepackage{framed}
\definecolor{shadecolor}{rgb}{.925,0.925,0.925}

%\usepackage{ifthen}
%Used for disabling hyperref
\ifx\dontloadhyperref\undefined
%\usepackage[pdftex,pdfborder={0 0 0 [1 1]}]{hyperref}
\usepackage[pdftex,pdfborder={0 0 .5 [1 1]}]{hyperref}
\else
\providecommand{\texorpdfstring}[2]{#1}
\fi
%>>>>>>>>>>>>>>>>>>>>>>>>>>>>>>
%<<<        Versions        <<<
%>>>>>>>>>>>>>>>>>>>>>>>>>>>>>>
%Add in the following line to include all the versions.
%\def\excludeversion#1{\includeversion{#1}}

%>>>>>>>>>>>>>>>>>>>>>>>>>>>>>>
%<<<       Better ToC       <<<
%>>>>>>>>>>>>>>>>>>>>>>>>>>>>>>
\setlength{\cftbeforesecskip}{0.5ex}

%>>>>>>>>>>>>>>>>>>>>>>>>>>>>>>
%<<<      Hyperref mod      <<<
%>>>>>>>>>>>>>>>>>>>>>>>>>>>>>>

%needs more testing
\newcounter{dummyforrefstepcounter}
\newcommand{\labelRIGHTHERE}[1]
{\refstepcounter{dummyforrefstepcounter}\label{#1}}


%>>>>>>>>>>>>>>>>>>>>>>>>>>>>>>
%<<<  Theorem Environments  <<<
%>>>>>>>>>>>>>>>>>>>>>>>>>>>>>>
\ifx\dontloaddefinitionsoftheoremenvironments\undefined
\theoremstyle{plain}
\newtheorem{thm}{Theorem}[section]
\newtheorem*{thm*}{Theorem}
\newtheorem{lem}[thm]{Lemma}
\newtheorem*{lem*}{Lemma}
\newtheorem{prop}[thm]{Proposition}
\newtheorem*{prop*}{Proposition}
\newtheorem{cor}[thm]{Corollary}
\newtheorem*{cor*}{Corollary}
\newtheorem{defprop}[thm]{Definition-Proposition}
\newtheorem*{punchline}{Punchline}
\newtheorem*{conjecture}{Conjecture}
\newtheorem*{claim}{Claim}

\theoremstyle{definition}
\newtheorem{defn}{Definition}[section]
\newtheorem*{defn*}{Definition}
\newtheorem{exmp}{Example}[section]
\newtheorem*{exmp*}{Example}
\newtheorem*{exmps*}{Examples}
\newtheorem*{nonexmp*}{Non-example}
\newtheorem{asspt}{Assumption}[section]
\newtheorem{notation}{Notation}[section]
\newtheorem{exercise}{Exercise}[section]
\newtheorem*{fact*}{Fact}
\newtheorem*{rmk*}{Remark}
\newtheorem{fact}{Fact}
\newtheorem*{aside}{Aside}
\newtheorem*{question}{Question}
\newtheorem*{answer}{Answer}

\else\relax\fi

%>>>>>>>>>>>>>>>>>>>>>>>>>>>>>>
%<<<      Fields, etc.      <<<
%>>>>>>>>>>>>>>>>>>>>>>>>>>>>>>
\DeclareSymbolFont{AMSb}{U}{msb}{m}{n}
\DeclareMathSymbol{\N}{\mathbin}{AMSb}{"4E}
\DeclareMathSymbol{\Octonions}{\mathbin}{AMSb}{"4F}
\DeclareMathSymbol{\Z}{\mathbin}{AMSb}{"5A}
\DeclareMathSymbol{\R}{\mathbin}{AMSb}{"52}
\DeclareMathSymbol{\Q}{\mathbin}{AMSb}{"51}
\DeclareMathSymbol{\PP}{\mathbin}{AMSb}{"50}
\DeclareMathSymbol{\I}{\mathbin}{AMSb}{"49}
\DeclareMathSymbol{\C}{\mathbin}{AMSb}{"43}
\DeclareMathSymbol{\A}{\mathbin}{AMSb}{"41}
\DeclareMathSymbol{\F}{\mathbin}{AMSb}{"46}
\DeclareMathSymbol{\G}{\mathbin}{AMSb}{"47}
\DeclareMathSymbol{\Quaternions}{\mathbin}{AMSb}{"48}


%>>>>>>>>>>>>>>>>>>>>>>>>>>>>>>
%<<<       Operators        <<<
%>>>>>>>>>>>>>>>>>>>>>>>>>>>>>>
\DeclareMathOperator{\ad}{\textbf{ad}}
\DeclareMathOperator{\coker}{coker}
\renewcommand{\ker}{\textup{ker}\,}
\DeclareMathOperator{\End}{End}
\DeclareMathOperator{\Aut}{Aut}
\DeclareMathOperator{\Hom}{Hom}
\DeclareMathOperator{\Maps}{Maps}
\DeclareMathOperator{\Mor}{Mor}
\DeclareMathOperator{\Gal}{Gal}
\DeclareMathOperator{\Ext}{Ext}
\DeclareMathOperator{\Tor}{Tor}
\DeclareMathOperator{\Map}{Map}
\DeclareMathOperator{\Der}{Der}
\DeclareMathOperator{\Rad}{Rad}
\DeclareMathOperator{\rank}{rank}
\DeclareMathOperator{\ArfInvariant}{Arf}
\DeclareMathOperator{\KervaireInvariant}{Ker}
\DeclareMathOperator{\im}{im}
\DeclareMathOperator{\coim}{coim}
\DeclareMathOperator{\trace}{tr}
\DeclareMathOperator{\supp}{supp}
\DeclareMathOperator{\ann}{ann}
\DeclareMathOperator{\spec}{Spec}
\DeclareMathOperator{\SPEC}{\textbf{Spec}}
\DeclareMathOperator{\proj}{Proj}
\DeclareMathOperator{\PROJ}{\textbf{Proj}}
\DeclareMathOperator{\fiber}{F}
\DeclareMathOperator{\cofiber}{C}
\DeclareMathOperator{\cone}{cone}
\DeclareMathOperator{\skel}{sk}
\DeclareMathOperator{\coskel}{cosk}
\DeclareMathOperator{\conn}{conn}
\DeclareMathOperator{\colim}{colim}
\DeclareMathOperator{\limit}{lim}
\DeclareMathOperator{\ch}{ch}
\DeclareMathOperator{\Vect}{Vect}
\DeclareMathOperator{\GrthGrp}{GrthGp}
\DeclareMathOperator{\Sym}{Sym}
\DeclareMathOperator{\Prob}{\mathbb{P}}
\DeclareMathOperator{\Exp}{\mathbb{E}}
\DeclareMathOperator{\GeomMean}{\mathbb{G}}
\DeclareMathOperator{\Var}{Var}
\DeclareMathOperator{\Cov}{Cov}
\DeclareMathOperator{\Sp}{Sp}
\DeclareMathOperator{\Seq}{Seq}
\DeclareMathOperator{\Cyl}{Cyl}
\DeclareMathOperator{\Ev}{Ev}
\DeclareMathOperator{\sh}{sh}
\DeclareMathOperator{\intHom}{\underline{Hom}}
\DeclareMathOperator{\Frac}{frac}



%>>>>>>>>>>>>>>>>>>>>>>>>>>>>>>
%<<<   Cohomology Theories  <<<
%>>>>>>>>>>>>>>>>>>>>>>>>>>>>>>
\DeclareMathOperator{\KR}{{K\R}}
\DeclareMathOperator{\KO}{{KO}}
\DeclareMathOperator{\K}{{K}}
\DeclareMathOperator{\OmegaO}{{\Omega_{\Octonions}}}

%>>>>>>>>>>>>>>>>>>>>>>>>>>>>>>
%<<<   Algebraic Geometry   <<<
%>>>>>>>>>>>>>>>>>>>>>>>>>>>>>>
\DeclareMathOperator{\Spec}{Spec}
\DeclareMathOperator{\Proj}{Proj}
\DeclareMathOperator{\Sing}{Sing}
\DeclareMathOperator{\shfHom}{\mathscr{H}\textit{\!\!om}}
\DeclareMathOperator{\WeilDivisors}{{Div}}
\DeclareMathOperator{\CartierDivisors}{{CaDiv}}
\DeclareMathOperator{\PrincipalWeilDivisors}{{PrDiv}}
\DeclareMathOperator{\LocallyPrincipalWeilDivisors}{{LPDiv}}
\DeclareMathOperator{\PrincipalCartierDivisors}{{PrCaDiv}}
\DeclareMathOperator{\DivisorClass}{{Cl}}
\DeclareMathOperator{\CartierClass}{{CaCl}}
\DeclareMathOperator{\Picard}{{Pic}}
\DeclareMathOperator{\Frob}{Frob}


%>>>>>>>>>>>>>>>>>>>>>>>>>>>>>>
%<<<  Mathematical Objects  <<<
%>>>>>>>>>>>>>>>>>>>>>>>>>>>>>>
\newcommand{\sll}{\mathfrak{sl}}
\newcommand{\gl}{\mathfrak{gl}}
\newcommand{\GL}{\mbox{GL}}
\newcommand{\PGL}{\mbox{PGL}}
\newcommand{\SL}{\mbox{SL}}
\newcommand{\Mat}{\mbox{Mat}}
\newcommand{\Gr}{\textup{Gr}}
\newcommand{\Squ}{\textup{Sq}}
\newcommand{\catSet}{\textit{Sets}}
\newcommand{\RP}{{\R\PP}}
\newcommand{\CP}{{\C\PP}}
\newcommand{\Steen}{\mathscr{A}}
\newcommand{\Orth}{\textup{\textbf{O}}}

%>>>>>>>>>>>>>>>>>>>>>>>>>>>>>>
%<<<  Mathematical Symbols  <<<
%>>>>>>>>>>>>>>>>>>>>>>>>>>>>>>
\newcommand{\DASH}{\textup{---}}
\newcommand{\op}{\textup{op}}
\newcommand{\CW}{\textup{CW}}
\newcommand{\ob}{\textup{ob}\,}
\newcommand{\ho}{\textup{ho}}
\newcommand{\st}{\textup{st}}
\newcommand{\id}{\textup{id}}
\newcommand{\Bullet}{\ensuremath{\bullet} }
\newcommand{\sprod}{\wedge}

%>>>>>>>>>>>>>>>>>>>>>>>>>>>>>>
%<<<      Some Arrows       <<<
%>>>>>>>>>>>>>>>>>>>>>>>>>>>>>>
\newcommand{\nt}{\Longrightarrow}
\let\shortmapsto\mapsto
\let\mapsto\longmapsto
\newcommand{\mapsfrom}{\,\reflectbox{$\mapsto$}\ }
\newcommand{\bigrightsquig}{\scalebox{2}{\ensuremath{\rightsquigarrow}}}
\newcommand{\bigleftsquig}{\reflectbox{\scalebox{2}{\ensuremath{\rightsquigarrow}}}}

%\newcommand{\cofibration}{\xhookrightarrow{\phantom{\ \,{\sim\!}\ \ }}}
%\newcommand{\fibration}{\xtwoheadrightarrow{\phantom{\sim\!}}}
%\newcommand{\acycliccofibration}{\xhookrightarrow{\ \,{\sim\!}\ \ }}
%\newcommand{\acyclicfibration}{\xtwoheadrightarrow{\sim\!}}
%\newcommand{\leftcofibration}{\xhookleftarrow{\phantom{\ \,{\sim\!}\ \ }}}
%\newcommand{\leftfibration}{\xtwoheadleftarrow{\phantom{\sim\!}}}
%\newcommand{\leftacycliccofibration}{\xhookleftarrow{\ \ {\sim\!}\,\ }}
%\newcommand{\leftacyclicfibration}{\xtwoheadleftarrow{\sim\!}}
%\newcommand{\weakequiv}{\xrightarrow{\ \,\sim\,\ }}
%\newcommand{\leftweakequiv}{\xleftarrow{\ \,\sim\,\ }}

\newcommand{\cofibration}
{\xhookrightarrow{\phantom{\ \,{\raisebox{-.3ex}[0ex][0ex]{\scriptsize$\sim$}\!}\ \ }}}
\newcommand{\fibration}
{\xtwoheadrightarrow{\phantom{\raisebox{-.3ex}[0ex][0ex]{\scriptsize$\sim$}\!}}}
\newcommand{\acycliccofibration}
{\xhookrightarrow{\ \,{\raisebox{-.55ex}[0ex][0ex]{\scriptsize$\sim$}\!}\ \ }}
\newcommand{\acyclicfibration}
{\xtwoheadrightarrow{\raisebox{-.6ex}[0ex][0ex]{\scriptsize$\sim$}\!}}
\newcommand{\leftcofibration}
{\xhookleftarrow{\phantom{\ \,{\raisebox{-.3ex}[0ex][0ex]{\scriptsize$\sim$}\!}\ \ }}}
\newcommand{\leftfibration}
{\xtwoheadleftarrow{\phantom{\raisebox{-.3ex}[0ex][0ex]{\scriptsize$\sim$}\!}}}
\newcommand{\leftacycliccofibration}
{\xhookleftarrow{\ \ {\raisebox{-.55ex}[0ex][0ex]{\scriptsize$\sim$}\!}\,\ }}
\newcommand{\leftacyclicfibration}
{\xtwoheadleftarrow{\raisebox{-.6ex}[0ex][0ex]{\scriptsize$\sim$}\!}}
\newcommand{\weakequiv}
{\xrightarrow{\ \,\raisebox{-.3ex}[0ex][0ex]{\scriptsize$\sim$}\,\ }}
\newcommand{\leftweakequiv}
{\xleftarrow{\ \,\raisebox{-.3ex}[0ex][0ex]{\scriptsize$\sim$}\,\ }}

%>>>>>>>>>>>>>>>>>>>>>>>>>>>>>>
%<<<    xymatrix Arrows     <<<
%>>>>>>>>>>>>>>>>>>>>>>>>>>>>>>
\newdir{ >}{{}*!/-5pt/@{>}}
\newcommand{\xycof}{\ar@{ >->}}
\newcommand{\xycofib}{\ar@{^{(}->}}
\newcommand{\xycofibdown}{\ar@{_{(}->}}
\newcommand{\xyfib}{\ar@{->>}}
\newcommand{\xymapsto}{\ar@{|->}}

%>>>>>>>>>>>>>>>>>>>>>>>>>>>>>>
%<<<     Greek Letters      <<<
%>>>>>>>>>>>>>>>>>>>>>>>>>>>>>>
%\newcommand{\oldphi}{\phi}
%\renewcommand{\phi}{\varphi}
\let\oldphi\phi
\let\phi\varphi
\renewcommand{\to}{\longrightarrow}
\newcommand{\from}{\longleftarrow}
\newcommand{\eps}{\varepsilon}

%>>>>>>>>>>>>>>>>>>>>>>>>>>>>>>
%<<<  1st-4th & parentheses <<<
%>>>>>>>>>>>>>>>>>>>>>>>>>>>>>>
\newcommand{\first}{^\text{st}}
\newcommand{\second}{^\text{nd}}
\newcommand{\third}{^\text{rd}}
\newcommand{\fourth}{^\text{th}}
\newcommand{\ZEROTH}{$0^\text{th}$ }
\newcommand{\FIRST}{$1^\text{st}$ }
\newcommand{\SECOND}{$2^\text{nd}$ }
\newcommand{\THIRD}{$3^\text{rd}$ }
\newcommand{\FOURTH}{$4^\text{th}$ }
\newcommand{\iTH}{$i^\text{th}$ }
\newcommand{\jTH}{$j^\text{th}$ }
\newcommand{\nTH}{$n^\text{th}$ }

%>>>>>>>>>>>>>>>>>>>>>>>>>>>>>>
%<<<    upright commands    <<<
%>>>>>>>>>>>>>>>>>>>>>>>>>>>>>>
\newcommand{\upcol}{\textup{:}}
\newcommand{\upsemi}{\textup{;}}
\providecommand{\lparen}{\textup{(}}
\providecommand{\rparen}{\textup{)}}
\renewcommand{\lparen}{\textup{(}}
\renewcommand{\rparen}{\textup{)}}
\newcommand{\Iff}{\emph{iff} }

%>>>>>>>>>>>>>>>>>>>>>>>>>>>>>>
%<<<     Environments       <<<
%>>>>>>>>>>>>>>>>>>>>>>>>>>>>>>
\newcommand{\squishlist}
{ %\setlength{\topsep}{100pt} doesn't seem to do anything.
  \setlength{\itemsep}{.5pt}
  \setlength{\parskip}{0pt}
  \setlength{\parsep}{0pt}}
\newenvironment{itemise}{
\begin{list}{\textup{$\rightsquigarrow$}}
   {  \setlength{\topsep}{1mm}
      \setlength{\itemsep}{1pt}
      \setlength{\parskip}{0pt}
      \setlength{\parsep}{0pt}
   }
}{\end{list}\vspace{-.1cm}}
\newcommand{\INDENT}{\textbf{}\phantom{space}}
\renewcommand{\INDENT}{\rule{.7cm}{0cm}}

\newcommand{\itm}[1][$\rightsquigarrow$]{\item[{\makebox[.5cm][c]{\textup{#1}}}]}


%\newcommand{\rednote}[1]{{\color{red}#1}\makebox[0cm][l]{\scalebox{.1}{rednote}}}
%\newcommand{\bluenote}[1]{{\color{blue}#1}\makebox[0cm][l]{\scalebox{.1}{rednote}}}

\newcommand{\rednote}[1]
{{\color{red}#1}\makebox[0cm][l]{\scalebox{.1}{\rotatebox{90}{?????}}}}
\newcommand{\bluenote}[1]
{{\color{blue}#1}\makebox[0cm][l]{\scalebox{.1}{\rotatebox{90}{?????}}}}


\newcommand{\funcdef}[4]{\begin{align*}
#1&\to #2\\
#3&\mapsto#4
\end{align*}}

%\newcommand{\comment}[1]{}

%>>>>>>>>>>>>>>>>>>>>>>>>>>>>>>
%<<<       Categories       <<<
%>>>>>>>>>>>>>>>>>>>>>>>>>>>>>>
\newcommand{\Ens}{{\mathscr{E}ns}}
\DeclareMathOperator{\Sheaves}{{\mathsf{Shf}}}
\DeclareMathOperator{\Presheaves}{{\mathsf{PreShf}}}
\DeclareMathOperator{\Psh}{{\mathsf{Psh}}}
\DeclareMathOperator{\Shf}{{\mathsf{Shf}}}
\DeclareMathOperator{\Varieties}{{\mathsf{Var}}}
\DeclareMathOperator{\Schemes}{{\mathsf{Sch}}}
\DeclareMathOperator{\Rings}{{\mathsf{Rings}}}
\DeclareMathOperator{\AbGp}{{\mathsf{AbGp}}}
\DeclareMathOperator{\Modules}{{\mathsf{\!-Mod}}}
\DeclareMathOperator{\fgModules}{{\mathsf{\!-Mod}^{\textup{fg}}}}
\DeclareMathOperator{\QuasiCoherent}{{\mathsf{QCoh}}}
\DeclareMathOperator{\Coherent}{{\mathsf{Coh}}}
\DeclareMathOperator{\GSW}{{\mathcal{SW}^G}}
\DeclareMathOperator{\Burnside}{{\mathsf{Burn}}}
\DeclareMathOperator{\GSet}{{G\mathsf{Set}}}
\DeclareMathOperator{\FinGSet}{{G\mathsf{Set}^\textup{fin}}}
\DeclareMathOperator{\HSet}{{H\mathsf{Set}}}
\DeclareMathOperator{\Cat}{{\mathsf{Cat}}}
\DeclareMathOperator{\Fun}{{\mathsf{Fun}}}
\DeclareMathOperator{\Orb}{{\mathsf{Orb}}}
\DeclareMathOperator{\Set}{{\mathsf{Set}}}
\DeclareMathOperator{\sSet}{{\mathsf{sSet}}}
\DeclareMathOperator{\Top}{{\mathsf{Top}}}
\DeclareMathOperator{\GSpectra}{{G-\mathsf{Spectra}}}
\DeclareMathOperator{\Lan}{Lan}
\DeclareMathOperator{\Ran}{Ran}

%>>>>>>>>>>>>>>>>>>>>>>>>>>>>>>
%<<<     Script Letters     <<<
%>>>>>>>>>>>>>>>>>>>>>>>>>>>>>>
\newcommand{\scrQ}{\mathscr{Q}}
\newcommand{\scrW}{\mathscr{W}}
\newcommand{\scrE}{\mathscr{E}}
\newcommand{\scrR}{\mathscr{R}}
\newcommand{\scrT}{\mathscr{T}}
\newcommand{\scrY}{\mathscr{Y}}
\newcommand{\scrU}{\mathscr{U}}
\newcommand{\scrI}{\mathscr{I}}
\newcommand{\scrO}{\mathscr{O}}
\newcommand{\scrP}{\mathscr{P}}
\newcommand{\scrA}{\mathscr{A}}
\newcommand{\scrS}{\mathscr{S}}
\newcommand{\scrD}{\mathscr{D}}
\newcommand{\scrF}{\mathscr{F}}
\newcommand{\scrG}{\mathscr{G}}
\newcommand{\scrH}{\mathscr{H}}
\newcommand{\scrJ}{\mathscr{J}}
\newcommand{\scrK}{\mathscr{K}}
\newcommand{\scrL}{\mathscr{L}}
\newcommand{\scrZ}{\mathscr{Z}}
\newcommand{\scrX}{\mathscr{X}}
\newcommand{\scrC}{\mathscr{C}}
\newcommand{\scrV}{\mathscr{V}}
\newcommand{\scrB}{\mathscr{B}}
\newcommand{\scrN}{\mathscr{N}}
\newcommand{\scrM}{\mathscr{M}}

%>>>>>>>>>>>>>>>>>>>>>>>>>>>>>>
%<<<     Fractur Letters    <<<
%>>>>>>>>>>>>>>>>>>>>>>>>>>>>>>
\newcommand{\frakQ}{\mathfrak{Q}}
\newcommand{\frakW}{\mathfrak{W}}
\newcommand{\frakE}{\mathfrak{E}}
\newcommand{\frakR}{\mathfrak{R}}
\newcommand{\frakT}{\mathfrak{T}}
\newcommand{\frakY}{\mathfrak{Y}}
\newcommand{\frakU}{\mathfrak{U}}
\newcommand{\frakI}{\mathfrak{I}}
\newcommand{\frakO}{\mathfrak{O}}
\newcommand{\frakP}{\mathfrak{P}}
\newcommand{\frakA}{\mathfrak{A}}
\newcommand{\frakS}{\mathfrak{S}}
\newcommand{\frakD}{\mathfrak{D}}
\newcommand{\frakF}{\mathfrak{F}}
\newcommand{\frakG}{\mathfrak{G}}
\newcommand{\frakH}{\mathfrak{H}}
\newcommand{\frakJ}{\mathfrak{J}}
\newcommand{\frakK}{\mathfrak{K}}
\newcommand{\frakL}{\mathfrak{L}}
\newcommand{\frakZ}{\mathfrak{Z}}
\newcommand{\frakX}{\mathfrak{X}}
\newcommand{\frakC}{\mathfrak{C}}
\newcommand{\frakV}{\mathfrak{V}}
\newcommand{\frakB}{\mathfrak{B}}
\newcommand{\frakN}{\mathfrak{N}}
\newcommand{\frakM}{\mathfrak{M}}

\newcommand{\frakq}{\mathfrak{q}}
\newcommand{\frakw}{\mathfrak{w}}
\newcommand{\frake}{\mathfrak{e}}
\newcommand{\frakr}{\mathfrak{r}}
\newcommand{\frakt}{\mathfrak{t}}
\newcommand{\fraky}{\mathfrak{y}}
\newcommand{\fraku}{\mathfrak{u}}
\newcommand{\fraki}{\mathfrak{i}}
\newcommand{\frako}{\mathfrak{o}}
\newcommand{\frakp}{\mathfrak{p}}
\newcommand{\fraka}{\mathfrak{a}}
\newcommand{\fraks}{\mathfrak{s}}
\newcommand{\frakd}{\mathfrak{d}}
\newcommand{\frakf}{\mathfrak{f}}
\newcommand{\frakg}{\mathfrak{g}}
\newcommand{\frakh}{\mathfrak{h}}
\newcommand{\frakj}{\mathfrak{j}}
\newcommand{\frakk}{\mathfrak{k}}
\newcommand{\frakl}{\mathfrak{l}}
\newcommand{\frakz}{\mathfrak{z}}
\newcommand{\frakx}{\mathfrak{x}}
\newcommand{\frakc}{\mathfrak{c}}
\newcommand{\frakv}{\mathfrak{v}}
\newcommand{\frakb}{\mathfrak{b}}
\newcommand{\frakn}{\mathfrak{n}}
\newcommand{\frakm}{\mathfrak{m}}

%>>>>>>>>>>>>>>>>>>>>>>>>>>>>>>
%<<<  Caligraphic Letters   <<<
%>>>>>>>>>>>>>>>>>>>>>>>>>>>>>>
\newcommand{\calQ}{\mathcal{Q}}
\newcommand{\calW}{\mathcal{W}}
\newcommand{\calE}{\mathcal{E}}
\newcommand{\calR}{\mathcal{R}}
\newcommand{\calT}{\mathcal{T}}
\newcommand{\calY}{\mathcal{Y}}
\newcommand{\calU}{\mathcal{U}}
\newcommand{\calI}{\mathcal{I}}
\newcommand{\calO}{\mathcal{O}}
\newcommand{\calP}{\mathcal{P}}
\newcommand{\calA}{\mathcal{A}}
\newcommand{\calS}{\mathcal{S}}
\newcommand{\calD}{\mathcal{D}}
\newcommand{\calF}{\mathcal{F}}
\newcommand{\calG}{\mathcal{G}}
\newcommand{\calH}{\mathcal{H}}
\newcommand{\calJ}{\mathcal{J}}
\newcommand{\calK}{\mathcal{K}}
\newcommand{\calL}{\mathcal{L}}
\newcommand{\calZ}{\mathcal{Z}}
\newcommand{\calX}{\mathcal{X}}
\newcommand{\calC}{\mathcal{C}}
\newcommand{\calV}{\mathcal{V}}
\newcommand{\calB}{\mathcal{B}}
\newcommand{\calN}{\mathcal{N}}
\newcommand{\calM}{\mathcal{M}}

%>>>>>>>>>>>>>>>>>>>>>>>>>>>>>>
%<<<<<<<<<DEPRECIATED<<<<<<<<<<
%>>>>>>>>>>>>>>>>>>>>>>>>>>>>>>

%%% From Kac's template
% 1-inch margins, from fullpage.sty by H.Partl, Version 2, Dec. 15, 1988.
%\topmargin 0pt
%\advance \topmargin by -\headheight
%\advance \topmargin by -\headsep
%\textheight 9.1in
%\oddsidemargin 0pt
%\evensidemargin \oddsidemargin
%\marginparwidth 0.5in
%\textwidth 6.5in
%
%\parindent 0in
%\parskip 1.5ex
%%\renewcommand{\baselinestretch}{1.25}

%%% From the net
%\newcommand{\pullbackcorner}[1][dr]{\save*!/#1+1.2pc/#1:(1,-1)@^{|-}\restore}
%\newcommand{\pushoutcorner}[1][dr]{\save*!/#1-1.2pc/#1:(-1,1)@^{|-}\restore}










\excludeversion{ASSp7-8}
\excludeversion{Algebroid Definition}
\includeversion{Gamma-comodules}
\includeversion{Maps of algebroids}
\begin{document}
\begin{ASSp7-8}

\subsection*{The Adams Spectral Sequence (pages 7-8)}
Choose a space $X=X_0$. Work only in the stable range \rednote{(?)} and on the $p$-component. We always use $\F_p$ coefficients, and write $A$ for the mod $p$ Steenrod algebra.

Define spaces $X_i$ and $K_i$ inductively by writing $K_i=\prod_{j>0}K(H^j(X_i),j)$, and letting $X_{i+1}$ be the fibre of the obvious map $X_i\to K_i$. We have a system of fibrations, with wavy arrows denoting maps $\phi:\Omega K_i\to X_{i+1}$:
\[\xymatrix{
\cdots\ar[r]&
X_2\ar[r]\ar[d]&
X_1\ar[r]\ar[d]&
X_0\ar@{=}[r]\ar[d]&
X_0\ar@{=}[r]\ar[d]&
\cdots\\
&
K_2\ar@{~>}[ul]&
K_1\ar@{~>}[ul]&
K_0\ar@{~>}[ul]&
\ast\ar@{~>}[ul]&
}\]
This yields an exact couple, after applying $\pi_*$, whose wavy arrows lower degree by one:
\[\xymatrix{
\cdots\ar[r]&
\pi_*X_2\ar[r]\ar[d]&
\pi_*X_1\ar[r]\ar[d]&
\pi_*X_0\ar@{=}[r]\ar[d]&
\pi_*X_0\ar@{=}[r]\ar[d]&
\cdots\\
&
\pi_*K_2\ar@{~>}[ul]&
\pi_*K_1\ar@{~>}[ul]&
\pi_*K_0\ar@{~>}[ul]&
0\ar@{~>}[ul]&
}
\]
From this we obtain a spectral sequence:
\[E_1^{st}=\pi_{s+t}(K_s)\implies \pi_{s+t}(X_0)\textup{\qquad (when there are \textit{no ancients}).}\]
Now consider the map $\psi:K_i\to\Sigma X_{i+1}$ adjoint to $\phi:\Omega K_i\to X_{i+1}$. Then we have a commuting diagram:
\[\xymatrix{
X_{i+1}\ar[d]\\
K_{i+1}\ar[d]^\eta&\Omega K_i\ar[l]_\phi\ar[ld]^{\Omega\psi}\ar@{~>}[lu]\\
\Omega\Sigma K_{i+1}
}
\qquad 
\xymatrix{
\pi_{*-1}(X_{i+1})\ar[d]\\
\pi_{*-1}(K_{i+1})\ar[d]^\Sigma&\pi_{*} (K_i)\ar[l]\ar[ld]^{\psi_*}\ar@{~>}[lu]\\
\pi_{*}(\Sigma K_{i+1})
}
\]
In the stable range, suspension is an isomorphism, so we only need to consider the map $\psi_*$ in order to calculate the $E_2$ page. Here's where it gets interesting. We consider instead the map $\psi^*$ on cohomology. It is the long diagonal in the following commuting diagram:
\[\xymatrix{
H^{*-1}(X_{i+1})\ar@{=}[d]\ar[r]^{\phi^*}&H^{*-1}(\Omega K_i)\ar@{=}[r]&H^*(\Sigma\Omega K_i)& H^*(K_i)\ar[l]_{\quad  \eta^*}\\
H^{*}(\Sigma X_{i+1})\ar[rru]_{(\Sigma\phi)^*}&\\
H^*(\Sigma K_{i+1})\ar[uurrr]^{\psi^*}\xyfib[u]
}\]
Now the top row of this diagram represents the transgression map in the cohomology Serre spectral sequence for the fibration $X_{i+1}\to X_i\to K_i$. By design, every third map in the Serre short exact sequence is surjective, so we obtain short exact sequences:
\[\xymatrix{
0&
H^*(X_i)\ar[l]&
H^*(K_i)\ar[l]&
H^{*-1}(X_{i+1})\ar[l]_\tau&
0\ar[l]
}\]
Splicing these short exact sequences together, we obtain an exact sequence:
\[\xymatrix@!0@R=12mm@C=1.2cm{
0&&H^*(X_0)\ar[ll]&&
H^*(K_0)\ar[ll]&&
H^*(\Sigma K_1)\ar[ll]_{\psi^*}\xyfib[ld]&&
H^*(\Sigma^2 K_2)\ar[ll]_{\psi^*}\xyfib[ld]&&
\cdots\ar[ll]\\
&&&&&H^*(\Sigma X_1)\xycofibdown[lu]_\tau&&
H^*(\Sigma^2 X_2)\xycofibdown[lu]_\tau&&
}\]
We have that the differential is $\psi^*$ because of the previous commuting diagram.

Now the cohomology groups $H^*(K_*)$ are all free $A$-modules \rednote{(in a range, or something)}. Thus, the above exact sequence gives a free resolution of $H^*(X_0)$. The idea is that after applying $\Hom_A(\DASH,\F_p)$ to this exact sequence {(with $H^*(X)$ removed)} we should get a cochain complex isomorphic to the $E_1$ page of the spectral sequence, so that the $E_2$-page is $\Ext_A^{**}(H^*(X),\F_p)$, as desired.

There is a map \[\pi_*(K(\F_p,i))\to\Hom_{\F_p}(H^*(K(\F_p,i)),\F_p)\textup{\qquad sending\qquad }\sigma\mapsto\langle \DASH,h(\sigma)\rangle.\]
Now we note that this map factors as
\[\xymatrix{
\ar[r]\ar[dr]^\theta_\exists\pi_*(K(\F_p,i))&
\Hom_{\F_p}(H^*(K(\F_p,i)),\F_p)\\
&\Hom_{A}(H^*(K(\F_p,i)),\F_p)\xycofib[u]
}\]
Moreover, the map $\theta$ is an isomorphism \rednote{(some stable stuff happens here?)}. So there is such an isomorphism $\theta$ whenever the EM space is replaced by a product of EM spaces. This gives an isomorphism of cochain complexes between $(\pi_*K_*,\psi_*)$ and $(\Hom_A(H^*(K_*),\F_p),(\psi^*)^*)$, which shows that the $E_2$-page of the spectral sequence is $\Ext_A^{**}(H^*(X),\F_p)$.
\end{ASSp7-8}
\begin{Algebroid Definition}
\subsection*{Hopf Algebroids}
\newcommand\OB{\textup{OB}}
\newcommand\ARR{\textup{ARR}}
\newcommand\COMP{\textup{COMP}}
From the text: A Hopf algebroid over a commutative ring $K$ is a cogroupoid object in the category of (graded or bigraded) commutative $K$-algebras,
i.e., a pair $(A, \Gamma)$ of commutative $K$-algebras with structure maps such that for
any other commutative $K$-algebra $B$, the sets $\Hom(A, B)$ and $\Hom(\Gamma, B)$ are the
objects and morphisms of a groupoid.

We use the following abbreviations for certain co-representable functors:
\begin{itemize}\squishlist
\item $\OB:=\Hom(A,\DASH)$
\item $\ARR:=\Hom(\Gamma,\DASH)$
\item $\COMP:=\Hom(\Gamma\otimes_A\Gamma,\DASH)=\ARR\times_{\OB}\ARR$
\end{itemize}
We tabulate the structure maps, and what they correspond to under the co-Yoneda embedding.
\[\xymatrix@R=.1cm{
A\ar[r]^{\eta_L}&\Gamma&& \ARR\ar[r]^{\textup{src}}&\OB&& \textit{left unit or source} \\
A\ar[r]^{\eta_R}&\Gamma&& \ARR\ar[r]^{\textup{targ}}&\OB&& \textit{right unit or target} \\
\Gamma\ar[r]^-{\Delta}&\Gamma\otimes_A\Gamma
&&\ARR\ar[r]^\circ&\COMP&&\textit{coproduct or composition} \\
\Gamma\ar[r]^{\epsilon}&A&&
\OB\ar[r]^{\textup{iden}}&\ARR&&\textit{counit or identity} \\
\Gamma\ar[r]^{c}&\Gamma&&
\ARR\ar[r]^{\textup{inv}}&\ARR&&\textit{conjugation or inverse} 
&
}\]
We demand that:
\begin{itemize}\squishlist
\item $\Gamma$ is a left $A$-module via $\eta_L$;
\item $\Gamma$ is a right $A$-module via $\eta_R$;
\item $\Delta$ and $\epsilon$ are $A$-bimodule maps;
\item The source (likewise, target) of an
identity morphism are the object on which it is defined:
\[\xymatrix{
A\ar@{=}[d]\ar[r]^{\eta_{L}}& \Gamma\ar[ld]^\epsilon&&
%
\OB\ar@{=}[d];[] \ar[r];[]_{\textup{src}} &\ARR\ar[ld];[]_{\textup{iden}}
\\
A&&&
\OB
}\]
\item Pre-composition (likewise, post-composition) with the identity leaves a morphism
unchanged:
\[\xymatrix{
\Gamma\ar@{=}[dr]\ar[r]^-{\Delta}& \Gamma\otimes_A\Gamma
\ar[d]^{(\eta_L\epsilon)\otimes 1}&&
%
\ARR\ar@{=}[dr]\ar[r];[]_-{\circ}& \COMP\ar[d];[]_{((\textup{iden}\circ\textup{src}),\id_{\ARR})}&&\\
&\Gamma&&
&\ARR
}\]
\item Composition of morphisms is associative:
\[\xymatrix{
\Gamma\ar[r]^-{\Delta}\ar[d]^{\Delta}&
\Gamma\otimes\Gamma\ar[d]^{1\otimes\Delta}&&
\ARR\ar[r];[]_-{\circ}\ar[d];[]_{\circ}&
\ARR\times_{\OB}\ARR\ar[d];[]_{\id\times\circ}
\\
\Gamma\otimes\Gamma\ar[r]^-{\Delta\otimes1}&
\Gamma\otimes\Gamma\otimes\Gamma&&
\ARR\times_{\OB}\ARR\ar[r];[]_-{\circ\times\id}&
\ARR\times_{\OB}\ARR\times_{\OB}\ARR
}\]

\item The inverse of the inverse is the original morphism:
\[\textup{The composites }
\xymatrix{\Gamma\ar[r]^c& \Gamma\ar[r]^c&\Gamma}\textup{ and }\xymatrix{\ARR\ar[r];[]_{\textup{inv}}& \ARR\ar[r];[]_{\textup{inv}}&\ARR}\textup{ are the identity.}\]
\item Inverting a morphism interchanges source and
target:
\[\xymatrix{
A\ar[r]^{\eta_R}\ar[rd]_{\eta_L}&\Gamma\ar[d]^c&&
\OB\ar[r];[]_{\textup{targ}}\ar[rd];[]^{\textup{src}}& \ARR\ar[d];[]_{\textup{inv}}\\
&\Gamma&&
&\ARR
}\]


\item Composition of a
morphism with its inverse on either side gives an identity morphism:
\[\xymatrix@C=2.5cm{
\Gamma&\Gamma\otimes_K\Gamma\ar@{->>}[d]
\ar[l]_-%
{c(\gamma_1)\gamma_2\,\reflectbox{\tiny$\mapsto$}\, \gamma_1\otimes\gamma_2}&
\ARR&\ARR\times\ARR\xycofib[d];[]
\ar[l];[]^-%
{\phi\,\mbox{\tiny$\mapsto$}\,(\phi^{-1}\!,\,\phi)}
\\
%\ar[l];[]^-{(\textup{inv,id})}\\
%
&\Gamma\otimes_A\Gamma\ar@{-->}[ul]^{\exists!}&
&\ARR\times_{\OB}\ARR\ar@{-->}[ul];[]_{\exists!}\\
%
A\ar[uu]^{\eta_R}&\Gamma\ar[u]^\Delta \ar[l]_\epsilon&
\OB\ar[uu];[]_{\textup{targ}}& \ARR\ar[u];[]_{\circ}\ar[l];[]^{\textup{iden}}
}\]
\end{itemize}
\end{Algebroid Definition}
\begin{Gamma-comodules}
\subsection*{$\Gamma$-comodules}

\begin{itemize}\squishlist
\item A \textbf{left $\Gamma$-comodule} $M$ is:
\begin{itemize}\squishlist
\item a left $A$-module $M$;
\item a left $A$-module map $\psi:M\to\Gamma\otimes_A M$ which is counitary and coassociative:
\[\xymatrix{
M\ar[r]^-{\psi}&\Gamma\otimes M\ar[d]^{\epsilon\otimes1}&&
M\ar[d]^{\psi}\ar[r]^-{\psi}&\Gamma\otimes M\ar[d]^{\Delta\otimes1}\\
&A\otimes M\ar@{=}[ul]&&
\Gamma\otimes M\ar[r]^-{1\otimes\psi}& \Gamma\otimes\Gamma\otimes M
}\]
\begin{shaded}
I think that if $M=\Gamma$, $\eta_L:A\to \Gamma$ is one of these --- you have the right hand diagram corresponding to $\Delta$ being a module map, and the left hand diagram is replicated above.
\end{shaded}

\end{itemize}
\item A \textbf{comodule algebra} $M$ is:
\begin{itemize}\squishlist
\item a commutative associative $A$-algebra;
\item a left $\Gamma$-comodule via an \emph{algebra map} $\psi:M\to\Gamma\otimes_A M$.
\end{itemize}

\item The \textbf{tensor product} of left $\Gamma$-comodules has an obvious structure of left $\Gamma$-comodule.
\item If $\Gamma$ is flat as a (right) $A$-module, then the category of left $\Gamma$-comodules is abelian. Thus we'll always assume that $\Gamma$ is flat over $A$.
\item We can define a cotensor product $M\Box_\Gamma N$ whenever $M$ (resp.\ $N$) is a right (resp.\ left) $\Gamma$-comodule. This fails to be a comodule, or even an $A$-module.
\item We can turn left comodules into right comodules using the inverse map $c:\Gamma\to\Gamma$, so it makes sense to claim that $M\square_\Gamma N=N\Box_\Gamma M$.




\begin{shaded}
\item For any $A$-modules $M$ and $N$, we have a natural map $\gamma_M:\Hom_A(M,A)\otimes_AN\to\Hom_A(M,N)$. This is an isomorphism when $M$ is projective.

\INDENT To see this, write \smash{$F=M\oplus M'$} with $F$ free. Then $\gamma_F=\gamma_M\oplus\gamma_{M'}$, and $\gamma_F$ is obviously an isomorphism, so that $\gamma_M$ must be an isomorphism.
\end{shaded}
\item In light of the previous, if $M$ is a left $\Gamma$-comodule that is projective over $A$, then $\Hom_A(M,A)$ is a left comodule, via (I suppose):
\[\Hom_A(M,A)\overset{(\eta_L)_*}{\to}\Hom_A(M,\Gamma)\overset{\cong}{\from} \Hom_A(M,A)\otimes\Gamma\]
In particular, we can form the cotensor product in the following isomorphism, for left $\Gamma$-modules $M$ and $N$ with $M$ projective:
\[\Hom_\Gamma(M,N)=\Hom_{A}(M,A)\square_\Gamma N.\]
\end{itemize}
\end{Gamma-comodules}
\begin{Maps of algebroids}
\begin{itemize}\squishlist
\item A map of Hopf algebroids is the obvious.
\item Suppose that $f=(f_1,f_2):(A,\Gamma)\to(B,\Sigma)$ is a map of Hopf algebroids. Then
\begin{itemize}\squishlist
\item $B$ is a left $A$-module via $f_1$.
\item $\Sigma$ is a left $A$-module by restriction along $f_1$.
\item For any left $\Sigma$-comodule $N$, $\Gamma\otimes_AN$ is a left $\Gamma$-comodule, via: %by co-restricting along $f_2$:
\smash{$\Gamma\otimes_AN\overset{\Delta\otimes N}{\to}\Gamma\otimes_A\Gamma\otimes_AN$}.
\[\xymatrix{
A\ar[r]^-{\eta_R}\ar[d]^-{f_1}&%r1c1
\Gamma\ar[d]\ar[r]^-{(1\otimes f_2)\circ\Delta}&%r1c2
\Gamma\otimes_A\Sigma\ar@{=}[d]\\%r2c1
B\ar@/_1em/[rr]\ar[r]&%r3c1
\Gamma\otimes_AB\ar@{-->}[r]&%r3c2
\Gamma\otimes_A B\otimes_B\Sigma
}\]
(note that $B$ (obviously) maps into any tensor product over $B$...)
\end{itemize}

\end{itemize}

\end{Maps of algebroids}

\end{document}


A-adams resolution:
\[\xymatrix{
X_0\ar[d]^-{f_0}&%r1c1
X_1\ar[l]_-{g_0}\ar[d]^-{f_1}&%r1c2
X_2\ar[d]^-{f_2}\ar[l]_-{g_1}&%r1c3
X_3\ar[l]_-{g_2}\\%r1c4
K_0&%r2c1
K_1&%r2c2
K_2&%r2c3
%r2c4
}\]

Jandr Claims:
\[\xymatrix{
\pi_n(X)\ar[rr]\ar[d]^-{\sim}&%r1c1
&%r1c2
H_n(X)\\%r1c3
\pi_{n-1}(\Omega X)\ar[r]&%r2c1
H_{n-1}{(\Omega X)}\ar[r]^-{\sim}&%r2c2
H_n(\Sigma\Omega X)\ar[u]%r2c3
}\]



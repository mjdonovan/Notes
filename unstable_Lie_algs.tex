% !TEX root = z_output/unstable_Lie_algs.tex
\documentclass[11pt]{amsart}
%\newcommand{\IncludeAll}{}% <--
\newcommand{\ShrinkPages}{}% <--
\ifx\ShrinkPages\undefined\else\usepackage[paperheight=160mm,paperwidth=160mm]{geometry}\fi%220/160
\usepackage{amsmath,amsthm,amssymb}
\usepackage{eucal}
\usepackage{mathrsfs,nicefrac}
\usepackage{amssymb}
\usepackage[all]{xy}
\usepackage{cancel}
\usepackage{color}
\usepackage{version}
\usepackage{enumerate}
\usepackage{mathtools}%used for mathclap
\setcounter{tocdepth}{2}% <--
\newcommand{\xclude}[1]{\excludeversion{#1}} \newcommand{\nclude}[1]{\includeversion{#1}}
\xclude{comp func sseq old version}
\xclude{Cohomology operations for unstable Lie algebras over P}%depr
\xclude{Unstable Lie algebras over the Lambda-algebra}%depr, with some expos
\xclude{Cohomology operations for unstable Lie algebras over Lambda}%depr
\xclude{todolist}
\ifx\IncludeAll\undefined\else\renewcommand{\xclude}[1]{\nclude{#1}}\fi
\xclude{Contents Page}
\xclude{Intro}
\xclude{Conventions and notation}
\xclude{CPiAlgs and CHalgs}
\nclude{Constructing homotopy and cohomotopy operations}
\xclude{External spectral sequence operations}
\xclude{Lie algebras in characteristic 2 and their homotopy operations}
\xclude{Cohomology operations for all unstable Lie algebras}
\xclude{Koszul complexes}
\xclude{Composite functor spectral sequences}
\xclude{tricky proofs of operation compatibilities}
\xclude{Calculations of HWn for n nonzero}
\xclude{Calculations of HW0}
\xclude{appendices}
\xclude{bibliog}

\usepackage[bookmarks=false,pdftex,pdfborder={0 0 0 [1 1]}]{hyperref}

\headheight=8pt
\topmargin=0pt
\oddsidemargin=18pt
\evensidemargin=18pt
\textheight=610pt
\ifx\ShrinkPages\undefined\else %if you want to shrink the page while editing, do it here...
  \topmargin=-87pt
  \oddsidemargin=-60pt
  \evensidemargin=-60pt
  \textheight=439pt %610/439
\fi
\textwidth=432pt
\footskip=25pt


%>>>>>>>>>>>>>>>>>>>>>>>>>>>>>>
%<<<  Theorem Environments  <<<
%>>>>>>>>>>>>>>>>>>>>>>>>>>>>>>
\theoremstyle{plain}
\newtheorem{thm}{Theorem}[section]
\newtheorem*{thm*}{Theorem}
\newtheorem{lem}[thm]{Lemma}
\newtheorem*{lem*}{Lemma}
\newtheorem{prop}[thm]{Proposition}
\newtheorem*{prop*}{Proposition}
\newtheorem{cor}[thm]{Corollary}
\newtheorem*{cor*}{Corollary}
\newtheorem{defprop}[thm]{Definition-Proposition}
\newtheorem{conjecture}{Conjecture}
\newtheorem*{conjecture*}{Conjecture}
\newtheorem*{claim}{Claim}

\theoremstyle{definition}
\newtheorem{defn}{Definition}[section]
\newtheorem*{defn*}{Definition}


%>>>>>>>>>>>>>>>>>>>>>>>>>>>>>>
%<<<       Operators        <<<
%>>>>>>>>>>>>>>>>>>>>>>>>>>>>>>
\DeclareMathOperator{\ad}{\textbf{ad}}
\DeclareMathOperator{\coker}{coker}
\renewcommand{\ker}{\textup{ker}\,}
\DeclareMathOperator{\End}{End}
\DeclareMathOperator{\Aut}{Aut}
\DeclareMathOperator{\Hom}{Hom}
\DeclareMathOperator{\Maps}{Maps}
\DeclareMathOperator{\Mor}{Mor}
\DeclareMathOperator{\Gal}{Gal}
\DeclareMathOperator{\Ext}{Ext}
\DeclareMathOperator{\Tor}{Tor}
\DeclareMathOperator{\Cotor}{Cotor}
\DeclareMathOperator{\Prim}{Prim}
\DeclareMathOperator{\Tot}{Tot}
\DeclareMathOperator{\Map}{Map}
\DeclareMathOperator{\Der}{Der}
\DeclareMathOperator{\Rad}{Rad}
\DeclareMathOperator{\rank}{rank}
\DeclareMathOperator{\ArfInvariant}{Arf}
\DeclareMathOperator{\KervaireInvariant}{Ker}
\DeclareMathOperator{\im}{im}
\DeclareMathOperator{\coim}{coim}
\DeclareMathOperator{\trace}{tr}
\DeclareMathOperator{\supp}{supp}
\DeclareMathOperator{\ann}{ann}
\DeclareMathOperator{\spec}{Spec}
\DeclareMathOperator{\SPEC}{\textbf{Spec}}
\DeclareMathOperator{\proj}{Proj}
\DeclareMathOperator{\PROJ}{\textbf{Proj}}
\DeclareMathOperator{\fiber}{fib}
\DeclareMathOperator{\cofiber}{cof}
\DeclareMathOperator{\cone}{cone}
\DeclareMathOperator{\skel}{sk}
\DeclareMathOperator{\coskel}{cosk}
\DeclareMathOperator{\conn}{conn}
\DeclareMathOperator*{\colim}{colim}
\DeclareMathOperator*{\limit}{lim}
\DeclareMathOperator*{\hocolim}{hocolim}
\DeclareMathOperator*{\holimit}{holim}
\DeclareMathOperator*{\holim}{holim}
\DeclareMathOperator*{\hofib}{hofib}
\DeclareMathOperator*{\hotfib}{thofib}
\DeclareMathOperator*{\equaliser}{eq}
\DeclareMathOperator*{\coequaliser}{coeq}
\DeclareMathOperator{\ch}{ch}
\DeclareMathOperator{\Thom}{Th}
\DeclareMathOperator{\GrthGrp}{GrthGp}
\DeclareMathOperator{\Sym}{Sym}
\DeclareMathOperator{\Prob}{\mathbb{P}}
\DeclareMathOperator{\Exp}{\mathbb{E}}
\DeclareMathOperator{\GeomMean}{\mathbb{G}}
\DeclareMathOperator{\Var}{Var}
\DeclareMathOperator{\Cov}{Cov}
\DeclareMathOperator{\Sp}{Sp}
\DeclareMathOperator{\Seq}{Seq}
\DeclareMathOperator{\Cyl}{Cyl}
\DeclareMathOperator{\Ev}{Ev}
\DeclareMathOperator{\sh}{sh}
\DeclareMathOperator{\intHom}{\underline{Hom}}
\DeclareMathOperator{\Frac}{frac}


%>>>>>>>>>>>>>>>>>>>>>>>>>>>>>>
%<<<  Mathematical Symbols  <<<
%>>>>>>>>>>>>>>>>>>>>>>>>>>>>>>
\newcommand{\DASH}{\textup{--}}

%>>>>>>>>>>>>>>>>>>>>>>>>>>>>>>
%<<<     Greek Letters      <<<
%>>>>>>>>>>>>>>>>>>>>>>>>>>>>>>
\let\oldphi\phi
\let\phi\varphi
\renewcommand{\to}{\longrightarrow}
\newcommand{\from}{\longleftarrow}
\newcommand{\eps}{\varepsilon}

%%>>>>>>>>>>>>>>>>>>>>>>>>>>>>>>
%%<<<     Environments       <<<
%%>>>>>>>>>>>>>>>>>>>>>>>>>>>>>>
\newcommand{\squishlist}{
  \setlength{\itemsep}{.5pt}
  \setlength{\parskip}{0pt}
  \setlength{\parsep}{0pt}}
%>>>>>>>>>>>>>>>>>>>>>>>>>>>>>>
%<<<     Script Letters     <<<
%>>>>>>>>>>>>>>>>>>>>>>>>>>>>>>
\newcommand{\scrQ}{\mathscr{Q}}
\newcommand{\scrW}{\mathscr{W}}
\newcommand{\scrE}{\mathscr{E}}
\newcommand{\scrR}{\mathscr{R}}
\newcommand{\scrT}{\mathscr{T}}
\newcommand{\scrY}{\mathscr{Y}}
\newcommand{\scrU}{\mathscr{U}}
\newcommand{\scrI}{\mathscr{I}}
\newcommand{\scrO}{\mathscr{O}}
\newcommand{\scrP}{\mathscr{P}}
\newcommand{\scrA}{\mathscr{A}}
\newcommand{\scrS}{\mathscr{S}}
\newcommand{\scrD}{\mathscr{D}}
\newcommand{\scrF}{\mathscr{F}}
\newcommand{\scrG}{\mathscr{G}}
\newcommand{\scrH}{\mathscr{H}}
\newcommand{\scrJ}{\mathscr{J}}
\newcommand{\scrK}{\mathscr{K}}
\newcommand{\scrL}{\mathscr{L}}
\newcommand{\scrZ}{\mathscr{Z}}
\newcommand{\scrX}{\mathscr{X}}
\newcommand{\scrC}{\mathscr{C}}
\newcommand{\scrV}{\mathscr{V}}
\newcommand{\scrB}{\mathscr{B}}
\newcommand{\scrN}{\mathscr{N}}
\newcommand{\scrM}{\mathscr{M}}


%\newcommand{\frakq}{\mathfrak{q}}
%\newcommand{\frakw}{\mathfrak{w}}
%\newcommand{\frake}{\mathfrak{e}}
%\newcommand{\frakr}{\mathfrak{r}}
%\newcommand{\frakt}{\mathfrak{t}}
%\newcommand{\fraky}{\mathfrak{y}}
%\newcommand{\fraku}{\mathfrak{u}}
%\newcommand{\fraki}{\mathfrak{i}}
%\newcommand{\frako}{\mathfrak{o}}
%\newcommand{\frakp}{\mathfrak{p}}
%\newcommand{\fraka}{\mathfrak{a}}
%\newcommand{\fraks}{\mathfrak{s}}
%\newcommand{\frakd}{\mathfrak{d}}
%\newcommand{\frakf}{\mathfrak{f}}
%\newcommand{\frakg}{\mathfrak{g}}
%\newcommand{\frakh}{\mathfrak{h}}
%\newcommand{\frakj}{\mathfrak{j}}
%\newcommand{\frakk}{\mathfrak{k}}
%\newcommand{\frakl}{\mathfrak{l}}
%\newcommand{\frakz}{\mathfrak{z}}
%\newcommand{\frakx}{\mathfrak{x}}
%\newcommand{\frakc}{\mathfrak{c}}
%\newcommand{\frakv}{\mathfrak{v}}
%\newcommand{\frakb}{\mathfrak{b}}
%\newcommand{\frakn}{\mathfrak{n}}
%\newcommand{\frakm}{\mathfrak{m}}

%>>>>>>>>>>>>>>>>>>>>>>>>>>>>>>
%<<<  Caligraphic Letters   <<<
%>>>>>>>>>>>>>>>>>>>>>>>>>>>>>>
\newcommand{\calQ}{\mathcal{Q}}
\newcommand{\calW}{\mathcal{W}}
\newcommand{\calE}{\mathcal{E}}
\newcommand{\calR}{\mathcal{R}}
\newcommand{\calT}{\mathcal{T}}
\newcommand{\calY}{\mathcal{Y}}
\newcommand{\calU}{\mathcal{U}}
\newcommand{\calI}{\mathcal{I}}
\newcommand{\calO}{\mathcal{O}}
\newcommand{\calP}{\mathcal{P}}
\newcommand{\calA}{\mathcal{A}}
\newcommand{\calS}{\mathcal{S}}
\newcommand{\calD}{\mathcal{D}}
\newcommand{\calF}{\mathcal{F}}
\newcommand{\calG}{\mathcal{G}}
\newcommand{\calH}{\mathcal{H}}
\newcommand{\calJ}{\mathcal{J}}
\newcommand{\calK}{\mathcal{K}}
\newcommand{\calL}{\mathcal{L}}
\newcommand{\calZ}{\mathcal{Z}}
\newcommand{\calX}{\mathcal{X}}
\newcommand{\calC}{\mathcal{C}}
\newcommand{\calV}{\mathcal{V}}
\newcommand{\calB}{\mathcal{B}}
\newcommand{\calN}{\mathcal{N}}
\newcommand{\calM}{\mathcal{M}}

\newcommand{\calq}{\mathcal{Q}}
\newcommand{\calw}{\mathcal{W}}
\newcommand{\cale}{\mathcal{E}}
\newcommand{\calr}{\mathcal{R}}
\newcommand{\calt}{\mathcal{T}}
\newcommand{\caly}{\mathcal{Y}}
\newcommand{\calu}{\mathcal{U}}
\newcommand{\cali}{\mathcal{I}}
\newcommand{\calo}{\mathcal{O}}
\newcommand{\calp}{\mathcal{P}}
\newcommand{\cala}{\mathcal{A}}
\newcommand{\cals}{\mathcal{S}}
\newcommand{\cald}{\mathcal{D}}
\newcommand{\calf}{\mathcal{F}}
\newcommand{\calg}{\mathcal{G}}
\newcommand{\calh}{\mathcal{H}}
\newcommand{\calj}{\mathcal{J}}
\newcommand{\calk}{\mathcal{K}}
\newcommand{\call}{\mathcal{L}}
\newcommand{\calz}{\mathcal{Z}}
\newcommand{\calx}{\mathcal{X}}
\newcommand{\calc}{\mathcal{C}}
\newcommand{\calv}{\mathcal{V}}
\newcommand{\calb}{\mathcal{B}}
\newcommand{\caln}{\mathcal{N}}
\newcommand{\calm}{\mathcal{M}}
\newcommand{\calmv}{\mathcal{M}_\textup{v}}
\newcommand{\calmh}{\mathcal{M}_\textup{h}}
\newcommand{\calMv}{\mathcal{M}_\textup{v}}
\newcommand{\calMh}{\mathcal{M}_\textup{h}}


\usepackage{framed}
\definecolor{shadecolor}{rgb}{.925,0.925,0.925}
\usepackage[style=numeric,%citestyle=numeric,
url=false,doi=false,isbn=false,eprint=false]{biblatex}%
\hypersetup{colorlinks=false,pdfborder={0 0 0}}



\makeatletter
\renewcommand{\@seccntformat}[1]{\csname the#1\endcsname.\quad}
\makeatother


\theoremstyle{plain}
\newtheorem{theorem}{Theorem}
\newtheorem*{completenesstheorem}{Completeness Theorem}
\newtheorem{twistinglemma}[thm]{Twisting Lemma}
\renewcommand*{\thetheorem}{\Alph{theorem}}
\newtheorem{conjectureAlpha}{Conjecture}
\renewcommand*{\theconjectureAlpha}{\Alph{conjectureAlpha}}

\newcommand{\PMonad}{{\calP^\textup{u}}}
\newcommand{\LambdaMonad}{\Lambda^\textup{u}}
\newcommand{\Palg}{{\calP}}
\newcommand{\deltaalg}{\Delta} %change me
\newcommand{\LieOperad}{{\scrL}}
\newcommand{\CommOperad}{{\scrC}}
\newcommand{\restn}[1]{#1^{[2]}}
\newcommand{\restnwithsubscript}[2]{#1^{[2]}_{#2}}
\newcommand{\restnRepeated}[2]{#1^{[2^{#2}]}}
\newcommand{\dualrestn}[1]{\sqrt[{[2]}]{#1}}
\renewcommand{\dualrestn}[1]{\sqrt[{\!\!\![2]}]{#1}}
\newcommand{\vect}[2]{\calV^{#1}_{#2}}
\newcommand{\BSW}{{\scrG}}
\newcommand{\BSWres}{B^\BSW}%didn't always remember to use
\newcommand{\PiAlg}{\textup{-$\Pi$-Alg}}
\newcommand{\HAlg}{\textup{-$H^*$-Alg}}
\newcommand{\HCoalg}{\textup{-$H_*$-Coalg}}
\newcommand{\quadratic}{\textup{qu}}
\newcommand{\quadgrad}[1]{\textup{q}_{#1}}
\newcommand{\crossterms}{\textup{cr}}
\newcommand{\ExtCohOp}{\textup{Sq}_\textup{ext}}
\newcommand{\vExtCohOp}{\textup{Op}_\textup{v,ext}}
\newcommand{\hExtCohOp}{\textup{Op}_\textup{h,ext}}
\newcommand{\ExtCohProd}{\mu_\textup{ext}}
\newcommand{\epi}{{\,\makebox[0cm][l]{\ensuremath\to}\to}}
\newcommand{\mono}{{\to}}
\newcommand{\minDimP}{\overline{m}}
\newcommand{\minDimDelta}{m}
\newcommand{\minDimSq}{\underline{m}}
\newcommand{\excess}{e}
\newcommand{\produces}[3]{#3:#1\sim #2}
\renewcommand{\produces}[3]{#1\rightarrow_{#3} #2}%{J}{I}{P}
\renewcommand{\produces}[3]{#1\overset{\smash{#3}}{\rightarrow} #2}%{J}{I}{P}
\newcommand{\twist}{\omega}
\newcommand{\DeltaUp}{\Delta}% <-- i changed the indices to upper, marking where I did it.
\newcommand{\Shuffles}[2]{\textup{Sh}_{#1#2}}
\newcommand{\HalfShuffles}[2]{\textup{Sh}_{#1#2}^{\smash{\!\div2}}}
\newcommand{\Nop}{N^{\smash{-}}}
\newcommand{\NOFULLPAGE}{\relax}
\newcommand{\UEA}{U'}%{U^{[2]}}%if this changes, so does some text
\newcommand{\UEAX}{\overline{X}'}%{U^{[2]}}%if this changes, so does some text
\newcommand{\Sq}{\mathrm{Sq}}
\newcommand{\Sqh}{\mathrm{Sq}_\textup{h}}
\newcommand{\Sqv}{\mathrm{Sq}_\textup{v}}
\newcommand{\deltav}{\delta^\textup{v}}
\newcommand{\LieSteen}{\calA}
\newcommand{\aDT}{\textup{adm}_+(\Delta,T)}
\newcommand{\aDTirr}{\textup{adm}_+^\textup{irr}(\Delta,T)}
\newcommand{\Comm}{\calC}
\newcommand{\F}{\mathbb{F}}
\newcommand{\Id}{\textup{id}}
\newcommand{\complexes}{\textup{ch}_+}
%\bibliography{papers}
\bibliography{../../Dropbox/logbook/_LOGBOOK/papers}




\title[Unstable Lie algebras and their cohomology]{Unstable Lie algebras and their cohomology}
%\author[M.\ Donovan]{Michael Donovan}

%\address{Department of Mathematics \\ Massachusetts Institute of Technology}
%\email{mdono@math.mit.edu}



\newcommand{\dupdown}[2]{D_{\smash{#1}}}
\newcommand{\caldup}[1]{\calD_{\smash{#1}}}
\newcommand{\caldupdown}[2]{\calD^{\smash{#1}}_{\smash{#2}}}

\begin{document}
\newcommand{\todo}[2]{\begin{shaded}\begin{itemize}
\setlength{\parindent}{.25in}
\item[{\Large$\smash\diamondsuit$}] #1
\ifblank{#2}{}{\tiny\begin{itemize}
\setlength{\parindent}{.25in}
\item #2
\end{itemize}}
\end{itemize}\end{shaded}
}
\newcommand{\tododone}[2]{}
\newcommand{\todoeasy}[2]{\begin{shaded}\begin{itemize}
\setlength{\parindent}{.25in}
\item[{\Large$\smash\spadesuit$}] #1
\ifblank{#2}{}{\tiny\begin{itemize}
\setlength{\parindent}{.25in}
\item #2
\end{itemize}}
\end{itemize}\end{shaded}
}

\begin{Contents Page}
%\begin{abstract}\end{abstract}
%\maketitle
\tableofcontents
\end{Contents Page}

\begin{Intro}
This paper is concerned with calculating the cohomology $H^{\calW(0)}_*X$ of $X\in \calW(0)$.
\end{Intro}

\begin{Conventions and notation}
\section{\textbf{Background and conventions}}

\subsection{Universal algebras}
[\textbf{repetitive}]: In this paper we will be dealing with various categories of universal algebras \cite[\S2.1]{Blanc_Stover-Groth_SS.pdf} monadic over $\vect{}{}$, a category either of graded or ungraded $\F_2$-vector spaces.
As such, it will be helpful to set notation globally. %[Until \S\ref{The example of simplicial commutative F2-algebras} [\textbf{current?}], our comments will apply to any of the algebraic categories $\calL(n)$, $\calU(n)$ or $\calW(n)$ to be defined below. Let $\calC$ be any of these categories, monadic over $\vect{+}{r}$.]
Until \S????, let $\calC$ be such a category, monadic over $\vect{}{}$. We will make some comments that hold in complete generality, and some that hold for all of the examples that we have in mind.

In our examples, an object $X$ of $\calC$ may be defined as an object $X$ of $\vect{}{}$ equipped with certain binary and unary operations on $X$ (and without any nullary operations). As such, there will be a forgetful functor $U_\calC:\calC\to\vect{}{}$, which will admit a left adjoint, always denoted $F_\calC:\vect{}{}\to\calC$ for notational consistency. The functor $U_{\calC}$ will always be monadic, in the sense that the natural comparison functor from $\calC$ to the category of algebras over the monad $U_{\calC}F_{\calC}$ on $\vect{}{}$ is an equivalence.

In our examples, the monad $U_{\calC}F_{\calC}$ will admit an augmentation (of monads) $\epsilon:U_{\calC}F_{\calC}\to\Id$, reflecting homogeneity in the relations defining $\calC$. This augmentation has the monad unit $\Id\to U_{\calC}F_{\calC}$ as a section, and may be thought of as projection back onto generators.


We will generally omit the functor $U_{\calC}$ from our notation, writing $F_{\calC}$ as shorthand for either the monad $U_{\calC}F_{\calC}$ on $\vect{}{}$ or the comonad $F_{\calC}U_{\calC}$ on $\calC$. We will refer to elements of a free construction $F_\calC U$ using notation such as $f(u_i)$, thought of as a composite $f$ of $\calC$-structure maps applied to generators $u_i\in U\subseteq F_\calC(V)$. We will say that $f(v_i)$ is a \emph{$\calC$-expression}. In this language, the linear maps % for $V\in \vect{}{}$, we have linear maps
\[F_{\calc}F_{\calc}V\overset{\mu}{\to} F_{\calc}V,\ \ V\overset{\eta}{\to}F_{\calc}V\ \ \textup{and}\ \ F_{\calc}V\overset{\epsilon}{\to}V, \]
constituting the augmented monad $F_{\calc}$ on $\vect{}{}$ may be decribed as follows: $\mu$ collapses a $\calc$-expression  in $\calc$-expressions into a single $\calc$-expression; $\eta$ sends a vector $v$ to the unit $\calc$-expression $v$; and $\epsilon$ projects a $\calc$-expression onto those summands to which non operations have been applied.
For $X\in \calc$, the comonad structure maps in $\calc$,
\[F_{\calc}F_{\calc}X\overset{\Delta}{\from} F_{\calc}X\ \ \textup{and}\ \ X\overset{\rho}{\from}F_{\calc}X,\]
are as follows: on an expression $f(x_i)$, $\Delta$ returns the same expression $f(x_i)$ in which the $x_i\in X$ are viewed as elements of $F_{\calc}X$, and $\rho$ is the evaluation map equivalent to the $\calc$-structure on $X$.

\subsection{The functor $Q^\calc$ of indecomposables}
Using the augmentation $\epsilon:F_\calc\to\Id$ of monads on $\vect{+}{r}$, any $V\in\vect{+}{r}$ becomes an $F_{\calc}$-module, i.e.\ an object of $\calc$. We denote this functor $K_\calC:\vect{+}{r}\to \calC$; it sends $V\in\vect{+}{r}$ to \emph{the trivial object on $V$}, a phrase that is meaningful  in each of our examples. In each of our examples,  $K_{\calC}$ has a left adjoint, $Q^{\calC}:\calC\to\vect{+}{r}$, which sends $X\in\calC$ to \emph{the quotient of $X$ by the image of its non-trivial operations}. %More precisely, $K_\calC$ sends any object $V\in\vect{+}{r}$ to the $U_{\calC}F_{\calC}$-algebra whose structure map is $\epsilon_V:U_{\calC}F_{\calC}V\to V$. 
%The functor $Q^{\calC}$ sends $X\in \calC$ with structure map $\rho:F_\calC U_\calC X\to X$ to the coequalizer in $\vect{+}{r}$ of the maps $U_\calC\rho$ and $\epsilon_{U_\calC X}$. \emph{or, less correctly}
The functor $Q^{\calC}$ sends $X\in \calC$ to the coequalizer in $\vect{+}{r}$ of $\rho,\epsilon:F_{\calc}X\to X$.

\subsection{Quillen's model structure on $s\calC$ and the bar construction}\label{ssec: quillen model and bar construction}
For any of the algebraic categories $\calC$ herein we use Quillen's simplicial model category structure on the category $s\calC$. In this structure, the weak equivalences (fibrations) are the maps which are weak equivalences (fibrations) of simplicial abelian groups, so that every object is fibrant. 

 A simplicial object $X$ is \emph{almost free} if there are subspaces $V_n\subseteq X_n$ for each $n\geq0$ such that the structure map $F_{\calC}U_\calC V_n\to X_n$ is an isomorphism for all $n$, and such that these subspaces are preserved by all of the degeneracies and face maps of $X$ except for $d_0$. Every cofibrant object is a retract of an almost free object \cite[\S3]{MillerSullivanConjecture.pdf}.

For a simplicial resolution functor $\calC\to s\calC$, we'll often use the standard simplicial bar construction arising from the $F_\calC\dashv U_\calC$ adjunction \cite{BlumRiehlResolutions.pdf}. More explicitly, given $X\in\calC$, we define $B^\calC X\in s\calC$ by iterated application of the comonad $F_\calC$ to $X\in \calC$:
\[B_s^\calC X=(F_\calC)^{s+1}X.\]
The face maps are given by the formula $d_i=(F_\calC)^i\rho$, and the degeneracies by $s_i=(F_\calC)^i\Delta$. %, where $\rho$ and $\Delta$ are respectively the counit and diagonal of the comonad $F_\calC U_\calC$. 
This object is almost free, with $B_s^\calC X$ generated by its subspace $V_s=(F_{\calC})^sX$. It is standard that the augmentation $B^\calC X\to X$ is an acyclic fibration.

The construction generalizes to yield a cofibrant replacement functor $B^{\calc}$ on $s\calc$: one applies  $B^{\calc}:\calc\to s\calc$ levelwise, to obtain a bisimplicial object, and then takes its diagonal.

\subsection{The $\calC$-homology and $\calC$-cohomology functors $H^{\calC}_*$ and $H_{\calC}^*$}

It need not be true that the $Q^\calC\dashv K_\calC$ adjunction models the abelianization functor of \cite[\S II.5]{QuillenHomAlg.pdf}, for example, this does not occur when $\calc$ is the category of restricted Lie algebras. Nevertheless, in this text we will always define the $\calC$-\emph{homology} of $X\in\calC$ by the formula:
\[H_*^{\calC}X=\pi_*(Q^\calC B^\calC X)=H_*N_*(Q^\calC B^\calC X).\]
This is well defined, as the $Q^\calC\dashv K_\calC$ adjunction is a Quillen adjunction (that $K_\calC$ preserves fibrations and acyclic fibrations is immediate), and indeed we are free to use any cofibrant replacement in place of $B^\calC X$.

As $\calC$ is monadic over $\vect{+}{r}$, we may view the groups $H_*^{\calC}X$ together as an object of $\vect{+}{r+1}$. That is, each homology group $H_s^\calC X$ retains the gradings of $X$, and a new homological grading is added (to the left of the existing homological gradings). We will often avoid substituting into the location of the $*$, for the sake of clarity, writing expressions such as $(H_*^\calC X)_{s_{r+1},\ldots,s_1}^t$ in place of $(H_{s_{r+1}}^\calC X)_{s_r,\ldots,s_1}^t$.

We define the \emph{$\calC$-cohomology} $H^*_\calC X$ of $X$ to be the linear dual of $H_*^\calC X$, or equivalently the cohomotopy groups $\pi^*(Q^\calC X)^*$ of the dual cosimplicial object.  If we dualize to obtain cohomology, the cohomological gradings and homological gradings are swapped, and $H^*_{\calC}X$ may be viewed as an object of $\vect{r+1}{+}$. 
\begin{lem}\label{lemma on homology class repd by normalized generator}
Suppose that $X$ is almost free with generating subspaces $V_n\subseteq X_n$. Then any homology class in $H_n^{\calC}X\cong \pi_nQ^{\calC}X$ can be represented by the image in $Q^{\calC}X_n$ of an element of $V_n\cap N_nX$.
\end{lem}
\begin{proof}
Choose a representative $w\in ZN_nQ^\calC X$. As the composite $V_n\subseteq X_n\epi Q^\calC X_n$ is an isomorphism, one can find some $v\in V_n$ such that the image of $v$ in $Q^\calC X_n$ is $w$, yet $v$ needn't lie in $N_nX$. Consider the endomorphism
\[c=(1+s_0d_1)(1+s_1d_2)\cdots (1+s_{n-1}d_n):Q^\calC X_n\to Q^\calC X_n.\]
By the simplicial identities, $c$ has image inside $N_n Q^\calC X$, and therefore acts as the identity on $N_nQ^\calC X$. The similarly defined operator on $X_n$ has image inside $N_n X$, and, in light of the fact that all the simplicial operators but $d_0$ preserve the subspaces $V_i$, preserves $V_n$. Thus the image in $Q^\calC X_n$ of $cv\in V_n\cap N_n X$ is $cw=w$.
\end{proof}
We will sometimes wish to work directly with the generating subspaces when working with (co)-homology, and will then apply the following easy observation:
\begin{lem}\label{identify almost free indecs with gens}
Suppose that $X$ is almost free with generating subspaces $V_n\subseteq X_n$. Then the simplicial object $(Q^{\calc}X)_{n}$ may be identified with the collection of vector spaces $\{V_n\}$, using  the composite
\[V_{n}\overset{d_0}{\to}X_n\cong F^{\calc}V_{n-1}\overset{\epsilon}{\to}V_{n-1}\]
as the zeroth face map of $\{V_n\}$, and using the other structure maps of $X$, which preserve the $V_{n}$, as the other structure maps of $\{V_n\}$.
\end{lem}



\subsection{The trace map and quadratic refinements}
For any vector space $V\in \vect{}{}$, the tensor power $V^{\otimes2}:=V\otimes V$ has an action of $\Sigma_2$ given by the map $T$ interchanging the two factors. We'll write $S_2V$ for the coinvariants and $S^2V$ for the invariants of this action, and $\Lambda^2V$ for the image in $S^2V$ of the \emph{trace map} $\trace=(1+T):S_2V\to S^2V$. Thus, we have written $\Lambda_2V$ as a subobject of $S^2V$.

On the other hand, one may view $\Lambda_2V$ as the quotient $S_2V/\ker(\trace)$, that is, the quotient of $S_2V$ by the subspace generated by elements of the form $v\otimes v$. It is often convenient to define maps out of $\Lambda_2V$ by viewing it as a subobject of $S^2V$, and maps into $\Lambda_2V$ by viewing it as a quotient of $S_2V$ (itself a quotient of $V^{\otimes2}$).

Suppose that $V$ and $W$ are $\F_2$-vector spaces, and $p:S_2V\to W$ is a linear map. A \emph{quadratic refinement of $p$} is a function $\sigma:V\to W$ satisfying, for $v_1,v_2\in V$ and $\alpha\in\F_2$:
\[\sigma(v_1+v_2)=\sigma(v_1)+\sigma(v_2)+p(v_1\otimes v_2)\textup{ and }\sigma(\alpha v_1)=\alpha^2\sigma(v_1).\]
In fact, the second condition is redundant (over $\F_2$), and these conditions are equivalent to the following condition. For any set $B$, define $\Lambda^2B$ to be the set of subsets of $B$ of cardinality exactly two. 
The equivalent condition is that, for every collection of vectors $v_b\in V$ and of coefficients $\alpha_b\in \F_2$ indexed by a set $B$, in which all but finitely many of the $\alpha_b$ are zero, the equation
\[\sigma\Bigl(\sum_{b\in B}\alpha_bv_b\Bigr)=\sum_{b\in B}\alpha_b^2\sigma(v_b)+\sum_{\{b,c\}\in \Lambda^2B}\alpha_b\alpha_cp(v_b\otimes v_c)\]
holds.
%\[v\otimes v
%=
%\sum_{b\in B}\alpha_b^2v_b+\sum_{\{b,c\}\in \Lambda^2B}\alpha_b\alpha_c\trace(v_b\otimes v_c)\]
%
%
%
%
%\begin{alignat*}{2}
%v\otimes v
%&=
%\sum_{b\in B}\alpha_b^2v_b+\sum_{\{b,c\}\in \Lambda^2B}\alpha_b\alpha_c\trace(v_b\otimes v_c)%
%\\
%f(v\otimes v)&:=
%\sum_{b\in B}\alpha_b^2\sigma(v_b)+\sum_{\{b,c\}\in \Lambda^2B}\alpha_b\alpha_cp(v_b\otimes v_c)%
%\\
%% Left hand side
%% Relation
%&=
%% Right hand side
%\textstyle\sum_{i<j}\alpha_i\alpha_jp(v_i\otimes v_j)+\textstyle\sum_{i}\alpha_i^2\sigma(v_i)%
%►% Comment
%&\qquad&\text{(ᾮ)}©
%\end{alignat*}
%
If $f:S^2V\to W$ is a linear map, the function
$v\mapsto f(v\otimes v)$
is a quadratic refinement of $\trace\circ f$, and indeed:
\begin{prop}\label{propOnExtendingToInvariants}
For any linear map $p:S_2V\to W$, extensions of $p$ to a linear map $f:S^2V\to W$ are in natural bijection  with quadratic refinements of $p$.
\end{prop}
\begin{proof}
%It is enough to prove this result for finite-dimensional vector spaces $V$, as any vector space $V$ is the filtered colimit of its finite-dimensional subspaces, and the functors $S_2$ and $S^2$ commute with filtered colimits.
Suppose that $V$ has basis $\{v_b\ |\ b\in B\}$. Then $S^2V$ has basis the set 
\[\{\trace(v_b\otimes v_c)\ |\ \{b,c\}\in\Lambda^2B\}\cup\{v_b\otimes v_b\ |\ b\in B\}.\]
This shows that an extension $f$ of $p$ is determined by the quadratic refinement $v\mapsto f(v\otimes v)$. Thus, as long as we can produce an extension $f$ with $\sigma(v)=f(v\otimes v)$ for any quadratic refinement $\sigma$ of $p$, we will have a construction which is natural in finite-dimensional $V$.

What remains to prove is that the linear map $f$ defined on this basis by
\[\trace(v_b\otimes v_c)\mapsto p(v_b\otimes v_c),\quad v_b\otimes v_b\mapsto \sigma(v_b)\]
does in fact have the property that $f(v\otimes v)=\sigma(v)$ for \textup{all} $v\in V$. Indeed, if we write $v$ in terms of the chosen basis as $v=\sum_{b\in B}\alpha_bv_b$, then
\[v\otimes v
=
\sum_{b\in B}\alpha_b^2v_b+\sum_{\{b,c\}\in \Lambda^2B}\alpha_b\alpha_c\trace(v_b\otimes v_c),\]
and we can apply our definition of the linear map $f$ to this expansion directly, obtaining
\[f(v\otimes v):=
\sum_{b\in B}\alpha_b^2\sigma(v_b)+\sum_{\{b,c\}\in \Lambda^2B}\alpha_b\alpha_cp(v_b\otimes v_c)=\sigma(v).\qedhere\]
\end{proof}
\subsection{Lie algebras}
As we work in characteristic 2, there is more than one available notion of a Lie algebra. An \emph{$S(\LieOperad)$-algebra} is a vector space $L$ equipped with a bracket $L\otimes L\to L$ satisfying the Jacobi identity and the (anti)-symmetry condition $[x,y]=[y,x]$. A \emph{Lie algebra} (or $\Lambda(\LieOperad)$-algebra) is a vector space $L$ equipped with a bracket $L\otimes L\to L$ satisfying the Jacobi identity and the alternating condition $[x,x]=0$. Finally, a \emph{restricted Lie algebra} \cite{CurtisSimplicialHtpy.pdf,6Author.pdf} (or $\Gamma(\LieOperad)$-algebra) is a Lie algebra equipped with a \emph{squaring} or \emph{restriction} function $\restn{(\DASH)}:L\to L$, satisfying the axioms
\[\restn{(x_1+x_2)}=\restn{x_1}+\restn{x_2}+[x_1,x_2]\textup{ \ and \ }[\restn{x_1},x_2]=[x_1,[x_1,x_2]].\]
The alternating condition implies the (anti)-symmetry condition, and these three types of Lie algebras form a hierarchy: a restricted Lie algebra is in particular a Lie algebra, and a Lie algebra is in particular an $S(\LieOperad)$-algebra.

Fresse \cite{FresseSimplicialAlgs.pdf} explains how to construct the monads $S(\LieOperad)$, $\Lambda(\LieOperad)$ and $\Gamma(\LieOperad)$ on $\vect{}{}$ which give rise to these structures, starting with the Lie operad $\LieOperad$. For $V\in\vect{}{}$, it is standard that the functor
\[S(\LieOperad):V\mapsto \bigoplus_{n\geq1}(\LieOperad(n)\otimes V^{\otimes n})_{\Sigma_n}\]
inherits the structure of a monad from the composition maps of $\LieOperad$. Fresse observes that the functor
\[\Gamma(\LieOperad):V\mapsto \bigoplus_{n\geq1}(\LieOperad(n)\otimes V^{\otimes n})^{\Sigma_n}\]
may also be equipped with a monad structure, such that the trace map $S(\LieOperad)\to \Gamma(\LieOperad)$
%\[\trace:S(\LieOperad)(V)\to \Gamma(\LieOperad)(V)\]
is a map of monads, and an intermediate monad may be defined by
\[\Lambda(\LieOperad):V\mapsto\im\bigl(\trace:S(\LieOperad)(V)\to \Gamma(\LieOperad)(V)\bigr).\]
These monads give rise to the three indicated forms of Lie algebras in characteristic 2. Now each of these functors supports an extra grading, the \emph{quadratic grading}, so named because \textbf{why?}. For any $k\geq1$, we'll say that $\Gamma(\LieOperad)V$ has quadratic grading $k$ part
\[\quadgrad{k}(\Gamma(\LieOperad)V):=(\LieOperad(k)\otimes V^{\otimes k})^{\Sigma_k},\]
and similarly for the other two monads.

 Now $\LieOperad(2)\cong\F_2$, so that
\[\quadgrad{2}(S(\LieOperad)V)\cong S_2V,\quad \quadgrad{2}(\Lambda(\LieOperad)V)\cong \Lambda^2V,\quad\textup{and}\quad\quadgrad{2}(\Gamma(\LieOperad)V)\cong S^2V.\]
%Each of these three monads is generated by a single binary operation, contained in quadratic grading 2. 
Moreover, one can view an $S(\LieOperad)$-algebra as a map $S_2L\to L$, a $\Lambda(\LieOperad)$-algebra as a map $\Lambda^2L\to L$, and a $\Gamma(\LieOperad)$-algebra as a map $S^2L\to L$, where in each case we demand the necessary compatibilities that these maps extend to the full monad. %Here, we used the analogous definitions $S_2V:=(V^{\otimes2})_{\Sigma_2}$, $S^2V:=(V^{\otimes2})_{\Sigma_2}$ and $\Lambda^2V:=\im(\trace:S_2V\to S^2V)$.
By pulling back along the natural maps
\[S_2V\to \Lambda_2 V\to S^2V\]
one can demote a restricted Lie algebra to a Lie algebra, or a Lie algebra to an $S(\LieOperad)$-algebra.

A \emph{restictable ideal} in a Lie algebra $L$ is a Lie ideal $I$ of $L$, equipped with a restriction function $\restn{(\DASH)}:I\to I$, satisfying the following axioms, for $x_1,x_2\in I$ and $x_3\in L$:
\[\restn{(x_1+x_2)}=\restn{x_1}+\restn{x_2}+[x_1,x_2]\textup{ \ and \ }[\restn{x_1},x_3]=[x_1,[x_1,x_3]].\]
In fact, let $\mathsf{PRL}$ denote the category whose objects are pairs of vector spaces $(L_+,L_0)$, equipped with a Lie algebra structure on $L_0\oplus L_+$ in which $L_+$ is a restrictable ideal, and whose maps are Lie algebra maps preserving the decomposition and commuting with the partial restrictions. This category is monadic over $\vect{}{}\times\vect{}{}$, the category of pairs of vector spaces, and the value of monad $F_{\mathsf{PRL}}$  on $(V_+,V_0)$ is just an appropriately chosen subalgebra of $\Gamma(\LieOperad)(V_+\oplus V_0)$.


\subsection{Non-unital commutative algebras}
In this paper we will always work with \emph{non-unital} commutative algebras, unless we specify otherwise. As for Lie algebras, there are three different notions of non-unital commutative algebra available in characteristic 2. A \emph{commutative algebra} (or $S(\CommOperad)$-algebra) is a vector space $A$ equipped with an associative commutative pairing $A\otimes A\to A$. An \emph{exterior algebra} (or $\Lambda(\CommOperad)$-algebra) is a commutative algebra $A$ with the property that $x^2=0$ for all $x\in A$. A \emph{divided power algebra} (or $\Gamma(\CommOperad)$-algebra) is a commutative algebra $A$ equipped with \emph{divided power} operations, as described in \cite[1.2.2]{FresseSimplicialAlgs.pdf} or \cite[\S2]{MR1089001}. In characteristic 2, these operations are all determined by a single operation, the \emph{divided square} $\gamma_2:A\to A$, which satisfies
\[\gamma_2(xy)=x^2\gamma_2(y),\ \gamma_2(\lambda x)=\lambda^2\gamma_2(x)\textup{ and }\gamma_2(x+y)=\gamma_2(x)+\gamma_2(y)+xy.\]
Note that the last condition implies that a divided power algebra is exterior. Thus, $\gamma_2(xy)=x^2\gamma_2(y)=0$.

There is a notion of a \emph{divided power ideal} of a commutative algebra: an ideal $I$ of a commutative algebra $A$ equipped with a compatible divided power structure on $I$. In this case, $I$ is necessarily exterior, although for $x\in A$ and $y\in L$, $\gamma_2(xy)=x^2\gamma_2(y)$ need not be zero.

Again, Fresse \cite{FresseSimplicialAlgs.pdf} explains how to construct the monads $S(\CommOperad)$, $\Lambda(\CommOperad)$ and $\Gamma(\CommOperad)$ on $\vect{}{}$ which give rise to these structures, using the commutative operad $\CommOperad$ instead of $\LieOperad$. Again, there is a quadratic grading definable on these three monads, and each monad is generated in degre 2, so that a commutative algebra may be thought of as a  map $S_2L\to L$, an exterior algebra as a map $\Lambda^2L\to L$, and a divided power algebra as a map $S^2L\to L$.








\subsection{Categories of $\F_2$-vector spaces}
[Repetitive:] In this paper we will deal with numerous categories monadic over a category of graded $\F_2$-vector spaces. We will write $\vect{}{}$ for a generic category of $\F_2$-vector spaces, either graded or ungraded. We will write $\vect{+}{r}$ for the category of vector spaces with $r$ non-negative homological gradings and a single strictly positive cohomological grading, so that an object $V$ of $\vect{+}{r}$ decomposes as
\[V=\bigoplus_{s_r,\ldots,s_1\geq0,\,t\geq 1}V^{t}_{s_r,\ldots,s_1}.\]
Similarly, we write $\vect{r}{+}$ for the category of $\F_2$-vector spaces with one positive homological grading and $r$ non-negative cohomological gradings. We follow the convention that (graded) dualization $V\mapsto V^*$ interchanges homological and cohomological gradings, and is thus a contravariant functor $\vect{+}{r}\to\vect{r}{+}$ defined by $(V^*)^{s_r,\ldots,s_1}_{t}:=(V_{s_r,\ldots,s_1}^{t})^*$.

The category $\vect{+}{r}$ is equipped with a tensor product:
\[(U\otimes V)^t_{s_r,\ldots,s_1}=\bigoplus_{s'_i+s''_i=s_i,\,t'+t''=t}U^{t'}_{s'_r,\ldots,s'_1}\otimes V^{t''}_{s''_r,\ldots,s''_1}.\]
\textbf{Comment} here on degree shifts.
%and if $V\in \vect{+}{r}$, then $V\otimes V$ has an action of $\Sigma_2$ given by the map $T$ interchanging the two factors. We'll write $S_2V$ for the coinvariants and $S^2V$ for the invariants of this action, and $\Lambda^2V$ for the image in $S^2V$ of the trace map $\trace=(1+T):S_2V\to S^2V$:\[S_2V\to \Lambda_2 V\to S^2V.\]

\subsection{Simplicial vector spaces}
In this paper we will use the following four chain complexes in $\complexes \vect{+}{r}$ associated with a simplicial graded vector-space $V\in s\vect{+}{r}$:
\begin{alignat*}{2}
C_nX&:=X_n&& \textup{ with differential }d=\textstyle\sum_{i=0}^{n}d_i;\\
N_nX&:=\textstyle\bigcap_{0< i\leq n}\ker(d_i:X_n\to X_{n-1})&& \textup{ with differential }d=d_0;\\
\Nop_nX&:=\textstyle\bigcap_{0\leq i< n}\ker(d_i:X_n\to X_{n-1})&& \textup{ with differential }d=d_n;\\
N_n^\div X&:=X_n/(\textup{degenerate $n$-simplices})&& \textup{ with differential }d=\textstyle\sum_{i=0}^{n}d_i.
\end{alignat*}
There are evident inclusions of $N_*X$ and $\Nop_*X$ into $C_*X$, and a projection of $N_*X$ onto $N_*^\div X$, and all of these maps are weak equivalences. Moreover, the composite $N_*X\to N_*^\div X$ is an isomorphism (as is the composite from $\Nop_*X$). Each $N_nX$ retains the internal gradings of $X$, and the functor $N_*$ appears in the Dold-Kan correspondence \cite[\S III.2]{goerss-jardine.pdf}, an adjoint equivalence of categories:
\[N:s\vect{+}{r}\rightleftarrows \complexes \vect{+}{r}:\Gamma,\]
under which the homotopy groups of $X\in s\vect{+}{r}$ (as a simplicial set)  are isomorphic to the homology groups of $N_*X$.
%
%As such, we may define the homotopy groups of $X$, written $\pi_*X\in \vect{+}{r+1}$, to be $H_*N_*X$, where $N_*$ is one of the inverse equivalences appearing in the Dold-Kan correspondence [somewhere]: \[N:s\vect{+}{r}\rightleftarrows \complexes \vect{+}{r}:\Gamma.\]
%There are three equivalent definitions of $N_*X$. We will almost always use the description of $N_nX$ as the mutual kernel of $d_1,\ldots,d_n$ in $X_n$, with differential $d_0$ (rather than the equivalent definition as a quotient of the unnormalized complex $C_nX=X_n$), and write $ZN_nX$ for the mutual kernel of $d_0,\ldots,d_n$. 
A cycle in $N_nX$ is an element $x\in X_n$ such that $d_ix=0$ for $0\leq i\leq n$. We write $ZN_nX$ for this group of cycles, and sometimes write $\overline{x}$ for the equivalence class of $x$ in $\pi_*X$.

It will be helpful to end the notational distinction between the chain complex dimension $n$ and the other homological dimensions $s_r,\ldots,s_1$. That is, we will view $\pi_*V$ as a single object of $\vect{+}{r+1}$, defined by
\[(\pi_*V)^t_{s_{r+1},\ldots,s_1}:=(\pi_{s_{r+1}}V)^t_{s_{r},\ldots,s_1}.\]
Now for any collection of indices $s_{r+1},\ldots,s_1\geq0$ and $t\geq1$, define:
%Let $\mathbb{S}_{n,s_r,\ldots,s_1}^t$ and $C\mathbb{S}_{n,s_r,\ldots,s_1}^t$ be the $\in \complexes \vect{+}{r}$ be the chain complexes, with $z$, $h$ and $dh$ all having internal gradings $t,s_r,\ldots,s_1$:
%%%%%%\begin{gather*}
%%%%%%\phantom{C\mathbb{S}_{n,s_r,\ldots,s_1}^t=\Gamma\Bigl(}\xymatrix@R=1.5mm@1@!C{
%%%%%%&{\smash{(n-1)}}&{\smash{(n)}}&{\smash{(n+1)}}&{\smash{(n+2)}}\\
%%%%%%{\makebox[0cm][r]{$\,\mathbb{S}_{n,s_r,\ldots,s_1}^t=\Gamma\Bigl($}\cdots\,} &%r1c1
%%%%%%{\,0\,}\ar[l]
%%%%%%&%r1c2
%%%%%%\F_2\langle z\rangle\ar[l]
%%%%%%&%r1c3
%%%%%%{\,0\,}\ar[l]
%%%%%%&{\,0\,}\ar[l]
%%%%%%&{\,\cdots\Bigr),} \ar[l]
%%%%%%\\
%%%%%%%&{\smash{(n-1)}}&{\smash{(n)}} &{\smash{(n+1)}}&{\smash{(n+2)}}\\
%%%%%%{\makebox[0cm][r]{$\,C\mathbb{S}_{n,s_r,\ldots,s_1}^t=\Gamma\Bigl($}\cdots\,} &%r1c1
%%%%%%{\,0\,}\ar[l]
%%%%%%&%r1c2
%%%%%%\F_2\langle dh\rangle\ar[l]
%%%%%%&%r1c3
%%%%%%\F_2\langle h\rangle\ar[l]
%%%%%%&{\,0\,}\ar[l]
%%%%%%&{\,\cdots\Bigr).} \ar[l]
%%%%%%}
%%%%%%\end{gather*}
\begin{gather*}
\phantom{C\mathbb{S}_{n,s_r,\ldots,s_1}^t=\Gamma\Bigl(}\xymatrix@R=0mm@1@!C{
{\makebox[0cm][r]{$\,\mathbb{S}_{n,s_r,\ldots,s_1}^t=\Gamma\Bigl($}\cdots\,} &%r1c1
{\,0\,}\ar[l]
&%r1c2
\F_2\langle z\rangle\ar[l]
&%r1c3
{\,0\,}\ar[l]
&{\,0\,}\ar[l]
&{\,\cdots\Bigr),} \ar[l]
\\
&\raisebox{.4mm}{\tiny$\smash{(s_{r+1}\!-\!1)}$}
&\raisebox{.4mm}{\tiny$\smash{(s_{r+1}  )}$}
&\raisebox{.4mm}{\tiny$\smash{(s_{r+1}\!+\!1)}$}
&\raisebox{.4mm}{\tiny$\smash{(s_{r+1}\!+\!2)}$}\\
%&{\smash{(n)}}&{\smash{(n+1)}}&{\smash{(n+2)}}\\
%&{\smash{(n-1)}}&{\smash{(n)}} &{\smash{(n+1)}}&{\smash{(n+2)}}\\
{\makebox[0cm][r]{$\,C\mathbb{S}_{n,s_r,\ldots,s_1}^t=\Gamma\Bigl($}\cdots\,} &%r1c1
{\,0\,}\ar[l]
&%r1c2
\F_2\langle dh\rangle\ar[l]
&%r1c3
\F_2\langle h\rangle\ar[l]
&{\,0\,}\ar[l]
&{\,\cdots\Bigr).} \ar[l]
}
\end{gather*}
Here $z$ and $h$ denote are both to lie in internal cohomological grading $t$ and homological gradings $s_r,\ldots,s_1$.
There is an evident inclusion $\imath:\mathbb{S}_{s_{r+1},s_r,\ldots,s_1}^t\to C\mathbb{S}_{s_{r+1},s_r,\ldots,s_1}^t$. For any $V\in s\vect{+}{r}$, we can identify the subspaces of cycles and boundaries with hom-sets:
\begin{alignat*}{2}
\hom_{s\vect{+}{r}}(\mathbb{S}^t_{s_{r+1},\ldots,s_1},X)&\cong (ZN_{s_{r+1}}X)_{s_r,\ldots,s_1}^t\makebox[0cm][l]{ and}\\
\hom_{s\vect{+}{r}}(C\mathbb{S}^t_{s_{r+1},\ldots,s_1},X)&\cong (N_{s_{r+1}+1}X)_{s_r,\ldots,s_1}^t.
\end{alignat*}
%\[\hom_{s\vect{+}{r}}(\mathbb{S}^t_{s_r,\ldots,s_1},V)\cong (ZN_{s_{r+1}}V)_{s_r,\ldots,s_1}^t\textup{ and }\hom_{s\vect{+}{r}}(C\mathbb{S}^t_{s_{r+1},\ldots,s_1},V)\cong (N_{s_{r+1}+1}V)_{s_r,\ldots,s_1}^t,\]
Under these isomorphisms the differential $N_{s_{r+1}+1}V\to ZN_{s_{r+1}}V$ corresponds to $\imath^*$.

\subsection{Dold's theorem}
According to Dold [], see also [\url{http://www3.amherst.edu/~mching/Work/homotopy_operations.pdf}, lemma 3.1], 
\begin{thm}[Dold's theorem]
Suppose that $F:s\vect{+}{r}\to s\vect{+}{r}$ is a functor preserving weak equivalences, for example, the prolongation of an endofunctor of $\vect{+}{r}$. Then there is a functor $\calF:\vect{+}{r+1}\to\vect{+}{r+1}$ such that the following diagram commutes:
\[\xymatrix@R=4mm{
s\vect{+}{r}\ar[r]^-{F}
\ar[d]^-{\pi}
&%r1c1
s\vect{+}{r}\ar[d]^-{\pi}
\\%r1c2
\vect{+}{r+1}\ar[r]^-{\calF}
&
\vect{+}{r+1}
}\]
Moreover, if $F\cong F_2\circ F_1$, then $\calF\cong \calF_2\circ \calF_1$.
\end{thm}
\noindent The idea here is that the functor $\pi$ induces an equivalence between the homotopy category of $s\vect{+}{r}$ and $\vect{+}{r+1}$. In fact, the inverse equivalence can be lifted to a functor into $s\vect{+}{r}$, namely 
\[V\mapsto \Gamma V,\ \ \vect{+}{r+1}\to s\vect{+}{r},\]
where we view $V$ as a trivial chain complex. Then $\calF$ can be constructed as $\calF V:=\pi_*(F\Gamma V)$.
%given that the homotopy category of simplicial vector spaces is graded vector spaces, the derived functors of a weak equivalence preserving functor $F:s\calV\to s\calV$ are well defined, and there's a commuting diagram

\end{Conventions and notation}


\begin{CPiAlgs and CHalgs}

\section{\textbf{Homotopy operations and cohomology operations}}
Let $\calC$ be a category of universal graded algebras, monadic over $\vect{+}{r}$. Our goal is to understand construct operations on the homotopy and cohomology of an object of $s\calC$. In \S\ref{homotopy and pialgs} and \S\ref{cohomology and Halgs}, we set out dual frameworks in which these operations can be organised, and in \ref{quadratic part section} and \ref{subseq:The smash coproduct}, we will describe some useful chain level operations that we will use to construct cohomology operations in \S\ref{sec:Constructing cohomology operations}.


\subsection{Homotopy groups and $\calC$-$\Pi$-algebras}\label{homotopy and pialgs}
%The forgetful functor $U_\calC:\calc\to \vect{+}{r}$ used to define the weak equivalences and fibrations in $s\calc$, allows us to consider any object $X\in s\calC$ as an object of $s\vect{+}{r}$. In particular, for any $X\in s\calc$ we may define homotopy groups
Using the forgetful functor $U_\calC:\calc\to \vect{+}{r}$, for any $X\in s\calc$ we may define the homotopy groups of $X$,
%\[(\pi_*X)_{s_{r+1},\ldots,s_1}^{t}:=\pi_{s_{r+1}}(X^t_{s_r,\ldots,s_1}),\]
which we view together as an object $\pi_*X$ of $\vect{+}{r+1}$. By the definition of the model structure, $\pi_*$ is a homotopical functor.
For any $X\in s\calC$, by adjunction:
%\[\hom_{s\calC}(F_\calC\mathbb{S}^t_{n,s_r,\ldots,s_1},X)\cong (ZN_nX)_{s_r,\ldots,s_1}^t\textup{ and }\hom_{s\calC}(F_{\calC}C\mathbb{S}^t_{n,s_r,\ldots,s_1},X)\cong (N_{n+1}X)_{s_r,\ldots,s_1}^t,\]
%\begin{gather*}
%\hom_{s\calC}(F_\calC\mathbb{S}^t_{s_{r+1},\ldots,s_1},X)\cong (ZN_{s_{r+1}}X)_{s_r,\ldots,s_1}^t\makebox[0cm][l]{ and}\\
%\hom_{s\calC}(F_{\calC}C\mathbb{S}^t_{s_{r+1},\ldots,s_1},X)\cong (N_{s_{r+1}+1}X)_{s_r,\ldots,s_1}^t,
%\end{gather*}
\begin{alignat*}{2}
\hom_{s\calC}(F_\calC\mathbb{S}^t_{s_{r+1},\ldots,s_1},X)&\cong (ZN_{s_{r+1}}X)_{s_r,\ldots,s_1}^t\makebox[0cm][l]{ and}\\
\hom_{s\calC}(F_{\calC}C\mathbb{S}^t_{s_{r+1},\ldots,s_1},X)&\cong (N_{s_{r+1}+1}X)_{s_r,\ldots,s_1}^t,
\end{alignat*}
and indeed $i^*$ plays the same role as above, representing the differential of $N_*X$. Moreover, in the homotopy category corresponding to the above model category structure, $F_\calC\mathbb{S}^t_{n,s_r,\ldots,s_1}$ represents $\pi_n(\DASH)^t_{s_r,\ldots,s_1}$, c.f.\ \cite[\S1]{MR1089001} or \cite[\S3.1.1]{Blanc_Stover-Groth_SS.pdf}.

%In light of this fact, the homotopy groups of $X$ possess
By virtue of the algebraic structure posessed by $X$, the homotopy groups $\pi_*X$ possess certain natural algebraic structure, that of a $\calc$-$\Pi$-algebra. Indeed, as any given homotopy group is a representable functor on the homotopy category, natural $N$-ary operations on homotopy groups
\[(\pi_*X)_{s_{r+1}^{1},\ldots,s_{1}^{1}}^{t^1}\times\cdots \times(\pi_*X)_{s_{r+1}^{N},\ldots,s_{1}^{N}}^{t^N}\to (\pi_*X)_{s_{r+1},\ldots,s_{1}}^{t}\]
are in bijective correspondence with elements of the group
\[ \pi_*\Bigl(F_\calC\mathbb{S}_{s_{r+1}^{1},\ldots,s_{1}^{1}}^{t^1}\sqcup\cdots\sqcup F_\calC\mathbb{S}_{s_{r+1}^{N},\ldots,s_{1}^{N}}^{t^N}\Bigr)\makebox[0cm][l]{}_{s_{r+1},\ldots,s_{1}}^{t}.\]
Blanc and Stover \cite{Blanc_Stover-Groth_SS.pdf} define a new category of graded universal algebras, the category of $\calC$-$\Pi$-algebras, monadic over $\vect{+}{r+1}$, whose objects are graded vector spaces $V\in\vect{+}{r+1}$ with a structure map 
\[V_{s_{r+1}^{1},\ldots,s_{1}^{1}}^{t^1}\times\cdots \times V_{s_{r+1}^{N},\ldots,s_{1}^{N}}^{t^N}\to V_{s_{r+1},\ldots,s_{1}}^{t}\]
for every such homotopy class, satisfying certain natural compatibilities.

It is a standard formalism to encode these compatibilities as follows. A \emph{model}  \cite{Blanc_Stover-Groth_SS.pdf} in $s\calc$ is an almost free object of $s\calc$ which is weakly equivalent to a coproduct of spheres. A \emph{finite model} is a model in which this coproduct is finite. Let $\Pi$  be the $\vect{}{}$-enriched category with objects the finite models in $s\calc$, and morphisms
\[\hom_{\Pi}(M,M'):=\hom_{\textup{ho}(s\calc)}(M,M').\]
Then the category of $\calc$-$\Pi$-algebras  may be defined as the category of $\vect{}{}$-enriched functors $\Pi^\textup{op}\to \vect{}{}$ that send finite coproducts into products. The category of $\calc$-$\Pi$-algebras is monadic over $\vect{+}{r+1}$. The forgetful functor is defined by:%$A$ under the forget Such a functor $A$ yields an object of $U_{\calc\PiAlg}A$ of $\vect{+}{r+1}$ via
\[(U_{\calc\PiAlg}A)^t_{s_{r+1},\ldots,s_1}:=A(F_\calC\mathbb{S}_{s_{r+1},\ldots,s_{1}}^{t}),\]
and any of the structure maps advertised above is induced by the corresponding homotopy class. %The natural compatibilities referred to above are the conditions required that a collection of structure maps extends to a functor on $\Pi^\textup{op}$.

One obtains the free $\calc$-$\Pi$-algebra on a graded vector space $V\in \vect{+}{r+1}$ using Dold's theorem. That is, one views $V$ as a chain complex in $\complexes\vect{+}{r}$ with zero differential, and applies the Dold-Kan correspondence and $\calc$-free functor, obtaining an object $F_\calc\Gamma V\in s\calc$. Then:
\[F_{\calc\PiAlg}V=\pi_*(F_\calc\Gamma V).\]
Moreover, as $F_{\calc}$ is an augmented monad, so is $F_{\calc\PiAlg}$, via the map
\[F_{\calc\PiAlg}V=\pi_*(F_\calc\Gamma V)\overset{\pi_*\epsilon}{\to}\pi_*(\Gamma V)=H_*V=V,\]
and in particular, there is an adjunction $Q^{\calc\PiAlg}\dashv K_{\calc\PiAlg}$.
%For any simplicial object $A\in s\calC$, there's a natural map $\gamma:Q^{\calC\PiAlg}\pi A\to \pi Q^{\calC}A$.
\begin{lem}
For any $A\in s\calC$, the map $\pi_*A\to \pi_*Q^\calc A$ descends to a map 
\[\gamma:Q^{\calC\PiAlg}\pi_* A\to \pi_* Q^{\calC}A.\]
If $A$ is a model in $\calc$ then $\gamma$ is an isomorphism.
\end{lem}
\begin{proof}{[\textbf{what's the status of this?}]} $A$ is homotopic to a wedge of spheres, and $\pi A$ is free on generators in correspondence with the wedge summands.
\end{proof}
\subsection{Cofibrant replacement via the small object argument}
The homotopy of an object $X$ of $s\calC$ was defined simply by application of the forgetful functor $U^{\calc}:\calc\to\vect{}{}$, a definition which is tautologically homotopically correct. On the other hand, in order to define the homology $H_*^{\calc}X$, as the left Quillen functor $Q^{\calc}$ does not preserve all weak equivalences, we must perform a cofibrant replacement before applying $Q^{\calc}$. While the comonadic bar construction $B^{\calc}$ described in \S\ref{ssec: quillen model and bar construction} suffices to define the groups $H_*^{\calc}X$, it lacks the structure that we will need at various points in this work.

Radulescu-Banu's innovation \cite{Radulescu-Banu.pdf} was to explain that the cofibrant replacement functor $c:s\calC\to s\calC$ constructed by Quillen's small object argument \cite{QuillenHomAlg.pdf}, which by design already posesses a natural acyclic fibration $\epsilon:c\to\Id$, in fact admits the full structure of a comonad, with diagonal $\psi:c\to cc$. As explained by Blumberg and Riehl \cite[Remark 4.12]{BlumRiehlResolutions.pdf},
\begin{prop}
The endofunctor $Q^\calc cK^\calc$ of $s\vect{}{}$ admits the structure of a comonad, via the maps
\[Q^\calc cK^\calc\overset{\smash{Q^\calc (\psi)}}{\to}Q^\calc ccK^\calc \overset{\smash{Q^\calc c(\eta)}}{\to} Q^\calc cK^\calc Q^\calc cK^\calc\text{ \ and \ }Q^\calc cK^\calc\overset{\smash{Q^\calc (\epsilon)}}{\to}Q^\calc K^\calc\cong\Id,\]
where $\eta$ denotes the unit of the $Q^\calC\dashv K_\calC$ adjunction.
\end{prop}

\subsection{Homology groups and $\calC$-$H_*$-coalgebras}\label{homology and Hcoalgs}
By Dold's theorem, there's a commuting diagram
\[\xymatrix@R=4mm@C=20mm{
s\vect{+}{r}\ar[r]^-{Q^\calc cK^\calc }
\ar[d]^-{\pi}
&%r1c1
s\vect{+}{r}\ar[d]^-{\pi}
\\%r1c2
\vect{+}{r+1}\ar[r]^-{C_{\calc\HCoalg}}
&
\vect{+}{r+1}
}\]
in which we are using Dold's theorem to \emph{define} $C_{\calc\HCoalg}$, the \emph{cofree $\calc$-$H_*$-coalgebra comonad}.
By the naturality of Dold's theorem, this is a comonad on $n\vect{+}{r+1}$. A $\calc$-$H_*$-coalgebra is simply a coalgebra over this monad, i.e.\ any $h\in\vect{+}{r+1}$ equipped with a coaction map $h\to C_{\calc\HCoalg}h$ satisfying the standard compatibilities. The homology $H_*^\calc X$ of $X\in s\calc$ is a $\calc$-$H_*$-coalgebra using the map
\[\pi_*(Q^{\calc}cX)\overset{\pi_*(Q^{\calc}(\psi))}{\to}\pi_*(Q^{\calc}ccX)\overset{\pi_*(Q^{\calc}c(\eta))}{\to}\pi_*(Q^{\calc}cK^{\calc}Q^{\calc}cX)=C_{\calc\HCoalg}(\pi_*(Q^{\calc}cX)).\]
If $X\sim K^{\calc}V$ for some $V\in s\vect{+}{r}$, then $H_*^{\calc}X\cong C_{\calc\HCoalg}(\pi_*(V))$, and the coaction map of $H_*^{\calc}X$ is none other than the diagonal map of the comonad.

Moreover, the comonad $C_{\calc\HCoalg}$ is coaugmented, via the map $a:\Id\to C_{\calc\HCoalg}$ induced by the Hurewicz map $\pi_*(V)\cong \pi_*(cKV)\to\pi_*(QcKV)$, in the sense that $\epsilon\circ a=1$ and $a\circ a=\Delta\circ a$. In particular, the monomorphism $a$ presents $W$ as a subspace of $CW$, and if $H$ is any $C$-coalgebra, we may define the \emph{primitives} in $H$ as the equaliser (of graded vector spaces):
\[\xymatrix@R=4mm@1{
\textup{Pr}(H)\ar[r]
&%r1c1
H\ar@<.5ex>[r]^-{\textup{coact}}
\ar@<-.5ex>[r]_-{a}
&%r1c2
CH%r1c3
}.\]
\begin{shaded}\small
\textbf{Move this stuff somewhere less general:} Include a bit about colimits, as in \verb|adamsops|
\begin{prop}
\textbf{Move/delete me}: For $W$ a graded vector space, the map $a:W\to CW$ is an isomorphism onto the primitives $\textup{Pr}(CW)$ of the cofree $C$-coalgebra $CW$.
\end{prop}
\begin{prop}
The Hurewicz map $\pi_*A\to H_*A$ factors through $\textup{Pr}(H_*A)$, and if $A$ is a simplicial zero-square algebra the resulting map $\pi_*A\to\textup{Pr}(H_*A)$ equals the isomorphism $a:\pi_*A\to\textup{Pr}(C\pi_*A)$.
\end{prop}
\begin{proof}
Represent an element $\alpha\in\pi_nA$ by a map $\alpha:S^n_\calC\to A$. Then, $h(\alpha)$ is the image of the fundamental homology class $\imath\in H_n(S^n_\calC)$ under the map $\alpha_*$. As $\imath$ is primitive, so is $h(\alpha)$.
\end{proof}
At this point, we examine the cosimplicial resolution of \cite[4.6]{BlumRiehlResolutions.pdf}:
\[
\vcenter{
\def\labelstyle{\scriptstyle}
\xymatrix@C=1.5cm@1{
cX_\bullet\,
\ar[r]
&
\,cKQcX_\bullet\,
\ar[r];[]
&
\,c(KQc)^2X_\bullet\,
\ar@<-1ex>[l];[]
\ar@<+1ex>[l];[]
\ar@<+1ex>[r];[]
\ar@<-1ex>[r];[]
&
\,c(KQc)^3X_\bullet\,\makebox[0cm][l]{\,$\cdots $}
\ar[l];[]
\ar@<-2ex>[l];[]
\ar@<+2ex>[l];[]
}}\]
Applying $\pi_*(Q\DASH)$, we obtain exactly the cobar construction of the $C$-coalgebra $H_*X_\bullet$:
\[
\vcenter{
\def\labelstyle{\scriptstyle}
\xymatrix@C=1.5cm@1{
HX\,
\ar[r]
&
\,CHX\,
\ar[r];[]
&
\,CCHX\,
\ar@<-1ex>[l];[]
\ar@<+1ex>[l];[]
\ar@<+1ex>[r];[]
\ar@<-1ex>[r];[]
&
\,c(KQc)^3X_\bullet\,\makebox[0cm][l]{\,$\cdots $}
\ar[l];[]
\ar@<-2ex>[l];[]
\ar@<+2ex>[l];[]
}}\]
\end{shaded}

\subsection{Cohomology groups and $\calC$-$H^*$-algebras}\label{cohomology and Halgs}
It will in general be preferable to for us to consider algebraic structure on cohomology, rather than coalgebraic structure on homology. Firstly, algebra is in general a more familiar subject than coalgebra, and secondly, cohomology has the advantage that it consists of representable functors. That is, in the homotopy category of $s\calc$, the object $K_\calC\mathbb{S}^t_{n,s_r,\ldots,s_1}$ represents $H^n_{\calc}(\DASH)_t^{s_r,\ldots,s_1}$,  c.f.\ \cite[Proposition 4.3]{MR1089001}. Using cohomology groups has the disadvantages associated with double-dualisation.


%The forgetful functor $U_\calC:\calc\to \vect{+}{r}$ used to define the weak equivalences and fibrations in $s\calc$, allows us to consider any object $X\in s\calC$ as an object of $s\vect{+}{r}$. In particular, for any $X\in s\calc$ we may define homotopy groups
%We have defined the cohomology of $X\in s\calc$ to be the dual left-derived functors of the indecomposables functor $Q_\calC:\calc\to \vect{+}{r}$,
%\[(H^{s_{r+1}}_\calc X)^{s_r,\ldots,s_1}_t:=((H_{s_{r+1}}^\calc X)_{s_r,\ldots,s_1}^t)^*:=(\pi_{s_{r+1}}(Q^{\calc}B^\calc X)_{s_r,\ldots,s_1}^t)^*,\]
%which we view together as an object of $\vect{r+1}{+}$.


%In light of this fact, the homotopy groups of $X$ possess
By virtue of the algebraic structure posessed by $X$, the cohomology groups $H_\calc^*X$ possess certain natural algebraic structure, that of a $\calc$-$H^*$-algebra. It will follow from the definitions that the dual of a $\calc$-$H_*$-algebra is a $\calc$-$H^*$-algebra, but the reverse will only hold under appropriate finiteness conditions. As for $\calc$-$\Pi$-algebras, natural $N$-ary operations on cohomology groups
\[(H^*_\calc X)^{s_{r+1}^{1},\ldots,s_{1}^{1}}_{t^1}\times\cdots \times(H^*_\calc X)^{s_{r+1}^{N},\ldots,s_{1}^{N}}_{t^N}\to (H^*_\calc X)^{s_{r+1},\ldots,s_{1}}_{t}\]
are in bijective correspondence with elements of the group
\[ H^*_\calc\Bigl(K_\calC\mathbb{S}_{s_{r+1}^{1},\ldots,s_{1}^{1}}^{t^1}\times\cdots\times K_\calC\mathbb{S}_{s_{r+1}^{N},\ldots,s_{1}^{N}}^{t^N}\Bigr)\makebox[0cm][l]{}^{s_{r+1},\ldots,s_{1}}_{t}.\]
The category of $\calC$-$H^*$-algebras, monadic over $\vect{r+1}{+}$, has objects graded vector spaces $V\in\vect{r+1}{+}$ with a structure map 
\[V^{s_{r+1}^{1},\ldots,s_{1}^{1}}_{t^1}\times\cdots \times V^{s_{r+1}^{N},\ldots,s_{1}^{N}}_{t^N}\to V^{s_{r+1},\ldots,s_{1}}_{t}\]
for every such cohomology class, satisfying certain natural compatibilities.

The formalism required to express these compatibilities is as follows. A \emph{generalized Eilenberg-Mac Lane object}, or \emph{GEM}, in $s\calc$ is an almost free object of $s\calc$ which is weakly equivalent to a product of objects of the form $K_\calC\mathbb{S}_{s_{r+1},\ldots,s_{1}}^{t}$. A \emph{finite GEM} is a GEM in which this product is finite. Let $K$  be the $\vect{}{}$-enriched category with objects the finite GEMs in $s\calc$, and morphisms
\[\hom_{\Pi}(M,M'):=\hom_{\textup{ho}(s\calc)}(M,M').\]
Then the category of $\calc$-$H^*$-algebras  may be defined as the category of $\vect{}{}$-enriched functors $\mathbb{K}\to \vect{}{}$ that preserve finite products. The category of $\calc$-$H^*$-algebras is monadic over $\vect{r+1}{+}$. The forgetful functor is defined, on $h:K\to \vect{}{}$, by:%$A$ under the forget Such a functor $A$ yields an object of $U_{\calc\HAlg}A$ of $\vect{r+1}{+}$ via
\[(U_{\calc\HAlg}h)_t^{s_{r+1},\ldots,s_1}:=h(K_\calC\mathbb{S}_{s_{r+1},\ldots,s_{1}}^{t}),\]
and any of the structure maps advertised above is induced by the corresponding cohomology class. %The natural compatibilities referred to above are the conditions required that a collection of structure maps extends to a functor on $\Pi^\textup{op}$.

One obtains the free $\calc$-$\Pi$-algebra on a graded vector space $V\in \vect{r+1}{+}$ as follows. One views $V$ as a chain complex in $\complexes\vect{+}{r}$ with zero differential, and applies the Dold-Kan correspondence and $\calc$-free functor, obtaining an object $F_\calc\Gamma V\in s\calc$. Then:
\[F_{\calc\HAlg}V=H^*_{\calc}K_\calc\Gamma V.\]
Moreover, $F_{\calc\HAlg}$ is an augmented monad: one applies $H^*_{\calc}$ to the natural collapse map $F^{\calc}\Gamma V\to K^{\calc}\Gamma V$, to obtain
\[K_{\calc\HAlg}V\cong H^*_{\calc} F^{\calc}\Gamma V\from H^*_{\calc} K^{\calc}\Gamma V=:F_{\calc\HAlg}V.\]

\[F_{\calc\HAlg}V=\pi_*F_\calc\Gamma V\overset{\pi_*\epsilon}{\to}\pi_*\Gamma V=H_*V=V,\]
and in particular, there is an adjunction $Q^{\calc\HAlg}\dashv K_{\calc\HAlg}$.
%For any simplicial object $A\in s\calC$, there's a natural map $\gamma:Q^{\calC\HAlg}\pi A\to \pi Q^{\calC}A$.


\subsection{The smash coproduct}\label{subseq:The smash coproduct}
For $X_1,X_2\in\calc$, we define the \emph{smash coproduct} $X_1\wedge X_2$ to be the kernel of the natural map $X_1\sqcup X_2\to X_1\times X_2$. When $X_1=X_2=X$, $X\wedge X$ has a natural action of $\Sigma_2$, and we write $X\wedge^{\Sigma_2} X$ for the subobject of invariant elements under this action.

\tododone{Various intro here}{Say what a $\calC$-$\Pi$-algebra is, even with all the gradings.
\item Construct the free $\calC$-$\Pi$-algebra on a set (\textbf{what happens on a vector space?}).
\item Say what a model is. Move the discourse on spheres in $\complexes\vect{}{}$ to somewhere above. Also, put some discourse on the smash coproduct into this paragraph.}
We will find it useful to be able to express the smash coproduct of free $\calC$-$\Pi$-algebras as the homotopy groups of an object of $s\calC$.
As every free $\calC$-$\Pi$-algebra is the homotopy of a model in $s\calc$, it will be enough to prove
\begin{prop}
For models $X_1$ and $X_2$ in $s\calc$, there is an isomorphism
\[i:\pi_* X_1\wedge \pi_* X_2\to\pi_*(X_1\wedge X_2),\]
natural in the models $X_1$ and $X_{2}$.

\end{prop}
\begin{proof}
The natural map $i:\pi_* X_1\sqcup \pi_* X_2\to \pi_*(X_1\sqcup X_2)$ is an isomorphism, since both source and target represent the free $\calC$-$\Pi$-algebra on generators corresponding to the wedge summands of $X_1$ and $X_2$ taken together. Moreover, the natural map $\pi_*(X_1\times X_2)\to \pi_* X_1\times \pi_* X_2$ is an isomorphism, as the forgetful functor is a right adjoint, and $\pi_*$ preserves products (of vector spaces). Thus we have a commuting square
\[\xymatrix@R=4mm{
\pi_* X_1\sqcup \pi_* X_2
\ar@{->>}[r]
\ar[d]^-{i}_-{\cong}
&%r1c3
\pi_* X_1\times \pi_* X_2
\ar[d]_-{\cong}\\
\pi_* (X_1\sqcup X_2)
\ar[r]
&%r2c3
\pi_* (X_1\times X_2)
}\]
and the homotopy long exact sequence of $X_1\wedge X_2\to X_1\sqcup X_2\to X_1\times X_2$ has connecting maps zero. The resulting morphism of short exact sequences
\[\xymatrix@R=4mm{
0\ar[r]&%r1c1
\pi_* X_1\wedge \pi_* X_2
\ar[r]
\ar[d]^-{i}_-{\cong}
&%r1c2
\pi_* X_1\sqcup \pi_* X_2
\ar[r]
\ar[d]^-{i}_-{\cong}
&%r1c3
\pi_* X_1\times \pi_* X_2
\ar[r]
\ar[d]_-{\cong}
&%r1c4
0\\%r1c5
0\ar[r]&%r1c1
\pi_* (X_1\wedge  X_2)
\ar[r]
&%r1c2
\pi_* (X_1\sqcup X_2)
\ar[r]
&%r2c3
\pi_* (X_1\times X_2)
\ar[r]
&%r1c4
0
}\]
provides the desired isomorphism.
\end{proof}



\subsection{The quadratic part of a $\calC$-expression}\label{quadratic part section}
In this paper, we'll often use a method of constructing cohomology operations used by Goerss in \cite[\S5]{MR1089001}, and here we'll set up a framework that can be applied to each case. Suppose that $\calC$ is an algebraic category, monadic over $\vect{}{}$, category of graded vector spaces. 



For $V\in\vect{}{}$, the diagonal map $\Delta:V\to V\oplus V$ of $V$ induces a `diagonal map' $F_\calC V\to F_\calC (V\oplus V)\cong (F_\calC (V))^{\sqcup 2}$, and writing $i_1$ and $i_2$ for the two summand inclusions $F_\calC (V)\to (F_\calC (V))^{\sqcup 2}$, consider the map
\[(F_\calC(\Delta)+i_1+i_2):F_\calC V\to (F_\calC V)^{\sqcup2}.\]
This map evidently factors through $(F_\calC V)^{\wedge 2}$, and is symmetric. We name this factoring the \emph{cross terms}:
\[\crossterms:F_\calC V\to (F_\calC V)\wedge^{\Sigma_2} (F_\calC V),\]
as it measures the nonlinearity in an expression in $F_\calC V$. For example, when $\calC=S(\CommOperad)$, then for $v,w\in V$, using subscripts to denote membership of the first or second copy of $V$:
\[\crossterms(vw)=(v_1+v_2)(w_1+w_2)+v_1w_1+v_2w_2=v_1w_2+w_1v_2.\]

In each case of interest to us, we will give a map, natural and symmetric in $X_1,X_2\in\calC$,
\[j_\calC:Q^\calC(X_1\wedge X_2)\to Q^\calC(X_1)\otimes Q^\calC(X_2),\]
and we'll write $\quadratic_\calC$ for \emph{quadratic part} map, the composite
\[\quadratic_\calC:\left(F_\calC V\overset{\crossterms}{\to}(F_\calC V)^{\wedge 2}\to Q^\calC((F_\calC V)\wedge^{\Sigma_2} (F_\calC V))\overset{j_\calC}{\to}S^2(Q_{\calc}F_{\calC}V)=S^2V\right).\]
When $\calC=S(\CommOperad)$, the map $j_{S(\CommOperad)}$ is defined by $x_1x_2\mapsto x_1\otimes x_2$ when $x_i\in X_i$, and $\quadratic_{S(\CommOperad)}$ is the composite $F_{S(\CommOperad)} V\epi S_2V\overset{\trace}{\to}S^2V$, for example,  $\quadratic_{S(\CommOperad)}(u+vw+xy^2)$ evaluates as:
\[j_{S(\CommOperad)}(v_1w_2+w_1v_2+x_1y_2^2+y_1^2x_2)=v\otimes w+w\otimes v+x\otimes y^2+y^2\otimes x,\]
which equals $v\otimes w+w\otimes v$ %when evaluated in $S^2(Q^{S(\CommOperad)}F^{S(\CommOperad)}V)=S^2V$, 
as hoped.

In each category of interest to us, the following equation of maps $F_\calC F_\calC V\to S^2 V$ will always be satisfied:
\[\quadratic_\calC\circ\mu_V=\quadratic_\calC\circ \epsilon_{F_\calC V} +\quadratic_\calC\circ {F_\calC \epsilon_V},\]
where $\mu$ here stands for the multiplication map of the monad $U_\calC F_\calC $, which is to say that if  $f(g_i)$ is a $\calC$-expression in various $\calC$-expressions $g_i(v_{ij})$, we have 
\[\quadratic(fg_i)(v_{ij})=\quadratic(f\epsilon(g_i))(v_{ij})+\epsilon(f)(\quadratic(g_i)(v_{ij})).\]
This is another expression of homogeneity in the relations defining $\calC$. For an example when $\calC=S(\CommOperad)$, we specify an expression $f(g_1,g_2,g_3):=g_1g_2+g_3\in F_{\calc}F_{\calc}V$ in expressions $g_i:=v_{i1}v_{i2}+v_{i3}\in F_{\calc}V$ for each $i=1,2,3$. Then
\begin{alignat*}{2}
\textup{\small$\quadratic(fg_i)(v_{ij})={}$}
&
\textup{\small$\quadratic((v_{11}v_{12}+v_{13})(v_{21}v_{22}+v_{23})+(v_{31}v_{32}+v_{33}))=\trace(v_{13}\otimes v_{23}+v_{31}\otimes v_{32}),$}
\\
\textup{\small$\quadratic(f\epsilon(g_i))(v_{ij})={}$}
&
\textup{\small$\quadratic((v_{13})(v_{23})+(v_{33}))=\trace(v_{13}\otimes v_{23}),\textup{ and}$}
\\
\textup{\small$\epsilon(f)(\quadratic(g_i)(v_{ij}))={}$}
&
\textup{\small$\quadratic(v_{31}v_{32}+v_{33})=\trace(v_{31}\otimes v_{32}).$}
\end{alignat*}










\end{CPiAlgs and CHalgs}



\begin{Constructing homotopy and cohomotopy operations}
\vfil\pagebreak


\section{\textbf{Constructing homotopy operations}}\label{sec:Constructing homotopy operations}

\subsection{Higher simplicial Eilenberg-Mac Lane maps}[\textbf{$\Delta$ because} it's the right symbol, should be an upper index since it decreases homological degree.]


In what follows, we will often have a natural map $F$ whose domain and codomain both support a switch map $T$, obtained by interchanging tensor factors. Furthermore, we will so often have use for the expression $TFT$, that we introduce the shorthand $\twist F:=TFT$. Although this notation has the potential for ambiguity, we will be careful to avoid confusion.

Let $\{\DeltaUp^k\}$ be a higher simplicial Eilenberg-Mac Lane map \cite[\S3]{DwyerHtpyOpsSimpComAlg.pdf}, i.e.\ a collection of maps
\[\DeltaUp^k:(CU\otimes CV)_{i+k}\to N(U\otimes V)_i\textup{\ \ defined for $0\leq k\leq i$}\]
natural in simplicial vectorspaces $U$ and $V$, such that for $k\geq0$, the identity
\[(1+\twist) \DeltaUp^k=\oldphi^k+\begin{cases}
\DeltaUp^{k-1}\partial+\partial\DeltaUp^{k-1},&\textup{if }k\geq1,\\
\Delta,&\textup{if }k=0,
%\\,&\textup{if }
\end{cases}
\]
holds on classes of dimension at least $2k$, and:
\begin{itemize}
\setlength{\parindent}{.25in}
\item $\Delta:CU\otimes CV\to N(U\times V)$ is the Eilenberg-Zilber map, a chain homotopy equivalence inducing the identity in dimension zero; and
\item $\oldphi^k$ is the map $(CU\otimes CV)_{i+k}\to N(U\otimes V)_i$ which vanishes except on $U_k\otimes V_k$, where its value is just the projection $U_k\otimes V_k\to N(U\times V)_k$.
\end{itemize}
Note that as $\oldphi^0$ commutes with symmetry isomorphisms, so does $\Delta$.





\subsection{External unary homotopy operations}
In this section we recall the definition of certain homotopy operations with domain $\pi_*V$ for any $V\in s\vect{}{}$, implicit in \cite[\S4]{DwyerHtpyOpsSimpComAlg.pdf} (or cartan or Bousfield - explicit?) and explicit in 
\cite[\S3]{MR1089001}, using the functions
\[a \mapsto \DeltaUp^{n-i}(a\otimes a),\ \ N_nV \to N_{n+i}(S_2V).\]
By postcomposing with the maps $S_2V\to \Lambda^{2}V\to S^2V$, we obtain functions from $N_nV$ to $N_{n+i}(\Lambda^2V)$ and $N_{n+i}(S^2V)$.
\begin{prop}[{\cite[Lemma 4.1]{DwyerHtpyOpsSimpComAlg.pdf},\ \cite[\S3]{MR1089001}}] \label{extUnaryHomotOps}
These functions descend to  well defined homotopy operations:
\begin{alignat*}{2}
\delta_i^\textup{ext}:\pi_nV&\to \pi_{n+i}(S_2V),&\quad&(2\leq i\leq n),\\
\lambda_i^\textup{ext}:\pi_nV&\to \pi_{n+i}(\Lambda^2V),&\quad&(1\leq i\leq n),\\
\sigma_i^\textup{ext}:\pi_nV&\to \pi_{n+i}(S^2V),&\quad&(1\leq i\leq n).
\end{alignat*}
The function $N_nV \to N_{n}(S^2V)$ given by $\overline{a}\mapsto \overline{a\otimes a}$ yields a well defined homotopy operation $\sigma_0^\textup{ext}:\pi_nV\to \pi_{n}(S^2V)$. 
For all $n\geq0$, the map $\sigma_n^\textup{ext}:\pi_nV\to\pi_{2n}(S^2V)$ satisfies
\[\sigma_n^\textup{ext}(\overline{x}+\overline{y})=\sigma_n^\textup{ext}(\overline{x})+\sigma_n^\textup{ext}(\overline{y})+\overline{(1+T)\Delta(x\otimes y)}\quad\text{for $x,y\in ZN_nV$}.\]
%For $2\leq i\leq n$, the function $N_nV \to N_{n+i}S_2V$ yields a well-defined homotopy operation $\sigma_i^\textup{ext}:\pi_nV\to \pi_{n+i}(S_2V)$. 
%For $1\leq i\leq n$, the function $N_nV \to N_{n+i}\Lambda^2V$ yields a well-defined homotopy operation $\sigma_i^\textup{ext}:\pi_nV\to \pi_{n+i}(\Lambda^2V)$. 
%%produces a well defined homotopy operation $\sigma_i^\textup{ext}:\pi_n(V)\to \pi_{n+i}(\Lambda^2V)$. 
%The function \[a\mapsto \oldphi^n(a\otimes a),\ \ N_nV \to N_{n}S^2V\] yields a well-defined homotopy operation $\sigma_0^\textup{ext}:\pi_nV\to \pi_{n}(S^2V)$.
\end{prop}
\begin{proof}
Although all of the operations  are defined in the cited references, we will be a little more explicit about the definition of $\sigma^\textup{ext}_0$, and the final equation of the proposition.

As described in \cite[\S3]{MR1089001}, we might choose to define $\sigma^\textup{ext}_0$ using a universal example, for which the cycle 
\[z\otimes z\in ZN_n(S^2\mathbb{S}_n)\cong \F_2\]
is the only possible representative, demonstrating that the formula $\overline{a}\mapsto \overline{a\otimes a}$ yields the correct (well defined) operation. To check that $\sigma_0^\textup{ext}$ satisfies the stated equation, we need only check that it holds on $z_1+z_2\in ZN_0(\mathbb{S}_0\oplus \mathbb{S}_0)\cong \F_2\oplus \F_2$. But
\[\sigma_0^\textup{ext}(z_1+z_2)-\sigma_0^\textup{ext}(z_1)-\sigma_0^\textup{ext}(z_2)=z_1\otimes z_2+z_2\otimes z_1=(1+T)\Delta(z_1\otimes z_2),\]
as $\Delta$ is the identity in dimension zero.

To explain the equation when $n\geq1$, as $\sigma_n(\overline{x}):=\overline{(1+T)\Delta^0(x\otimes x)}$, we obtain %into the expression  $\sigma_n^\textup{ext}(x+y)-\sigma_n^\textup{ext}(x)-\sigma_n^\textup{ext}(y)$ to obtain
\[\sigma_n^\textup{ext}(\overline{x}+\overline{y})-\sigma_n^\textup{ext}(\overline{x})-\sigma_n^\textup{ext}(\overline{y})
=
\overline{(1+T)\Delta^0(1+T)(x\otimes y)}\]
and using the symmetry  $T\bigl((1+T)(x\otimes y)\bigr)=(1+T)(x\otimes y)$, and the fact that $\oldphi^0$ vanishes on $(1+T)(x\otimes y)$:
%\begin{alignat*}{2}
%(1+T)\Delta^0(1+T)(x\otimes y)
%&=
%(1+\omega\Delta^0)(1+T)(x\otimes y)%
%\\
%&=
%(\Delta+\oldphi^0)(1+T)(x\otimes y)%
%\\
%&=
%\Delta(1+T)(x\otimes y)%
%\end{alignat*}
\[
(1+T)\Delta^0(1+T)(x\otimes y)
=
(1+\omega\Delta^0)(1+T)(x\otimes y)
=
\Delta(1+T)(x\otimes y).\qedhere
\]
%where we have used the symmetry
%\[T\bigl((1+T)(x\otimes y)\bigr)=(T+T^2)(x\otimes y)=(1+T)(x\otimes y)\]
%in order to exchange $T\Delta^0$ and $\omega\Delta^0$ at the second equality.
\end{proof}

\subsection{External binary homotopy operations}
We'll now give an account of various natural external homotopy operations, most of which are binary operations, induced by the  Eilenberg-Mac Lane shuffle map $\nabla:N_*(V)\otimes N_*(V)\to N_*(V\otimes V)$, which is also known as the Eilenberg-Zilber map.
%The homotopy operations we'll present will be in the form of a natural transformation of functors $s\vect{+}{n}\to \vect{+}{n+1}$, which will be induced by the Eilenberg-Mac Lane shuffle map $\nabla:N_*(V)\otimes N_*(V)\to N_*(V\otimes V)$. \textbf{Need to clarify what's meant by quad grading.}
\begin{prop}[Various authors]\label{the top external homotopy operations}
There is a natural commuting diagram:
\[\xymatrix@R=4mm{
S_2(\pi_*V)\ar[r]^-{\smash{\widetilde{\nabla}}}
\ar[d]^-{\textup{proj}}
&%r1c1
\pi_*(S_2V)\ar[d]^-{\pi_*(\textup{proj})}
\\%r1c2
\Lambda^2(\pi_*V)\ar[r]^-{\widetilde{\nabla}}
\ar[d]^-{\textup{incl}}
&%r1c1
\pi_*(\Lambda^2V)\ar[d]^-{\pi_*(\textup{incl})}
\\%r1c2
S^2(\pi_*V)\ar[r]^-{\widetilde{\nabla}}&%r2c1
\pi_*(S^2V)%r2c2
}\]
For cycles $x,y\in ZN_*(V)$ and $z\in ZN_n(V)$, the upper horizontal is determined by:
\[\overline{x}\otimes\overline{y}\mapsto \overline{x\otimes y},\]
and the lower horizontal is determined by:
\[\overline{x}\otimes\overline{y}+\overline{y}\otimes \overline{x}\mapsto\overline{\nabla(x\otimes y+y\otimes x)}\textup{, and }\overline{z}\otimes\overline{z}\mapsto\sigma^\textup{ext}_n(\overline{z}).\]
\end{prop}
\begin{proof}
During this proof, write $\widetilde{\nabla}_U$, $\widetilde{\nabla}_M$ and $\widetilde{\nabla}_L$ for the upper, middle and lower horizontal maps. We must demonstrate: that $\widetilde{\nabla}_U$  is well defined; that
\[\ker(\pi_*(\textup{proj})\circ\widetilde{\nabla}_U)\supseteq\ker(\textup{proj}),\]
so that there is a unique map $\widetilde{\nabla}_M$ for which the upper square commutes; and that one may extend the composite $\pi_*(\trace)\circ\widetilde{\nabla}_U$ along the trace map $S_2(\pi_*A)\to S^2(\pi_*A)$ using the operations
$\sigma_n^\textup{ext}$.
A simple diagram chase would then reveal that the bottom square must also commute.
%in order that $\widetilde{\nabla}_U$ extends to  a unique map $\widetilde{\nabla}_M$ for which the diagram commutes.

As $\nabla$ is a chain  map, it produces a well defined map $(\pi_*V)^{\otimes2}\to \pi_*(V^{\otimes2})$, and the fact that $\nabla=\twist\nabla$ implies that this map descends to a well defined map $\nabla_U$.

The kernel of the projection $S_2(\pi_*V)\to \Lambda^2(\pi_*V)$ is spanned by classes of the form $\overline{x}\otimes \overline{x}$, and the image under $\pi_*(\textup{proj})\circ\widetilde{\nabla}_U$ of such a class may be represented by $x\otimes x\in \Lambda^2V$, which equals zero, proving the inclusion of kernels.

Finally, to extend the composite $\pi_*(\trace)\circ\widetilde{\nabla}_U$, we simply need the operations $\sigma_n$ to satisfy the equations of proposition \ref{propOnExtendingToInvariants}, which are part of proposition \ref{extUnaryHomotOps}.
%
%To produce $\widetilde{\nabla}_L$ is to extend the composite $\pi_*(\trace)\circ\widetilde{\nabla}_U$, along the trace map $S_2(\pi_*A)\to S^2(\pi_*A)$.
%
%Since $\nabla$ is a symmetric chain map, the formula given for $\widetilde{\nabla}_L$ does indeed return homotopy classes in $\pi_*(S_2V)$.
%
% If these maps are well defined, the rest is clear. %, however, it is not clear that either of the maps $\widetilde{\nabla}$ is well defined. 
%
%%In fact, we only need to check that  $\widetilde{\nabla}_L$ is well defined: as $\pi_*(\textup{incl})$ is a monomorphism (by \cite[Prop 5.6]{BousOpnsDerFun.pdf}), it will follow that the upper map is well defined. 
%
%By general observations on natural transformations with domain the endofunctor $S^2$ of $\vect{}{}$, in order to extend
%\[\pi_*(\trace)\circ\widetilde{\nabla}_U:S_2(\pi_*V)\to \pi_*(S^2V)\]
%to a natural map $\widetilde{\nabla}_L$, it suffices to specify a natural function $\restn{(\DASH)}:\pi_n(V)\to \pi_{2n}(S^2V)$ satisfying the equation
%\[\restn{(x+y)}=\restn{x}+\restn{y}+\pi_*(\trace)(\widetilde{\nabla}_U(x\otimes y)).\] 
% Then $\widetilde{\nabla}_L$ is the unique extension of $\pi_*(\trace)\circ\widetilde{\nabla}_U$  such that $\widetilde{\nabla}_L(\alpha\otimes \alpha)=\restn{\alpha}$ for $\alpha\in\pi_*\left(V\right)$.
%%\[p=\left(S_2\pi_*(V)\overset{\trace}{\to} S^2\pi_*(V)\overset{\widetilde{\nabla}'}{\to} \pi_*(S^2V)\right)\textup{ and }\restn{(\DASH)}=\left(\pi_*(V)\overset{v\mapsto v\otimes v}{\to} S^2\pi_*(V)\overset{\widetilde{\nabla}'}{\to} \pi_*(S^2V)\right).\]
%We define $\widetilde{\nabla}_L$ in this way, with $\restn{(\DASH)}$ the operation $\sigma_n:\pi_n(V)\to \pi_{2n}(S^2V)$, which satisfies this equation by proposition \ref{extUnaryHomotOps}.
%%This operation does not depend on the choice of representative $x$. Indeed,
%%\[\nabla((x+dh)\otimes (x+dh))=\nabla(x\otimes x)+d\trace(x\otimes h)+d(h\otimes dh)\]
%%
%%
%%
%%When $n\geq0$, this map is defined on cycles by $z\mapsto \nabla(z\otimes z)$, and the required equation is simply the expansion
%%\[\Delta((x+y)\otimes(x+y))=\Delta(x\otimes x)+\Delta(y\otimes y)+(1+T)\Delta(x\otimes y).\]
\end{proof}

\subsection{Homotopy operations for simplicial commutative algebras}\label{Homotopy operations for simplicial commutative algebras}
Suppose that $A\in s S(\CommOperad)$ is a simplicial non-unital commutative algebra, with multiplication map $\mu:S_2A\to A$. Then by composition with the map $\pi_*(\mu):\pi_*(S_2A)\to \pi_*A$, one obtains unary operations:
\[\delta_i:=\pi_*(\mu)\circ\delta_i^\textup{ext}:\pi_nV\to \pi_{n+i}V,\textup{ defined when }2\leq i\leq n,\]
and a pairing\[\mu:=\pi_*(\mu)\circ\widetilde{\nabla}:S_2(\pi_*V)\to \pi_{*}V.\]
\begin{prop}[Dwyer {\cite{DwyerHtpyOpsSimpComAlg.pdf}}]\label{omnibus on htpy of simp algs}These operations have the following properties:
\begin{enumerate}
\item the pairing $\mu$ gives $\pi_*A$ the structure of a non-unital commutative algebra;
\item the ideal $\bigoplus_{n\geq1}\pi_nA$ is an exterior algebra;
\item the ideal $\bigoplus_{n\geq2}\pi_nA$ is a divided power algebra, with divided square given by the \emph{top} $\delta$-operation, i.e.\ $x\mapsto \delta_nx\text{ for }x\in\pi_nA$;
\item the \emph{non-top operations}, $\delta_i:\pi_nA\to \pi_{n+i}A$ for $2\leq i<n$, are linear;
\item there holds the following \emph{Cartan formula}: for $x\in\pi_nA$, $y\in \pi_mA$ and $2\leq i\leq n$
\[\delta_i(xy)=\begin{cases}
y^2\delta_i(x),&\text{if }m=0;\\
0,&\text{otherwise};
\end{cases}
\]
\item \label{deltaademsunstable} the \emph{$\delta$-Adem relations} hold: if $\delta_i\delta_jx$ is defined, and $i<2j$, then
\[\delta_i\delta_jx:=\sum_{s=\lceil(i+1)/2\rceil}^{\lfloor(i+j)/3\rfloor}{j+s-i-1\choose j-s}\delta_{i+j-s}\delta_sx.\]
\end{enumerate}
\end{prop}
A few comments are in order. Firstly, the proposition distinguishes between the \emph{top} and \emph{non-top} $\delta$ operations, as they have rather different behaviour --- this will be a recurring pattern. Secondly, it is not immediately obvious that the $\delta$-Adem relations make sense, in that it is not obvious that every term in the right hand side is defined. This does indeed happen, by lemma \ref{lemOnAdemChangeInMDeltaPlain} (to follow). 

Now we may define a non-unital associative algebra, $\deltaalg$, as the algebra generated by $\delta_i$ for $i\geq2$, subject to relations
\[\delta_i\delta_j:=\sum_{s=\lceil(i+1)/2\rceil}^{\lfloor(i+j)/3\rfloor}{j+s-i-1\choose j-s}\delta_{i+j-s}\delta_s\textup{ when $i<2j$}.\]
We will say that a sequence $I=(i_\ell,\ldots,i_1)$ of integers $i_j\geq2$ is \emph{$\delta$-admissible} if $i_{j+1}\geq 2i_j$ for $1\leq j <\ell$. For any sequence $I=(i_\ell,\ldots,i_1)$, write $\delta_I$ for the composite $\delta_{i_\ell}\cdots \delta_{i_1}$. Then this relation evidently allows us to write any $\delta_I$ in $\deltaalg$ as a sum of composites $\delta_J$ in which $J$ is $\delta$-admissible. In fact, it follows from \cite[Proposition 2.7]{MR1089001} that the algebra $\deltaalg$ has an \emph{admissible basis}, consisting of those $\delta_I=\delta_{i_\ell}\cdots \delta_{i_{1}}$ with $I$ a $\delta$-admissible sequence. 

It then makes sense to make the following definition. Suppose that $I$ is any nonempty sequence of integers at least $2$, and $J$ is a $\delta$-admissible sequence. Then we will say that \emph{$I$ produces $J$ in $\deltaalg$}, denoted $\produces{I}{J}{\deltaalg}$ if, when $\delta_I$ is written in the $\delta$-admissible basis of $\deltaalg$, $\delta_J$ appears with non-zero coefficient. In this case, $J$ must be $\delta$-admissible, and $I$ must be $\delta$-inadmissible unless $J=I$.

Proposition \ref{omnibus on htpy of simp algs} does not state that $\pi_*A$ is a left module over $\deltaalg$, due to the fact that the $\delta$-operations are not always defined (or even linear). We define
\[\minDimDelta(I):=\max\{(i_1),\,(i_2-i_1),\,(i_3-i_2-i_1),\,\ldots,\,(i_{\ell}-\cdots-i_1)\},
\]
following the convention that $\max(\emptyset)=-\infty$, for any sequence $I$ of integers $i_j\geq2$ (or more generally, for any sequence of non-negative integers).  The intent of this definition is that the composite $\delta_I$, by which we mean
\[\pi_{n}A\overset{\delta_{i_1}}{\to}\pi_{n+i_1}A\overset{\delta_{i_2}}{\to}\cdots \overset{\delta_{i_\ell}}{\to}\pi_{n+i_1+\cdots +i_\ell}A\]
is defined if and only if $n\geq \minDimDelta(I)$. 
Note that when $I$ is a non-empty \emph{$\delta$-admissible} sequence,
\[\minDimDelta(I)=i_{\ell}-i_{\ell-1}-\cdots -i_1=:e(I),\]
the \emph{Serre excess of $I$}. Moreover, if $I$ is $\delta$-admissible, then for any expression $\delta_{i_{\ell}}\cdots \delta_{i_{1}}x$ there is some $k$ with $0\leq k\leq \ell$ such that each of the $k$ the operations $\delta_{i_{\ell}}\cdots \delta_{i_{\ell-k+1}}$ are acting as top operations, and each of the remaining $\ell-k$ are acting as non-top operations.

The following lemma assures us that the $\delta$-Adem relations make sense even in \emph{(\ref{deltaademsunstable})}.
%\begin{lem}\label{lemOnAdemChangeInMDeltaPlain}
%Suppose that $i,j\geq2$ and $i<2j$, and $(i+1)/2\leq s\leq (i+j)/3$. Then $i+j-s\geq 2s$, $i+j-s\geq2$, $s\geq2$, so that the $\delta$-Adem relation writes $\delta_i\delta_j$ as a sum of $\delta$-admissible composites. Moreover,  $\minDimDelta(i,j)\geq \minDimDelta(i+j-s,s)$.
%\end{lem}
%\begin{proof}
%The only tricky inequality is $\minDimDelta(i,j)\geq \minDimDelta(i+j-s,s)$. Now the right hand side must equal $i+j-2s$, and the left hand side is at least $j$. The result follows, since $2s\geq i+1$.
%\end{proof}
\begin{lem}\label{lemOnAdemChangeInMDeltaPlain}
If $\produces{I}{J}{\deltaalg}$, then $\minDimDelta(I)\geq\minDimDelta(J)$.
\end{lem}
\begin{proof}
It is enough to show this result when $I$ and $J$ are disjoint and have length two, in light of the evident algorithm for expressing $\delta_I$ in terms of admissible composites. In the length two case it can be checked directly from the format of the $\delta$-Adem relation, and the inequality is in fact strict (unless $I$ is itself $\delta$-admissible).
\end{proof}
%\noindent Thus, even though $\pi_*$ is not a $\deltaalg$-module, any composite $\delta_I:\pi_nA\to \pi_mA$ that is defined may be written as a sum
Finally, one should note that these operations generate all of the operations in the category $S(\CommOperad)\PiAlg$, and that all of the relations between the operations in $S(\CommOperad)\PiAlg$ are implied by those presented here. Goerss \cite[\S2]{MR1089001} presents this information as follows. First, he observes that there is a simple \emph{Hilton-Milnor theorem} available:
\begin{prop}
Suppose that $A_1$ and $A_2$ are models in $sS(\CommOperad)$. Then $\pi_*(A_1\sqcup A_2)$, which is the coproduct of $\pi_*A_1$ and $\pi_*A_2$ in $S(\CommOperad)\PiAlg$, may be calculated as the non-unital commutative algebra coproduct of $\pi_*A_1$ and $\pi_*A_2$.
\end{prop}
%This may be read as a theorem about coproducts of free $S(\CommOperad)$-$\Pi$-algebras, since $\pi_*(A_1\sqcup A_2)$ is the coproduct in $S(\CommOperad)\PiAlg$ of $\pi_*(A_1)$ and $\pi_*(A_2)$, the theorem being that coproducts of free $S(\CommOperad)$-$\Pi$-algebras may be taken in the category $S(\CommOperad)$.
\noindent Thus, after giving the calculation on a single sphere, the homotopy of finite models is determined, and thus the structure of $S(\CommOperad)\PiAlg$ is well understood:
\begin{prop}[{\cite[Proposition 2.7]{MR1089001}}]
\noindent For $n\geq0$, let $\imath_n$ be the fundamental class in $\pi_n(S(\CommOperad)\mathbb{S}_n)$. There  are isomorphisms of non-unital commutative algebras:
\begin{alignat*}{2}
\pi_*(S(\CommOperad)\mathbb{S}_0)
&\cong
S(\CommOperad)[\imath_0];%
\\
\pi_*(S(\CommOperad)\mathbb{S}_n)
&\cong
\Lambda(\CommOperad)[\delta_I(\imath_n)\ |\ \textup{$J$ is $\Sq$-admissible, $e(I)\leq n$}]%
&\ \ &\text{for $n\geq1$};\\
% Left hand side
\pi_*(S(\CommOperad)\mathbb{S}_n)
% Relation
&\cong
% Right hand side
\Gamma(\CommOperad)[\delta_I(\imath_n)\ |\ \textup{$J$ is $\Sq$-admissible, $e(I)< n$}]%
% Comment
&\ \ &\text{for $n\geq2$.}
\end{alignat*}
\end{prop}

\subsection{Homotopy operations for simplicial Lie algebras}\label{Homotopy operations for simplicial Lie algebras}
Suppose that $L\in s \Lambda(\LieOperad)$ is a simplicial Lie algebra with bracket $[\,,]:\Lambda^2L\to L$. There are unary operations:
\[\lambda_i:=\pi_*([\,,])\circ\lambda_i^\textup{ext}:\pi_nV\to \pi_{n+i}V,\textup{ defined when }1\leq i\leq n,\]
which we write on the right as $x\mapsto x\lambda_i$, and a bracket
\[[\,,]:=\pi_*([\,,])\circ\widetilde{\nabla}:\Lambda^2(\pi_*V)\to \pi_{*}V.\]

Alternatively, one can suppose that $L\in s \Gamma(\LieOperad)$ is a simplicial restricted Lie algebra with bracket $[\,,]:S^2L\to L$, and construct operations:
\begin{gather*}
\lambda_i:=\pi_*([\,,])\circ\sigma_i^\textup{ext}:\pi_nV\to \pi_{n+i}V,\textup{ defined when }0\leq i\leq n\textup{, and}\\
[\,,]:=\pi_*([\,,])\circ\widetilde{\nabla}:S^2(\pi_*V)\to \pi_{*}V.
\end{gather*}
\begin{prop}[????????]\label{omnibus on htpy of Lie algs}
For $L\in s\Lambda(\LieOperad)$, these operations satisfy:
\begin{enumerate}
\item \label{meep1} the bracket gives $\pi_*L$ the structure of a Lie algebra;
\item \label{meep2} the ideal $\bigoplus_{n\geq1}\pi_nL$ is a restricted Lie algebra, with restriction given by the \emph{top} $\delta$-operation, i.e.\ $\restn{x}=\lambda_nx\text{ for }x\in\pi_nL$;
\item \label{meep3} the \emph{non-top} operations, $\lambda_i:\pi_nL\to \pi_{n+i}L$ for $1\leq i<n$, are linear;
\item \label{meep4} there holds the following \emph{Cartan formula}, for $x\in \pi_*L$, $y\in \pi_nL$ and $1\leq i\leq n$:
\[[x,y\lambda_i]=\begin{cases}
[y,[x,y]],&\text{if }i=n;\\
0,&\text{otherwise};
\end{cases}
\]
\item \label{meep5} the \emph{$\lambda$-Adem relations} hold: if $x\lambda_j\lambda_i$ is defined, and $i>2j$, then
\[x\lambda_j\lambda_i=\sum_{k=0}^{(i-2j)/2-1}{i-2j-2-k\choose k}x\lambda_{i-j-1-k}\lambda_{2j+1+k}.\]
\end{enumerate}
For $L\in s\Gamma(\LieOperad)$, we may omit (\ref{meep1}), modify (\ref{meep2}) to state that the whole of $\pi_*L$ is restricted, and modify (\ref{meep3})-(\ref{meep5}) to include $\lambda_0$.
\end{prop}
Similar comments apply as for commutative algebras, for example, one needs lemma \ref{lemOnAdemChangeInMLambdaPlain} (to follow) to understand why this unstable relation makes sense. 

Now we will view the well known \emph{$\Lambda$-algebra} as the non-unital associative algebra generated by $\lambda_i$ for $i\geq0$, subject to relations \[\lambda_j\lambda_i=\sum_{k=0}^{(i-2j)/2-1}{i-2j-2-k\choose k}\lambda_{i-j-1-k}\lambda_{2j+1+k}\textup{\ for $i>2j$.}\]
We say that a sequence $I=(i_\ell,\ldots,i_1)$ of non-negative integers is \emph{$\lambda$-admissible} if $i_{j+1}\leq 2i_j$ for $1\leq j <\ell$. 
For any sequence $I=(i_\ell,\ldots,i_1)$, if we write $\lambda_I$ for the element $\lambda_{i_1}\cdots \lambda_{i_\ell}$ in $\Lambda$, then the $\Lambda$-algebra has the evident admissible basis, and we may make sense of the symbol $\produces{I}{J}{\Lambda}$. Note that the ordering of the generators in $\lambda_I$ and $\delta_I$ are opposite, for consistency with the fact that we write the $\lambda$-operations on $\pi_*L$ \emph{on the right}. Thus, we may think of $\lambda_I$ as the composite operator  
\[\pi_{n}L\overset{\lambda_{i_1}}{\to}\pi_{n+i_1}L\overset{\lambda_{i_2}}{\to}\cdots \overset{\lambda_{i_\ell}}{\to}\pi_{n+i_1+\cdots +i_\ell}L,\]
again defined only when $\minDimDelta(I)\leq n$, so that $\pi_*L$ is \emph{not} a right module over $\Lambda$.
Note that when $I$ is a non-empty \emph{$\lambda$-admissible} sequence,
$\minDimDelta(I)=i_1$, not the Serre excess, reflecting the observation that when $x\lambda_{i_1}\cdots \lambda_{i_\ell}$ is a $\lambda$-admissible composite, the only operation that  can be a top (i.e.\ restriction) operation is $\lambda_{i_1}$.
The following lemma assures us that the $\lambda$-Adem relations make sense in \emph{(\ref{meep5})}.
%\begin{lem}\label{lemOnAdemChangeInMDeltaPlain}
%Suppose that $i,j\geq2$ and $i<2j$, and $(i+1)/2\leq s\leq (i+j)/3$. Then $i+j-s\geq 2s$, $i+j-s\geq2$, $s\geq2$, so that the $\delta$-Adem relation writes $\delta_i\delta_j$ as a sum of $\delta$-admissible composites. Moreover,  $\minDimDelta(i,j)\geq \minDimDelta(i+j-s,s)$.
%\end{lem}
%\begin{proof}
%The only tricky inequality is $\minDimDelta(i,j)\geq \minDimDelta(i+j-s,s)$. Now the right hand side must equal $i+j-2s$, and the left hand side is at least $j$. The result follows, since $2s\geq i+1$.
%\end{proof}
\begin{lem}\label{lemOnAdemChangeInMLambdaPlain}
If $\produces{I}{J}{\Lambda}$, then $\minDimDelta(I)\geq\minDimDelta(J)$, and $J$ does not contain zero unless $I$ does.
\end{lem}
These operations generate all of the operations in each of the categories $\Lambda(\LieOperad)\PiAlg$ and $\Gamma(\LieOperad)\PiAlg$, and the relations presented here are sufficient. Indeed, although we don't have a Hilton-Milnor theorem available, from \cite[Theorem 8.8 and proof]{CurtisSimplicialHtpy.pdf} (and the original work, \cite{6Author.pdf}) we gather
\begin{prop}
{\textup{PRLieAlgs,Prop3.3}}... Basically, the free functor in $\Gamma(\LieOperad)\PiAlg$ takes the free restricted Lie algebra and attaches non-top $\lambda$ operations on the right.
\end{prop}

\section{\textbf{Constructing cohomology operations}}\label{sec:Constructing cohomology operations}

\subsection{Higher simplicial Alexander-Whitney map}
Let $\{D_k\}$ be a \emph{special simplicial Alexander-Whitney map},  i.e.\  maps
\[D_k:N(U\otimes V)_{n}\to (CU\otimes CV)_{n+k}\textup{\ \ defined for $n,k\geq0$},\]
natural in simplicial vectorspaces $U$ and $V$, such that for $k\geq1$, the identity
\[dD_k+D_kd=(1+\twist) D_{k-1}
\]
holds on classes of any dimension, and:
\begin{itemize}
\setlength{\parindent}{.25in}
\item $D_0$ is the Alexander-Whitney map, a chain homotopy equivalence inducing the identity in dimension zero;
\item the image  of $D_k:C_n(U\otimes V)\to (CU\otimes CV)_{n+k}$ is contained in the subspace
$\textstyle\bigoplus_{i,j\leq n}(C_iU\otimes C_jV)$, and is thus zero unless $k\leq n$; and
\item $D_k$ maps $C_{k}(U\otimes V)$ identically onto $C_{k}U\otimes C_{k}V$.
\end{itemize}
Such maps are described in detail, under the name \emph{special cup-$k$ product}, by Singer in \cite[Definitions 1.91 and 1.94]{MR2245560}, and developed originally in \cite{DoldUber}.
It is \textbf{??? prob not ???} a natural convention to define $D_k=0$ for whenever either $k<0$ or $i<0$, in which case the relation $dD_k+D_kd=D_{k-1}+TD_{k-1}T$ holds for any $k$.

\begin{shaded}
\textbf{Move, maybe.}
Let $\{D^k\}$ be a special dual Alexander-Whitney map \cite[Proposition 5.2]{turner_opns_and_sseqs_I.pdf}, i.e.\  maps
\[D^k:(CR\otimes CS)_{-i-k}\to N(R\otimes S)_{-i}\textup{\ \ for $i,k\geq0$},\]
natural in cosimplicial vectorspaces $R,S$,
with the properties:
\begin{itemize}
\setlength{\parindent}{.25in}
\item $dD^k+D^kd=D^{k-1}+TD^{k-1}T$ for $k\geq1$;
\item $D^0$ is a chain homotopy equivalence inducing the identity in dimension zero;
\item the restriction of $D^k$ to $C_{-i}R\otimes C_{-j}S$ is zero unless $i\geq k$ and $j\geq k$; and
\item $D^k$ maps $C_{-k}R\otimes C_{-k}S$ identically onto $C_{-k}(R\otimes S)$.
\end{itemize}
It is a natural convention to define $D^k=0$ for whenever either $k<0$ or $i<0$, in which case the relation $dD^k+D^kd=D^{k-1}+TD^{k-1}T$ holds for any $k$.
\end{shaded}

\subsection{External unary cohomotopy operations}
In this section we recall the definition of certain homotopy operations with domain $(\pi_*V)^*$ for any $V\in s\vect{}{}$, \textbf{reference}, using the function
\[D^*_{n-i}\oldphi(\alpha\otimes \alpha)+D^*_{n-i+1}\oldphi(\alpha\otimes d\alpha),\ \ (N_nV)^* \to (N_{n+i}(S^2V))^*,\]
where we regard the $D^*_k$ as maps into $(N_*(S^2V))^*$, and the map $\oldphi$ is that described in [Singer \S1.9]. Namely, [\textbf{potentially redundant:} as the normalized complex is naturally a retract of the unnormalized complex, we may define] $\oldphi:\left(N_mV\right)^*\otimes \left(N_mV\right)^*\to \left(N_m(V\otimes V)\right)^*$ to be
\[\left(C_mV\overset{\alpha}{\to}\F_2\right)\otimes \left(C_mV\overset{\beta}{\to}\F_2\right)\mapsto\left(C_m(V\otimes V)\overset{\alpha\otimes\beta}{\to}\F_2\otimes \F_2\overset{\textup{mult}}{\to}\F_2\right).\]
\begin{prop}[same refs, I guess] \label{extUnaryCohomotOps}
These functions descend to  well defined cohomotopy operations:
\[\ExtCohOp^k:(\pi_{n}V)^*\to (\pi_{n+k}(S^2V))^*,\text{ \ zero unless $0\leq k\leq n$.}\]
\end{prop}
\begin{proof}
One checks that these cochain operations induce well defined operations $\ExtCohProd$ and $\ExtCohOp^k$ by the same arguments as in \cite[\S1.12]{MR2245560}.
\end{proof}

\subsection{Linearly dual homotopy operations}
The linear maps $\ExtCohOp^k$ induce dual operators
\[(\ExtCohOp^k)^*:\pi_*(S^2V)\to \pi_{*-k}(V)\]
whenever these groups are locally finite dimensional. Following \cite[\S3]{MR1089001}, one can define the operations $(\ExtCohOp^k)^*$ directly, without the double dualization as described below. %Setting $V=Q^{\calC}B^{\calC}X$ and precomposing with the map $(\psi_\calC)_*:\pi_*(V)\to\pi_{*-1}(S^2V)$ yields homology operations on $H_*^\calC$ whose duals are the cohomology operations of \S\ref{generic coh ops section}.
Goerss \cite[Proposition 3.7]{MR1089001}
%If $\sigma_i$ and $\pi_*(1+T)$ are the maps of \cite[\S3]{MR1089001}, then 
shows that any element of $x\in \pi_m(S^2V)$ can be written as a sum
\[x=\sum_j\pi_*(1+T)(y_j\otimes z_j)+\sum_{\smash{0\leq k\leq \lfloor m/2\rfloor}}\sigma_k(w_k),\]
where $w_k\in \pi_{m-k}(V)$ for $0\leq k\leq\lfloor m/2\rfloor$, and $y_j,z_j\in \pi_{*}(V)$, and that we may \emph{define}:
\[(\ExtCohOp^k)^*(x)=w_k.\]
That is, we are defining a map $(\ExtCohOp^k)^*$ by this requirement, and the method of \cite[\S3]{MR1089001} shows that its linear dual, $((\ExtCohOp^k)^*)^*$ equals $\ExtCohOp^k$. Of course, the chosen \emph{symbol} $(\ExtCohOp^k)^*$ is then only appropriate in the finite-dimensional case, but this definition of a map $(\ExtCohOp^k)^*$ works in any setting.

One might only need to determine the operations $(\ExtCohOp^k)^*$ for $k<m/2$, so that when $m$ is even we may ignore the dual of the top operation, $(\ExtCohOp^{m/2})^*$. In this case, it is more convenient to rewrite the above equation as follows: 
\[x=\widetilde{\nabla}(v)+\sum_{\smash{0\leq k< m/2}}\sigma_k(w_k),\]
where $v\in(S^2(\pi_*(V)))_{m}$ and $w_k\in \pi_{m-k}(V)$ for $0\leq k<m/2$.


\subsection{External binary cohomotopy operations}
Consider the cochain level pairing
\[\alpha\otimes\beta\mapsto D_0^*\oldphi(\alpha\otimes\beta),\quad(N_*V)^*\otimes (N_*V)^*\to (N_*(S^2V))^*.\]
\begin{prop}\label{the external cohomotopy pairing}
This pairing induces a pairing on cohomotopy,
\[\ExtCohProd:S_2((\pi_*V)^*)\to (\pi_*(S^2V))^*,\]
with the property that $\ExtCohProd(\alpha\otimes \alpha)=\ExtCohOp^k\alpha$ for $\alpha\in(\pi_kV)^*$.
\end{prop}



\subsection*{stuff from dead chapter}
\textbf{\tiny Typical $\calC$-cohomology operations}\label{generic coh ops section}
We will use $\psi_\calC$ to define operations on $\calC$-cohomology, employing 
%\begin{shaded}\tiny
%certain generic operations, defined for any $U\in s\calV$:
%\begin{gather*}
%\ExtCohProd:\pi^{n_1}(U^*)\otimes \pi^{n_2}(U^*)\to \pi^{n_1+n_2}((S^2U)^*),\textup{ and}\\
%\ExtCohOp^k:\pi^{n}(U^*)\to \pi^{n+k}((S^2U)^*),\textup{ (zero unless $0\leq k\leq n$).}
%\end{gather*}
%These operations are defined as follows. In all that follows, let $\{D_k\}$ be a special cup-$k$ product [Singer \S1.9]. As the normalized complex is naturally a retract of the unnormalized complex, we may define $\oldphi:\left(N_mU\right)^*\otimes \left(N_mU\right)^*\to \left(N_m(U\otimes U)\right)^*$ using the map
%\[\left(C_mU\overset{\alpha}{\to}\F_2\right)\otimes \left(C_mU\overset{\beta}{\to}\F_2\right)\mapsto\left(C_m(U\otimes U)\overset{\alpha\otimes\beta}{\to}\F_2\otimes \F_2\overset{\textup{mult}}{\to}\F_2\right).\]
%Then define a cochain map:
%\[m_\textup{ext}:(N_*U)^*\otimes (N_*U)^*\to (N_*(S^2U))^*\]
%and functions
%\[S^k_\textup{ext}:(N_mU)^*\to (N_{m+k}(S^2U))^*\]
%by the formulae, for $\alpha\in (N_{n_1}U)^*$ and $\beta\in (N_{n_2}U)^*$
%\[m_\textup{ext}(\alpha\otimes\beta)=D_0^*\oldphi(\alpha\otimes\beta),\quad S^k_\textup{ext}(\alpha)=D^*_{n_1-k}\oldphi(\alpha\otimes \alpha)+D^*_{n_1-k+1}\oldphi(\alpha\otimes d\alpha),\]
%where we regard the $D^*_k$ as maps into $(N_*(S^2U))^*$ by composition with the projection from $(N_*(U\otimes U))^*$. One checks that these operations induce well defined operations $\ExtCohProd$ and $\ExtCohOp^k$ by the same arguments as in \cite[\S1.12]{MR2245560}.
%\end{shaded}

Given these external operations, we can define $\calC$-cohomology operations, natural in $X\in s\calC$, on $H^*_\calC X=\pi^*((Q^\calC B^\calC X)^*)$, by the composites:
\begin{gather*}
H_\calC^{n_1}(X)\otimes H_\calC^{n_2}(X)\overset{\ExtCohProd}{\to} \pi^{n_1+n_2}((S^2(Q^\calC B^\calC X))^*)\overset{\psi_\calC^*}{\to} H_\calC^{n_1+n_2+1}(X),\\
H_\calC^{n}(X)\overset{\ExtCohOp^k}{\to} \pi^{n+k}((S^2(Q^\calC B^\calC X))^*)\overset{\psi_\calC^*}{\to} H_\calC^{n+k+1}(X).
\end{gather*}
\begin{shaded}\tiny\textbf{the first paragraph here is still relevant}

\textbf{\tiny Linearly dual $\calC$-homology operations}
The linear maps $\ExtCohOp^k$ induce dual operators
\[(\ExtCohOp^k)^*:\pi_*(S^2U)\to \pi_{*-k}(U)\]
whenever these groups are locally finite dimensional. Following \cite[\S3]{MR1089001}, one can define the operations $(\ExtCohOp^k)^*$ directly, without the double dualization as described below. Setting $U=Q^{\calC}B^{\calC}X$ and precomposing with the map $(\psi_\calC)_*:\pi_*(U)\to\pi_{*-1}(S^2U)$ yields homology operations on $H_*^\calC$ whose duals are the cohomology operations of \S\ref{generic coh ops section}.

%If $\sigma_i$ and $\pi_*(1+T)$ are the maps of \cite[\S3]{MR1089001}, then any element of $x\in \pi_m(S^2U)$ can be written as a sum
%\[x=\sum_j\pi_*(1+T)(y_j\otimes z_j)+\sum_{\smash{0\leq k\leq \lfloor m/2\rfloor}}\sigma_k(w_k),\]
%where $w_k\in \pi_{m-k}(U)$ ($0\leq k\leq\lfloor m/2\rfloor$), and $y_j,z_j\in \pi_{*}(U)$.
%Then the map $(\ExtCohOp^k)^*$ sends $x$ to this $w_k$. Alternatively, if we only wish to determine the $(\ExtCohOp^k)^*$ for $k<m/2$ (and not the dual of the top operation $(\ExtCohOp^{m/2})^*$ when $m$ is even), we can rewrite this equation: 
%\[x=\widetilde{\nabla}(v)+\sum_{\smash{0\leq k< m/2}}\sigma_k(w_k),\]
%where $v\in(S^2(\pi_*(U)))_{m}$ and $w_k\in \pi_{m-k}(U)$ ($0\leq k<m/2$). Here, $\widetilde{\nabla}:S^2\pi_*V\to \pi_*S^2V$ is the map that we discuss in \S\ref{the top homotopy operations for Lie algebras}.
\end{shaded}

\subsection{Chain level structure for cohomology operations}
\hfil
\begin{shaded}
\begin{enumerate}
\item frame the $j$ by analogy with the pairing $S_2X\to X$
\item examples - explain Lie stuff's Koszul duality in terms of Goerss' work
\end{enumerate}
\end{shaded}


We'll now give generalisations of Goerss' constructions in \cite[\S5]{MR1089001} to yield useful structure on the complexes calculating cohomology.  Suppose that $X\in s\calC$ is almost free, with $V_s\subset X_s$ the freely generating subspace. Then for each $s$, the functor 
\[\Hom_\calC(X_s,\DASH)\cong \Hom_\calV(V_s,U_\calC\DASH)\]
is naturally an $\F_2$-vector space. Writing $\phi_s=\F_\calC(\Delta):X_s\to X_s\sqcup X_s$, the addition operation `$\star$' on $\Hom_\calC(X_s,\DASH)$ is given by $f\star g= (f\sqcup g)\circ\phi_s$. Now let
$\overline{\xi}_\calC=((d_0\sqcup d_0)\phi_s)\star(\phi_{s-1}d_0)$, the sum of maps in the $\F_2$-vector space $\textup{Hom}_\calC(X_s,X_{s-1}\sqcup X_{s-1})$. It is completely formal to check that $\overline{\xi}_\calC$ maps to zero in the group
\[\textup{Hom}_\calC(X_s,X_{s-1}\times X_{s-1})=\textup{Hom}_\calC(X_s,X_{s-1})\times \textup{Hom}_\calC(X_s,X_{s-1}),\]
and thus $\overline{\xi}_\calC$ factors through a unique map $\xi_\calC:X_s\to X_{s-1}\wedge X_{s-1}$. Furthermore, $\xi_\calC$ enjoys the symmetry $\tau\xi_\calC=\xi_\calC$, and it is again formal to verify the analogue of \cite[Lemma 5.5]{MR1089001}:
\begin{lem}\label{psi is basically the quadratic part}
The map $Q^\calC\xi_\calC$ induces a chain map of degree $-1$ on normalized complexes:
\[N_s(Q^{\calC}X)\to N_{s-1}((Q^{\calC}(X\wedge X))^{\Sigma_2}).\]
The composite
\[\psi_\calC:\left(N_s(Q^{\calC}X)\overset{Q^\calC\xi_\calC}{\to} N_{s-1}((Q^{\calC}(X\wedge X))^{\Sigma_2})\overset{j_\calC}{\to} N_{s-1}(S^2(Q^{\calC}X))\right),\]
is essentially $\quadratic_\calC$, in that if $v\in V_s\cap N_sX$ represents an element of $N_sQ^\calC X$, writing $d_0v=f(w_j)$ for $w_j\in V_{s-1}$, we have $\psi_\calC(v)=\quadratic_\calC(f)(w_j)\in S^2(V_{s-1})$.
\end{lem}
We may reproduce this result somewhat more generally:
\begin{prop}\label{general CohOpns given irreducibility}
Suppose that $\theta:F_\calC V\to GV$ is a natural transformation from $F_\calC$ to another endofunctor $G$ of $\vect{}{}$. Suppose further that $\theta$ satisfies the condition:
\[\smash{\theta\circ\mu_V=\theta\circ \epsilon_{F_\calC V} +\theta\circ {F_\calC \epsilon_V}:F_\calC F_\calC V\to G V.}\]
Write $\smash{\widetilde{\theta}}:Q^{\calC}X_s\to G(Q^{\calC}X_{s-1})$ for the following composite, depending on the almost free structure chosen:
\[\smash{Q^{\calC}X_s\overset{\cong}{\to} V_s\overset{d_0}{\to}F_{\calC}V_{s-1}\overset{\theta}{\to}GV_{s-1}\overset{\cong}{\to}G(Q^{\calC}X_{s-1})}\]
Then $d_0\circ\widetilde{\theta}=\widetilde{\theta}\circ(d_0+d_1)$, and $d_j\circ\widetilde{\theta}=\widetilde{\theta}\circ d_{j+1}$ for $j\geq1$, so that $\widetilde{\theta}$ restricts to a degree $-1$ chain map $N_sQ^\calC X\to N_{s-1}Q^\calC X$.
%For $v\in V_s$, write $d_0v=g(v_i)$ as an expression in certain $v_i\in V_{s-1}$. Then $\widetilde{\theta}(v):=\theta(g)(v_i)$. Here, we view the $g(v_i)$ as lying in $F_\calC V_{s-1}\cong X_{s-1}$, so that $\theta(g(v_i))\in G V_{s-1}\cong G(Q^\calC X_{s-1})$. Then $\widetilde{\theta}$ is a chain map, even descending to the subspace $N_*Q^\calC X$.
\end{prop}
\begin{proof}
In order to see that $d_j\circ\widetilde{\theta}=\widetilde{\theta}\circ d_{j+1}$ for $j\geq1$, we examine the diagram
\[{\xymatrix@R=4mm{
V_s\ar[r]^-{d_0}
\ar@{..>}[d]^-{d_{j+1}}
&%r1c1
F_\calC V_{s-1}\ar[r]^-{\theta}
\ar[d]^-{d_j}
&GV_{s-1}\ar[r]^-{\cong}
\ar@{..>}[d]^-{d_j}
&%r2c2
GQ^{\calC}X_{s-1}\ar[d]^-{GQ^{\calC}(d_j)}
\\%r1c3
V_{s-1}\ar[r]^-{d_0}
&%r2c1
F_\calC V_{s-2}\ar[r]^-{\theta}
&GV_{s-2}\ar[r]^-{\cong}
&%r2c2
GQ^{\calC}X_{s-2}
}}\]
The dotted verticals are available since $X$ is almost free. The left square commutes by the simplicial identities, and the center square commutes by naturality of $\theta$. For the other relationship, we use the following diagram, which commutes except for the leftmost square:
\[\xymatrix@R=4mm{
V_s\ar[r]^-{d_0}
\ar@{..>}[d]_-{d_{1}+\epsilon\circ d_0}
&%r1c1
F_\calC V_{s-1}\ar[r]^-{\theta}
\ar[d]^-{F_{\calC}(\epsilon\circ d_0)}
&GV_{s-1}\ar[r]^-{\cong}
\ar@{..>}[d]^-{G(\epsilon\circ d_0)}
&%r2c2
GQ^{\calC}X_{s-1}\ar[d]^-{GQ^{\calC}(d_0)}
\\%r1c3
V_{s-1}\ar[r]^-{d_0}
&%r2c1
F_\calC V_{s-2}\ar[r]^-{\theta}
&GV_{s-2}\ar[r]^-{\cong}
&%r2c2
GQ^{\calC}X_{s-2}
}\]
Now the rightmost two squares both commute, so to show that the outer rectangle commutes, it is enough to see that the two composites $V_{s}\to F_{\calc}V_{s-2}$ are coequalized by $\theta$. Using the simplicial identity $d_0 d_1=d_0d_0$, we are trying to show that $\theta d_0d_0+\theta d_0\epsilon d_0$ and $\theta F_{\calC}(\epsilon d_0) d_0$ are the same map from $V_s$ to $GV_{s-2}$. Even more, we will show that $\theta d_0+\theta d_0\epsilon$ and $\theta F_{\calC}(\epsilon d_0)$ are the same map $F_\calC V_{s-1}$ to $GV_{s-2}$.

Starting with an expression $f(v_i)$ with $v_i\in V_{s-1}$, we calculate $\theta d_0 f(v_i)=\theta (fd_0v_i)$, $\theta d_0\epsilon f(v_i)=\theta (\epsilon (f)(d_0v_i))$ and $\theta F_{\calC}(\epsilon d_0)(f(v_i))=\theta(f)(\epsilon(d_0v_i))$. That these three terms add to zero was the requirement specified for $\theta$.
\end{proof}



\subsection{Cohomology operations for simplicial commutative $\F_2$-algebras}\label{The example of simplicial commutative F2-algebras}
Goerss \cite[\S5]{MR1089001} defines cohomology operations, natural in $A\in s S(\CommOperad)$:
\begin{gather*}
P^i=\psi^*_{S(\CommOperad)}\circ\ExtCohOp^i:H^n_{S(\CommOperad)}A\to H_{S(\CommOperad)}^{n+i+1}A;\textup{ and}\\
[\,,]=\psi^*_{S(\CommOperad)}\circ\ExtCohProd :H_{S(\CommOperad)}^nA\otimes H_{S(\CommOperad)}^mA\to H_{S(\CommOperad)}^{n+m+1}A.
\end{gather*}
He also defines a natural operation $\beta:H_{S(\CommOperad)}^0A\to H_{S(\CommOperad)}^1A$. Note that as a result of our use of $\psi^*_{S(\CommOperad)}$, these operations have a grading shift.
\begin{prop}[{\cite[\S5]{MR1089001}}]\label{omnibus on coh of simp algs}These operations have the following properties:
\begin{enumerate}
\item the bracket gives $H^*_{S(\CommOperad)}A$ the structure of an $S(\LieOperad)$-algebra (with grading shift);
\item the operation $\beta$ acts as a restriction defined only in dimension zero, so that for $x,y\in H^0_{S(\CommOperad)}A$ and $z\in H^*_{S(\CommOperad)}A$:
\[\beta(x+y)=\beta(x)+\beta(y)=[x,y],\text{\ \ and \ }[\beta(x),z]=[x,[x,z]];\]
\item the self-square operation on $H^*_{S(\CommOperad)}A$ equals the top $P$-operation:\[P^nx=[x,x]\text{\ \ for $x\in H^n_{S(\CommOperad)}A$};\]
\item \label{P unstable vanishing} if $x\in H^n_{S(\CommOperad)}A$, then $P^ix=0$ unless $2\leq i\leq n$;
\item every $P$-operation is linear;
\item there holds the following \emph{Cartan formula}:  for all $x,y\in   H^n_{S(\CommOperad)}A$ and $i\geq0$,
\[[x,P^iy]=0\]
\item \label{yeah H is a Pmodule}the \emph{$P$-Adem relations} hold: if $i\geq 2j$, then
\[P^iP^jx=\sum_{s=i-j+1}^{i+j-2}{2s-i-1\choose s-j}P^{i+j-s}P^sx.\]
\end{enumerate}
\end{prop}
In this case, \emph{(\ref{yeah H is a Pmodule})} does state that $H^*_{S(\CommOperad)}A$ is a left module over $\Palg$, the \emph{Steenrod algebra for commutative algebras over $\F_2$}, the non-unital associative algebra generated by symbols $P^i$ for $i\geq0$, modulo the two sided ideal generated by $P^0$, $P^1$, and the evident \emph{$P$-Adem relations}.


A sequence $I=(i_\ell,\ldots,i_1)$ of integers $i_j\geq2$ is \emph{$P$-admissible} if $i_{j+1}<2i_j$ for $1\leq j <\ell$. For any sequence $I=(i_\ell,\ldots,i_1)$, write $P^I$ for the composite $P^{i_\ell}\cdots P^{i_1}$. It follows from \cite[Theorem I]{MR1089001} that $\Palg$ has an \emph{admissible basis}, consisting of those $P^I=P^{i_\ell}\cdots P^{i_{1}}$ with $I$ a $P$-admissible sequence.  Again, we will say that \emph{$I$ produces $J$ in $\Palg$}, denoted $\produces{I}{J}{\Palg}$ if, when $P_I$ is written in the $P$-admissible basis of $\Palg$, $P_J$ appears with non-zero coefficient. In this case, $J$ must be $P$-admissible, and $I$ must be $P$-inadmissible unless $J=I$.

We define
\[\minDimP(I):=\max\{(i_1),\,(i_2-i_1-1),\,(i_3-i_2-i_1-2),\,\ldots,\,(i_{\ell}-\cdots-i_1-\ell+1)\},\]
for a rather different purpose than in \S\ref{Homotopy operations for simplicial commutative algebras} and \S\ref{Homotopy operations for simplicial Lie algebras}: the composite 
\[P^I:\bigl(H_{S(\CommOperad)}^{n}A\overset{P^{i_1}}{\to}H_{S(\CommOperad)}^{n+i_1+1}A\overset{P^{i_2}}{\to}\cdots \overset{P^{i_\ell}}{\to}H_{S(\CommOperad)}^{n+i_1+\cdots +i_\ell+\ell}A\bigr)\]
is \emph{forced to be zero} by \emph{(\ref{P unstable vanishing})} alone precisely when $n<\minDimDelta(I)$.

Unlike in \S\ref{Homotopy operations for simplicial commutative algebras} and \S\ref{Homotopy operations for simplicial Lie algebras}, having a non-empty sequence $I$  be \emph{$P$-admissible} does not allow us identify which term is largest in the maximum defining $\minDimP(I)$. More explicitly, in an admissible expression $P^{i_\ell}\cdots P^{i_1}x$, for $x\in H^n_{S(\CommOperad)}A$, top $P$-operations may be interspersed with non-top $P$-operations.
%\begin{lem}\label{lemOnAdemChangeInMDeltaPlain}
%Suppose that $i,j\geq2$ and $i<2j$, and $(i+1)/2\leq s\leq (i+j)/3$. Then $i+j-s\geq 2s$, $i+j-s\geq2$, $s\geq2$, so that the $\delta$-Adem relation writes $\delta_i\delta_j$ as a sum of $\delta$-admissible composites. Moreover,  $\minDimDelta(i,j)\geq \minDimDelta(i+j-s,s)$.
%\end{lem}
%\begin{proof}
%The only tricky inequality is $\minDimDelta(i,j)\geq \minDimDelta(i+j-s,s)$. Now the right hand side must equal $i+j-2s$, and the left hand side is at least $j$. The result follows, since $2s\geq i+1$.
%\end{proof}

The following result shows that whenever an expression $P^Jx$ is forced to be zero by \emph{(\ref{P unstable vanishing})}  and we reduce said expression to a sum of $P$-admissible composites, all of the summands are forced to be zero by \emph{(\ref{P unstable vanishing})}.
\begin{lem}\label{lemOnAdemChangeInMP}
If $\produces{I}{J}{\Palg}$, then $\minDimP(J) \geq \minDimP(I)$, with strict inequality when $I$ and $J$ are distinct and of length two.
\end{lem}


\textbf{messed up from here:} Finally, one should note that these operations generate all of the operations in the category $S(\CommOperad)\PiAlg$, and that all of the relations between the operations in $S(\CommOperad)\PiAlg$ are implied by those presented here. Goerss \cite[\S2]{MR1089001} presents this information as follows. First, he observes that there is a simple \emph{Hilton-Milnor theorem} available:
\begin{prop}
Suppose that $A_1$ and $A_2$ are GEMs in $sS(\CommOperad)$. Then $H^*_{S(\CommOperad)}(A_1\times A_2)$, which is the coproduct of $H^*_{S(\CommOperad)}A_1$ and $H^*_{S(\CommOperad)}A_2$ in $S(\CommOperad)\PiAlg$, may be calculated as the non-unital commutative algebra coproduct of $H^*_{S(\CommOperad)}A_1$ and $H^*_{S(\CommOperad)}A_2$.
\end{prop}
%This may be read as a theorem about coproducts of free $S(\CommOperad)$-$\Pi$-algebras, since $\pi_*(A_1\sqcup A_2)$ is the coproduct in $S(\CommOperad)\PiAlg$ of $\pi_*(A_1)$ and $\pi_*(A_2)$, the theorem being that coproducts of free $S(\CommOperad)$-$\Pi$-algebras may be taken in the category $S(\CommOperad)$.
\noindent Thus, after giving the calculation on a single sphere, the homotopy of finite models is determined, and thus the structure of $S(\CommOperad)\PiAlg$ is well understood:
\begin{prop}[{\cite[Proposition 2.7]{MR1089001}}]
\noindent For $n\geq0$, let $\imath_n$ be the fundamental class in $H^n_{S(\CommOperad)}(S(\CommOperad)\mathbb{S}_n)$. There  are isomorphisms of non-unital commutative algebras:
\begin{alignat*}{2}
H^*_{S(\CommOperad)}(S(\CommOperad)\mathbb{S}_0)
&\cong
S(\CommOperad)[\imath_0];%
\\
H^*_{S(\CommOperad)}(S(\CommOperad)\mathbb{S}_n)
&\cong
\Lambda(\CommOperad)[\delta_I(\imath_n)\ |\ \textup{$J$ is $\Sq$-admissible, $e(I)\leq n$}]%
&\ \ &\text{for $n\geq1$};\\
% Left hand side
H^*_{S(\CommOperad)}(S(\CommOperad)\mathbb{S}_n)
% Relation
&\cong
% Right hand side
\Gamma(\CommOperad)[\delta_I(\imath_n)\ |\ \textup{$J$ is $\Sq$-admissible, $e(I)< n$}]%
% Comment
&\ \ &\text{for $n\geq2$.}
\end{alignat*}
\end{prop}




\subsection{Definition of $\calW(0)$}
Now let $\calW(0)$ be the category of connected objects in Goerss' category `$\calW$' \cite[Definition G]{MR1089001}. Explicitly, $\calW(0)$ has objects $\vect{+}{0}$-graded vector spaces $X$, equipped with:
\begin{enumerate}
\item a commutative, bilinear product $[\,,]:X^n\otimes X^m\to X^{n+m+1}$ satisfying the Jacobi identity; and
\item linear maps $P^i:X^n\to X^{n+i+1}$ such that
\begin{enumerate}
\item $P^i=0$ unless $2\leq i\leq n$, and $P^nx=[x,x]$;
\item $[x,P^iy]=0$ for all $x$, $y$ and $i$; and
\item satisfying the $P$-Adem relation:
\[P^iP^j=\sum_{s=i-j+1}^{i+j-2}{2s-i-1\choose s-j}P^{i+j-s}P^s\textup{ whenever $i\geq 2j$.}\]
\end{enumerate}
\end{enumerate}
Then Goerss's results \cite[Theorem H]{MR1089001}, restricted to the full subcategory of connected objects in $sS(\CommOperad)$ (i.e.\ objects $A$ with $\pi_0A=0$, or equivalently $H^0_{S(\CommOperad)}A=0$), state that the operations $[\,,]$ and $P^i$ give $H^*A$ the structure of an element of $\calW(0)$, and moreover, that these are the only natural operations on the cohomology of connected objects of $sS(\CommOperad)$ \cite[Theorem I]{MR1089001}. That is, Goerss shows that the above relations are satisfied by the operations defined, and then calculates the cohomology of an abelian object in $sS(\CommOperad)$; neither of these two steps is straightforward.

Let $\calU(0)$ be the category whose objects are $\vect{+}{0}$-graded vector spaces $V$ equipped with linear left $P$-operations $P^i:V^n\to V^{n+i+1}$, which are zero unless  $2\leq i\leq n$, and satisfy the $P$-Adem relation as above.

Any object $X$ of $\calW(0)$ is in particular an object of $\calU(0)$. It is \emph{not} true, however, that $X$ is a Lie algebra. Indeed, the relations above do not imply that $[x,x]=0$ for any $x\in X$ (see \cite[p.\ 18]{MR1089001}). Rather, $X$ is an algebra over the monad $S(\LieOperad)$ discussed in \S\ref{sec on Lie algs and homotopy ops}. We will describe the construction of the monads $F_{\calW(0)}$ and $F_{\calU(0)}$ in \S\ref{Construction of the monads}.



\end{Constructing homotopy and cohomotopy operations}


\begin{External spectral sequence operations}
\section{\textbf{External spectral sequence operations}}

[\textbf{move me} to appropriate part of spectral sequence section:] {\tiny It is not a natural convention to regard $\DeltaUp^k$ as zero where it is not defined, however, we will perform this abuse shortly for notational convenience. Under this convention we will be able to use the equation
\[(d\DeltaUp^k+\DeltaUp^kd)z=(\oldphi^{k+1} +\DeltaUp^{k+1}+T\DeltaUp^{k+1}T)z\]
whenever $k\geq0$ and the simplicial dimension of $z$ does not equal either of $2k$ and $2k+1$.}


\end{External spectral sequence operations}







\begin{Lie algebras in characteristic 2 and their homotopy operations}

\section{\textbf{Graded Lie algebras in characteristic 2}}\label{sec on Lie algs and homotopy ops}

\subsection{Graded and partially restricted Lie algebras}
[\textbf{Old version} of spiel on PRLie algs commented out here.%
%Let $\LieOperad$ be the Lie operad in characteristic 2. As explained in [Fresse], the monad $S(\LieOperad)$ on $\vect{}{}$ defined by
%\[S(\LieOperad)(V)=\bigoplus_{n\geq1}(\LieOperad(n)\otimes V^{\otimes n})_{\Sigma_n}\]
%does not return the free Lie algebra on $V$ in the traditional sense. Rather, an $S(\LieOperad)$-algebra is a vector space $L$ equipped with a bracket $L\otimes L\to L$ satisfying the Jacobi identity and the antisymmetry condition $[x,y]=-[y,x]$. This condition does \emph{not} imply the alternating condition $[x,x]=0$ (which is impossible to encode operadically). We will refer to structures of this type as $S(\scrL)$-algebras, to distinguish them from Lie algebras, which we demand satisfy the alternating condition.
%
%Fresse constructs a monad $\Gamma(\LieOperad)$ defined by
%\[\Gamma(\LieOperad)(V)=\bigoplus_{n\geq1}(\LieOperad(n)\otimes V^{\otimes n})^{\Sigma_n},\]
%which represents the free restricted [See Curtis or 6A or something] Lie algebra. He further constructs a norm map $\textup{Tr}:S(\LieOperad)\to \Gamma(\LieOperad)$, and defines a monad $\Lambda(\LieOperad)$ by the formula
%\[\Lambda(\LieOperad)V=\im (\textup{Tr}:S(\LieOperad)V\to \Gamma(\LieOperad)V).\]
%Then $\Lambda(\LieOperad)(V)$ is the free Lie algebra on $V$, subject to the standard axiom, $[x,x]=0$.
%
%We'll be interested in \emph{partially restricted Lie algebras}. That is, Lie algebras $L$ equipped with a decomposition $L=L_0\oplus L_1$ such that $L_1$ is a Lie ideal and a map $\restn{(\DASH)}:L_1\to L_1$ satisfying the axioms one would expect for a restriction map. Namely, for any $a,b\in L_1$ and $c\in L$, $\restn{(a+b)}=\restn{a}+\restn{b}+[a,b]$, and $[\restn{a},c]=[a,[a,c]]$. There is a free functor from $\vect{}{}\times \vect{}{}$ to partially restricted Lie algebras, sending $(V_0,V_1)$ to the algebra generated by restrictable elements $V_1$ and non-restrictable elements $V_0$. This free functor may be described as follows. There is a natural $\Sigma_n$-equivariant decomposition: %The expressions for $S(\LieOperad)(V_0\oplus V_1)$ and $\Gamma(\LieOperad)(V_0\oplus V_1)$ both contain the terms:
%\[\LieOperad(n)\otimes (V_0\oplus V_1)^{\otimes n}=\Bigl(\LieOperad(n)\otimes (V_0)^{\otimes n}\Bigr)\oplus\Bigl(\bigoplus(\LieOperad(n)\otimes V_{i_1}\otimes V_{i_2}\otimes\cdots\otimes V_{i_n})\Bigr)\]
%where the direct sum is taken over all sequences $(i_1,\ldots,i_n)\in\{0,1\}^n$ other than $(0,\ldots,0)$.
%%and this decomposition is of $\Sigma_n$-modules.
%The free partially restricted Lie algebra is the subspace of $\Gamma(\LieOperad(V_0\oplus V_1))$ given by
%\[\im \Bigl(S(\LieOperad)(V_0)\overset{\textup{Tr}}{\to}\Gamma(\LieOperad)(V_0\oplus V_1)\Bigr)\oplus\bigoplus_{n\geq1} \Bigl(\bigoplus(\LieOperad(n)\otimes V_{i_1}\otimes V_{i_2}\otimes\cdots\otimes V_{i_n})\Bigr)^{\Sigma_n}\]
]%END commented out old material
For each $n\geq0$, we'll be interested in certain categories of Lie algebras monadic over $\vect{+}{n}$, with a grading shift. Broadly, a $\vect{+}{n}$-graded Lie algebra is an element $L\in\vect{+}{n}$ with a structure map $\Lambda^2 L\to L$ which treats gradings as follows
\[L^{t}_{s_n,\ldots,s_1}\otimes L^{t'}_{s'_n,\ldots,s'_1}\to L^{t+t'+1}_{s_n+s'_n,\ldots,s_1+s'_1}.\]
More precisely, we view the Lie operad as an operad in $\vect{+}{n}$, such that
\[\LieOperad(r)=(\LieOperad(r))^{r-1}_{0,\ldots,0},\]
and then a $\vect{+}{n}$-graded Lie algebra is an algebra over the corresponding monad $\Lambda(\LieOperad)$ on $\vect{+}{n}$. In our context, the Lie operad arises as the Koszul dual of the commutative operad, through the constructions in \cite[\S5]{MR1089001}. See \cite[\S5.3.4]{FresseKoszulDuality.pdf} for a discussion of Koszul duality of operads in positive characteristic.

A $\vect{+}{n}$-graded \emph{partially restricted} Lie algebra is to be a graded Lie algebra such that certain graded parts admit a restriction operation. Specifically, there is to be defined a restriction operation
\[\restn{(\DASH)}:L^t_{s_n,\ldots,s_1}\to L^{2t+1}_{2s_n,\ldots,2s_1}\]
whenever not all of $s_n,\ldots,s_{1}$ are zero. We will denote the category of such objects $\calL(n)$. It is monadic over $\vect{+}{n}$, with an adjunction
\[F_{\calL(n)}:\vect{+}{n}\rightleftarrows \calL(n):U_{\calL(n)}.\]
The monad of this adjunction may be constructed as an appropriately chosen submonad of $\Gamma(\LieOperad):\vect{+}{n}\to \vect{+}{n}$ ($\LieOperad$ shifted as above), containing $\Lambda(\LieOperad)$. As such, the free construction $F_{\calL(n)}V$ admits a quadratic grading, which we denote $F_{\calL(n)}^{(k)}V$.  (\textbf{say something} about grading shifts here?)
%To describe the free functor, if $V\in\vect{+}{n}$, write:
%\[V_-=\bigoplus_{t\geq1}V^t_{0,\ldots,0},\textup{ and }V_+=\bigoplus_{t\geq1}\bigoplus_{\exists\,i\textup{\,s.t.\,}s_i\neq0}V^t_{s_n,\ldots,s_1}.\]
%There is a natural $\Sigma_n$-equivariant decomposition: 
%\[\LieOperad(n)\otimes (V_0\oplus V_1)^{\otimes n}=\Bigl(\LieOperad(n)\otimes (V_0)^{\otimes n}\Bigr)\oplus\Bigl(\bigoplus(\LieOperad(n)\otimes V_{i_1}\otimes V_{i_2}\otimes\cdots\otimes V_{i_n})\Bigr)\]
%where the direct sum is taken over all sequences $(i_1,\ldots,i_n)\in\{0,1\}^n$ other than $(0,\ldots,0)$.
%The free partially restricted Lie algebra is the subspace of $\Gamma(\LieOperad(V_0\oplus V_1))$ given by
%\[\im \Bigl(S(\LieOperad)(V_0)\overset{\textup{Tr}}{\to}\Gamma(\LieOperad)(V_0\oplus V_1)\Bigr)\oplus\bigoplus_{n\geq1} \Bigl(\bigoplus(\LieOperad(n)\otimes V_{i_1}\otimes V_{i_2}\otimes\cdots\otimes V_{i_n})\Bigr)^{\Sigma_n}\]

\subsection{Construction of the monads $F_{\calW(0)}$ and $F_{\calU(0)}$}\label{Construction of the monads}
\tododone{Construct monad $U$ for $\calW$}{Write in terms of monads $\PMonad$ and $S(\scrL)$.\item Decide whether to give the distributive law.
}
Denote by $\Palg$ the algebra arising in Goerss' work, defined as a quotient of a free unital associative algebra:
\[\Palg=F_{\textup{ass}}(P_0,P_1,P_2,\ldots)/\left(P_0=P_1=0,\ \textup{$P$-Adem relation}\right)\]
This associative algebra can be $\vect{+}{0}$-graded, by thinking of the generator $P^i$ as having grading $i+1$ (\textbf{or} even in $\vect{1}{1}$, taking $P^i\in(\Palg)^{i+1}_1$). For any $x\in\Palg$, there is a homomorphism, \emph{right multiplication} by $x$:
\[\textup{m}_x:\Palg\to \Palg,\quad y\mapsto yx.\]
Moreover, $\Palg$ has an \emph{admissible basis}, consisting of those monomials
\[P^I:=P^{i_\ell}P^{i_{\ell-1}}\cdots P^{i_1}\]
for which $I=(i_\ell,\ldots,i_1)$ is a sequence of integers such that each $i_j\geq2$, and  each comparison $i_{j+1}< 2i_j$ holds. Moreover, $\Palg$ has a decreasing filtration $\Palg\supset F^1\Palg\supset F^2\Palg\supset\cdots $, where
\[F^n\Palg=\F_2\left[P^I\,:\,\textup{$I$ nonempty and admissible, }i_j+\cdots +i_1\geq n+j\textup{ for some $j\geq 1$}\right].\]
By examining the $P$-Adem relation, one can determine that this is a filtration by left ideals. For a vectorspace $V\in\vect{+}{0}$, we have
\[F_{\calU(0)}V=\Palg\otimes V/\bigoplus_{i\geq1}(F^n\Palg\otimes V^n)\]
Next, writing $S(\LieOperad)$ for the monad on $\vect{+}{0}$ with cohomological grading shifted, we may construct the monad $F_{\calW(0)}$ (the restriction of the monad $U$ in \cite[p.\ 18]{MR1089001}), by coequalizing two maps $\Palg\otimes S(\LieOperad)V\to F_{\calU(0)} (S(\LieOperad)V)$. The first map is the composite:
\[\xymatrix@R=4mm{
\Palg\otimes S(\LieOperad)V\ar[r]^-{\Id\otimes \textup{frob}}
&%r1c1
\Palg\otimes S(\LieOperad)V\ar@{->>}[r]&%r1c2
F_{\calU(0)}(S(\LieOperad)V).%r1c3
}\]
The second map restricts to each subset $\Palg\otimes (S(\LieOperad)V)^r$ to the composite:
\[\xymatrix@R=4mm{
\Palg\otimes (S(\LieOperad)V)^r\ar[r]^-{\textup{m}_{P^r}\otimes\Id}
&
\Palg\otimes (S(\LieOperad)V)^r\ar@{^{(}->}[r]
&%r1c1
\Palg\otimes S(\LieOperad)V\ar@{->>}[r]&%r1c2
F_{\calU(0)}(S(\LieOperad)V).%r1c3
}\]
One can check, using \cite[p.\ 18]{MR1089001} the following is a coequalizer in $\vect{+}{0}$:
\[\xymatrix@R=4mm{
\Palg\otimes S(\LieOperad)V\ar@<.5ex>[r]\ar@<-.5ex>[r]
&%r1c1
F_{\calU(0)}(S(\LieOperad)V)\ar@{->>}[r]
&%r1c2
F_{\calW(0)}V%r1c3
}\]
The monad $F_{\calW(0)}$ possesses a quadratic grading, as do the $\calW(n)$ for $n\geq1$. One views $V\subset F_{\calW(0)}V$ as lying in quadratic grading 1, and demands that quadratic grading adds under taking brackets and doubles when applying $P$-operations.


An object of $\calW(0)$ is in particular an object of $\calU(0)$, and (as all of the $P$ operations are linear), we can define a functor $Q^{\calU(0)}:\calW(0)\to\vect{+}{0}$ by quotienting by the image of these operations. One may check:
\begin{lem}\label{Kill P ops gives lie alg}
For $X\in\calW(0)$, the vector space $Q^{\calU(0)}(X)$ inherits the structure of an object of $\calL(0)$ from the bracket of $X$, yielding a factorization:% of indecomposable functors, $Q^{\calW(n)}=Q^{\calL(n)}\circ Q^{\calU(n)}$
\[Q^{\calW(0)}=Q^{\calL(0)}\circ Q^{\calU(0)}:\left(\calW(0)\to \calL(0)\to \vect{+}{0}\right).\]
Moreover the composite $Q^{\calU(0)}\circ F_{\calW(0)}$ equals the free construction $F_{\calL(0)}$.
\end{lem}
\begin{proof}
One checks that the bracket is well defined in the quotient, and that taking the quotient by the top $P$ operation imposes the relation $[x,x]=0$, to create a $\Lambda(\LieOperad)$-algebra from the pre-existing $S(\LieOperad)$-algebra structure. The final claim follows from \cite[p.\ 18]{MR1089001}.
\end{proof}


\subsection{Homotopy operations for partially restricted Lie algebras}
\textbf{much of this now moved...} We'll now give an account of the natural operations on the homotopy of a simplicial object of $\calL(n)$.

The homotopy operations we'll present will be in the form of a natural transformation of functors $s\vect{+}{n}\to \vect{+}{n+1}$, which will be induced by the Eilenberg-Mac Lane shuffle map $\nabla:N_*(V)\otimes N_*(V)\to N_*(V\otimes V)$. \textbf{Need to clarify what's meant by quad grading.}
\begin{prop}[Various authors]\label{the top homotopy operations for Lie algebras}
There is a natural commuting diagram:
\[\xymatrix@R=4mm{
F^{(2)}_{\calL(n+1)}(\pi_*V)\ar@{-->}[r]^-{\widetilde{\nabla}}
\ar[d]^-{\textup{incl}}
&%r1c1
\pi_*(F^{(2)}_{\calL(n)}V)\ar[d]^-{\pi_*(\textup{incl})}
\\%r1c2
S^2(\pi_*V)\ar[r]^-{\widetilde{\nabla}}&%r2c1
\pi_*(S^2V)%r2c2
}\]
%\[\widetilde{\nabla}:F^{(2)}_{\calL(n+1)}(\pi_*V)\to \pi_*(F^{(2)}_{\calL(n)}V)\]
with horizontal maps defined by the formulae (for cycles $x,y,z\in ZN_*(V)$):
\[\widetilde{\nabla}([x]\otimes[y]+[y]\otimes [x])=[\nabla(x\otimes y+y\otimes x)]\textup{, and }\widetilde{\nabla}([x]\otimes[x])=[\nabla(x\otimes x)].\]
\end{prop}
\begin{proof}
Since $\nabla$ is symmetric, the formulae given for $\widetilde{\nabla}$ do indeed return homotopy classes in either $\pi_*(F^{(2)}_{\calL(n)}V)$ or $\pi_*(S^2V)$, however, it is not clear that either of the maps $\widetilde{\nabla}$ is well defined. We'll only need to check that the lower map is well defined: as $\pi_*(\textup{incl})$ is a monomorphism (by \cite[Prop 5.6]{BousOpnsDerFun.pdf}), it will follow that the upper map is well defined. 
%Although it is not clear that $\widetilde{\nabla}$ is well defined, $\widetilde{\nabla}$ will prove to be the unique lift in a diagram: %We will construct it as the unique lift (dotted) of an analogous map $\widetilde{\nabla}'$:
%\[\xymatrix@R=4mm{
%F^{(2)}_{\calL(n+1)}(\pi_*V)\ar@{-->}[r]^-{\widetilde{\nabla}}
%\ar[d]^-{\textup{incl}}
%&%r1c1
%\pi_*(F^{(2)}_{\calL(n)}V)\ar[d]^-{\pi_*(\textup{incl})}
%\\%r1c2
%S^2(\pi_*V)\ar[r]^-{\widetilde{\nabla}'}&%r2c1
%\pi_*(S^2V)%r2c2
%}\]
%in which the map $\pi_*(\textup{incl})$ is a monomorphism (by \cite[Prop 5.6]{BousOpnsDerFun.pdf}). Now $\widetilde{\nabla}'$ will be defined by the same formulae as $\widetilde{\nabla}$. 
\begin{shaded}\tiny
By general observations on natural transformations out of the endofunctor $S^2$ of $\vect{}{}$, to construct a natural map $\widetilde{\nabla}:S^2(\pi_*V)\to\pi_*(S^2V)$ is to give a natural linear map $p:S_2(\pi_*(V))\to \pi_*(S^2V)$ and a natural function $\restn{(\DASH)}:\pi_n(V)\to \pi_{2n}(S^2V)$ satisfying $\restn{(x+y)}=\restn{x}+\restn{y}+p(x\otimes y)$. Then $\widetilde{\nabla}$ is the unique natural map such that $\widetilde{\nabla}\circ\trace=p$ and $\widetilde{\nabla}(a)=p(a\otimes a)$ for $a\in\pi_*\left(V\right)$.
%\[p=\left(S_2\pi_*(V)\overset{\trace}{\to} S^2\pi_*(V)\overset{\widetilde{\nabla}'}{\to} \pi_*(S^2V)\right)\textup{ and }\restn{(\DASH)}=\left(\pi_*(V)\overset{v\mapsto v\otimes v}{\to} S^2\pi_*(V)\overset{\widetilde{\nabla}'}{\to} \pi_*(S^2V)\right).\]
We define $\widetilde{\nabla}$ in this way, with
\[p=\pi_*(\trace)\circ\nabla:\left(S_2(\pi_*(V))\overset{\nabla}{\to}\pi_*(S_2(V))\overset{\pi_*(\trace)}{\to}\pi_*(S^2(V))\right)\]
and with $\restn{(\DASH)}$ the operation $\sigma_n:\pi_n(V)\to \pi_{2n}(S^2V)$ detailed in \cite[\S3]{MR1089001}.
\end{shaded}
%In general, to construct a natural map $g$ from $S^2$ to any other endofunctor $G$ of the category of $\F_2$-vectorspaces, it is enough to define a natural linear map $p:S_2V\to GV$ and natural function $\restn{(\DASH)}:V\to GV$ satisfying $\restn{(x+y)}=\restn{(x)}+\restn{(y)}+p(x\otimes y)$. Then there is a unique natural map $g:S^2\to G$ such that we have
%\[p=\left(S_2V\overset{\trace}{\to} S^2V\to GV\right)\textup{ and }\restn{(\DASH)}=\left(V\overset{v\mapsto v\otimes v}{\to} S^2V\to GV\right).\]
%The map $\widetilde{\nabla}'$ is exactly obtained by this method, using 
%
%\[p([x]\otimes [y])=[\nabla(x\otimes y+y\otimes x)],\textup{ and }\restn{([x])}=[\nabla(x\otimes x)].\qedhere\]
\end{proof}
It is classical that via this map, the homotopy of a simplicial object $L\in s\calL(n)$ obtains the structure of an object of $\calL(n+1)$:
\begin{prop}[Curtis, 6-A]\label{prop on top operations}
If $L\in s\calL(n)$ has structure map $\rho:F_{\calL(n)}L\to L$, then $\pi_*(L)\in\calL(n+1)$, with structure map extending $\pi_*(\rho)\circ\widetilde{\nabla}:F^{(2)}_{\calL(n+1)}(\pi_*L)\to \pi_*L$.
\end{prop}
\begin{proof}
This theorem follows from the analogous theorem for fully restricted simplicial Lie algebras $L$, once the operations $\widetilde{\nabla}$ have been identified as the operations of \cite[\S8.5]{CurtisSimplicialHtpy.pdf}. In the proposed $\calL(n+1)$-structure, for $x,y
\in N_*L$ the bracket of $[x],[y]\in\pi_*L$ is given by
\[[\rho(\nabla(x\otimes y+y\otimes x))]=[\rho(\trace(\nabla(x\otimes y)))]=[\textup{br}(\nabla(x\otimes y))],\]
and if $n>0$, the restriction of an element $[x]\in\pi_nL$ is given by
\[[\rho(\nabla(x\otimes x))]=[\rho(\trace(\nabla^0(x\otimes x)))]=[\textup{br}(\nabla^0(x\otimes x))],\]
and these two formulae coincide with those of \cite{CurtisSimplicialHtpy.pdf}. The point here was that both before and after taking homotopy groups, brackets are defined using the trace. We should also check that in the proposed structure, the proposed restriction $\pi_0L\to \pi_0L$ is given on the chain level by the restriction $L_0\to L_0$, as in \cite{CurtisSimplicialHtpy.pdf}. As $\nabla^0:N_0L\otimes N_0L\to N_0(L\otimes L)$ is the identity, for $[x]\in\pi_*L$:
\[(\pi_*(\rho)\circ\widetilde{\nabla})([x])=[\rho(\nabla(x\otimes x))]=[\rho(x\otimes x)]=[\restn{x}].\qedhere\]
\end{proof}
We'll now turn to some further operations defined on the homotopy of an object of $s\calL(n)$.
\begin{prop}[\cite{CurtisSimplicialHtpy.pdf,6Author.pdf}]\label{linear operations on homotopy of lie alg}
If $L\in s\calL(n)$ has structure map $\rho:F_{\calL(n)}L\to L$, then there are right operations
\[(\DASH)\lambda_i:(\pi_{s_{n+1}}L)_{s_n,\ldots,s_1}^t\to (\pi_{s_{n+1}+i}L)_{2s_n,\ldots,2s_1}^{2t+1}\]
defined whenever $0\leq i\leq s_{n+1}$ and not all of $i,s_n,\ldots,s_1$ equal zero by the composite
\[(\pi_{s_{n+1}}L)_{s_n,\ldots,s_1}^t\overset{\sigma_i}{\to}(\pi_{s_{n+1}+i}(F^{(2)}_{\calL(n)}L))_{2s_n,\ldots,2s_1}^{2t}\overset{\pi_*(\rho)}{\to}(\pi_{s_{n+1}+i}(L))_{2s_n,\ldots,2s_1}^{2t+1}.\]
The operation $\lambda_i$ is linear when $i<s_{n+1}$, while the top operation $\lambda_{s_{n+1}}$ equals the restriction defined in proposition \ref{prop on top operations}. The operations satisfy the Adem relations for the $\Lambda$-algebra. That is, whenever $i>2j$, and $x$ is a homogeneous element of $\pi_*L$ such that $x\lambda_j\lambda_i$ is defined:
\[x\lambda_j\lambda_i=\sum_{k=0}^{(i-2j)/2-1}{i-2j-2-k\choose k}x\lambda_{i-j-1-k}\lambda_{2j+1+k}.\]
\end{prop}
\begin{proof}
As in the previous proposition, we only need to identify the operations defined with those given by Curtis in the case of a (fully) restricted Lie algebra. This is verified when $i\geq1$ using [Dwyer, 4.4], and when $i=0$ using the fact that $\sigma_0:\pi_*L\to \pi_*(S^2L)$ is the map $[x]\mapsto[x\otimes x]$.
\end{proof}
For $n\geq0$, let $\calU(n+1)$ denote the algebraic category of $\vect{+}{n+1}$-graded vectorspaces $V$ equipped with only those right $\lambda$-operations in proposition \ref{linear operations on homotopy of lie alg} that are \emph{linear}. That is, linear operations 
\[(\DASH)\lambda_i:V_{s_{n+1},s_n,\ldots,s_1}^t\to V_{s_{n+1}+i,2s_n,\ldots,2s_1}^{2t+1}\]
defined whenever $0\leq i< s_{n+1}$ and not all of $i,s_{n},\ldots,s_{1}$ are zero, satisfying the Adem relations of proposition \ref{linear operations on homotopy of lie alg}. Summarizing classical results to be found in \cite{CurtisSimplicialHtpy.pdf}:

\begin{prop}\label{compatibilities between U and L in W}
For $L\in s\calL(n)$, the homotopy groups $\pi_*(L)$ are simultaneously an object of $\calL(n+1)$ and of $\calU(n+1)$. The $\calU(n+1)$-operations are annihilated by the bracket. That is, for $i<s_{n+1}$, $x\in\pi_{s_{n+1}}L$ and $y\in \pi_*L$ such that $x\lambda_i$ is defined, we have $[x\lambda_i,y]=0$.
\end{prop}
For $n\geq0$, let $\calW(n+1)$ denote the algebraic category of $\vect{+}{n+1}$-graded vectorspaces which are simultaneously an object of $\calU(n+1)$ and $\calL(n+1)$, such that the compatibilities of \ref{compatibilities between U and L in W} are satisfied. This category has a number of useful properties, following from the calculations of \cite{6Author.pdf}, primarily:
\begin{prop}\label{prop on Wnplus1 being the pialgs for Wn}
The category $\calW(n+1)$ is isomorphic to the category $\calL(n)\PiAlg$ of $\calL(n)$-$\Pi$-algebras, so that the operations defined above exhaust the set of natural operations on the homotopy of simplicial objects of $\calL(n)$.
The monad $F_{\calW(n+1)}$ on $\vect{+}{n+1}$ factors as a composite $F_{\calU(n+1)}\circ F_{\calL(n+1)}$, with monad structure arising from a distributive law \cite{BeckDistLaws} of monads on $\vect{+}{n+1}$.
\end{prop}
\begin{proof}
All of these facts are easy to prove after observing that, for $W\in s\calV$ a wedge of spheres, $\pi_*(F_{\calL(n)}W)$ embeds in $\pi_*(F_{\Gamma(\LieOperad)}W)$, which, along with $\pi_*(F_{\Lambda(\LieOperad)}W)$, is well studied \cite{6Author.pdf}. In order to make this observation, let write $W^\textup{z}$ for $\bigoplus_{t\geq1}W_{0,\ldots,0}^t$, the non-restrictable part of $W$. This is actually a sub-wedge of $W$, the wedge of those summands of $W$ which lie in homological dimension $(0,\ldots,0)$. Now there is a commuting diagram of simplicial vector spaces, containing two short exact sequences:
\[\xymatrix@R=-2mm{
0
\ar[r]
&%r1c1
F_{\calL(n)}W
\ar[r]^{\alpha}
&%r1c2
F_{\Gamma(\LieOperad)}W
\ar[dr]^-(.4){\rho\gamma}\ar@{->>}[dd]^-{\gamma}
&%r1c3
%\Fr{\RestLie{n}}W/\Fr{\PRLie{n}}W
%\ar[r]
%\ar[d]^-{\cong}
%&%r1c4
\\%r1c5
&&&\frac{\displaystyle F_{\Gamma(\LieOperad)}W^\textup{z}}{\displaystyle F_{\Lambda(\LieOperad)}W^\textup{z}}
\ar[r]
&
0
\\
0
\ar[r]
&%r1c1
F_{\Lambda(\LieOperad)}W^\textup{z}
\ar[r]^{\beta}
&%r1c2
F_{\Gamma(\LieOperad)}W^\textup{z}
\ar[ru]^-(.4){\rho}&%r1c3
%\Fr{\RestLie{n}}W^\textup{z}/ F_{\Lambda(\LieOperad)}W^\textup{z}
%\ar[r]
%&%r1c4
%0
}\]
On homotopy groups: $\beta_*$ is injective (its source and target are well understood), so that $\rho_*$ is surjective. $\gamma_*$ is surjective (after all, $\gamma$ is an isomorphism in those internal degrees in which its codomain in nonzero), so that $(\rho\gamma)_*$ is surjective. Thus $\alpha_*$ is injective.
\end{proof}
It will be useful in what follows that $\calW(n+1)$ is monadic over $\vect{+}{n+1}$, not just over a category of graded sets.
We define a quadratic grading on the monad $F_{\calW(n+1)}$ by specifying that elements of $V$ have quadratic grading 1, and demanding that quadratic grading adds under taking brackets and doubles when applying $\lambda$-operations.

Moreover, we may define a functor $Q^{\calU(n+1)}:\calW(n+1)\to\vect{+}{n+1}$, by quotienting by the image of the \emph{non-top} $\lambda$-operations (which, fortunately, are linear), and one can prove the following lemma, the direct analogue of lemma \ref{Kill P ops gives lie alg}:
\begin{lem}\label{Kill lambda ops gives lie alg}
For $X\in\calW(n+1)$, $X$ is in particular an object of $\calL(n+1)$, and the vector space $Q^{\calU(n+1)}(X)$ retains this structure, yielding a factorization:% of indecomposable functors, $Q^{\calW(n)}=Q^{\calL(n)}\circ Q^{\calU(n)}$
\[Q^{\calW(n+1)}=Q^{\calL(n+1)}\circ Q^{\calU(n+1)}:\left(\calW(n+1)\to \calL(n+1)\to \vect{+}{n+1}\right).\]
Moreover the composite $Q^{\calU(n+1)}\circ F_{\calW(n+1)}$ equals the free construction $F_{\calL(n+1)}$.
\end{lem}
\begin{proof}
Similar to the proof of lemma \ref{Kill P ops gives lie alg}, using the observation from the proof of proposition \ref{prop on Wnplus1 being the pialgs for Wn} that $\pi_*(F_{\calL(n)}W)\subseteq\pi_*(F_{\Gamma(\LieOperad)}W)$.
\end{proof}



\subsection{Decomposition maps for $\calL(n)$ and $\calW(n)$}
\tododone{Describe the composition maps for these categories}{show the compatibility under $\calL(n)\PiAlg\cong \calW(n+1)$
\item show that the $j$ are \emph{quadratic decomposition maps}, if that language is still around}
Here we'll introduce decomposition maps for the categories $\calL(n)$ and $\calW(n)$, and prove a certain compatibility between them. The definitions are simple enough, and the reader can verify that each is well defined. Choose $n\geq0$, then we define decomposition maps $\calC:Q(X\wedge Y)\to QX\otimes QY$:
\begin{alignat*}{2}
%\calW(0):&\quad &
j_{\calW(0)}:P^{i_\ell}\cdots P^{i_1}[z_1,\cdots ,z_a]&\longmapsto\begin{cases}
z_1\otimes z_2,&\textup{if }\ell=0,\,a=2,\,z_1\in X,\,z_2\in Y,
\\0,&\textup{otherwise.}
\end{cases}
\\
%\calW(n+1):&&
j_{\calW(n+1)}:[z_1,\cdots ,z_a]\lambda_{i_1}\cdots \lambda_{i_\ell}&\longmapsto\begin{cases}
z_1\otimes z_2,&\textup{if }\ell=0,\,a=2,\,z_1\in X,\,z_2\in Y,
\\0,&\textup{otherwise.}
\end{cases}
\\
%\calL(n):&&
j_{\calL(n)}:\restnRepeated{[z_1,\cdots ,z_a]}{r}&\longmapsto\begin{cases}
z_1\otimes z_2,&\textup{if }r=0,\,a=2,\,z_1\in X,\,z_2\in Y,
\\0,&\textup{otherwise.}
\end{cases}
\end{alignat*}
[or \textbf{uniformize} all this using the phrase $\calU(n)$-operations.]
\begin{prop}
If $V\in\vect{+}{n}$ then $\quadratic_{\calL(n)}$ equals the composite $F_{\calL(n)} V\epi F^{(2)}_{\calL(n)} V\subseteq S^2V$, and $\quadratic_{\calW(n)}$ equals the composite $F_{\calW(n)} V\epi F_{\calL(n)} V\epi F^{(2)}_{\calL(n)} V\subseteq S^2V$.
\end{prop}
\begin{proof}
Consider the case $\calW(n+1)$ for $n\geq1$. As $\quadratic_\calC$ vanishes except on quadratic grading 2, one only checks terms $[x,y]$, $\restn{x}$, and $x\lambda_i$ (a $\calU(n+1)$-operation, not the restriction):
\begin{alignat*}{2}
\quadratic_{\calW(n+1)}([x,y])&=j_{\calW(n+1)}([{x_1}+{{x_2}},{y_1}+{{y_2}}]+[{x_1},{y_1}]+[{{x_2}},{{y_2}}])\\
&=j_{\calW(n+1)}([{x_1},{{y_2}}]+[{{x_2}},{y_1}])=x\otimes y+y\otimes x,
\end{alignat*}
which is precisely the representation of $[x,y]$ in $F^{(2)}_{\calL(n+1)}V\subseteq V^{\otimes2}$. Similarly, $\quadratic_{\calW(n+1)}(x\lambda_i)$ vanishes (as $\calU(n+1)$-operations are linear), while $\quadratic_{\calW(n+1)}(\restn{x})$ equals $x\otimes x$ as desired. The other cases, including the case of $\calW(0)$, are barely any different.
\end{proof}

\subsection{Cohomology operations for simplicial (restricted) Lie algebras}\label{section: Cohomology operations for simplicial (restricted) Lie algebras}
\todo{Explain the four Koszul algebras in these two sections}{Explain why you get the Koszul dualities
\item refer to appendix for proof of the fact about Lie cohomology}

\textbf{Probably move this to the end of the chapter on lie algs... but edit it here for convenience.}

Old \textbf{spiel} from definition of operations on $H^*_{\calW(n)}$:
By Priddy's work in \cite{PriddySimplicialLie.pdf}, Lie algebra cohomology has Steenrod operations and a product. However, the formulae given by Priddy (involving the $\overline{W}$-construction) are not obviously equivalent to what we have given here. However, it is not hard to check that the definition we have given using $\psi_\calW$ is equivalent to an analogous definition for Lie algebra cohomology (given in [placemarker in appendix]), and theorem [BLAH again] identifies these operations with Priddy's.


\end{Lie algebras in characteristic 2 and their homotopy operations}


\begin{Cohomology operations for all unstable Lie algebras}
\vfil\pagebreak
\section{\textbf{Operations on $\calW(n)$- and $\calU(n)$-cohomology}}\label{Cohomology Operations chapter}

%\subsection{Cohomology for objects of $\calW$}
\tododone{Define $Q^\calW:\calW\to \vect{+}{0}$, and thus the cohomology functor on $\calW$}{Give the bar construction $B^{\calW}$
\item Explain significance as Adams $E^2$}
%We will use the standard simplicial bar construction as our simplicial resolution of $X\in\calW$, writing $B^\calW_sX=U^{s+1}X$. Then the cohomology of $X$, $H^{s}_t(X)\in\vect{1}{+}$, is the cohomology of the cochain complex
%\[C^{s}_{t}=\Hom(N_s(Q^\calW B_{\bullet}X)^{t},\F_2).\]
%We'll frequently use the identification $Q^\calW B_{s}X=U^{s}X$.


\subsection{Vertical $\delta$ operations on $H^*_{\calW(0)}$ and $H^*_{\calU(0)}$}
\tododone{Define the operations $\gamma^i$, state and define the $\delta_i$ operations}{still need to clarify the end of the proof of the relations.}
For $V\in \vect{+}{0}$, we will define natural homomorphisms
\[\theta_i:(F_{\calW(0)}V)^{t+i+1}\to V^{t}, \textup{ for $2\leq i<t$}.\]
Indeed, there are natural homomorphisms into the quadratic grading 2 part of $F_{\calW(0)}V$:
\begin{align*}
P^i:V^{t}&\to F_{\calW(0)}^{(2)}V^{t+i+1}, \textup{\quad  for $2\leq i<t$}\\
[\,,]:(S_2V)^{t}&\to F_{\calW(0)}^{(2)}V^{t+1},
\end{align*}
and for given $m\geq1$, the degree $m$, quadratic grading 2 part $F_{\calW(0)}^{(2)}V^m$ decomposes as
%\[UV^{m,(2)}=\left(\bigoplus_{n_1+n_2=m-1}\im ([\,,]_{n_1,n_2})\right) \oplus\left(\bigoplus_{2\leq i< (m-1)/2}\im (P^i_{m-i-1})\right)\]
%
\[F_{\calW(0)}^{(2)}V^{m}=%\left(\bigoplus_{n_1+n_2=m-1}
\im \bigl((S_2V)^{m-1}\overset{[,]}{\to} F_{\calW(0)}^{(2)}V\bigr)%\right) 
\oplus\bigoplus_{\!\!\!\!\!\!2\leq i< (m-1)/2\!\!\!\!\!\!}\im \bigl(V^{m-i-1}\overset{P^i}{\to}F_{\calW(0)}^{(2)}V\bigr).\]
Moreover, each map $P^i:V^t\to F_{\calW(0)}^{(2)}V^{t+i+1}$ appearing in this decomposition is an isomorphism onto its image, so that for $2\leq i <t$ we may construct $\theta_i$ as the composite
\[\theta_i:\left((F_{\calW(0)}V)^{t+i+1}\overset{\textup{proj}}{\makebox[.06ex][l]{$\to$}\to} (F_{\calW(0)}^{(2)}V)^{t+i+1}\overset{\textup{proj}}{\makebox[.06ex][l]{$\to$}\to} \im (P^i)\overset{(P^i)^{-1}}{\to}V^{t}\right).\]
Here we have projected onto the quadratic filtration 2 part, and then further onto the relevant summand in its natural decomposition. Note that although $P^t:V^t\to F_{\calW(0)}^{(2)}V^{2t+1}$ is a nontrivial linear map when $t\geq2$, its image is entangled with the image of the bracket, and we are not able to split it off. Thus we are not able to improve on the bounds $2\leq i< t$.
\begin{prop}\label{operations on goerss homology}
Suppose that $X\in s\calW(0)$ is almost free, so that we may identify $H^*_{\calW(0)}X$ with $(\pi_*Q^{\calw(0)}X)^*$. Then for $2\leq i <t$, the chain map $\widetilde{\theta_i}$ of proposition \ref{general CohOpns given irreducibility} induces a linear operation
\[\delta_i:(H^*_{\calW(0)}X)^{s}_t\to (H^*_{\calW(0)}X)^{s+1}_{t+i+1}.\] 
These operations are natural in maps preserving the generating subspaces, and satisfy the $\delta$-Adem relation of [Dwyer].
\end{prop}
\noindent For any $X\in s\calW(0)$, the bar construction $Q^{\calW(0)}B^{\calW(0)}X$, whose dual homotopy is $H^*_{\calW(0)}X$, has a natural almost free structure, so that proposition \ref{operations on goerss homology} constructs natural  operations on $H^*_{\calw(0)}X$ for all $X\in s\calw(0)$.

Note that \textbf{these are the tricky, unstable operations.}
%Now suppose that $X\in\calW(0)$, and consider the bar construction $Q^{\calW(0)}B^{\calW(0)}X$ whose cohomotopy is $H^*_{\calW(0)}X$. Until the end of \S\ref{Cohomology Operations chapter}, we'll write $Q\mathbb{B}$ for this simplicial object, write $F$ for $F_{\calW(0)}$, and identify $Q\mathbb{B}_s$ with $F^sX$.
\begin{proof}[Proof of \ref{operations on goerss homology}]
The conditions of proposition \ref{general CohOpns given irreducibility} are satisfied when $\theta=\theta_i$ and $G$ is the identity functor. It just remains to prove the $\delta$-Adem relations, which we will do using the technique of \cite{PriddyKoszul.pdf}, the point being that the algebra of $\delta$-operations is Koszul dual to $\Palg$. For this, we define a map $\theta_{ij}$, whenever $i<2j$, $2\leq j<t$ and $2\leq i<t+j+1$:
\[\theta_{ij}:\left((F_{\calW(0)}V)^{t+i+j+2}\overset{\textup{proj}}{\makebox[.06ex][l]{$\to$}\to} (F_{\calW(0)}^{(4)}V)^{t+i+j+2}\overset{\textup{proj}}{\makebox[.06ex][l]{$\to$}\to} \im (P^{i,j})\overset{(P^{i,j})^{-1}}{\to}V^{t}\right),\]
where we have split off the image of 
$P^{i,j}=P^iP^j$ %:V^n\to (F_{\calW(0)}^{(4)}V)^{n+i+j+2}\]
as before, which is possible, since neither $P^j$ nor $P^i$ are entangled with the bracket in these ranges. We may identify $Q^{\calW(0)}X_s$ with $V_s$, at the cost of replacing $d_0$ with $\epsilon\circ d_0$, as in lemma \ref{identify almost free indecs with gens}.  Define $\widetilde{\theta_{ij}}$ to be the composite $V_{s+1}\overset{d_0}{\to}FV_s\overset{\theta_{ij}}{\to}V_s$. This will be the nullhomotopy giving the $\delta$-Adem relation. %It'll feel easier to think of it as a nullhomotopy of the un-normalized complex. 
As in the proof of proposition \ref{general CohOpns given irreducibility}, we have $d_k\circ\widetilde{\theta_{ij}}=\widetilde{\theta_{ij}}\circ d_{k+1}$ for $k\geq1$, so its nullhomotopy is the sum
\[\epsilon d_0\widetilde{\theta_{ij}}+\widetilde{\theta_{ij}}(\epsilon d_0+d_1)=(\epsilon d_0\theta_{ij}+\theta_{ij}d_0\epsilon+\theta_{ij}d_0)d_0,\]
using the simplicial identity $d_0d_1=d_0d_0$, and the $\delta$-Adem relation will follow from
\[\theta_{ij}d_0=\Bigl(\epsilon d_0\theta_{ij}+\theta_{ij}d_0\epsilon+\sum\theta_\beta d_0\theta_\alpha\Bigr):FV_{s+1}\to V_s,\]
where the sum is taken over all the combinations of $\alpha,\beta\geq2$ such that $P^iP^j$ appears when $P^\alpha P^\beta$ is written in the $P$-admissible basis. This identity states the following: if $V\in\vect{+}{0}$, and $f(g_k)$ is a nested $\calw(0)$-expression with $g_k\in F_{\calw(0)}V$ and $f(g_k)\in F_{\calw(0)}F_{\calw(0)}V$, then if we write $d_0:F_{\calw(0)}F_{\calw(0)}V\to F_{\calw(0)}V$ for the monad product map, there are only three ways that one may obtain expressions of the form $P^iP^jv$ in $d_0(f(g_k))$: for some $k$, $g_k=P^iP^jv$, and $f $ adds no further operations to this term; $f=P^iP^jg_{k}$ for some $k$ for which $g_{k}=v$ is a unit expression; or for some $k$, $g_k=P^\beta v$, and  $f$ has $P^\alpha(g_k)$ as a summand. In this last case, after applying $d_0$, the composite $P^{\alpha}P^{\beta}v$ may need to be rearranged using the $P$-Adem relation, and we sum over those $\alpha$,$\beta$ producing a summand $P^iP^jv$.

This shows that the nullhomotopy proposed equals the sum of the maps $\widetilde{\theta_\beta}\circ\widetilde{\theta_\alpha}$, and as Goerss \cite{MR1089001} \emph{constructs} the $P$-algebra as the Koszul dual, in the sense of \cite{PriddyKoszul.pdf}, of the algebra of $\delta$-operations, $P^{\alpha }P^{\beta}$ produces a summand $P^iP^j$ if and only if $\delta_i\delta_j$ produces a summand $\delta_{\alpha }\delta_{\beta}$. This proves the result.
\end{proof}
The same constructions work in the category $s\calU(0)$ of simplicial unstable \emph{modules}, the only difference being that when we define $\theta_i$, we needn't worry about brackets, and we can define a map
\[\theta_i:F_{\calU(0)}V^{t+i+1}\to V^t\]
whenever $2\leq i\leq t$, so that there is one more operation available on $H^*_{\calU(0)}$ than on $H^*_{\calW(0)}$. It will be useful to encode this structure in a definition. Write $\calMv(1)$ for the algebraic category whose objects are vectorspaces $M\in\vect{1}{+}$ with left $\delta$-operations
\[\delta_i:M^{s}_t\to M^{s+1}_{t+i+1},\textup{ defined whenever $2\leq i\leq t$,}\]
satisfying the \textbf{tricky} $\delta$-Adem relations of [Dwyer].
\begin{prop}\label{operations on untable P homology}
Suppose that $X\in s\calU(0)$ is almost free. Then the chain maps $\widetilde{\theta_i}$ of proposition \ref{general CohOpns given irreducibility} give $H^*_{\calU(0)}X$ the structure of an object of $\calMv(1)$, natural in maps preserving the generating subspaces.
\end{prop}
In order to give a basis for a free object in $\calMv(1)$, for a sequence $I=(i_\ell,\ldots,i_1)$ of integers $i_j\geq2$, we define
\[\minDimP(I):=\max\{(i_1),\,(i_2-i_1-1),\,(i_3-i_2-i_1-2),\,\ldots,\,(i_{\ell}-\cdots-i_1-\ell+1)\},
\]
following the convention that $\max(\emptyset)=-\infty$. Moreover, say that the sequence is \emph{$\delta$-admissible} if each $i_j\geq2$ and  $i_{j+1}\geq 2i_j$ for $1\leq j <\ell$.
\begin{lem}\label{basis of element of M(0)}
%For $V\in\vect{1}{+}$ with homogeneous basis $B$, a basis of $F_{\calMv(1)}V$ consists of the classes $\delta_Ib$ where $b\in B\cap V^s_t$ is a basis element and $I$ is a $\delta$-admissible sequence with $t\geq\minDimP(I)$.
For $V\in\vect{1}{+}$ with homogeneous basis $B$, a basis of $F_{\calMv(1)}V$ consists of
\[\left\{\delta_Ib\ \middle|\ b\in B^{s}_t\textup{, $I$ $\delta$-admissible with }\minDimP(I)\leq t\right\}.\]
\textbf{extend to case where $V$ concentrated in Koszul deg 0}\end{lem}
\begin{shaded}
Note that the forgetful functor $\calW(0)\to\calU(0)$ does not preserve almost free simplicial objects! However, it does preserve levelwise free simplicial objects, which create the right homology groups (see PRLieAlgs.tex). Thus there's a disconnect in the number of available operations on the Adams $E^2$ and on the first Koszul resolution.
\end{shaded}

\subsection{Vertical Steenrod operations for $H^*_{\calW(n)}$ and $H^*_{\calU(n)}$ when $n\geq1$}\label{section: vertical Koszul operations n positive}
For $V\in \vect{+}{n}$, we will define natural homomorphisms
\[\theta^i:(F_{\calW(n)}V)^{2t+1}_{s_n+i-1,2s_{n-1},\ldots,2s_1}\to V^{t}_{s_n,\ldots,s_1},\]
which are defined for all $i,s_1,\ldots,s_n\geq0$ and $t\geq1$, but are zero except when $1\leq i \leq s_n$ and not all of $i-1,s_{n-1},\ldots,s_1$ are zero.
These are rather easier to define than in the $n=0$ case above, as the monad $F_{\calW(n)}$ is a simple composite $F_{\calu(n)}F_{\call(n)}$ of monads.
Indeed, there are natural monomorphisms
\[(\DASH)\lambda_{i-1}:V^{t}_{s_n,\ldots,s_1}\to (F^{(2)}_{\calW(n)}V)^{2t+1,(2)}_{s_n+i-1,2s_{n-1},\ldots,2s_1}\]
defined only when   $1\leq i\leq n$ and $i-1,s_{n-1},\ldots,s_1$ are not all zero, and an inclusion
\[\textup{incl}:F^{(2)}_{\calL(n)}V\to F^{(2)}_{\calW(n)}V.\]
As in the $n=0$ case, the images of the listed maps are linearly independent and span the quadratic grading 2 part of $F_{\calW(n)}V^*$. We define $\theta^i$ to be zero unless $1\leq i\leq n$ and $i-1,s_{n-1},\ldots,s_1$ are not all zero, in which case we define it as the composite: %project onto the quadratic grading 2 part, then further onto the summand corresponding to the image of $\lambda_{i-1}$, and finally use the inverse of the map $\lambda_{i-1}$:
\[\theta^i:\Bigl((F_{\calW(n)}V)^{2t+1}_{s_n+i-1,2s_{n-1},\ldots,2s_1}\overset{\textup{proj}\circ\textup{proj}}{\makebox[.06ex][l]{$\to$}\to} \im (\lambda_{i-1})\overset{(\lambda_{i-1})^{-1}}{\to}V^{t}_{s_{n,\ldots,s_1}}\Bigr).\]
One can give exactly the same definitions for the free construction in $\calU(n)$, producing functions $\theta^i:F_{\calU(n)}V\to V$ which are zero under the same conditions as for $\calW(n)$.
Write $\calMv(n+1)$ for the algebraic category whose objects are vectorspaces $M\in\vect{n+1}{+}$ with left Steenrod operations
\[\Sqv^i:M^{s_{n+1},\ldots,s_1}_t\to M^{s_{n+1}+1,s_n+i-1,2s_{n-1},\ldots,2s_1}_{2t+1},\]
which are zero except when $1\leq i \leq s_n$ and not all of $i-1,s_{n-1},\ldots,s_1$ are zero, and which
satisfy the $\Sq$-Adem relations of [wherever-Priddy?].

In the present case, $n\geq1$, there is no disparity between the unstableness conditions on $\calw(n)$- and $\calu(n)$-cohomology, so that we can combine the analogues of \ref{operations on goerss homology} and \ref{operations on untable P homology} into:
\begin{prop}\label{vertical steenrod operations prop}
Suppose that $X\in s\calC$ is almost free, where $\calC$ stands for either $\calW(n)$ or $\calU(n)$ with $n\geq1$. Then the chain maps $\smash{\widetilde{\theta^i}}$ of proposition \ref{general CohOpns given irreducibility} give $H^*_{\calC}X$ the structure of an object of $\calMv(n+1)$, natural in maps preserving the generating subspaces.
\end{prop}
\textbf{give some discussion. Point out that the nullhomotopy is unstable.}

In order to give a basis for a free object in $\calMv(n+1)$, for a sequence $I=(i_\ell,\ldots,i_1)$ of integers $i_j\geq1$, we define
\[\minDimSq(I):=\max\{(i_1),\,(i_2-i_1+1),\,(i_3-i_2-i_1+2),\,\ldots,\,(i_{\ell}-\cdots-i_1+(\ell-1))\},
\]
and say that $I$ is \emph{$\Sq$-admissible} if each $i_j\geq1$ and $i_{j+1}\geq 2i_j$ for $1\leq j <\ell$ \textbf{(uniformize this convention)}.
\begin{lem}
For $V\in\vect{n+1}{+}$ with homogeneous basis $B$, a basis of $F_{\calMv(n+1)}V$ consists of
\[\left\{\Sq^J_{\textup{v}}b\ \middle|\ \genfrac{}{}{0pt}{}{b\in B^{s_{n+1},\ldots,s_1}_t\textup{, $J$ $\Sq$-admissible with }\minDimSq(J)\leq s_n,}{\textup{if }s_{n-1}\!=\!\cdots\!=\!s_1\!=\!0\textup{ then $J$ doesn't contain 1}}\right\}.\]


\end{lem}

\subsection{Horizontal Steenrod operations and a product for $H^*_{\calW(n)}$}
For any $n\geq 0$, we'll construct operations on the homology $H^*_{\calW(n)}$ arising from the Lie structure.

Indeed, suppose that $X\in s\calW(n)$ is almost free. Then $Q^{\calU(n)}X\in s\calL(n)$ is also almost free, on essentially the same generating subspaces. Thus, the cohomotopy of $Q^{\calW(n)}X=Q^{\calL(n)}Q^{\calU(n)}X$ is an instance of simplicial partially restricted Lie algebra cohomology. Cohomology operations of this type are discussed in \S\ref{section: Cohomology operations for simplicial (restricted) Lie algebras} and appendix \ref{appendix on Lie coh ops}. In the present context, we have two equivalent definitions, one using $\psi_{\calL(n)}$ and one using $\psi_{\calW(n)}$. Until the appendix, we will use $\psi_{\calW(n)}$, defining operations
\begin{gather*}
\mu:\left(H_{\calW(n)}^{n_1}(X)\otimes H_{\calW(n)}^{n_2}(X)\overset{\ExtCohProd}{\to} \pi^{n_1+n_2}((S^2(QX))^*)\overset{\psi_{\calW(n)}^*}{\to} H_{\calW(n)}^{n_1+n_2+1}(X)\right)\textup{ and}\\
\Sqh^k:\left(H_{\calW(n)}^{n}(X)\overset{\ExtCohOp^{k-1}}{\to} \pi^{n+k-1}((S^2(QX))^*)\overset{\psi_{\calW(n)}^*}{\to} H_{\calW(n)}^{n+k}(X)\right).
\end{gather*}
More explicitly:
\begin{prop}
For $X$ an object of $\calW(n)$, $H^*_{\calW(n)}X$ possesses a left action of the Lie Steenrod algebra, consisting of natural homomorphisms
\[\Sqh^j:(H^*_{\calW(n)}X)_t^{s_{n+1},\ldots,s_1}\to (H^*_{\calW(n)}X)_{2t+1}^{s_{n+1}+j,2s_{n},\ldots,2s_1},\]
zero unless $1\leq j\leq s_{n+1}+1$ and not all of $j-1,s_{n},\ldots,s_1$ equal zero (so that if $n=0$, $\Sqv^1$ is necessarily zero). %If $n=0$, $\Sqh^1$ and $\Sqh^2$ are both zero 
[\textbf{$\Sqh^2=0$ when $n=0$?}] Moreover, $H^*_{\calW(n)}X$ supports a nonunital commutative algebra pairing
\[(H^*_{\calW(n)}X)_t^{s_{n+1},\ldots,s_1}\otimes (H^*_{\calW(n)}X)_q^{p_{n+1},\ldots,p_1}\to (H^*_{\calW(n)}X)_{t+q+1}^{s_{n+1}+p_{n+1}+1,s_{n}+p_{n},\ldots,s_1+p_1}.\]
These satisfy the unstableness condition
\[x^2=\Sqh^{s_{n+1}+1}x\text{ for }x\in (H^*_{\calW(n)}X)_t^{s_{n+1},\ldots,s_1}\]
and the Cartan formula
\[\Sqh^k(xy)=\textstyle\sum_{i=0}^k(\Sqh^ix)(\Sqh^{k-i}y).\]
\end{prop}
\begin{proof}
This theorem follows from theorem [BLAH] in the appendix.
\end{proof}
For $n\geq0$, write $\calMh(n+1)$ for the algebraic category whose objects are vectorspaces $M\in\vect{n+1}{+}$ with left Steenrod operations and a commutative pairing satisfying the conditions of the proposition. We have simply shown that $H^*_{\calW(n)}$ takes values in $\calMv(n+1)$.

Note that the unstableness condition implies that if $x\in M_t^{0,\ldots,0}$, then $x^2=0$. Indeed
\begin{prop}\label{basis of free horizontal operations algebra}
Suppose that $n\geq1$. For $V\in\vect{n+1}{+}$ with homogeneous basis $B$. Then $F_{\calMh(n+1)}V$ is the quotient of the algebra
\[\F_2\left[\Sq^J_{\textup{v}}b\ \middle|\ \genfrac{}{}{0pt}{}{b\in B^{s_{n+1},\ldots,s_1}_t\textup{, $J$ $\Sq$-admissible with }\excess(J)\leq s_{n+1},}{\textup{if }s_{n}=\cdots=s_1=0\textup{ then $J$ does not contain 1}}\right]\]
by the relation $b^2=0$ if $b\in B_t^{0,\ldots,0}$. Here, $e(J):=j_\ell-j_{\ell-1}-\cdots j_1$ is the Serre excess of $J$. \textbf{{check this!}}
\end{prop}
\begin{proof}
By \cite[6.1]{PriddySimplicialLie.pdf}, the true free object is at worst a quotient of what we propose. It is in fact equal to what we propose, because the two-sided ideal in $\LieSteen$ generated by $\Sqh^1$ is spanned by those admissible sequences ending in $\Sqh^1$, so that forcing $\Sq^1_h=0$ in the relevant degrees has no unintended consequences. Another way to express this fact is to say that left multiplication by $\Sq^1_h$ annihilates the augmentation ideal in $\LieSteen$ (the Steenrod algebra for Lie algebras).
\end{proof}

\subsection{Relations between the horizontal and vertical operations}
%With such a broad range of available operations, 
It will be helpful if we are able to reduce %We are interested in calculating the composite $\delta_i\Sqh^jx$, or $\delta_i(xy)$, for $x,y\in H^*_{\calW(n)}X$ \textbf{but only for $n=0$}, so that 
expressions in the the various available operations in a standard format, namely:
\[\prod_i \Sqh^{J_{i}}\deltav_{I_{i}}x_i\textup{ when $n=0$, or }\prod_i \Sqh^{J_{i}}\Sqv^{I_{i}}x_i\textup{ when $n\geq1$}.\]
This is possible, thanks to:
\begin{prop}
Suppose that $x,y\in H^*_{\smash{\calW(0)}}X$. If $\Sqh^jx\in (H^*_{\smash{\calW(0)}}X)^{s}_{t}$, then $\deltav_i\Sqh^{j}x=0$ for  $2\leq i<t$, and if $xy\in (H^*_{\smash{\calW(0)}}X)^{s}_{t}$, then $\deltav_i(xy)=0$ for  $2\leq i<t$.

Suppose that $x,y\in H^*_{\smash{\calW(n)}}X$ for $n\geq1$. Then $\Sqv^{i}\Sqh^{j}x=0$ and $\Sqv^i(xy)=0$.
%The expressions $\deltav_i\Sqh^jx$ and $\deltav_i(xy)$ for $x,y\in H^*_{\calW(0)}X$ are zero whenever defined. 
\end{prop}
\begin{proof}
For the case $n=0$, suppose that $X\in s\calw(0)$ is almost free on generating subspaces $V_s$. It is enough to prove that the composite
\[N_{s+1}(Q_{\calW(0)}X_{s+2})^{t+i+1}
\overset{\smash{\widetilde{\theta_i}}}{\to}
N_s(Q_{\calW(0)}X_{s+1})^{t}
\overset{{\psi_{\calw(n)}}}{\to}
N_{s-1}(S^2(Q_{\calW(0)}X_{s}))^{t-1}
\]
is nullhomotopic, by a similar method to that used in the proof of \ref{operations on goerss homology}. For any $V\in \vect{+}{0}$, there is a natural composite
\[(S_2V)^{t-1}\underset{\smash{\alpha}}{\overset{[\,,\,]}{\to}} (F^{(2)}_{\calW(0)}V)^{t}\underset{\smash{\beta}}{\overset{P^i}{\mono}} (F^{(4)}_{\calW(0)}V)^{t+i+1},\]
whose maps we have labeled $\alpha$ and $\beta$ for convenience.
The map $\beta$ is not a monomorphism on $\im([\,,])$ when $i=t-1$ is even, as in this case, for any $v\in V^{i/2}$,
\[P^i[v,v]=P^{i}P^{i/2}v=\textstyle\sum_{k=i/2+1}^{3i/2-2}{2(k-i/2)-1\choose k-i/2}P^{3i/2-k}P^kv=0,\]
by the Adem relation and the unstableness condition. However, the $\ker(\beta)$ is contained in $\ker(\quadratic_{\calw(0)})$, 
and $\im(\beta\circ\alpha)$ does naturally split off as a direct summand of $(F^{(4)}_{\calW(0)}V)^{t+i+1}$.
 We write $h_i$ for the composite:
\[h_{i}:\Bigl((F_{\calW(0)}V)^{t+i+1}\overset{\textup{proj}}{\epi}
(\im(\beta\circ\alpha))^{t+i+1}\overset{\beta^{-1}}{\to}\frac{\im(\alpha)}{\ker(\beta)\cap\im(\alpha)}\overset{\quadratic_{\calw(0)}}{\to}(S^2V)^{t-1}\Bigr).\]
Identifying $Q^{\calW(0)}X_s$ with $V_s$ as in the proof of \ref{operations on goerss homology}, the nullhomotopy associated with the composite $\widetilde{h_i}:(V_{s+1}\overset{d_0}{\to}FV_s\overset{h_i}{\to}V_s)$ is the sum
\[(\epsilon d_0h_i+h_id_0\epsilon+h_id_0)d_0,\]
and the relation we seek will follow from the identity
\[h_id_0=\Bigl(\epsilon d_0h_i+h_id_0\epsilon+\quadratic_{\calw(0)} d_0\theta_i\Bigr):FV_{s+1}\to V_s,\]
as then $\psi_{\calw(0)}\smash{\widetilde{\theta}_i}=\quadratic_{\calw(0)} d_0\theta_id_0=d_0\widetilde{h_i}+\widetilde{h_i}d_0$. This identity states the following: if $V\in\vect{+}{0}$, and $f(g_k)$ is a nested $\calw(0)$-expression with $g_k\in F_{\calw(0)}V$ and $f(g_k)\in F_{\calw(0)}F_{\calw(0)}V$, then if we write $d_0:F_{\calw(0)}F_{\calw(0)}V\to F_{\calw(0)}V$ for the monad product map, there are only three ways that one may obtain summands of the form $P^i[v_1,v_2]$ in $d_0(f(g_k))\in (F_{\calw(0)}V)^{t+i+1}$: for some $k$, $g_k=P^i[v_1,v_2]$, and $f $ adds no further operations to this term; $f=P^i[g_{k_1},g_{k_2}]$, where $g_{k_1}=v_1$ and $g_{k_2}=v_2$ are unit expressions; or for some $k$, $g_k=[v_1,v_2]$, and  $f$ has $P^i(g_k)$ as a summand.

For the case $n\geq0$, the proof only becomes easier, the main difference being that in the corresponding composite:
\[(F_{\call(n)}^{(2)})^{t-1}\underset{\smash{\alpha}}{{\to}} (F^{(2)}_{\calW(0)}V)^{t}\underset{\smash{\beta}}{\overset{\lambda_{i-1}}{\mono}} (F^{(4)}_{\calW(0)}V)^{t+i-1},\]
both $\alpha$ and $\beta|_{\im(\alpha)}$ are monomorphisms.
\end{proof}

\subsection{Compressing sequences of Steenrod operations}
\begin{thm}\label{thm on compressing seqs of steenrod ops}
Suppose that $n\geq1$ and $V\in \vect{n}{+}$. Then there is a decreasing filtration on $F_{\calMh(n)}V$, the `target filtration', and an isomorphism
\[ f:(F_{\calMh(n+1)}F_{\calMv(n+1)}V)^{s_{n+1},\ldots,s_1}_t\overset{\cong}{\to} E_0^{s_{n+1}}(F_{\calMh(n)}V)^{s_n,\ldots,s_1}_t,\]
defined by requiring that
$f(\Sqv^Iv)=\Sqh^Iv$ for $v\in V$, that $f(w_1w_2)=f(w_1)f(w_1)$ for $w_1,w_2\in F_{\calMh(n+1)}F_{\calMv(n+1)}V$,
and that
\[f(\Sqh^jw)=\Sqh^{j+s_n}f(w)\textup{ for }w\in (F_{\calMh(n+1)}F_{\calMv(n+1)}V)^{s_{n+1},\ldots,s_1}_t.\]
\end{thm}
\begin{proof}
The map $f$ is not a well defined map to $F_{\calMh(n)}V$ since the Adem relations between the $\Sqh$ are not preserved by the proposed map $f$. Write $W(V)$ for the quotient of $\F_2[\LieSteen\otimes F_{\calMv(n+1)}V]$ by the unstableness relations and Cartan formula, so that $F_{\calMh(n+1)}F_{\calMv(n+1)}V$ is obtained from $W(V)$ by taking the quotient by the two-sided ideal generated by the (horizontal) Adem relations. Then may define a map $\overline{f}:W(V)\to F_{\calMh(n)}V$ by requiring the same of $\overline{f}$ as of $f$. There is a decreasing filtration on $W(V)$, given by 
\[F^pW(V)=\bigoplus_{s_{n+1}\geq p}\bigoplus_{s_n,\ldots,s_1\geq0}\bigoplus_{t\geq1}W(V)^{s_{n+1},\ldots,s_1}_t\]
 and we define the \emph{target filtration} on the target by $F^p(F_{\calMh(n)}V):=\overline{f}(F^pW(V))$.

The map $\overline{f}$ fails to descend to a well-defined map  $F_{\calMh(n+1)}F_{\calMv(n+1)}V\to F_{\calMh(n)}V$, because it does not annihilate the Adem relations. However, we will show that it does send them into higher filtration, so that $\overline{f}$ induces a well defined map $f$ as advertised: if $w\in W(V)^{s_{n+1},\ldots,s_1}_t$ and $i<2j$, then
%\[f(\Sqh^i\Sqh^jw):=\Sqh^{i+2s_n}\Sqh^{j+s_n}f(w),\]
%and such an assignment does not preserve the Adem relations. Indeed, the Adem relation arising when $i<2j$:
%\[\Sqh^i\Sqh^jw-\sum_{k=0}^{\lfloor i/2\rfloor}{j-k-1\choose i-2k}\Sqh^{i+j-k}\Sqh^{k}w\]
%is sent to
%%\[\Sqh^{i+2s_n}\Sqh^{j+s_n}f(w)-\sum_{k=0}^{\lfloor i/2\rfloor}{j-k-1\choose i-2k}\Sqh^{i+j-k+2s_n}\Sqh^{k+s_n}f(w)\]
\begin{alignat*}{2}
\overline{f}\Bigl(\Sqh^i&\Sqh^jw-\textstyle\sum_{k=0}^{\lfloor i/2\rfloor}{j-k-1\choose i-2k}\Sqh^{i+j-k}\Sqh^{k}w\Bigr)\\
{}:={}&\Sqh^{i+2s_n}\Sqh^{j+s_n}\overline{f}(w)-\textstyle\sum_{k=0}^{\lfloor i/2\rfloor}{j-k-1\choose i-2k}\Sqh^{i+j-k+2s_n}\Sqh^{k+s_n}\overline{f}(w)\\
{}={}&
\Sqh^{i+2s_n}\Sqh^{j+s_n}\overline{f}(w)-\textstyle\sum_{k=s_n}^{\lfloor (i+2s_n)/2\rfloor}{j+s_n-k-1\choose i+2s_n-2k}\Sqh^{(i+2s_n)+(j+s_n)-k}\Sqh^{k}\overline{f}(w)\\
{}={}&\textstyle\sum_{k=0}^{s_n-1}{j+s_n-k-1\choose i+2s_n-2k}\Sqh^{(i+2s_n)+(j+s_n)-k}\Sqh^{k}\overline{f}(w)\\
{}=\makebox[0cm][r]{$:$}{}&\textstyle\sum_{k=0}^{s_n-1}{j+s_n-k-1\choose i+2s_n-2k}\overline{f}(\Sqh^{i+j+2(s_n-k)+1}\Sqv^{k}w),
\end{alignat*}
which is in filtration $s_{n+1}+i+j+2(s_n+1-k)>s_{n+1}+i+j$ (the second equality holds by simply shifting the dummy variable $k$, the third by an Adem relation in the codomain).
%This quantity is nonzero, but is written as the image under $f$ of an element of filtration 

What remains is to show that $f$ is an isomorphism as in the theorem statement, which we approach simply by choosing a set of multiplicative generators for both the domain and codomain. The domain is generated by those expressions $\Sqh^I\Sqv^Jv$, for $v\in V^{s_{n},\ldots,s_1}_t$ running through a basis of $V$, and appropriate $\Sq$-admissible sequences $J$ and $I$. The codomain is generated by expressions $\Sqh^Kv$, for $v\in V^{s_{n},\ldots,s_1}_t$ running through a basis of $V$, and appropriate $\Sq$-admissible sequences $K$. It is a  combinatorial  exercise in the properties of admissible sequences to show that these sets of generators are put in bijection by $f$, and this bijection sends polynomial generators to polynomial generators and exterior generators to exterior generators.
\end{proof}
\end{Cohomology operations for all unstable Lie algebras}















\begin{Cohomology operations for unstable Lie algebras over P}
\vfil\pagebreak
\section{Cohomology operations for unstable Lie algebras over $P$}\label{Koszul delta opns}

%\subsection{Cohomology for objects of $\calW$}
\tododone{Define $Q^\calW:\calW\to \vect{+}{0}$, and thus the cohomology functor on $\calW$}{Give the bar construction $B^{\calW}$
\item Explain significance as Adams $E^2$}
%We will use the standard simplicial bar construction as our simplicial resolution of $X\in\calW$, writing $B^\calW_sX=U^{s+1}X$. Then the cohomology of $X$, $H^{s}_t(X)\in\vect{1}{+}$, is the cohomology of the cochain complex
%\[C^{s}_{t}=\Hom(N_s(Q^\calW B_{\bullet}X)^{t},\F_2).\]
%We'll frequently use the identification $Q^\calW B_{s}X=U^{s}X$.


\subsection{$\delta$ operations on $H^*_{\calW(0)}$ and $H^*_{\calU(0)}$}
\tododone{Define the operations $\gamma^i$, state and define the $\delta_i$ operations}{still need to clarify the end of the proof of the relations.}
For $V\in \vect{+}{0}$, we will define natural homomorphisms
\[\theta^i:(F_{\calW(0)}V)^{n+i+1}\to V^{n}, \textup{ for $2\leq i<n$}.\]
Indeed, there are natural homomorphisms into the quadratic grading 2 part of $F_{\calW(0)}V$:
\begin{align*}
P^i:V^{n}&\to F_{\calW(0)}^{(2)}V^{n+i+1}, \textup{\quad  for $2\leq i<n$}\\
[\,,]:(S_2V)^{n}&\to F_{\calW(0)}^{(2)}V^{n+1},
\end{align*}
and for given $m\geq1$, the degree $m$, quadratic grading 2 part $F_{\calW(0)}^{(2)}V^m$ decomposes as
%\[UV^{m,(2)}=\left(\bigoplus_{n_1+n_2=m-1}\im ([\,,]_{n_1,n_2})\right) \oplus\left(\bigoplus_{2\leq i< (m-1)/2}\im (P^i_{m-i-1})\right)\]
%
\[F_{\calW(0)}^{(2)}V^{m}=%\left(\bigoplus_{n_1+n_2=m-1}
\im \bigl((S_2V)^{m-1}\overset{[,]}{\to} F_{\calW(0)}^{(2)}V\bigr)%\right) 
\oplus\bigoplus_{\!\!\!\!\!\!2\leq i< (m-1)/2\!\!\!\!\!\!}\im \bigl(V^{m-i-1}\overset{P^i}{\to}F_{\calW(0)}^{(2)}V\bigr).\]
Moreover, each map $P^i:V^n\to F_{\calW(0)}^{(2)}V^{n+i+1}$ appearing in this decomposition is an isomorphism onto its image, so that for $2\leq i <n$ we may construct $\theta^i$ as the composite
\[\theta^i:\left((F_{\calW(0)}V)^{n+i+1}\overset{\textup{proj}}{\makebox[.06ex][l]{$\to$}\to} (F_{\calW(0)}^{(2)}V)^{n+i+1}\overset{\textup{proj}}{\makebox[.06ex][l]{$\to$}\to} \im (P^i)\overset{(P^i)^{-1}}{\to}V^{n}\right).\]
Here we have projected onto the quadratic filtration 2 part, and then further onto the relevant summand in its natural decomposition. Note that although $P^n:V^n\to F_{\calW(0)}^{(2)}V^{2n+1}$ is a nontrivial linear map when $n\geq2$, its image is entangled with the image of the bracket, and we are not able to split it off. Thus we are not able to improve on the bounds $2\leq i< n$.

Now suppose that $X\in\calW(0)$, and consider the bar construction $Q^{\calW(0)}B^{\calW(0)}X$ whose cohomotopy is $H^*_{\calW(0)}X$. Until the end of \S\ref{Koszul delta opns}, we'll write $Q\mathbb{B}$ for this simplicial object, write $F$ for $F_{\calW(0)}$, and identify $Q\mathbb{B}_s$ with $F^sX$.
\begin{prop}
For any $i$ and $t$ such that $2\leq i <t$, the map $\theta^i:(F^{s+1}X)^{t+i+1}\to (F^sX)^{t}$ induces a degree $-1$ chain map $N_{s+1}(Q\mathbb{B}^{t+i+1})\to N_{s}(Q\mathbb{B}^{t})$. These chain maps induce a natural linear operations
\[\delta_i=(\theta^i)^*:(H^*_{\calW(0)}X)^{s}_t\to (H^*_{\calW(0)}X)^{s+1}_{t+i+1}\textup{\ \ whenever\ \ }2\leq i <t,\] 
which satisfy the Adem relation [Dwyer] for $\delta$ operations.
\end{prop}
\begin{proof}
One needs to see that $\theta^i$ is a chain homomorphism respecting the normalization functor.
%\[\xymatrix@R=4mm{
%N_{s+1}(Q^U B_{\bullet}W)^{t+i+1}\ar[r]^-{\theta^i}
%\ar[d]^-{\partial}&%r1c1
%N_s(Q^U B_{\bullet}W)^{t}
%\ar[d]^-{\partial}
%\\%r1c2
%N_{s}(Q^U B_{\bullet}W)^{t+i+1}\ar[r]^-{\theta^i}&%r2c1
%N_{s-1}(Q^U B_{\bullet}W)^{t}%r2c2
%}\]
For this, it is enough to produce commuting squares:
\[\xymatrix@R=4mm{
(F^sX)^{t+i+1}\ar[r]^-{\theta^i}
\ar[d]^-{d_0+d_1}&%r1c1
(F^{s-1}X)^{t}
\ar[d]^-{d_0}
\\%r1c2
(F^{s-1}X)^{t+i+1}\ar[r]^-{\theta^i}&%r2c1
(F^{s-2}X)^{t}%r2c2
}
\raisebox{-5mm}{\ \ and\ \ }
\xymatrix@R=4mm{
(F^sX)^{t+i+1}\ar[r]^-{\theta^i}
\ar[d]^-{d_{j+1}}&%r1c1
(F^{s-1}X)^{t}
\ar[d]^-{d_j}
\\%r1c2
(F^{s-1}X)^{t+i+1}\ar[r]^-{\theta^i}&%r2c1
(F^{s-2}X)^{t}%r2c2
}\raisebox{-5mm}{\ \ for $j\geq 1$.}\]
The later squares are easy. We'll check the first square on an expression $e=f(g_j(h_{jk}))$, for certain expressions $g_j(h_{jk})\in F^{s-1}X$ on certain $h_{jk}$ in $F^{s-2}X$. We find
\[\theta^i d_0e=\epsilon(f)\theta^i (g_j)(h_{jk}),\quad \theta^i d_1e=\theta^i (f(g_j))(h_{jk}),\quad d_0\theta^i e=\theta^i (f)\epsilon(g_j)(h_{jk}).\]
% and and for the first one, starting with $f_{(s)}g_{(s-1)}h_{(s-2)}\in F^sW$, we find $\theta^i d_0=u(f)\theta^i (g)(h_{(s-2)})$,
$\theta^i d_1=\theta^i (fg)(h_{(s-2)})$, and
$d_0\theta^i =\theta^i (f)u(g)(h_{(s-2)})$. 
These three quantities sum to zero, as the single operation $P^i$ is indecomposable. This proves that $\theta^i$ induces a chain map on normalized complexes, and established that the $\delta$-operations are well-defined.

What remains is to check that the Adem relations are satisfied. The necessary argument is a modification of that given in Priddy's work [Koszul], which accommodates for the presence of the extra operations. As the algebra of $\delta$-operations is Koszul dual (in Priddy's sense) to $\Palg$, the Adem relations follow. [$\diamondsuit$ \textbf{Should I include} the compression to the quadratic grading 2 parts?]% For this, show that we can compress to the terms in which there is quadratic grading 2 in each construction. Then the Adem relation is the image under the differential of something obvious.
\end{proof}
\todo{Note that the same method defines operations on $H^*_{\calU(n)}$}{}



\subsection{Horizontal Steenrod operations and a commutative product for $H^*_{\calW}$}
We'll now give a construction of operations on the homology $H^*_{\calW}$ which arise from the Lie structure on elements of $\calW$. {\tiny\textbf{This section made tiny since it duplicates the more general derivation above, although I'll still need to be explicit, especially about the shift.} We'll follow the framework set out by Goerss in the derivation [] of the $\calW$ structure operations on Andr\'e-Quillen homology. The first construction is that of the smash coproduct of objects $X_1,X_2\in\calW$, which is defined to be the kernel of the natural map from the coproduct $X_1\sqcup X_2$ to the product $X_1\times X_2$, so that there is a short exact sequence:
\[\xymatrix@R=4mm{
0\ar[r]
&%r1c1
X_1\wedge X_2\ar[r]
&%r1c2
X_1\sqcup X_2\ar[r]
&%r1c3
X_1\times X_2\ar[r]&0.
}\]
We leave the interested reader to verify:
\begin{lem}
There is a unique map $j_\calW:Q(X_1\wedge X_2)^t\to (QX_1\otimes QX_2)^{t-1}$ such that
\[P^{i_1}\cdots P^{i_r}[z_1,\cdots ,z_a]\overset{j_\calW}{\mapsto}\begin{cases}
z_1\otimes z_2,&\textup{if }r=0,\,a=2,\,z_1\in X_1\textup{ and } z_2\in X_2,
\\0,&\textup{otherwise, }
\end{cases}
\]
where each of $z_1,\ldots,z_a$ is an element of either $X_1$ or $X_2$, and $[z_1,\ldots,z_s]$ represents some choice of bracketing. It is natural in $X_1$ and $X_2$.
\end{lem}
Next, if $X\in s\calW$ is an almost free simplicial object, there is a cogroup object structure $\phi_s:X_s\to X_s\sqcup X_s$ on each level $X_s$ induced by the diagonal map $V_s\to V_s\oplus V_s$ of the generating subspace $V_s$. Write $\overline{\xi}_\calW$ for the difference of the maps $(d_0\sqcup d_0)\phi_s$ and $\phi_{s-1}d_0$ in the group $\textup{Hom}_\calW(X_s,X_{s-1}\sqcup X_{s-1})$. Then $\overline{\xi}_\calW$ factors through a unique map $\xi_\calW:X_s\to X_{s-1}\wedge X_{s-1}$, and one can verify:
\begin{lem}
The map $Q^\calW\xi_\calW$ induces a chain map of (lower) degree $-1$ on normalized complexes:
\[N_s(Q^{\calW}X)^t\to N_{s-1}(Q^{\calW}(X\wedge X))^t.\]
\end{lem}
We now have access to the composite
\[\psi_\calW:\left(N_s(Q^{\calW}X)^t\overset{Q^\calW\xi_\calW}{\to} N_{s-1}(Q^{\calW}(X\wedge X))^t\overset{j_\calW}{\to} N_{s-1}(Q^{\calW}X\otimes Q^\calW X)^{t-1}\right),\]
which is in fact symmetric, giving a map
\[\psi_{\calW}:N_s(Q^{\calW}X)^t\to N_{s-1}(S^2(Q^{\calW}X))^{t-1}.\]

From this map we can construct Steenrod operations and a commutative product on the cohomology groups $H^*_\calW X$. Explicitly, there are natural homomorphisms
\[\Sq^j:H_t^{s}X\to H_{2t+1}^{s+j}X,\]
defined as follows. Suppose that $\alpha\in C_t^{s}$ is a cocycle. Then we define $\Sq^j\alpha$ as the composite
\[N_{s+j}(Q^UB_{\bullet}X)^{2t+1}\overset{\psi_\calW}{\to}N_{s+j-1}((Q^UB_{\bullet}X)^{\otimes2})^{2t}\overset{\mathbb{D}_{s-j+1}}{\to}
((N_*Q^UB_{\bullet}X)^{\otimes2})^{2t}_{2s} \overset{\alpha\otimes\alpha}{\to}\F_2.
\]
Another way to say this, in the style of Goerss (and others), is to define a function
\[\Theta^j:C_{t}^{s}\to C_{2t+1}^{s+j+1},\ \ \Theta^j(\alpha)=\psi^*_\calW\mathbb{D}_{s-j+1}^*(\alpha\otimes\alpha)+ \psi^*_\calW\mathbb{D}_{s-j+2}^*(\alpha\otimes d\alpha).\]
Note the shift of one in the indices in this formula, as compared with similar formulae [Goerss or Dwyer].
Then, (like Goerss shows), $d\Theta^j\alpha=\Theta^jd\alpha$, and we can define $\Sq^j$ to be the map on cohomotopy defined by $\Theta^j$. Similarly, we can define a cochain complex homomorphism
\[\Psi:C_t^{s}\otimes C_{t'}^{s'}\to C_{t+t'+1}^{s+s'+1}\]
by the formula $\psi^*_\calW\mathbb{D}_0^*$, and use this to define a pairing of cohomology.}
\begin{thm}
These operations satisfy the Axioms *write them out*.
\end{thm}
This theorem follows from theorem [BLAH] in the appendix, after the observation that 
%The functor $Q^\calW:\calW\to \vect{+}{0}$ may be written as a composite of two functors:
%\[\calW\overset{Q^\Palg}{\to}\Lambda(\LieOperad)\textup{-mod}\overset{Q^{\LieOperad}}{\to}\vect{+}{0},\]
%where the functor $Q^\Palg$ takes the quotient by the image of every $P$-operation. It follows from the axioms in $\calW$ that this quotient bears the structure of a (good) Lie algebra. 
the simplicial Lie algebra $Q^\PMonad B X$ is an almost free simplicial Lie algebra, so that the dual homotopy of $Q^\calW B X=Q^{\LieOperad} Q^\PMonad B X$ is an instance of Lie algebra cohomology. [\textbf{Is this exactly true?} It doesn't matter much if not... just need to give extensive enough calculation in the appendix, and modify the text around this comment accordingly]

By [Priddy's prim coh ops], Lie algebra cohomology has Steenrod operations and a product. However, the formulae given by Priddy (involving the $\overline{W}$-construction) are not obviously equivalent to what we have given here. However, it is not hard to check that the definition we have given using $\psi_\calW$ is equivalent to an analogous definition for Lie algebra cohomology (given in [placemarker in appendix]), and theorem [BLAH again] identifies these operations with Priddy's.

{\tiny
\textbf{This section tiny because it contains some duplication, although the typesetting of the proposition may end up useful. Maybe it'd be better to make it lokk more like Priddy's format, though.} We can identify the bar construction $Q^UB_sX$ with
$Q^{\Lambda(\LieOperad)}\Lambda(\LieOperad)U^sX$,
which is $Q^{\Lambda(\LieOperad)}$ applied to the almost free simplicial Lie algebra $\Lambda(\LieOperad)U^sX$. As such, its homotopy has operations induced by the map
\[\xi_\LieOperad:\left(Q^{\Lambda(\LieOperad)}\Lambda(\LieOperad)U^{s+1}X
\overset{Q(d_0\sqcup d_0\psi+\psi d_0)}{\to}
Q^{\Lambda(\LieOperad)}((\Lambda(\LieOperad)U^sX)^{\wedge2})\overset{j}{\to}
(Q^{\Lambda(\LieOperad)} \Lambda(\LieOperad)U^sX)^{\otimes2}\right)\]
where by $d_0$ we mean the map $\Lambda(\LieOperad)(P\circ S(\LieOperad)/\sim)\to \Lambda(\LieOperad)$ arising from killing the $P$ part and identifying all the Lie stuff together. This only sees the Lie stuff, not the part that has any $P$ in it. Thus it could be written as $f(g)\mapsto q^\LieOperad(f)(g)$.

The operations $\Sq^j:H_t^{s}X\to H_{2t+1}^{s+j}X$ are defined as follows. Suppose that $\alpha\in C_t^{s}$ is a cocycle. Then we define $\Sq^j\alpha$ as the composite
\[N_{s+j}(Q^UB_{\bullet}X)^{2t+1}\overset{\xi_\calL}{\to}N_{s+j-1}((Q^UB_{\bullet}X)^{\otimes2})^{2t}\overset{\mathbb{D}_{s-j+1}}{\to}
((N_*Q^UB_{\bullet}X)^{\otimes2})^{2t}_{2s} \overset{\alpha\otimes\alpha}{\to}\F_2.
\]
Another way to say this, in the style of Goerss (and others), is to define a function
\[\Theta^j:C_{t}^{s}\to C_{2t+1}^{s+j+1},\ \ \Theta^j(\alpha)=\phi^*_\calL\mathbb{D}_{s-j+1}^*(\alpha\otimes\alpha)+ \phi^*_\calL\mathbb{D}_{s-j+2}^*(\alpha\otimes d\alpha).\]
Then, (like Goerss shows), $d\Theta^j\alpha=\Theta^jd\alpha$, and we can define $\Sq^j$ to be the map on cohomotopy defined by $\Theta^j$. Similarly, we can define a cochain complex homomorphism
\[\Psi:C_t^{s}\otimes C_{t'}^{s'}\to C_{t+t'+1}^{s+s'+1}\]
by the formula $\psi^*_\calL\mathbb{D}_0^*$, and use this to define a pairing of cohomology.
\begin{prop}
There are natural homomorphisms
$\Sq^j:H_t^{s}X\to H_{2t+1}^{s+j}X$,
zero unless $3\leq j\leq s+1$, and satisfying the Adem relations for the Steenrod algebra. There is a natural nonunital commutative algebra pairing
$H_t^{s}X\otimes H_{t'}^{s'}X\to H_{t+t'+1}^{s+s'+1}X.$
These satisfy an unstableness condition:
$x^2=\Sq^{s+1}x\text{ for }x\in H^{s}_tX$
and the standard Cartan formula:
$\Sq^k(xy)=\sum_{i=0}^k(\Sq^ix)(\Sq^{k-i}y)$.
\end{prop}
\begin{proof}
\textbf{To do:} Insert the proof of the Steenrod relations from prliealgs.tex, citing Priddy or really Singer.
\end{proof}
}
\todo{Give theorem relating all the given operations}{prove the $\delta$ Adem relations using Priddy compression
\item show that the Sq-ops and products are instances of those on the cohomology of a Lie algebra (in particular that $j_\calW$ restricts to $j_\calL$), then quote the appendix
\item The commutation relation for $\delta$ vs $\Sq$/product is written on the wall
\item need to figure out what happens in low dimensions}
\subsection{Relations between the horizontal and vertical operations}

Do a Steenrod or product operation, and then a $\delta_i$, and you're using the composite of this map with
\[Q^\calW U^{s+3}X=U^{s+2}X\to U^{s+1}X=Q^\calW U^{s+2}X\]
which all boils down to
\[P^i_{(s+2)}f_{(s+1)}g\mapsto q^{\calL}(f)(g)\]
Anyway, I have a nullhomotopy of this map, which might be called $((P^i)^{-1}q^{\calL})$, and is stuck on the wall in the upper right corner.
\end{Cohomology operations for unstable Lie algebras over P}

\begin{Unstable Lie algebras over the Lambda-algebra}
\vfil\pagebreak
\section{Unstable Lie algebras over the $\Lambda$-algebra (depreciated?)}
\textbf{This section has probably been depreciated by the general section on lie algs in characteristic 2 and their homotopy operations.}
\todoeasy{Define the categories $\calW(n)$, $\calL(n)$ ($n=0$ is Goerss' case)}{$\calW(n)$ and $\calL(n)$ are monadic over $\vect{+}{n}$, but we won't carefully describe the monads.
\item note that partially restricted Lie algs will appear in the appendix}
In preparation for our calculations of cohomology in $\calW$, we introduce here a family of algebraic categories of a similar nature. That is, we will introduce, for each $n\geq 0$, algebraic categories $\calW(n)$ and $\calL(n)$, such that $\calW(0)=\calW$, and for each $n\geq0$:
\begin{itemize}
\setlength{\parindent}{.25in}
\item objects of $\calW(n)$ and $\calL(n)$ are $\vect{+}{n}$-graded;
\item the indecomposables functor $Q^{\calW(n)}:\calW(n)\to \vect{+}{n}$ factors through the indecomposables functor $Q^{\calL(n)}:\calL(n)\to \vect{+}{n}$; and
\item the category $\calW(n+1)$ is equivalent to the category $\calL(n)\PiAlg$ of algebras for the homotopy operations of an object of $s\calL(n)$.
\end{itemize}
The definitions of the categories $\calW(n)$ for $n\geq1$ are all of a similar flavor, and a little different to the definition above of $\calW(0)=\calW$. 

For $n\geq1$, $\calL(n)$ is a category of (good) Lie algebras, with a restriction operation partially defined. An object $Y$ of $\calL(n)$ is a Lie algebra with bracket
\[[\DASH,\DASH]:Y^t_{s_n,\ldots,s_1}\otimes Y^q_{p_n,\ldots,p_1}\to Y^{t+q+1}_{s_n+p_n,\ldots,s_1+p_1}\]
(satisfying the Jacobi identity and $[x,x]=0$ for all $x$), and restriction operations
\[\restn{(\DASH)}:Y_{s_n,\ldots,s_1}^t\to Y_{2s_n,\ldots,2s_1}^{2t+1}\]
defined whenever not all of $s_n,\ldots,s_{1}$ are zero.
For $x,y\in Y^t_{s_n,\ldots,s_1}$ and $z\in Y$ homogeneous, the restriction operations satisfy \[\restn{(x+y)}=\restn{x}+\restn{y}+[x,y]\textup{ and }[\restn{x},z]=[x,[x,z]].\]
We have defined the bracket and restrictions on homogeneous elements only, but there is no problem extending these operations to non-homogeneous elements using the bilinearity of the bracket and the above formula for the restriction of a sum.


Building on this definition, an object $X$ of the category $\calW(n)$ is to be an object of $\calL(n)$ admitting certain unstable right $\Lambda$ operations. Namely, whenever $0\leq i\leq s_n$ and not all of $i,s_{n-1},\ldots,s_{1}$ are zero, there is defined an operation
\[(\DASH)\lambda_i:X^t_{s_n,\ldots,s_1}\to X^{2t+1}_{s_n+i,2s_{n-1},\ldots,2s_1}.\]
%The operation $(\DASH)\lambda_i$ is required to be linear whenever $i< s_n$, and 
When $i=s_n$ (and $\lambda_i$ is defined) it is required to equal the restriction:
\[(\DASH)\lambda_{s_n}=\restn{(\DASH)}:X^t_{s_n,\ldots,s_1}\to X^{2t+1}_{2s_n,2s_{n-1},\ldots,2s_1}.\]
In particular, we have already specified that this top $\lambda$-operation is a quadratic refinement \textbf{(acceptable language?)} of the bracket, and how the two operations interact [label that equation]. Contrastingly, the non-top operations are required to be linear, and to be killed by the bracket, so that for any $x,y\in X^t_{s_n,\ldots,s_1}$ and $i<s_n$ (with not all of $i,s_{n-1},\ldots,s_{1}$ being zero) and homogeneous $z\in L$:
\[(x+y)\lambda_i=x\lambda_i+y\lambda_i\textup{ and }[x\lambda_i,z]=0.\]
Finally, these right operations satisfy the $\lambda$-Adem relations. That is, whenever $i>2j$, and $x$ is a homogeneous element of $X$ with $x\lambda_j\lambda_i$ defined:
\[x\lambda_j\lambda_i=\sum_{k=0}^{(i-2j)/2-1}{i-2j-2-k\choose k}x\lambda_{i-j-1-k}\lambda_{2j+1+k}.\]

{\tiny[Moved]Next, we'll define another category, $\LambdaMonad(n)$, whose objects are $\vect{+}{n}$-graded vector spaces with certain unstable $\lambda$ operations. An object $X$ of $\LambdaMonad(n)$ is simply an object of $\vect{+}{n}$ with operations
\[(\DASH)\lambda_i:X^t_{s_n,\ldots,s_1}\to X^{2t+1}_{s_n+i,2s_{n-1},\ldots,2s_1}\]
defined whenever $0\leq i< s_n$ and not all of $i,s_{n-1},\ldots,s_{1}$ are zero. Note the strict inequality $i<s_n$, so that we define all the $\lambda$ operations except for the top $\lambda$ operation. These operations are all assumed to be linear and to satisfy the same $\lambda$-Adem relation.

The objects of $\calW(n)$ may be described as those graded vector spaces with both $\calL(n)$ and $\LambdaMonad(n)$ structure satisfying certain compatibilities. In fact, it is not difficult to show that the free $\calW(n)$ object monad factors as a composite
\[F_{\calW(n)}=\LambdaMonad(n)\circ F_{\calL(n)}:\vect{+}{n}\to \vect{+}{n},\]
and that the monad structure on $F_{\calW(n)}$ arises from a distributive law [{\tiny Beck Dist Laws in seminar on triples and cat homology thy}]. It is important to note that this is monadic down only as far as $\vect{+}{n}$. In particular, we may define a functor $Q^{\LambdaMonad(n)}:\calW(n)\to\vect{+}{n}$, by quotienting by the image of the \emph{non-top} $\lambda$-operations (which, fortunately, are linear). It follows from the axioms in $\calW(n)$ that:
\begin{lem}
For $X\in\calW(n)$, the vector space $Q^{\LambdaMonad(n)}(X)$ inherits the structure of an object of $\calL(n)$ from the bracket and restriction of $X$, yielding a factorization:% of indecomposable functors, $Q^{\calW(n)}=Q^{\calL(n)}\circ Q^{\LambdaMonad(n)}$
\[Q^{\calW(n)}=Q^{\calL(n)}\circ Q^{\LambdaMonad(n)}:\left(\calW(n)\to \calL(n)\to \vect{+}{n}\right).\]
Moreover the composite $Q^{\LambdaMonad(n)}\circ F_{\calW(n)}$ equals the free construction $F_{\calL(n)}$.
\end{lem}
}






































\todoeasy{Derived functors of $Q^{\calW(n)}:\calW(n)\to \vect{+}{n}$ are $H^*_{\calW(n)}:\calW(n)\to \vect{n+1}{+}$}{and we're interested in these for $n=0$.}
\todoeasy{Define functors $Q^P:\calW(0)\to\calL(0)$ and $Q^{\Lambda(n)}:\calW(n)\to \calL(n)$ for $n\geq1$}{explain that when $n>0$ you divide by non-top operations only, and the opposite for $n=0$
\item point is that $Q^{\calW(n)}$ factors as these followed by $Q^{\calL(n)}:\calL(n)\to \vect{+}{n}$}
\todoeasy{Note that $\calL(n)\PiAlg$ is the category $\calW(n+1)$}{comment that this is why $\calW(n)$ will be interesting for $n>0$}
%\todo{Define cohomology $H_{\calW(n)}^*:\calW(n)\to \vect{n+1}{+}$}{}
\todoeasy{Define homology $H^{\Lambda(n)}_*:\calW(n)\to \vect{+}{n+1}$ and enrich codomain to $\calW(n+1)$}{also do $n=0$ case: define homology $H^P_*:\calW(0)\to \vect{+}{1}$ and enrich codomain to $\calW(1)$}
\end{Unstable Lie algebras over the Lambda-algebra}

\begin{Cohomology operations for unstable Lie algebras over Lambda}
\vfil\pagebreak
\section{Cohomology operations for unstable Lie algebras over $\Lambda$}
\todoeasy{Do all the same work as in the $n=0$ case, to get horizontal and vertical $\Sq$ and products}{must explain in detail what the available operations are
\item relations between them less important}
\subsection{Vertical steenrod operations for $H^*_{\calW(n)}$ and $H^*_{\calU(n)}$, $n\geq1$}\label{section: vertical Koszul operations n positive}
Fix $n\geq1$. For $V\in \vect{+}{n}$, we will define natural homomorphisms
\[\gamma_i:(F_{\calW(n)}V)^{2t+1}_{s_n+i-1,2s_{n-1},\ldots,2s_1}\to V^{t}_{s_n,\ldots,s_1},\]
whenever $1\leq i \leq s_n$ and not all of $i-1,s_{n-1},\ldots,s_1$ are zero.
{\tiny Meh: These are rather easier to define than in the $n=0$ case above, as the monad $F_{\calW(n)}$ is a simple composite of monads.}
Indeed, there are natural monomorphisms
\[(\DASH)\lambda_{i-1}:V^{t}_{s_n,\ldots,s_1}\to (F^{(2)}_{\calW(n)}V)^{2t+1}_{s_n+i-1,2s_{n-1},\ldots,2s_1}\]
defined whenever $1\leq i\leq n$ and $i-1,s_{n-1},\ldots,s_1$ are not all zero, and an inclusion
\[\textup{incl}:(F^{(2)}_{\calL(n)}V)^{*}\to F^{(2)}_{\calW(n)}V^{*}.\]
As in the $n=0$ case, the images of the listed maps are linearly independent and span the quadratic grading 2 part of $F_{\calW(n)}V^*$. We define $\gamma_i$ as before --- project onto the quadratic grading 2 part, then further onto the summand corresponding to the image of $\lambda_{i-1}$, and finally use the inverse of the map $\lambda_{i-1}$:
\[\gamma_i:\left((F_{\calW(n)}V)^{2t+1}_{s_n+i-1,2s_{n-1},\ldots,2s_1}\overset{\textup{proj}\circ\textup{proj}}{\makebox[.06ex][l]{$\to$}\to} \im (\lambda_{i-1})\overset{(\lambda_{i-1})^{-1}}{\to}V^{t}_{s_{n,\ldots,s_1}}\right).\]
\begin{prop}
There are natural operations
\[\Sqv^i:(H^*_{\calW(n)}X)^{s_{n+1},\ldots,s_1}_t\to (H^*_{\calW(n)}X)^{s_{n+1}+1,s_n+i-1,2s_{n-1},\ldots,2s_1}_{2t+1}\]
defined whenever $1\leq i \leq s_n$ and not all of $i-1,s_{n-1},\ldots,s_1$ are zero. If $\alpha\in C^{s_{n+1},\ldots,s_1}_t$ is a cocycle, $\Sqv^i\alpha$ is the composite
\[N_{s_{n+1}+1}(Q^U B_{\bullet}X)^{2t+1}_{s_{n}+i-1,2s_{n-1},\ldots,2s_1}\overset{\gamma_i}{\to}N_{s_{n+1}}(Q^U B_{\bullet}X)^{t}_{s_{n},s_{n-1},\ldots,s_1}\overset{\alpha}{\to}\F_2,\]
after identifying $Q^U B_{s+1}X$ with $U^{s}X$ and $Q^U B_{s}X$ with $U^{s+1}X$. These operations satisfy the Adem relation for Steenrod operations.
\end{prop}
\begin{proof}
The proof here is obtained by modification of the corresponding proof in the $n=0$ case. Note that we use the Koszul duality between the Steenrod and $\Lambda$ algebras (Priddy Koszul Resolutions). \textbf{One might} make a note of the fact that all of the necessary relation terms are actually available --- nothing is ruled out due to dimension issues.
\end{proof}
\todo{Note that the same method defines operations on $H^*_{\calU(n)}$}{}
\subsection{Horizontal steenrod operations and a commutative product for $H^*_{\calW}$}
Say `it follows as in the $n=0$ case'

\end{Cohomology operations for unstable Lie algebras over Lambda}
\begin{Koszul complexes}
\vfil\pagebreak

\section{\textbf{Koszul complexes calculating $\calU(n)$-homology}}
In this section we'll discuss the Koszul resolutions that one may use to calculate $H_*^{\calU(n)}X$ for $X$ a (non-simplicial) locally finite object of $\calU(n)$ or $\calW(n)$, using Priddy's technique \cite{PriddyKoszul.pdf}, adapted to an unstable context, as in \cite{CurtisSimplicialHtpy.pdf}.

\subsection{The (co-)Koszul complex}
Write $\Nop_*,N_*^\div,C_*$ for three of the four chain complexes associated with $Q^{\calU(n)}B^{\calU(n)}X$. Each of these complexes admits an increasing filtration, the length filtration, with $F_\ell C_s$ generated by those terms in which there are at most $\ell$ generators appearing in the $s$ free constructions in $C_s= Q^{\calU(n)}B^{\calU(n)}_sX\cong (F^{\calU(n)})^sX$, and the filtrations on $\Nop_*$ and $N^\div_*$ induced by that on $C_*$. Note that $F_{s-1}N^\div_s=0$, so that  $F_{s-1}\Nop_s=0$. Note further that $d(F_sN_s^-)\subseteq F_{s-1}N_{s-1}^-$, which is easily checked on the isomorphic filtered complex $N_*^\div$.

Write $E^r_{\ell,s}$ for the spectral sequence of the filtered complex $\Nop_*X$, so that $E^0_{\ell,p}$ is the associated graded complex. As $F_{s-1} \Nop_s=0$, $E^r_{\ell,s}=0$ for $\ell<s$, and $E^0_{s,s}$ is the subspace $F_s\Nop_s$ of $\Nop_s$. The $d_0$-differential vanishes on $E^0_{s,s}$ since $d(F_sN_s^-)\subseteq F_{s-1}N_{s-1}^-$ (\textbf{?????}). We conclude that $E^1_{s,s}=F_s\Nop_s$, with $d^1$-differential $E^1_{s,s}\to E^1_{s-1,s-1}$ identified with the differential of $\Nop_*$. On the other hand, Priddy shows that $E^1_{\ell,s}=0$  for $\ell>s$. Thus, the groups
\[K_s^{\calU(n)}X:=E^1_{s,s}=d^{-1}(F_{s-1}\Nop_{s-1})=F_s\Nop_s\]
form a subcomplex of $\Nop_*$ whose inclusion is a homotopy equivalence. Rather than determining these groups directly, Priddy's theory works with their duals, $K^*_{\calU(n)}X$, which form a cochain complex with homology $H^*_{\calU(n)}X$. In fact, Priddy's theory shows that the cochain complex $K^*_{\calU(n)}X$, the \emph{co-Koszul complex}, is actually a differential unstable left module over the same operations as its cohomology $H^*_{\calU(n)}X$, and indeed that this unstable module is \emph{free}. More precisely,  $K_0^{\calU(n)}X= X$, and $K^*_{\calU(n)}X$ is free on the  subspace $X^*$.
\begin{prop}\label{the cokoszul complex is free}
Suppose that $n\geq0$, and $X$ is a locally finite object of $\calU(n)$. The chain maps $\widetilde{\theta^i}$ ($\widetilde{\theta_i}$ when $n=0$) on $C_*Q^{\calU(n)}B^{\calU(n)}X$ restrict to the subcomplex $K_*^{\calU(n)}X$, and induce an $\calMv(n+1)$-structure on $K^*_{\calU(n)}X$ which commutes with the differentials. The inclusion $X^*\cong K^0_{\calU(n)}X\subseteq K^*_{\calU(n)}X$ induces an isomorphism $F_{\calMv(n+1)}(X^*)\to K^*_{\calU(n)}X$. Moreover, this $\calMv(n+1)$-structure on $K^*_{\calU(n)}$ induces the $\calMv(n+1)$-structure on $H^*_{\calU(n)}X$ of [above]. 
\end{prop}


%\subsection{The co-Koszul complex for $H_*^{\calU(0)}$}
Although it is easier to calculate the co-Koszul complex, we will need to understand the Koszul complex itself, in order to calculate the $\calW(n+1)$-structure of $H_*^{\calU(n)}$. For this, we'll introduce a little notation. Firstly, we'll return to the convenient bar notation from which the bar construction gets its name, c.f.\ \cite{PriddyKoszul.pdf}. Next, we'll write $\produces{J}{I}{\deltaalg}$ \cancel{whenever $J\neq I$} and the element $\delta_J\in\deltaalg$ contains the term $P^I$ when written in the admissible basis. \textbf{Clarify that $\leftarrow$.}

\begin{prop}\label{propDerivedIndTrivialUobject n=0}
Suppose that $X\in\calU(0)$ has homogeneous basis $B$. Then $K_*^{\calU(0)}X$ has basis
\[\left\{\delta(I,b)\ \middle|\ b\in B^t\textup{, $I$ $\delta$-admissible with }\minDimP(I)\leq t\right\},\]
where we define
\[\delta(I,x):=
%\cancel{\left[P^{i_\ell}\middle|\cdots\middle|P^{i_1} \right]x}+
\sum_{\produces{K}{I}{\deltaalg}}\left[P^{k_\ell} \middle|\cdots\middle|P^{k_1} \right]x\textup{ for $x\in X^t$}.\]
%Supposing that $X$ is locally finite, let $B^*\subset X^*$ represent the basis dual to $B$, and denote by $b^*\in B^*$ the element dual to $b\in B$. Then $K^*_{\calU(0)}X$ has basis $\{\delta_I(b^*)\}$, and this basis is dual to the basis just presented.
%
If $X$ is locally finite, there is a basis $\{\delta_I(b^*)\}$ of $K^*_{\calU(0)}X$ constructed using proposition \ref{the cokoszul complex is free}, lemma \ref{basis of element of M(0)} and the basis of $X^*$ dual to $B$. The bases $\{\delta_I(b^*)\}$ and $\{\delta(I,b)\}$ are dual.

The differential of $K^{\calU(0)}_*X$ is given by the formula:
\[d(\delta(I,x))=\sum_{ \substack{\produces{K}{I}{\deltaalg}\\(k_{\ell},\ldots,k_2){\,\deltaalg\textup{-admis.}}}}\!\!\!\!\!\!\!\!\!\!\!\! \delta((k_{\ell},\ldots,k_2),P^{k_1}x),\]
summing over those $K=(k_{\ell},\ldots,k_1)$ such that $(k_{\ell},\ldots,k_2)$ is $\delta$-admissible, and yet $\produces{K}{I}{\deltaalg}$.
\end{prop}
Note that the class $\delta(I,x)$ is still defined for $x\in X^t$ and $\minDimP(I)>t$, it is simply equal to zero. This definition is well behaved, since $\delta$-Adem relations only decrease $\minDimP$. Indeed, we may further restrict the two sums appearing in this proposition be requiring that $\minDimP(K)\leq t$ in each case, but there's no need. Dually, in the coKoszul complex, the operations $\delta_i$ are \emph{undefined} when out of range!
\begin{proof}\textbf{Need something here showing that this sum is actually finite. Same in next proof, need well definedness.}
Firstly, we may assume that $X$ is locally finite, as any object of $\calU(0)$ is the union of its locally finite subobjects. It is enough to check that $\delta(I,b)$ is in fact a member of $\Nop_*$, not just of $C_*$, as then the collection $\delta(I,b)$ will evidently be the dual basis to the $\delta_I(B^*)$: in the sum defining $\delta(I,b)$, the only $\delta$-admissible sequence $K$ appearing is $K=I$.  % the basis element $\delta_J(b^*_2)\in K^*$ pairs nontrivially with $\delta(I,b_1)\in K_*$ if and only if $I=J$ and $b_1=b_2$. 

Using \cite[Lemma 3.2]{PriddyKoszul.pdf}, to check that $\delta(I,b)\in\Nop_*$, we only need to check that $d(\delta(I,b))\in F_{s-1}C_{s-1}$.
To check this membership condition is to check that $\delta(I,b)$ pairs to zero with $\im(d^*:(F_sC_{s-1})^*\to(F_sC_s)^*)$. Priddy's proof shows that $(F_sC_s)^*$ is spanned by functionals $[(P^{k_s})^*|\cdots |(P^{k_1})^*]f$, for $f\in X^*$.  This functional evaluates to zero on $\delta(I,b)$ unless $f(b)=1$ and $\produces{K}{I}{\deltaalg}$. However, the image of $d^*$, as determined by Priddy, is spanned by the space of `$\delta$-Adem relations' (see \cite[Theorem 2.5 and proof]{PriddyKoszul.pdf}). These evaluate to zero on any $\delta(I,b)$, tautologically: in reducing a $\delta$-relation to a sum of $\delta$-admissible expressions using $\delta$-relations, one obtains zero.

\textbf{Alternative ending: }To check the membership condition is to check that $\delta(I,b)$ pairs to zero with $\im(d^*:(F_sN_{s-1})^*\to(F_sN_s)^*)$. Priddy's proof shows that $(F_sN_s)^*$ is spanned by functionals $[(P^{k_s})^*|\cdots |(P^{k_1})^*]b^*$. %Viewing $\delta(I,c)$ as an element of the double dual,
\[\left([(P^{k_s})^*|\cdots |(P^{k_1})^*]b^*\right)\left(\delta(I,c)\right)= b^*(c)\cdot\left(\textup{$\delta_I$ coeff.\ of when $\delta_K\in\deltaalg$}\right).\]
% $\delta(I,c)$ is the functional which sends, and this functional evaluates to zero on $\delta(I,b)$ unless $b=b_2$ and $\produces{K}{I}{\deltaalg}$. 
However, the image of $d^*$, as determined by Priddy, is spanned by the space of `$\delta$-Adem relations' (see \cite[Theorem 2.5 and proof]{PriddyKoszul.pdf}), and these tautologically evaluate to zero on any $\delta(I,b)$.
\end{proof}
Now the same analysis applies in the $n\geq1$ case. Here, we write the bar construction on the right, as we have a right action. This turns back into a left action, as the Lie Steenrod algebra is Koszul dual to the \emph{opposite} of the $\Lambda$-algebra, with an index shift, so that $\Sq^i$ corresponds to $\lambda_{i-1}$ for $i\geq1$ \cite[\S7.1]{PriddyKoszul.pdf}.
\begin{prop}\label{propDerivedIndTrivialUobject n at least 1}
Suppose that $n\geq1$ and $X\in\calU(n)$ has homogeneous basis $B$. Then $K_*^{\calU(n)}X$ has basis
\[\left\{\Sq_{\textup{v}}(J,b)\ \middle|\ \genfrac{}{}{0pt}{}{b\in B_{s_{n},\ldots,s_1}^t\textup{, $J$ $\Sq$-admissible with }\minDimSq(J)\leq s_n,}{\textup{if }s_{n-1}\!=\!\cdots\!=\!s_1\!=\!0\textup{ then $J$ doesn't contain 1}}\right\}.\]
where we define
\[\Sqv(J,b):=
%\cancel{\left[P^{i_\ell}\middle|\cdots\middle|P^{i_1} \right]b}+
\sum_{\produces{K}{J}{\Sq}}b\left[\lambda_{k_1-1} \middle|\cdots\middle|\lambda_{k_\ell-1} \right],\]
but only when $J$ and $b$ satisfy the conditions on $\minDimSq(J)$ and on the appearance of 1 in $J$.
If $X$ is locally finite, this basis is dual to the $\{\Sqv^J\}$ basis of $K^*_{\calU(n)}X$, as in proposition \ref{propDerivedIndTrivialUobject n=0}. The differential of $K^{\calU(n)}_*X$ is given by the formula:
\[d(\Sqv(I,x))=\sum_{ \substack{\produces{K}{I}{\Sq}\\(k_{\ell},\ldots,k_2){\,\Sq\textup{-admis.}}}}\!\!\!\!\!\!\!\!\!\!\!\! \Sqv((k_{\ell},\ldots,k_2),x\lambda_{k_1-1}),\]
summing over $K=(k_{\ell},\ldots,k_1)$ such that $(k_{\ell},\ldots,k_2)$ is $\Sq$-admissible, $K$ doesn't contain 1 if $s_{n-1}=\cdots s_1=0$, and yet $\produces{K}{I}{\Sq}$.
\end{prop}
[\textbf{On well definedness of the term $\Sqv(J,b)$:} {Search for }``Lemma \ref{lemOnAdemChangeInM} demonstrates that if $I$ and $J$ are sequences of'' in PRLieAlgs.
In this case, the condition $\produces{K}{J}{\Sq}$ forces \textbf{[look above, ineq \& nonzero]}, so that every term appearing in the sum is automatically nonzero. On the other hand, the class $\Sqv(J,b)$ is only defined under certain conditions. This adds up to \textbf{further evidence} that the Steenrod operations are all defined, just with some zero.]
\subsection{The $\calW(n+1)$-structure on $H_*^{\calU(n)}X$ for $X\in\calW(n)$}\label{section on structure on homology of koszul cx}
Suppose that $X\in\calW(n)$. We now have two monomorphic quasi-isomorphisms of chain complexes with homology is $H_*^{\calU(n)}X$, and we denote their composite $\jmath$:
\[\jmath:\left(K_*^{\calU(n)}X\subseteq \Nop_*Q^{\calU(n)}B^{\calU(n)}X\subseteq \Nop_*Q^{\calU(n)}B^{\calW(n)}X\right).\]
Now $H^{\calU(n)}_*X$ is an element of $\calW(n+1)$, since it can be calculated as the homotopy of $Q^{\calU(n)}B^{\calW(n)}X\in s\calL(n)$, and this is the structure needed for the composite functor spectral sequences discussed above. We will go some way to calculating this structure in this section --- our method will be to take classes in the Koszul complex, map them into the large complex using $\jmath$, perform the operations in question, and then homotope the outcome back into the Koszul complex.

We'll need a little notation for elements of the various bar constructions. We'll label the $s+1$ free constructions in $B^{\calW(n)}_{s}X$ with subscripts in angle brackets: 
\[B^{\calW(n)}_{s}X= F^{\calW(n)}_{\langle -1\rangle}F^{\calW(n)}_{\langle 0\rangle}\cdots F^{\calW(n)}_{\langle s-1\rangle}X\]%corrected
so that we can then indicate in which free construction operations are being performed. For example, when $n=0$ and $x,y\in X$, $B_2^{\calW(0)}X$ contains an element
\[[P^i_{\langle 0\rangle}x,P^j_{\langle 1\rangle}y]_{\langle -1\rangle}:=[\eta P^i\eta^2 x,\eta^2P^j\eta y]\]%corrected
where we write $\eta:V\to F_{\calW(n)}$ for the unit of the monad on $\vect{+}{n}$ (omitting as always the forgetful functor). That is: we apply $P^j$, not to $y\in X$, but rather to $\eta y$, the corresponding generator of $F_{\langle 1\rangle}^{\calW(n)}X$; we only apply $P^i$ to $\eta^2 x$, the generator of $F^{\calW(n)}_{\langle 0\rangle}F^{\calW(n)}_{\langle 1\rangle}X$; and finally the bracket is taken in the outermost construction.%corrected

With this notation in hand, the map $\jmath$ is induced by the assignment
\begin{alignat*}{2}
[P^{i_s}|\cdots |P^{i_1}]x&\longmapsto P_{\langle 0\rangle}^{i_s}P_{\langle 1\rangle}^{i_s-1}\cdots P_{\langle s-1\rangle}^{i_1}x &\quad&(\textup{if }n=0),\\
x[\lambda_{i_1}|\cdots |\lambda_{i_s}]&\longmapsto x\lambda_{i_1\langle s-1\rangle}\cdots \lambda_{i_{s-1}\langle1\rangle}\lambda_{i_s\langle 0\rangle} &\quad&(\textup{if }n\geq1).%both corrected
\end{alignat*}
Before making calculations, we recall the formulae for Lie algebra homotopy operations [Curtis]. Let $\Shuffles{p}{q}$ be the set of $(p,q)$-shuffles, that is, pairs $(\alpha,\beta)$ where $\alpha=(\alpha_{p-1},\ldots,\alpha_0)$ and $\beta=(\beta_{q-1},\ldots,\beta_0)$ are disjoint monotonically decreasing sequences that together partition the set $\{0,\ldots,p+q-1\}$. Let $s_{\alpha}$ denote the iterated degeneracy operator $s_{\alpha_{p-1}}\cdots s_{\alpha_0}$.
Finally, let $\HalfShuffles{i}{i}$ denote the subset of $\Shuffles{i}{i}$ consisting of those shuffles $(\alpha,\beta)\in\Shuffles{i}{i}$ such that $\beta_{i-1}=2i-1$. The formulae of \cite[\S8]{CurtisSimplicialHtpy.pdf}, for $z\in ZK_p(X)$ and $w\in ZK_q(X)$ cycles representing classes $\overline{z},\overline{w}\in H_*^{\calU(n)}X$ are as follows:
\begin{alignat*}{3}
[\overline{z},\overline{w}]\textup{ is represented by }&&\sum_{(\alpha,\beta)\in\Shuffles{p}{q}}[s_\beta(\jmath z), s_\alpha(\jmath w)]_{\langle -1\rangle}&\in Q^{\calU(n)}B^{\calW(n)}_{p+q}X,\\
\overline{z}\lambda_i\textup{ is represented by }&&\sum_{(\alpha,\beta)\in\HalfShuffles{i}{i}}[s_\beta(\jmath z), s_\alpha(\jmath z)]_{\langle -1\rangle}&\in Q^{\calU(n)}B^{\calW(n)}_{p+i}X,&\quad&(0<i\leq p),\\
\overline{z}\lambda_0\textup{ is represented by }&&\restnwithsubscript{(z)}{\langle -1\rangle}&\in Q^{\calU(n)}B^{\calW(n)}_{p}X,&&(\textup{when defined}).
\end{alignat*}%all three corrected
It will be important to understand these sums. Suppose that $z\in ZK_p^{\calU(n)}X$ (for $n\geq1$). Then  $z$ may be written as a sum of terms of the form $x\lambda_{i_1\langle p-1\rangle}\cdots \lambda_{i_p\langle 0\rangle}$, and
\begin{lem}\label{what degens do to iterated operations}
If $(\alpha,\beta)\in\Shuffles{p}{q}$, then $s_\beta(x\lambda_{i_1\langle p-1\rangle}\cdots \lambda_{i_p\langle 0\rangle})=x\lambda_{i_1\langle \alpha_{p-1}\rangle}\cdots \lambda_{i_p\langle \alpha_0\rangle}$.
\end{lem}
We'll also need the following consequence of the simplicial identities:

\begin{lem}\label{LemmaOnSimplicialRelations}
Choose $i\geq1$ and $\alpha=(\alpha_{p-1},\ldots,\alpha_0)$ with $\alpha_{p-1}>\cdots >\alpha_0\geq0$.
\begin{enumerate}[i)]\squishlist
\setlength{\parindent}{.25in}
\item[i)] If neither $i-1$ nor $i-2$ appear in $\alpha$, then  $d_{i-1}s_\alpha=s_{\alpha'}d_{i'}$ for some $\alpha'$ and $i'$.
\item[ii)] If exactly one of $i-1$ and $i-2$ appears in $\alpha$, then  $d_{i-1}s_\alpha$ does not depend on which of $i-1$ and $i-2$ appeared in $\alpha$.
\end{enumerate}
\end{lem}

\begin{prop}\label{LieBracketsTrivial}
The Lie bracket $H_p^{\calU(n)}X\otimes H_q^{\calU(n)}X\to H_{p+q}^{\calU(n)}X$ vanishes except when $p+q=0$. 
The Lie algebra structure on $H_0^{\calU(n)}X$ is induced by that on $X$: if $z,w\in X$ represent $\overline{z},\overline{w}\in H_0^{\calU(n)}X$, then $[\overline{x},\overline{y}]$ is represented by the cycle $[x,y]\in ZC_0(Q^{\calU(n)}B^{\calW(n)}X)$.
\end{prop}
\noindent This theorem shows that $H_*^{\calU(X)}$ is trivial in positive dimensions \emph{as a Lie algebra}, but the \emph{restriction} need not be trivial (c.f.\ propositions \ref{QkTrivial} and \ref{Q0ZeroByPriddyAlg}).
\begin{proof}
We'll give the proof for $n\geq1$, but it works the same way for $n=0$. In fact, when $n=0$ we can ignore all discussion of top and non-top operations.

Use the abbreviation $\mathbb{B}:=Q^{\calU(n)}B^{\calW(n)}X\in s\calL(n)$. Then $\mathbb{B}$ is almost free on the subspaces $V_s=F^{\calW(n)}_{\langle 0\rangle}\cdots F^{\calW(n)}_{\langle s-1\rangle}X$. Choose representatives $z\in ZK_p^{\calU(n)}X$ and $w\in ZK_q^{\calU(n)}X$. Now  for any $(\alpha,\beta)\in\Shuffles{p}{q}$, the elements $s_{0}s_\beta(\jmath z)$ and $s_{0}s_\alpha(\jmath w)$ both lie in $V_{p+q+1}$, and it is only a minor abuse of notation to define:
\[a:=\sum_{(\alpha,\beta)\in\Shuffles{p}{q}}[s_{0}s_\beta(\jmath z), s_{0}s_\alpha(\jmath w)]_{\langle 0\rangle}\in C_{p+q+1}\mathbb{B}.\]
Using the simplicial identity $d_0s_0=\Id$, we have $d_{0}a=\sum [s_\beta(\jmath z), s_\alpha(\jmath w)]_{\langle -1\rangle}$, the representative given for $[\overline{z},\overline{w}]$. Moreover, we'll find that $d_ia=0$ for $i>0$, except when $p=q=0$, in which case $d_1a=[x,y]$. Thus, in either case, $a$ is the required homotopy in $C_*\mathbb{B}$.

Using the simplicial identity $d_1s_0=\Id$, we have $d_{1}a=\sum [s_\beta(\jmath z), s_\alpha(\jmath w)]_{\langle 0\rangle}$. Now for every pair $(\alpha,\beta)$ indexing this sum, unless $p=q=0$, one of $\alpha$ or $\beta$, say $\beta$, will contain $0$. Then by lemma \ref{what degens do to iterated operations}, every summand in $s_{\alpha}(\jmath z)$ is in the image of some \emph{non-top} $\lambda_{i\langle 0\rangle}$, and as $[x\lambda_i,y]=0$ whenever $\lambda_i$ is not a top operation, the entire expression vanishes in the construction $F^{\calW(n)}_{\langle 0\rangle}$.

What remains is to show that $d_{i}a=0$ for $2\leq i\leq p+q+1$. As $d_is_0=s_0d_{i-1}$ for $i\geq2$:
\[d_{i}a=\sum [s_{0}d_{i-1}s_\beta(\jmath z), s_{0}d_{i-1}s_\alpha(\jmath w)]_{\langle 0\rangle}.\]
For this, we'll define an involution $\rho_i$ of the set $\Shuffles{p}{q}$ indexing the sum, for $2\leq i\leq p+q+1$.
If $\alpha$ and $\beta$ do not each contain exactly one of $i-1$ and $i-2$, then $\rho_i$ fixes $(\alpha,\beta)$. Otherwise, $\rho_i$ interchanges the positions of $i-1$ and $i-2$ in $(\alpha,\beta)$. To avoid confusion, we note that $\rho_{p+q+1}$ is the identity, as neither $\alpha$ nor $\beta$ ever contain $p+q$.

If $(\alpha,\beta)$ is a fixed point of $\rho_i$, then one of $\alpha$ and $\beta$, say $\alpha$, contains neither of $i$ and $i-1$. Then by lemma \ref{LemmaOnSimplicialRelations}(i), $d_{i-1}s_\alpha(\jmath w)=s_{\alpha'}d_{i'}(\jmath w)=0$, as $\jmath w\in Z\Nop_*$. Thus, the summands corresponding to fixed points vanish.
On the other hand, given a shuffle $(\alpha,\beta)$ which is not fixed by $\rho_i$, lemma \ref{LemmaOnSimplicialRelations}(ii) shows that the summand corresponding to $(\alpha,\beta)$ equals the summand corresponding to $\rho_i(\alpha,\beta)$, so these two summands cancel with each other.
\end{proof}
\begin{prop}\label{QkTrivial}
Suppose that $z\in ZK^{\calU(n)}_{s_{n+1}}(X)_{s_n,\ldots,s_1}^t$. If $n>0$, $\overline{z}\lambda_k=0$ for $1\leq k\leq {s_{n+1}}$. If $n=0$, writing $s=s_1$ we have $\overline{z}\lambda_k=0$ for $2\leq k\leq {s}$, and $\lambda_1$ may be defined at the level of the Koszul complex by the assignment
\[z=\textstyle\sum_{j}P(i_{s}^{(j)},\ldots,i_1^{(j)})x^{(j)}{\longmapsto}\textstyle\sum_{j}P(t,i_{s}^{(j)},\ldots,i_1^{(j)})x^{(j)}\in ZK^{\calU(n)}_{s+1}(X)^{2t+1},\]
where we have written $(i_{s}^{(j)},\ldots,i_1^{(j)})$ for the various $\delta$-admissible sequences corresponding to the summands of $z$. That is, $\lambda_1$ adjoins a top operation to the left of each sequence $I^{(j)}$.
\end{prop}
\begin{proof}
We'll first prepare for the calculation of $\lambda_1$ in case $n=0$.
Write $e$ for the proposed representative $\sum_{j}P(t,i_{s}^{(j)},\ldots,i_1^{(j)})x^{(j)}$ of $\overline{z}\lambda_1$. This $e$ does in fact lie in the Koszul complex: the sequences $(t,i_{s}^{(j)},\ldots,i_1^{(j)})$ are indeed $\delta$-admissible, since
\[t=(i_{s}+1)+|P(i_{s-1},\ldots,i_1)x^{(j)}|\geq2i_{s}+1,\]
since we have demanded that $\minDimP(i_{s},\ldots,i_1)\leq|x^{(j)}|$. Then, as long as the image $\jmath e$ in $\Nop_{s+1}Q^{\calU(n)}B^{\calW(n)}X^{2t+1}$ is homotopic to the standard representative of $\overline{z}\lambda_1$, the calculation for $\lambda_1$ is complete. This chain is given by the formula
\[\!\!\!\!\sum_{j,\,\produces{K}{(t,i_{s}^{(j)},\ldots,i_1^{(j)})}{\deltaalg}}\!\!\!\!\left[P^{k_{s+1}} \middle|\cdots\middle|P^{k_1} \right]x^{(j)}
 =\!\!\!\!\sum_{j,\,\produces{H}{(i_{s}^{(j)},\ldots,i_1^{(j)})}{\deltaalg}}\!\!\!\!\left[P^t\middle|P^{h_{s}} \middle|\cdots\middle|P^{h_1} \right]x^{(j)}=P^{t}_{\langle 0\rangle}s_0(\jmath z),\]
where the equality here comes from the observation that for each $j$:
%\[\produces{(k_{s_1+1},\ldots,k_1)}{(t,i_{s}^{(j)},\ldots,i_1^{(j)})}{\deltaalg}\textup{ iff }k_{s_1+1}=t\textup{ and }\produces{(k_{s_1},\ldots,k_1)}{(i_{s}^{(j)},\ldots,i_1^{(j)})}{\deltaalg}\]
\[\textup{if }\produces{(k_{s+1},\ldots,k_1)}{(t,i_{s}^{(j)},\ldots,i_1^{(j)})}{\deltaalg}\textup{ and }k_{s+1}\neq t\textup{, then }\minDimP(k_{s+1},\ldots,k_1)>|x^{(j)}|.\]
To understand this observation, as $\delta$-Adem relations only decrease $\minDimP$ [above somewhere], we may reduce to the case where $(k_s,\ldots,k_1)$ is already $\delta$-admissible and $\produces{(k_{s+1},k_{s})}{(t,k_{s+1}+k_{s}-t)}{\deltaalg}$. Then
\begin{alignat*}{2}
\minDimP(k_{s+1},\ldots)&\geq k_{s}-(k_{s-1}+1)-(k_{s-2}+1)-\cdots  \\
&> t-(k_{s+1}+k_s-t+1)-(k_{s-1}+1)-(k_{s-2}+1)-\cdots  \\
&=t-i^{(j)}_{s}-\cdots -s=|x^{(j)}|
\end{alignat*}
where: the non-strict inequality is by definition of $\minDimP$; the strict inequality follows by examining the $\delta$-Adem relation for $\delta_{k_{s+1}}\delta_{k_{s}}$; the first equality holds as $\deltaalg$ is graded by the sum of the indices; and the second equality holds as $t$ is the top operation, so that the maximum $\minDimP(t,i^{(j)}_{s},\ldots)$ is attained at its final argument.

With this in hand, we return the general case, $1\leq k\leq p$ and $n\geq0$, our goal being to produce a nullhomotopy, except when $n=0$ and $k=1$, when we need a homotopy to $P^t_{\langle 0\rangle}s_0(\jmath z)$. We proceed as in the previous proof, defining
\[a:=\sum_{(\alpha,\beta)\in\HalfShuffles{k}{k}}[s_{0}s_\beta(\jmath z), s_{0}s_\alpha(\jmath z)]_{\langle 0\rangle}\in C_{p+k+1}\mathbb{B}_{2s_n,\ldots,2s_1}^{2t+1}.\]
Then $d_0a$ is the representative for $\overline{z}\lambda_k$, and $d_1a=0$ as in the previous proof (with no analogue of the special case $p=q=0$). Now consider the same involutions $\rho_i$ as in the previous proof, now action on $\Shuffles{k}{k}$. When $2\leq i< 2k$, these $\rho_i$ preserves $\HalfShuffles{k}{k}$. When $2k<i\leq p+k+1$, $\rho_i$ is the identity, so preserves $\HalfShuffles{k}{k}$ trivially. Thus, $d_ia=0$ for all $2\leq i\leq p+k+1$ with $i\neq2k$, as  the cancellations still all occur within the smaller sum
\[d_{i}a=\sum_{(\alpha,\beta)\in\HalfShuffles{k}{k}} [s_{0}d_{i-1}s_\beta(\jmath z), s_{0}d_{i-1}s_\alpha(\jmath z)]_{\langle 0\rangle}.\]
To address the question of $\rho_{2k}$, we define an alternative involution $\widetilde{\rho}_{2k}$ of $\Shuffles{k}{k}$ as follows.  If $\alpha$ and $\beta$ do not each contain exactly one of $2k-2$ and $2k-1$ each, then $\widetilde{\rho}_{2k}$ fixes $(\alpha,\beta)$. Otherwise, we define $\widetilde{\rho}_{2k}(\alpha,\beta):=\rho_{2k}(\beta,\alpha)$, which is to say that $\widetilde{\rho}_{2k}$ swaps \emph{everything but} $2k-2$ and $2k-1$.

Now the summands in this formula exhibit an extra symmetry, given that $z$ is repeated. This symmetry, along with lemma \ref{LemmaOnSimplicialRelations}(ii), shows that all the summands corresponding to shuffles not fixed by $\widetilde{\rho}_{2k}$ cancel out. When $k>1$, the fixed points of $\widetilde{\rho}_{2k}$ are only those shuffles in which one of $\alpha$ and $\beta$ contains neither $2k-2$ nor $2k-1$, and these summands vanish, by \ref{LemmaOnSimplicialRelations}(i), as in previous arguments. When $k=1$, however, $\widetilde{\rho}_{2k}$ has an \emph{extra} fixed point, the shuffle $((0),(1))$, which fails to differ from its image under $\widetilde{\rho}_{2k}$.

In sum, we have shown that $d_0a=0$ represents $\overline{z}\lambda_i$, and that $d_ia=0$ whenever $1\leq i\leq p+k+1$, except when $k=1$ and $i=2$. In this case, we perform the final calculation:
\[d_2a=[s_{0}d_1s_1(\jmath z), s_{0}d_1s_0(\jmath z)]_{\langle 0\rangle}=[s_{0}(\jmath z), s_{0}(\jmath z)]_{\langle 0\rangle}=\begin{cases}
0,&\textup{if }n\geq1,\\
P^{t}_{\langle 0\rangle}s_0(\jmath z),&\textup{if }n=0.
\end{cases}\]
That is, if $n\geq1$, this self-bracket vanishes (an object of $\calW(n)$ for $n\geq1$ is a Lie algebra), while if $n=0$, the self-bracket is equal to the top $P$-operation, in this case $P^t$. %Thus, for $n\geq1$, we have $\overline{z}\lambda_1=0$ as needed, and for $n=0$, we have shown that $\overline{z}\lambda_1=0$ may be represented by $P^{t}_{\langle 0\rangle}s_0(\jmath z)$, and one simply observes. %The result is then proven for $n\geq1$, as $a$ turns out to be a nullhomotopy for all $k\geq1$. When $n=0$, this only holds for $k\geq2$, and when
\end{proof}
\begin{prop}\label{Q0ZeroByPriddyAlg}
Suppose that $n>0$, and $z\in ZK^{\calU(n)}_{s_{n+1}}(X)_{s_n,\ldots,s_1}^t$ where not all of $s_n,\ldots,s_1$ equal zero. If $s_{n+1}=0$ then $\overline{z}\lambda_0$ is represented by $z\lambda_{s_n}\in X_{2s_n,\ldots,2s_1}^{2t+1}$. Suppose instead that $s_{n+1}>0$, and 
\[z=\textstyle\sum_{j}\Sqv(I^{(j)},x^{(j)}),\]
for some collection $I^{(j)}=(i^{(j)}_{p},\ldots,i^{(j)}_{1})$ of  $\Sq$-Admissible sequences, and corresponding homogeneous elements $x^{(j)}$ of $X$. Suppose further that, for all $j$, $x^{(j)}\lambda_{i-1}=0$ whenever $i\geq i^{{(j)}}_1$. Then $\overline{z}\lambda_0=0$.
\end{prop}
\begin{proof}
Write $p:=s_{n+1}$. The same homotopy $a$ as in the previous cases shows that $\widetilde{z}\lambda_0$ is represented by $\restnwithsubscript{(z)}{\langle 0\rangle}=z\lambda_{s_n\langle 0\rangle}$ when $p>0$, and by $z\lambda_{s_n}\in X$ when $p=0$, so that we may restrict to the case $p>0$. Then, $z\lambda_{s_n\langle 0\rangle}$ is the image of
\[E:=\sum_{j,\,\produces{K}{I^{(j)}}{\Sq}}x^{(j)}\left[\lambda_{k_1^{(j)}-1} \middle|\cdots\middle| \lambda_{k_{p-1}^{(j)}-1} \middle|\lambda_{k_{p}^{(j)}-1}\lambda_{s_n} \right]\in ZF_{p+1}N^\div_{p}Q^{\calU(n)}B^{\calU(n)}X\]
Dualizing Priddy's work, namely \cite[proof of Theorem 5.2]{PriddyKoszul.pdf}, gives a sequence of homotopies which move this cycle into $F_pN_p^\div$. Indeed, given an expression
\[e=y\left[\lambda_{g_1-1} \middle|\cdots \middle|\lambda_{g_{r-2}-1}\middle|
\lambda_{g_{r-1}-1}\lambda_{g_r-1}\middle|\lambda_{g_{r+1}-1}\middle|\cdots\middle|\lambda_{g_{p+1}-1}\right]\in F_{p+1}N_p^\div,\]
where $g_{r}-1\leq 2(g_{r-1}-1)$ (so that the expression $\lambda_{g_{r-1}-1}\lambda_{g_r-1}$ is $\Lambda$-admissible), define:
\[\Gamma(e):=\begin{cases}
y\left[\lambda_{g_1-1} \middle|\cdots \middle|
\lambda_{g_{r-1}-1}\middle|\lambda_{g_r-1}\middle|\cdots\middle|\lambda_{g_{p+1}-1}\right],&\textup{if }(g_{p+1},\ldots,g_{1})\textup{ is $\Sq$-admissible};\\
0,&\textup{otherwise}.
\end{cases}\]
If we further define $\Gamma$ to be zero on $F_pN_p^\div$, then $\Gamma:F_{p+1}N^\div_p\to F_{p+1}N^\div_{p+1}$ may be used as a chain homotopy to compress $E\in ZF_{p+1}N^\div_p$ into $ZF_pN^\div_p$:
\[(\Id+d\Gamma)^uE\textup{ stabilizes to an element of } ZF_pN_p^\div\textup{ as $u\to\infty$.}\]
As we repeatedly  apply $(\Id+d\Gamma)$ to this $e$, because $a_1\geq b_1$ whenever $\produces{(b_2,b_1)}{(a_2,a_1)}{\Lambda}$, the very leftmost $\lambda$-operation in any of the expressions that appear is $\lambda_{m-1}$ for some $m\geq g_1$, and every term in $(\Id+d\Gamma)^ue\in ZF_pN_p^\div$ will be of the form $y\lambda_{m-1}[{}\cdots{}]$ for some $m\geq g_1$.

Applying these observations in the very specific circumstances of this proposition, along with the observation that for each $j$ and $K$ indexing the sum $E$ we have $k_1^{(j)}\geq i_1^{(j)}$ (as $a_1\geq b_1$ whenever $\produces{(a_2,a_1)}{(b_2,b_1)}{\Sq}$ \textbf{would prefer to say `by x.y'}), one derives that $(\Id+d\Gamma)^uE=0$, so that $E$ is nullhomotopic. 
%
%
%Priddy's process returns a (large) sum of terms of the form
%\[x^{(j)}\lambda_{\kappa_0-1}\left[\lambda_{\kappa_1-1} \middle|\cdots\middle| \lambda_{\kappa_{p-1}-1} \middle|\lambda_{\kappa_p-1}\right]\textup{ where }\kappa_0\geq i_1^{(j)}.\]
%Under the very specific circumstances of this proposition, each of these terms vanishes.
\end{proof}
\end{Koszul complexes}




\begin{Composite functor spectral sequences}
\vfil\pagebreak
\section{\textbf{Composite functor spectral sequences}}
The subject of this paper is to identify the derived functors $H^*_{\calW(0)}X:=(\mathbb{L}_*Q^{\calW(0)}X)^*$, for $X\in\calW(0)$. More generally, we'll now present a spectral sequence whose goal is to calculate $H^*_{\calW(n)}X$ for $X\in\calW(n)$. This will be a composite functor spectral sequence analogous to Miller's spectral sequence in \ref{MillerSullivanConjecture.pdf}. The factorization of $Q^{\calW(n)}$ we will use is of course 
\[Q^{\calW(n)}=\left(\calW(n)\overset{Q^{\calU(n)}}{\to}\calL(n)\overset{Q^{\calL(n)}}{\to}\vect{+}{n}\right)\]
There is an added challenge in this context --- indeed, the available factorization of $Q^{\calW(n)}$ is through a non-abelian category. Thus, the standard technology for composite functor spectral sequences does not apply, and we must use Blanc and Stover's methods \cite{Blanc_Stover-Groth_SS.pdf}. They observe that the left derived functors $\mathbb{L}_*Q^{\calU(n)}X$ are calculated as the homotopy groups of a simplicial object in $\calL(n)$, namely $Q^{\calU(n)}B^{\calW(n)}X$. As such, they have the structure of a $\calL(n)\textup{-$\Pi$-algebra}$, i.e.\ they form an object of $\calW(n+1)$.  After verifying that the functor $Q^{\calU(n)}$ satisfies the requisite acyclicity condition (indeed it preserves free objects), one may apply \cite[Theorem 4.4]{Blanc_Stover-Groth_SS.pdf}: there is a spectral sequence, with $E_r\in\vect{+}{n+2}$,
\[(E^2)_{s_{n+2},\ldots,s_1}^t=((H_*^{\calW(n+1)})(\mathbb{L}_*Q^{\calU(n)})X)_{s_{n+2},\ldots,s_1}^t\implies ((H_*^{\calW(n)})X)_{s_{n+2}+s_{n+1},s_n,\ldots,s_1}^t\]
\begin{prop}
For $X\in s\calW(n)$, the groups $\mathbb{L}_*Q^{\calU(n)}X$ are isomorphic to $H_*^{\calU(n)}X$, in which we view $X$ as an element of $s\calU(n)$.
\end{prop}
\begin{proof}
We may take $X$ to be almost free in $s\calW(n)$, and calculate $\mathbb{L}_*Q^{\calU(n)}X$ simply as $\pi_*Q^{\calU(n)}X$. Then $X$, viewed as an object of $s\calU(n)$, is levelwise free, but potentially not almost free. We need to show then that $\pi_*Q^{\calU(n)}X$ does indeed calculate $H_*^{\calU(n)}X$ whenever $X\in s\calU(n)$ is \emph{levelwise} free, which is to say that the map $Q^{\calU(n)}B^{\calU(n)}X\to Q^{\calU(n)}X$ is a weak equivalence in $s\vect{}{}$. For this, $Q^{\calU(n)}B^{\calU(n)}X$ is the diagonal of a bisimplicial vectorspace, say $Q^{\calU(n)}B_q^{\calU(n)}X_p$, and we use the spectral sequence arising from filtering in the $p$ direction. As $X$ is levelwise free, the $E^1$-page is concentrated in $q=0$, and is isomorphic to the chain complex $N_p(Q^{\calU(n)}X)$.
\end{proof}
We'll prefer to work with the dual spectral sequence, (\cancel{which is} equivalent, as we have specified that elements of $\calW(n)$ are locally finite), which has $E_r\in\vect{n+2}{+}$, and we may write as:
\[(E_2)^{s_{n+2},\ldots,s_1}_t=((H^*_{\calW(n+1)})(H_*^{\calU(n)})X)^{s_{n+2},\ldots,s_1}_t\implies ((H^*_{\calW(n)})X)^{s_{n+2}+s_{n+1},s_n,\ldots,s_1}_t\]
These are the homology and cohomology spectral sequences for a bisimplicial abelian group (precisely, an object of $ss\vect{+}{n}$), and we will need to work with this object directly. Before we do, we'll give some general recollections and constructions relevant to the spectral sequence.

\subsection{The Blanc-Stover comonad in categories monadic over $\F_2$-vector spaces}
Fix an algebraic category $\calC$, monadic over a category of graded $\F_2$-vector spaces $\vect{}{}$. % --- although we intend to work over a category of graded vector spaces, but for now we opt for notational simplicity \textbf{do we really need to do that?}). 
As we are working over a category of vector spaces, rather than a category of graded sets, we can refine Blanc and Stover's definition  of a useful comonad on $s\calC$. While Blanc and Stover use the notation `$W$' in \cite{Blanc_Stover-Groth_SS.pdf} and $\scrV$ in \cite{StoverVanKampen.pdf}, we will use the symbol $\BSW$, to avoid notational confusion in the present work. \textbf{{work on this!}}

%\begin{shaded}\tiny
%It will help to fix a little notation. First, for $n\geq0$, let $\mathbb{S}^n\in \complexes \vect{}{}$ and $C\mathbb{S}^n\in \complexes \vect{}{}$ be the chain complexes
%\[(\mathbb{S}^n)_j=\begin{cases}
%\F_2\langle z\rangle,&\textup{if }j=n;\\
%0,&\textup{otherwise},
%\end{cases}\qquad 
%(C\mathbb{S}^n)_j=\begin{cases}
%\F_2\langle h\rangle,&\textup{if }j=n+1;\\
%\F_2\langle dh\rangle,&\textup{if }j=n;\\
%0,&\textup{otherwise},
%\end{cases}
%\]
%%with $d:(C\mathbb{S}^n)_{n+1}\to (C\mathbb{S}^n)_{n}$ the identity of $\F_2$.
%There is an evident inclusion $\imath:\mathbb{S}\to C\mathbb{S}$. For any $V\in\complexes\vect{}{}$, we have
%\[\hom_{\complexes\vect{}{}}(\mathbb{S}^n,V)\cong ZN_nV,\textup{ and }\hom_{\complexes\vect{}{}}(C\mathbb{S}^n,V)\cong N_{n+1}V,\]
%and the differential $d:N_{n+1}V\to ZN_nV$ corresponds to $\imath^*$ under these isomorphisms. If $N:s\vect{}{}\rightleftarrows \complexes \vect{}{}
%:\Gamma$ are the inverse equivalences of the Dold-Kan correspondence, define
%\[S^n_{\calC}:=F_{\calC}\Gamma(\mathbb{S}^n)\textup{ and }CS^n_{\calC}:=F_{\calC}\Gamma(C\mathbb{S}^n).\]
%Corresponding to the above structure, $S^n_{\calC}$ contains an element $z$ in dimension $n$, $CS^n_{\calC}$ contains an element $h$ in dimension $n+1$, there is a map $\imath:S^n_{\calC}\to CS^n_{\calC}$ sending $z$ to $d_0h$. For $L\in s\calC$, $\imath^*$ represents the differential under the isomorphisms
%\[\hom_{s\calC}(S_\calC^n,L)\cong ZN_nL,\textup{ and }\hom_{s\calC}(CS_\calC^n,L)\cong N_{n+1}L.\]
%\end{shaded}

In this context [\textbf{Change} all $S_\calC^n$ to the new notation], Blanc-Stover's comonad $\BSW$, applied to $L\in s\calC$, is the pushout
\[\xymatrix@R=4mm{
\bigsqcup_{n\geq0}\bigsqcup_{N_{n+1}L}S_\calC^n
\ar[r]\ar[d]^-{\bigsqcup\bigsqcup\imath}
&%r1c1
\bigsqcup_{n\geq0}\bigsqcup_{ZN_{n}L}S_\calC^n\ar[d]\\%r1c2
\bigsqcup_{n\geq0}\bigsqcup_{N_{n+1}L}CS_\calC^n
\ar[r]&%r2c1
\BSW L%r2c2
}\]
where under the top horizontal map, the copy of $S^n_\calC$ corresponding to $x\in N_{n+1}L$ maps identically onto the copy of $S^n_\calC$ corresponding to $dx=\imath^*x$. It will be useful to write $h_x$ for the element of $\BSW L$ corresponding to $h$ in the copy of $CS^n_\calC$ corresponding to $x\in N_{n+1}L$, and similarly, $z_x$ for the element of $\BSW L$ corresponding to $z$ in the copy of $S^n_\calC$ corresponding to $x\in ZN_{n}L$. The comonad structure maps $\epsilon:\BSW L\to L$ and $\Delta:\BSW L\to \BSW^2L$ are then determined by the equations
\[\epsilon(h_x)=x,\ \epsilon(z_y)=y,\ \Delta(h_x)=h_{h_x},\textup{ and }\Delta(z_y)=z_{z_y}\textup{ for $x\in N_{n+1}L$ and $y\in ZN_nL$.}\]

Moreover, $\BSW L$ is homotopy equivalent to a coproduct of spheres. Indeed, for each $n$, let $B_nL=\im (d:N_{n+1}L\to ZN_nL)$, and choose a section $h_0$ of the surjection $d:N_{n+1}L\to B_nL$. Then $\BSW L$ contains the contractible subobject $C_0:=\bigsqcup_n\bigsqcup_{h\in \im (h_0)} CS^n_{h_0(f)}$, and $\BSW L/C_0$ is homotopic to a wedge of spheres:
%\[\BSW L/C_0\cong \left(\bigsqcup_{h\in N_{n+1}L\setminus\im (h_0)}CS^n_h/\partial(CS^n_h)\right) \sqcup\left(\bigsqcup_{f\in ZN_nL\setminus B_nL}S^n_f\right)\]
\[\BSW L/C_0\cong \left(\bigsqcup_{N_{n+1}L\setminus\im (h_0)}F_\calC (CS^n_{\vect{}{}}/\partial(CS^n_{\vect{}{}}))\right) \sqcup\left(\bigsqcup_{f\in ZN_nL\setminus B_nL}S^n_\calC\right)\]
We would like to find a \emph{subspace} of $\pi_*(\BSW L)$ which freely generates it as a $\calC\textup{-$\Pi$-algebra}$. Even better, we have the following rendition of [Stover? Blanc-Stover?]. We give the proof since we will need to be explicit about some parts of it in what follows.
\begin{prop}
For $L\in s\calC$ (\textbf{s, right?}), $\pi_*(B^{\BSW}L)$ is an almost free (monadic over $\vect{}{}$) simplicial $\calC\textup{-$\Pi$-algebra}$ weakly equivalent to $\pi_*(L)$.
\end{prop}
\begin{proof}
That the augmentation to $\pi_*(L)$ is a weak equivalence follows from Stover's result \cite[2.7]{StoverVanKampen.pdf}. The only change from Blanc-Stover is that $\pi_*(B^{\BSW}L)$ is almost free over the category $\vect{}{}$, rather than the category of pointed sets.
For a set $S$, write $\F_2\langle S\rangle$ for the vector space generated by the symbols $\underline{s}$ for $s\in S$. There's a natural map $\F_2\langle d\rangle :\F_2\langle N_{n+1}L\rangle \to \F_2\langle ZN_nL\rangle $, and a natural monomorphism $\alpha:\ker(\F_2\langle d\rangle)\to \pi_*(\BSW L)$, defined by
\[\alpha(\underline{x_1}-\underline{x_0})=[h_{x_1}-h_{x_0}],\textup{ for $x_1,x_2\in N_{n+1}L$ with $dx_1=dx_2$}.\]
Moreover, there's a natural map $\beta:\F_2[ZN_nL]\to\pi_*(\BSW L)$ (which is not monomorphic) defined by
\[\beta(\underline{x})=[z_{x}],\textup{ for $x\in ZN_{n}L$}.\]
From the above expression for $\BSW L/C_0$, one sees that $\im (\alpha)$ and $\im (\beta)$ are linearly independent subspaces of $\pi_*(\BSW L)$, and $\pi_*(\BSW L)$ is free on $\im (\alpha)\oplus\im (\beta)$. From this description it is clear that the generating subspaces are preserved by maps in the image of $\BSW $. Thus, every face and degeneracy map except for $s_0$ and $d_0$ preserves the generators.

In order that $s_0$ preserves generators, we note that the diagonal of $\BSW$ sends the subspaces $\im (\alpha_L)$ and $\im (\beta_L)$ into the subspaces $\im (\alpha_{\BSW L})$ and $\im (\beta_{\BSW L})$. That $\im (\beta_L)$ maps into $\im (\beta_{\BSW L})$ is immediate. For $\im (\alpha_L)$, the image of $h_{x_1}-h_{x_0}$ under the diagonal is $h_{h_{x_1}}-h_{h_{x_0}}$, which is in $\im(\alpha_{\BSW L})$, since $dh_{x_1}=z_{dx_1}=z_{dx_0}=dh_{x_0}$.
\end{proof}



%There's a natural map $\F_2[d]:\F_2[N_{n+1}X]\to \F_2[ZN_nX]$, and a natural monomorphism $\alpha:\ker(\F_2[d])\to\pi_*(\BSW X)$. Indeed, writing $\underline{z}$ for the generator of $\F_2[N_{n+1}X]$ corresponding to $z\in N_{{n+1}X}$, the group $\ker(\F_2[d])$ is generated by differences $\underline{z}-\underline{z}'$ for $z,z'\in N_{n+1}X$ having $dz=dz'$, and we define $\alpha(\underline{z}-\underline{z}')$ to be the homotopy class of the difference of the cones corresponding to $z$ and $z'$, whose boundaries have been identified in the colimit. Moreover, there's a natural map $\beta:\F_2[ZN_nX]\to\pi_*(\BSW X)$ (which is not monomorphic), which sends a generator $z\in ZN_nX$ to the homotopy class in $\BSW X$ of the corresponding summand in the top right corner of the colimit diagram. From the above expression for $\BSW X/C_0$, one sees that $\im (\alpha)$ and $\im (\beta)$ are linearly independent subspaces of $\pi_*(\BSW X)$, and $\pi_*(\BSW X)$ is free on $\im (\alpha)\oplus\im (\beta)$. From this description it is clear that the generating subspaces are preserved by maps in the image of $\BSW $.
%
%Moreover, the diagonal of the comonad also preserves the subspaces $\im (\alpha)$ and $\im (\beta)$. That $\im (\alpha)$ is preserved is evident from the definitions. For $\im (\beta)$, note that $\im (\beta)\subset\pi_*(\BSW X)$ is spanned by terms $D_h-D_{h'}$ for $h,h':CS\to X$ satisfying $h\imath=h'\imath$. The diagonal applied to $D_h-D_{h'}$ may be written $D_{D_h}-D_{D_{h'}}$, and one notes that $D_h\imath=S_{h\imath}=S_{h'\imath}=D_{h'}\imath$.
%
%\textbf{Move this para: }Blanc and Stover explain that $\BSW $ actually has the structure of a comonad on $s\calL(n)$. As such, for any simplicial Lie algebra $L\in s\calL(n)$, there is a bisimplicial object $B^\BSW L$, the bar construction using the comonad $\BSW $, given by $B_{s_2}^\BSW L_{s_1}=(\BSW^{s_2+1}L)_{s_1}$. Flesh out the point here...
%
%
%Now the utility of the comonad structure lies in resolving an object $L\in s\calL(n)$ by the comonadic bar construction, $B^\BSW(L)$. This discussion shows that, except for $d_0$, all the maps in $B^\BSW L$ preserve the subspaces of generators of $\pi_*(B^\BSW L)$, which is to say that this object of $s\calW(n+1)$ is almost free. Moreover, according to [Stover 2.6], the augmentation map $\pi_*(B^\BSW L)\to \pi_*(L)$ is a weak equivalence in $s\calW(n+1)$. Thus, this Blanc-Stover $\BSW $-construction provides an almost-free replacement of $\pi_*(L)$ in $s\calW(n+1)$.





\subsection{A chain-level diagonal on the $\BSW $ construction}
\label{Subsection: Chain level diagonal}
We have seen, for $L\in s\calC$, that $\pi_*(\BSW L)$ is naturally a free object in $\calC\PiAlg$. As such, there is a diagonal
\[\phi_{\calC\PiAlg}:\pi_*(\BSW L)\to \pi_*(\BSW L)\sqcup \pi_*(\BSW L).\]
In this section, we'll describe how $\phi_{\calC\PiAlg}$ is the map on homotopy induced by a morphism $\phi_\BSW :\BSW L\to \BSW L\sqcup \BSW L$ in $s\calC$ (so that $i\circ\phi_{\calC\PiAlg}=\pi_*(\phi_{\BSW})$, where $i$ is the natural isomorphism $\pi_*(\BSW L)\sqcup\pi_*(\BSW L)\to \pi_*(\BSW L\sqcup \BSW L)$).

In order to construct a map $\phi_\BSW $, we construct a commuting diagram
\[\xymatrix@R=4mm{
S^n_{\calC}\ar[r]^-{\phi_1}
\ar[d]^-{\imath}
&%r1c1
S^n_{\calC}\sqcup S^n_{\calC}\ar[d]^-{\imath\sqcup\imath}
\\%r1c2
CS^n_{\calC}\ar[r]^-{\phi_2}
&%r2c1
CS^n_{\calC}\sqcup CS^n_{\calC}%r2c2
}\qquad \textup{\raisebox{-6mm}{by applying $F_\calC$ to }}\qquad \xymatrix@R=4mm{
S^n_{\vect{}{}}\ar[r]^-{\Delta}
\ar[d]^-{\imath}
&%r1c1
S^n_{\vect{}{}}\sqcup S^n_{\vect{}{}}\ar[d]^-{\imath\sqcup\imath}
\\%r1c2
CS^n_{\vect{}{}}\ar[r]^-{\Delta}
&%r2c1
CS^n_{\vect{}{}}\sqcup CS^n_{\vect{}{}}%r2c2
}\]
The maps $\phi_1$ and $\phi_2$ can then be applied respectively to all of the sphere and cone classes appearing in $\BSW L$.
%In fact, this diagram can even be formed in $s\vect{+}{n}$, before application of $F_{\calC}:\vect{+}{n}\to\calC$, and $\phi_1$ and $\phi_2$ are just the levelwise application of the diagonal map on a simplicial vector space. 
To understand the effect of $\phi_{\BSW }$ on homotopy, it is enough to identify where the generators of $\pi(\BSW L)$ are sent in $\pi(\BSW L)\sqcup\pi(\BSW L)$, which is easy.
\begin{lem}
$\BSW L$ is naturally a (strict) commutative cogroup object, having comultiplication map $\phi_{\BSW }$, counit map $0:\BSW L\to 0$, and inverse map $\Id:\BSW L\to \BSW L$. In particular, $\hom(\BSW L,\DASH)$ takes values in $\ensuremath{\F_2}$-vector spaces.
\end{lem}
\begin{proof}
There are a few axioms to check, for example, left counitality: that the composite 
\[\xymatrix@R=4mm@1{
\BSW L\ar[r]^-{\phi_{\BSW }}
&%r1c1
\BSW L\sqcup \BSW L\ar[r]^-{\Id\sqcup0}
&%r1c2
\BSW L%r1c3
}\] is the identity. This follows since $(\Id\sqcup0)\phi_1$ is the identity of $S^n$ and $(\Id\sqcup0)\phi_2$ is the identity of $CS^n$. The other axioms follow similarly.
\end{proof}

Writing$\star$ for the group operation on $\hom_{s\calC}(\BSW L,L')$, we have the following:
\begin{lem}
For maps $f,g:\BSW L\to L'$ we have 
\[Q^{\calC}(f\star g)=(Q^{\calC}f+Q^{\calC}g):Q^{\calC}\BSW L\to Q^{\calC}L'.\]
\end{lem}
\begin{proof}
It's enough to check that $Q^{\calC}(\phi_\BSW ):Q^{\calC}\BSW L\to Q^{\calC}(\BSW L\sqcup \BSW L)$ equals the diagonal map $Q^{\calC}\BSW L\to Q^{\calC}\BSW L\oplus Q^{\calC}\BSW L$. For this, $Q^{\calC}$ converts all the colimits involved in the construction of $\BSW L$ to colimits in $s\vect{+}{n}$, and $Q^{\calC}\phi_1$ and $Q^{\calC}\phi_2$ are both precisely the diagonal map.
\end{proof}

Now let $\overline{\xi}_{\BSW }$ denote the following composite:
%\[W^2X\overset{\phi_{\BSW }}{\to}(W^2X)^{\sqcup2}\overset{\phi_{\BSW }\sqcup \epsilon}{\to}(W^2X)^{\sqcup2}\sqcup(WX) \overset{\epsilon^{\sqcup2}\sqcup\phi_{ch}}{\to}((WX)^{\sqcup2})^{\sqcup2}\overset{\textup{fold}}{\to}(WX)^{\sqcup2}\]
\[\makebox[0mm][r]{$\overline{\xi}_{\BSW }:\ $}\BSW^2L\overset{\phi_{\BSW }}{\to}(\BSW^2L)^{\sqcup2}\overset{a\sqcup b}{\to}(\BSW L)^{\sqcup2}\]
where $a,b:\BSW^2L\to(\BSW L)^{\sqcup2}$ are the composites
\[\xymatrix@R=2mm{
\makebox[0mm][r]{$a:\ $}\BSW^2L\ar[r]^-{\phi_{\BSW }}
&%r1c1
(\BSW^2L)^{\sqcup2}\ar[r]^-{\epsilon^{\sqcup2}}
&%r1c2
(\BSW L)^{\sqcup2}\\
\makebox[0mm][r]{$b:\ $}\BSW^2L\ar[r]^-{\epsilon}
&%r1c1
(\BSW L)\ar[r]^-{\phi_{\BSW }}
&%r1c2
(\BSW L)^{\sqcup2}
}\]
\begin{lem}
The composite 
$\BSW^2L\overset{\overline{\xi}_{\BSW }}{\to}(\BSW L)^{\sqcup2}\to(\BSW L)^{\times2}$ is zero.
\end{lem}
\begin{proof}
This follows from the observation that both composites $(\Id\sqcup0)\overline{\xi}_{\BSW }$ and $(0\sqcup\Id)\overline{\xi}_{\BSW }$ equal $\epsilon:\BSW^2L\to \BSW L$.
\end{proof}
In particular, $\overline{\xi}_{\BSW }$ factors through the smash coproduct, defining a natural map
\[\xi_{\BSW }:\BSW^2L\to (\BSW L)^{\wedge 2}.\]

\begin{lem}\label{LemmaOn xi}
We have $(i\circ \xi_{\calC\PiAlg})=\pi_*(\xi_{\BSW}):\pi_*(\BSW^{p+k+1}L)\to\pi_*((\BSW^{p+k}L)^{\wedge 2})$, where $i$ is the natural isomorphism $(\pi_*(\BSW^{p+k}))^{\wedge 2}\to \pi_*((\BSW^{p+k})^{\wedge 2})$ of \S[\textbf{whatever}].
\end{lem}
\begin{proof}
Given the observation [?from above on SESs], this is equivalent to $(i\circ \overline{\xi}_{\calC\PiAlg})=\pi_*(\overline{\xi}_{\BSW}):\pi_*(\BSW^{p+k+1}L)\to\pi_*((\BSW^{p+k}L)^{\sqcup 2})$, which holds as $(i\circ\phi_{\calC\PiAlg})=\pi_*(\phi_{\BSW})$.
\end{proof}
\begin{lem}
$(d_i)^{\wedge 2}\xi_\BSW =\xi_\BSW d_{i+1}$ for $i\geq1$, and $(d_0)^{\wedge 2}\xi_\BSW = (\xi_\BSW d_{0})\star(\xi_\BSW d_{1})$, so that the map $Q^{\calC}\xi_\BSW $ induces a degree $(-1,0)$ bicomplex map:
\[N_*N_*(Q^{\calC}B^\BSW_{\bullet}L)_{s_2,s_1}\to
  N_*N_*(Q^{\calC}((B^\BSW_{\bullet}L)^{\wedge 2}))_{s_2-1,s_1}.\]
\end{lem}
\begin{proof}
I used to say `it's just like Goerss'... \textbf{give proof.}
\end{proof}
\textbf{Maybe define $\psi_\BSW$} here, as opposed to defining it in a couple of pages, and give a description of it.

\subsection{Definition of the spectral sequence}
\todo{Give the construction of the spectral sequence}{Do it in general, then in the specific cases
\item Conclude with `we'll now look at some constructions required to give operations on the sseq.'}
Fix $X\in\calW(n)$, and define $L=Q^{\calU(n)}B^{\calW(n)}X$. Let $\BSWres L$ be the bisimplicial object obtained from the simplicial bar construction using the comonad $\BSW$. Then the spectral sequence we are interested in is the spectral sequence of the bisimplicial object
\[Q^{\calL(n)}\BSWres L.\]
Now $L$ is an almost free object of $\calL(n)$, with $\pi_*L=H_*^{\calU(n)}X$. Thus $\pi_*(\BSWres L)$ is an almost free resolution of $H_*^{\calU(n)}X$ in $s\calW(n+1)$. Moreover, $\pi_*(Q^{\calL(n)}\BSWres L)=Q^{\calW(n+1)}\pi_*(\BSWres L)$, so that
\[E^2_{**}=H_*^{\calW(n+1)}H_*^{\calU(n)}X.\]
\textbf{Some} comments here on the convergence target.

%\subsection{Singer's theory of spectral sequence operations}
\subsection{Spectral sequence operations}
[\textbf{This whole subsection needs} to have its notation brought into line with the rest of the document. We \textbf{also need} to mention $S^k$ somewhere, and discuss the external operations at $E_2$!] In what follows, we'll need to use Singer's techniques for deriving operations in spectral sequences. He defines operations in the cohomology spectral sequence of a bisimplicial cocommutative coalgebra. While the comultiplicative structure that we can define on the resolution $Q^{\calL(n)}\BSWres L$ is not that of a coalgebra, we can still use his techniques after a single simple observation. That is, for every bisimplicial vector space $V$, if we write $E_r^{**}(V)$ for the cohomology spectral sequence obtained from the double cochain complex $(N_*N_*V)^*$, then Singer's methods yield external operations of type:
\[E_r^{**}(V)\to E_{r'}^{**}(S^2V)\textup{ and }E_r^{**}(V)\otimes E_r^{**}(V)\to E_r^{**}(S^2V),\]
which will be compatible with external operations of type:
\[H^{*}((CV)^*)\to H^{*}((CS^2V)^*)\textup{ and }H^{*}((CV)^*)\otimes H^{*}((CV)^*)\to H^{*}((CS^2V)^*).\]

When $V$ is a cocommutative coalgebra, the diagonal map $V\to S^2V$ induces a map $E_r^{**}(S^2V)\to E_r^{**}(V)$, postcomposition with which yields Singer's operations, however we will not have access to such structure in what follows. Thus, we will briefly reprise Singer's results in this \emph{external} language, in order that we have them available.
\begin{thm}[Singer's Book, thm 2.15-2.17,2.22]
For all $p,q\geq0$ and all $r\geq2$, there are well-defined vector space homomorphisms:
\begin{alignat*}{2}
\ExtCohOp^k:E^{p,q}_r(V)&\to E^{p,q+k}_r(S^2V)&\qquad&\text{if }0\leq k \leq q,\\
\ExtCohOp^k:E^{p,q}_r(V)&\to E^{p+k-q,2q}_{r+k-q}(S^2V)&\qquad&\text{if }q\leq k\leq q+r-2,\\
\ExtCohOp^k:E^{p,q}_r(V)&\to E^{p+k-q,2q}_{2r-2}(S^2V)&\qquad&\text{if }q+r-2\leq k.
\end{alignat*}
The operations $\ExtCohOp^k$ are all defined at $E_2$, and determine the operations at each $E_r$, $r>2$. The operations $\ExtCohOp^k$ commute with differentials. Finally, the $\ExtCohOp^k$ converge to well defined maps on $E_\infty$, and there is a commuting diagram
\[\xymatrix@R=4mm{
E^{p,q}_\infty(V)\ar[r]^-{\ExtCohOp^k}\ar[d]^-{\pi}
&%r1c1
E^{p,q+k}_\infty(S^2V)\ar[d]^-{\pi}
\\%r1c2
E_0^{p,q}H^*((CV)^*)\ar[r]^-{\ExtCohOp^k}&%r2c1
E_0^{p,q+k}H^*((CS^2V)^*)%r2c2
}\]
whenever $0\leq k\leq q$, and a commuting diagram
\[\xymatrix@R=4mm{
E^{p,q}_\infty(V)\ar[r]^-{\ExtCohOp^k}\ar[d]^-{\pi}
&%r1c1
E^{p+k-q,2q}_\infty(S^2V)\ar[d]^-{\pi}
\\%r1c2
E_0^{p,q}H^*((CV)^*)\ar[r]^-{\ExtCohOp^k}&%r2c1
E_0^{p+k-q,2q}H^*((CS^2V)^*)%r2c2
}\]
whenever $q\leq k$ (which summarises also Singer's computation of how the $\ExtCohOp^k$ interact with the filtration on cohomology).

Similar theory holds for an exterior product operation
\[\ExtCohProd:E_r^{p_1,q_1}(V)\otimes E_r^{p_2,q_2}(V)\to E_r^{p_1+p_2,q_1+p_2}(S^2V)\]
%Moreover, for $r\geq2$ and $p,q,k\geq0$ given, if $\alpha\in E_r^{p,q}$, then $\Sq_\textup{ext}^k\alpha$ and $\Sq_\textup{ext}^kd_r\alpha$ both survive to $E_t$, where
%\[t=\begin{cases}
%r,&\textup{if }0\leq k \leq q-r+1;\\
%2r+k-q-1,&\textup{if }q-r+1\leq k\leq q;
%\\2r-1,&\textup{if }q\leq k,
%\end{cases}
%\]
%and in $E_t$ we have the relation $d_t\Sq^k_textup{ext}\alpha=\Sq^k_textup{ext}d_r\alpha$.
\end{thm}
Moreover, if the bisimplicial object $V$ admits an augmentation $\lambda:V_{0\bullet}\to U_{\bullet}$, (\textbf{didn't I} do away with $\bullet$?) then %such that the natural map $\lambda_*:CV\to CU$ is a weak equivalence,
Singer proves the following relationship between the spectral sequence operations and the operations of \S\ref{generic coh ops section}:
\begin{thm}[Proposition 2.1]
The $\ExtCohOp^k$ and $\ExtCohProd$ commute with $\lambda^*$, so that there are commuting diagrams
\[\xymatrix@R=4mm{
H^{m}((CU)^*)\ar[r]^-{\ExtCohOp^k}
\ar[d]^-{\lambda^*}
&%r1c1
H^{m+k}((CS^2U)^*)\ar[d]^-{(S^2\lambda)^*}
&%r1c3
H^{m_1}((CU)^*)\otimes H^{m_2}((CU)^*)\ar[r]^-{\ExtCohProd}
\ar[d]^-{\lambda^*}
&%r1c1
H^{m_1+m_2}((CS^2U)^*)\ar[d]^-{(S^2\lambda)^*}
\\%r1c5
H^{m}((CV)^*)\ar[r]^-{\ExtCohOp^k}
&%r2c1
H^{m+k}((CS^2V)^*)
&%r2c3
H^{m_1}((CV)^*)\otimes H^{m_2}((CV)^*)\ar[r]^-{\ExtCohProd}
&%r2c1
H^{m_1+m_2}((CS^2V)^*)
}\]
\end{thm}

Singer separates the operations $\ExtCohOp^k:E^{p,q}_2\to E_{2}$ according to the pattern of change in the spectral sequence coordinates of their output. Indeed, define
\begin{alignat*}{2}
\vExtCohOp^k=\ExtCohOp^k:E^{p,q}_2(V)&\to E^{p,q+k}_2(S^2V)&\qquad&\text{if }0\leq k \leq q,\\
\hExtCohOp^k=\ExtCohOp^{q+k}:E^{p,q}_2(V)&\to E^{p+k,2q}_2(S^2V)&\qquad&\text{if }0\leq k \leq p,
\end{alignat*}
and gives explicit formulae for these $E_2$ operations, and for the product at $E_2$ [Singer 2.23].


In order to use these constructions in the present work, we will use the map of double complexes:
\[\psi_{\BSW}=j_{\calL(n)}\circ Q^{\calL(n)}\xi_\BSW:N_{p+1}N_q(Q^{\calL(n)}\BSWres L)_{s_n,\ldots,s_1}^{t+1}\to N_{p}N_q(S^2(Q^{\calL(n)}\BSWres L))_{s_n,\ldots,s_1}^{t}\]
to define a spectral sequence map
\[E_2^{p,q}(S^2(Q^{\calL(n)}\BSWres L))^{s_n,\ldots,s_1}_t\overset{(\psi_\BSW)^*}{\to} E_2^{p+1,q}(Q^{\calL(n)}\BSWres L)^{s_n,\ldots,s_1}_{t+1}\]
We may then define internal spectral sequence operations
\begin{alignat*}{2}
\Sq^k=\psi_{\BSW}^*\circ\ExtCohOp^{k-1}:(E^{p,q}_r)^{s_n,\ldots,s_1}_t&\to (E^{p+1,q+k-1}_r)^{2s_n,\ldots,2s_1}_{2t+1}&\qquad&\text{if }0\leq k-1 \leq q,\\
\Sq^k=\psi_{\BSW}^*\circ\ExtCohOp^{k-1}:(E^{p,q}_r)^{s_n,\ldots,s_1}_t&\to (E^{p+k-q,2q}_{r+k-q-1})^{2s_n,\ldots,2s_1}_{2t+1}&\qquad&\text{if }q\leq k-1\leq q+r-2,\\
\Sq^k=\psi_{\BSW}^*\circ\ExtCohOp^{k-1}:(E^{p,q}_r)^{s_n,\ldots,s_1}_t&\to (E^{p+k-q,2q}_{2r-2})^{2s_n,\ldots,2s_1}_{2t+1}&\qquad&\text{if }q+r-2\leq k-1.
\end{alignat*}
Which, at $E_2$, we may write (dropping internal degrees) as \textbf{(precisely?:)}
\begin{alignat*}{2}
\Sqv^k=\psi_{\BSW}^*\circ\vExtCohOp^{k-1}= \psi_{\BSW}^*\circ\ExtCohOp^{k-1}:E^{p,q}_2&\to E^{p+1,q+k-1}_2&\qquad&\text{if }0\leq k-1 \leq q,\\
\Sqh^k=\psi_{\BSW}^*\circ\hExtCohOp^{k-1}= \psi_{\BSW}^*\circ\ExtCohOp^{q+k-1}:E^{p,q}_2&\to E^{p+k,2q}_2&\qquad&\text{if }0\leq k-1 \leq p,
\end{alignat*}

And similarly, a pairing
\[\mu=\psi_{\BSW}^*\circ\ExtCohProd:(E^{p,q}_r)^{s_n,\ldots,s_1}_t\otimes (E^{p',q'}_r)^{s'_n,\ldots,s'_1}_{t'}\to (E^{p+p'+1,q+q'}_r)^{s_n+s'_n,\ldots,s_1+s'_1}_{t+t'+1}\]
The reader might now guess the next results:
\begin{thm}\label{E2CompFuncLieOperationsID}
At $E_2$, the operations $\Sqh^k$ and $\mu$ are equal to the $H^*_{\calW(n+1)}$-cohomology operations on $E_2=H^*_{\calW(n+1)}H_*^{\calU(n)}X$ of the same name, as defined in [\S?].
\end{thm}
\begin{thm}\label{E2CompFuncKosOperationsID}
At $E_2$, the operations $\Sqv^k$ are equal to the $H^*_{\calW(n+1)}$-cohomology operations on $E_2=H^*_{\calW(n+1)}H_*^{\calU(n)}X$ of the same name, as defined in [\S?].
\end{thm}
\begin{thm}\label{EInftyCompFuncOperationsID}
At $E_\infty$, the operations $\Sq^k$ are compatible with the operations $\Sqh^k$ defined on the target $H^*_{\calW(n)}(X)$, as defined in [\S?].
\end{thm} 
\begin{tricky proofs of operation compatibilities}
\begin{proof}[Proof of \ref{E2CompFuncLieOperationsID}]
We will prove this theorem in various parts. Firstly, we'll work on the operations $\Sqh^k$ and $\mu$ at $E_2$.
For this we will rely on a commuting diagram, where we write $L=Q^{\calU(n)}X^c\in s\calL(n)$ and $Y=\BSWres L\in ss\calL(n)$:
\[\xymatrix@C=13mm@R=4mm{
(N_p^\textup{h}\pi_*^{\textup{v}}Q^{\calL(n)}Y)^{\otimes2}&%r1c1
(N_p^\textup{h}Q^{\calW(n+1)}\pi_*^{\textup{v}}Y)^{\otimes2}
\ar[l]_-{(N^\textup{h}_*(\gamma))^{\otimes 2}}
\\%r1c2
N_{p+k-1}^\textup{h}((\pi_*^{\textup{v}}Q^{\calL(n)}Y)^{\otimes2})
\ar[u]^-{D_{p-k+1}^\textup{h}}
&%r2c1
N_{p+k-1}^\textup{h}((Q^{\calW(n+1)}\pi_*^{\textup{v}}Y)^{\otimes2})
\ar[u]^-{D_{p-k+1}}
\ar[l]_-{N^\textup{h}_*(\gamma^{\otimes 2})}
\\%r2c2
N_{p+k-1}^\textup{h}(\pi_*^{\textup{v}}((Q^{\calL(n)}Y)^{\otimes2}))
\ar[u]^-{N^\textup{h}_*\pi_*^\textup{v}(D_0^\textup{v})}
&%r3c1
\\%r3c2
N^\textup{h}_{p+k}(\pi_*^{\textup{v}} Q^{\calL(n)}Y)
\ar[u]^-{\pi_*^\textup{v}(\psi_{\BSW})}
&%r4c1
N^\textup{h}_{p+k}(Q^{\calW(n+1)}\pi_*^{\textup{v}} Y)
\ar[uu]_-{\psi_{\calW(n+1)}}
\ar[l]_-{N^\textup{h}_*(\gamma)}
%r4c2
}\]
In this diagram, all of the horizontal maps are isomorphisms, and the two vertical composites are chain maps. (by Singer's Theorem 2.23),  the left hand vertical composite is that used to define the horizontal operations $\Sq^{k}_h$ on $E_2$ (\textbf{(right?)}). On the other hand, the right vertical is used to define the operations on the $\calW(n+1)$-cohomology groups with which the $E_2$-page can be identified. The same proof will apply for $\mu$, if we replace the maps $D_j$ in the top square with $D_0^\textup{h}$, and modify the subscripts $p$ as needed.

What remains is to prove that the bottom square commutes. It may be expanded into the \emph{outer square} in the following larger commuting diagram, which we will abbreviate by omitting the subscripts on indecomposables functors:
%\[
%\def\objectstyle{\scriptstyle}
%\xymatrix@R=8mm@C=13mm@!0{
%%&%r1c1
%%(\pi (QW^{n+1}L))^{\otimes2}\ar[rrrrrr];[]_-{=}
%%&%r1c2
%%&%r1c3
%%&%r1c4
%%&%r1c5
%%&%r1c6
%%&%r1c7
%%(\pi (QW^{n+1}L))^{\otimes2}\\\\%r1c8
%&%r2c1
%\pi((QW^{p+k}L)^{\otimes2}) \ar[rrr]_-{D_0^{\textup{v}}}
%&%r2c2
%&%r2c3
%&%r2c4
%(\pi (QW^{p+k}L))^{\otimes2}&%r2c5
%&%r2c6
%&%r2c7
%(Q\pi(W^{p+k}L))^{\otimes2} \ar[lll]^-{\gamma^{\otimes2}}\\\\%r2c8
%&%r3c1
%\pi Q((W^{p+k}L)^{\wedge 2})\ar[uu]^-{\pi(j)}
%\ar[dl]
%&%r3c2
%&%r3c3
%&%r3c4
%Q\pi((W^{p+k}L)^{\wedge 2})\ar[lll]_-{\gamma}\ar[dl]
%&%r3c5
%&%r3c6
%&%r3c7
%Q((\pi( W^{p+k}L))^{\wedge 2})\ar[dl]
%\ar[uu]_-{j}
%\ar[lll]_-{Q(i)}
%\\
%%r3c1
%\pi Q((W^{p+k}L)^{\sqcup 2})
%&%r3c2
%&%r3c3
%&%r3c4
%Q\pi((W^{p+k}L)^{\sqcup 2})\ar[lll]_-(.3){\gamma}
%&%r3c5
%&%r3c6
%&%r3c7
%Q((\pi( W^{p+k}L))^{\sqcup 2})
%\ar[lll]_-(.3){Q(i)}
%%r3c8
%\\\\
%&%r3c1
%\pi Q(W^{p+k+1}L)
%\ar@/_.5em/[uuu]_-(.25){\pi Q(\xi_{\textup{ch}})}
%\ar[uul]^-(.65){\pi Q(\overline{\xi}_{\textup{ch}})}
%&%r3c2
%&%r3c3
%&%r3c4
%Q\pi(W^{p+k+1}L)\ar[lll]_-{\gamma}
%\ar@/_.5em/[uuu]_-(.25){Q\pi(\xi_{\textup{ch}})}
%\ar[uul]^-(.65){Q\pi(\overline{\xi}_{\textup{ch}})}
%&%r3c5
%&%r3c6
%&%r3c7
%Q(\pi( W^{p+k+1}L))
%\ar@/_.5em/[uuu]_-(.25){Q(\xi_{\calW(n+1)})}
%\ar[lll]_-{=}
%\ar[uul]^-(.65){Q(\overline{\xi}_{\calW(n+1)})}
%}\]
\[
\def\objectstyle{\scriptstyle}
\xymatrix@R=6mm@C=13mm@!0{
%&%r1c1
%(\pi (QW^{n+1}L))^{\otimes2}\ar[rrrrrr];[]_-{=}
%&%r1c2
%&%r1c3
%&%r1c4
%&%r1c5
%&%r1c6
%&%r1c7
%(\pi (QW^{n+1}L))^{\otimes2}\\\\%r1c8
&%r2c1
\pi((Q\BSW^{p+k}L)^{\otimes2}) \ar[rrr]_-{D_0^{\textup{v}}}
&%r2c2
&%r2c3
&%r2c4
(\pi (Q\BSW^{p+k}L))^{\otimes2}&%r2c5
&%r2c6
&%r2c7
(Q\pi(\BSW^{p+k}L))^{\otimes2} \ar[lll]^-{\gamma^{\otimes2}}\\\\%r2c8
&%r3c1
\pi Q((\BSW^{p+k}L)^{\wedge 2})\ar[uu]^-{\pi(j)}
&%r3c2
&%r3c3
&%r3c4
Q\pi((\BSW^{p+k}L)^{\wedge 2})\ar[lll]_-{\gamma}
&%r3c5
&%r3c6
&%r3c7
Q((\pi( \BSW^{p+k}L))^{\wedge 2})
\ar[uu]_-{j}
\ar[lll]_-{Q(i)}
\\\\
&%r3c1
\pi Q(\BSW^{p+k+1}L)
\ar[uu]_-{\pi Q(\xi_{\BSW})}
&%r3c2
&%r3c3
&%r3c4
Q\pi(\BSW^{p+k+1}L)\ar[lll]_-{\gamma}
\ar[uu]_-{Q\pi(\xi_{\BSW})}
&%r3c5
&%r3c6
&%r3c7
Q(\pi( \BSW^{p+k+1}L))
\ar[uu]_-{Q(\xi_{\calW(n+1)})}
\ar[lll]_-{=}
}\]
The bottom left square commutes simply by naturality of $\gamma$, while the bottom right square is an instance of lemma \ref{LemmaOn xi}. What remains is to check that the hexagon commutes. For notational convenience, write $A=\BSW^{p+k}L$, $\textup{br}:(\pi_* A)^{\otimes 2}\to (\pi_* A)^{\wedge 2}$ for the bracket $a\otimes b\mapsto [a,b]$ on homotopy, and similarly write $\textup{br}:A^{\otimes 2}\to A^{\wedge 2}$ for the chain level bracket. Now any element of $Q((\pi_* A)^{\wedge 2})$ is the class of an expression $\textup{br}(\overline{x}\otimes\overline{y})+E$ where $\overline{x},\overline{y}\in\pi_* A$ are represented by $x,y\in N_*A$, and $E\in(\pi_* A)^{\wedge 2}$ is a sum of terms which are $\lambda$-operations applied to 3-fold brackets of elements of $\pi_* A$ \textbf{[explain this better]}. The map $Q(i)$ is induced by the Eilenberg-Mac Lane map shuffle map $\nabla$ [](\textbf{what exactly?}), and
\[\textup{br}(\overline{x}\otimes\overline{y})\overset{\gamma\circ Q(i)}{\mapsto}\overline{\textup{br}(\nabla(x\otimes y))}\overset{\pi_*(j)}{\mapsto}\overline{(\nabla(x\otimes y))}\overset{D_0}{\mapsto}\overline{x}\otimes \overline{y}.\]
The last mapping follows from the fact that $D_0\circ\nabla=\Id$, as $\{D_k\}$ is special.
Similarly, our understanding of the construction of $\lambda$-operations and iterated brackets shows that $E$ is annihilated by $\pi_*(j)\circ\gamma\circ Q(i)$. As $E$ is also annihilated by $j$, and $j(\textup{br}(\overline{x}\otimes\overline{y})) =\overline{x}\otimes\overline{y}$, the hexagon commutes. \textbf{This is poorly written.}
\end{proof}
\begin{proof}[Proof of \ref{E2CompFuncKosOperationsID}]
We must identify the operations $\Sqv^i=\psi_{\BSW}^*\circ\vExtCohOp^{i-1}$ with the $\calW(n+1)$-cohomology operations $\Sqv^i$ defined in \S\ref{section: vertical Koszul operations n positive} using the map $\gamma_i$. However, the $\gamma_i$ are defined on the bar construction, while $\psi_{\BSW}^*$ is defined on the Blanc-Stover resolution. In order to make the comparison, we will need to choose a weak equivalence of resolutions of $\pi_* L$ in $s\calW(n+1)$
\[\chi:B^{\calW(n+1)}_{\bullet}\pi_*L\to \pi_*(\BSWres L).\]
We'll use Miller's techniques [Miller p. 55] to define such a $\chi$ explicitly. For brevity, write $\mathbb{B}$ for the object $B_i^{\calW(n+1)}\pi_*L\in s\calW(n+1)$. Write $V_i$ for the subspace $(F_{\calW(n+1)})^{i}\subset \mathbb{B}$ of generators. For each $m\geq0$, write $F_mV_i$ for the subspace of $V_i$ consisting of degeneracies of elements of $V_j$ for $j\leq m$. Then write $F_m\mathbb{B}$ for the subobject of $\mathbb{B}$ which is almost free on the subspaces $F_mV_i$.

Now for each $m$, $V_m$ splits into a direct sum
$V_m=V'_m\oplus F_{m-1}V_m$,
where $V'_m\subset N_m^\textup{h}\mathbb{B}$. To make this classical insight explicit, consider the endomorphism $c:\mathbb{B}_m\to \mathbb{B}_m$ introduced in the proof of lemma \ref{lemma on homology class repd by normalized generator}, an idempotent with image $N_m^\textup{h}\mathbb{B}$, preserving the subspace $V_m$. As $\im((\Id-c)|_{V_m})\subset F_{m-1}V_m$, we may define $V'_m=c(V_m)$. \textbf{(read this paragraph sometime)}

In order to define $\chi$, we recursively define its restriction to the skeleta $F_m\mathbb{B}$. In order to extend a map $\chi_{m-1}:F_{m-1}\mathbb{B}\to \pi_*(\BSWres L)$ to a map $\chi_m:F_m\mathbb{B}\to \pi_*(\BSWres L)$, we need only to specify the values of $\chi_m$ on $V'_m$, which is to choose a lift in the diagram
\[\xymatrix@R=4mm{
V'_m\ar@{-->}[r]^-{\chi_m}
\ar[d]^-{d_0^\textup{h}}
&%r1c1
N^\textup{h}_m\pi_*(\BSWres L)\ar[d]^-{d_0^\textup{h}}
\\%r1c2
ZN^\textup{h}_{m-1}\mathbb{B}\ar[r]^-{\chi_{m-1}}
&%r2c1
ZN^\textup{h}_{m-1}\pi_*(\BSWres L)%r2c2
}\]
However, we'll need to record some chain level information for what follows. Recursively for $m\geq-1$ [\textbf{say} ``view these as augmented simplicial objects, and start with $\chi_{-1}$ the identity''], we'll construct functions $\overline{\chi}_m:V'_m\to ZN^\textup{v}_*\BSWres_m L$, with the property that $\im(\overline{\chi}_m)$ is contained in the span of the classes $z_{w}$ for $w\in ZN^\textup{v}_*\BSWres_{m-1}L$. In order to do this, one may choose a basis of $V'_m$, and then for each basis element $v$, choose a $\calW(n+1)$-expression $e(s^\textup{h}_{\alpha_j}w_j)$ for $v$, with $w_j\in V'_{n_j}$ for integers $n_j\leq m-1$ and degeneracy operators $s_{\alpha_j}:V'_{n_j}\to V_{m-1}$. Then we'll define
\[\overline{\chi}_m(v)=z_{e^\textup{rep}(s^\textup{h}_{\alpha_j}\overline{\chi}_{n_j}(w_j))},\]
where by $e^\textup{rep}$, we mean the cycle in $ZN_*^\textup{v}(\BSWres_mL)$ obtained from the cycles $s^\textup{h}_{\alpha_j}\overline{\chi}_{n_j}(w_j)\in ZN_*\BSWres_{m-1}L$ using the formulae of [Curtis' explicit formulae], according to the chosen expression $e$ for $v$.
Now the class of $\overline{\chi}_m(v)$ in $\pi_*(\BSWres_mL)$ is in fact in $N^\textup{h}_m\pi_*(\BSWres L)$, since (c.f.\ [Stover lemma 2.7]) for $1\leq i \leq m$:
\[d_i^\textup{h}\overline{\chi}_m(v) =z_{d^\textup{h}_{i-1}e^\textup{rep}(s^\textup{h}_{\alpha_j}\overline{\chi}_{n_j}(w_j))}\]
where by construction $e^\textup{rep}(s^\textup{h}_{\alpha_j}\overline{\chi}_{n_j}(w_j))\in ZN^\textup{v}_*(\BSWres_{m-1} L)$ represents $\chi_{m-1}(d_{0}^\textup{h}(v))\in ZN_{m-1}^\textup{h}\pi_*(\BSWres  L)$. As such, the subscript $d^\textup{h}_{i-1}e^\textup{rep}(s^\textup{h}_{\alpha_j}\overline{\chi}_{n_j}(w_j))\in ZN^\textup{v}_*(\BSWres_{m-2} L)$ represents $0\in\pi_*(\BSWres_{m-2}L)$. Choosing a nullhomotopy $H\in N^\textup{v}_{*+1}(\BSWres_{m-2}L)$, $h_H$ is a null-homotopy of $d_i^\textup{h}\overline{\chi}_m(v)$, showing that the class of  $\overline{\chi}_m(v)$ lies in $N^\textup{h}_m\pi_*(\BSWres L)$.
%\[e^\textup{rep}(s^\textup{h}_{\alpha_j}\overline{\chi}_{n_j}(w_j))=\chi_{m-1}(d_{0}^\textup{h}(v))\in ZN^\textup{v}_*\]
Thus $\overline{\chi}_m$ does induce a map $\chi_m:V'_m\to N_m^\textup{h}\pi_*(\BSWres L)$, completing the construction of $\chi$.






Recall that the operations of \S\ref{section: vertical Koszul operations n positive} are the maps induced on cohomology by the degree -1 endomorphism $\gamma_i$ of the chain complex $N_*(Q^{\calW(n+1)} B^{\calW(n+1)}_{\bullet}\pi_*L$:
\[\gamma_i:N_{s_{n+1}+1}(Q^{\calW(n+1)} B^{\calW(n+1)}_{\bullet}\pi_*L)^{2t+1}_{s_{n+1}+i-1,\ldots}\to N_{s_{n+1}}(Q^{\calW(n+1)} B^{\calW(n+1)}_{\bullet}\pi_*L)^{t}_{s_{n+1},\ldots}.\]
If we write $V=Q^{\calL(n)}\BSWres L$ for the double complex yielding the spectral sequence, our goal is to identify these operations with the spectral sequence operations
\[\psi_{\BSW}^*\circ\vExtCohOp^{i-1}:\left(E^{p,q}_2(V)\overset{\vExtCohOp^{i-1}}{\to} E^{p,q+i-1}_2(S^2V)\overset{\psi_{\BSW}^*}{\to}E^{p+1,q+i-1}_2(V)\right)\]
using the equivalence $Q\chi$ in $s\calV$ induced by $\chi$:%a weak equivalence $Q^{\calW(n+1)}(\chi)$ in , which we denote:
\[Q\chi:\left(Q^{\calW(n+1)}B^{\calW(n+1)}_{\bullet} \pi_*L\overset{Q^{\calW(n+1)}(\chi)}{\to} Q^{\calW(n+1)}\pi_*(\BSWres )\overset{\cong }{\to}\pi_*^\textup{v}(V)\right).\]
%Now, $\gamma_i$ induces a map on $\calW(n+1)$-homology, whose dual we would like to identify with the composite:

The composite $\psi_{\BSW}^*\circ\vExtCohOp^{i-1}$ is itself the dual of an operator:
\[\left(E_{p,q}^2(V)\overset{(\vExtCohOp^{i-1})^*}{\from} E_{p,q+i-1}^2(S^2V)\overset{\psi_{\BSW}}{\from}E_{p+1,q+i-1}^2(V)\right),\]
as Singer gives the formula on $E^2$ \textbf{and it's easy} for vertical operations, and we talked about the dual operations $(\vExtCohOp^{i-1})^*$ like this in []. Thus, what we are to prove is equivalent to a commuting diagram, for $1\leq i\leq q$:
\[\xymatrix@R=4mm{
N_{p+1}(QB\pi_*L)_{q+i-1}\ar[rr]^-{(\gamma_i)_*}
\ar[d]^-{Q\chi}
&%r1c1
&%r1c2
N_p(QB\pi_*L)_q\ar[d]^-{Q\chi}
\\%r1c3
N_{p+1}(\pi_*(V))_{q+i-1}\ar[r]^-{\psi_{\BSW}}
&%r2c1
N_{p}(\pi_*(S^2V))_{q+i-1}\ar[r]^-{(\vExtCohOp^{i-1})^*}
&%r2c2
N_p(\pi_*(V))_q%r2c3
}\]

Given the description of the operations $(\vExtCohOp^{i-1})^*$, it will suffice to show that the composite
\[(QB_{p+1}\pi_*L)_{q+i-1}\overset{Q\chi}{\to}(\pi_*(Q^{\calL(n)}\BSWres_{p+1}L))_{q+i-1}\overset{\psi_{\BSW}}{\to}(\pi_*(S^2(Q^{\calL(n)}\BSWres_pL)))_{q+i-1}\]
equals the sum of the composites, for $1\leq i \leq q$ (and not all of ... are zero?): (these are the non-top)
\[(QB_{p+1}\pi_* L)_{q+i-1}\overset{(\gamma_i)_*}{\to} (QB_{p}\pi_* L)_q\overset{Q\chi}{\to} (\pi_*Q\BSWres_{p}L)_q\overset{\sigma_{i-1}}{\to} (\pi_*(S^2(Q^{\calL(n)}\BSWres_{p}L)))_{q+i-1},\]
added to the composite:
\[(QB_{p+1}\pi_* L)_{q+i-1}\overset{\psi_{\calW(n+1)}}{\to}(S^2(QB_{p}\pi_* L))_{q+i-1}\overset{S^2(Q\chi)}{\to}
(S^2(\pi_*Q\BSWres_pL))_{q+i-1}\overset{\widetilde{\nabla}}{\to}
(\pi_*(S^2(Q^{\calL(n)}\BSWres_{p}L)))_{q+i-1},\]


By lemma \ref{lemma on homology class repd by normalized generator}, we may represent any homology class of interest by an element $e\in V'_{p+1}$. We name the element in question:
\[e:=f^{p+1}_{p+1}(u_j)\in (V'_{p+1})_{q+i-1}\subset F_{\calW(n+1)}V_p\textup{ for various }u_j\in V_{p}\]
and if $f_{p+1}^\textup{rep}$ is a $\calL(n)$-expression which implements the $\calW(n+1)$-expression $f_{p+1}$ at the chain level:
\begin{alignat*}{2}
%e&:=f^{p+1}_{p+1}\cdots f^1_1x_0\in V'_{p+1}\\
\chi(e)&=z_{f_{p+1}^\textup{rep}\chi(u_j)}\\
\psi_{\BSW}(\chi(e))&=\quadratic_{\calL(n)}(f_{p+1}^\textup{rep})(\chi(u_j))
\end{alignat*}
since, by construction, $\chi(u_j)$ is contained in the span of the $z$ classes.  Taking $\quadratic(f_{p+1}^\textup{rep})$, we extract the part of $f_{p+1}^\textup{rep}$ corresponding to the quadratic grading 2 part of $f_{p+1}$, in $F_{\calW(n+1)}^{(2)}$. That is, we may write $f_{p+1}\in F_{\calW(n+1)}V_p$ as
\[f_{p+1}=\quadratic_{\calW(n+1)}(f)(u_j)+\textstyle\sum_{1\leq i\leq q} \lambda_{i-1}(\gamma_ie) + w\in F_{\calW(n+1)}V_p,\]
where $w\in F_{\calW(n+1)}(V_p)$ is the quadratic grading$\makebox[0cm][l]{}\neq2$ part of $f_{p+1}$, if we view $\quadratic_{\calW(n+1)}(f)\in S^2V_p$ as an element of $F_{\calW(n+1)}V_p$ via the inclusion $F_{\calL(n+1)}V_p\to F_{\calW(n+1)}V_p$. Then
%\[\quadratic_{\calL(n)}(f_{p+1}^\textup{rep})(\chi(u_k ))=\quadratic_{\calL(n)} (\textup{!}(\widetilde{\nabla}(y)(\chi(u_k))))+\sum_j \quadratic_{\calL(n)}(\lambda^\textup{rep}_{n_j})(\chi(x_j))\in \pi_*(S^2(Q^{\calL(n)}G_pL))\]
\begin{alignat*}{2}
\psi_\BSW(\chi(e))&=\quadratic_{\calL(n)} \left(\widetilde{\nabla}(\quadratic_{\calW(n+1)}(f))(\chi(u_j))+\sum \sigma_{i-1}(\gamma_if)(\chi(u_j))\right)\\
&=\widetilde{\nabla}(\quadratic_{\calW(n+1)}(f))(\chi(u_j))+\sum \sigma_{i-1}(\gamma_if)(\chi(u_j))\\
&=\left(\widetilde{\nabla}\circ S^2(Q\chi)\circ \psi_{\calW(n+1)}+\sum\sigma_{i-1}\circ Q\chi\circ(\gamma_i)_*\right)(e)
\end{alignat*}
Where we have removed the function $\quadratic_{\calL(n)}$ as its argument already has quadratic grading 2.
\end{proof}

\begin{proof}[Proof of \ref{EInftyCompFuncOperationsID}]
Given the general theory of Singer's external operations, we only need to show that the diagram of chain complexes
\[\xymatrix@R=4mm{
T_{m}(Q^{\calL(n)}\BSWres L)
\ar[r]^-{\psi_{\BSW}}
\ar[d]^-{\epsilon}
&%r1c1
T_{m-1}(S^2(Q^{\calL(n)}\BSWres L))\ar[d]^-{\epsilon}
\\%r1c3
N_{m}(Q^{\calL(n)}L)
\ar[r]^-{\psi_{\calW(n)}}
&%r1c1
N_{m-1}(S^2(Q^{\calL(n)} L))
}\]
commutes up to homotopy (recall that $\psi_{\BSW}$ reduces filtration by one). The required chain homotopy $\Phi$ is constructed as follows. Let $\Phi$ be zero except on 
\[N_0^\textup{h}N_*^\textup{v}(Q\BSWres L)=N_*Q^{\calL(n)}\BSW Q^{\calU(n)}B^{\calW(n)}X.\]
The generic element of $N_mQ^{\calL(n)}\BSW Q^{\calU(n)}B^{\calW(n)}X$ may be written as a sum of terms
\[z_{f^0(g^1_{i_1}(h^2_{i_1i_2}))}\textup{\quad where $f^0(g^1_{i_1}(h^2_{i_1i_2}))\in ZN_m Q^{\calU(n)}B^{\calW(n)}X$}\]
and further terms
\[h_{f^0(g^1_{i_1}(h^2_{i_1i_2}))}\textup{\quad where $f^0(g^1_{i_1}(h^2_{i_1i_2}))\in N_{m} Q^{\calU(n)}B^{\calW(n)}X$}.\]
In these descriptions, $f$ is an operator in $F_{\calL(n)}$, while $g$, $h$, etc.\ are operators in $F_{\calW(n)}$.
For brevity we will write $K=k_{f^0(g^1_{i_1}(h^2_{i_1i_2}))}$ for either of $z_{f^0(g^1_{i_1}(h^2_{i_1i_2}))}$ and $h_{f^0(g^1_{i_1}(h^2_{i_1i_2}))}$.
We then define a homotopy 
\[\Phi:T_*(Q^{\calL(n)}\BSWres L)\to N_*(S^2(Q^{\calW(n)}B^{\calW(n)}X))\]
by the formula
\[K\mapsto q(f)(g^1_{i_1}(h^2_{i_1i_2})).\]
This definition makes sense (and yields a nontrivial map) because $f^0$ is an operator in $Q^{\calU(n)}F_{\calW(n)}=F_{\calL(n)}$.
The chain map $d\Phi+\Phi d$ is a sum of three terms:
\begin{enumerate}\squishlist
\setlength{\parindent}{.25in}
\item[(a)] $d\circ\Phi:N_0^\textup{fil}N_*^\textup{int}(Q\BSWres L)\overset{\Phi}{\to} N_*(S^2(QL))\overset{d}{\to} N_{*-1}(S^2(QL))$ \textbf{sub in for X?}
\item[(b)] $\Phi\circ d^\textup{int}:N_0^\textup{fil}N_{*}^\textup{int}(Q\BSWres L)\overset{d^\textup{int}}{\to} N_0^\textup{fil}N_{*-1}^\textup{int}(Q\BSWres L)\overset{\Phi}{\to} N_{*-1}(S^2(QL))$
\item[(c)] $\Phi\circ d^\textup{fil}:N_1^\textup{fil}N_{*-1}^\textup{int}(Q\BSWres L)\overset{d^\textup{fil}}{\to} N_0^\textup{fil}N_{*-1}^\textup{int}(Q\BSWres L)\overset{\Phi}{\to} N_{*-1}(S^2(QL))$
\end{enumerate}
Let's start by identifying (a) and (b) applied to $K$:
\begin{alignat*}{2}
(d\circ\Phi)(K)&=d( q(f)(g^1_{i_1}(h^2_{i_1i_2})))\\
&=q(f)(g^0_{i_1}(h^1_{i_1i_2}))\\
&=q(f)(u(g^0_{i_1})(h^1_{i_1i_2}))\quad\textup{(as we calculate in $S^2(QL)$)}\\
(\Phi\circ d^\textup{int})(K)&=\begin{cases}
\Phi( k_{f^0(g^0_{i_1})(h^1_{i_1i_2}))}),&\textup{if `$k$' stands for `$h$'};\\
\Phi(0),&\textup{if `$k$' stands for `$z$'};
%\\,&\textup{if }
\end{cases}\\
&=q(f(g_{i_1}))(h^1_{i_1i_2})\quad\textup{(in either case).}
\end{alignat*}
By (the definition of quadratic part? calculation for this case? Decide.), the sum of these two terms equals $q(uf(g_{i_1}))(h^1_{i_1i_2})$, which is exactly the formula for $(\psi_{\calW(n)}\circ\epsilon)(K)$.

It remains to show that $\Phi\circ d^\textup{fil}$ coincides with $\epsilon^{\otimes 2}\circ\psi_{\calW(n)}$. These two maps are only nonzero on the graded part $N_1^\textup{fil}N_{*-1}^\textup{int}(Q\BSWres L)\subseteq Q^{\calL(n)}\BSW^2 L$ of the total complex, and an element therein is a linear combination
\[K':=\sum_j k_{e_j\left(s_{\alpha_{ji_0}} k_{f^0_{ji_0}g^1_{ji_0i_1}h^2_{ji_0i_1i_2}}\right)}\]
which satisfies the equation $d_1^\textup{h}(K')=0$, i.e.:
%\[d_1^\textup{fil}(K')=\sum_j k_{e_j\left(s_{\alpha_{ji_0}} {f^0_{ji_0}g^1_{ji_0i_1}h^2_{ji_0i_1i_2}}\right)}=0.\]
%We may rewrite this equation as
\[\sum_j k_{e_j(f_{ji_0})\left(s_{\alpha_{ji_0}} {g^1_{ji_0i_1}h^2_{ji_0i_1i_2}}\right)}=0\textup{ in }N_{*-1}Q^{\calL(n)}\BSW Q^{\calU(n)}B^{\calW(n)}X.\]
Since the symbol $k$ is used in this condition, it is \emph{very} strong, implying that as $j$ varies, the subscripts 
\[e_j(f_{ji_0})\left(s_{\alpha_{ji_0}} {g^1_{ji_0i_1}h^2_{ji_0i_1i_2}}\right)\in Q^{\calU(n)}(F_{\calW(n)})^*X\]
each repeat an even number of times. %As the subscripts are elements of $Q^{\Lambda}[F_{\calW(n)}]^{*}M$,
These coincidences also occur if we take quadratic parts of the $e_j(f_{ji_0})$, so that there holds the following equation in $N_{*-1}(S^2(QL))$:
%\[\sum_j {q(e^0_j(f^0_{ji_0}))\left(s_{\alpha_{ji}} g^1_{ji_1}h^2_{ji_1i_2}\right)}=0\in N_{*-1}((QX)^{\otimes2}).\]
\begin{gather*}
\sum q(e_j(f_{ji_0}))\left(s_{\alpha_{ji_0}} {g^1_{ji_0i_1}h^2_{ji_0i_1i_2}}\right)=0,\textup{ or equivalently}\\
\sum q(e_j(u(f_{ji_0})))\left(s_{\alpha_{ji_0}} {g^1_{ji_0i_1}h^2_{ji_0i_1i_2}}\right)
=
\sum q(u(e_j)(f_{ji_0}))\left(s_{\alpha_{ji_0}} {g^1_{ji_0i_1}h^2_{ji_0i_1i_2}}\right).
\end{gather*}
The proof is completed upon noting that the left hand side of this equation equals $(\epsilon^{\otimes 2}\circ \psi_{\BSW})(K')$, while the right hand side equals $(\Phi\circ d^\textup{fil})(K')$. Indeed:
\begin{alignat*}{2}
(\epsilon^{\otimes 2}\circ \psi_{\BSW})(K')
&=
\epsilon^{\otimes 2}\left(\sum q(e_j)\left(s_{\alpha_{ji_0}} k_{f^0_{ji_0}g^1_{ji_0i_1}h^2_{ji_0i_1i_2}}\right)\right)%
\\
&=
\sum q(e_j)\left(s_{\alpha_{ji_0}} f^0_{ji_0}g^1_{ji_0i_1}h^2_{ji_0i_1i_2}\right)%
\\
&=
\sum q(e_j(u(f_{ji_0})))\left(s_{\alpha_{ji_0}} g^1_{ji_0i_1}h^2_{ji_0i_1i_2}\right)=\textup{LHS}
\end{alignat*}
\textbf{(clarify $\psi_{\BSW}$?)} where we may simplify $f_{ji_0}$ to $u(f_{ji_0})$ since we take values in $S^2(QL)$. \textbf{(subscripts on $Q$)} Moreover,
\begin{alignat*}{2}
d^\textup{fil}(K')
&=
\sum e_j\left(s_{\alpha_{ji_0}} k_{f^0_{ji_0}g^1_{ji_0i_1}h^2_{ji_0i_1i_2}}\right)%
\\
&=
\sum u(e_j)\left(s_{\alpha_{ji_0}} k_{f^0_{ji_0}g^1_{ji_0i_1}h^2_{ji_0i_1i_2}}\right)%
&\qquad&\text{(in $N_{*-1}^\textup{int}(Q\BSW L)$)}\\
\Phi(d^\textup{fil}(K'))&=
\sum u(e_j)\left( q(f_{ji_0})(s_{\alpha_{ji_0}}g^1_{ji_0i_1}h^2_{ji_0i_1i_2})\right)%
&\qquad&\text{(relevant $s_{\alpha_{ji_0}}$ are $\Id$)}\\
&=
\sum q(u(e_j)(f_{ji_0}))\left(s_{\alpha_{ji_0}} {g^1_{ji_0i_1}h^2_{ji_0i_1i_2}}\right) =\textup{RHS}.
\end{alignat*}
To explain further the third equality,  the expression $\sum e_j(s_{\cdots }k_{\cdots })$ represents a sum of $\calL(n)$-expressions $e_j$ in various generators $s_{\cdots }k_{\cdots }$ of the almost free object $\BSW L\in s\calL(n)$. As we have passed to the indecomposables $Q^{\calL(n)}\BSW L$, (\textbf{note}\emph{ above (or here) that we can only say this kind of thing as the monad is really homogeneous}), any non-unit expression is sent to zero, and an expression only lies in the normalization $N_{*-1}Q^{\calL}\BSW L$ if its unital part consists only of classes $z_{\cdots}$ and $h_{\cdots }$, and not of their degeneracies. Thus, any degeneracies $s_{\cdots }$ appearing in the unital part $u(e_j)$ are in fact simply the identity, and can be ignored when we apply $\Phi$.
\end{proof}
\end{tricky proofs of operation compatibilities}
\subsection{The edge homomorphism}
For $X\in s\calW(n)$, the spectral sequence
\[(E_2)^{s_{n+2},\ldots,s_1}_t=(H^*_{\calW(n+1)}(H_*^{\calU(n)}X))^{s_{n+2},\ldots,s_1}_t\implies (H^*_{\calW(n)}X)^{s_{n+2}+s_{n+1},s_n,\ldots,s_1}_t\]
has edge homomorphism
\[(H^*_{\calW(n)}X)^{s_{n+1},\ldots,s_{1}}_t\epi
E_0^{0}(H^*_{\calW(n)}X)^{s_{n+1},\ldots,s_{1}}_t\cong (E_\infty)^{0,s_{n+1},\ldots,s_1}_t\subset (E_2)^{0,s_{n+1},\ldots,s_1}_t\]
which we may compose with the inclusion
\[(E_2)^{0,s_{n+1},\ldots,s_1}_t=((Q^{\calW(n+1)}H_*^{\calU(n)}X)^*)^{s_{n+1},\ldots,s_1}_t\subseteq (H^*_{\calU(n)}X)^{s_{n+1},\ldots,s_1}_t.\]
\textbf{to form the ``edge composite''}
One may then ask whether this composite commutes with the operations defined in \S\ref{Cohomology Operations chapter}.
\begin{prop}\label{edgehomproposition}
Suppose that $n\geq1$, $1\leq i \leq s_n$, and not all of $i-1,s_{n-1},\ldots,s_1$ are zero. This composite commutes with the vertical Steenrod operations of proposition \ref{vertical steenrod operations prop} (\textbf{don't need these conditions}):
\[\xymatrix@R=4mm{
(H^*_{\calW(n)}X)^{s_{n+1},\ldots,s_1}_t\ar[r]^-{\Sqv^i}
\ar[d]^-{\textup{edge comp.}}
&%r1c1
(H^*_{\calW(n)}X)^{s_{n+1}+1,s_n+i-1,2s_{n-1},\ldots,2s_1}_{2t+1}\ar[d]^-{\textup{edge comp.}}
\\%r1c2
(H^*_{\calU(n)}X)^{s_{n+1},\ldots,s_1}_t\ar[r]^-{\Sqv^i}
&%r1c1
(H^*_{\calU(n)}X)^{s_{n+1}+1,s_n+i-1,2s_{n-1},\ldots,2s_1}_{2t+1}
}\]
Setting $n=0$, suppose that $2\leq i <t$. The same composite commutes with the (vertical) $\delta$-operations of propositions \ref{operations on goerss homology} and \ref{operations on untable P homology}:
\[\xymatrix@R=4mm{
(H^*_{\calW(0)}X)^{s}_t\ar[r]^-{\delta_i}
\ar[d]^-{\textup{edge comp.}}
&%r1c1
(H^*_{\calW(0)}X)^{s+1}_{t+i+1}\ar[d]^-{\textup{edge comp.}}
\\%r1c2
(H^*_{\calU(0)}X)^{s}_t\ar[r]^-{\delta_i}
&%r1c1
(H^*_{\calU(0)}X)^{s+1}_{t+i+1}
}\]
\end{prop}
\begin{proof}For this proof, we'll suppress the `$(n)$' notation, as the proof is the same for all $n\geq0$. We'll also suppress all internal gradings, and write $*$ for the grading $s_{n+1}$. The edge homomorphism composite is dual to
\[H_*^{\calW}X:=\pi_*(Q^{\calW}B^{\calW}X)\overset{d_0^\textup{h}}{\from}\pi_0^\textup{h}\pi_*^\textup{v}(Q^{\calL}B^{\BSW} Q^{\calU}B^{\calW}X)\overset{z_{\DASH}}{\from}\pi_*(Q^{\calU}B^{\calW}X)\cong H_*^{\calU}X.\]
Abbreviating further by setting $D:=Q^{\calU}B^{\calW}X$ and $C:=Q^{\calL}B^{\BSW} Q^{\calU}B^{\calW}X$, the map $z_{\DASH}$ sends the class of $x\in ZN_*D$ to $[z_x]\in \pi_0^\textup{h}\pi^\textup{v}_*C$. This assignment does not produce a well defined map $\pi_*D\to N^\textup{h}_0\pi^\textup{v}_*C$, as if $y\in ZN_* D$ represents the same class as $x$, $[z_y]$ needn't equal $[z_x]$ in $N^\textup{h}_0\pi^\textup{v}_*C$: we only know that $[z_{x-y}]=0\in N_0^\textup{h}\pi_*^\textup{v}C$.
%SPELLING OUT THAE NULLHOMOTOPY: However, $x-y=d^\textup{v}_0w$ for some $w\in N_{n+1}D$, and $h_w\in N^\textup{h}_0N^\textup{v}_{n+1}C$ satisfies $d_0^\textup{v}h_w=z_{x-y}$, so that $[z_{x-y}]=0\in N_0^\textup{h}\pi_*^\textup{v}C$. 
Although $[z_{x-y}]$ needn't equal $[z_{x}-z_{y}]$ in $N_0^\textup{h}\pi_*^\textup{v}C$, the element [$z_{z_{x}-z_{y}}-z_{z_{x-y}}]\in N_1^\textup{h}\pi_*^\textup{v}C$ provides a homotopy:
%
% there is a (horizontal) homotopy between the two, which we'll present as the element $z_{z_{x}-z_{y}}\in C_1^\textup{h}\pi_*^\textup{v}C$ in the unnormalized complex:
\[d_0^\textup{h}\!\left([z_{z_{x}-z_{y}}-z_{z_{x-y}}]\right)=[z_x-z_y]-[z_{x-y}]\textup{, \  and \ }d_1^\textup{h}\!\left([z_{z_{x}-z_{y}}-z_{z_{x-y}}]\right)=[z_{{x}-{y}}-z_{{x}-{y}}]=0.\]
We may model the final isomorphism as follows. Write $U_{\calW,\calU}:\calW\to\calU$ for the forgetful functor. For any $V\in\vect{+}{n}$, there is a natural inclusion $F_{\calU}V\to U_{\calW,\calU} F_{\calW}V$ in the category  $\calU$, adjoint to the inclusion $V\to F_{\calW}V$. This morphism yields an inclusion of bar constructions, a weak equivalence $B^{\calU}U_{\calW,\calU}X\to U_{\calW,\calU}B^{\calW}X$ in $s\calU$. Suppressing the forgetful functors, for $X\in\calW$, we have a weak equivalence $Q^{\calU}B^{\calU}X\to Q^{\calU}B^{\calW}X$ inducing the isomorphism. Our conclusion is then that the entire composite $H_*^{\calU}X\to H_*^{\calW}X$ is the map on homotopy induced by the composite 
\[Q^{\calU}B^{\calU}X\to Q^{\calU}B^{\calW}X \epi Q^{\calW}B^{\calW}X,\]
and the operations we are considering are easily understood in relation to this map.
\end{proof}


\end{Composite functor spectral sequences}


\begin{comp func sseq old version}
\vfil\pagebreak
\section{\textbf{Composite functor spectral sequences OLD VERSION}}
\todo{Define the spectral sequence $H^*_{\calW(n)}\Longleftarrow H^*_{\calW(n+1)}H_*^{\Lambda(n)}$}{reference Blanc-Stover, but then give the construction of bisimplicial object (using `$W$' comonad)}
The subject of this paper is to identify the derived functors $H^*_{\calW(0)}X:=(\mathbb{L}_*Q^{\calW(0)}X)^*$, for $X\in\calW(0)$. More generally, we'll now present a spectral sequence that may be used to calculate $H^*_{\calW(n)}X$ for $X\in\calW(n)$. This will a composite functor spectral sequence analogous to Miller's spectral sequence in []. The factorization of $Q^{\calW(n)}$ we will use is of course 
\[Q^{\calW(n)}=\left(\calW(n)\overset{Q^{\calU(n)}}{\to}\calL(n)\overset{Q^{\calL(n)}}{\to}\vect{+}{n}\right)\]
There is an added challenge in this context --- indeed, the available factorization of $Q^{\calW(n)}$ is through a non-abelian category. Thus, the standard technology for composite functor spectral sequences does not apply, and we must use Blanc and Stover's methods []. They observe that the left derived functors $\mathbb{L}_*Q^{\calU(n)}X$ are calculated as the homotopy groups of a simplicial object in $\calL(n)$, namely $Q^{\calU(n)}B^{\calW(n)}X$. As such, they have the structure of a $\calL(n)\textup{-$\Pi$-algebra}$, i.e.\ they form an object of $\calW(n+1)$.  After verifying that the functor $Q^{\calU(n)}$ satisfies the requisite acyclicity condition (indeed it preserves free objects), one may apply [relevant BS theorem]: there is a spectral sequence, with $E_r\in\vect{+}{n+2}$,
\[(E_2)_{s_{n+2},\ldots,s_1}^t=((H_*^{\calW(n+1)})(\mathbb{L}_*Q^{\calU(n)})X)_{s_{n+2},\ldots,s_1}^t\implies ((H_*^{\calW(n)})X)_{s_{n+2}+s_{n+1},s_n,\ldots,s_1}^t\]
We'll prefer to work with the dual spectral sequence, (which is equivalent, as we have specified that elements of $\calW(n)$ are locally finite), which has $E^r\in\vect{n+2}{+}$:
\[(E^2)^{s_{n+2},\ldots,s_1}_t=((H^*_{\calW(n+1)})(\mathbb{L}_*Q^{\calU(n)})X)^{s_{n+2},\ldots,s_1}_t\implies ((H^*_{\calW(n)})X)^{s_{n+2}+s_{n+1},s_n,\ldots,s_1}_t\]
These are the homology and cohomology spectral sequences for a bisimplicial abelian group (precisely, an object of $ss\vect{+}{n}$), and we will need to work with this object directly.
\subsection{The Blanc-Stover comonad}
Blanc and Stover \cite{Blanc_Stover-Groth_SS.pdf} define a comonad $\BSW$ (they use the character $W$, which is already overused in this article) on $s\calL(n)$ by defining $\BSW Z$ to be the pushout
\[\xymatrix@R=4mm{
\bigsqcup_{h,n\geq0}S^n_h
\ar[r]\ar[d]^-{\bigsqcup\imath}
&%r1c1
\bigsqcup_{f,n\geq0}S^n_f\ar[d]\\%r1c2
\bigsqcup_{h,n\geq0}CS^n_h
\ar[r]&%r2c1
\BSW X%r2c2
}\]
where the leftmost two colimits are taken over all maps $h:CS^n\to X$ and the top right colimit is take over all maps $f:S^n\to X$.
%where $H^n_X:=\hom_{\calL(n)}(CS^n,X)$ and $S^n_X:=\hom_{\calL(n)}(S^n,X)$
Here, the left vertical is the coproduct of many copies of the standard inclusion $\imath:S^n\to CS^n$, and the top horizontal map sends a summand $S^n_h$ onto the sphere $S^n_{h\imath}$.

It will be useful to reinterpret this construction via the Dold-Kan correspondence. Indeed, $\hom_{s\calL(n)}(S^n,X)=\hom_{s\vect{+}{n}}(S^n,UX)=\hom_{\textup{ch}\vect{+}{n}}(S^n,UX)=$

Blanc and Stover explain that $\BSW $ actually has the structure of a comonad on $s\calL(n)$. As such, for any simplicial Lie algebra $L\in s\calL(n)$, there is a bisimplicial object $B^\BSW L$, the bar construction using the comonad $\BSW $, given by $B_{s_2}^\BSW L_{s_1}=(\BSW^{s_2+1}L)_{s_1}$. Flesh out the point here...

Now $\BSW X$ is homotopy equivalent to a coproduct of spheres. Indeed, for each map $f:S^n\to X$ that extends to a map $CS^n\to X$, \emph{choose} an extension $h_0(f):CS^n\to X$ (so that $h_0(f)\imath=f$). Then $\BSW X$ contains the contractible subobject $C_0:=\bigsqcup_f CS^n_{h_0(f)}$. Moreover, $\BSW X/C_0$ is a wedge of spheres:
\[\BSW X/C_0\cong \left(\bigsqcup_{h\textup{ not chosen}}CS^n_h/\partial(CS^n_h)\right) \sqcup\left(\bigsqcup_{f\textup{ not null}}S^n_f\right)\]
Representatives for a \emph{set} of generators of $\pi_*(\BSW X)$ are 
\[\{CS^n_h-CS^n_{h_0(h\imath)}\ |\ \textup{$h$ was not chosen}\}\cup\{S^n_f\ |\ \textup{$f$ not null}\}.\]
However, it is preferable to find a subspace of $\pi_*(\BSW X)$ which freely generates it as an element of $\calW(n+1)$.
%A thought here is that there are maps of vector spaces we can use. Let $VH_X$ be the free vector space on the homologies to $X$, and let $VS_X$ be the free vector space on the spheres.
For this purpose, observe that there's a natural map $i^*:\F_2[H_X]\to \F_2[S_X]$, and a natural map $\alpha:\ker(i^*)\to\pi_*(\BSW X)$. Moreover, there's a natural map $\beta:\F_2[S_X]\to\pi_*(\BSW X)$. Then $\im (\alpha)$ and $\im (\beta)$ are linearly independent subspaces of $\pi_*(\BSW X)$, and $\pi_*(\BSW X)$ is free on $\im (\alpha)\oplus\im (\beta)$. From this description it is clear that the generating subspaces are preserved by maps in the image of $\BSW $. Moreover, the diagonal of the comonad also preserves the subspaces $\im (\alpha)$ and $\im (\beta)$. That $\im (\alpha)$ is preserved is evident from the definitions. For $\im (\beta)$, note that $\im (\beta)\subset\pi_*(\BSW X)$ is spanned by terms $D_h-D_{h'}$ for $h,h':CS\to X$ satisfying $h\imath=h'\imath$. The diagonal applied to $D_h-D_{h'}$ may be written $D_{D_h}-D_{D_{h'}}$, and one notes that $D_h\imath=S_{h\imath}=S_{h'\imath}=D_{h'}\imath$.

Now the utility of the comonad structure lies in resolving an object $L\in s\calL(n)$ by the comonadic bar construction, $B^\BSW(L)$. This discussion shows that, except for $d_0$, all the maps in $B^\BSW L$ preserve the subspaces of generators of $\pi_*(B^\BSW L)$, which is to say that this object of $s\calW(n+1)$ is almost free. Moreover, according to [Stover 2.6], the augmentation map $\pi_*(B^\BSW L)\to \pi_*(L)$ is a weak equivalence in $s\calW(n+1)$. Thus, this Blanc-Stover $\BSW $-construction provides an almost-free replacement of $\pi_*(L)$ in $s\calW(n+1)$.
%
%
%
% Now we may choose the contractible subobjects of each $W^{n+1}X$ in such a way that the degeneracies $W^{n+1}X\to W^{n+2}X$ are homotopic to wedge inclusions of spheres, c.f.\ \cite[2.5]{StoverVanKampen.pdf}. In particular, the degeneracies of $\pi(W^{n+1}X)\in s\calW(n+1)$ preserve the chosen subspaces of generators. Moreover, all the face maps except $d_0$ are in the image of the functor $W$, and thus preserve subspaces of generators.
%
%
%
% which on homotopy yields an almost free simplicial resolution of $\pi_*(L)\in \calW(n+1)$:
%\[\vcenter{
%\def\labelstyle{\scriptstyle}
%\xymatrix@C=1.5cm@1{
%\pi_*(L)
%&
%\pi_*(WL)
%\ar[r]|(.65){s_0}
%\ar@{-->}[l]|(.65){d_0}
%&
%\pi_*(W^2L)
%\ar@<-1ex>[l]|(.65){d_0}
%\ar@<+1ex>[l]|(.65){d_1}
%\ar@<+1ex>[r]|(.65){s_0}
%\ar@<-1ex>[r]|(.65){s_1}
%&
%\pi_*(W^3L)
%\ar[l]|(.65){d_1}
%\ar@<-2ex>[l]|(.65){d_0}
%\ar@<+2ex>[l]|(.65){d_2}
%\ar[r]|(.65){s_1}
%\ar@<+2ex>[r]|(.65){s_0}
%\ar@<-2ex>[r]|(.65){s_2}
%&
%\pi_*(W^4L)\makebox[0cm][l]{\,$\cdots $}
%\ar@<-3ex>[l]|(.65){d_0}
%\ar@<-1ex>[l]|(.65){d_1}
%\ar@<+1ex>[l]|(.65){d_2}
%\ar@<+3ex>[l]|(.65){d_3}
%}}\]
%
%
%
% So, if by almost free we mean that there are vector spaces of generators which are preserved by everything except $d_0$, we have shown that $\pi(W^{\bullet+1}X)$ is an almost free object in $s\calW(n+1)$.
%Now there is a bisimplicial object $W^{n+1}X$ using the comonad structure, and we may choose the contractible subobjects of each $W^{n+1}X$ in such a way that the degeneracies $W^{n+1}X\to W^{n+2}X$ are homotopic to wedge inclusions of spheres, c.f.\ \cite[2.5]{StoverVanKampen.pdf}. In particular, the degeneracies of $\pi(W^{n+1}X)\in s\calW(1)$ preserve generators.


\subsection{The Blanc-Stover Spectral sequence}
We'll now give the derivation of the spectral sequence for $X\in\calW(n)$. Indeed, writing $B^{\calW(n)}_{\bullet}X\in s\calW(n)$ for the simplicial bar construction on $X$, we use the double complex associated with the bisimplicial object
\[(Q^{\calL(n)}B^\BSW_{\bullet}(Q^{\calU(n)}X^c))\in ss\vect{+}{n},\]
where $X^c$ is shorthand for the cofibrant replacement of $X$ in $s\calW(n)$ given by the bar construction: $B^{\calW(n)}X$.









\begin{shaded}\tiny
\subsection{A chain-level diagonal on the $\BSW $ construction}
\todo{give the construction of the chain level diagonal}{}
We have seen, for $L\in s\calL(n)$, that $\pi_*(\BSW L)$ is a free object in $\calW(n+1)$. As such, there is a diagonal
\[\phi_{\calW(n+1)}:\pi_*(\BSW L)\to \pi_*(\BSW L)\sqcup \pi_*(\BSW L).\]
In this section, we'll describe how this map is the map on homotopy induced by a morphism $\phi_\textup{ch}:\BSW L\to \BSW L\sqcup \BSW L$ in $s\calL(n)$ (note that homotopy preserves coproducts of coproducts of spheres). 

In order to construct a map $\phi_\textup{ch}$, it is enough to construct a commuting diagram
\[\xymatrix@R=4mm{
S^n\ar[r]^-{\phi_1}
\ar[d]^-{\imath}
&%r1c1
S^n\sqcup S^n\ar[d]^-{\imath\sqcup\imath}
\\%r1c2
CS^n\ar[r]^-{\phi_2}
&%r2c1
CS^n\sqcup CS^n%r2c2
}\]
The maps $\phi_1$ and $\phi_2$ can then be applied respectively to all of the sphere and cone classes appearing in $\BSW L$.
In fact, this diagram can even be formed in $s\vect{+}{n}$, before application of $F_{\calL(n)}:\vect{+}{n}\to\calL(n)$, and $\phi_1$ and $\phi_2$ are just the levelwise application of the diagonal map on a simplicial vector space. To understand the effect of $\phi_{\textup{ch}}$ on homotopy, it is enough to identify where the generators of $\pi(\BSW L)$ are sent in $\pi(\BSW L)\sqcup\pi(\BSW L)$, which is easy.
\begin{lem}
$\BSW L$ is naturally a (strict) commutative cogroup object, having comultiplication map $\phi_{\textup{ch}}$, counit map $0:\BSW L\to 0$, and inverse map $\Id:\BSW L\to \BSW L$. In particular, $\hom(\BSW L,\DASH)$ takes values in $\ensuremath{\F_2}$-vector spaces.
\end{lem}
\begin{proof}
There are a few axioms to check, for example, left counitality: that the composite 
\[\xymatrix@R=4mm@1{
\BSW L\ar[r]^-{\phi_{\textup{ch}}}
&%r1c1
\BSW L\sqcup \BSW L\ar[r]^-{\Id\sqcup0}
&%r1c2
\BSW L%r1c3
}\] is the identity. This follows since $(\Id\sqcup0)\phi_1$ is the identity of $S^n$ and $(\Id\sqcup0)\phi_2$ is the identity of $CS^n$. The other axioms follow similarly.
\end{proof}

Using the notation $*$ for the group operation on $\hom_{s\calL(n)}(\BSW L,L')$, we have the following:
\begin{lem}
For maps $f,g:\BSW L\to L'$ we have 
\[Q^{\calL(n)}(f*g)=(Q^{\calL(n)}f+Q^{\calL(n)}g):Q^{\calL(n)}\BSW L\to Q^{\calL(n)}L'.\]
\end{lem}
\begin{proof}
It's enough to check that $Q^{\calL(n)}(\phi_\textup{ch}):Q^{\calL(n)}\BSW L\to Q^{\calL(n)}(\BSW L\sqcup \BSW L)$ equals the diagonal map $Q^{\calL(n)}\BSW L\to Q^{\calL(n)}\BSW L\oplus Q^{\calL(n)}\BSW L$. For this, $Q^{\calL(n)}$ converts all the colimits involved in the construction of $\BSW L$ to colimits in $s\vect{+}{n}$, and $\phi_1$ and $\phi_2$ are both precisely the diagonal map.
\end{proof}

Now let $\overline{\xi}_{\textup{ch}}$ denote the following composite:
%\[W^2X\overset{\phi_{\textup{ch}}}{\to}(W^2X)^{\sqcup2}\overset{\phi_{\textup{ch}}\sqcup \epsilon}{\to}(W^2X)^{\sqcup2}\sqcup(WX) \overset{\epsilon^{\sqcup2}\sqcup\phi_{ch}}{\to}((WX)^{\sqcup2})^{\sqcup2}\overset{\textup{fold}}{\to}(WX)^{\sqcup2}\]
\[\makebox[0mm][r]{$\overline{\xi}_{\textup{ch}}:\ $}\BSW^2L\overset{\phi_{\textup{ch}}}{\to}(\BSW^2L)^{\sqcup2}\overset{a\sqcup b}{\to}(\BSW L)^{\sqcup2}\]
where $a,b:\BSW^2L\to(\BSW L)^{\sqcup2}$ are the composites
\[\xymatrix@R=2mm{
\makebox[0mm][r]{$a:\ $}\BSW^2L\ar[r]^-{\phi_{\textup{ch}}}
&%r1c1
(\BSW^2L)^{\sqcup2}\ar[r]^-{\epsilon^{\sqcup2}}
&%r1c2
(\BSW L)^{\sqcup2}\\
\makebox[0mm][r]{$b:\ $}\BSW^2L\ar[r]^-{\epsilon}
&%r1c1
(\BSW L)\ar[r]^-{\phi_{\textup{ch}}}
&%r1c2
(\BSW L)^{\sqcup2}
}\]
\begin{lem}
The composite 
$\BSW^2L\overset{\overline{\xi}_{\textup{ch}}}{\to}(\BSW L)^{\sqcup2}\to(\BSW L)^{\times2}$ is zero.
\end{lem}
\begin{proof}
This follows from the easy observation that both composites $(\Id\sqcup0)\overline{\xi}_{\textup{ch}}$ and $(0\sqcup\Id)\overline{\xi}_{\textup{ch}}$ equal $\epsilon:\BSW^2L\to \BSW L$.
\end{proof}
In particular, $\overline{\xi}_{\textup{ch}}$ factors through the smash coproduct, defining a natural map $\xi_{\textup{ch}}:\BSW^2L\to (\BSW L)^{\wedge 2}$.

\textbf{For the time being, write $L$ for the object $Q^{\calU(n)}X^c$ in $s\calL(n)$.}
\begin{lem}
$(d_i)^{\wedge 2}\xi_\textup{ch}=\xi_\textup{ch}d_{i+1}$ for $i\geq1$, and $(d_0)^{\wedge 2}\xi_\textup{ch}= (\xi_\textup{ch}d_{0})*(\xi_\textup{ch}d_{1})$.
The map $Q^{\calL(n)}\xi_\textup{ch}$ induces a degree $(-1,0)$ bicomplex map:
\[N_*N_*(Q^{\calL(n)}B^\BSW_{\bullet}L)^t_{s_{n+2},\ldots,s_1}\to
  N_*N_*(Q^{\calL(n)}((B^\BSW_{\bullet}L)^{\wedge 2}))^t_{s_{n+2}-1,s_{n+1},\ldots,s_1}.\]
\end{lem}
\begin{proof}
I used to say `it's just like Goerss'...
\end{proof}
\end{shaded}
As in the earlier discourse, composition with $j_{\calL(n)}$ yields degree $(-1,0)$ bicomplex map:
\[\psi_\textup{ch}:N_*N_*(Q^{\calL(n)} B^\BSW_{\bullet}L)^t_{s_{n+2},\ldots,s_1}\to
N_*N_*(S^2(Q^{\calL(n)} B^\BSW_{\bullet}L))^{t-1}_{s_{n+2}-1,s_{n+1},\ldots,s_1}.
\]


\subsection{The edge homomorphism}

\todo{Discuss the edge homomorphism $e:\left(E^2(n)\to\!\!\!\!\!\!\!\!\!\to E^\infty(n+1)\subseteq E^2_{0}(n+1)\right)$}{Look inside the hom - what happens with the Koszul differential?}

\hfil

\end{comp func sseq old version}

\begin{Calculations of HWn for n nonzero}
\section{\textbf{Calculations of ${\calW(n)}$-cohomology for $n\geq 1$}}
In this section, we'll calculate the value of $H^*_{\calW(n)}X$ for certain locally finite-dimensional objects $X$ of $\calW(n)$. In each subsection, we will write $X=V^*_{(n)}$, so that $X$ has underlying vectorspace dual to $V_{{(n)}}\in\vect{n}{+}$, and define:
\[V_{(k+1)}:=H^*_{\calU(k)}V^*_{(k)}\textup{ and }V^*_{(k+1)}:=H_*^{\calU(k)}V^*_{(k)}\textup{ for $k\geq n$},\]
so that $V^*_{(k+1)}$ is an object of $\calW(k+1)$ for each $k$, vectorspace dual to $V_{(k+1)}\in\vect{k+1}{+}$.
%We will calculate $V^*_{(k+1)}$ as an object of $\calW(k+1)$ at each stage. 
Having all of this data will allow us to draw conclusions about $H^*_{\calW(n)}V^*_{(n)}$, using, for each $k\geq n$, the $(k+1)^{\textup{st}}$ composite functor spectral sequence:
\[(E_{2,(k+1)})^{s_{k+2},\ldots,s_1}_{t}:=(H^*_{\calW(k+1)}V^*_{(k+1)})^{s_{k+2},\ldots,s_1}_{t}\implies (H^*_{\calW(k)}V^*_{(k)})^{s_{k+2}+s_{k+1},s_k,\ldots,s_1}_{t}.\]
The $1^{\textup{st}}$ composite functor spectral sequence, which calculates $H^{*}_{\calW(0)}$ from $H^{*}_{\calW(1)}$, will only appear [later].



\subsection{When $X\in\calW(n)$ is one-dimensional}
Let $X=V^*_{(n)}\in\calW(n)$ be a one dimensional object of $\calW(n)$, %dual to a vector space $V_{(n)}=(V_{(n)})^S_T=\langle v\rangle\in\vect{n}{+}$, where $T\geq1$ and $S\geq0$ (\textbf{better:} 
dual to a one-dimensional vector space $V_{(n)}\in\vect{n}{+}$, with nonzero element $v\in(V_{(n)})^S_T$. Write $v^*\in X^T_{S}$ for the non-zero element of $X$. As every $\calW(n)$-operation changes degrees, $X$ is necessarily trivial.
\begin{prop}\label{iterative calc of the Vk all trivial}
For each $k\geq n$
\[V_{(k)}=F_{\calMv(k)}F_{\calMv(k-1)}\cdots F_{\calMv(n+1)}V_{(n)},\]
and $V_{(k)}^*$ is a trivial object of $\calW(k)$.
\end{prop}
\begin{proof}
The proof is by induction, with the case $k=n$ simply our standing assumptions. If the statement holds for $V_{(k)}$, then proposition \ref{propDerivedIndTrivialUobject n at least 1} shows that the Koszul complex calculating $V_{(k+1)}$ has zero differentials, as $V_{(k)}$ has trivial $\calW(k)$-structure, so that $V_{(k+1)}=F_{\calM_{\textup{v}}(k+1)}V_{(k)}$. This has trivial $\calW(k+1)$-structure, by the results of \S\ref{section on structure on homology of koszul cx}.
\end{proof}
Our next step is to calculate, for $k\geq n$, the groups:
\[(E_{2,(k+1)})^{s_{k+2},0,s_k,\ldots,s_1}_{t}:=(H^*_{\calW(k+1)}V^*_{(k+1)})^{s_{k+2},0,s_k,\ldots,s_1}_{t}\cong H^{s_{k+2}}_{\calL(k)}V_{(k)}^*.\]
The isomorphism shown here follows from the observation that in dimension $s_{k+1}=0$, an object of $\calW(k+1)$ is nothing more than an object of $\calL(k)$. More precisely, consider the functor $\DASH_\textbf{0}:\vect{+}{k+1}\to \vect{+}{k}$ given by
\[(Y_\textbf{0})^t_{s_k,\ldots,s_1}:=Y^t_{0,s_k,\ldots,s_1}.\]
Then $\DASH_\textbf{0}$ induces a functor $\DASH_\textbf{0}:\calW(k+1)\to \calL(k)$, such that, for all $Y\in\calW(k+1)$:
\[(F_{\calW(k+1)}(Y))_\textbf{0}\cong F_{\calL(k)}(Y_\textbf{0})\textup{ and }(Q^{\calW(k+1)}Y)_\textbf{0}\cong Q^{\calL(k)}(Y_\textbf{0}),\]
so that $(Q^{\calW(k+1)}B^{\calW(k+1)}Y)_{\textbf{0}}\cong(Q^{\calL(k)}B^{\calL(k)}Y_{\textbf{0}})$ for any $Y\in s\calW(k+1)$, and thus:
\begin{prop}
Suppose that $Y\in s\calW(k+1)$, where $k\geq0$. Then
\[(H^*_{\calW(k+1)}Y)^{s_{k+2},0,s_k,\ldots,s_1}_{t}\cong (H^{s_{k+2}}_{\calL(k)}Y_\textup{\textbf{0}})^{s_k,\ldots,s_1}_t.\]
\end{prop}
Returning to the calculation at hand,
\begin{prop}\label{calculation in internal dimension zero}
For each $k\geq n$, there is an isomorphism of commutative algebras:
\[(E_{2,(k+1)})^{s_{k+2},0,s_k,\ldots,s_1}_{t}\cong(\UEAX(V_{(k)}^*)^*)^{s_{k+2},s_k,\ldots,s_1}_t.\]
%where the argument $V_{(k)}^*=(V_{(k+1)}^*)_\textbf{\textup{0}}$ of the Chevalley-Eilenberg-May complex functor $\UEAX$ is a trivial object of $\calL(k)$. 
When $v\in V_{(n)}$ is restrictable,  $\UEAX(V_{(k)}^*)^*$ is the polynomial algebra $\F_2[V_{(k)}]$ (with degree shifts: see appendix \ref{The Chevalley-Eilenberg-May complex}). When $v\in V_{(n)}$ is not restrictable, $V_{(k)}=\F_{2}\{v\}$ is one-dimensional, and $\UEAX(V_{(k)}^*)^*$ is the exterior quotient $E(v)$ of $\F_2[v]$, generated by the element $v$. In either case, for each individual value of the grading $t$, the group $(E_{2,(k+1)})^{s_{k+2},0,s_k,\ldots,s_1}_{t}$ is finite-dimensional.
\end{prop}
\begin{proof}
The only further observation necessary to prove this isomorphism is that if $v\in V_{(n)}$ is restrictable, every element of the trivial partially restricted Lie algebra $V_{(k)}$ is in restrictable degree, and that if $v\in V_{(n)}$ is not restrictable, each $V_{(k)}$ is one-dimensional, concentrated in non-restrictable degree. For the finiteness property, one simply notes that the $V_{(k)}$ have such a property, and that there is a degree shift in the algebra structure.
\end{proof}
%\begin{thm}
%There is an isomorphism $F_{\calMh(2)}F_{\calMv(2)}V_{(1)}\to H^*_{\calW(1)}V_{(1)}^* $, under which the filtration on $H^*_{\calW(1)}V_{(1)}^*$ arising from the composite functor spectral sequence coincides with the filtration on $F_{\calMh(2)}F_{\calMv(2)}V_{(1)}$ given in theorem \ref{thm on compressing seqs of steenrod ops}.
%\end{thm}

Consider the diagram:
\[\xymatrix@R=1.5mm{
&
H^*_{\calW({n+1})}H_*^{\calU(n)}V^*_{(n)}\ar@{=>}[dl]^-{g_{n+1}}\ar@{=}[d]&
H^*_{\calW({n+2})}H_*^{\calU({n+1})}V^*_{({n+1})}\ar@{=>}[dl]^-{g_{n+2}}\ar@{=}[d]&
%H^*_{\calW({n+3})}H_*^{\calU({n+2})}V^*_{({n+2})}\ar@{=>}[dl]^-{g_{n+3}}\ar@{=}[d]&
%H^*_{\calW(5)}H_*^{\calU({n+3})}V^*_{({n+3})}\ar@{=>}[dl]^-{g_5}\ar@{=}[d]
\\
 H_{\calW(n)}^{*}V^{*}_{(n)}
&H_{\calW({n+1})}^{*}V^{*}_{({n+1})}
&H_{\calW({n+2})}^{*}V^{*}_{({n+2})}
%&H_{\calW({n+3})}^{*}V^{*}_{({n+3})}
%&H_{\calW(5)}^{*}V^{*}_{(5)}
\\
&&\makebox[5cm][r]{\,$\smash{\cdots }$}
\\
 F_{\calM_{\textup{h}}({n+1})}F_{\calM_{\textup{v}}({n+1})}V_{(n)}\ar[uu]_-{\rho_n}\ar@{=}[d]
&F_{\calM_{\textup{h}}({n+2})}F_{\calM_{\textup{v}}({n+2})}V_{({n+1})}\ar[uu]_-{\rho_{n+1}}\ar@{=}[d]\ar@{=>}[ld]^-{f_{n+1}}
&F_{\calM_{\textup{h}}({n+3})}F_{\calM_{\textup{v}}({n+3})}V_{({n+2})}\ar[uu]_-{\rho_{n+2}}\ar@{=}[d]\ar@{=>}[ld]^-{f_{n+2}}
%&F_{\calM_{\textup{h}}(5)}F_{\calM_{\textup{h}}(5)}V_{({n+3})}\ar[uu]_-{\rho_{n+3}}\ar@{=}[d]\ar@{=>}[ld]
%&F_{\calM_{\textup{h}}(6)}F_{\calM_{\textup{h}}(6)}V_{(5)}\ar[uu]_-{\rho_5}\ar@{=}[d]\ar@{=>}[ld]
\\
 F_{\calMh({n+1})}V_{({n+1})}
&F_{\calMh({n+2})}V_{({n+2})}
&F_{\calMh({n+3})}V_{({n+3})}
%&F_{\calMh(5)}V_{(5)}
%&F_{\calMh(6)}V_{(6)}
}\]
For each $k\geq n$, the map $\rho_k$ is induced by the inclusion $V_{(k)}\cong H^{0}_{\calW(k)}V_{(k)}^*\subseteq H^{*}_{\calW(k)}V_{(k)}^*$ (which exists as $V_{(k)}$ is trivial) and the $F_{\calMh(k+2)}$- and $F_{\calMv(k+2)}$-operations defined on $H^{*}_{\calW(k)}V_{(k)}$. (Note that $\rho_{k}$ is a graded map, since the effect of these operations on dimensions is the same in its domain and codomain.)

The double arrow $g_{k+1}$, representing the convergence of the $(k+1)^{\textup{st}}$ composite functor spectral sequence, is in truth shorthand for the function
\[g_{k+1}:\left(E_{\infty,(k+1)}\to E_0(H_{\calW(k)}^{*}V^{*}_{(k)})\right)\]
so that $g_{k+1}$ may only be defined on the permanent cycles within $H^*_{\calW(k+1)}H_*^{\calU(k)}V^*_{(k)}$, and lands in the graded object associated to the \emph{target filtration} on $H_{\calW(k)}^{*}V^{*}_{(k)}$, the filtration associated with the spectral sequence $E_{2,(k+1)}\implies H_{\calW(k)}^{*}V^{*}_{(k)}$. Similarly, we employ the double arrow $f_{k+1}$ as shorthand for the function of theorem \ref{thm on compressing seqs of steenrod ops}, which is defined on the entirety of $F_{\calM_{\textup{h}}(k+2)}F_{\calM_{\textup{v}}(k+2)}V_{(k+1)}$, but whose true codomain is the graded object $E_0(F_{\calMh(k+1)}V_{(k+1)})$ associated with the target filtration defined in theorem \ref{thm on compressing seqs of steenrod ops}.
\begin{thm}\label{thm on collapsing of most sseqs}
For each $k\geq n$, $\im(\rho_{k+1})$ consists of permanent cycles and $\rho_k$ preserves the target filtrations, so that it is possible to form the composites $g_{k+1}\circ \rho_{k+1}$ and $E_0(\rho_{k})\circ f_{k+1}$. These composites are equal, and moreover, $\rho_k$ is an isomorphism. In particular, for $k\geq n$, the  $(k+1)^{\textup{st}}$ composite functor spectral sequence collapses at $E_2$.
\end{thm}
Before giving the proof, we remark that in some dimensions, $\rho_k$ is already known to be an isomorphism.
\begin{prop}\label{isomorphism rho k in some dims}
For $k\geq n$, $\rho_k$ is an isomorphism in dimension $s_k=0$:
\[\rho_k:(F_{\calM_{\textup{h}}(k+1)}F_{\calM_{\textup{v}}(k+1)}V_{(k)})^{s_{k+1},0,s_{k-1},\ldots,s_1}_{t} \overset{\cong}{\to}(H_{\calW(k)}^{*}V^{*}_{(k)})^{s_{k+1},0,s_{k-1},\ldots,s_1}_{t}.\]
\end{prop}
\begin{proof}
In this dimension, $\rho_k$ factors as
\begin{alignat*}{2}
(F_{\calM_{\textup{h}}(k+1)}F_{\calM_{\textup{v}}(k+1)}V_{(k)})^{s_{k+1},0,s_{k-1},\ldots,s_1}_{t}
&=
(F_{\calM_{\textup{h}}(k+1)}F_{\calM_{\textup{v}}(k+1)}V_{(k)}^{\textbf{\textup{0}}})^{s_{k+1},0,s_{k-1},\ldots,s_1}_{t}\\
&=(F_{\calM_{\textup{h}}(k+1)}V_{(k)}^{\textbf{\textup{0}}})^{s_{k+1},0,s_{k-1},\ldots,s_1}_{t}\\
&\cong (\UEAX((V_{(k)}^{\textbf{\textup{0}}})^*)^*)^{s_{k+1},0,s_{k-1},\ldots,s_1}_{t}\\
&\cong (H_{\calW(k)}^{*}V^{*}_{(k)})^{s_{k+1},0,s_{k-1},\ldots,s_1}_{t}
\end{alignat*}
Here, we are viewing $V_{(k)}^{\textbf{\textup{0}}}$, the subspace of $V_{(k)}$ in degree $s_k=0$, as an object of $\vect{k+1}{+}$ in order to apply $F_{\calMv(k+1)}$. The inclusion $F_{\calM_{\textup{h}}(k+1)}F_{\calM_{\textup{v}}(k+1)}V_{(k)}^{\textbf{\textup{0}}}\subseteq F_{\calM_{\textup{h}}(k+1)}F_{\calM_{\textup{v}}(k+1)}V_{(k)}$ restricts to the identity in degree $s_k=0$, explaining the first equality. The second equality is similar: any nontrivial $\calMv(k+1)$-operation lands outside degree $s_k=0$. The first isomorphism follows from proposition \ref{basis of free horizontal operations algebra} --- as $V_{(k)}^{\textbf{\textup{0}}}$ is concentrated in dimension $s_{k+1}=0$, the proposition ensures that $F_{\calM_{\textup{h}}(k+1)}V_{(k)}^{\textbf{\textup{0}}}$ is a quotient of the polynomial algebra on $V_{(k)}^{\textbf{\textup{0}}}$, and indeed, the same quotient as $\UEAX((V_{(k)}^{\textbf{\textup{0}}})^*)^*$. The second isomorphism is proposition \ref{calculation in internal dimension zero}.
\end{proof}
\begin{proof}[Proof of \ref{thm on collapsing of most sseqs}]
For each $k\geq n$, we'll use the diagram
\[\xymatrix@R=4mm{
%&&H_{\calW(k+1)}^{0}H_*^{\calU(k)}V_{(k)}^*\\
%V_{(k)}\ar@{..>}[r]^-{i_0}\ar@{..>}[rd]_-{j_0}&
H_{\calW(k)}^{*}V^{*}_{(k)}\ar@{..>}@/^1.22em/[rr]|-{\textup{\,edge composite\,}}
&H_{\calW(k+1)}^{*}H_*^{\calU(k)}V_{(k)}^*\ar@{=>}[l]^-{g_{k+1}}%\ar@{..>}@<.35ex>[r]^-{}%\ar[u]_-{\textup{proj onto $*=0$}}
&H^*_{\calU(k)}V^*_{(k)}\ar@{..>}[l]^-{i_1}
&V_{(k)}\ar@{..>}[l]|-{\,i_2\,}\ar@{..>}[ld]^-{j_2}\ar@{..>}@/_1.72em/[lll]|-{\,i_0\,}
\\
%V_{(k)}\ar@{..>}[r]\ar@{..>}[u]^-{\cong}&
F_{\calM_{\textup{h}}(k+1)}F_{\calM_{\textup{v}}(k+1)}V_{(k)}\ar[u]^-{\rho_k}
&W(F_{\calMv(k+1)}V_{(k)})\ar[u]\ar@{->>}[l]_-{\overline{f}_{k+1}}\ar[u]_-{\rho_{k+1}}
&F_{\calMv(k+1)}V_{(k)}\ar@{..>}[l]_-{j_1}\ar@{=}[u]%\ar@{..>}@/^1.02em/[ll]|-{\,j_3\,}
%&V_{(k)}\ar@{..>}[l]\ar@{..>}[u]_-{=}\ar@{..>}@/^1.52em/[lll]|-{\textup{inc}}
%\\
% F_{\calMh(k+1)}F_{\calMv(k+1)}V_{(k)}
%&F_{\calMh(k+2)}V_{(k+2)}
}\]
where $W(F_{\calMv(k+1)}V_{(k)})$ is the object introduced in the proof of theorem \ref{thm on compressing seqs of steenrod ops}, of which $F_{\calM_{\textup{h}}(k+2)}F_{\calM_{\textup{v}}(k+2)}F_{\calMv(k+1)}V_{(k)}$ is a quotient. Here, the maps $j_1,j_2$ are the evident inclusions of generators, while the maps $i_0,i_1,i_2$ are the inclusions arising because $V_{(k)}$ is trivial.

 We may define 
\[c:=(\rho_k\circ \overline{f}_{k+1}\circ j_1):F_{\calMv(k+1)}V_{(k)}\to H_{\calW(k)}^{*}V^{*}_{(k)},\]
without the need to pass to any associated graded objects. By construction of $\overline{f}_{k+1}$,  $c$ is induced by the inclusion $i_0$ and the $\calMv(k+1)$-structure of $H^{*}_{\calW(k)}V^{*}_{(k)}$.

The edge composite is the composite of a surjection, a monomorphism $m_1$, and an isomorphism $m_2$ (with inverse $i_1$):
\[H^*_{\calW(k)}V^*_{(k)}\epi E_{\infty,(k+1)}^{0,*}\overset{m_1}{\to}E_{2,(k+1)}^{0,*}\overset{m_2}{\to}H^*_{\calU(k)}V^*_{(k)}.\]
Moreover, $c$ is a section of the edge composite, since both maps are compatible with $\calMv(k+1)$-structures (proposition \ref{edgehomproposition}), and their composite is the identity on $V_{(k)}\subseteq  F_{\calMv(k+1)}V_{(k)}$.
In particular, the edge composite is a surjection, so that $m_{1}$ is an isomorphism. That is, every class in $\im(i_1)$ is a permanent cycle. Singer's work (c.f.\ \S??????) then shows that $\im(\rho_{k+1})$ consists of permanent cycles, as permanent cycles are preserved by the $F_{\calMh(k+2)}$- and $F_{\calMv(k+2)}$-operations on $E_{2,(k+1)}$.

%The claim that $g_{k+1}\circ \rho_{k+1}=E_0(\rho_{k})\circ f_{k+1}$ is best proven together with the claim that $\rho_k$ preserves filtration.

Any section of $H^*_{\calW(k)}V^*_{(k)}\epi E_{\infty,(k+1)}^{0,*}=E_{2,(k+1)}^{0,*}$ will realize, up to filtration, the restriction of $g_{k+1}$ to $E_{\infty,(k+1)}^{0,*}$, so we choose
\[E_{2,(k+1)}^{0,*}\overset{m_2}{\to} H^*_{\calU(k)}V^*_{(k)}\cong F_{\calMv(k+1)}V_{(k)}\overset{c}{\to}H_{\calW(k)}^{*}V^{*}_{(k)}.\]
In particular, $g_{k+1}\circ \rho_{k+1}\circ j_1=g_{k+1}\circ i_1=c\circ m_2\circ i_1=c$, up to filtration. More precisely, $g_{k+1}\circ \rho_{k+1}\circ j_1$ equals the composite
\begin{gather*}
F_{\calMv(k+1)}V_{(k)}\overset{c}{\to} H_{\calW(k)}^{*}V^{*}_{(k)}\epi E_0^0(H_{\calW(k)}^{*}V^{*}_{(k)}).
\end{gather*}
Now the target filtrations on the domain and codomain of $\rho_k$ are induced by the filtrations on the domain and codomain of $\rho_{k+1}$ by cohomological dimension $s_{k+2}$, and $\rho_{k+1}$ is a graded map. Thus, for any $w\in F^pW(F_{\calMv(k+1)}V_{(k)})$, we must see that $\rho_k(\overline{f}_{k+1}(w))$ coincides with $g_{k+1}(\rho_{k+1}(w))$ modulo $F^{p+1}H_{\calW(k)}^{*}V^{*}_{(k)}$, as this will prove both that $\rho_k$ preserves target filtrations and that  $g_{k+1}\circ \rho_{k+1}=E_0(\rho_{k})\circ f_{k+1}$. However, this coincidence follows from the fact that $c=g_{k+1}\circ \rho_{k+1}\circ j_1$, as $W(F_{\calMv(k+1)}V_{(k)})$ is generated by $\im(j_1)$ under $F_{\calMh(k+2)}$- and $F_{\calMv(k+2)}$-operations, and the definition of $\overline{f}_{k+1}$ is modeled on the interaction of $g_{k+1}$ with these operations, as studied by Singer (c.f.\ \S??????).

What remains is to show that the maps $\rho_k$ are isomorphisms. Suppose that 
\[x_{(k)}\in (E_{2,(k)})^{s_{k+1}^{k},\ldots,s_1^{k}}_t=(H^*_{\calW(k)}V^*_{(k)})^{s_{k+1}^{k},\ldots,s_1^{k}}_t.\]
Now $x_{(k)}$ is detected by some permanent cycle $x_{(k+1)}\in E_{2,(k+1)}$, which is detected by some permanent cycle $x_{(k+2)}\in E_{2,(k+2)}$, and so on, giving a sequence of elements
\begin{gather*}
x_{(r)}\in (E_{2,(r)})^{s_{r+1}^{r},\ldots,s_1^{r}}_t=(H^*_{\calW(r)}V^*_{(r)})^{s_{r+1}^{r},\ldots,s_1^{r}}_t\textup{ for $r\geq k$,}\\
\textup{where }s^{r}_{r+1}+s^{r}_{r}=s^{r-1}_{r}\textup{ and }s^{r}_{i}=s^{r-1}_{i}\textup{  for $1\leq i\leq r-1$ and $r>k$}.
\end{gather*}
We will say that $x_{(k)}$ has iterated filtration at least $(s^{k+1}_{k+2},s^{k+2}_{k+3},s^{k+3}_{k+4},\ldots)$ whenever a sequence of such classes $x_{(r)}$ exists, and partially order the set of possible iterated filtrations lexicographically. Then $x_{(r)}$ only determines $x_{(k)}$ modulo elements of $E_{2,(k)}$ of higher iterated filtration.

Simply because these gradings are always nonnegative, it is inevitable that $s_r^r=0$ for some $r\geq k$, so that by proposition \ref{isomorphism rho k in some dims}, $x_{(r)}=\rho_ry_{(r)}$ for some $y_{(r)}\in F_{\calM_{\textup{h}}(r+1)}F_{\calM_{\textup{v}}(r+1)}V_{(r)}$. Moreover, one only needs to examine finitely many sequences of gradings, each of the form
\[(s^r_{r+1},0,s^{r}_{r-1},\ldots, s^{r}_{k+1},s^k_{k},\ldots,s^k_1)\textup{ where }s^k_{k+1}=s^r_{r+1}+s^{r}_{r-1}+s^{r}_{r-2}+\cdots +s^{r}_{k+1}.\]
This, along with proposition \ref{calculation in internal dimension zero}, shows that $(H^*_{\calW(k)}V^*_{(k)})^{s_{k+1}^{k},\ldots,s_1^{k}}_t$ is finite dimensional for each given value of $t$.



By the commutativity established above, $x_{(k)}\equiv\rho_{k}f_{k+1}\cdots f_{r-1}f_r(y_{(r)})$, modulo higher iterated filtration. As this equality holds in a group which is finite dimensional for each given $t$, this establishes the surjectivity of $\rho_k$, and that
%
%
%
%{calculation in internal dimension zero},
%\[x_{(r)}\in (E_{2,(r)})^{s_{r+1}^{r},0,s_{r-1}^{r},\ldots,s_1^{r}}_{t}\cong\UEAX(V_{(r-1)}^*)^*,\]
%which is by definition generated by products of elements of 
%\[V_{(r-1)}\cong H^{0}_{\calW(r)}H_0^{\calU(r-1)}V_{(r-1)}\subseteq E_{2,(r)}.\]
%Such products are all contained in the image of $\rho_r$, so that they are all permanent cycles. But even more, we have seen that elements in the image of some $\rho_r$ are permanent cycles that detect elements in the image of $\rho_{r-1}$ whenever $r>1$. That is, \textbf{(filtration confusing)}, there is a sequence of permanent cycles (well defined up to higher filtration each time), each detecting the next. One of these is $x_{(k)}$, if we desire. Thus each $\rho_k$ is surjective, and 
every one of the spectral sequences is degenerate. Thus, we have shown that all of the maps $g_k$ are in fact isomorphisms, or rather that in the following commuting square, the $g_{k+1}$ are isomorphisms:
\[\xymatrix@R=4mm{
E_{0}(H_{\calW(k)}^{*}V^{*}_{(k)})
&H_{\calW(k+1)}^{*}H_*^{\calU(k)}V_{(k)}^*\ar@{->}[l]^-{\cong}_-{{g}_{k+1}}
\\
%V_{(k)}\ar@{..>}[r]\ar@{..>}[u]^-{\cong}&
E_0(F_{\calM_{\textup{h}}(k+1)}V_{(k+1)})\ar@{->>}[u]^-{E_0(\rho_k)}
&F_{\calM_{\textup{h}}(k+2)}F_{\calM_{\textup{v}}(k+2)}F_{\calMv(k+1)}V_{(k)}\ar@{->}[l]^-{\cong}_-{{f}_{k+1}}\ar@{->>}[u]_-{\rho_{k+1}}
}\]
For each $k$, $\rho_k$ is injective if and only if $E_0(\rho_k)$ is injective. This holds by repeated application of the snake lemma, using the fact that $\rho_k$ is surjective, and the observation that for any given value of the grading $t$, the group $(F_{\calM_{\textup{h}}(k+1)}V_{(k+1)})_t$ is finite dimensional, so that the filtrations of both the domain and codomain of $\rho_k$ are eventually zero in each degree $t$. More specifically,
\[\rho_k:(F_{\calM_{\textup{h}}(k+1)}V_{(k+1)})_t^{s_{k+1},\ldots,s_1}\to (H_{\calW(k)}^{*}V^{*}_{(k)})_t^{s_{k+1},\ldots,s_1}\]
is injective if and only if
\[E_0(\rho_k):E_0^{s'_{k+2}}(F_{\calM_{\textup{h}}(k+1)}V_{(k+1)})_t^{s'_{k+1},s_k,\ldots,s_1}\to E_0^{s'_{k+2}}(H_{\calW(k)}^{*}V^{*}_{(k)})_t^{s'_{k+1},s_k,\ldots,s_1}\]
is injective whenever $s'_{k+2}+s'_{k+1}=s_{k+1}$. As in the argument for surjectivity, in order to check that all the $\rho_k$ are injective, we now only need to check that every map
\[(F_{\calM_{\textup{h}}(r+1)}F_{\calM_{\textup{v}}(r+1)}V_{(r)})^{s_{r+1}^{r},0,s_{r-1}^{r},\ldots,s_1^{r}}_{t} \overset{\rho_r}{\to}(H_{\calW(r)}^{*}V^{*}_{(r)})^{s_{r+1}^{r},0,s_{r-1}^{r},\ldots,s_1^{r}}_{t}\]
is injective, which is part of proposition \ref{isomorphism rho k in some dims}.
%%%%%%However, in this degree, $\rho_r$ is an isomorphism, as it factors as
%%%%%%\begin{alignat*}{2}
%%%%%%(F_{\calM_{\textup{h}}(r+1)}F_{\calM_{\textup{v}}(r+1)}V_{(r)})^{s_{r+1}^{r},0,s_{r-1}^{r},\ldots,s_1^{r}}_{t}
%%%%%%&=
%%%%%%(F_{\calM_{\textup{h}}(r+1)}F_{\calM_{\textup{v}}(r+1)}V_{(r-1)})^{s_{r+1}^{r},0,s_{r-1}^{r},\ldots,s_1^{r}}_{t}\\
%%%%%%&=(F_{\calM_{\textup{h}}(r+1)}V_{(r-1)})^{s_{r+1}^{r},0,s_{r-1}^{r},\ldots,s_1^{r}}_{t}\\
%%%%%%&\cong (\UEAX(V_{(r-1)}^*)^*)^{s_{r+1}^{r},0,s_{r-1}^{r},\ldots,s_1^{r}}_{t}.
%%%%%%\end{alignat*}
%%%%%%Here, we have written $V_{(r-1)}$ for the subspace of $V_{(r)}$ in degree $s_r=0$, which we view as an object of $\vect{r+1}{+}$. Then the inclusion $F_{\calM_{\textup{h}}(r+1)}F_{\calM_{\textup{v}}(r+1)}V_{(r-1)}\subseteq F_{\calM_{\textup{h}}(r+1)}F_{\calM_{\textup{v}}(r+1)}V_{(r)}$ restricts to the identity in degree $s_r=0$, explaining the first equality. The second equality is similar: any nontrivial $\calMv(r+1)$-operation lands outside degree $s_r=0$. The final isomorphism follows from proposition \ref{basis of free horizontal operations algebra} --- as $V_{(r-1)}$ is concentrated in dimension $s_{n+1}=0$, the proposition ensures that $F_{\calM_{\textup{h}}(r+1)}V_{(r-1)}$ is a quotient of the polynomial algebra on $V_{(r-1)}$, and indeed, the same quotient as $\UEAX(V_{(r-1)}^*)^*$.
%
%, note that on $V_{(r-1)}$, there are no nontrivial horizontal Steenrod operations except for the top Steenrod operation, the squaring operation. That is, for any $v\in (V_{(r-1)})^{0,0,s_{r-1},\ldots, s_1}_t$, suppose that $\Sqh^{i_\ell}\cdots\Sqh^{i_1}$ is an admissible sequence such that $\Sqh^{i_\ell}\cdots\Sqh^{i_1}v$ is not forced to be zero \textbf{(did we ever use the word \emph{excess}, or give a basis of $\calMh$?)}. Then we must have $i_j=2^j$ for each $j$, and  $\Sqh^{i_\ell}\cdots\Sqh^{i_1}v=v^{2^\ell}$. Note that the axiom $\Sqh^1v=0$ applies iff $s_{r-1}=\cdots =s_1=0$, i.e.\ iff $v$ is not restrictable, and so is an exterior generator in $\UEAX(V_{(r-1)}^*)^*$. That is, $F_{\calM_{\textup{h}}(r+1)}V_{(r-1)}$ and $\UEAX(V_{(r-1)}^*)^*$ are generated by $V_{(r-1)}$ using the same operations (commutative products), and these operations satisfy the same relations (only that classes in $(V_{(r-1)})_t^{0,\ldots,0}$ are exterior), proving the isomorphism. See also proposition \ref{basis of free horizontal operations algebra}.
\end{proof}
\subsection{A Koszul dual Hilton-Milnor theorem}
This is an opportune moment to prove:
\begin{thm}\label{Koszul-dual Hilton-Milnor theorem}
Suppose that $X,Y\in\calW(n)$ are locally finite, with $n\geq0$. Then 
\[H^*_{\calW(n)}(X\oplus Y)\cong H^*_{\calW(n)}(X)\otimes H^*_{\calW(n)}(Y).\]
\end{thm}
\begin{proof}
Basically, $H_{*}^{\calU(k)}$ is additive, and $H^*_{\calL(k)}$ turns $\oplus$ into $\otimes $. We do induction on iterated filtration. Importantly, all of the spectral sequences involved are multiplicative. \textbf{Note that $n=0$ works as well?}
\end{proof}

\subsection{A two-dimensional example in $\calW(2)$}
In this section, we suppose that $T\geq1$, and let $X=V^*_{(2)}\in\calW(2)$ be the two-dimensional object of $\calW(2)$ spanned by non-zero classes 
\[v_{0}^{*}\in (V^*_{(2)})^{T}_{0,1}\textup{ and }v_{1}^{*}\in (V^*_{(2)})^{2T+1}_{0,2}\]
such that $v_{1}^*=v^*_0\lambda_{0}$, and all other operations are trivial. Note that this nontrivial operation may be written $v_{1}^*=\restn{(v^*_0)}$: the $\lambda_0$ is in fact a top operation.
\begin{prop}\label{2d example in w2}
For all $k\geq2$, $V_{(k)}^{*}$ is two-dimensional, spanned by
\[v_{0}^{*}\in (V^*_{(k)})^{T}_{0,\ldots,0,1}\textup{ and }v_{1}^{*}\in (V^*_{(k)})^{2T+1}_{0,\ldots,0,2},\]
with  $v_{1}^*=\restn{(v^*_0)}$ the only nontrivial operation.
\end{prop}
\begin{proof}
An induction as in the proof of proposition \ref{iterative calc of the Vk all trivial}, using the fact that at each stage, the only non-trivial $\lambda$-operation is a \emph{top} operation, and thus does not yield a differential in $K_*^{\calU(k)}V_{(k)}$. One also uses propositions \ref{LieBracketsTrivial}, \ref{QkTrivial} and \ref{Q0ZeroByPriddyAlg} to calculate the $\calW(k+1)$-structure of $V_{(k+1)}$ at each stage.
\end{proof}
\begin{prop}
For each $k\geq2$,
\[(H^*_{\calW(k+1)}V^*_{(k+1)})^{s_{k+2},0,s_k,\ldots,s_1}_{t}\cong (H^{s_{k+2}}_{\calL(k)}V^*_{(k)})^{s_k,\ldots,s_1}_t\cong (\F_2[v_1^{2}]\otimes E(v_0))_t^{s_{k+2},s_{k},\ldots,s_1}.\]
%The only nonzero classes in these groups are %
These groups are zero unless $s_k=\cdots =s_2=0$.
\end{prop}
\begin{proof}
One performs this calculation in the Chevalley-Eilenberg-May complex $\UEAX(V_{(k)}^*)^*$, which is the differential graded algebra $\F_2[v_0,v_1]$ with differential defined by
\[d(v_0)=(\dualrestn{v_0})^2=0,\ \ d(v_1)=(\dualrestn{v_1})^2=v_0^2.\qedhere\]
\end{proof}
By a greatly simplified version of the proof of theorem \ref{thm on collapsing of most sseqs}:
\begin{cor}\label{statement of result on 2d w2 example}
For each $k\geq2$, $(E_{2,(k+1)})^{s_{k+2},s_{k+1},\ldots,s_1}_{t}$ is zero unless $s_{k+1}=\cdots s_2=0$, so that the spectral sequence $E_{2,(k+1)}\implies H^{*}_{\calW(k)}V^*_{(k)}$ collapses, and in particular,
\[H^*_{\calW(2)}V_{(2)}^*\cong \F_2[v_1^{2}]\otimes E(v_0).\]
\end{cor}

%\subsection{An infinite-dimensional example in $\calW(1)$}
%In this section, we suppose that $S,T\geq1$, and let $X=V^*_{(1)}\in\calW(1)$ be the infinite dimensional object of $\calW(1)$ spanned by non-zero classes
%\[v_{j}^{*}\in (V^*_{(1)})^{2^{j}(T+1)-1}_{S+j}\textup{ for $j\geq 0$},\] such that $v_{j+1}^*=v^*_j\lambda_{1}$ for $j\geq 0$, and all other operations are trivial. %(We must have $S\geq1$ for this definition to make sense: $v_0\lambda_1$ must be defined.)
%\begin{prop}\label{calc of koszul complex in inf dim example}
%The Koszul complex $K_*^{\calU(1)}V_{(1)}^*$ has basis
%\[\left\{\Sq_{\textup{v}}(J,v_j^*)\ \middle|\ \textup{$j\geq0$, $J$ is $\Sq$-admissible, $\minDimSq(J)\leq S+j$ and $1\notin J$}\right\} \]
%%\ \genfrac{}{}{0pt}{}{\textup{$J$ $\Sq$-admissible with }\minDimSq(J)\leq S+k,}{\textup{$J$ doesn't contain 1}}\right\},\]
%and all differentials zero except for:
%\[\Sq_{\textup{v}}((i_\ell,\ldots,i_2,2),v^*_{j})\longmapsto \Sq_{\textup{v}}((i_\ell,\ldots,i_2),v^*_{j+1}),\]
%so that $V^*_{(2)}:=H_*^{\calU(1)}V_{(1)}^{*}$ is the following subquotient of $K_*^{\calU(1)}V_{(1)}^*$:
%\[\frac{{\F_{2}\left\{\Sq_{\textup{v}}(J,v_j^*)\ \middle|\ \makebox[\widthof{$\textup{$j\geq0$}$}][c]{$\textup{$j\geq0$}$}\textup{, $J$ is $\Sq$-admissible, $\minDimSq(J)\leq S+j$ and}\makebox[\widthof{$\textup{ $1,2,3\notin J$}$}][c]{$\textup{ $1,2\notin J$}$}\right\}}}{{\F_{2}\left\{\Sq_{\textup{v}}(J,v_j^*)\ \middle|\ \makebox[\widthof{$\textup{$j\geq0$}$}][c]{$\textup{$j\geq1$}$}\textup{, $J$ is $\Sq$-admissible, $\minDimSq(J)\leq S+j$ and}\makebox[\widthof{$\textup{ $1,2,3\notin J$}$}][c]{$\textup{ $1,2,3\notin J$}$}\right\}}}.\]
%Equivalently, $V_{(2)}$ is the subquotient of $F_{\calM_v(2)}V_{(1)}$ in which we restrict to the sub-$\calMv(2)$-object generated by the elements%\textbf{{(Can some of these be 0? yes if $S=1$, at least. How does this effect the calculation here?)}}
%\[\{v_0,\Sqv^2v_{1},\Sqv^3v_{1},\Sqv^2v_{2},\Sqv^3v_{2},\Sqv^2v_{3},\Sqv^3v_{3},\ldots\}\]
%and in which we set $\Sqv^2 v_{j}$ to zero for all $j\geq0$.  All of the $\Sqv^3v_{j}$ are nonzero, except when $j=S=1$. If $S\geq2$, then $V_{(2)}^{*}$ is trivial as an object of $\calW(2)$. %, and
%%\[V_{(k)}=F_{\calMv(k)}F_{\calMv(k-1)}\cdots F_{\calMv(3)}V_{(2)}.\]
%If $S=1$, then $V^*_{(2)}$ supports a single nonzero operation:
%\[(V_{(2)}^{*})_{0,1}^{T}\ni v_0^*\overset{\lambda_0}{\longmapsto}v_1^*\in (V_{(2)}^{*})_{0,2}^{2T+1}.\]
%% $\Sqv(\emptyset,v_0^*)\lambda_0=\Sqv(\emptyset,v_1^*)$. \textbf{fill in details on higher $V_{(k)}$}
%\end{prop}
%\begin{proof}
%The basis for the Koszul complex is just a reading of [above], but we must think a little about the differentials. As $\lambda_1$ is the only nonzero operation:
%\[d(\Sqv(J,v^*_j))=\sum_{ \substack{\produces{(k_{\ell},\ldots,k_2,2)}{J}{\Sq}\\(k_{\ell},\ldots,k_2){\,\Sq\textup{-admis.}}}}\!\!\!\!\!\!\!\!\!\!\!\! \Sqv((k_{\ell},\ldots,k_2),v_{j+1}^*).\]
%Consider a sequence $(k_\ell,\ldots,k_2,2)$ corresponding to a summand of this formula. Supposing that $\ell\geq2$ and $\Sqv^{k_2}\Sqv^2$ is not $\Sq$-admissible, it follows that $k_2$ is either 3 or 2, so that $\Sqv^{k_2}\Sqv^2$ is either zero or $\Sqv^{3}\Sqv^1$. As $J$ does not contain 1, and the two-sided ideal in $\LieSteen$ generated by $\Sqh^1$ is spanned by those admissible sequences ending in $\Sqh^1$, it cannot happen that $\produces{(k_{\ell},\ldots,k_2,2)}{J}{\Sq}$. Thus, the only summand appearing is that in which $(k_{\ell},\ldots,k_2,2)=J$, confirming our description of the differential. Taking the homology of this differential provides the formula for $V_{(2)}^*$, and dualizing provides that for $V_{(2)}$.
%
%In order to determine $V_{(2)}^*$ as an object of $\calW(2)$, note first that \ref{LieBracketsTrivial} and \ref{QkTrivial} show that all operations are zero except perhaps for $\lambda_0$. %Now $(\Sqv(J,w^*))\lambda_0$ is evidently zero, 
%Consider the operation $\lambda_0$ applied to a cycle of the form $\Sqv(J,v_{j}^*)\in K_{*}^{\calU(1)}V^*_{(1)}$ with $J\neq\emptyset$ (so that $1,2\notin J$). As $J$ ends in an integer no less than 3, and as $\lambda_1$ is the only nonzero operation in $V_{(1)}^*$, the second part of proposition \ref{Q0ZeroByPriddyAlg} implies that $\Sqv(J,v_{j}^*)\lambda_0=0$.
%
%In the case $J=\emptyset$, proposition \ref{Q0ZeroByPriddyAlg} states that $(\Sqv(\emptyset,v_{j}^*))\lambda_0\in V_{(2)}^*$ is represented by $(\Sqv(\emptyset,v_{j}^*\lambda_{S+j}))$, which is zero unless $j=0$ and $S=1$. Thus the only nonzero operation on $V_{(2)}^*$ is $v_0^*\lambda_0=v_1^*$ in the case $S=1$.
%\end{proof}
%
%\begin{prop}
%The spectral sequence $H^*_{\calW(2)}V_{(2)}^*\implies H^*_{\calW(1)}V_{(1)}^*$ collapses.
%\end{prop}
%\begin{proof}
%Initially, we assume that $S\geq2$. The first point is to observe that the generators $v_0$ and $\Sqv^3v_j$ ($j\geq1$) of $V_{(2)}$ are all permanent cycles in
%$(H^{*}_{\calW(2)}V_{(2)})^{0**}_{*}$. For $v_0\in (E_2)^{00S}_{T}$, this is obvious. It is less obvious for $\Sqv^3v_j$ ($j\geq1$), whose only opportunity to support a differential is
%\[\Sqv^3v_j\in (E_2)^{0,1,2+S+j}_{2^{j+1}(T+1)-1}\overset{d_2}{\to}(E_2)^{2,0,2+S+j}_{2^{j+1}(T+1)-1}.\]
%Fortunately, this target group is zero, due to the constraint that $s_2=0$. To see this, note that this group is spanned by the products of three classes in $(E_{2})^{00*}_*$, namely:
%\[v_{j_1}v_{j_2}v_{j_3}\in (E_{2})^{2,0,3S+j_1+j_2+j_3}_{(2^{j_1}+2^{j_2}+2^{j_3})(T+1)-1},\]
%and if this target group is nonzero, these indices must coincide.
%In order that $2^{j+1}$ equals $2^{j_{1}}+2^{j_{2}}+2^{j_{3}}$ it must happen that $j_1,j_2,j_3$ equal $j,j-1,j-1$ (in some order), but then $2+S+j=3S+j_1+j_2+j_3$ implies that $S+j=2$. This is impossible, as $S\geq2$ and $j\geq1$.
%
%Next, we can derive that $\Sqv^Jv_j$ is a permanent cycle for all $J,j$ such that $J$ is $\Sq$-admissible, with final entry 3 when $j>0$. For this, we'll use  proposition \ref{edgehomproposition}, that there is a commuting diagram:%as $V_{(2)}^*$ is a trivial object, there is an isomorphism $V_{(2)}\cong H^{0}_{\calW(2)}V_{(2)}^*$. In light of this isomorphism, proposition \ref{edgehomproposition} shows that the edge homomorphism itself is compatible with the action of the vertical Steenrod operations, in the sense that
%\[\xymatrix@R=4mm{
%(H^*_{\calW(1)}V_{(1)}^*)^{s_{2},s_1}_t\ar[r]^-{\Sqv^i}
%\ar[d]^-{\textup{edge hom}}
%&%r1c1
%(H^*_{\calW(1)}V_{(1)}^*)^{s_{2}+1,s_1+i-1}_{2t+1}\ar[d]^-{\textup{edge hom}}
%\\%r1c2
%(E_{2})^{0,s_{2},s_1}_t\ar@{=}[d]%\ar[r]^-{\Sqv^i}
%&%r1c1
%(E_2)^{0,s_{2}+1,s_1+i-1}_{2t+1}\ar@{=}[d]\\
%(V_{(2)})^{s_{2},s_1}_t\ar[r]^-{\Sqv^i}
%&%r1c1
%(V_{(2)})^{s_{2}+1,s_1+i-1}_{2t+1}
%}\]
%Now as we have shown that the classes $v_0$ and $\Sqv^{3}v_j$  ($j\geq1$) are all permanent cycles, they are in the image of the edge homomorphism. Then this diagram shows that all of $V_{(2)}$ is in the image of the edge homomorphism, so that every element  of $V_{(2)}$ is a permanent cycle. Finally, as $V_{(2)}$ is a trivial object of $\calW(2)$, $E_{2}$ is (freely) generated by $V_{(2)}$ under the $\calmh(3)$- and $\calmv(3)$-operations, by theorem \ref{thm on collapsing of most sseqs}. As we understand how these operations interact with the differential, we have shown that the spectral sequence collapses.
%
%When $S=1$ we must modify this proof a little. Indeed, in this case, $E_{2}=H^*_{\calW(2)}V_{(2)}^*$ is no longer generated by $V_{(2)}$ under the $\calmh(3)$- and $\calmv(3)$-operations. Rather, propositions \ref{2d example in w2} and \ref{Koszul-dual Hilton-Milnor theorem}, theorem \ref{thm on collapsing of most sseqs} and corollary \ref{statement of result on 2d w2 example} show that 
%%\[H^*_{\calW(2)}V_{(2)}^*\cong H^*_{\calW(2)}(\F_2\{v_0^*,v_1^*\}\oplus (V'_{(2)})^*)\cong \F_2[v_1^{2}]\otimes E(v_0)\otimes \calmh(3)\calmv(3)V'_{(2)} ,\]
%\begin{alignat*}{2}
%H^*_{\calW(2)}V_{(2)}^*&=H^*_{\calW(2)}(\F_2\{v_0^*,v_1^*\}\oplus (V'_{(2)})^*)\\
%&\cong \F_2[v_1^{2}]\otimes E(v_0)\otimes \calmh(3)\calmv(3)V'_{(2)}
%\end{alignat*}
%where $(V'_{(2)})^*$ is the trivial object of $\calW(2)$ dual to $V'_{(2)}$, the subquotient of $F_{\calM_v(2)}V_{(1)}$ in which we restrict to the sub-$\calMv(2)$-object generated by the elements
%\[\{\Sqv^2v_{2},\Sqv^3v_{2},\Sqv^2v_{3},\Sqv^3v_{3},\ldots\}\]
%an in which we set $\Sqv^2 v_{j}$ to zero for all $j\geq2$. This object is chosen so that $V_{(2)}^*$ decomposes as a direct sum $\F_2\{v_0^*,v_1^*\}\oplus (V'_{(2)})^*$ (of objects of $\calW(2)$), in light of the description given in  proposition \ref{calc of koszul complex in inf dim example}.
%
%Now we may proceed as before, except that in the case $S=1$, the class $v_1^{2}$ becomes relevant, as $v_1$ is no longer present in $H^*_{\calW(2)}V_{(2)}^*$, while $\Sqv^{3}v_1$ becomes zero. That $v_1^2\in(H^*_{\calW(2)}V_{(2)}^*)^{1,0,4}_{4T-3}$ cannot support differentials is obvious, while for $j\geq2$, the same degree argument as before shows that $\Sqv^3v_j$ is a permanent cycle. As before, an argument with the edge homomorphism shows that every element of $V'_{(2)}$ is a permanent cycle, and we have found enough permanent cycles to generate $E_2$ under  the $\calmh(3)$- and $\calmv(3)$-operations.
%
% \textbf{Obviously we need better notation for the nested vertical constructions.}
%\end{proof}
%
%
%






\subsection{An infinite-dimensional example in $\calW(1)$}
In this section, we suppose that $S,T\geq1$, and let $X=V^*_{(1)}\in\calW(1)$ be the infinite dimensional object of $\calW(1)$ spanned by non-zero classes
\[v_{j}^{*}\in (V^*_{(1)})^{2^{j}(T+1)-1}_{S+j}\textup{ for $j\geq 0$},\] such that $v_{j+1}^*=v^*_j\lambda_{1}$ for $j\geq 0$, and all other operations are trivial. %(We must have $S\geq1$ for this definition to make sense: $v_0\lambda_1$ must be defined.)
\begin{prop}\label{calc of koszul complex in inf dim example}
The Koszul complex $K_*^{\calU(1)}V_{(1)}^*$ has basis
\[\left\{\Sq_{\textup{v}}(J,v_j^*)\ \middle|\ \textup{$j\geq0$, $J$ is $\Sq$-admissible, $\minDimSq(J)\leq S+j$ and $1\notin J$}\right\} \]
%\ \genfrac{}{}{0pt}{}{\textup{$J$ $\Sq$-admissible with }\minDimSq(J)\leq S+k,}{\textup{$J$ doesn't contain 1}}\right\},\]
and all differentials zero except for:
\[\Sq_{\textup{v}}((i_\ell,\ldots,i_2,2),v^*_{j})\longmapsto \Sq_{\textup{v}}((i_\ell,\ldots,i_2),v^*_{j+1}).\]
\end{prop}
\begin{proof}
The basis for the Koszul complex is just a reading of [above], but we must think a little about the differentials. As $\lambda_1$ is the only nonzero operation:
\[d(\Sqv(J,v^*_j))=\sum_{ \substack{\produces{(k_{\ell},\ldots,k_2,2)}{J}{\Sq}\\(k_{\ell},\ldots,k_2){\,\Sq\textup{-admis.}}}}\!\!\!\!\!\!\!\!\!\!\!\! \Sqv((k_{\ell},\ldots,k_2),v_{j+1}^*).\]
Consider a sequence $(k_\ell,\ldots,k_2,2)$ corresponding to a summand of this formula. Supposing that $\ell\geq2$ and $\Sqv^{k_2}\Sqv^2$ is not $\Sq$-admissible, it follows that $k_2$ is either 3 or 2, so that $\Sqv^{k_2}\Sqv^2$ is either zero or $\Sqv^{3}\Sqv^1$. As $J$ does not contain 1, and the two-sided ideal in $\LieSteen$ generated by $\Sqh^1$ is spanned by those admissible sequences ending in $\Sqh^1$, it cannot happen that $\produces{(k_{\ell},\ldots,k_2,2)}{J}{\Sq}$. Thus, the only summand appearing is that in which $(k_{\ell},\ldots,k_2,2)=J$, confirming our description of the differential.
\end{proof}
\begin{prop}\label{Sg1 calc of V2}
When $S\geq2$, $V^*_{(2)}:=H_*^{\calU(1)}V_{(1)}^{*}$ is the subquotient
\[\frac{{\F_{2}\left\{\Sq_{\textup{v}}(J,v_j^*)\ \middle|\ \makebox[\widthof{$\textup{$j\geq0$}$}][c]{$\textup{$j\geq0$}$}\textup{, $J$ is $\Sq$-admissible, $\minDimSq(J)\leq S+j$ and}\makebox[\widthof{$\textup{ $1,2,3\notin J$}$}][c]{$\textup{ $1,2\notin J$}$}\right\}}}{{\F_{2}\left\{\Sq_{\textup{v}}(J,v_j^*)\ \middle|\ \makebox[\widthof{$\textup{$j\geq0$}$}][c]{$\textup{$j\geq1$}$}\textup{, $J$ is $\Sq$-admissible, $\minDimSq(J)\leq S+j$ and}\makebox[\widthof{$\textup{ $1,2,3\notin J$}$}][c]{$\textup{ $1,2,3\notin J$}$}\right\}}}\]
 of $K_*^{\calU(1)}V_{(1)}^*$. Equivalently, $V_{(2)}$ is the subquotient of $F_{\calM_v(2)}V_{(1)}$ in which we restrict to the sub-$\calMv(2)$-object generated by the elements%\textbf{{(Can some of these be 0? yes if $S=1$, at least. How does this effect the calculation here?)}}
\[\{v_0,\Sqv^2v_{1},\Sqv^3v_{1},\Sqv^2v_{2},\Sqv^3v_{2},\Sqv^2v_{3},\Sqv^3v_{3},\ldots\}\]
and in which we set $\Sqv^2 v_{j}$ to zero for all $j\geq0$.   As an object of $\calW(2)$, $V_{(2)}^{*}$ is trivial.
\end{prop}
\begin{prop}\label{Se1 calc of V2}
When $S=1$, $V^*_{(2)}:=H_*^{\calU(1)}V_{(1)}^{*}$ is the subquotient
\[\frac{{\F_{2}\left\{\Sq_{\textup{v}}(J,v_j^*)\ \middle|\ \makebox[\widthof{$\textup{$j\geq2$}$}][c]{$\textup{$j\geq0$}$}\textup{, $J$ is $\Sq$-admissible, $\minDimSq(J)\leq S+j$ and}\makebox[\widthof{$\textup{ $1,2,3\notin J$}$}][c]{$\textup{ $1,2\notin J$}$}\right\}}}{{\F_{2}\left\{\Sq_{\textup{v}}(J,v_j^*)\ \middle|\ \makebox[\widthof{$\textup{$j\geq2$}$}][c]{$\textup{$j\geq2$}$}\textup{, $J$ is $\Sq$-admissible, $\minDimSq(J)\leq S+j$ and}\makebox[\widthof{$\textup{ $1,2,3\notin J$}$}][c]{$\textup{ $1,2,3\notin J$}$}\right\}}}\]
 of $K_*^{\calU(1)}V_{(1)}^*$. Equivalently, $V_{(2)}$ is the subquotient of $F_{\calM_v(2)}V_{(1)}$ in which we restrict to the sub-$\calMv(2)$-object generated by the elements%\textbf{{(Can some of these be 0? yes if $S=1$, at least. How does this effect the calculation here?)}}
\[\{v_0,v_1,\Sqv^2v_{2},\Sqv^3v_{2},\Sqv^2v_{3},\Sqv^3v_{3},\ldots\}\]
and in which we set $\Sqv^2 v_{j}$ to zero for all $j\geq1$.  As an object of $\calW(2)$, $V_{(2)}^{*}$ admits a single nonzero operation, $\lambda_0:v_0^*\longmapsto v_1^*$, and so decomposes as the direct sum of
$\F_2\{v_0^*,v_1^*\}$ with a trivial object $(V'_{(2)})^*$, dual to $V'_{(2)}$, the subquotient of $F_{\calM_v(2)}V_{(1)}$ in which we restrict to the sub-$\calMv(2)$-object generated by
$\{\Sqv^2v_{2},\Sqv^3v_{2},\Sqv^2v_{3},\Sqv^3v_{3},\ldots\}$ and set $\Sqv^2 v_{j}$ to zero for all $j\geq2$.
\end{prop}
%
%\begin{prop}
%If $S\geq2$, then $V_{(2)}^{*}$ is trivial as an object of $\calW(2)$. %, and
%%\[V_{(k)}=F_{\calMv(k)}F_{\calMv(k-1)}\cdots F_{\calMv(3)}V_{(2)}.\]
%If $S=1$, then $V^*_{(2)}$ supports a single nonzero operation:
%\[(V_{(2)}^{*})_{0,1}^{T}\ni v_0^*\overset{\lambda_0}{\longmapsto}v_1^*\in (V_{(2)}^{*})_{0,2}^{2T+1}.\]
%% $\Sqv(\emptyset,v_0^*)\lambda_0=\Sqv(\emptyset,v_1^*)$. \textbf{fill in details on higher $V_{(k)}$}
%\end{prop}
\begin{proof}[Proof of \ref{Sg1 calc of V2} and \ref{Se1 calc of V2}]
For any $S\geq1$, taking the homology of this differential provides the formula for $V_{(2)}^*$, and dualizing provides that for $V_{(2)}$.
In order to determine $V_{(2)}^*$ as an object of $\calW(2)$, note first that \ref{LieBracketsTrivial} and \ref{QkTrivial} show that all operations are zero except perhaps for $\lambda_0$. %Now $(\Sqv(J,w^*))\lambda_0$ is evidently zero, 
Consider the operation $\lambda_0$ applied to a cycle of the form $\Sqv(J,v_{j}^*)\in K_{*}^{\calU(1)}V^*_{(1)}$ with $J\neq\emptyset$ (so that $1,2\notin J$). As $J$ ends in an integer no less than 3, and as $\lambda_1$ is the only nonzero operation in $V_{(1)}^*$, the second part of proposition \ref{Q0ZeroByPriddyAlg} implies that $\Sqv(J,v_{j}^*)\lambda_0=0$.

In the case $J=\emptyset$, proposition \ref{Q0ZeroByPriddyAlg} states that $(\Sqv(\emptyset,v_{j}^*))\lambda_0\in V_{(2)}^*$ is represented by $(\Sqv(\emptyset,v_{j}^*\lambda_{S+j}))$, which is zero unless $j=0$ and $S=1$. Thus the only nonzero operation on $V_{(2)}^*$ is $v_0^*\lambda_0=v_1^*$ in the case $S=1$.
\end{proof}
\begin{thm}\label{W2 to W1 collapse}
The spectral sequence $H^*_{\calW(2)}V_{(2)}^*\implies H^*_{\calW(1)}V_{(1)}^*$ collapses, with
\[E_2=H^*_{\calW(2)}V_{(2)}^*\cong\begin{cases}
F_{\calmh(3)}F_{\calmv(3)}V_{(2)},&\textup{if }S\geq2;\\
F_{\calmh(3)}F_{\calmv(3)}V'_{(2)}\otimes \F_2[v_1^{2}]\otimes E(v_0),&\textup{if }S=1.
%\\,&\textup{if }
\end{cases}
\]
\end{thm}
\begin{proof}
The calculations of $E_2$ follow from propositions \ref{Koszul-dual Hilton-Milnor theorem}, \ref{2d example in w2}, \ref{Sg1 calc of V2} and \ref{Se1 calc of V2}, theorem \ref{thm on collapsing of most sseqs} and corollary \ref{statement of result on 2d w2 example}. What remains is to prove the collapsing result in each case.

Suppose that $S\geq2$. The first point is to observe that the generators $v_0$ and $\Sqv^3v_j$ ($j\geq1$) of $V_{(2)}$ are all permanent cycles in
$(H^{*}_{\calW(2)}V_{(2)})^{0**}_{*}$. For $v_0\in (E_2)^{00S}_{T}$, this is obvious. It is less obvious for $\Sqv^3v_j$ ($j\geq1$), whose only opportunity to support a differential is
\[\Sqv^3v_j\in (E_2)^{0,1,2+S+j}_{2^{j+1}(T+1)-1}\overset{d_2}{\to}(E_2)^{2,0,2+S+j}_{2^{j+1}(T+1)-1}.\]
Fortunately, this target group is zero, due to the constraint that $s_2=0$. To see this, note that this group is spanned by the products of three classes in $(E_{2})^{00*}_*$, namely:
\[v_{j_1}v_{j_2}v_{j_3}\in (E_{2})^{2,0,3S+j_1+j_2+j_3}_{(2^{j_1}+2^{j_2}+2^{j_3})(T+1)-1},\]
and if this target group is nonzero, these indices must coincide.
In order that $2^{j+1}$ equals $2^{j_{1}}+2^{j_{2}}+2^{j_{3}}$ it must happen that $j_1,j_2,j_3$ equal $j,j-1,j-1$ (in some order), but then $2+S+j=3S+j_1+j_2+j_3$ implies that $S+j=2$. This is impossible, as $S\geq2$ and $j\geq1$.

Next, we can derive that $\Sqv^Jv_j$ is a permanent cycle for all $J,j$ such that $J$ is $\Sq$-admissible, with final entry 3 when $j>0$. For this, we'll use  proposition \ref{edgehomproposition}, that there is a commuting diagram:%as $V_{(2)}^*$ is a trivial object, there is an isomorphism $V_{(2)}\cong H^{0}_{\calW(2)}V_{(2)}^*$. In light of this isomorphism, proposition \ref{edgehomproposition} shows that the edge homomorphism itself is compatible with the action of the vertical Steenrod operations, in the sense that
\[\xymatrix@R=4mm{
(H^*_{\calW(1)}V_{(1)}^*)^{s_{2},s_1}_t\ar[r]^-{\Sqv^i}
\ar[d]^-{\textup{edge hom}}
&%r1c1
(H^*_{\calW(1)}V_{(1)}^*)^{s_{2}+1,s_1+i-1}_{2t+1}\ar[d]^-{\textup{edge hom}}
\\%r1c2
(E_{2})^{0,s_{2},s_1}_t\ar@{=}[d]%\ar[r]^-{\Sqv^i}
&%r1c1
(E_2)^{0,s_{2}+1,s_1+i-1}_{2t+1}\ar@{=}[d]\\
(V_{(2)})^{s_{2},s_1}_t\ar[r]^-{\Sqv^i}
&%r1c1
(V_{(2)})^{s_{2}+1,s_1+i-1}_{2t+1}
}\]
Now as we have shown that the classes $v_0$ and $\Sqv^{3}v_j$  ($j\geq1$) are all permanent cycles, they are in the image of the edge homomorphism. Then this diagram shows that all of $V_{(2)}$ is in the image of the edge homomorphism, so that every element  of $V_{(2)}$ is a permanent cycle. Finally, as $E_{2}$ is (freely) generated by $V_{(2)}$ under the $\calmh(3)$- and $\calmv(3)$-operations, and we understand how these operations interact with the differential, this shows that the spectral sequence collapses.

%%When $S=1$ we must modify this proof a little, as $E_{2}=H^*_{\calW(2)}V_{(2)}^*$ is no longer generated by $V_{(2)}$ under the $\calmh(3)$- and $\calmv(3)$-operations. Rather, propositions \ref{2d example in w2} and \ref{Koszul-dual Hilton-Milnor theorem}, theorem \ref{thm on collapsing of most sseqs} and corollary \ref{statement of result on 2d w2 example} show that 
%%%\[H^*_{\calW(2)}V_{(2)}^*\cong H^*_{\calW(2)}(\F_2\{v_0^*,v_1^*\}\oplus (V'_{(2)})^*)\cong \F_2[v_1^{2}]\otimes E(v_0)\otimes F_{\calmh(3)}F_{\calmv(3)}V'_{(2)} ,\]
%%\begin{alignat*}{2}
%%H^*_{\calW(2)}V_{(2)}^*&=H^*_{\calW(2)}(\F_2\{v_0^*,v_1^*\}\oplus (V'_{(2)})^*)\\
%%&\cong \F_2[v_1^{2}]\otimes E(v_0)\otimes F_{\calmh(3)}F_{\calmv(3)}V'_{(2)}.
%%\end{alignat*}
%where $(V'_{(2)})^*$ is the trivial object of $\calW(2)$ dual to $V'_{(2)}$, the subquotient of $F_{\calM_v(2)}V_{(1)}$ in which we restrict to the sub-$\calMv(2)$-object generated by the elements
%\[\{\Sqv^2v_{2},\Sqv^3v_{2},\Sqv^2v_{3},\Sqv^3v_{3},\ldots\}\]
%an in which we set $\Sqv^2 v_{j}$ to zero for all $j\geq2$. This object is chosen so that $V_{(2)}^*$ decomposes as a direct sum $\F_2\{v_0^*,v_1^*\}\oplus (V'_{(2)})^*$ (of objects of $\calW(2)$), in light of the description given in  proposition \ref{calc of koszul complex in inf dim example}.
Instead $E_2$ is generated by the class $v_1^{2}$  (and we no longer need to consider $\Sqv^{3}v_1$, as it is zero when $S=1$). That $v_1^2\in(H^*_{\calW(2)}V_{(2)}^*)^{1,0,4}_{4T-3}$ cannot support differentials is obvious, while for $j\geq2$, the same degree argument as before shows that $\Sqv^3v_j$ is a permanent cycle. As before, an argument with the edge homomorphism shows that every element of $V'_{(2)}$ is a permanent cycle, so $E_2$ is again generated by permanent cycles under the $\calmh(3)$- and $\calmv(3)$-operations, completing the proof.

 \textbf{Obviously we need better notation for the nested vertical constructions.}
\end{proof}

\begin{cor}
Some corollary giving full structure of $H^*_{(1)}$.
\end{cor}




\end{Calculations of HWn for n nonzero}

\begin{Calculations of HW0}
\vfil\pagebreak

\section{\textbf{Calculations of ${\calW(0)}$-cohomology}}
Let $X=V^*_{(0)}\in\calW(0)$ be a one dimensional object of $\calW(0)$, dual to a one-dimensional vector space $V_{(0)}\in\vect{0}{+}$, with nonzero element $\imath\in(V_{(0)})_T$. Write $\imath^*\in X^T$ for the non-zero element of $X$. As every $\calW(0)$-operation changes degrees, $X$ is necessarily trivial.

In order to state our next result, define
\[\aDT:=\left\{I\ \middle|\ \textup{$I$ a non-empty $\delta$-admissible sequence with }\minDimP(I)\leq T\right\},\]
and define a function $t:\aDT\to \aDT$ by
\[I=(i_{\ell},\ldots,i_1)\overset{\smash{t}}{\longmapsto }(T+nI+\ell,i_\ell,\ldots,i_1)\]
This is indeed a well defined injective endomorphism of the set $\aDT$, in that it preserves admissibility and the condition $\minDimP(I)\leq T$. The claim about $\minDimP(I)$ holds essentially by definition. For $\delta$-admissibility:
\begin{alignat*}{2}
i_{\ell}
&\leq
\ell-1+i_{\ell-1}+\cdots +i_1+T&\qquad&\textup{(as $\minDimP(I)\leq T$)}%
\\
% Left hand side
2i_{\ell}
% Relation
&\leq
% Right hand side
\ell-1+i_{\ell}+\cdots +i_1+T\\
&< \ell+i_{\ell}+\cdots +i_1+T%
\end{alignat*}
as required. The strict inequality here is interesting: suppose that $I=(i_\ell,\ldots,i_1)\in \aDT$ and 
\[(i_{j+1},\ldots,i_1)=t(i_{j},\ldots,i_1)\textup{ for some $j$, $1\leq j<i-1$.}\]
Thanks to this strict inequality, it need not be true that $I=t^{i-1-j}(i_{j+1},\ldots,i_1)$. %This is in contrast to the situation for standard unstable algebras over the Steenrod algebra. Suppose that $x\in H^n(X;\F_2)$, for $X$ a topological space, and $I$ is a $\Sq$-admissible sequence with excess $e(I)\leq $
Define:
\[\aDTirr:= \aDT\setminus\im(t:\aDT\to\aDT),\]
the set of sequences in $\aDT$ not in the image of $t$, so that we may decompose $\aDT$ as the disjoint union
\[\aDT=\bigsqcup_{\smash{I\in \aDTirr}}\left\{I,tI,t^2I,\ldots\right\}.\]
\begin{prop}
\label{calc of V1 from W0 sphere}
$V^*_{(1)}:=H_*^{\calU(0)}V_{(0)}^{*}$ has basis $\{\imath^*\}\sqcup\left\{\delta(I,\imath^*)\ \middle|\ I\in\aDT\right\}$ and all $\calW(1)$-operations trivial except for $\lambda_1$, which is defined (only when $\ell(I)\geq1$) by
\[\delta(I,\imath^*)\overset{\smash{\lambda_1}}{\longmapsto }\delta(tI,\imath^*).\]
Thus, as an object of $\calW(1)$, $V_{(1)}^*$ decomposes as a direct sum
\[\F_2\{\imath^*\}\oplus\bigoplus_{\smash{I\in \aDTirr}}\F_2\left\{\delta(I,\imath^*)\lambda_1^j\ |\ j\geq0\right\}.\]
\end{prop}
\begin{proof}
The basis of the Koszul complex was described in proposition \ref{propDerivedIndTrivialUobject n=0}, and the Koszul differential is zero as $X$ is trivial. The $\lambda$-operations were calculated in proposition \ref{QkTrivial}.
\end{proof}
Now we have put considerable effort into calculating $H^*_{\calw(1)}$ of each summand in this decomposition: theorem \ref{thm on collapsing of most sseqs} proves that
\[H^*_{\calw(1)}(\F_2\{\imath^*\})\cong F_{\calmh(2)}F_{\calmv(2)}(\F_2\{\imath\})\cong E(\imath),\]
while propositions \ref{Sg1 calc of V2} and \ref{Se1 calc of V2} calculate
\[H^*_{\calw(1)}\left(\F_2\left\{\delta(I,\imath^*)\lambda_1^j\ |\ j\geq0\right\}\right)\textup{ for $I\in\aDTirr$.}\]
With a view to calculating the first composite functor spectral sequence,
we catalogue a collection of generators of $E_{2,(1)}$ under the $\calmv(2)$- and $\calmh(2)$-operations. The `fundamental class' $\imath\in (E_2)^{0,0,(1)}_{d}$ is an exterior generator (arising in theorem  \ref{thm on collapsing of most sseqs}). %, as are the classes $\delta_k^\textup{v}\imath\in (E_2)^{0,1,(2)}_{d+k+1}$ for $2\leq k\leq d$ (denoted $v_0$ in theorem \ref{W2 to W1 collapse} \textbf{or the corollary to that: write it)).
Moreover, for all $I\in\aDTirr$, there are further generators, arising in theorem \ref{W2 to W1 collapse}: \textbf{or the corollary to that: write it)}
\begin{alignat*}{2}
\delta_I^{\textup{v}}\imath&\in (E_2)^{0,\ell I,(2^{\ell I})}_{d+nI+\ell I}&\qquad&\\
\Sq_{\textup{v}}^{3}\delta_{t^j\!I}^{\textup{v}}\imath&\in (E_2)^{1,2+\ell I+j,(2^{1+\ell I+j})}_{2^{j+1}(d+nI+\ell I+1)-1}&\qquad&\textup{(when $j\geq1$, but not $j=\ell I=1$),}\\
(\delta_{tI}^{\textup{v}}\imath)^{2}&\in (E_2)^{1,4,(2^{3})}_{2^{2}(d+nI+\ell I+1)-1}&\qquad&\textup{(when $\ell I=1$)},
\end{alignat*}
where they are referred to  as $v_0$, $v_j$ and $v_1^2$ respectively.
Note that this final generator, $(\delta_{tI}^{\textup{v}}\imath)^{2}$, has the same degrees as the \emph{missing} generator $\Sq_{\textup{v}}^{3}\delta_{t^j\!I}^{\textup{v}}\imath$ when $j=\ell I=1$.
\begin{thm}
The first composite functor spectral sequence collapses at $E_2$:
\[E_{2,(1)}=H^*_{\calw(1)}V^*_{(1)}\implies H^*_{\calw(0)}V^*_{(0)}.\]
\end{thm}
\begin{proof}
Evidently the funcdamental class is a permanent cycle, so prove that the spectral sequence collapses, it is enough to show that no classes
%
%\vfil\pagebreak
%
%
%\[x\in (E_2)^{0,\ell I,(2^{\ell I})}_{d+nI+\ell I}\textup{ or }y\in (E_2)^{1,2+\ell I+j,(2^{1+\ell I+j})}_{2^{j+1}(d+nI+\ell I+1)-1}\]
%can support a differential, whenever $j\geq1$ and $I$ is a non-empty $\delta$-admissible sequence.
%
%Now the whole $E_2$ page is a subquotient of the polynomial algebra on symbols
%\[\Sqh^A\Sqv^B\delta_C^\textup{v}\imath\in (E_2)^{\ell B+nA,2^{\ell A}(nB-\ell B+\ell C),(2^{\ell A+\ell B+\ell C})}_{2^{\ell A+\ell B}(d+nC+\ell C+1)-1},\]
%in which either $B$ is empty or has rightmost entry a 3.
%Thus, if there is to be a differential $d_r$ (for $r\geq2$) supported by $y$, %a class in $(E_2)^{1,2+\ell I+j,(2^{1+\ell I+j})}_{2^{j+1}(d+nI+\ell I+1)-1}$, 
%then $d_{r}(y) $ must be a product of $N$ such classes:
%\[\textstyle\prod_{k=1}^{N}\Sqh^{A_k}\Sqv^{B_k}\delta_{C_k}^\textup{v}\imath\in (E_2)^{-1+\sum(\ell B_k+nA_k+1),\sum2^{\ell A_k}(nB_k-\ell B_k+\ell C_k),(\sum2^{\ell A_k+\ell B_k+\ell C_k})}_{-1+\sum2^{\ell A_k+\ell B_k}(d+nC_k+\ell C_k+1)},\]
%and $r=-2+\sum(\ell B_k+nA_k+1)$.
%Now $d_r$ preserves the quatratic grading, and convexity of the exponential function implies
%\[1+\ell I+j\leq \textstyle\sum_k \left[\ell A_k+\ell B_k+\ell C_k\right].\]
%As the total degree of the differential is one, we must also have
%\[4+\ell I+j=-1+\textstyle\sum_{k}\left[\ell B_k+nA_{k}+1\right]+\textstyle\sum_{k}\left[2^{\ell A_k}(nB_k-\ell B_k+\ell C_k)\right]\]
%and combining this with the above inequality yields, after a little manipulation
%\[4\geq\textstyle\sum_k\left[(nA_k-\ell A_k)+1+(2^{\ell A_k}-1)\ell C_k+2^{\ell A_k}(nB_k-\ell B_k)\right],\]
%where each quantity $(nA_k-\ell A_k)$, $1$, $(2^{\ell A_k}-1)\ell C_k$ and $2^{\ell A_k}(nB_k-\ell B_k)$ is nonnegative.
%One sees readily that if any $\ell A_k$ is nonzero, the corresponding summand would exceed 4, as in this case, $B_k$ must be non-empty, so that $nB_k-\ell B_k\geq2$. Thus we may take all $A_k=0$, and the inequality becomes
%\[4\geq\textstyle\sum_k\left[1+(nB_k-\ell B_k)\right].\]
%Now if $\ell B_k\geq1$ then  $(nB_k-\ell B_k)\geq2$ and if $\ell B_k\geq2$ then  $(nB_k-\ell B_k)\geq7$. Thus there can be at most one term with $\ell B_k\neq0$, and for any such term, $\ell B_k=1$. In the case $\ell B_1=1$, we can have at most one other term, and the target group is either $(E_2)_*^{1,*,*}$ or $(E_2)_*^{2,*,*}$ --- neither of these groups can be the target of a $d_r$ for $r\geq2$. This leaves us investigating the case in which all $A_k$ and $B_k$ are empty, where this inequality becomes $N\leq 4$, so that  $r=N-2$ must equal 2. That is, in this case we are demanding that the gradings in
%\[(E_2)^{3,\sum \ell C_k,(\sum2^{\ell C_k})}_{-1+\sum(d+nC_k+\ell C_k+1)}\textup{ and }(E_2)^{3,1+\ell I+j,(2^{1+\ell I+j})}_{2^{j+1}(d+nI+\ell I+1)-1}\]
%coincide
%
%\vfil\pagebreak
%
%
\[x\in (E_2)^{0,\ell I,(2^{\ell I})}_{d+nI+\ell I}\textup{ or }y\in (E_2)^{1,2+\ell I+j,(2^{1+\ell I+j})}_{2^{j+1}(d+nI+\ell I+1)-1}\]
can support a differential, $I$ a non-empty $\delta$-admissible sequence.

To see this, all one needs to know about the entire $E_2$ page is that it is a subquotient (in which $\imath^2=0$) of the polynomial algebra on symbols
\[\Sqh^A\Sqv^B\delta_C^\textup{v}\imath\in (E_2)^{\ell B+nA,2^{\ell A}(nB-\ell B+\ell C),(2^{\ell A+\ell B+\ell C})}_{2^{\ell A+\ell B}(d+nC+\ell C+1)-1},\]
in which $B$ is $\Sq$-admissible, B does not contain $1$ or $2$, if $C$ is empty then so is $B$, and if $B$ is empty then so is $A$. These conditions imply that $nB-2\ell B\geq2^{\ell B}-1$ (\textbf{recheck!}). %Moreover, in this subquotient, $\imath^2=0$ (among other relations explained above).

If there is to be a differential $d_r$ supported by $y\in E_2$, %a class in $(E_2)^{1,2+\ell I+j,(2^{1+\ell I+j})}_{2^{j+1}(d+nI+\ell I+1)-1}$, 
then $d_{r}(y) $ must be a sum of products of $N\geq1$ such classes. The generic such monomial may be written as:
\[\textstyle\prod_{k=1}^{N}\Sqh^{A_k}\Sqv^{B_k}\delta_{C_k}^\textup{v}\imath\in (E_2)^{\sum(\ell B_k+nA_k)+N-1,\sum2^{\ell A_k}(nB_k-\ell B_k+\ell C_k),(\sum2^{\ell A_k+\ell B_k+\ell C_k})}_{\sum2^{\ell A_k+\ell B_k}(d+nC_k+\ell C_k+1)-1}\]
in which $\ell C_k=0$ for at most one $k$. We derive the following:
\begin{gather}
\textstyle \sum(\ell B_k+nA_k)\geq4-N,\label{1eqnofindices}\\
\log_2(N)+\textstyle \frac{1}{N}\textstyle \sum_k \left[\ell A_k+\ell B_k+\ell C_k\right]\geq1+\ell I+j,\label{2eqnofindices}\\
4+\ell I+j=\textstyle\sum_{k}\left[\ell B_k+nA_{k}\right]+N-1+\textstyle\sum_{k}\left[2^{\ell A_k}(nB_k-\ell B_k+\ell C_k)\right]\label{3eqnofindices},\\
\log_2(N)\geq \textstyle\sum_{k}\left((2^{\ell A_k}-\frac{1}{N})\ell C_k+\left[\left(2^{\ell A_k}(nB_k-\ell B_k)-\frac{1}{N}\ell B_k\right)-\frac{1}{N}(\ell A_k)\right]\right).\label{4eqnofindices}
\end{gather}
The inequality (\ref{1eqnofindices}) is just the requirement that $r\geq2$, while (\ref{2eqnofindices}) results from the observation that $d_r$ preserves the quadratic grading and the convexity of the exponential function. Equation (\ref{3eqnofindices}) holds since the total degree of the differential is one, and (\ref{4eqnofindices}) is derived by rearranging the sum of (\ref{1eqnofindices}), (\ref{2eqnofindices}) and (\ref{3eqnofindices}).
%\[r=N-2+\sum(\ell B_k+nA_k)\geq2.\]
%Now $d_r$ preserves the quadratic grading, and convexity of the exponential function implies that
%\[1+\ell I+j\leq \log_2(N)+\frac{1}{N}\textstyle \sum_k \left[\ell A_k+\ell B_k+\ell C_k\right].\]
%%with equality if and only if all of the $\ell A_k+\ell B_k+\ell C_k$ coincide. This obviously occurs when $N=1$, and also when $N=2$ as we are considering an integer equation $2^\alpha=2^\beta+2^\gamma$.
%As the total degree of the differential is one, we must also have
%\[4+\ell I+j=-1+\textstyle\sum_{k}\left[\ell B_k+nA_{k}+1\right]+\textstyle\sum_{k}\left[2^{\ell A_k}(nB_k-\ell B_k+\ell C_k)\right]\]
%and combining this with the above inequalities yields, after a little manipulation:
%\[\log_2(N)\geq \textstyle\sum_{k}\left((2^{\ell A_k}-\frac{1}{N})\ell C_k+\left[\left(2^{\ell A_k}(nB_k-\ell B_k)-\frac{1}{N}\ell B_k\right)-\frac{1}{N}(\ell A_k)\right]\right).\]
This is a very strong inequality, since the expression $2^{\ell A_k}(nB_k-\ell B_k)-\frac{1}{N}\ell B_k$ is at least $2^{\ell B_k}-\frac{1}{N}$, and $nB_k-\ell B_k\geq2$ of $\ell B_k\neq0$. Thus, in  (\ref{4eqnofindices}), each expression in square brackets is always non-negative, is at least $2-\frac{1}{N}$ when $\ell B_k\neq0$, and exceeds $2-\frac{1}{N}$ if $\ell B_k\geq2$ or $\ell A_k\neq0$. 

When $N=1$ or $N=3$, $\log_2(N)<2-\frac{1}{N}$, so that (\ref{4eqnofindices}) implies that $\ell B_k=0$ for all $k$, violating (\ref{1eqnofindices}).
When $N\leq2$, $\log_2(N)\leq 2-\frac{1}{N}$, so that (\ref{4eqnofindices}) implies that $\ell B_k\neq0$ for at most one $k$, with $\ell B_k=1$, violating (\ref{1eqnofindices}).
When $N\geq4$, all but at most one of the summands $(2^{\ell A_k}-\frac{1}{N})\ell C_k$ in (\ref{4eqnofindices}) is at least $\frac{3}{4}$, and as $\frac{3}{4}(N-1)\geq\log_2(N)$ when $N\geq4$, (\ref{4eqnofindices}) is violated. Thus $y\in E_2$ is a permanent cycle.

Performing the same calculations for $d_r(x)$, we find that the inequality (\ref{4eqnofindices}) is unchanged, while (\ref{1eqnofindices}) is replaced by
\begin{gather}
\textstyle \sum(\ell B_k+nA_k)\geq3-N.\label{5eqnofindices}\end{gather}
The argument is unchanged when $N=1$ or $N\geq4$, while if $2\leq N\leq3$ we may still draw the samw conclusions from (\ref{4eqnofindices}). When $N=2$, we may assume that $\ell B_1=1$ and $\ell B_2=0$, and although (\ref{5eqnofindices}) is not violated, (\ref{4eqnofindices}) is violated as $\ell C_1\neq0$. When $N=3$, we must have $\ell C_k=0$ for each $k$, and the following equations must be satisfied
\[\ell I-1=\ell C_1+\ell C_2+\ell C_3,\ \  2^{\ell I}=2^{\ell C_1}+2^{\ell C_2}+2^{\ell C_3}.\]
As in the proof of theorem \ref{W2 to W1 collapse}, these equations imply that $\ell C_1,\ell C_2,\ell C_3$ equal $\ell I-1,\ell I-2,\ell I-2$, in some order. The first equation then implies that $\ell I=2$, implying that $\ell C_k=0$ for more than one $k$, which we have prohibited. Thus $x\in E_2$ is a permanent cycle.
\end{proof}
\begin{cor}
Some corollary giving full structure of $H^*_{(0)}$.
\end{cor}
%
%\[3+\log_2(N)+\frac{1}{N}\textstyle \sum_k \left[\ell A_k+\ell B_k+\ell C_k\right]\geq -1+\textstyle\sum_{k}\left[\ell B_k+nA_{k}+1\right]+\textstyle\sum_{k}\left[2^{\ell A_k}(nB_k-\ell B_k+\ell C_k)\right]\]
%
%\[4+\log_2(N)+\frac{1}{N}\textstyle \sum_k \left[\ell A_k+\ell B_k+\ell C_k\right]\geq \textstyle\sum_{k}\left[\ell B_k+nA_{k}+1\right]+\textstyle\sum_{k}\left[2^{\ell A_k}(nB_k-\ell B_k+\ell C_k)\right]\]
%
%%\[4+\log_2(N)+\frac{1}{N}\textstyle \sum_k \left[\ell A_k+\ell C_k\right]\geq \textstyle\sum_{k}\left[\frac{N-1}{N}\ell B_k+nA_{k}+1+2^{\ell A_k}(nB_k-\ell B_k+\ell C_k)\right]\]
%
%\[4+\log_2(N)-N\geq \textstyle\sum_{k}\left[\frac{N-1}{N}\ell B_k+(nA_{k}-\frac{1}{N}\ell A_k)+2^{\ell A_k}(nB_k-\ell B_k)+(2^{\ell A_k}-\frac{1}{N})\ell C_k\right].\]
%We have written the right hand side as a sum of $4k$ non-negative terms, so this is quite a strong inequality.
%\begin{alignat*}{2}
%3
%&=
%\textstyle\ell B_1+(nA_{1}-\ell A_1)+2^{\ell A_1}(nB_1-\ell B_1)+(2^{\ell A_1}-1)\ell C_1%
%&\qquad&\text{($N=1$)}\\
%3
%&=
%\textstyle\sum_{k}\left[\frac{1}{2}\ell B_k+(nA_{k}-\frac{1}{2}\ell A_k)+2^{\ell A_k}(nB_k-\ell B_k)+(2^{\ell A_k}-\frac{1}{2})\ell C_k\right]%
%&\qquad&\text{($N=2$)}\\
%8/3
%&>
%\textstyle\sum_{k}\left[\frac{2}{3}\ell B_k+(nA_{k}-\frac{1}{3}\ell A_k)+2^{\ell A_k}(nB_k-\ell B_k)+(2^{\ell A_k}-\frac{1}{3})\ell C_k\right]%
%&\qquad&\text{($N=3$)}\\
%2
%&\geq
%\textstyle\sum_{k}\left[\frac{3}{4}\ell B_k+(nA_{k}-\frac{1}{4}\ell A_k)+2^{\ell A_k}(nB_k-\ell B_k)+(2^{\ell A_k}-\frac{1}{4})\ell C_k\right]%
%&\qquad&\text{($N\geq 4$)}
%\end{alignat*}
%Now we just go about checking that these inequalities show that no non-zero differential $d_r(y)$ can exist for $r\geq2$. Indeed, the requirement that $r\geq2$ reduces to the requirement
%\[\sum(\ell B_k+nA_k)\geq4-N\]
%which forces
%\[\log_2(N)\geq \textstyle\sum_{k}\left[-\frac{1}{N}(\ell A_k)+[2^{\ell A_k}(nB_k-\ell B_k)-\frac{1}{N}\ell B_k]+(2^{\ell A_k}-\frac{1}{N})\ell C_k\right].\]
%
%N=1: $\ell B_k=0$, done.
%
%N=2: $\ell C_1=\ell C_2=1$
%
%
%When $N=1$, demanding that $r\geq2$ implies that $\ell B_1$
%
%\[4\geq \textstyle\sum_{k}\left[\frac{N-1}{N}\ell B_k+(nA_{k}-\frac{1}{N}\ell A_k)+\frac{2}{5}+2^{\ell A_k}(nB_k-\ell B_k)+(2^{\ell A_k}-\frac{1}{N})\ell C_k\right]\]
%
%\[3= \textstyle\sum_{k}\left[\frac{1}{2}\ell B_k+(nA_{k}-\frac{1}{2}\ell A_k)+2^{\ell A_k}(nB_k-\ell B_k)+(2^{\ell A_k}-\frac{1}{2})\ell C_k\right] (\textup{N=2})\]
%
%\[\frac{8}{3}>1+\log_2(3)\geq \textstyle\sum_{k}\left[\frac{2}{3}\ell B_k+(nA_{k}-\frac{1}{3}\ell A_k)+2^{\ell A_k}(nB_k-\ell B_k)+(2^{\ell A_k}-\frac{1}{3})\ell C_k\right] (\textup{N=3})\]
%
%
%\[2\geq4-\frac{N}{2}\geq \textstyle\sum_{k}\left[\frac{3}{4}\ell B_k+(nA_{k}-\frac{1}{4}\ell A_k)+2^{\ell A_k}(nB_k-\ell B_k)+(2^{\ell A_k}-\frac{1}{4})\ell C_k\right] (\textup{$N\geq4$})\]
%\textup{}
%
%%\[4\geq\textstyle\sum_k\left[(nA_k-\ell A_k)+1+(2^{\ell A_k}-1)\ell C_k+2^{\ell A_k}(nB_k-\ell B_k)\right],\]
%where each quantity $(nA_k-\ell A_k)$, $1$, $(2^{\ell A_k}-1)\ell C_k$ and $2^{\ell A_k}(nB_k-\ell B_k)$ is nonnegative.
%One sees readily that if any $\ell A_k$ is nonzero, the corresponding summand would exceed 4, as in this case, $B_k$ must be non-empty, so that $nB_k-\ell B_k\geq2$. Thus we may take all $A_k=0$, and the inequality becomes
%\[4\geq\textstyle\sum_k\left[1+(nB_k-\ell B_k)\right].\]
%Now if $\ell B_k\geq1$ then  $(nB_k-\ell B_k)\geq2$ and if $\ell B_k\geq2$ then  $(nB_k-\ell B_k)\geq7$. Thus there can be at most one term with $\ell B_k\neq0$, and for any such term, $\ell B_k=1$. In the case $\ell B_1=1$, we can have at most one other term, and the target group is either $(E_2)_*^{1,*,*}$ or $(E_2)_*^{2,*,*}$ --- neither of these groups can be the target of a $d_r$ for $r\geq2$. This leaves us investigating the case in which all $A_k$ and $B_k$ are empty, where this inequality becomes $N\leq 4$, so that  $r=N-2$ must equal 2. That is, in this case we are demanding that the gradings in
%\[(E_2)^{3,\sum \ell C_k,(\sum2^{\ell C_k})}_{-1+\sum(d+nC_k+\ell C_k+1)}\textup{ and }(E_2)^{3,1+\ell I+j,(2^{1+\ell I+j})}_{2^{j+1}(d+nI+\ell I+1)-1}\]
%coincide
\end{Calculations of HW0}

%\begin{The cohomology of a trivial unstable Lie algebra over P}
%\vfil\pagebreak
%\section{\textbf{The cohomology of a trivial unstable Lie algebra over $P$}}
%There may be a wrinkle here answering the well definedness vs vanishing question --- maybe I have a mistake, and some operation is only well defined on $E^1$.
%\todo{State interest in this calculation}{write $X=\bigoplus_\alpha \F_2[r_\alpha]$, a finite direct sum}
%\todo{Construct the relevant elements $\Sq^J\delta_I\imath_\alpha$ and state a theorem on $H^*X$}{$\imath_\alpha\in H^0_{r_\alpha}X$ is the functional which projects $QX\cong X$ onto $\F_2[r_\alpha]$.
%\item Clearly state the contraints relevant for $K$ and $I$.
%\item Also describe which products thereof are not known to be zero, and state a theorem.}
%\todo{Proposition: any of the $\Sq^J\delta_I\imath_\alpha$ is detected by $\Sqv^{J_n}e\cdots e\Sqv^{J_1}e\delta_I\imath_\alpha$}{again, clearly state the constraints on the $J_i$.
%\item Push out to $ee\Sqv^{J_n}e\cdots e\Sqv^{J_1}e\delta_I\imath_\alpha\in E^2(n+2)_*^{00**\cdots *}$}
%\todo{Calculate the groups $E^2(n)_*^{*0**\cdots *}$}{Observe importance of these groups, since everything's detected in one}
%\todo{Identify products of $ee\Sqv^{J_{n-2}}e\cdots e\Sq^{J_1}e\delta_I\imath_\alpha\in E^2(n)_*^{00**\cdots *}$ in $E^2(n)_*^{*0**\cdots *}$}{I still find this hard to write down}
%\todo{Prove the theorem on $H^*X$}{start with anything in $E^2(n)$
%\item it's detected by something in $E^2(N)_*^{*0**\cdots *}$
%\item anything there is a product of $ee\Sqv^{J_{N-2}}e\cdots e\Sq^{J_1}e\delta_I\imath_\alpha$
%\item this converges down to $\Sq^K_h \Sqv^{J_n}e\cdots e\Sqv^{J_1}e\delta_I\imath_\alpha$, which must equal the original element
%\item this is a permanent cycle, proving that there are never any differentials in any of the cfsseqs
%\item thus $E^2(0)^*_*$ is a direct sum of certain of the $E^2(n)^{*0**\cdots *}_*$ via collapsing sequences
%\item stocktake reveals that these terms are everything we hoped}
%\end{The cohomology of a trivial unstable Lie algebra over P}
\appendix
\begin{appendices}
\vfil\pagebreak

\section{\textbf{A May spectral sequence for $H^*_{\calW(0)}$}}
\todo{Explain the quadratic filtration of the bar construction}{derive a spectral sequence
\item figure out what it implies about the whole thing
}

\section{\textbf{Cohomology operations for Lie algebras}}\label{appendix on Lie coh ops}
\subsection{The partially restricted universal enveloping algebra}
\todo{Recall Priddy's definition of Lie algebra cohomology, and results}{Explain that they don't calculate everything for us already, since the actual simplicial Lie algebras we're looking at are not GEMs --- their homotopy supports nontrivial operations.
\item maybe compare $\overline{W}$ and the bar construction
\item maybe not, but if you don't, point out that the operations you get come from the diagonal of the bar construction below which is a genuine simplicial coalgebra}
In this section we will prove [????????], using an improved version of Priddy's [Proof operations coincide]. We'll need one last category of graded vector spaces, $\vect{-}{n}$, an object of which is simply the direct sum of an object $V$ of $\vect{+}{n}$ and a vector space $V_{0,\ldots,0}^{-1}$:
\[V=V^{-1}_{0,\ldots,0}\oplus\bigoplus_{t\geq{1}}\bigoplus_{s_n,\ldots,s_1\geq0}V^t_{s_n,\ldots,s_1}\in\vect{-}{n}.\]
Denote by $\calA(n)$ the following category of graded augmented associative algebras. An object of $\calA(n)$ is a graded vector space
%\[A=A^{-1}_{0,\ldots,0}\oplus\bigoplus_{t\geq{1}}\bigoplus_{s_n,\ldots,s_1\geq0}A^t_{s_n,\ldots,s_1},\]
$A\in \vect{-}{n}$ such that $A^{-1}_{0,\ldots,0}=\F_2\langle 1\rangle$ is one-dimensional, spanned by the unit of an associative unital pairing
\[A^{t}_{s_n,\ldots,s_1}\otimes A^{q}_{p_n,\ldots,p_1}\to A^{t+q+1}_{s_n+p_n,\ldots,s_1+p_1}.\]
That is, $A_{0,\ldots,0}^{-1}$ is not part of the data of $A$, but only a graded piece added to hold the unit. Such an algebra is certainly augmented, and the augmentation ideal may be viewed as a forgetful functor $I:\calA(n)\to\calL(n)$, which sends $A$ to the partially restricted Lie algebra
\[\bigoplus_{t\geq{1}}\bigoplus_{s_n,\ldots,s_1\geq0}A^t_{s_n,\ldots,s_1},\]
with bracket $[x,y]:=xy-yx$, and restriction operation $\restn{x}:=x^2$ whenever $x\in A^t_{s_n,\ldots,s_1}$ and not all of $s_n,\ldots,s_1$ zero.

The composite forgetful functor $\calA(n)\overset{I}{\to}\calL(n)\to\vect{+}{n}$ has a left adjoint, none other than the tensor algebra functor $T$ (with unit placed in $A^{-1}_{0,\ldots,0}$ as appropriate). Moreover, the functor $I$ has a left adjoint, $\UEA$, \emph{the partially restricted universal enveloping algebra} functor, with $\UEA L$ obtained as the quotient of $T(L)$ by the two-sided ideal generated by any $[x,y]-xy-yx$ and by $\restn{x}-x^2$ with $x$ of restrictable degree. Indeed, there is a composite of adjunctions
\[\xymatrix@R=.3cm@C=1cm{
\vect{+}{n}  \ar@<.6ex>[r]^{F_{\calL(n)}}&
\calL(n)  \ar@<.4ex>[l]^{\textup{forget}} \ar@<.6ex>[r]^{\UEA}&
\calA(n),  \ar@<.4ex>[l]^{I} 
}
\]
showing that $\UEA\circ F^{\calL(n)}\cong T$. As in the non-restricted and fully restricted case, $\UEA L$ is naturally a Hopf algebra, having diagonal defined by the requirement $\Delta x=1\otimes x+x\otimes 1$ for $x\in L\subset \UEA L$.
\begin{lem}[Partially restricted PBW theorem]\label{Partially restricted PBW theorem}
If $L\in\calL(n)$, then there is a natural increasing filtration of $\UEA L$, the Lie filtration (by powers of $\langle 1\rangle\oplus \im(L\to \UEA(L))$), and the associated graded algebra is naturally isomorphic to $\F_2[L_{\textup{\textbf{0}}}]\otimes E[L_{\neq\textup{\textbf{0}}}]$, where $L=L_{\textbf{\textup{0}}}\oplus L_{\neq\textbf{\textup{0}}}$ is the decomposition of $L$ into the sum of its subspaces of in non-restrictable and restrictable degrees respectively.
\end{lem}
\begin{lem}
The prolonged functor $\UEA:s\calL(n)\to s\calA(n)$ preserves weak equivalences.
\end{lem}
\begin{proof}
Suppose that $L\to L'$ is a weak equivalence in $s\calL(n)$. The Lie filtration makes $C_*(\UEA L)\to C_*(\UEA L')$ a map of filtered commutative differential graded algebras, so there is an induced map of the resulting spectral sequences. By \ref{Partially restricted PBW theorem}, the $E^0$ page of the spectral sequence for $\UEA L$ is the differential graded algebra $C_*(\F_2[L_{\textbf{0}}]\otimes E[L_{\neq\textbf{0}}])$, where $\F_2[L_{\textbf{0}}]$ denotes a polynomial algebra, while $E[L_{\neq\textbf{0}}]$ denotes an exterior algebra. By Dold's theorem [], the $E^1$ page is a functor of $\pi_*(L_{\textbf{0}})$ and $\pi_*(L_{\neq\textbf{0}})$. As the induced maps $\pi_*(L_{\textbf{0}})\to\pi_*(L'_{\textbf{0}})$ and $\pi_*(L_{\neq\textbf{0}})\to\pi_*(L'_{\neq\textbf{0}})$ are isomorphisms, the map of spectral sequences is an isomorphism from $E^1$.
\end{proof}

\subsection{The proof of [????????]}

Let $ L_\bullet\in s\calL(n)$ be almost free on a fixed choice of subspaces $V_p\subseteq  L_p$.
We're interested in using the bisimplicial vector space:
\[\textbf{B}_{pq}:=\overline{B}_q \UEA L_p=(\UEA L_p)^{\otimes q}\in ss\vect{-}{n},\]
which in each simplicial level $p$ is the standard simplicial bar construction [Priddy early chapter] calculating $\Tor_{\UEA L_p}(\F_2,\F_2)$.
This is a bisimplicial cocommutative coalgebra, with diagonal:
\[\psi_{\textup{\textbf{B}}}:\left(\overline{B}_\bullet (\UEA L_\bullet)\overset{\overline{B}(\Delta)}{\to}\overline{B}_\bullet (\UEA L_\bullet\otimes \UEA L_\bullet)\cong\overline{B}_\bullet (\UEA L_\bullet)\otimes \overline{B}_\bullet (\UEA L_\bullet)\right).\]
We're also going to use the (somewhat artificial) simplicial chain complex $\textbf{Q}\in s\,\complexes\vect{-}{n}$
\[\textbf{Q}_{\bullet *}:=\begin{cases}
Q^{\calL(n)} L_\bullet,&\textup{if }*=1;\\
\F_2\langle 1\rangle,&\textup{if }*=0;
\\0,&\textup{otherwise.}
\end{cases}
\]
with zero differentials in each simplicial level. Now there's a map of simplicial chain complexes $r:N_*^\textup{v}\textbf{B}_\bullet\to \textbf{Q}_{\bullet*}$ \textbf{(what kind of normalization is convenient?)}, defined in level $p$ by the identification $N^\textup{v}_0\textbf{B}_p=\F_2\langle 1\rangle=\textbf{Q}_{p0}$ and the composite:
\[N^\textup{v}_1\textbf{B}_{p\bullet}=I\UEA L_p\epi I\UEA L_p/(I\UEA L_p)^2\cong Q^{\calL(n)} L_p.\]
In order to prove [????????], we plan to show
\begin{prop}\label{the point of the appendix} The composite
\[N_*\Delta\textbf{\textup{B}}\simeq\textup{Tot}(N^\textup{h}_*N^\textup{v}_*\textbf{\textup{B}})\overset{r}{\to} \textup{Tot}(N^\textup{h}_*\textbf{\textup{Q}}_{\bullet*})=\F_2\oplus \Sigma N_*Q^{\calL(n)} L\]
is a weak equivalence of chain complexes under which the operations $\Sq^k$ and $\mu$ on $\pi^*(\Delta\textup{\textbf{B}}^*)$, which arise from the simplicial coalgebra structure $\psi_{\textup{\textbf{B}}}$ of $\Delta \textup{\textbf{B}}$, correspond to the operations $\Sq^k:=\psi_{\calL(n)}\circ\ExtCohOp^{k-1}$ and $\mu:=\psi_{\calL(n)}\circ\ExtCohProd$ on $\pi^*((Q^{\calL(n)} L)^*)=:H^*_{\calL(n)} L$.
\end{prop}

As $ L$ is levelwise free, the evident map $T(V_p)\to \UEA L_p$ is an isomorphism for each $p$, and we define a vertical homotopy $h:N_*^\textup{v}\textbf{B}_{p\bullet}\to N_{*+1}^\textup{v}\textbf{B}_{p\bullet}$ by the following formulae (in which the $v_{i_{j}}$ are taken to be in $V_p\subseteq  L_p\subseteq\UEA L_p$):
\begin{alignat*}{2}
h_q:N^\textup{v}_q\overline{B}\UEA L_p&\overset{}{\longrightarrow} N^\textup{v}_{q+1}\overline{B}\UEA L_p\\
[\smash{\underbrace{v_{i_1}|\cdots |v_{i_{k-1}}}_{\textup{length 1 bars}}}|v_{i_{k}}v_{i_{k+1}}\cdots |\cdots ]
&\overset{}{\longmapsto}
[v_{i_1}|\cdots |v_{i_{k}}|v_{i_{k+1}}\cdots |\cdots ]%
\\
[v_{i_1}|\cdots |v_{i_{q}}]
&\overset{}{\longmapsto}0.
\end{alignat*}
This homotopy is of the same type as that used in \S[???????] and [Priddy], and commutes with all of the horizontal simplicial structure except $d^\textup{h}_0$, so that $d^\textup{h}h_q+h_qd^\textup{h} =d^\textup{h}_0h_q+h_qd^\textup{h}_0$.
\begin{lem}\label{barConstNullHtpyLemma}
Under the map $(\textup{Id}+h_{q-1}d^\textup{v}+d^\textup{v}h_{q}): N^\textup{v}_q\overline{B}\UEA L_p\to N^\textup{v}_q\overline{B}\UEA L_p$,
\begin{alignat*}{2}
[v_{i_1}\cdots |\cdots |\cdots v_{i_r}]
&\longmapsto0\text{ unless $r=q=1$, in which case}%
\\
[v_{i_1}]
&\longmapsto [v_{i_{1}}].
\end{alignat*}
\end{lem}
[\textbf{modernize this spiel:}] Now Singer \cite{SingerSteen1.pdf} gives a method of defining horizontal Steenrod operations on the spectral sequence of $\textbf{B}_{\bullet\bullet}$. His method is to define a \emph{bisimplicial Eilenberg-Zilber map} $K_k:C\left(X\otimes Y\right)\to CX\otimes CY$ for each $k$, a degree $k$ map satisfying $dK_k+K_kd=K_{k-1}+TK_{k-1}T$. His formulae are given in terms of a \emph{special} simplicial Eilenberg-Zilber map. On the dual total complex, the resulting cohomology operations satisfy the Adem operations.

Our method of proving [] will be to modify Singer's definition, using the chain homotopy $h$ to shift the horizontal operations one higher in filtration. They will still satisfy the same relations at the abutment, and as the abutment filtration is trivial, they must satisfy the same relations at $E^2$. Finally, we'll note that what we have produced at $E^2$ is the definition of the Steenrod operations from \S[??????].
\begin{lem}\label{firstCompositeLemma}
The composite
\[N_{p}^\textup{h}N_2^\textup{v}\textup{\textbf{B}}\overset{\psi_{\textup{\textbf{B}}}}{\to} N_{p}^\textup{h}N_2^\textup{v}(\textup{\textbf{B}}\otimes\textup{\textbf{B}}) \overset{D_0^\textup{v}}{\to} N_{p}^\textup{h}(N_1^\textup{v}\textup{\textbf{B}}\otimes N_1^\textup{v}\textup{\textbf{B}})\overset{r\otimes r}{\to}N^\textup{h}_p(\textup{\textbf{Q}}_{\bullet 1}\otimes \textup{\textbf{Q}}_{\bullet 1})\]
vanishes except on terms $[x|y]$ where $x$ and $y$ are generators of $ L_p$, which have image $x\otimes y$.
\end{lem}
\begin{proof}
A generic element of the domain is a sum of terms $[x_1\cdots x_I|y_1\cdots y_J]$, with $x_1,\ldots,x_I$ and $y_1,\ldots,y_J$ in $V_p\subseteq L_p$. This element maps under $\psi_{\textup{\textbf{B}}}$ to
%\[\sum_{a_1=0}^1\cdots \sum_{a_I=0}^1
%\sum_{b_1=0}^1\cdots \sum_{b_J=0}^1
%\left[\prod_{i=1}^Ix_i^{a_i}\middle|\prod_{\smash{j=1}}^Jy_j^{b_j}\right]\otimes
%\left[\prod_{i=1}^Ix_i^{1-a_i}\middle|\prod_{\smash{j=1}}^Jy_j^{1-b_j}\right]\]
\[\sum_{\smash{a_1,\ldots,a_I,b_1,\ldots,b_J\in\{0,1\}}}
\left[\prod_{i=1}^Ix_i^{a_i}\middle|\prod_{\smash{j=1}}^Jy_j^{b_j}\right]\otimes
\left[\prod_{i=1}^Ix_i^{1-a_i}\middle|\prod_{\smash{j=1}}^Jy_j^{1-b_j}\right]
\in N_{p}^\textup{h}N_2^\textup{v}(\textup{\textbf{B}}\otimes\textup{\textbf{B}}),
\]
and $D_0^\textup{v}$ annihilates all terms except for those in which all $a_i$ are $1$ and all $b_j$ are $0$, leaving
\[\left[\prod_{i=1}^Ix_i\right]\otimes
\left[\prod_{\smash{j=1}}^Jy_j\right]\in N_{p}^\textup{h}(N_1^\textup{v}\textup{\textbf{B}}\otimes N_1^\textup{v}\textup{\textbf{B}}).\]
Finally, $r\otimes r$ annihilates this term unless $I=J=1$.
\end{proof}
\begin{lem}\label{secondCompositeLemma}
The composite
\[N_{p+1}^\textup{h} N_1^\textup{v}\textup{\textbf{B}}\overset{\psi_{\textup{\textbf{B}}}}{\to} N_{p+1}^\textup{h}N_1^\textup{v}(\textup{\textbf{B}}\otimes\textup{\textbf{B}}) \overset{D_1^\textup{v}}{\to} N_{p+1}^\textup{h}(N_1^\textup{v}\textup{\textbf{B}}\otimes N_1^\textup{v}\textup{\textbf{B}})\overset{r\otimes r}{\to}N^\textup{h}_{p+1}(\textup{\textbf{Q}}_{\bullet 1}\otimes \textup{\textbf{Q}}_{\bullet 1})\]
vanishes except on terms $[xy]$ where $x$ and $y$ are generators of $ L_{p+1}$, which have image $x\otimes y+y\otimes x$.
\end{lem}
\begin{proof}
A generic element of the domain is a sum of terms $[x_1\cdots x_I]$, with $x_1,\ldots,x_I$ in $V_{p+1}\subseteq L_{p+1}$. This element maps under $\psi_{\textup{\textbf{B}}}$ to
\[\sum_{\smash{a_1,\ldots,a_I\in\{0,1\}}}
\left[\prod_{i=1}^Ix_i^{a_i}\right]\otimes
\left[\prod_{i=1}^Ix_i^{1-a_i}\right].\]
As $\{D_k\}$ was chosen to be a special $k$-cup product, $D_1^\textup{v}$ acts as the identity.
Finally, $r\otimes r$ annihilates this term unless $I=2$ and $a_1\neq a_2$.
\end{proof}
\begin{lem}\label{commuting rectangle lemma for lie operations}
There is a commuting diagram:
\[\xymatrix@R=4mm@C=18mm{
N_{n+k}^\textup{h}N_1^\textup{v}\textup{\textbf{B}} \ar[r]^-{d^\textup{h}_0h_{n+k}+h_{n+k-1}d^\textup{h}_0}
\ar[d]^-{r}&%r1c1
N_{n+k-1}^\textup{h}N_2^\textup{v}\textup{\textbf{B}} \ar[r]^-{D^\textup{v}_0\circ\psi_{\textup{\textbf{B}}}}&%r1c2
N_{n+k-1}^\textup{h}(N_1^\textup{v}\textup{\textbf{B}}\otimes N_1^\textup{v}\textup{\textbf{B}})\ar[d]^-{r\otimes r}
\\%r1c3
N_{n+k}^\textup{h}Q^{\calL(n)} L  \ar[r]^-{Q^{\calL(n)}(\xi_{\calL(n)})}&%r1c1
N_{n+k-1}^\textup{h}Q^{\calL(n)}( L\wedge  L) \ar[r]^-{j_{\calL(n)}}&%r1c2
N_{n+k-1}^\textup{h}(Q^{\calL(n)} L \otimes Q^{\calL(n)} L )
}\]
\end{lem}
\begin{proof}
Write
$\textup{LHS}=(r\otimes r)\circ D^\textup{v}_0\circ\psi_{\textup{\textbf{B}}}\circ(d_0^\textup{h}h+hd^\textup{h}_0)$ and $\textup{RHS}= \psi_{\calL(n)}\circ r$.
Consider first an element $e=[v_1v_2\cdots v_b]$ of $N_{n+k}^\textup{h}N_1^\textup{v}\textbf{B}$ with $b\geq2$. By definition, $r$ vanishes on such an element, so that $\textup{RHS}(e)=0$. Lemma \ref{firstCompositeLemma} states that the map $(r\otimes r)\circ D^\textup{v}_0\circ\psi_{\textup{\textbf{B}}}$ vanishes except on expressions of the form $[u|w]$ for $u,w\in V_{n+k-1}$. However, the expressions of this form appearing in $d^\textup{h}_0h(e)$ coincide with such expressions in $hd^\textup{h}_0(e)$, so that there is a cancellation, and $\textup{LHS}(e)=0$ as hoped. [\textbf{maybe} explain further.]

Next, consider an element $[v]$ of $N_{n+k}^\textup{h}N_1^\textup{v}\textbf{B}$. As $h[v]=0$, and in light of lemma \ref{firstCompositeLemma}, $\textup{LHS}([v])$ equals the quadratic part of $d_0^\textup{h}v$, after writing $d_0^\textup{h}v$ as an expression in elements of $V_{n+k-1}$. This is exactly the description given in lemma \ref{psi is basically the quadratic part} of $\textup{RHS}([v])=\psi_{\calL(n)}(v)$.
%\[\textup{RHS}([v])=(j_{\calL(n)}\circ Q^{\calL(n)}(\xi_{\calL(n)}))(v)=\psi_{\calL(n)}(v).\qedhere\]
\end{proof}
\begin{proof}[Proof of \ref{the point of the appendix}]
%Having put in place these preparations, let's start working on the theorem. 
Fix a cocycle $\alpha\in (N_{n}Q^{\calL(n)} L_\bullet)^*$. %We may assume without loss of generality (or are we \textbf{forced} to assume) that  $\alpha$ evaluates to zero... wait... wtf am I talking about?
Then $\alpha$ may be viewed an a permanent cocycle at $E_2^{n,1}$ in the spectral sequence of $(\textbf{Q}_{\bullet*})^*$. We'll apply the chain-level Steenrod operation $S^k$ of [Singer ???], (which defines the Steenrod operation $\Sq^k$ on the cohomology of the total complex), to the class $r^*\alpha$. As $\alpha$ is a permanent cycle, $\delta(r^*\alpha)=0$, and
%\[S^k(r^*\alpha):=\psi^*K^*_{n+1-k}\oldphi(r^*\alpha\otimes r^*\alpha)=T_1+T_2,\textup{ where}\]
\begin{alignat*}{2}
S^k(r^*\alpha):={}&\psi_{\textup{\textbf{B}}}^*K^*_{n+1-k}\oldphi(r^*\alpha\otimes r^*\alpha)=T_1+T_2,\textup{ where}
\\
T_1
={}&
\psi_{\textup{\textbf{B}}}^*(D^\textup{v}_0)^*(D^\textup{h}_{n+1-k})^*
\oldphi(r^*\alpha\otimes r^*\alpha)\in (N^\textup{h}_{n+k-1}N^\textup{v}_2\textbf{B})^*%
\\
T_2
={}&
\psi_{\textup{\textbf{B}}}^*(D^\textup{v}_1)^*(TD^\textup{h}_{n-k}T)^*
\oldphi(r^*\alpha\otimes r^*\alpha)\in (N^\textup{h}_{n+k}N^\textup{v}_1\textbf{B})^*
\end{alignat*}
%
% $S^k(r^*\alpha)=\psi^*K^*_{n+1-k}\oldphi(r^*\alpha\otimes r^*\alpha)$, which has two terms:
%\[\underbrace{\psi^*(D^\textup{v}_0)^*(D^\textup{h}_{n+1-k})^*\oldphi(r^*\alpha\otimes r^*\alpha)}_{\textup{$T_1$, filtration $n+k-1$}}+
%\underbrace{\psi^*(D^\textup{v}_1)^*(TD^\textup{h}_{n-k}T)^*
%\oldphi(r^*\alpha\otimes r^*\alpha)}_{\textup{$T_2$, filtration $n+k$}}\]
Our method will be to compress each of these terms into filtration one higher, using the cochain homotopy $h^*:(N^\textup{h}_* N^\textup{v}_*\textbf{B})^*\to (N^\textup{h}_*N^\textup{v}_{*-1}\textbf{B})^*$
Using lemma \ref{barConstNullHtpyLemma}:
\[(\textup{Id}+\delta^\textup{v}h^*+ h^*\delta^\textup{v})T_1=0\textup{ and }(\textup{Id}+\delta^\textup{v}h^*+ h^*\delta^\textup{v})T_2=0.\]
The first equation holds as $(\textup{Id}+hd^\textup{v}+d^\textup{v}h)$ is zero on $N^\textup{v}_2\textbf{B}$. For the second equation, on $N^\textup{v}_1\textbf{B}$,  $(\textup{Id}+hd^\textup{v}+d^\textup{v}h)$ is the projection onto terms of the form $[v]$, yet lemma \ref{secondCompositeLemma} shows that the composite
\[((r\otimes r)\circ(TD^\textup{h}_{n-k}T)\circ D^\textup{v}_1\circ\psi_{\textup{\textbf{B}}}): N^\textup{h}_{n+k}N^\textup{v}_1\textbf{B}\to N^\textup{v}_{n+k}(\textbf{Q}_{\bullet1}\otimes \textbf{Q}_{\bullet1})\]
vanishes except on terms of the form $[vw]$ (recall that $r$ commutes with the horizontal simplicial structure).

As $\delta^\textup{h}h^*+ h^*\delta^\textup{h}$ increases filtration, we have compressed $S^k(r^*\alpha)$ to the filtration $n+k$ expression $(\delta^\textup{h}h^*+ h^*\delta^\textup{h})T_1$, modulo even higher filtration.
%up to even higher filtration ($n+k+1$), we have compressed $S^k(r^*\alpha)$ to the expression $(\delta^\textup{v}h^*+ h^*\delta^\textup{v})\textup{Term}_1$.
Now the commuting diagram of lemma \ref{commuting rectangle lemma for lie operations} is the left square in a larger commuting diagram:
\[\xymatrix@R=4mm@C=15mm{
N_{n+k}^\textup{h}N_1^\textup{v}\textbf{B} \ar[d]^-{r}
 \ar[r]^-{D^\textup{v}_0\circ\psi_{\textup{\textbf{B}}}\circ (d^\textup{h}h+hd^\textup{h})}&%r1c2
N_{n+k-1}^\textup{h}(N_1^\textup{v}\textbf{B}\otimes N_1^\textup{v}\textbf{B})\ar[d]^-{r\otimes r}\ar[r]^-{ D^\textup{h}_{n+1-k}}
&
N^\textup{h}_nN^\textup{v}_1\textbf{B}\otimes N^\textup{h}_nN^\textup{v}_1\textbf{B}\ar[d]^-{r\otimes r}
\\%r1c3
N_{n+k}^\textup{h}Q^{\calL(n)} L  \ar[r]^-{\psi_{\calL(n)}}&%r1c2
N_{n+k-1}^\textup{h}(Q^{\calL(n)} L \otimes Q^{\calL(n)} L )
\ar[r]^-{ D^\textup{h}_{n+1-k}}
&
N^\textup{h}_nQ^{\calL(n)} L \otimes N^\textup{h}_nQ^{\calL(n)} L 
}\]
Now $(N^\textup{h}_nQ^{\calL(n)} L \otimes N^\textup{h}_nQ^{\calL(n)} L )^*$ contains the cocycle $\oldphi(\alpha\otimes\alpha)$, and pulling $\oldphi(\alpha\otimes\alpha)$ back to $(N_{n+k}^\textup{h}N_1^\textup{v}\textbf{B})^*$ along the lower composite yields $r^*\psi_{\calL(n)}^*\ExtCohOp^{k-1}(\alpha)$. Pulling back along the upper composite yields the $E^2$ representative of the shifted version of Singer's operations. As $r^*$ is an $E^2$-equivalence, and the filtrations are trivial, this proves the result. A simple modification proves the result for pairings.
\end{proof}

\subsection{The Chevalley-Eilenberg-May complex}\label{The Chevalley-Eilenberg-May complex}
Suppose that $L\in\calL(n)$ is locally finite dimensional. We define a differential coalgebra, the \emph{Chevalley-Eilenberg-May complex}, to be the subcoalgebra $\UEAX(L):= E(L_{\textbf{0}})\otimes \Gamma(L_{\neq\textbf{0}})$ of the divided power coalgebra $\Gamma(L)$ with its usual coalgebra structure. This is the homology (in the sense of \cite{PriddyKoszul.pdf}) of the associated graded algebra appearing in the partially restricted PBW theorem, lemma \ref{Partially restricted PBW theorem}. On \cite[p.\ 141]{MayRestLie.pdf}, the typical element of the coalgebra $\Gamma(L)$ is written as 
\[f=\gamma_{r_1}(y_1)\cdots \gamma_{r_m}(y_m),\]
for homogeneous $y_i\in L$. We define $\UEAX(L)$ to be the subcoalgebra given by the restriction that $r_i\leq1$ when $y_m\in L_{\textbf{0}}$. The coalgebra structure map and differential are the restriction to $E(L_{\textbf{0}})\otimes \Gamma(L_{\neq\textbf{0}})$ of those given on \cite[p.\ 141]{MayRestLie.pdf} (after tensoring the formula  \cite[(6.19)]{MayRestLie.pdf} down to a formula on $\overline{X}(L)$, which kills the first term $\Sigma_{i=1}^nf_iy_i$).

Now let $ L=B^{\calL(n)}L\in s\calL(n)$. Using the weak equivalence $\Delta\textbf{B}\to \overline{B}_\bullet(\UEA L)$ of simplicial coalgebras, and  May's injection \cite[Theorem 18 and  (7.8)]{MayRestLie.pdf} of $\UEAX(L)$ into the bar construction, there are maps (up to a zigzag):
\[\UEAX(L)\to N_*\overline{B}_\bullet(U'L)\simeq\textup{Tot}(N^\textup{h}_*N^\textup{v}_*\textbf{B})\overset{r}{\to} \textup{Tot}(N^\textup{h}_*\textbf{Q}_{\bullet*})=\F_2\langle 1\rangle\oplus \Sigma N_*Q^{\calL(n)} L.\]
Now the first map is a map of differential coalgebras, so that proposition \ref{the point of the appendix} implies that $\F_2\langle 1\rangle\oplus \Sigma H^*_{\calL(n)}L$ may be calculated, as an \emph{algebra}, by the differential graded algebra $(\UEAX(L))^*$, naturally defined as the quotient $(\UEAX(L))^*=E(L^*_{\textbf{0}}) \otimes\F_2[L_{\neq\textbf{0}}^*]$ of $\F_2[L^*]$.

We're interested in the case that $L$ is \emph{trivial as a Lie algebra}, but may still have nonzero restriction. In this case, the restriction is in fact a \emph{linear} map, and we may write $\dualrestn{\DASH}:L^*\to L^*$ for its dual (a map which we consider to be everywhere defined, but necessarily equal to zero on $L_{\textbf{0}}$). Examination of  \cite[(6.19)]{MayRestLie.pdf} shows that:
\begin{prop}
If $L$ has bracket zero, then the differential on $(\UEAX(L))^*$ is defined on $\alpha\in L^*\subseteq (\UEAX(L))^*$ by the formula
\[\alpha\longmapsto (\!\sqrt[{[2]}]{\alpha}\,)^2.\]

\end{prop}

\section{\textbf{Unstable Lie coalgebras over the dual $P$-algebra (move me)}}
\todo{Give a formal definition of these coalgebras}{goes down to level of simplicial commutative algebras, and brings in the SOA comonad structure}
\todo{Give the construction (via equalizer diagram, etc) of the comonad}{}
\todo{Write the maps $j$ and $\gamma_i$ in terms of maps out of $C_{ij}$ and $C_i^j$}{include generating cofibrations such as $S^{i+j}\to C_{ij}$ and $\Delta[1]\times S^{i+j}\to \Delta[1]\times C_{ij}$}
\todo{Construct operations on homology of such coalgebras}{they agree with those in double dual construction when locally finite
\item if not locally finite, by studying gradings, can see that every coalgebra is the union of its finite type (even just `finite') subcoalgebras, and that the cobar construction respects such unions, thus get the same results in infinite case as we did from double dualization.}


\end{appendices}

\begin{todolist}
\section{\textbf{To do/consider:}}
\begin{enumerate}\squishlist
\setlength{\parindent}{.25in}
\item introduce the category $\calU(n)$, uniformize notation for free constructions. Scrap Goerss' $U$ notation, instead write $F_{\calW}$.
\item which side do you index bar constructions from?
\item Do I insert a section of conventions, including: %\cancel{cugas} and \cancel{model structures}, free functors using standard notation 
$F_?$ thought of as monads without further notation, %indecomposable, categories of vector spaces, good and bad Lie algebras, 
higher Eilenberg-Zilber maps, %never substituting out stars for gradings and
having all gradings at the very right, tensor products of vector spaces,\ldots?
\item Comment that we care about derived indecomposables, not AQ homology.
\item choose a notation for the quadratic grading? like $\textup{qu}_2V^{*}_{***}$?
\item Monadicity over vector spaces is important and should be emphasised, especially for the diagonal on the Blanc-Stover resolution.
\item Maybe I should keep finiteness to myself, given that the SOA might be around, or even the W comonad.
\item change all $\calW$'s to $\calK$'s?
\item choose a consistent set of sub/superscript conventions for the categories you're using.
\item Change the notations $i$ and $\gamma$ for $\Pi$-algebra stuff.
\item Uniformize notation for the total complex of a double complex, and for the cohomotopy of the dual of a simplicial vector space.
\item In bit about quadratic decomposition, say ``I'll just prove well definedness in specific cases.''
\item Uniformize bracketing and other notation for quadratic decompositions.
\item Need a bit of theory about splittings of monads, homogeneity, and unital part of an expression.
\item delete all pagebreaks
\item At present the notation $\gamma$ is repeated all over the place.
\item make a note about the difference between AQ homology and derived indecomposables.
\item Either use or lose the $\quadratic$ notation --- in general, sort out a consistent use for it.
\item $\gamma_i$ and $\gamma^i$ notations must go.
\item Need a cogent section on stuff like: $S^2_\textup{pr?}\pi_*M\to \pi_*M$. What's a restricted Lie algebra, what's are the ``top operations'' $S^2\pi_*(L)\to \pi_*(S^2L)\to \pi_*(L)$ - are they defined by $\nabla$? (yes). Use $\widetilde{\nabla}$ for the top ops. Change them all to $\Delta$.
\item Uniform notation for the class represented by $x$... $[x]$? $\overline{x}$?
\item get an arrow for epimorphisms and one for monomorphisms, and use them
\item vector space vs vectorspace
\item upper and lower scripts for algebraic structure - sort it out! E.g. $Q^\calC$ and $H_*^\calC$.
\item don't confuse ``quadratic grading 2'' with ``quadratic operations''! The quadratic grading 2 part in a $\Pi$-algebra corresponds to the quadratic operations. So far I haven't really made that clear --- actually, this thought is the motivation for the name ``quadratic grading''! Should it be the `arity grading'?
\item Need discourse on that grading on $\calW(n)$.
\item Point out that $F_{\calU(n+1)}^{(2)}(V)\oplus F_{\calL(n+1)}^{(2)}(V)=F_{\calW(n+1)}^{(2)}(V)$, and that the two monoids both include as subs of $F_{\calW(n+1)}$, and one is a quotient monoid. Broadly just need to fully explain all this grading stuff in all of the free constructions.
\item Consider making every symbol $(E_2)^{s_5,\ldots,s_1}_t$ to $(E_2^{s_5,s_4})^{s_3,\ldots,s_1}_t$.
\item Remove all bullets from $B^{\DASH}.$
\item ``unnormalized'' and ``non(-)normalized''... also, introduce the notation $C_*$
\item $\longmapsto$ instead of $\mapsto$
\item Point out why $F_\calC$ is a section of $Q_{\calC}$ --- compose adjunctions.
\item maybe raise the subscripts in $F_\calC$ and in $U_\calC$ and in $K_\calC$ and in $Q_\calC$ (which is done)
\item $E_2$ should be $E_2X$ or $E_2(X)$, also, need to figure out how to number certain of the spectral sequences
\item say what locally finite means and point out that things with a shifted degree are all the colimit of their locally finite subobjects
\item ``$\delta$-algebra'' isn't a good name, $\delta$-admissible too!
\item General principle: homotopy operations ($\delta$, $\lambda$) are partially defined, cohomology operations are sometimes zero. The two notions interchange upon taking cohomology.
\item co-Koszul vs coKoszul.
\item ``goerss looked at unstable ops in the reverse adams SSeq, which is how we understand w. I do something for the forwards ass, and the story comes out rather differently.''
\item 
``Goerss' text explains why one might expect these dualities. (After giving the four koszul algebras in the first chapter). Put 2 algs in the example of commutative algs, and then 2 under lie algs.''
\item Install a $\textup{v}$ in each Koszul $\delta$ operation.
\item Am I writing $n>0$ or $n\geq1$? Choose one.
\item $\Sq$-admissible? Steenrod-Admissible? \textbf{it means ``not containing 0''}
\item search all ``ref''s and check they have the right word, like ``lemma'' before them.
\item search $F_?$, $B_?$, $Q_?$, $U_?$, $\xi_?$, $j_?$, $\psi_?$...?
\item get a better macro for $\epi$
\item Identify and remove certain common phrases, like `Now $x=y$, but $y\neq z$...'
\item Introduce uniform notation for connective chain complexes.
\item once and for all, say that Sq-admiss means $\geq1$ and $\delta,P$ means $\geq2$ and $\Lambda$ means $\geq0$
\item forecast the pr UEA in the section where we discuss lie algebras first.
\item Search lemma, theorem, proposition, etc, and fix it all.
\item mention %SuccessiveSpectralSequences_22draft.pdf and po hu
\item say `edge composite', or similar, not edge hom thm
\item search mapsto, replace with longmapsto, or just redefine
\item subset to be replaced by subseteq
\item say ``sometimes view element of $\vect{r}{+}$ as an element of $\vect{r+1}{+}$ concentrated in degree $s_{r+1}=0$'' and say ``we'll use the phrase \emph{degree $s_k=0$}, or \emph{degree $t=0$}''
\item define top and non-top
\item define``restrictable''
\item search ``non'', and have uniform convention on dashes
\item include references to the Chevalley-Eilenberg-May complex section of the appendix
\item explain the frequent use of the word `trivial'
\item mention that $\F_2\{\}$ just means `spanned by', while $\F_2[]$ means poly alg.
\item in koszul resolutions, often write $x$ instead of $\Sqv(\emptyset,x)$. Maybe want to write $\Sqv^*(\emptyset,x)$ instead, in general.
\item should probably say what the structure  on the tensor product of $E_2$-pages is.
\item most $\delta$ should be $\deltav$
\item $\smash{\widetilde{f}}$ is normally best smashed
\item  \textbf{remove any $[x]$'s}, replace with $\overline{x}$.
\item flesh out the looking glass by comparing unstable homotopy operations and cohomology operations using Koszul duality
\item point out: knowing $\PiAlg$ structure is knowing htpy of spheres
\item Bring in GoerssHiltonMilnor.pdf
\item probably want to make the big four algebras unital, so that they are easy to tensor with... 
\end{enumerate}
\end{todolist}
\begin{bibliog}
\printbibliography
\end{bibliog}

\end{document}


%Goerss:MR1089001
%Singer:MR2245560
%Priddy:PriddySimplicialLie.pdf or PriddyKoszul.pdf
%Curtis:CurtisSimplicialHtpy.pdf
%6Author.pdf
%MillerSullivanConjecture.pdf
%Blanc_Stover-Groth_SS.pdf
%\cite{FresseSimplicialAlgs.pdf}
%DwyerHtpyOpsSimpComAlg.pdf
% !TEX root = z_output/unstable_Lie_algs.tex
\documentclass[11pt]{amsart}
\usepackage{amsmath,amsthm,amssymb}
\usepackage{eucal}
\usepackage{mathrsfs,nicefrac}
\usepackage{amssymb}
\usepackage[all]{xy}
\usepackage{cancel}
\usepackage{color}
\usepackage{enumerate}
\usepackage{mathtools}%used for mathclap
\usepackage{dsfont}%creates \mathds{1}

\usepackage[bookmarks=false,pdftex,pdfborder={0 0 0 [1 1]}]{hyperref}

\headheight=8pt
\topmargin=0pt
\textheight=610pt
\textwidth=432pt
\oddsidemargin=18pt
\evensidemargin=18pt
\footskip=25pt


%>>>>>>>>>>>>>>>>>>>>>>>>>>>>>>
%<<<  Theorem Environments  <<<
%>>>>>>>>>>>>>>>>>>>>>>>>>>>>>>
\theoremstyle{plain}
\newtheorem{thm}{Theorem}[section]
\newtheorem*{thm*}{Theorem}
\newtheorem{lem}[thm]{Lemma}
\newtheorem*{lem*}{Lemma}
\newtheorem{prop}[thm]{Proposition}
\newtheorem*{prop*}{Proposition}
\newtheorem{cor}[thm]{Corollary}
\newtheorem*{cor*}{Corollary}
\newtheorem{defprop}[thm]{Definition-Proposition}
\newtheorem*{punchline}{Punchline}
\newtheorem{conjecture}{Conjecture}
\newtheorem*{conjecture*}{Conjecture}
\newtheorem*{claim}{Claim}

\theoremstyle{definition}
\newtheorem{defn}{Definition}[section]
\newtheorem*{defn*}{Definition}


%>>>>>>>>>>>>>>>>>>>>>>>>>>>>>>
%<<<       Operators        <<<
%>>>>>>>>>>>>>>>>>>>>>>>>>>>>>>
\DeclareMathOperator{\ad}{\textbf{ad}}
\DeclareMathOperator{\coker}{coker}
\renewcommand{\ker}{\textup{ker}\,}
\DeclareMathOperator{\End}{End}
\DeclareMathOperator{\Aut}{Aut}
\DeclareMathOperator{\Hom}{Hom}
\DeclareMathOperator{\Maps}{Maps}
\DeclareMathOperator{\Mor}{Mor}
\DeclareMathOperator{\Gal}{Gal}
\DeclareMathOperator{\Ext}{Ext}
\DeclareMathOperator{\Tor}{Tor}
\DeclareMathOperator{\Cotor}{Cotor}
\DeclareMathOperator{\Prim}{Prim}
\DeclareMathOperator{\Tot}{Tot}
\DeclareMathOperator{\Map}{Map}
\DeclareMathOperator{\Der}{Der}
\DeclareMathOperator{\Rad}{Rad}
\DeclareMathOperator{\rank}{rank}
\DeclareMathOperator{\ArfInvariant}{Arf}
\DeclareMathOperator{\KervaireInvariant}{Ker}
\DeclareMathOperator{\im}{im}
\DeclareMathOperator{\coim}{coim}
\DeclareMathOperator{\trace}{tr}
\DeclareMathOperator{\supp}{supp}
\DeclareMathOperator{\ann}{ann}
\DeclareMathOperator{\spec}{Spec}
\DeclareMathOperator{\SPEC}{\textbf{Spec}}
\DeclareMathOperator{\proj}{Proj}
\DeclareMathOperator{\PROJ}{\textbf{Proj}}
\DeclareMathOperator{\fiber}{fib}
\DeclareMathOperator{\cofiber}{cof}
\DeclareMathOperator{\cone}{cone}
\DeclareMathOperator{\skel}{sk}
\DeclareMathOperator{\coskel}{cosk}
\DeclareMathOperator{\conn}{conn}
\DeclareMathOperator*{\colim}{colim}
\DeclareMathOperator*{\limit}{lim}
\DeclareMathOperator*{\hocolim}{hocolim}
\DeclareMathOperator*{\holimit}{holim}
\DeclareMathOperator*{\holim}{holim}
\DeclareMathOperator*{\hofib}{hofib}
\DeclareMathOperator*{\hotfib}{thofib}
\DeclareMathOperator*{\equaliser}{eq}
\DeclareMathOperator*{\coequaliser}{coeq}
\DeclareMathOperator{\ch}{ch}
\DeclareMathOperator{\Thom}{Th}
\DeclareMathOperator{\GrthGrp}{GrthGp}
\DeclareMathOperator{\Sym}{Sym}
\DeclareMathOperator{\Prob}{\mathbb{P}}
\DeclareMathOperator{\Exp}{\mathbb{E}}
\DeclareMathOperator{\GeomMean}{\mathbb{G}}
\DeclareMathOperator{\Var}{Var}
\DeclareMathOperator{\Cov}{Cov}
\DeclareMathOperator{\Sp}{Sp}
\DeclareMathOperator{\Seq}{Seq}
\DeclareMathOperator{\Cyl}{Cyl}
\DeclareMathOperator{\Ev}{Ev}
\DeclareMathOperator{\sh}{sh}
\DeclareMathOperator{\intHom}{\underline{Hom}}
\DeclareMathOperator{\Frac}{frac}


%>>>>>>>>>>>>>>>>>>>>>>>>>>>>>>
%<<<  Mathematical Symbols  <<<
%>>>>>>>>>>>>>>>>>>>>>>>>>>>>>>
\newcommand{\DASH}{\textup{--}}

%>>>>>>>>>>>>>>>>>>>>>>>>>>>>>>
%<<<     Greek Letters      <<<
%>>>>>>>>>>>>>>>>>>>>>>>>>>>>>>
\let\oldphi\phi
\let\phi\varphi
\renewcommand{\to}{\longrightarrow}
\newcommand{\from}{\longleftarrow}
\newcommand{\eps}{\varepsilon}

%%>>>>>>>>>>>>>>>>>>>>>>>>>>>>>>
%%<<<     Environments       <<<
%%>>>>>>>>>>>>>>>>>>>>>>>>>>>>>>
\newcommand{\squishlist}{
  \setlength{\itemsep}{.5pt}
  \setlength{\parskip}{0pt}
  \setlength{\parsep}{0pt}}
%>>>>>>>>>>>>>>>>>>>>>>>>>>>>>>
%<<<     Script Letters     <<<
%>>>>>>>>>>>>>>>>>>>>>>>>>>>>>>
\newcommand{\scrQ}{\mathscr{Q}}
\newcommand{\scrW}{\mathscr{W}}
\newcommand{\scrE}{\mathscr{E}}
\newcommand{\scrR}{\mathscr{R}}
\newcommand{\scrT}{\mathscr{T}}
\newcommand{\scrY}{\mathscr{Y}}
\newcommand{\scrU}{\mathscr{U}}
\newcommand{\scrI}{\mathscr{I}}
\newcommand{\scrO}{\mathscr{O}}
\newcommand{\scrP}{\mathscr{P}}
\newcommand{\scrA}{\mathscr{A}}
\newcommand{\scrS}{\mathscr{S}}
\newcommand{\scrD}{\mathscr{D}}
\newcommand{\scrF}{\mathscr{F}}
\newcommand{\scrG}{\mathscr{G}}
\newcommand{\scrH}{\mathscr{H}}
\newcommand{\scrJ}{\mathscr{J}}
\newcommand{\scrK}{\mathscr{K}}
\newcommand{\scrL}{\mathscr{L}}
\newcommand{\scrZ}{\mathscr{Z}}
\newcommand{\scrX}{\mathscr{X}}
\newcommand{\scrC}{\mathscr{C}}
\newcommand{\scrV}{\mathscr{V}}
\newcommand{\scrB}{\mathscr{B}}
\newcommand{\scrN}{\mathscr{N}}
\newcommand{\scrM}{\mathscr{M}}


\newcommand{\frakq}{\mathfrak{q}}
\newcommand{\frakw}{\mathfrak{w}}
\newcommand{\frake}{\mathfrak{e}}
\newcommand{\frakr}{\mathfrak{r}}
\newcommand{\frakt}{\mathfrak{t}}
\newcommand{\fraky}{\mathfrak{y}}
\newcommand{\fraku}{\mathfrak{u}}
\newcommand{\fraki}{\mathfrak{i}}
\newcommand{\frako}{\mathfrak{o}}
\newcommand{\frakp}{\mathfrak{p}}
\newcommand{\fraka}{\mathfrak{a}}
\newcommand{\fraks}{\mathfrak{s}}
\newcommand{\frakd}{\mathfrak{d}}
\newcommand{\frakf}{\mathfrak{f}}
\newcommand{\frakg}{\mathfrak{g}}
\newcommand{\frakh}{\mathfrak{h}}
\newcommand{\frakj}{\mathfrak{j}}
\newcommand{\frakk}{\mathfrak{k}}
\newcommand{\frakl}{\mathfrak{l}}
\newcommand{\frakz}{\mathfrak{z}}
\newcommand{\frakx}{\mathfrak{x}}
\newcommand{\frakc}{\mathfrak{c}}
\newcommand{\frakv}{\mathfrak{v}}
\newcommand{\frakb}{\mathfrak{b}}
\newcommand{\frakn}{\mathfrak{n}}
\newcommand{\frakm}{\mathfrak{m}}

%>>>>>>>>>>>>>>>>>>>>>>>>>>>>>>
%<<<  Caligraphic Letters   <<<
%>>>>>>>>>>>>>>>>>>>>>>>>>>>>>>
\newcommand{\calQ}{\mathcal{Q}}
\newcommand{\calW}{\mathcal{W}}
\newcommand{\calE}{\mathcal{E}}
\newcommand{\calR}{\mathcal{R}}
\newcommand{\calT}{\mathcal{T}}
\newcommand{\calY}{\mathcal{Y}}
\newcommand{\calU}{\mathcal{U}}
\newcommand{\calI}{\mathcal{I}}
\newcommand{\calO}{\mathcal{O}}
\newcommand{\calP}{\mathcal{P}}
\newcommand{\calA}{\mathcal{A}}
\newcommand{\calS}{\mathcal{S}}
\newcommand{\calD}{\mathcal{D}}
\newcommand{\calF}{\mathcal{F}}
\newcommand{\calG}{\mathcal{G}}
\newcommand{\calH}{\mathcal{H}}
\newcommand{\calJ}{\mathcal{J}}
\newcommand{\calK}{\mathcal{K}}
\newcommand{\calL}{\mathcal{L}}
\newcommand{\calZ}{\mathcal{Z}}
\newcommand{\calX}{\mathcal{X}}
\newcommand{\calC}{\mathcal{C}}
\newcommand{\calV}{\mathcal{V}}
\newcommand{\calB}{\mathcal{B}}
\newcommand{\calN}{\mathcal{N}}
\newcommand{\calM}{\mathcal{M}}

\newcommand{\calq}{\mathcal{Q}}
\newcommand{\calw}{\mathcal{W}}
\newcommand{\cale}{\mathcal{E}}
\newcommand{\calr}{\mathcal{R}}
\newcommand{\calt}{\mathcal{T}}
\newcommand{\caly}{\mathcal{Y}}
\newcommand{\calu}{\mathcal{U}}
\newcommand{\cali}{\mathcal{I}}
\newcommand{\calo}{\mathcal{O}}
\newcommand{\calp}{\mathcal{P}}
\newcommand{\cala}{\mathcal{A}}
\newcommand{\cals}{\mathcal{S}}
\newcommand{\cald}{\mathcal{D}}
\newcommand{\calf}{\mathcal{F}}
\newcommand{\calg}{\mathcal{G}}
\newcommand{\calh}{\mathcal{H}}
\newcommand{\calj}{\mathcal{J}}
\newcommand{\calk}{\mathcal{K}}
\newcommand{\call}{\mathcal{L}}
\newcommand{\calz}{\mathcal{Z}}
\newcommand{\calx}{\mathcal{X}}
\newcommand{\calc}{\mathcal{C}}
\newcommand{\calv}{\mathcal{V}}
\newcommand{\calb}{\mathcal{B}}
\newcommand{\caln}{\mathcal{N}}
\newcommand{\calm}{\mathcal{M}}

\newcommand{\twist}{\sigma}

\usepackage{framed}
\definecolor{shadecolor}{rgb}{.925,0.925,0.925}
\usepackage[style=numeric,%citestyle=numeric,
url=false,doi=false,isbn=false,eprint=false]{biblatex}%
\hypersetup{colorlinks=false,pdfborder={0 0 0}}



\makeatletter
\renewcommand{\@seccntformat}[1]{\csname the#1\endcsname.\quad}
\makeatother


\theoremstyle{plain}
\newtheorem{theorem}{Theorem}
\newtheorem*{completenesstheorem}{Completeness Theorem}
\newtheorem{twistinglemma}[thm]{Twisting Lemma}
\renewcommand*{\thetheorem}{\Alph{theorem}}
\newtheorem{conjectureAlpha}{Conjecture}
\renewcommand*{\theconjectureAlpha}{\Alph{conjectureAlpha}}

\newcommand{\pgr}{\mathds{1}}
\renewcommand{\pgr}{\textup{+}}
\newcommand{\PMonad}{{\calP^\textup{u}}}
\newcommand{\Palg}{{\calP}}
\newcommand{\LieOperad}{{\scrL}}

\newcommand{\NOFULLPAGE}{\relax}

\newcommand{\Sq}{\mathrm{Sq}}
\newcommand{\Comm}{\calC}
\newcommand{\F}{\mathbb{F}}

%\bibliography{papers}
\bibliography{../../Dropbox/logbook/_LOGBOOK/papers}




\title[Unstable Lie algebras and their cohomology]{Unstable Lie algebras and their cohomology}
%\author[M.\ Donovan]{Michael Donovan}

%\address{Department of Mathematics \\ Massachusetts Institute of Technology}
%\email{mdono@math.mit.edu}



\newcommand{\dupdown}[2]{D_{\smash{#1}}}
\newcommand{\caldup}[1]{\calD_{\smash{#1}}}
\newcommand{\caldupdown}[2]{\calD^{\smash{#1}}_{\smash{#2}}}

\begin{document}
\newcommand{\todo}[2]{\begin{shaded}\begin{itemize}
\setlength{\parindent}{.25in}
\item[{\Large$\smash\diamondsuit$}] #1
\ifblank{#2}{}{\tiny\begin{itemize}
\setlength{\parindent}{.25in}
\item #2
\end{itemize}}
\end{itemize}\end{shaded}
}
%\begin{abstract}\end{abstract}
%\maketitle
%\tableofcontents

\section{Unstable Lie algebras over Goerss' $P$-algebra}
\todo{Recall definition of Goerss' category $\calW$}{Define them to be locally finite and connected.}
\todo{Construct monad $U$ for $\calW$}{Write in terms of monads $\PMonad$ and $S(\scrL)$.\item Decide whether to give the distributive law.
\item give the $\calV(m,n)$ notation here.}
Write $\Palg$ for Goerss' algebra, generated by $P^2,P^3,\ldots,$. This can be viewed as a graded associative algebra in $\calV(1,0)$, by thinking of the generator $P^i$ as having grading $i+1$ (or even in $\calV(1,1)$, taking $P^i\in(\Palg)^{i+1}_1$). For an object $V\in\calV(+,0)$, the tensor product $\Palg\otimes V$ is an object of $\calV(+,0)$ in the evident way. Write $\PMonad$ for the monad for unstable modules (in which everything is defined, but you set various $P$ to zero).

Next, the Lie operad $\LieOperad$ can be $\calV(1,0)$-graded by viewing the elements of $\scrL(n)$ as having grading $n-1$. Write $S(\LieOperad)$ for the free bad Lie algebra monad (following FresseSimplicialAlgs.pdf and his posets paper, 1.2.16), and $\Lambda(\LieOperad)$ for the good Lie algebra monad. Both of these monads are now monads on $\scrV(1,0)$.

In order to construct the monad $U$ of [Goerss], we'll coequalize two maps:
\[\xymatrix@R=4mm{
\Palg\otimes S(\LieOperad)V\ar[r]^-{\textup{id}\otimes \textup{frob}}
&%r1c1
\Palg\otimes S(\LieOperad)V\ar@{->>}[r]&%r1c2
\PMonad(S(\LieOperad)V)%r1c3
}\]
and the map $\Palg\otimes S(\LieOperad)V\to\PMonad (S(\LieOperad)V)$ whose restriction to each subset $\Palg\otimes (S(\LieOperad)V)^r$ is the composite
\[\xymatrix@R=4mm{
\Palg\otimes (S(\LieOperad)V)^r\ar[r]^-{\textup{mult}_{P^r}\otimes\textup{id}}
&
\Palg\otimes (S(\LieOperad)V)^r\ar@{^{(}->}[r]
&%r1c1
\Palg\otimes S(\LieOperad)V\ar@{->>}[r]&%r1c2
\PMonad(S(\LieOperad)V)%r1c3
}\]
One can check, using Goerss' description of the monad $U$, that it can be described by the coequaliser in $\calV(+,0)$:
\[\xymatrix@R=4mm{
\Palg\otimes S(\LieOperad)V\ar@<.5ex>[r]\ar@<-.5ex>[r]
&%r1c1
\PMonad(S(\LieOperad)V)\ar@{->>}[r]
&%r1c2
UV%r1c3
}\]
The functor $U$ becomes a monad using the commutation relations given by Goerss, so that all of $U,\PMonad,S(\LieOperad),\Lambda(\LieOperad)$ are monads on graded vector spaces $\calV(+,0)$. \textbf{Quadratic grading?}
Note further that there is a natural isomorphism of functors $Q_P\circ U=\Lambda(\LieOperad)$.



\todo{Explain why $H^*_{\calC}$ is in $\calW$ (optional)}{Give $j$ map to construct everything, along with the relevant AW/EZ.
\item The the AW/EZ in a way chosen to be consistent with what will be done in the Adams operations paper. Look in adamsops.tex}
\todo{State Goerss' theorem in these terms}{talk about cohomology operations for $\calC$ and Yoneda's lemma}
\todo{Define $Q^\calW:\calW\to \calV^1$, and thus the cohomology functor on $\calW$}{Give the bar construction $B_\bullet^{\calW}$
\item Explain significance as Adams $E^2$}
We will use the standard simplicial bar construction as our simplicial resolution of $X\in\calW$, writing $B^\calW_sX=U^{s+1}X$. Then the cohomology of $X$, $H^{s}_t(X)\in\calV(1,+)$, is the cohomology of the cochain complex
\[C^{s}_{t}=\Hom(N_s(Q_\calW B_{\bullet}X)^{t},\F_2).\]
We'll frequently use the identification $Q_\calW B_{s}X=U^{s}X$.

\section{Cohomology operations for unstable Lie algebras over $P$}
\todo{Define the operations $\gamma_i$, state and define the $\delta_i$ operations}{still need to clarify the end of the proof of the relations.}
For $V^{*}\in \calV(+,0)$, we will define natural homomorphisms
\[\gamma^i:(UV)^{n+i+1}\to V^{n}, \textup{ for $2\leq i<n$}.\]
Indeed, there are natural maps
\begin{align*}
P^i:V^{n}&\to UV^{n+i+1}, \textup{ for $2\leq i<n$}\\
[\,,]:V^{n_1}\otimes V^{n_2}&\to UV^{n_1+n_2+1},
\end{align*}
whose images are linearly independent and span the quadratic filtration 2 part of $UV^*$, and this restricted $P^i$ is an injection. We define $\gamma^i$ to be the composite
\[(UV)^{n+i+1}\to (UV)^{n+i+1,(2)}\to \textup{Im}(P^i)\overset{(P^i)^{-1}}{\to}V^{n}.\]


\begin{prop}
There are natural operations $\delta_i:H^{s}_tW\to H^{s+1}_{t+i+1}W$ defined whenever $2\leq i <n$. If $\alpha\in C^{s}_t$ is a cocycle, $\delta_i\alpha$ is the composite
\[N_{s+1}(Q_U B_{\bullet}W)^{t+i+1}\overset{\gamma^i}{\to}N_s(Q_U B_{\bullet}W)^{t}\overset{\alpha}{\to}\F_2,\]
after identifying $Q_U B_{s+1}W$ with $U^{s}W$ and $Q_U B_{s}W$ with $U^{s+1}W$. These operations satisfy the Adem relation (see Dwyer) for $\delta$ operations.
\end{prop}
\begin{proof}
One needs to see that $\gamma_i$ is a chain homomorphism respecting the normalisation functor:\[\xymatrix@R=4mm{
N_{s+1}(Q_U B_{\bullet}W)^{t+i+1}\ar[r]^-{\gamma^i}
\ar[d]^-{\partial}&%r1c1
N_s(Q_U B_{\bullet}W)^{t}
\ar[d]^-{\partial}
\\%r1c2
N_{s}(Q_U B_{\bullet}W)^{t+i+1}\ar[r]^-{\gamma^i}&%r2c1
N_{s-1}(Q_U B_{\bullet}W)^{t}%r2c2
}\]
For this, it is enough to produce commuting squares:
\[\xymatrix@R=4mm{
(U^sW)^{t+i+1}\ar[r]^-{\gamma^i}
\ar[d]^-{d_0+d_1}&%r1c1
(U^{s-1}W)^{t}
\ar[d]^-{d_0}
\\%r1c2
(U^{s-1}W)^{t+i+1}\ar[r]^-{\gamma^i}&%r2c1
(U^{s-2}W)^{t}%r2c2
}\]
and for $j\geq1$:
\[\xymatrix@R=4mm{
(U^sW)^{t+i+1}\ar[r]^-{\gamma^i}
\ar[d]^-{d_{j+1}}&%r1c1
(U^{s-1}W)^{t}
\ar[d]^-{d_j}
\\%r1c2
(U^{s-1}W)^{t+i+1}\ar[r]^-{\gamma^i}&%r2c1
(U^{s-2}W)^{t}%r2c2
}\]
The later squares are easy, and and for the first one, starting with $f_{(s)}g_{(s-1)}h_{(s-2)}\in U^sW$, we find $\gamma^i d_0=u(f)\gamma^i (g)(h_{(s-2)})$,
$\gamma^i d_1=\gamma^i (fg)(h_{(s-2)})$, and
$d_0\gamma^i =\gamma^i (f)u(g)(h_{(s-2)})$. These three quanitities sum to zero, as the single operation $P^i$ is indecomposable. This proves the well-definedness of these operations.

\textbf{To do:} What remains is to check that the Adem relations are satisfied. For this, show that we can compress to the terms in which there is quadratic grading 2 in each construction. Then the Adem relation is the image under the differential of something obvious.
\end{proof}

\todo{Define and state $\Sq$ operations and products}{Need to define $j:Q(X\wedge Y)\to QX\otimes QY$}
We can identify
\[Q_UB_sX=Q_{\Lambda(\calL)}\Lambda(\calL)U^sX\]
as $Q_{\Lambda(\calL)}$ applied to an almost free simplicial Lie algebra. As such, its homotopy has operations induced by the map
\[\xi_\calL:\left(Q_{\Lambda(\calL)}\Lambda(\calL)U^{s+1}X
\overset{Q(d_0\sqcup d_0\psi+\psi d_0)}{\to}
Q_{\Lambda(\calL)}((\Lambda(\calL)U^sX)^{\wedge2})\overset{j}{\to}
(Q_{\Lambda(\calL)}\Lambda(\calL)U^sX)^{\otimes2}\right)\]
where by $d_0$ we mean the map $\Lambda(\calL)(P\circ S(\calL)/\sim)\to \Lambda(\calL)$ arising from killing the $P$ part and identifying all the Lie stuff together. This only sees the Lie stuff, not the part that has any $P$ in it. Thus it could be written as $f(g)\mapsto q^\calL(f)(g)$.

The operations $\Sq^j:H_t^{s}X\to H_{2t+1}^{s+j}X$ are defined as follows. Suppose that $\alpha\in C_t^{s}$ is a cocycle. Then we define $\Sq^j\alpha$ as the composite
\[N_{s+j}(Q_UB_{\bullet}X)^{2t+1}\overset{\xi_\calL}{\to}N_{s+j-1}((Q_UB_{\bullet}X)^{\otimes2})^{2t}\overset{\mathbb{D}_{s-j+1}}{\to}
((N_*Q_UB_{\bullet}X)^{\otimes2})^{2t}_{2s}\overset{\alpha\otimes\alpha}{\to}\F_2.
\]
Another way to say this, in the style of Goerss (and others), is to define a function
\[\Theta^j:C_{t}^{s}\to C_{2t+1}^{s+j+1},\ \ \Theta^j(\alpha)=\phi^*_\calL\mathbb{D}_{s-j+1}^*(\alpha\otimes\alpha)+ \phi^*_\calL\mathbb{D}_{s-j+2}^*(\alpha\otimes d\alpha).\]
Then, (like Goerss shows), $d\Theta^j\alpha=\Theta^jd\alpha$, and we can define $\Sq^j$ to be the map on cohomotopy defined by $\Theta^j$. Similarly, we can define a cochain complex homomorphism
\[\Psi:C_t^{s}\otimes C_{t'}^{s'}\to C_{t+t'+1}^{s+s'+1}\]
by the formula $\psi^*_\calL\mathbb{D}_0^*$, and use this to define a pairing of cohomology.
\begin{prop}
There are natural homomorphisms
\[\Sq^j:H_t^{s}X\to H_{2t+1}^{s+j}X,\]
zero unless $3\leq j\leq s+1$, and satisfying the Adem relations for the Steenrod algebra. There is a natural nonunital commutative algebra pairing
\[H_t^{s}X\otimes H_{t'}^{s'}X\to H_{t+t'+1}^{s+s'+1}X.\]
These satisfy an unstableness condition:
\[x^2=\Sq^{s+1}x\text{ for }x\in H^{s}_tX\]
and the standard Cartan formula:
\[\Sq^k(xy)=\sum_{i=0}^k(\Sq^ix)(\Sq^{k-i}y).\]
\end{prop}
\begin{proof}
\textbf{To do:} Insert the proof of the Steenrod relations from prliealgs.tex, citing Priddy or really Singer.
\end{proof}
\todo{Give theorem relating all the given operations}{prove the $\delta$ Adem relations using Priddy compression
\item show that the Sq-ops and products are instances of those on the cohomology of a Lie algebra (in particular that $j_\calW$ restricts to $j_\calL$), then quote the appendix
\item The commutation relation for $\delta$ vs $\Sq$/product is written on the wall
\item need to figure out what happens in low dimensions}
Do a Steenrod or product operation, and then a $\delta_i$, and you're using the composite of this map with
\[Q_\calW U^{s+3}X=U^{s+2}X\to U^{s+1}X=Q_\calW U^{s+2}X\]
which all boils down to
\[P^i_{(s+2)}f_{(s+1)}g\mapsto q^{\calL}(f)(g)\]
Anyway, I have a nullhomotopy of this map, which might be called $((P^i)^{-1}q^{\calL})$, and is stuck on the wall in the upper right corner.


\section{Unstable Lie algebras over the $\Lambda$-algebra}
\todo{Define the categories $\calW(n)$, $\calL(n)$ ($n=0$ is Goerss' case)}{need to be clear about categories of vector spaces
\item Write $\calV(m,n)$ for the category of loc fin vecspaces with $m$ upper, $n$ lower indices
\item Use blackboard bold symbol $\pgr$ to denote only strictly positive gradings
\item $\calW(n)$ and $\calL(n)$ are monadic over $\calV(\pgr,n)$, but we won't carefully describe the monads.}
\todo{Derived functors of $Q^{\calW(n)}:\calW(n)\to \calV(\pgr,n)$ are $H^*_{\calW(n)}:\calW(n)\to\calV(n+1,\pgr)$}{and we're interested in these for $n=0$.}
\todo{Define functors $Q^P:\calW(0)\to\calL(0)$ and $Q^{\Lambda(n)}:\calW(n)\to \calL(n)$ for $n\geq1$}{explain that when $n>0$ you divide by nontop operations only, and the opposite for $n=0$
\item point is that $Q^{\calW(n)}$ factors as these followed by $Q^{\calL(n)}:\calL(n)\to \calV(\pgr,n)$}
\todo{Note that $\calL(n)$-$\Pi$-\textup{alg} is the category $\calW(n+1)$}{comment that this is why $\calW(n)$ will be interesting for $n>0$}
%\todo{Define cohomology $H_{\calW(n)}^*:\calW(n)\to \calV(n+1,\pgr)$}{}
\todo{Define homology $H^{\Lambda(n)}_*:\calW(n)\to \calV(\pgr,n+1)$ and enrich codomain to $\calW(n+1)$}{also do $n=0$ case: define homology $H^P_*:\calW(0)\to \calV(\pgr,1)$ and enrich codomain to $\calW(1)$}

\section{Cohomology operations for unstable Lie algebras over $\Lambda$}
\todo{Do all the same work as in the $n=0$ case, to get horizontal and vertical $\Sq$ and products}{must explain in detail what the available operations are
\item relations between them less important}

\section{Composite functor spectral sequences}
\todo{Define the spectral sequence $H^*_{\calW(n)}\Longleftarrow H^*_{\calW(n+1)}H_*^{\Lambda(n)}$}{reference Blanc-Stover, but then give the construction of bisimplicial object (using `$W$' comonad)}
\todo{give the construction of the chain level diagonal}{}
\todo{use Singer's techniques to give spectral sequence operations}{prove compatibility at $E^2$ for horizontal $\Sq$ and product
\item prove compatibility at $E^2$ for vertical $\Sq$ with Koszul operation
\item prove compatibility at $E^\infty$ for all $\Sq$ and product}
\todo{Discuss the edge homomorphism $e:\left(E^2(n)\to\!\!\!\!\!\!\!\!\!\to E^\infty(n+1)\subseteq E^2_{0}(n+1)\right)$}{figure out what the relevant theorem here should be}

\section{The cohomology of a trivial unstable Lie algebra over $P$}
\todo{State interest in this calculation}{write $X=\bigoplus_\alpha \F_2[r_\alpha]$, a finite direct sum}
\todo{Construct the relevant elements $\Sq^J\delta_I\imath_\alpha$ and state a theorem on $H^*X$}{$\imath_\alpha\in H^0_{r_\alpha}X$ is the functional which projects $QX\cong X$ onto $\F_2[r_\alpha]$.
\item Clearly state the contraints relevant for $K$ and $I$.
\item Also describe which products thereof are not known to be zero, and state a theorem.}
\todo{Proposition: any of the $\Sq^J\delta_I\imath_\alpha$ is detected by $\Sq^{J_n}_\textup{v}e\cdots e\Sq^{J_1}_\textup{v}e\delta_I\imath_\alpha$}{again, clearly state the constraints on the $J_i$.
\item Push out to $ee\Sq^{J_n}_\textup{v}e\cdots e\Sq^{J_1}_\textup{v}e\delta_I\imath_\alpha\in E^2(n+2)_*^{00**\cdots *}$}
\todo{Calculate the groups $E^2(n)_*^{*0**\cdots *}$}{Observe importance of these groups, since everything's detected in one}
\todo{Identify products of $ee\Sq^{J_{n-2}}_\textup{v}e\cdots e\Sq^{J_1}_\textup{v}e\delta_I\imath_\alpha\in E^2(n)_*^{00**\cdots *}$ in $E^2(n)_*^{*0**\cdots *}$}{I still find this hard to write down}
\todo{Prove the theorem on $H^*X$}{start with anything in $E^2(n)$
\item it's detected by something in $E^2(N)_*^{*0**\cdots *}$
\item anything there is a product of $ee\Sq^{J_{N-2}}_\textup{v}e\cdots e\Sq^{J_1}_\textup{v}e\delta_I\imath_\alpha$
\item this converges down to $\Sq^K_h \Sq^{J_n}_\textup{v}e\cdots e\Sq^{J_1}_\textup{v}e\delta_I\imath_\alpha$, which must equal the original element
\item this is a permanent cycle, proving that there are never any differentials in any of the cfsseqs
\item thus $E^2(0)^*_*$ is a direct sum of certain of the $E^2(n)^{*0**\cdots *}_*$ via collapsing sequences
\item stocktake reveals that these terms are everything we hoped}

\appendix
\section{Unstable Lie coalgebras over the dual $P$-algebra (optional)}
\todo{Give a formal definition of these coalgebras}{goes down to level of simplicial commutative algebras, and brings in the SOA comonad structure}
\todo{Give the construction (via equaliser diagram, etc) of the comonad}{}
\todo{Write the maps $j$ and $\gamma_i$ in terms of maps out of $C_{ij}$ and $C_i^j$}{include generating cofibrations such as $S^{i+j}\to C_{ij}$ and $\Delta[1]\times S^{i+j}\to \Delta[1]\times C_{ij}$}
\todo{Construct operations on homology of such coalgebras}{they agree with those in double dual construction when locally finite
\item if not locally finite, by studying gradings, can see that every coalgebra is the union of its finite type (even just `finite') subcoalgebras, and that the cobar construction respects such unions, thus get the same results in infinite case as we did from double dualisation.}


\section{Operations on Lie algebra cohomology}
\todo{Recall Priddy's definition of Lie algebra cohomology, and results}{Explain that they don't calculate everything for us already, since the actual simplicial Lie algebras we're looking at are not GEMs --- their homotopy supports nontrivial operations.}
\todo{Modify Priddy's spectral sequence argument to identify his operations with ours}{uses Singer's work}









\end{document}



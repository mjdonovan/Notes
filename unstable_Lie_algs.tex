% !TEX root = z_output/unstable_Lie_algs.tex
\documentclass[11pt]{amsart} \renewcommand{\baselinestretch}{1.2}
%\newcommand{\IncludeAll}{}%
\newcommand{\xShrinkPages}{}%
\newcommand{\xPostponeContents}{}%
%%%%%%%%%%%%%%%%%%% 
%%%%%%%%%%%%%%%%%%%Begin preamble for pasteover
%%%%%%%%%%%%%%%%%%% 
\ifx\ShrinkPages\undefined\else\usepackage[paperheight=220mm,paperwidth=160mm]{geometry}\fi%220/160 height options
\usepackage{amsmath,amsthm,amssymb} \usepackage{eucal} \usepackage{mathrsfs,nicefrac} \usepackage{amssymb} \usepackage[all]{xy} \usepackage{cancel} \usepackage[textsize=small,color=green!40]{todonotes} \usepackage{color} \usepackage{version} \usepackage{enumerate} \usepackage{mathtools} \newcommand{\ooooo}[2][]{\excludeversion{#2}} \newcommand{\wooooo}[2][]{\includeversion{#2}}%mathtools is just used for `mathclap'
\ooooo[	do]{todolist}\ooooo[]{random junk}
\ifx\IncludeAll\undefined\else\renewcommand{\ooooo}[2][]{\wooooo{#2}}\fi
\ooooo[]{ThankS}
\ooooo[	ct]{Contents Page}
\wooooo[	 1]{Introduction}
\ooooo[	 2]{Conventions and notation}
\ooooo[	 3]{Pi-algebras and cohomology algebras}
\ooooo[	 4]{Bousfield-Kan spectral sequence}
\ooooo[	 5]{Constructing homotopy operations}
\ooooo[	 6]{Constructing cohomology operations}
\ooooo[	 7]{homotopy operations for PRLs}
\ooooo[	 8]{Cohomology Operations for W and U}
\ooooo[	 9]{Koszul complexes}
\ooooo[	10]{second quadrant homotopy sseq operations}
\ooooo[	11]{Operations on the Bousfield-Kan spectral sequence}
\ooooo[	12]{Comp funct sseqs}
\ooooo[	13]{Operations in composite functor spectral sequences}
\ooooo[	14]{Calculations of HWn}
\ooooo[	15]{The Bousfield-Kan spectral sequence for a sphere}
\ooooo[	16]{May sseq and vanishing line}
\ooooo[	ap]{appendices}
\ooooo[	bb]{bibliog}


\usepackage[bookmarks=false,pdftex,pdfborder={0 0 0 [1 1]}]{hyperref}



%>>>>>>>>>>>>>>>>>>>>>>>>>>>>>>
%<<<  Theorem Environments  <<<
%>>>>>>>>>>>>>>>>>>>>>>>>>>>>>>
\makeatletter
\def\thmhead@plain#1#2#3{%
  \thmname{#1}\thmnumber{\@ifnotempty{#1}{ }\@upn{#2}}%
  \thmnote{ {\the\thm@notefont#3}}}
\let\thmhead\thmhead@plain
\makeatother

\ifx\ISMITTHESIS\undefined
\theoremstyle{plain}
\newtheorem{thm}{Theorem}[section] %CHANGE to chapter
\newtheorem{lem}[thm]{Lemma}
\newtheorem{prop}[thm]{Proposition}
\newtheorem{cor}[thm]{Corollary}
\newtheorem{conjecture}{Conjecture}
\numberwithin{equation}{section} %CHANGE to chapter
\else
\theoremstyle{plain}
\newtheorem{thm}{Theorem}[chapter] %CHANGE to chapter
\newtheorem{lem}[thm]{Lemma}
\newtheorem{prop}[thm]{Proposition}
\newtheorem{cor}[thm]{Corollary}
\newtheorem{conjecture}{Conjecture}
\numberwithin{equation}{chapter} %CHANGE to chapter
\fi

%>>>>>>>>>>>>>>>>>>>>>>>>>>>>>>
%<<<       Operators        <<<
%>>>>>>>>>>>>>>>>>>>>>>>>>>>>>>
\DeclareMathOperator{\ad}{\textbf{ad}}
\DeclareMathOperator{\coker}{coker}
\renewcommand{\ker}{\mathrm{ker}\,}
\DeclareMathOperator{\End}{End}
\DeclareMathOperator{\Aut}{Aut}
\DeclareMathOperator{\Hom}{Hom}
\DeclareMathOperator{\Maps}{Maps}
\DeclareMathOperator{\Mor}{Mor}
\DeclareMathOperator{\Gal}{Gal}
\DeclareMathOperator{\Ext}{Ext}
\DeclareMathOperator{\Tor}{Tor}
\DeclareMathOperator{\Cotor}{Cotor}
\DeclareMathOperator{\Prim}{Pr}
\DeclareMathOperator{\Tot}{Tot}
\DeclareMathOperator{\Map}{Map}
\DeclareMathOperator{\Der}{Der}
\DeclareMathOperator{\Rad}{Rad}
\DeclareMathOperator{\rank}{rank}
\DeclareMathOperator{\ArfInvariant}{Arf}
\DeclareMathOperator{\KervaireInvariant}{Ker}
\DeclareMathOperator{\im}{im}
\DeclareMathOperator{\coim}{coim}
\DeclareMathOperator{\trace}{tr}
\DeclareMathOperator{\supp}{supp}
\DeclareMathOperator{\ann}{ann}
\DeclareMathOperator{\spec}{Spec}
\DeclareMathOperator{\SPEC}{\textbf{Spec}}
\DeclareMathOperator{\proj}{Proj}
\DeclareMathOperator{\PROJ}{\textbf{Proj}}
\DeclareMathOperator{\fiber}{fib}
\DeclareMathOperator{\cofiber}{cof}
\DeclareMathOperator{\cone}{cone}
\DeclareMathOperator{\skel}{sk}
\DeclareMathOperator{\coskel}{cosk}
\DeclareMathOperator{\conn}{conn}
\DeclareMathOperator*{\colim}{colim}
\DeclareMathOperator*{\limit}{lim}
\DeclareMathOperator*{\hocolim}{hocolim}
\DeclareMathOperator*{\holimit}{holim}
\DeclareMathOperator*{\holim}{holim}
\DeclareMathOperator*{\hofib}{hofib}
\DeclareMathOperator*{\hocof}{hocof}
\DeclareMathOperator*{\hotfib}{thofib}
\DeclareMathOperator*{\equaliser}{eq}
\DeclareMathOperator*{\coequaliser}{coeq}
\DeclareMathOperator{\ch}{ch}
\DeclareMathOperator{\Thom}{Th}
\DeclareMathOperator{\GrthGrp}{GrthGp}
\DeclareMathOperator{\Sym}{Sym}
\DeclareMathOperator{\Prob}{\mathbb{P}}
\DeclareMathOperator{\Exp}{\mathbb{E}}
\DeclareMathOperator{\GeomMean}{\mathbb{G}}
\DeclareMathOperator{\Var}{Var}
\DeclareMathOperator{\Cov}{Cov}
\DeclareMathOperator{\Sp}{Sp}
\DeclareMathOperator{\Seq}{Seq}
\DeclareMathOperator{\Cyl}{Cyl}
\DeclareMathOperator{\Ev}{Ev}
\DeclareMathOperator{\sh}{sh}
\DeclareMathOperator{\intHom}{\underline{Hom}}
\DeclareMathOperator{\Frac}{frac}
\DeclareMathOperator{\homog}{hg}

%>>>>>>>>>>>>>>>>>>>>>>>>>>>>>>
%<<<  Mathematical Symbols  <<<
%>>>>>>>>>>>>>>>>>>>>>>>>>>>>>>
\newcommand{\DASH}{\mathrm{-}}

%>>>>>>>>>>>>>>>>>>>>>>>>>>>>>>
%<<<     Greek Letters      <<<
%>>>>>>>>>>>>>>>>>>>>>>>>>>>>>>
\let\oldphi\phi
\let\phi\varphi
\renewcommand{\to}{\longrightarrow}
\newcommand{\from}{\longleftarrow}
\newcommand{\eps}{\varepsilon}

%%>>>>>>>>>>>>>>>>>>>>>>>>>>>>>>
%%<<<     Environments       <<<
%%>>>>>>>>>>>>>>>>>>>>>>>>>>>>>>
\newcommand{\squishlist}{
  \setlength{\itemsep}{.5pt}
  \setlength{\parskip}{0pt}
  \setlength{\parsep}{0pt}}
%>>>>>>>>>>>>>>>>>>>>>>>>>>>>>>
%<<<     Script Letters     <<<
%>>>>>>>>>>>>>>>>>>>>>>>>>>>>>>
\newcommand{\scrQ}{\mathscr{Q}}
\newcommand{\scrW}{\mathscr{W}}
\newcommand{\scrE}{\mathscr{E}}
\newcommand{\scrR}{\mathscr{R}}
\newcommand{\scrT}{\mathscr{T}}
\newcommand{\scrY}{\mathscr{Y}}
\newcommand{\scrU}{\mathscr{U}}
\newcommand{\scrI}{\mathscr{I}}
\newcommand{\scrO}{\mathscr{O}}
\newcommand{\scrP}{\mathscr{P}}
\newcommand{\scrA}{\mathscr{A}}
\newcommand{\scrS}{\mathscr{S}}
\newcommand{\scrD}{\mathscr{D}}
\newcommand{\scrF}{\mathscr{F}}
\newcommand{\scrG}{\mathscr{G}}
\newcommand{\scrH}{\mathscr{H}}
\newcommand{\scrJ}{\mathscr{J}}
\newcommand{\scrK}{\mathscr{K}}
\newcommand{\scrL}{\mathscr{L}}
\newcommand{\scrZ}{\mathscr{Z}}
\newcommand{\scrX}{\mathscr{X}}
\newcommand{\scrC}{\mathscr{C}}
\newcommand{\scrV}{\mathscr{V}}
\newcommand{\scrB}{\mathscr{B}}
\newcommand{\scrN}{\mathscr{N}}
\newcommand{\scrM}{\mathscr{M}}


%\newcommand{\frakq}{\mathfrak{q}}
%\newcommand{\frakw}{\mathfrak{w}}
%\newcommand{\frake}{\mathfrak{e}}
%\newcommand{\frakr}{\mathfrak{r}}
\newcommand{\frakt}{\mathfrak{t}}
%\newcommand{\fraky}{\mathfrak{y}}
%\newcommand{\fraku}{\mathfrak{u}}
%\newcommand{\fraki}{\mathfrak{i}}
%\newcommand{\frako}{\mathfrak{o}}
%\newcommand{\frakp}{\mathfrak{p}}
%\newcommand{\fraka}{\mathfrak{a}}
\newcommand{\fraks}{\mathfrak{s}}
\newcommand{\frakd}{\mathfrak{d}}
%\newcommand{\frakf}{\mathfrak{f}}
%\newcommand{\frakg}{\mathfrak{g}}
%\newcommand{\frakh}{\mathfrak{h}}
%\newcommand{\frakj}{\mathfrak{j}}
%\newcommand{\frakk}{\mathfrak{k}}
%\newcommand{\frakl}{\mathfrak{l}}
%\newcommand{\frakz}{\mathfrak{z}}
%\newcommand{\frakx}{\mathfrak{x}}
%\newcommand{\frakc}{\mathfrak{c}}
%\newcommand{\frakv}{\mathfrak{v}}
\newcommand{\frakb}{\mathfrak{b}}
%\newcommand{\frakn}{\mathfrak{n}}
%\newcommand{\frakm}{\mathfrak{m}}

%>>>>>>>>>>>>>>>>>>>>>>>>>>>>>>
%<<<  Caligraphic Letters   <<<
%>>>>>>>>>>>>>>>>>>>>>>>>>>>>>>
\newcommand{\calQ}{\mathcal{Q}}
\newcommand{\calW}{\mathcal{W}}
\newcommand{\calE}{\mathcal{E}}
\newcommand{\calR}{\mathcal{R}}
\newcommand{\calT}{\mathcal{T}}
\newcommand{\calY}{\mathcal{Y}}
\newcommand{\calU}{\mathcal{U}}
\newcommand{\calI}{\mathcal{I}}
\newcommand{\calO}{\mathcal{O}}
\newcommand{\calP}{\mathcal{P}}
\newcommand{\calA}{\mathcal{A}}
\newcommand{\calS}{\mathcal{S}}
\newcommand{\calD}{\mathcal{D}}
\newcommand{\calF}{\mathcal{F}}
\newcommand{\calG}{\mathcal{G}}
\newcommand{\calH}{\mathcal{H}}
\newcommand{\calJ}{\mathcal{J}}
\newcommand{\calK}{\mathcal{K}}
\newcommand{\calL}{\mathcal{L}}
\newcommand{\calZ}{\mathcal{Z}}
\newcommand{\calX}{\mathcal{X}}
\newcommand{\calC}{\mathcal{C}}
\newcommand{\calV}{\mathcal{V}}
\newcommand{\calB}{\mathcal{B}}
\newcommand{\calN}{\mathcal{N}}
\newcommand{\calM}{\mathcal{M}}

\newcommand{\calq}{\mathcal{Q}}
\newcommand{\calw}{\mathcal{W}}
\newcommand{\cale}{\mathcal{E}}
\newcommand{\calr}{\mathcal{R}}
\newcommand{\calt}{\mathcal{T}}
\newcommand{\caly}{\mathcal{Y}}
\newcommand{\calu}{\mathcal{U}}
\newcommand{\cali}{\mathcal{I}}
\newcommand{\calo}{\mathcal{O}}
\newcommand{\calp}{\mathcal{P}}
\newcommand{\cala}{\mathcal{A}}
\newcommand{\cals}{\mathcal{S}}
\newcommand{\cald}{\mathcal{D}}
\newcommand{\calf}{\mathcal{F}}
\newcommand{\calg}{\mathcal{G}}
\newcommand{\calh}{\mathcal{H}}
\newcommand{\calj}{\mathcal{J}}
\newcommand{\calk}{\mathcal{K}}
\newcommand{\call}{\mathcal{L}}
\newcommand{\calz}{\mathcal{Z}}
\newcommand{\calx}{\mathcal{X}}
\newcommand{\calc}{\mathcal{C}}
\newcommand{\calv}{\mathcal{V}}
\newcommand{\calb}{\mathcal{B}}
\newcommand{\caln}{\mathcal{N}}
\newcommand{\calm}{\mathcal{M}}
\newcommand{\calmv}{\mathcal{M}\dver}
\newcommand{\calmh}{\mathcal{M}\dhor}
\newcommand{\calMv}{\mathcal{M}\dver}
\newcommand{\calMh}{\mathcal{M}\dhor}
\newcommand{\calMhv}{\mathcal{M}_\mathrm{hv}}
\newcommand{\spheres}[1]{\mathrm{sph}(#1)}
\newcommand{\cones}[1]{\mathrm{con}(#1)}
\newcommand{\DeltatubfD}{\Delta_{\mathrm{\textbf{B}}}}

\usepackage{framed}
\definecolor{shadecolor}{rgb}{.925,0.925,0.925}
\usepackage[style=numeric,%citestyle=numeric,
url=false,doi=false,isbn=false,eprint=false]{biblatex}%
\hypersetup{colorlinks=false,pdfborder={0 0 0}}

\newcommand{\citeBOX}[2][]{\cite[\mbox{#1}]{#2}}

\newdir{ >}{{}*!/-7pt/@{>}}
\newdir{ >>}{{}*!/-14pt/@{>}}

\makeatletter
\renewcommand{\@seccntformat}[1]{\csname the#1\endcsname.\quad}
\makeatother

\setcounter{tocdepth}{2}

\newcommand{\PMonad}{{\calP^\mathrm{u}}}
\newcommand{\LambdaMonad}{\Lambda^\mathrm{u}}
\newcommand{\Palg}{{\calP}}
\newcommand{\deltaalg}{\Delta} %change me
\newcommand{\LieOperad}{{\scrL}}
\newcommand{\CommOperad}{{\scrC}}
\newcommand{\restn}[1]{#1^{[2]}}
\newcommand{\restnwithsubscript}[2]{#1^{[2]}_{#2}}
\newcommand{\restnRepeated}[2]{#1^{[2^{#2}]}}
\newcommand{\dualrestn}[1]{\sqrt[{[2]}]{#1}}
\renewcommand{\dualrestn}[1]{\sqrt[{\!\!\![2]}]{#1}}
\newcommand{\vect}[2]{\calV^{#1}_{#2}}
\newcommand{\BSW}{{\scrG}}
\newcommand{\BSWres}{B^\BSW}%didn't always remember to use
%\newcommand{\PiAlg}{\textup{-$\Pi$-Alg}}
%\newcommand{\HAlg}{\textup{-$H^*$-Alg}}
%\newcommand{\HCoalg}{\textup{-$H_*$-Coalg}}

\newcommand{\PA}[1]{\pi#1}
\newcommand{\HA}[1]{H#1}
\newcommand{\HC}[1]{H#1\mathrm{-coalg}}

\newcommand{\quadratic}{\mathrm{qu}}
\newcommand{\quadgrad}[1]{\mathrm{q}_{#1}}
\newcommand{\crossterms}{\mathrm{cr}}
\newcommand{\ExtCohOp}{\mathrm{Sq}_\mathrm{ext}}
\newcommand{\vExtCohOp}{\mathrm{Sq}_\mathrm{v,ext}}
\newcommand{\hExtCohOp}{\mathrm{Sq}_\mathrm{h,ext}}
\newcommand{\ExtCohProd}{\mu_\mathrm{ext}}
\newcommand{\epi}{{\,\makebox[0cm][l]{\ensuremath\to}\to{}}}
\newcommand{\epifrom}{{\,\makebox[0cm][l]{\ensuremath\from}\from}}
\newcommand{\mono}{{\to}}
\newcommand{\minDimP}{\overline{m}}
\newcommand{\minDimDelta}{m}
\newcommand{\minDimSq}{\underline{m}}
\newcommand{\excess}{e}
\newcommand{\produces}[3]{#3:#1\sim #2}
\renewcommand{\produces}[3]{#1\rightarrow_{#3} #2}%{J}{I}{P}
\renewcommand{\produces}[3]{#1\overset{\smash{#3}}{\rightarrow} #2}%{J}{I}{P}
\newcommand{\twist}{\omega}
\newcommand{\DeltaUp}{\Delta}% <-- i changed the indices to upper, marking where I did it.
\newcommand{\Nabla}{\nabla}
\newcommand{\Shuffles}[2]{\mathrm{Sh}_{#1#2}}
\newcommand{\HalfShuffles}[2]{\mathrm{Sh}_{#1#2}^{\smash{\!\div2}}}
\newcommand{\Nop}{N^{\smash{-}}}
\newcommand{\NOFULLPAGE}{\relax}
\newcommand{\UEA}{U'}%{U^{[2]}}%if this changes, so does some text
\newcommand{\UEAX}{\bar{X}'}%{U^{[2]}}%if this changes, so does some text
\newcommand{\Sq}{\mathrm{Sq}}
\newcommand{\LieSteen}{\calA}
\newcommand{\aDT}{\mathrm{adm}_+(\Delta,T)}
\newcommand{\aDTnoplus}{\mathrm{adm}(\Delta,T)}
\newcommand{\aDTe}{\mathrm{adm}^{\smash{e}}(\Delta,T)}
\newcommand{\aDTm}{\mathrm{adm}^{\smash{\minDimP}}(\Delta,T)}
\newcommand{\aDTme}{\mathrm{adm}^{\smash{\minDimP\setminus e}}(\Delta,T)}
\newcommand{\aD}[1]{\mathrm{adm}_+(\Delta,#1)}
\newcommand{\aS}[1]{\mathrm{adm}(\LieSteen_{>1},#1)}
\newcommand{\aSirr}[1]{\mathrm{adm}^\mathrm{irr}(\LieSteen_{>1},#1)}
\newcommand{\aDTirr}{\mathrm{adm}_+^\mathrm{irr}(\Delta,T)}
\newcommand{\aDirr}[1]{\mathrm{adm}_+^\mathrm{irr}(\Delta,#1)}
\newcommand{\F}{\mathbb{F}}
\newcommand{\Id}{\mathrm{id}}
\newcommand{\complexes}{\mathrm{ch}_+}
\newcommand{\algs}{{\scrC\!\textit{\normalfont\textit{om}}}}
\newcommand{\assocn}{{\scrA\!\textit{ss}(n)}}
\newcommand{\liealgs}{{\scrL\!\textit{ie}}}
\newcommand{\restliealgs}{{\scrL\!\textit{ie}^\textit{r}}}
\newcommand{\algcat}{{\calc}}%supposed to be temporary - only used in adams convergence
\newcommand{\Ftwo}{\F_2}
\newcommand{\TOP}{\mathfrak{T}}
\newcommand{\STOP}{\mathfrak{S}}

\newcommand{\Eprime}[5]{[E'_{#2}#3]^{#4}_{#5}}
\newcommand{\E}[5]{[E^{#1}_{#2}#3]^{#4}_{#5}}
\newcommand{\Edown}[4]{[E_{#1}#2]^{#3}_{#4}}
\newcommand{\EBdown}[4]{[B_{#1}#2]^{#3}_{#4}}
\newcommand{\Edownup}[5]{[E_{#1}^{#2}#3]^{#4}_{#5}}
\newcommand{\EZ}[5]{[Z^{#1}_{#2}#3]^{#4}_{#5}}
\newcommand{\EZdownup}[5]{[Z^{#2}_{#1}#3]^{#4}_{#5}}
\newcommand{\filt}{\mathrm{filt}}

\newcommand{\uver}{^\mathrm{v}}
\newcommand{\uhor}{^\mathrm{h}}
\newcommand{\dver}{_\mathrm{v}}
\newcommand{\dhor}{_\mathrm{h}}
\newcommand{\Sqh}{\mathrm{Sq}\dhor}
\newcommand{\Sqv}{\mathrm{Sq}\dver}
\newcommand{\Sqvstar}[1]{\mathrm{Sq}\dver^{#1\star}}
\newcommand{\deltav}{\delta\uver}
\newcommand{\lambdav}{\lambda\uver}
\newcommand{\Ph}{P\dhor}
\newcommand{\deltavstar}{\delta^{\mathrm{v}\star}}
\newcommand{\lambdavstar}{\lambda^{\mathrm{v}\star}}

\newcommand{\diag}[1]{|#1|}
\newcommand{\dual}{\mathbf{D}}

\newcommand{\smashprod}{\barwedge}%{\blacktriangle}
\newcommand{\Lsmashprod}{\barwedge^\mathrm{L}}%{\blacktriangle}
\newcommand{\smashcoprod}{\veebar}%{\blacktriangledown}
\newcommand{\Dendo}{R}

\renewcommand{\mapsto}{\longmapsto}

\newcommand{\dupdown}[2]{R_{\smash{#1}}}
\newcommand{\caldup}[1]{\calR_{\smash{#1}}}
\newcommand{\caldupdown}[2]{\calR^{\smash{#1}}_{\smash{#2}}}
\newcommand{\plainD}{R}
\newcommand{\algCat}{\calc}
\newcommand{\trip}[3]{{#1}_{\smash{#2}}^{\smash{#3}}}
\newcommand{\barConstructionMightAbbreviate}{b}

\newcommand{\BKSS}{BKSS}
\newcommand{\CFSS}{CFSS}
\newcommand{\CFSSs}{CFSSs}

%%%%%%%%%%%%%%%%%%% 
%%%%%%%%%%%%%%%%%%%End preamble for pasteover
%%%%%%%%%%%%%%%%%%% 

\newcommand{\vfpb}{\vfil\pagebreak}
\headheight=8pt
\topmargin=0pt
\oddsidemargin=18pt
\evensidemargin=18pt
\textheight=610pt
\ifx\ShrinkPages\undefined\else %if you want to shrink the page while editing, do it here...
  \topmargin=-87pt
  \oddsidemargin=-60pt
  \evensidemargin=-60pt
  \textheight=610pt %610/439
\fi
\textwidth=432pt
\footskip=25pt


\title[The Bousfield-Kan spectral sequence for simplicial algebras]{Unstable operations in the Bousfield-Kan spectral sequence for simplicial commutative $\Ftwo$-algebras}
\author[M.\ Donovan]{Michael Donovan}

%\address{Department of Mathematics \\ Massachusetts Institute of Technology}
%\email{mdono@math.mit.edu}

%\bibliography{papers}
\bibliography{../../Dropbox/logbook/_LOGBOOK/papers}
\newcommand{\SectionOrChapter}[1]{\section{\textbf{#1}}}
\newcommand{\SubsectionOrSection}[1]{\subsection{#1}}

\begin{document}

\begin{Contents Page}
\begin{abstract}
%%%%%%%%%%%%%%%%%%% 
%%%%%%%%%%%%%%%%%%%Begin abstract for pasteover
%%%%%%%%%%%%%%%%%%% 
Write one of these.
%%%%%%%%%%%%%%%%%%% 
%%%%%%%%%%%%%%%%%%%End abstract for pasteover
%%%%%%%%%%%%%%%%%%% 
\end{abstract}
\maketitle
\tableofcontents
\end{Contents Page}




\begin{ThankS}

First and foremost, I would like to thank Haynes Miller for his ongoing support and guidance over the course of this Ph.D. Without doubt I learned more algebraic topology from my meetings with Haynes than from all other sources combined, and at every meeting he contributed materially to my intellectual development. This work owes much to his influence, and to the ideas to be found in his Sullivan Conjecture paper \cite{MillerSullivanConjecture.pdf}.

I would also like to thank Paul Goerss, for visiting to take part in my examination committee, and whose Asterisque volume \cite{MR1089001} has had great influence on my work.
Thanks too to Michael Ching, Bill Dwyer, Benoit Fresse and John Harper, for helpful conversations on various topics appearing in this thesis.

I would like to thank my wonderful classmates, in particular Michael Andrews, Nate Bottman, Saul Glasman, Jiayong Li, Dana Mendelson and Alex Moll, for sharing this experience with me. I would also like to thank Michael Andrews for always being available to answer questions and share ideas.

%Michael Andrews, Nate Bottman, Saul Glasman, Jiayong Li, Dana Mendelson, Alex Moll, Roberto Svaldi, Xuwen Zhu.



Sonia V.\ and Eric M.\ and Bella and Gilad.

Parents.

%Finally, I would like to thank
Dana for being awesome.

\end{ThankS}




%%%%%%%%%%%%%%%%%%% 
%%%%%%%%%%%%%%%%%%%Begin thecontent for pasteover
%%%%%%%%%%%%%%%%%%% 

\begin{Introduction}
\SectionOrChapter{Introduction}
\label{intropart1}

\SubsectionOrSection{The classical Bousfield-Kan spectral sequence}
\label{Classicalstuff}

The primary object of study in this thesis is the Bousfield-Kan spectral sequence (\BKSS) in the Quillen model category $s\algs$ of simplicial non-unital commutative  $\Ftwo$-algebras. This spectral sequence  calculates the homotopy groups of the homology completion $X\hat{\ }$ of $X\in s\algs$, with $E_2$-page given by certain derived functors applied to the Andr\'e-Quillen cohomology groups $H^*_{\algs}X$. \todo{write whole transition to classical}

The homotopy theory $s\algs$ has much in common with the homotopy theory of topological spaces,  and before we introduce our main results, we briefly recall the analogous classical theory. 
The intention of this thesis is to produce an enriched version in the model category $s\algs$ of the classical theory which we now recall. %Some of the results also pass over to the category $s\liealgs$ of simplicial Lie algebras.  

Suppose that $X$ is a pointed connected topological space with $\pi_*X$ finitely generated in each degree. 
The (absolute) \emph{Bousfield-Kan spectral sequence of $X$} (over $\Ftwo$) is a second quadrant spectral sequence
\[\E{}{2}{X}{s}{t}\cong \Ext^s_{\calk}(\overline{H}^*(X;\Ftwo),\overline{H}^*(S^t;\Ftwo))\implies \pi_{t-s}X\hat{\ }_{\!\!\!2},\]
where $X\hat{\ }_{\!\!\!2}$ is the completion of $X$ at the prime 2. Throughout this thesis, we use the notation $\E{}{2}{X}{s}{t}$ rather than the more standard $E_{2}^{s,t}$ for the pages of the spectral sequence.


At least when $X$ is simply connected and $\pi_*X$ is of finite type, one may view this spectral sequence as  a tool for calculating $\pi_*X$, as
$\pi_*(X\hat{\ }_{\!\!\!2})\cong (\pi_*X)\hat{\ }_{\!\!\!2}$
determines the 2-torsion in $\pi_*X$. Under certain hypotheses (satisfied for example when $X=S^n$ for $n\geq1$) the \BKSS\ admits a vanishing line at $E_2$  \cite{MR0266212}, and is thus strongly convergent.

The non-abelian derived functors $\Ext^s$ are calculated in the category 
%of $\calc\cala_2$ of connected unstable coalgebras over the mod 2 Steenrod algebra $\cala_2$ (\cite[\S11 and \S12]{BousKanSSeq.pdf} or \citeBOX[\S2]{BlancRealizations.pdf}).  The unusual notation $\E{}{2}{X}{s}{t}$ is not an unreasonable choice, as $t$ is certainly a homological grading, and such notation will be quite necessary throughout the calculations to follow.
%
%Although it is preferable in general to work with homology coalgebras as above, when $\pi_*X$ is of finite type, this spectral sequence may be rewritten as
%\[\E{}{2}{X}{s}{t}\cong \Ext^s_{\calk}(\overline{H}^*(X;\Ftwo),\overline{H}^*(S^t;\Ftwo))\implies \pi_{t-s}X\hat{\ }_{\!\!\!2},\]
%where 
$\calk$ of non-unital unstable algebras over the Steenrod algebra  (c.f.\  \citeBOX[\S1.4]{MR1282727}). If we write $\vect{+}{}$ for the category of cohomologically graded vector spaces \[W=\textstyle \bigoplus_{n\geq 1} W^n,\] the objects of $\calk$ are graded non-unital $\Ftwo$-algebras
$W\in\vect{+}{}$
equipped with an unstable left action of the Steenrod algebra, i.e.\ maps:
%\[\Sq^i: W^t\to W^{t+i},\]
\begin{gather*}
\Sq^i: W^t\to W^{t+i},\\
\mu: W^t\otimes W^{t'}\to W^{t+t'},
\end{gather*}
satisfying the usual properties --- Adem relations, unstableness relations, and the Cartan formula.
%with the following properties:
%\begin{enumerate}
%\item $\Sq^0$ is the identity;
%\item the squaring operation on $\overline{H}^*X$ equals the \emph{top Steenrod operation}:
%\[\Sq^{t}x=x^{2}\text{\ \ for $x\in \overline{H}^tX$};\]
%\item \label{Sq unstable vanishing intro} if $x\in \overline{H}^tX$, then $\Sq^ix=0$ unless $0\leq i\leq t$;
%\item every Steenrod operation is linear;
%\item the \emph{Cartan formula} holds:  for all $x,y\in   \overline{H}^*X$ and $i\geq0$,
%\[\Sq^i(xy)=\textstyle\sum_{k=0}^{i}(\Sq^kx)(\Sq^{i-k}y);\]
%\item the Adem relations are satisfied.
%\end{enumerate}
We take a moment to set notation, defining the functor of \emph{indecomposables}\todo{meep} 
$Q^{\calk}:\calk\to \vect{+}{}$
by the formula
\[W\overset{Q^{\calk}}{\mapsto}W/\left(\im\Bigl(W\otimes W\overset{\mu}{\rightarrow} W\Bigr)\oplus\textstyle\bigoplus_{i\geq1}\im\Bigl(W\overset{\Sq^i}{\rightarrow} W\Bigr)\right)\in\vect{+}{}.\]
The \BKSS\ $E_2$-page can be  rewritten as the dual left derived functors
\[\E{}{2}{X}{s}{t}\cong H^s_{\calk}(\overline{H}^*(X;\Ftwo))_t:=\dual((\mathbb{L}_sQ^{\calk})(\overline{H}^*(X,\Ftwo) )^t),\]
where we write  $\dual V$ for the linear dual of a vector space $V$, and insist that $\dual$ interchanges homological and cohomological dimensions. We will use notation following this pattern for the rest of the thesis.



One useful idea is to search for operations which act on the \BKSS. Spectral sequence operations are typically used to produce new elements on the $E_2$-page and to compute differentials on those elements. Bousfield and Kan \cite[\S14]{BK_pairings_products.pdf} construct a Lie bracket:
\[\E{}{r}{X}{s}{t} \otimes \E{}{r}{X}{s'}{t'}\to \E{}{r}{X}{s+s'+1}{t+t'}\textup{ \ for $1\leq r\leq \infty$},\]
with the bracket on $E_r$ satisfying  a Liebniz formula and inducing the bracket on $E_{r+1}$.

There are two reasons why one might expect such a Lie algebra structure. %, one reason at $E_2$ and one at $E_\infty$. At $E_2$: 
First, the commutative operad $\CommOperad$ and the Lie operad $\LieOperad$ are Koszul dual, and even though the theory of Koszul homology is complicated by the non-zero characteristic, there is an action of $\LieOperad$ on the derived functors calculating $E_2$. %, which might then extend to the higher pages. At $E_\infty$: 
Next, there is a graded Lie algebra structure on homotopy groups given by the Whitehead bracket \cite{MR0004123}:
\[[\,,]:\pi_nX\hat{\ }_{\!\!\!2}\otimes \pi_{n'}X\hat{\ }_{\!\!\!2}\to \pi_{n+n'-1}X\hat{\ }_{\!\!\!2},\]
and one may ask whether or not this action preserves the filtration, in which case it would define a Lie algebra structure on $E_\infty$.
Bousfield and Kan answer this question in the affirmative by proving that the bracket at $E_\infty$ is compatible with the Whitehead bracket, and they also show that the pairing given at $E_2$ has the correct homological description.

This appears to be as far as it is possible to pursue this strategy, as at both $E_2$ and $E_\infty$ we lose hope of finding structure that can be readily described. %We lose hope at $E_\infty$, since w
We do not expect to extract further structure on $E_2$ using the Steenrod algebra action in $\calK$, or at least not any that can be described so explicitly. The Steenrod algebra $\cala_2$ suffers from the inhomogeneity $\Sq^0=1$. Were it a homogeneous Koszul algebra (in the sense of \cite{PriddyKoszul.pdf}), then its Koszul dual would at very least act on $\Ext_{\cala_2}(\Ftwo,M)$ for an $\cala_2$-module $M$, but even this is not the case. 
There is no particular reason to think that the situation should be any better for the non-abelian derived functors defining $E_2$.
Moreover, we simply do not understand the natural operations that exist on $\pi_*$ in enough detail to expect to see uniform structure appearing on $E_\infty$. After all, by the Hilton-Milnor theorem \citeBOX[\S4]{Neisendorfer.pdf}, all natural operations on the homotopy groups of pointed spaces are composites of the Whitehead bracket and unary operations, and a natural  homotopy operation $\pi_n X\to \pi_m X$ is equivalent to an element of $\pi_m S^n$.



Before we break from our extended analogy, we will discuss the considerable task of calculating the $E_2$-page of this classical \BKSS. Performing this calculation is at least as difficult as the calculation of the $E_2$-page for the classical (stable) Adams spectral sequence, which appears to be rather difficult. There is, however, the following method due to Miller \cite{MillerSullivanConjecture.pdf} for extracting information about the derived functors $H^*_{\calk}$. There is a factorization of $Q^{\calk}$ into 
\[\calk\overset{Q^{\algs}}{\to} \Sigma\calu \overset{Q^{\Sigma\calu}}{\to}\vect{+}{},\]
where $\Sigma\calu$ is the algebraic category whose objects are vector spaces $V\in\vect{+}{}$ equipped with a left action of $\cala_2$ such that $\Sq^i:V^n\to V^{n+i}$ is zero unless $0\leq i <n$.
This modified unstableness condition is neccessary in order that $Q^\algs$ satisfies an acyclicity condition, so that for $W\in\calk$ there is a composite functor spectral sequence
\[\E{\textup{cf}}{2}{W}{s_2,s_1}{t}=H^{s_2}_{\Sigma\calu}(H_*^{\algs}(W))^{s_1}_t\implies H^{s_1+s_2}_\calk (W)_t.\]
This spectral sequence was an integral part of Miller's proof of the Sullivan conjecture. The functor $H_*^{\algs}$ appearing in the above description is the \emph{Andr\'e-Quillen  homology} functor on $s\algs$. 

\SubsectionOrSection{The various categories $s\calc$}
In this thesis will will use quite a number of categories of universal algebras, such as the category $\algs$ of non-unital commutative $\Ftwo$-algebras, or the category $\liealgs$ of Lie algebras over $\Ftwo$. While we introduce certain general notions we will write $\calc$ for any one of these categories.

For any category $\calc $ of universal algebras, the category $s\calc$ of simplicial objects in $\calc$ is a Quillen model category \cite{QuillenHomAlg.pdf}. These model categories have much in common with the category of topological spaces. For example, an  object $X\in s\calc$ posesses homotopy groups $\pi_*X$ and homology groups $H_*^{\calc}X$. We use the  pragmatic notion of homology that appears in the spectral sequences that appear in this context, and it does not always coincide with Quillen's notion of homology \emph{derived abelianization}. The cohomology groups $H^*_{\calc}X$ are defined to be the linear duals of the homology groups.

%The object $X$ also posesses homotopy groups $\pi_*X$ and cohomology groups $H^*_{\calc}X$. The homotopy groups are those of the underlying simplicial vector space, while that cohomology groups are defined to be linear dual to the homology. 
In \S\ref{Pi-algebras and cohomology algebras}, we recall the  definition of  spheres and Eilenberg-Mac Lane objects in $s\calc$. These play the same role in $s\calc$ as their namesakes in the category of pointed topological spaces, which is to represent the homotopy and cohomology functors on the homotopy category of $s\calc$, respectively.
We also present a unified treatment of homotopy and cohomology operations (and of homology co-operations) for such categories. 

In \S\ref{Constructing homotopy operations} and \S\ref{Constructing cohomology operations} we present a number of existing examples of homotopy and cohomology operations in a common framework, with the construction of cohomology operation following Goerss' method from \cite{MR1089001}. In particular, it will be useful for us to understand the well-known cohomology operations for simplicial Lie algebras in this same framework, and in Appendix \ref{appendix on Lie coh ops} \todo{yeah say it}.




In \S\ref{Bousfield-Kan spectral sequence}, we recall Radulescu-Banu's \cite{Radulescu-Banu.pdf} cosimplicial resolution of $X\in s\algs$, which we denote by $\calx\in cs\algs$.
The resolution $\calx$ is suitable for the construction of a \BKSS\ for $X$. This construction is rather more difficult than that of Bousfield and Kan's $\Ftwo$-resolution, as the na\"{\i}ve monadic cobar construction in $s\calc$ is not homotopically correct.
The totalization of $\calx$ is the  \emph{homology completion $X\hat{\ }$ of $X$}, and the (absolute) \BKSS\ is the spectral sequence associated with the totalization tower. In  \S\ref{Idnt E1 E2} we perform the homotopical algebra needed to identify the $E_1$ and $E_2$-pages arising from Radulescu-Banu's resolution.
%
%
%Quillen \cite{QuillenHomAlg.pdf} formulated the notion of homotopy theory in a variety of algebraic contexts, in particular, there is a homotopy theory of any given type of universal algebra.  
%The homotopy theory of non-unital $\Ftwo$-algebras is an important and interesting example,  of interest both in topology and commutative algebra, and yet, many elements of this homotopy theory are not as well understood. 






% We will produce not one but an infinite sequence of composite functor spectral sequences, each calculating the $E_2$-page of the last, which together may be used to perform calculations of the Bousfield-Kan $E_2$-page. We equip each of the composite functor spectral sequences with, again, three varieties of operations, necessary in order to perform these calculations. We use these composite functor spectral sequences to calculate the $E_2$-page of the Bousfield-Kan spectral sequence in the most important case, that of a `sphere' in $s\algs$, and produce a May-Koszul spectral sequence calculating the $E_2$-page from the $E_2 $-pages for spheres. 
%
%give a calculation of the $E_2$-page of the spectral sequence
%
%surrounding the Bousfield-Kan spectral sequence, and to perform calculations in certain examples.  More specifically, we will 

\SubsectionOrSection{The Bousfield-Kan spectral sequence in $s\algs$}
At this point we depart from generalities, turning to the homotopy theory of simplicial non-unital commutative algebras in earnest.
% The category $s\algs$ is a one of the first examples of a Quillen model category \cite{QuillenHomAlg.pdf}, as is $s\calc$ for any category $\calc$ of universal algebras, and we will work in this model category for various categories $\calc$.
%In \S\ref{Conventions and notation} we give the background on the model category $s\calc$ for $\calc$ a category of universal algebras, and choose our conventions for the notation for spectral sequences we will use.
We will restrict to the \emph{connected} objects $X\in s\algs$, which simplifies various aspects of our analysis.
%The spectral sequence will be of the form
%\[\E{}{2}{\calX}{s}{t} \implies \pi_{t-s}X\hat{\ },\]
%where $X\hat{\ }$ is the \emph{homology completion} of $X$. 
As in the classical case, one must know how the homotopy groups of the homology completion $X\hat{\ }$ determine those of $X$. We will demonstrate (Theorem \ref{completenesstheorem}) that $X\hat{\ }$ is equivalent to $X$ as long as $X$ is connected. Moreover, we will prove in Theorem \ref{vanishing line prop} that the \BKSS\ admits a vanishing line from $E_2 $ in this case, and thus strongly converges to the homotopy of $X$.

As forecast by the discussion in \S\ref{Classicalstuff}, it will help to know a little about the natural operations on the homotopy of simplicial $\Ftwo$-algebras in advance. Fortunately, we have 
the explicit description of homotopy operations which is lacking in the category of pointed spaces, since they have been completely calculated by Dwyer \cite{DwyerHtpyOpsSimpComAlg.pdf} (and were studied earlier by Bousfield \cite{BousOpnsDerFun.pdf,BousHomogFunctors.pdf} and Cartan \cite{CartanDivSquares}). In summary, $\pi_*X$ supports operations
\begin{gather*}
\delta_i:\pi_nX\to \pi_{n+i}X,\textup{ defined when }2\leq i\leq n,\\
\mu:\pi_nX\otimes \pi_{n'}X\to \pi_{n+n'}X,
\end{gather*}
with $\mu$ a graded non-unital commutative algebra product, and %the top with left operations
%\[\delta_i:\pi_nX\to \pi_{n+i}X,\textup{ defined when }2\leq i\leq n,\]
the $\delta_i$ satisfying various compatibilities which we discuss in detail in \S\ref{Homotopy operations for simplicial commutative algebras}. In fact, these $\delta$-operations satisfy a $\delta$-Adem relation \emph{which is homogeneous}, and there is a corresponding unital associative algebra $\Delta$. Note that $\pi_*X$ is \emph{not} a left module over the algebra $\Delta$, because the operations are not defined in every dimension. This situation can not be remedied simply be defining the missing operations to be zero, as doing so is incompatible with the Adem relations on homotopy. Instead, we must adopt language for such situations, saying that $\Delta$ has a partially defined unstable left action on $\pi_*X$. In general, unstable homotopy operations will be  partially defined, whereas unstable cohomolgy operations will be everywhere defined but vanish in certain ranges.

Goerss \cite{MR1089001} described the analogue for $s\algs$ of the category $\calk$, and  all of the natural operations on the Andr\'e-Quillen cohomology $H^*_\algs X$ of $X\in s\algs$ are generated by:
\begin{gather*}
P^i:H^n_{\algs}X\to H_{\algs}^{n+i+1}X;\\
[\,,]:H_{\algs}^nX\otimes H_{\algs}^mX\to H_{\algs}^{n+m+1}X;\\
\beta:H^0_{\algs}X\to H_{\algs}^{1}X.
\end{gather*}
As we restrict to connect objects of $s\algs$, the operation $\beta$ can be ignored.
These operations satisfy various compatibilities which we recount in detail in \S\ref{The example of simplicial commutative F2-algebras}, and we will denote by $\calw(0)$ the category whose objects are vector spaces $W\in\vect{+}{}$ equipped with the $P^i$-operations and the bracket.
The bracket satisfies the Jacobi identity but falls just short of being a Lie algebra pairing as $[x,x]$ is not always zero. The $P^i$ satisfy a $P$-Adem relation \emph{that is homogeneous}. The evident unital associative algebra $P$ acting on $H^{*}_{\algs}$ is a homogeneous Koszul algebra, the \emph{$P$-algebra} with Koszul dual the algebra $\Delta$, and indeed, this is how it was originally described by Goerss. In \S\ref{Idnt E1 E2}, we identify the $E_2$-page (for connected $X\in s\algs$ with $\pi_*X$ of finite type) as the  non-abelian derived functors
\[\E{}{2}{\calx}{s}{t}\cong H^{s}_{\calw(0)}(H^{*}_{\algs}X)_t.\]
This description begets a laundry list of operations that we expect to see on $E_2$. The cohomology of a Lie algebra enjoys a commutative product and an action of the homogeneous Steenrod algebra $\LieSteen:=E_0\cala_2$, due to Priddy \cite{PriddySimplicialLie.pdf}. We will discuss these operations in detail in \S\ref{section: Cohomology operations for simplicial (restricted) Lie algebras} and Appendix \ref{The partially restricted universal enveloping algebra}. %This structure is visible at $E_2$ 
We construct in Proposition \ref{Wn Halg omnibus} the corresponding natural `horizontal' operations on $E_2$:
\begin{gather*}
\Sqh^j:(H^s_{\calw(0)}X)_t\to (H^{s+j}_{\calw(0)}X)_{2t+1};\\
\mu:(H^s_{\calw(0)}X)_t\otimes (H^{s'}_{\calw(0)}X)_{t'}\to (H^{s+s'+1}_{\calw(0)}X)_{t+t'+1}.
\end{gather*}
Moreover, we construct in Proposition \ref{operations on goerss homology} natural vertical operations constituting a  (partially defined) action of the Koszul dual $\Delta$ of the $P$-algebra:
\[\deltav_i:(H^s_{\calw(0)}X)_t\to (H^{s+1}_{\calw(0)}X)_{t+i+1}\textup{ \ defined for $2\leq i <t$}.\]%\max\{t-s,t-1\}$}.\] 
These operations satisfy various compatibilities (c.f.\ \S\ref{Relations between the horizontal and vertical operations} and propositions \ref{operations on goerss homology} and \ref{Wn Halg omnibus}).

Although the product and $\delta$-operations on $E_2$ look encouraging, there is the following issue: if $x\in \E{}{2}{\calx}{s}{t}$ is  a permanent cycle detecting  a class $\overline{x}\in \pi_*X$, then at least when $s\geq2$ there are more operations $\deltav_2x,\ldots,\deltav_{t-1}x$ defined on $E_2$ than there are operations $\delta_2\overline{x},\ldots,\delta_{t-s}\overline{x}$ defined on homotopy.  Moreover, the Steenrod operations at $E_2$ have no couterpart in homotopy.

This situation is quite reminiscent of that described by Dwyer \cite{DwyerHigherDividedSquares.pdf}, who works in the spectral sequence of a cosimplicial simplicial coalgebra (such as the Eilenberg-Moore spectral sequence). In such a spectral sequence, one expects to find Steenrod operations at $E_2$ but finds \textbf{too many}. Dwyer constructs $\delta$-operations and differentials mapping the excess Steenrod operations to the $\delta$ operations. In this way, the excess Steenrod operations fail to be defined at $E_\infty$, and the $\delta$-operations become zero by $E_\infty$, an excellent resolution to this problem.

Unfortunately, we cannot use Dwyer's operations. Indeed,  although the linear dual of a cosimplicial simplicial coalgebra is a cosimplicial simplicial algebra (of which the resolution $\calx$ is an example), the choice of filtration direction is transposed. Instead, we perform analogous constructions in the dual setting, and describe in \S\ref{The case of a cosimplicial simplicial commutative algebra} a theory of operations on the spectral sequence of a cosimplicial simplicial $\F_2$-algebra, which may be of independent interest. While defining these operations is a good first step, they are not yet what we require, as it happens that they can be lifted one filtration higher  when we are working in the \BKSS. In \S\ref{second quadrant homotopy sseq operations} and \S\ref{Operations on the Bousfield-Kan spectral sequence}, we explain how to construct operations
\begin{gather*}
\deltav_i:\Edownup{r}{}{\calx}{s}{t} \to \Edownup{r}{}{\calx}{s+1}{t+i+1},
\\
\Sqh^j:\Edownup{r}{}{\calx}{s}{t}   \to
\Edownup{r}{}{\calx}{s+j}{2t+1},\\
\mu:\Edownup{r}{}{\calx}{s}{t}\otimes \Edownup{r}{}{\calx}{s'}{t'}\to
\Edownup{r}{}{\calx}{s+s'+1}{t+t'+1},
\end{gather*}
with the $\deltav_i$ potentially multi-valued functions,  defined when $2\leq i\leq \max\{n,t-(r-1)\}$, and single-valued whenever $i\leq\min\{n+1,t+1-2(r-1)\}$, and the $\Sqh^j$ potentially multi-valued functions with indeterminacy vanishing by $E_{2r-2}$, and which equal zero unless $\min\{t,r\}< j\leq s+1$. All of the functions that are  defined on $E_2$ are single valued, and indeed, they coincide with the operations defined on $H^*_{\calw(0)}$, as we show in Proposition \ref{adams operations are right for comm}.

For $x\in \E{}{r}{\calx}{s}{t}$ such that  $\deltav_ix $ is defined:
\[d_r\deltav_i(x)+ \deltav_i(d_rx)=\begin{cases}
\Sqh^{t-i+2}(x),&\textup{if $i>t-s$ and $r =t-i+1$};\\
\mu(x\otimes d_rx),&\textup{if $i=t-s$,  $s=0$ and $r\geq2$};\\
%\Sqh^{t-i+1}(x),&\textup{if $t-s<i =t-(r-1)$}\\
0,&\textup{otherwise.}
\end{cases}\]
This is Corollary \ref{cor on delta external composed with lift}. In particular, if $i>t-s$ and $x\in\E{}{t-i+1}{\calx}{s}{t}$ survives to $\E{}{t-i+2}{\calx}{s}{t}$, then $d_{t-i+1}$ maps $\deltav_ix$ to $\Sqv^{t-i+2}x$. These formulae explain how the $\Sqv$ serve to absorb differentials supported by the excess $\deltav$.

\SubsectionOrSection{The first composite functor spectral sequence}
\label{The first composite functor spectral sequence}
Now that we have a theory of the operations available on the \BKSS\ in $s\algs$, we turn to the question of calculating it. If we hope to imitate Miller's use of a composite functor spectral sequence (\CFSS), %we have only one option for the factorization of $Q^{\calw(0)}$.: there are functors
using the factorization
\[Q^{\calw(0)}=Q^{\call(0)}\circ Q^{\calu(0)}  : \Bigl(\calw(0)\overset{Q^{\calu(0)}}{\to}\call(0)\overset{Q^{\call(0)}}{\to}\vect{+}{}\Bigr),\]
where $\call(0)$ is the category whose objects are graded vector spaces $W\in\vect{+}{}$ which are Lie algebras under a bracket which shifts gradings,
\[W^{t}\otimes W^{t'}\to W^{t+t'+1}.\]
%\subsubsection{Abelian derived functors}
We write $\calu(0)$ for the category whose objects are vector spaces $V\in \vect{+}{}$ equipped with an unstable action of the $P$-algebra given by operations
\[P:V^t\to V^{t+i+1}\]
which are zero unless $2\leq i\leq t$. The functor $Q^{\calu(0)}$ is defined for $W\in\calw(0)$ by
\[W\mapsto W/\textstyle\bigoplus_{i\geq2}\im(W\overset{P^i}{\to}W).\]

As the category $\call(0)$ is not an abelian category, it is a more technical task to form a \CFSS, and we use the method of Blanc and Stover \cite{Blanc_Stover-Groth_SS.pdf}. The key idea in their presentation is that the derived functors
$H_*^{\calu(0)}:=\mathbb{L}_*Q^{\calu(0)}$
take values in the category  $\calw(1)$ of $\call(0)$-$\Pi$-algebras, as they are calculated as the homotopy of an object of $s\call(0)$.
The first \CFSS\ takes the following form  for $W\in\calw(0)$:
\[\E{\textup{cf}}{2}{W}{s_2,s_1}{t}=H^*_{\calw(1)}(H_*^{\calu(0)}W)^{s_2,s_1}_t\implies (H^*_{\calw(0)} W)_t^{s_1+s_2}.\]
We will now unpack this somewhat dense expression, and explain how various unstable operations defined at the $E_2$-page and the target interact with the spectral sequence.

Objects of the category $\calw(1)$ are certain bigraded Lie algebras with a certain partially defined right action of the $\Lambda$-algebra (c.f.\ \S\ref{Higher composite functor spectral sequences}).
In \S\ref{section on structure on homology of koszul cx} we calculate the structure of $H_*^{\calu(0)}W$ as an object of $\calw(1)$ by explicit chain-level computation,
after defining in \S\ref{The Koszul complex and co-Koszul complex} an unstable version of Priddy's Koszul resolution \cite{PriddyKoszul.pdf} for the functors $H_*^{\calu(0)}$.

The linear duals $H^*_{\calU(0)}W$ admit an unstable partially defined left $\Delta$-algebra action, since the algebras $\Delta$ and $P$ are Koszul dual, and by Proposition \ref{edgehomproposition} there is a commuting diagram (for $2\leq i<t$): %this commutes In fact, we can form the composite
\[\xymatrix@R=4mm@C=17mm{
(H^*_{\calw(0)}W)^{s}_t\ar[r]^-{\textup{edge hom}}\ar[d]_-{\deltav_i}
&%r1c1
\E{\textup{cf}}{2}{W}{0,s}{t}\ar@{ >->}[r]
&%r1c2
(H^*_{\calU(0)}W)^{s}_t\ar[d]^-{\deltav_i}
\\
(H^*_{\calw(0)}W)^{s+1}_{t+i+1}\ar[r]^-{\textup{edge hom}}
&%r1c1
\E{\textup{cf}}{2}{W}{0,s+1}{t+i+1}\ar@{ >->}[r]
&%r1c2
(H^*_{\calU(0)}W)^{s+1}_{t+i+1}
}\]
In this  sense the $\deltav_i$-operations on the \BKSS\ $E_2$-page are compatible with the \CFSS.

The \BKSS\ $E_2$-page also supports products and horizontal Steenrod operations, and we should attempt to identify them in the \CFSS. The functor $H^*_{\calw(1)}$ may also be viewed as a Lie algebra cohomology functor, so that we expect horizontal Steenrod operations and products to appear in $\E{\textup{cf}}{2}{W}{s_2,s_1}{t}$. We use a new definition of these operations that fits into the framework set out in \S\ref{Constructing cohomology operations} (defering to Appendix \ref{appendix on Lie coh ops} the work of showing that these operations coincide with those constructed by Priddy \cite{PriddySimplicialLie.pdf}.)

Moreover, just as we expected $\deltav_i$ operations on $H^*_{\calw(0)}$, we expect a `vertical' left action of the homogeneous Steenrod algebra on $H^*_{\calw(1)}$, as it is Koszul dual to the $\Lambda$-algebra. Indeed, we construct in Proposition \ref{Wn Halg omnibus} such operations on the derived functors $H^*_{\calw(1)}$. Moreover,
Proposition \ref{vertical steenrod operations prop} applies to $H^*_{\calw(1)}$ just as it applies to $H^*_{\calw(0)}$, yielding horizontal Steenrod operations and products, so that ultimately we obtain operations
\begin{gather*}
\Sqv^i:\E{\textup{cf}}{2}{W}{s_{2},s_1}t\to \E{\textup{cf}}{2}{W}{s_2+1,s_1+i-1}{2t+1},\\
\Sqh^j:\E{\textup{cf}}{2}{W}{s_2,s_1}{t}\to \E{\textup{cf}}{2}{W}{s_{2}+j,2s_1}{2t+1},\\
\mu:\E{\textup{cf}}{2}{W}{s_2,s_1}{t}\otimes \E{\textup{cf}}{2}{W}{p_2,p_1}{q}\to \E{\textup{cf}}{2}{W}{s_2+p_2+1,s_1+p_1}{t+q+1},
\end{gather*}
with both the horizontal and vertical Steenrod operations satifying their own unstableness conditions.

Now suppose that $x\in \E{\textup{cf}}{2}{W}{s_{2},s_1}{t}$ is a permanent cycle detecting an element $\overline{x}\in (H^*_{\calw(0)}W)^{s_2+s_1}_t$. The $s_2{+}s_1{-}1$ operations $\Sqh^3\overline{x},\ldots,\Sqh^{s_2+s_1+1}\overline{x}$ are the potentially non-zero Steenrod operations on $\overline{x}$. The $s_1{-}2$ vertical operations $\Sqv^3x,\ldots,\Sqv^{s_1}x$ and the $s_2{+}1$ horizontal operations $\Sqh^1x,\ldots,\Sqh^{s_2+1}x$ are the potentially non-zero  Steenrod operations on $x$.
This is quite reminiscent of Singer's framework \cite{MR2245560} (c.f.\ \S\ref{singer ext ops sect}), and in \S\ref{Application to composite functor spectral sequences} we use Singer's methods  to extend the operations on $\E{\textup{cf}}{2}{W}{}{}$ to the entire \CFSS. The upshot is that if $x\in \E{\textup{cf}}{2}{W}{s_{2},s_1}{t}$ is a permanent cycle, then so are all of the above mentioned $\Sqv^ix$ and $\Sqh^jx$, and moreover,
\[\Sqv^ix\textup{ detects }\Sqh^i\overline{x}\ (3\leq i\leq s_1)\textup{ and }\Sqh^jx\textup{ detects }\Sqh^{s_1+i}\overline{x}\ (1\leq j\leq s_2+1).\]
That is, the horizontal and vertical Steenrod operations \emph{combined} account for the horizontal Steenrod operations on the target. We examine how this plays out for admissible sequences of Steenrod operations in Theorem \ref{thm on compressing seqs of steenrod ops}.\todo{it}



\SubsectionOrSection{Higher composite functor spectral sequences}\label{Higher composite functor spectral sequences}
We have constructed a comprehensive theory of the operations in the first \CFSS, but it may still be the case that the $H^*_{\calw(1)}$ is as difficult to calculate as $H^*_{\calw(0)}$, which would mean that the \CFSS\ is of little use for the calculation of $H^*_{\calw(0)}$. Rather than being discouraged, we will turn this similarity to our advantage by iterating our approach. %After all, the categories $\calw(0)$ and $\calW(1)$ are similar in structure: each of these categories is (close enough to) a category of Lie algebras with unstable partially defined operations of a homogeneous Koszul algebra.
 In \S\ref{homotopy operations for PRLs} we extend the constructions summarized in \S\ref{The first composite functor spectral sequence}, defining algebraic categories $\calw(n)$ and $\call(n)$ for $n\geq1$ such that $\calw(n)$ is the category of $\call(n-1)$-$\Pi$-algebras and there are factorizations 
\[Q^{\calw(n)}=Q^{\call(n)}\circ Q^{\calu(n)}:\Bigl(\calw(n)\overset{Q^{\calu(n)}}{\to}\call(n)\overset{Q^{\call(n)}}{\to}\vect{+}{n}\Bigr).\]
%For each $n\geq1$, $\call(n)$ is a category of $\vect{+}{n}$-graded partially restricted Lie algebras and $\calw(n)$ is a category of  $\vect{+}{n}$-graded partially restricted Lie algebras with certain unstable $\lambda$-operations. These definitions must be made carfully. Indeed, We write $\calu(n)$ for the category whose objects are the . The definition of the functor $\calU$
There are \CFSSs\ for $W\in\calw(n)$:
\[\E{\textup{cf}}{2}{W}{s_{n+2},\ldots,s_1}{t}=H^*_{\calw(n+1)}(H_*^{\calu(n)}W)^{s_{n+2},s_{n+1},\ldots,s_1}_t\implies (H^*_{\calw(n)} W)_t^{s_{n+2}+s_{n+1},s_n,\ldots,s_1},\]
and we equip each of these spectral sequences with a theory of operations which generalize those discussed in \S\ref{The first composite functor spectral sequence}. \todo{say koszul changes.}

At this point it is useful to summarize the definitions. For $n\geq1$,
let $\calU(n)$ denote the category whose objects are vector spaces $V\in \vect{+}{n}$ equipped with  \emph{linear} right $\lambda$-operations
\begin{equation}\label{lambdaopshighn}
(\DASH)\lambda_i:V_{s_{n},\ldots,s_1}^t\to V_{s_{n}+i,2s_{n-1},\ldots,2s_1}^{2t+1}
\end{equation}
defined whenever $0\leq i< s_{n+1}$ and not all of $i,s_{n},\ldots,s_{1}$ are zero. Let $\call(n)$ denote the category whose objects are $V\in \vect{+}{n}$ equipped with a (typically non-linear) $\lambda$-operation as in (\ref{lambdaopshighn}) defined whenever $i= s_{n+1}$ and not all of $i,s_{n},\ldots,s_{1}$ are zero, which acts as a partial restriction for a Lie algebra bracket
\[[\,,]:V^{t}_{s_n,\ldots,s_1}\otimes V^{t'}_{s'_n,\ldots,s'_1}\to V^{t+t'+1}_{s_n+s'_n,\ldots,s_1+s'_1}.\]
Finally, let $\calw(n)$ be the category whose objects are simulteneously objects of $\calU(n)$ and $\call(n)$ subject to certain compatibilities.

The functor $H_*^{\calu(n)}W$ may be calculated by an unstable Koszul resolution, and both its linear dual $H^*_{\calu(n)}W$ and the functor $H^*_{\calw(n)}W$ are naturally objects of $\calMv(n+1)$, the category whose objects are graded vector spaces $M\in\vect{n+1}{+}$ with an unstable left action of the Steenrod algebra, operations%:. Specifically
%Write $\calMv(n+1)$ for the algebraic category whose objects are vector spaces $M\in\vect{n+1}{+}$ with left Steenrod operations
\[\Sqv^i:M^{s_{n+1},\ldots,s_1}_t\to M^{s_{n+1}+1,s_n+i-1,2s_{n-1},\ldots,2s_1}_{2t+1},\]
which are zero except when $1\leq i \leq s_n$ and  $i-1,s_{n-1},\ldots,s_1$ are not all zero. This structure is derived in \S\ref{section: vertical Koszul operations n positive}, using the Koszul duality between the $\Lambda$-algebra and the homogeneous Steenrod algebra. This differs from the analogous constructions for $\calw(0)$- and $\calu(0)$-cohomology, in that $H^*_{\calw(0)}$ supports one fewer vertical $\delta$-operation than $H^*_{\calu(0)}$.

On the other hand, as in the $n=0$ case, $H^*_{\calw(n)}$ is an example of (partially restricted) Lie algebra cohomology, so that  `horizontal' Steenrod operations and products  appear. In \S\ref{Horizontal Steenrod operations and a product for HWn} we define these operations:
\begin{gather*}
\Sqh^j:(H^*_{\calw(n)}X)_t^{s_{n+1},\ldots,s_1}\to (H^*_{\calw(n)}X)_{2t+1}^{s_{n+1}+j,2s_{n},\ldots,2s_1},\\
\mu:(H^*_{\calw(n)}X)_t^{s_{n+1},\ldots,s_1}\otimes (H^*_{\calw(n)}X)_q^{p_{n+1},\ldots,p_1}\to (H^*_{\calw(n)}X)_{t+q+1}^{s_{n+1}+p_{n+1}+1,s_{n}+p_{n},\ldots,s_1+p_1},
\end{gather*}
so that $\calw(n)$-cohomology is also a certain type of unstable algebra over the homogeneous Steenrod algebra, %\emph{another} action of the homogeneous Steenrod algebra on $\calw(n)$-cohomology, 
with the horizontal Steenrod action. We write $\calMh(n+1)$ for the resulting category of $\vect{n+1}{+}$-graded unstable algebras over the homogeneous Steenrod algebra.


We identify in \S\ref{Relations between the horizontal and vertical operations} the relations betwen the $\calMv(n+1)$- and $\calMh(n+1)$-operations, which leads to the defintion of an algebraic category $\calMhv(n+1)$ in which $\calw(n)$-cohomology takes values for $n\geq1$.

Consider again the \CFSS\ for $W\in\calw(n)$:
\[\E{\textup{cf}}{2}{W}{s_{n+2},\ldots,s_1}{t}=H^*_{\calw(n+1)}(H_*^{\calu(n)}W)^{s_{n+2},s_{n+1},\ldots,s_1}_t\implies (H^*_{\calw(n)} W)_t^{s_{n+2}+s_{n+1},s_n,\ldots,s_1}.\]
The target is an object of $\calMhv(n+1)$, while the $E_2$-page is an object of $\calMhv(n+2)$. %The compatibility of the spectral sequence with the $\calMv(n+1)$-structure on the target is again via the edge homomorphism. 
We prove in Proposition \ref{edgehomproposition} that there is a commuting diagram relating the $\calMv(n+1)$-structures on $H^*_{\calw(n)}W$ and $H^*_{\calU(n)}W$ under the edge homomorphism.
%\[\xymatrix@R=4mm@C=17mm{
%(H^*_{\calw(n)}W)\ar[r]^-{\textup{edge hom}}\ar[d]_-{\Sqv^i}
%&%r1c1
%\E{\textup{cf}}{2}{W}{\textbf{0}}{}\ar@{ >->}[r]
%&%r1c2
%(H^*_{\calU(n)}W)\ar[d]^-{\Sqv^i}
%\\
%(H^*_{\calw(n)}W)\ar[r]^-{\textup{edge hom}}
%&%r1c1
%\E{\textup{cf}}{2}{W}{\textbf{0}}{}\ar@{ >->}[r]
%&%r1c2
%(H^*_{\calU(n)}W).
%}\]
As in the $n=0$ case, after extending the $\calMhv(n+2)$-structure on $E_2$ to the whole spectral sequence, this structure converges to the $\calMh(n+1)$-structure on the target.

\SubsectionOrSection{Computing with the composite functor spectral sequences}
So far, we have not explained how the \CFSSs\ may be used for calculation. First, we make the following simple observation. Suppose we wish to calculate the group $(H^*_{\calw(n)} W)_t^{s_{n+1},s_n,\ldots,s_1}$ for a given choice of indices. The part of the $E_2$-page that contributes to this particular group is 
\[\bigoplus_{s'_{n+2}+s'_{n+1}=s_{n+1}}H^*_{\calw(n+1)}(H_*^{\calu(n)}W)^{s'_{n+2},s'_{n+1},s_n,\ldots,s_1}_{t}\]
Now in each summand, either $s'_{n+1}=0$ or $s'_{n+2}<s_{n+1}$. Except for the challenges of understanding the differentials and hidden extensions of algebraic structure, it suffices then to calculate the groups
\[(H^{*}_{\calW(n+k)}H_*^{\calu(n+k-1)}\cdots H_*^{\calu(n)}W)^{s'_{n+k+1},\ldots,s'_{n+1},s_n,\ldots,s_1}_{t}\]
for all $k\geq1$ and for all indices $s'_{n+k+1}+\cdots+s'_{n+1}=s_{n+1}$ satisfying either $s'_{n+k+1}=0$ or $s'_{n+k}=0$. 
It is easy to calculate these groups in either case, as long as we understand the derived functors
\[H_*^{\calu(n+k-1)}\cdots H_*^{\calu(n)}W\]
 as objects of $\calw(n+k)$. We undertake these calculations in \S\ref{section on structure on homology of koszul cx}. When $s'_{n+k+1}=0$ there are no derived functors being taken, and when $s'_{n+k}=0$ the derived functors may be calculated simply as the cohomology of a (constant, not simplicial) partially restricted Lie algebra.

With this computation in mind we define the Chevalley-Eilenberg-May complex of a partially restricted Lie algebra in \S\ref{The Chevalley-Eilenberg-May complex}. This complex interpolates between the Chevalley-Eilenberg complex for the homology of Lie algebras and May's $\overline{X}$ complex \cite{MayRestLie.pdf} for the homology of restricted Lie algebras.
%\[H^*_{\calw(n+k)}(H_*^{\calu(n+k-1)}\cdots H_*^{\calu(n)}W)^{s'_{n+k+1},\ldots,s_1}_{t}\cong (Q^{\calw(n+k)}H_*^{\calu(n+k-1)}\cdots H_*^{\calu(n)}W)^{0,s'_{n+k+1},\ldots,s_1}_{t}\]
%and
%\[H^*_{\calw(n+k)}(H_*^{\calu(n+k-1)}\cdots H_*^{\calu(n)}W)^{s'_{n+k+1},\ldots,s_1}_{t}\cong (Q^{\calw(n+k)}H_*^{\calu(n+k-1)}\cdots H_*^{\calu(n)}W)^{0,s'_{n+k+1},\ldots,s_1}_{t}\]
%as long as we understand $H_*^{\calu(n)}W$.

This method is employed to prove theorems \ref{thm on collapsing of most sseqs} and \ref{Koszul-dual Hilton-Milnor theorem}, which together imply Corollary \ref{yeah it's Wn-coh-algs}, that $\calMhv(n+1)$ is the category of $\calw(n)$-cohomology algebras for $n\geq1$. That is, the $\calMhv(n+1)$-structure is \emph{all} of the natural structure on $\calW(n)$-cohomology for $n\geq1$.

Finally, we are able to use all of this structure together to calculate, at least as a vector space, the \BKSS\ $E_2$-page for the commutative algebra $T$-sphere $\mathbb{S}^{\algs}_T$ whenever $T\geq1$. That is, we calculate the derived functors $H^*_{\calw(0)}W$, where $W=H^*_{\algs}\mathbb{S}^{\algs}_T$ is a one-dimensional trivial object concentrated in dimension $T\geq1$.

Finally, we derive in \S\ref{The quadratic filtration section} a convergent spectral sequence which calculates the $E_2$-page for any connected $X\in s\algs$ of finite type, which we name the May-Koszul spectral sequence. Its $E_1$-page may be described in terms of the \BKSS\ $E_2$-pages of the spheres (using Theorem \ref{Koszul-dual Hilton-Milnor theorem}), and information about the $E_2$-operations in the \BKSS\ for a sphere passes over to information about the general \BKSS\ $E_2$-page via the May-Koszul spectral sequence.


\SubsectionOrSection{The Bousfield-Kan spectral sequence for $\mathbb{S}^{\algs}_T$}
In \S\ref{An alternative Bousfield-Kan E1}, we present a small model for the \BKSS\ $E_1$-page for a commutative algebra sphere. 
Given our knowledge of the operations on the \BKSS, of the $E_2$-page for $\mathbb{S}^{\algs}_T$ and of the homotopy groups $\pi_*\mathbb{S}^{\algs}_T$ (c.f.\ \S\ref{Homotopy operations for simplicial commutative algebras}), a natural goal is the complete computation of the \BKSS\ for $\mathbb{S}^{\algs}_T$. In \S\ref{Some conjectures on the E1-level structure}, we make two conjectures which would together allow us to make this complete computation. It turns out that $E_2$ is not the right place to start this computation, and we need to consider classes on $E_1$ and $d_1$ differentials in order to see the full picture. The problem is that certain relations involving the $\deltav$- and $\Sqh$-operations only hold from $E_2$. The conjectures we make would overcome these problems, and would lead to the description given in \S\ref{The resulting differentials} of the full structure of the \BKSS\ for $\mathbb{S}^{\algs}_T$.\todo{meep moop}



%
%\ref{theMaySseq}
%The cohomotopy spectral sequence of the quadratic filtration is a strongly convergent spectral sequence, the \emph{May-Koszul} spectral sequence:
%\[\E{\textup{MK}}{1}{N_*\textup{QBX}}{m,s_n,\ldots,s_1}{t}\cong \quadgrad{m}(H^*_{\calw(n)}K^{\calw(n)}U^{\calw(n)}X)^{s_n,\ldots,s_1}_{t}\implies (H^*_{\calw(n)}X)^{s_n,\ldots,s_1}_{t}.\]
%If $\pi_*X$ is of finite type, the $E_1$-page may be rewritten as:
%\[\E{\textup{MK}}{1}{N_*\textup{QBX}}{m,s_n,\ldots,s_1}{t}\cong \quadgrad{m}(F^{\HA{\calw(n)}}\dual(\pi_*X))^{s_n,\ldots,s_1}_{t},\]
%which reduces when $n\geq1$ to:
%\[\E{\textup{MK}}{1}{N_*\textup{QBX}}{m,s_n,\ldots,s_1}{t}\cong \quadgrad{m}(F^{\calMhv(n+1)}\dual(\pi_*X))^{s_n,\ldots,s_1}_{t}.\]
%
%
%An atomic example that we may seek to calculate is that of a \emph{sphere} $\mathbb{S}_n^{\algs} $ in $\algs$. This is an object of $s\algs$ which represents the functor $\pi_n$ on the homotopy category $\textup{ho}(s\algs)$. It has $H^*_{\algs}\mathbb{S}_n^{\algs}$ the trivial object of $\calw(0)$, concentrated in dimension $n$



%
%. Moreover, Propositions 
%
%The idea is then to use these spectral sequences to calculate $H^{*}_{\calw(0)}$. To do this would be difficult without a theory of unstable operations for these spectral sequences as well, so this is the next work to be performed. 
%
%Before forecasting the operations to be expected on these composite functor spectral sequences, we must say a little  about the categories $\calw(n)$, defined in \S\ref{wn+1 or pialgs section}. An object of $\calw(n)$ is a vector space in $\vect{+}{n}$, equipped with certain unstable right $\lambda$-operations and a Lie bracket. The top $\Lambda$-operation, when defined, acts as a restriction for the Lie bracket. We write $\calu(n)$ for the category whose objects are graded vector spaces equipped with an action of the \emph{non-top} $\lambda$-operations alone.
%
%%The $\Lambda$-algebra does not act on such an object, given that the $\lambda$-operations are only partially defined, but nonetheless, 
%For $n\geq1$, we may hope for an action of the Koszul dual of the $\Lambda$-algebra on $H^*_{\calw(n)}$, and the Koszul dual algebra is the homogeneous Steenrod algebra. In \S\ref{section: vertical Koszul operations n positive}, for $W\in\calw(n)$ with $n\geq1$, we are indeed able to define `vertical' operations:
%\[\Sqv^i:(H^*_{\calw(n)}W)^{s_{n+1},\ldots,s_1}_t\to (H^*_{\calw(n)}W)^{s_{n+1}+1,s_n+i-1,2s_{n-1},\ldots,2s_1}_{2t+1}.\]
%These operations have their own unstableness conditions, and we write $\calMv(n+1)$ for the resulting category of $\vect{n+1}{+}$-graded unstable modules over the homogeneous Steenrod algebra.
%
%On the other hand, for $n\geq0$, $H^*_{\calw(n)}$ is an example of (partially restricted) Lie algebra cohomology, so that we expect Steenrod operations and products to appear. In \S\ref{Horizontal Steenrod operations and a product for HWn} we define these operations:
%\begin{gather*}
%\Sqh^j:(H^*_{\calw(n)}X)_t^{s_{n+1},\ldots,s_1}\to (H^*_{\calw(n)}X)_{2t+1}^{s_{n+1}+j,2s_{n},\ldots,2s_1},\\
%\mu:(H^*_{\calw(n)}X)_t^{s_{n+1},\ldots,s_1}\otimes (H^*_{\calw(n)}X)_q^{p_{n+1},\ldots,p_1}\to (H^*_{\calw(n)}X)_{t+q+1}^{s_{n+1}+p_{n+1}+1,s_{n}+p_{n},\ldots,s_1+p_1},
%\end{gather*}
%so that $\calw(n)$-cohomology is also a certain type of unstable algebra over the homogeneous Steenrod algebra, %\emph{another} action of the homogeneous Steenrod algebra on $\calw(n)$-cohomology, 
%with the Steenrod action `horizontal'. We write $\calMh(n+1)$ for the resulting category of $\vect{n+1}{+}$-graded unstable algebras over the homogeneous Steenrod algebra.
%(We use the definition of the Steenrod action and products that fits into the framework set out in \S\ref{Constructing cohomology operations}, and the work of showing that these operations coincide with those constructed 
%by Priddy \cite{PriddySimplicialLie.pdf} is deferred until Appendix \ref{appendix on Lie coh ops}.)
%
%We identify in \S\ref{Relations between the horizontal and vertical operations} the relations betwen the $\calMv(n+1)$ and $\calMh(n+1)$, which leads to the defintion of an algebraic category $\calMhv(n+1)$ in which $\calw(n)$-cohomology takes values, for $n\geq1$. Theorems \ref{thm on collapsing of most sseqs} and \ref{Koszul-dual Hilton-Milnor theorem} together show that for $n\geq1$, $\calMhv(n+1)$ is the category of $\calW(n)$-cohomology algebras.
%
%For the calculation of the spectral sequence
%\[\E{\textup{cf}}{2}{W}{}{}=H^{*}_{\calw(n+1)}(H_*^{\calu(n)}(W))\implies H^{*}_{\calw(n)} (W)\]
%for $W\in \calw(n)$ with $n\geq0$, we rely on the following structure. The derived functors $H_{*}^{\calu(n)}W$ are calculated in \S\ref{The Koszul complex and co-Koszul complex} using unstable Koszul resolutions, and in \S\ref{section on structure on homology of koszul cx}, we calculate their structure as objects of $\calw(n+1)$. We relate them to the vertical operations on  $H^{*}_{\calw(n)} (W)$ under the edge homomorphism. For $n\geq1$, In each case, the category of $\calU(n)$-cohomology algebras equals $\calMv()$.
%
% On the other hand, we enrich the $\calMhv(n+2)$-structure on $H^{*}_{\calw(n+1)}(H_*^{\calu(n)}(W))$ to a theory of spectral sequence operations. The core of the method is Singer's theory \cite{MR2245560} of operations in first quadrant spectral sequences, although there is a little more work required in order to produce the operations. The $\calMhv(n+2)$-structure then converges to the $\calMh(n+1)$ structure on the target.
%
%Singer's method was developed in order to rectify the following problem ---
%\TODOCOMMENT[inline]{introduction to be continued}
%
%
%
%, and to retain as much information as possible about its products and the $\deltav_i$- and  $\Sqh^j$-operations. 
%
%
%The category $\PA{(\call(0))}$ is not too difficult to determine, and we name it $\calw(1)$. Objects of $\calw(1)$ are now certain bigraded Lie algebras with partially defined right $\lambda$-operations
%
%
%
%
%
%An atomic example that we may seek to calculate is that of a \emph{sphere} $\mathbb{S}_n^{\algs} $ in $\algs$. This is an object of $s\algs$ which represents the functor $\pi_n$ on the homotopy category $\textup{ho}(s\algs)$. It has $H^*_{\algs}\mathbb{S}_n^{\algs}$ the trivial object of $\calw(0)$, concentrated in dimension $n$
%
%
%\TODOCOMMENT[inline]{\tiny From here, the introduction gives way to a list of components of the thesis. This will be turned into prose asap, although I wanted to give my examiners at least a week with the document! After these dotpoints comes the start of said prose.}
%%We lay out here an overview of  the work that is to follow.
%\begin{itemize}
%\item In \S\ref{Idnt E1 E2}, we derive the expected homological description of Radulescu-Banu's \BKSS\ of $X\in s\algs$. For $X$ connected, this is denoted:
%\[\E{}{2}{\calx}{s}{t}\cong H^{s}_{\calw(0)}(H^{*}_{\algs}X)_t\implies \pi_{t-s}X\hat{\ },\]
%where $\calw(0)$ is the category of connected objects in Goerss' category $\calw$, the natural target category for the Andr\'e-Quillen cohomology functor $H^*_\algs$ on $s\algs$.
%\item In \S\ref{Cohomology Operations for W and U}, we construct three families of operations on the derived functors $H^*_{\calw(0)}$:
%\begin{itemize}
%\item `vertical' higher divided squares $\deltav_i$ (Koszul dual to the action of the algebra $P$ which acts on $H^*_{\algs}$);
%\item a commutative product and `horizontal' Steenrod operations $\Sqh^j$ forming an action of the (homogeneous) Steenrod algebra $\LieSteen:=E_0\cala_2$.  These operations correspond to Priddy's operations \cite{PriddySimplicialLie.pdf} on the cohomology of a simplicial Lie algebra. In Appendix \ref{appendix on Lie coh ops}, we demonstrate that Priddy's operations coincide with those defined in \S\ref{section: Cohomology operations for simplicial (restricted) Lie algebras}.
%\end{itemize}
%\item In \S\ref{second quadrant homotopy sseq operations}, we construct a theory of homotopy operations in the spectral sequence of a cosimplicial simplicial commutative algebra. The key chain-level construction is inspired by Dwyer's work \cite{DwyerHigherDividedSquares.pdf} on the spectral sequence of a cosimplicial simplicial coalgebra, although the two settings are not equivalent  --- although linear dualization interchanges (finite type) algebras and coalgebras, this symmetry is broken by the choice of filtration direction.
%\item In \S\ref{Operations on the Bousfield-Kan spectral sequence}, we construct three corresponding families of unstable operations on the \BKSS\ in $s\algs$. The definition of these operations requires the operations just described, along with a shift in filtration, which we can perform after shifting to a convenient model of the Adams tower in \S\ref{An alternate definition of the Adams tower}.
%\item We use two- and three-cell complexes defined in \S\ref{three cell complex} and \S\ref{two-cell complex for the deltas} to prove that the spectral sequence operations  on $E_2$ agree with those constructed on $H^*_{\calw(0)}$.
%\item The commutative product and higher divided squares abut to those on $\pi_*X\hat{\ }$ when possible. There are, however, \emph{too many} higher divided squares  defined at each $E_r$, and we construct in \S\ref{second quadrant homotopy sseq operations} and \S\ref{Operations on the Bousfield-Kan spectral sequence} differentials from the excess divided squares to the Steenrod operations which correct for this discrepancy.
%\item  In \S\ref{Comp funct sseqs}, we produce an infinite sequence of composite functor spectral sequences, each calculating the $E_2$-page of the last. That is, for $n\geq0$ we define categories and functors 
%\[\calw(n)\overset{Q^{\calu(n)}}{\to}\call(n)\overset{Q^{\call(n)}}{\to}\vect{+}{n},\]
%such that the derived functors $H_*^{\calu(n)}:=\mathbb{L}_*Q^{\calu(n)}$ take values in $\calw(n+1)$, and there are composite functor spectral sequences (for $W\in\calw(n)$):
%\[\E{\textup{cf}}{2}{W}{s_{n+2},\ldots,s_1}{t}=H^{s_{n+2}}_{\calw(n+1)}(H_*^{\calu(n)}(W))^{s_{n+1},\ldots,s_1}_t\implies H^{s_{n+2}+s_{n+1}}_{\calw(n)} (W)_t^{s_n,\ldots,s_1}.\]
%\item In \S\ref{Cohomology Operations for W and U}, we construct three families of operations on the derived functors $H^*_{\calw(n)}$ for $n\geq1$:
%\begin{itemize}
%\item `vertical' Steenrod operations  $\Sqv^i$ (the operations Koszul dual to the action of the $\Lambda$-algebra which is part of the structure of $\calw(n)$);
%\item a commutative product and `horizontal' Steenrod operations $\Sqh^j$ forming an action of the (homogeneous) Steenrod algebra $\LieSteen:=E_0\cala_2$ (again arising from a Lie algebra structure that is part of $\calw(n)$)
%\end{itemize}
%\item In \S\ref{Operations in composite functor spectral sequences}, for each $n\geq0$, using Singer's constructions and certain  chain-level structure on Blanc-Stover's resolution,  we construct three families of unstable operations on the composite functor spectral sequence calculating $H^*_{\calw(n)}$.  The horizontal and vertical Steenrod operations on $E_2=H^*_{\calw(n+1)}$ together abut to the horizontal Steenrod operations on $H^*_{\calw(n)}$, as in Singer's work [].
%\item We prove that these operations agree on $E_2$ with those constructed on $H^*_{\calw(n+1)}$.
%\item We use Priddy's Koszul resolutions to calculate the derived functors $H_*^{\calu(n)}$, and perform chain level calculations of their structure as objects of $\calw(n+1)$.
%\item We use the sequence of spectral sequences to calculate certain examples of the groups $H^*_{\calw(n)}$, including the Bousfield-Kan $E_2$-page for a sphere in $s\algs$.
%\item We produce a May-Koszul spectral sequence calculating the $E_2$-page of the Bousfield-Kan spectral sequence, and describe its $E^1$-page (in terms of the Bousfield-Kan $E_2$-pages of spheres).
%\item We construct the Adams tower of $X\in s\algs$, and study its connectivity properties in order to prove a completeness result for connected $X$.
%\item We prove a vanishing line theorem for the \BKSS\ for a connected object in $s\algs$.
%\item In \S\ref{An alternative Bousfield-Kan E1}, we present a small model for the Bousfield-Kan $E_1$-page for a commutative algebra sphere. This is a good point of reentry for the reader bemused by \S\S\ref{one-dimensional, n geq1}-\ref{Calculations of HW0}.
%\item Finally, we conjecture the full structure of the \BKSS\ for a `sphere'.
%\end{itemize}

%\begin{enumerate}
%\item easier due to homotopy operations
%\item dual dwyer type stuff.
%\end{enumerate}



\SubsectionOrSection{Overview}
The primary object of study in this thesis is the Bousfield-Kan spectral sequence (\BKSS) in the Quillen model category $s\algs$ of simplicial non-unital commutative  $\Ftwo$-algebras. This spectral sequence, which we discuss in \S\ref{Bousfield-Kan spectral sequence}, calculates the homotopy groups of the homology completion $X\hat{\ }$ of $X\in s\algs$, with $E_2$-page given by certain derived functors applied to the Andr\'e-Quillen cohomology groups $H^*_{\algs}X$. \todo{change up whole section}

In \S\ref{Bousfield-Kan spectral sequence} we work directly with the Adams tower to show that whenever $X$ is connected, $X\hat{\ }$ is  equivalent to $X$. Together with the vanishing line theorem we prove in \S\ref{May sseq and vanishing line}, this shows that whenever $X$ is connected, the \BKSS\ is strongly convergent to the homotopy of $X$. 

In \S\ref{Pi-algebras and cohomology algebras} we give an introduction to the theory of homotopy and cohomology algebras and homology coalgebras. In \S\S\ref{Constructing homotopy operations}-\ref{Constructing cohomology operations} %  order to construct the  various operations needed for this thesis, it was helpful to 
we construct a framework in which a number of classically known homotopy and cohomology operations may be considered together. In \S\S\ref{homotopy operations for PRLs}-\ref{Cohomology Operations for W and U} we construct a number of homotopy and cohomology operations in preparation for the following chapters, and in \S\ref{Koszul complexes} we study the unstable Koszul resolutions related to certain of these operations.


We define and study three families of unstable spectral sequence operations on the \BKSS.  Our approach is to perform  a generic construction of spectral sequence operations in a cosimplicial simplicial vector space in \S\ref{second quadrant homotopy sseq operations}, and then perform a shift in filtration using properties of Radulescu-Banu's resolution in \S\ref{Operations on the Bousfield-Kan spectral sequence}.


In \S\ref{Comp funct sseqs} we define a sequence of composite functor spectral sequences (\CFSSs) which we use in \S\ref{Calculations of HWn} to make calculations of the \BKSS\ $E_2$-page in the most important case, when $X$ is a \emph{sphere} in $s\algs$. In order for these spectral sequences to be of any use, we must equip them in \S\ref{Operations in composite functor spectral sequences} with various unstable spectral sequence operations, using a technique due to Singer. 

We define a May-Koszul spectral sequence in \S\ref{May sseq and vanishing line} which converges to the \BKSS\ $E_2$-page for any connected simplicial algebra $X$, and describe the May-Koszul $E^1$-page using the data of the \BKSS\ $E_2$-pages for spheres. Using this spectral sequence one can transfer information about the spectral sequence for spheres to the general setting.

%Given our description of the natural homotopy operations for the \BKSS, The difficulty in proving this conjecture is due to the failure at the $E_1$-page of certain relations that hold on higher pages.
Using the operations on the \BKSS\ we conjecture the full structure of the spectral sequence for a sphere in $s\algs$ in \S\ref{The Bousfield-Kan spectral sequence for a sphere}. 




%, of interest both to topologists and commutative algebraists


\end{Introduction}



\begin{Conventions and notation}
\SectionOrChapter{Background and conventions}
\label{Conventions and notation}

\SubsectionOrSection{Universal algebras}
\label{Universal algebras}
In this thesis we will be dealing with various categories $\calc$ of universal graded algebras over $\Ftwo $. Following \citeBOX[\S2.1]{Blanc_Stover-Groth_SS.pdf}, these are categories whose objects are $G$-graded $\Ftwo $-vector spaces $X=\{X_g\}_{g\in G}$, for some set $G$ of gradings, equipped with a set of operators of the form $X_{g_1}\times \cdots \times X_{g_n}\to X_k$ (with $n\geq1$)  satisfying a set identities, and whose morphisms are graded vector space maps preserving this structure. We assume that the vector space addition  maps $X_g\times X_g\to X_{g}$ are included in the set of operators, in order that the morphisms in $\calc$ are linear maps. Note that we have explicitly excluded nullary operations. 

It need not  be true that all of the defining operators must be linear, but in each of our examples, $\calc$ will be monadic over the category of $G$-graded $\Ftwo $-vector spaces. To explain this, there is a forgetful functor $U^\calc:\calc\to\vect{}{}$, where $\vect{}{}$ is the category of $G$-graded $\Ftwo $-vector spaces, and $U^{\calc}$ admits a left adjoint  $F^\calc:\vect{}{}\to\calc$. %. The functor $U^{\calc}$ will always be monadic, in the sense that 
When the natural comparison functor from $\calc$ to the category of algebras over the monad $U^{\calc}F^{\calc}$ on $\vect{}{}$ is an equivalence, we say that $\calc$ is monadic over $\vect{}{}$.

%graded vector spaces equipped with 
%
%universal graded algebras \citeBOX[\S2.1]{Blanc_Stover-Groth_SS.pdf}: categories whose objects are $G$-graded $\Ftwo $-vector spaces $X=\{X_g\}_{g\in G}$, for some set $G$ of gradings, equipped with an action of set of operators of the form $X_{g_1}\times \cdots X_{g_n}\to X_k$  which satisfy a set identities.
% Such a category possesses certain important structure, that we introduce now. The examples in this paper will have rather more structure than this, and. Firstly, each of our examples will be monadic over the category of $G$-graded $\Ftwo $-vector spaces.
%
%All of the eWe will assume that 
% monadic over $\vect{}{}$, a category either of graded or ungraded $\Ftwo $-vector spaces. That is, 
%
%We will be more specific about the category $\vect{}{}$ shortly.
%As such, it will be helpful to set notation globally. %[Until \S\ref{The example of simplicial commutative F2-algebras} [\textbf{current?}], our comments will apply to any of the algebraic categories $\calL(n)$, $\calU(n)$ or $\calw(n)$ to be defined below. Let $\calc$ be any of these categories, monadic over $\vect{+}{r}$.]
%Until \S?x?x?x?x, we will denote by $\calc$ such a category, although we will need some further assumptions on $\calc$ to proceed. %We will make some further assumptions about the category $\calc$ that will hold in all the examples that we use in this paper.
%
%A category $\calc$ of universal graded algebra is a category whose objects areIn our examples, an object $X$ of $\calc$ may be defined as an object $X$ of $\vect{}{}$ equipped with certain binary and unary operations on $X$ (and without any nullary operations). 
%As such, there will be a forgetful functor $U^\calc:\calc\to\vect{}{}$, which will admit a left adjoint, always denoted $F^\calc:\vect{}{}\to\calc$ for notational consistency. The functor $U^{\calc}$ will always be monadic, in the sense that the natural comparison functor from $\calc$ to the category of algebras over the monad $U^{\calc}F^\calc $ on $\vect{}{}$ is an equivalence.

In our examples, the monad $U^{\calc}F^{\calc}$ will admit an augmentation (of monads) $\epsilon:U^{\calc}F^{\calc}\to\Id$, reflecting homogeneity in the relations defining $\calc$. This augmentation has the monad unit $\eta:\Id\to U^{\calc}F^{\calc}$ as a section, and may be thought of as \emph{projection onto generators}.


We will generally omit the functor $U^{\calc}$ from our notation, writing $F^{\calc}$ as shorthand for either the monad $U^{\calc}F^{\calc}$ on $\vect{}{}$ or the comonad $F^{\calc}U^{\calc}$ on $\calc$. We will refer to elements of a free construction $F^\calc U$ using notation such as $f(v_i)$, thought of as a composite $f$ of $\calc$-structure maps applied to generators $v_i\in V\subseteq F^\calc(V)$. We will say that $f(v_i)$ is a \emph{$\calc$-expression}. In this language, the linear maps % for $V\in \vect{}{}$, we have linear maps
\[F^\calc F^\calc V\overset{\mu}{\to} F^\calc V,\ \ V\overset{\eta}{\to}F^\calc V\ \ \textup{and}\ \ F^\calc V\overset{\epsilon}{\to}V, \]
constituting the augmented monad $F^\calc $ on $\vect{}{}$ may be described as follows: $\mu$ collapses a $\calc$-expression  in $\calc$-expressions into a single $\calc$-expression; $\eta$ sends a vector $v$ to the $\calc$-expression $v$; and $\epsilon$ projects a $\calc$-expression onto those summands to which no (non-trivial) operations have been applied.
For $X\in \calc$, the comonad structure maps in $\calc$,
\[F^\calc F^\calc X\overset{\Delta}{\from} F^\calc X\ \ \textup{and}\ \ X\overset{\rho}{\from}F^\calc X,\]
are as follows: on an expression $f(x_i)$, $\Delta$ returns the same expression $f(x_i)$ in which the $x_i\in X$ are viewed as elements of $F^\calc X$, and $\rho$ is the evaluation map equivalent to the $\calc$-structure on $X$.

\SubsectionOrSection{The functor $Q^\calc$ of indecomposables}
Using the augmentation $\epsilon:F^\calc\to\Id$ of monads on $\vect{}{}$, any $V\in\vect{}{}$ becomes an $F^\calc $-algebra, i.e.\ an object of $\calc$. We denote this functor $K^\calc:\vect{}{}\to \calc$; it sends $V\in\vect{}{}$ to \emph{the trivial object on $V$}, which is $V$ equipped with coaction map the projection $\epsilon:F^{\calc}V\to V$. Whenever we say \emph{trivial} in what follows, we will mean \emph{having no non-zero operations}, and not \emph{equal to zero}.

In each of our examples,  $K^{\calc}$ has a left adjoint, $Q^{\calc}:\calc\to\vect{}{}$, which sends $X\in\calc$ to \emph{the quotient of $X$ by the image of its non-trivial operations}. %More precisely, $K^\calc$ sends any object $V\in\vect{+}{r}$ to the $U^{\calc}F^\calc $-algebra whose structure map is $\epsilon_V:U^{\calc}F^\calc V\to V$. 
%The functor $Q^{\calc}$ sends $X\in \calc$ with structure map $\rho:F^\calc U^\calc X\to X$ to the coequalizer in $\vect{+}{r}$ of the maps $U^\calc\rho$ and $\epsilon_{U^\calc X}$. \emph{or, less correctly}
The functor $Q^{\calc}$ sends $X\in \calc$ to the coequalizer in $\vect{}{}$ of $\rho,\epsilon:F^\calc X\to X$.

Note that $F^\calc$ is a section of $Q^{\calc}$, since $Q^{\calc}\circ F^{\calc}$ is adjoint to $U^\calc K^\calc$, which is obviously the identity.

\SubsectionOrSection{Quillen's model structure on $s\calc$ and the bar construction}\label{ssec: quillen model and bar construction}
For any of the algebraic categories $\calc$ appearing in this thesis we use Quillen's simplicial model category structure on the category $s\calc$ \cite{QuillenHomAlg.pdf}, \cite{MillerSullivanConjecture.pdf}, or \cite{Blanc_Stover-Groth_SS.pdf}. In this structure, the weak equivalences (fibrations) are the maps which are weak equivalences (fibrations) of simplicial abelian groups, so that every object is fibrant. 

 A simplicial object $X$ is \emph{almost free} if there are subspaces $V_n\subseteq X_n$ for each $n\geq0$ such that the composite $F^\calc  V_n\to F^\calc  X_n\overset{\rho}{\to} X_n$ is an isomorphism for all $n$, and such that the subspaces $V_n$ are preserved by all of the degeneracies and face maps of $X$ except for $d_0$. An almost free object is cofibrant, and every cofibrant object is a retract of an almost free object \citeBOX[\S3]{MillerSullivanConjecture.pdf}. 

There is a richer notion, that of an almost free map --- a map $X\to Y$ such that $Y_n$ contains a subspace $V_n$ for each $n$ such that the $V_n$ are preserved by all faces and degeneracies except for $d_0$, and such that the natural map $X_n\sqcup F^\calc(V_n) \to Y_n$ is an isomorphism for each $n$. An almost free map is cofibrant, and every cofibrant map is a retract of an almost free map.

A \emph{cofibrant replacement functor} for $s\calc$  is an endofunctor $f$ of $s\calc$ whose image consists only of cofibrant objects equipped with  a natural acyclic fibration $\epsilon:f\Rightarrow \Id $. One classical way to define such a functor is as follows.

%For a simplicial resolution functor $\calc\to s\calc$, we will often use 
Consider the standard simplicial bar construction $\calc\to s\calc$ arising from the $F^\calc\dashv U^\calc$ adjunction \cite{BlumRiehlResolutions.pdf}. More explicitly, given $X\in\calc$, we define $B^\calc X\in s\calc$ by iterated application of the comonad $F^\calc$ to $X\in \calc$:
\[B_s^\calc X=(F^\calc)^{s+1}X.\]
The face maps are given by the formula $d_i=(F^\calc)^i\rho$, and the degeneracies by $s_i=(F^\calc)^i\Delta$. %, where $\rho$ and $\Delta$ are respectively the counit and diagonal of the comonad $F^\calc U^\calc$. 
This object is almost free, with $B_s^\calc X$ generated by its subspace $V_s=(F^\calc )^sX$, moreover, it is standard \citeBOX[\S4]{BlumRiehlResolutions.pdf} that the augmentation $B^\calc X\to X$ is an acyclic fibration.

The construction generalizes to yield a cofibrant replacement functor $B^{\calc}$ on $s\calc$: one applies  $B^{\calc}:\calc\to s\calc$ levelwise to obtain a bisimplicial object, and then takes the diagonal. A standard spectral sequence argument shows that this is a cofibrant replacement functor. We may write $B^{\calc}$ for any of the functors $\calc\to s\calc$, $s\calc\to ss\calc$ and $s\calc\to s\calc$, depending on context.




\SubsectionOrSection{Categories of graded $\Ftwo $-vector spaces and linear dualization}
In order that we can be more specific about the types of gradings we have in mind, in this section we introduce  notation for the key categories of graded vector spaces. We will still sometimes write $\vect{}{}$ for a generic category of graded vector spaces, or for the category of ungraded vector spaces, as necessary.
%We write $\vect{}{}$ for the category of ungraded vector spaces.
%That said, when we are introducing general notions, we will use the symbol $\vect{}{}$ to refer either to this category or to one of the categories of graded vector spaces which we now introduce.   %We will write $\vect{}{}$ for a generic category of $\Ftwo $-vector spaces, either graded or ungraded. 

Write $\vect{q}{r}$ for the category of vector spaces with $r$ non-negative homological gradings and $q$ non-negative cohomological gradings, so that an object $V$ of $\vect{q}{r}$ decomposes as
\[V=\bigoplus_{s_r,\ldots,s_1,t_q,\ldots,t_1\geq0}V^{t_q,\ldots,t_1}_{s_r,\ldots,s_1}.\]
The category $\vect{q}{r}$ is equipped with a tensor product:
\[(U\otimes V)^{t_q,\ldots,t_1}_{s_r,\ldots,s_1}=\bigoplus_{s'_i+s''_i=s_i,\,t_j'+t_j''=t_j}U^{t'_q,\ldots,t'_1}_{s'_r,\ldots,s'_1}\otimes V^{t''_q,\ldots,t''_1}_{s''_r,\ldots,s''_1}.\]

We will often discuss maps between graded vector spaces which do not preserve degrees. %For a near-example from classical topology, take the whitehead product \[\pi_{t_1}X\times\pi_{t_2}X\to \pi_{t_1+t_2-1}.\] Writing $\pi_{\geq2}X$ for the graded abelian group $\bigoplus_{i\geq 2}\pi_iX$, this product produces a grading-preserving map $(\pi_{\geq2}X)^{\otimes2}\to\Sigma\pi_{\geq2}X$, for $\Sigma$ the evident suspension operator on homologically graded abelian groups.
Although we could encode such maps as grading-preserving maps between appropriate suspensions, we will systematically avoid being so systematic. For example, we will often write $V\otimes V\to V$ for a map which in fact adds one to certain gradings of $V$, and will avoid confusion by explicitly stating the effect of such a map on degrees.

We will often need to consider the linear dual of a vector space $V$, and the standard symbol, $V^*$, will cause ambiguity. Indeed, we will already be using superscripts intensively, so we opt for a modifier written prefix, defining the dualization functor
$\dual:(\vect{q}{r})^\textup{op}\to\vect{r}{q}$ by:
\[(\dual V)_{t_q,\ldots,t_1}^{s_r,\ldots,s_1}:=\hom(V^{t_q,\ldots,t_1}_{s_r,\ldots,s_1},\Ftwo ).\]
We will shortly define cohomology functors
$H_{\calc}^*X:=\dual H^{\calc}_*X$, and we will use the position of the asterisk to demonstrate which of homology and cohomology we mean. This is not precisely an exception to our convention, but was worth mentioning.

Often, the vector spaces we are interested in will support an \emph{extra} grading, the \emph{quadratic grading}, so called because certain operations derived from an underlying squaring operation tend to double this extra grading. We do not think of the quadratic grading as either homological or cohomological, so we write it prefix:
\[V=\textstyle\bigoplus_{k\geq1}q_kV.\]
We write $\quadgrad{}\vect{q}{r}$ for the category of objects of $\vect{q}{r}$ equipped with this extra grading.

A common pattern for us will be to consider vector spaces with $r$ non-negative homological gradings and a single \emph{strictly positive} cohomological grading:
\[V=\bigoplus_{s_r,\ldots,s_1\geq0,\,t\geq 1}V^{t}_{s_r,\ldots,s_1},\]
and we write $\vect{+}{r}$ for the category of such objects. Similarly, there is a category $\vect{r}{+}$, and  dualization is a functor $\dual:(\vect{+}{r})^\textup{op}\to\vect{r}{+}$.


\SubsectionOrSection{The Dold-Kan correspondence}\label{The Dold-Kan correspondence}
In this thesis we will use each of the following five chain complexes in $\complexes \vect{+}{r}$ associated with a simplicial graded vector-space $V\in s\vect{+}{r}$:
\begin{alignat*}{2}
C_nV&:=V_n&& \textup{ with differential }d=\textstyle\sum_{i=0}^{n}d_i;\\
N_nV&:=\textstyle\bigcap_{0< i\leq n}\ker(d_i:V_n\to V_{n-1})&& \textup{ with differential }d=d_0;\\
\Nop_nV&:=\textstyle\bigcap_{0\leq i< n}\ker(d_i:V_n\to V_{n-1})&& \textup{ with differential }d=d_n;\\
\textup{Deg}_n V&:=\textstyle\sum_{0\leq i< n}\im(s_i:V_{n-1}\to V_{n})
&& \textup{ with differential }d=\textstyle\sum_{i=0}^{n}d_i.\\
N_n^\div V&:=V_n/\textup{Deg}_n V&& \textup{ with differential }d=\textstyle\sum_{i=0}^{n}d_i.
\end{alignat*}
There are evident inclusions of $N_*V$ and $\Nop_*V$ into $C_*V$, and a projection of $C_*V$ onto $N_*^\div V$, and all of these maps are weak equivalences. Moreover, the composite $N_*V\to N_*^\div V$ is an isomorphism (as is the composite from $\Nop_*V$). It will be helpful to have an explicit formula for the composite
\[C_*V\to N_n^\div V \overset{\cong}{\to}N_nV. \]
\begin{lem}
\label{the map nml}
The \emph{normalization} map
\[\textup{nml}=(1+s_0d_1)(1+s_1d_2)\cdots (1+s_{n-1}d_n):V_n\to V_n\]
is an idempotent chain complex endomorphism with image $N_*V$ and kernel the degenerate $n$-simplices of $V$, so that there is a commuting diagram
\[\xymatrix@R=4mm{
N_nV\ar@{ >->}[r]\ar@{=}[d]&%r1c1
C_nV\ar[dl]_-{\textup{nml}}
\ar@{->>}[dr]
&%r1c2
\\%r1c3
N_nV\ar@{ >->}[r]
\ar@/_1em/[rr]|-{\cong}
&%r2c1
C_nV\ar@{->>}[r]&%r2c2
N^\div_nV%r2c3
}\]
\end{lem}
\begin{proof}
It is obvious that $\textup{nml}$ restricts to the identity on $N_nV$, and that $\textup{nml}-\Id$ has image consisting of degenerate simplices. By the simplicial identities, for $1\leq i\leq n$:
\[d_i(1+s_{i-1}d_i)=d_i+d_is_{i-1}d_i=d_i+\Id d_i=0.\]
As for $1\leq j<i$, we also have
\[d_i(1+s_{j-1}d_j)%=d_i+d_is_{j-1}d_j
=d_i+s_{j-1}d_{i-1}d_j=(1+s_{j-1}d_j)d_i,\]
this proves that $d_i\circ\textup{nml}=0$ for $1\leq i\leq n$, or that  $\textup{nml}$ has image inside $N_n V$. Thus $\textup{nml}$ is an idempotent with image $N_nV$. As $N_nV\to  N^\div_nV$ is as isomorphism, the rest is easy.
\end{proof}


Each $N_nV$ retains the internal gradings of $V$, and the functor $N_*$ appears in:
\begin{prop}[(The Dold-Kan correspondence), {\citeBOX[\S III.2]{goerss-jardine.pdf}}]
There is an adjoint equivalence of categories:
\[N_*:s\vect{+}{r}\rightleftarrows \complexes \vect{+}{r}:\Gamma,\]
under which the homotopy groups of $V\in s\vect{+}{r}$ (as a simplicial set)  are isomorphic to the homology groups of $N_*V$:
\[(\pi_n V)_{s_r,\ldots,s_1}^t=(H_n N_*V)_{s_r,\ldots,s_1}^t.\]
\end{prop}


%
%As such, we may define the homotopy groups of $V$, written $\pi_*V\in \vect{+}{r+1}$, to be $H_*N_*V$, where $N_*$ is one of the inverse equivalences appearing in the Dold-Kan correspondence [somewhere]: \[N:s\vect{+}{r}\rightleftarrows \complexes \vect{+}{r}:\Gamma.\]
%There are three equivalent definitions of $N_*V$. We will almost always use the description of $N_nV$ as the mutual kernel of $d_1,\ldots,d_n$ in $V_n$, with differential $d_0$ (rather than the equivalent definition as a quotient of the unnormalized complex $C_nV=V_n$), and write $ZN_nV$ for the mutual kernel of $d_0,\ldots,d_n$. 
A cycle in $N_nV$ is an element $x\in V_n$ such that $d_ix=0$ for $0\leq i\leq n$. We write $ZN_nV$ for this group of cycles, referring to elements of $ZN_*V$ as \emph{normalized cycles}. Note that $Z\Nop_*V=ZN_*V$ is the same group of normalized cycles.

For $x$ a cycle in any of the four chain complexes associated with $V$, we will write $\overline{x}$ for the equivalence class of $x$ in $\pi_*V$.

It will often be helpful to remove the notational distinction between the chain complex dimension $n$ and the other homological dimensions $s_r,\ldots,s_1$. That is, we may view $\pi_*V$ as a single object of $\vect{+}{r+1}$, defined by
\[(\pi_*V)^t_{s_{r+1},\ldots,s_1}:=(\pi_{s_{r+1}}V)^t_{s_{r},\ldots,s_1}.\]
Now for any collection of indices $s_{r+1},\ldots,s_1\geq0$ and $t\geq1$, define:
%Let $\commentmathbb{K}_{n,s_r,\ldots,s_1}^t$ and $C\commentmathbb{K}_{n,s_r,\ldots,s_1}^t$ be the $\in \complexes \vect{+}{r}$ be the chain complexes, with $z$, $h$ and $dh$ all having internal gradings $t,s_r,\ldots,s_1$:
%%%%%%\begin{gather*}
%%%%%%\phantom{C\commentmathbb{K}_{n,s_r,\ldots,s_1}^t=\Gamma\Bigl(}\xymatrix@R=1.5mm@1@!C{
%%%%%%&{\smash{(n-1)}}&{\smash{(n)}}&{\smash{(n+1)}}&{\smash{(n+2)}}\\
%%%%%%{\makebox[0cm][r]{$\,\commentmathbb{K}_{n,s_r,\ldots,s_1}^t=\Gamma\Bigl($}\cdots\,} &%r1c1
%%%%%%{\,0\,}\ar[l]
%%%%%%&%r1c2
%%%%%%\Ftwo \langle z\rangle\ar[l]
%%%%%%&%r1c3
%%%%%%{\,0\,}\ar[l]
%%%%%%&{\,0\,}\ar[l]
%%%%%%&{\,\cdots\Bigr),} \ar[l]
%%%%%%\\
%%%%%%%&{\smash{(n-1)}}&{\smash{(n)}} &{\smash{(n+1)}}&{\smash{(n+2)}}\\
%%%%%%{\makebox[0cm][r]{$\,C\commentmathbb{K}_{n,s_r,\ldots,s_1}^t=\Gamma\Bigl($}\cdots\,} &%r1c1
%%%%%%{\,0\,}\ar[l]
%%%%%%&%r1c2
%%%%%%\Ftwo \langle dh\rangle\ar[l]
%%%%%%&%r1c3
%%%%%%\Ftwo \langle h\rangle\ar[l]
%%%%%%&{\,0\,}\ar[l]
%%%%%%&{\,\cdots\Bigr).} \ar[l]
%%%%%%}
%%%%%%\end{gather*}
\begin{gather*}
\phantom{C\mathbb{K}^{t}_{s_{r+1},s_r,\ldots,s_1}=\Gamma\Bigl(}\xymatrix@R=0mm@1@!C{
{\makebox[0cm][r]{$\,\mathbb{K}^{t}_{s_{r+1},s_r,\ldots,s_1}=\Gamma\Bigl($}\cdots\,} &%r1c1
{\,0\,}\ar[l]
&%r1c2
\Ftwo \{ z\}\ar[l]
&%r1c3
{\,0\,}\ar[l]
&{\,0\,}\ar[l]
&{\,\cdots\Bigr),} \ar[l]
\\
&\raisebox{.4mm}{\tiny\makebox[0cm][r]{degrees:\ \ \ \ \ }$\smash{s_{r+1}\!-\!1}$}
&\raisebox{.4mm}{\tiny$\smash{s_{r+1}  }$}
&\raisebox{.4mm}{\tiny$\smash{s_{r+1}\!+\!1}$}
&\raisebox{.4mm}{\tiny$\smash{s_{r+1}\!+\!2}$}\\
%&{\smash{(n)}}&{\smash{(n+1)}}&{\smash{(n+2)}}\\
%&{\smash{(n-1)}}&{\smash{(n)}} &{\smash{(n+1)}}&{\smash{(n+2)}}\\
{\makebox[0cm][r]{$\,C\mathbb{K}^{t}_{s_{r+1},s_r,\ldots,s_1}=\Gamma\Bigl($}\cdots\,} &%r1c1
{\,0\,}\ar[l]
&%r1c2
\Ftwo \{ dh\}\ar[l]
&%r1c3
\Ftwo \{ h\}\ar[l]
&{\,0\,}\ar[l]
&{\,\cdots\Bigr).} \ar[l]
}
\end{gather*}
Here $z$ and $h$ denote are both to lie in internal cohomological grading $t$ and homological gradings $s_r,\ldots,s_1$.
There is an evident inclusion $\imath:\mathbb{K}^{t}_{s_{r+1},s_r,\ldots,s_1}\to C\mathbb{K}^{t}_{s_{r+1},s_r,\ldots,s_1}$. For any $V\in s\vect{+}{r}$, we can identify the subspaces of cycles and boundaries with hom-sets:
\begin{alignat*}{2}
\hom_{s\vect{+}{r}}(\mathbb{K}^t_{s_{r+1},\ldots,s_1},V)&\cong (ZN_{s_{r+1}}V)_{s_r,\ldots,s_1}^t\makebox[0cm][l]{ and}\\
\hom_{s\vect{+}{r}}(C\mathbb{K}^t_{s_{r+1},\ldots,s_1},V)&\cong (N_{s_{r+1}+1}V)_{s_r,\ldots,s_1}^t.
\end{alignat*}
%\[\hom_{s\vect{+}{r}}(\commentmathbb{K}^t_{s_r,\ldots,s_1},V)\cong (ZN_{s_{r+1}}V)_{s_r,\ldots,s_1}^t\textup{ and }\hom_{s\vect{+}{r}}(C\commentmathbb{K}^t_{s_{r+1},\ldots,s_1},V)\cong (N_{s_{r+1}+1}V)_{s_r,\ldots,s_1}^t,\]
Under these isomorphisms the chain complex differential $N_{s_{r+1}+1}V\to ZN_{s_{r+1}}V$ corresponds to $\imath^*$. In fact, $\mathbb{K}^t_{s_{r+1},\ldots,s_1}$ represents $\pi_*(\DASH)t_{s_{r+1},\ldots,s_1}$ in the homotopy category of $s\vect{+}{r}$: in a category of simplicial vector spaces, there is no distinction between spheres and Eilenberg-Mac Lane spaces.

A dual theory exists for cosimplicial vector spaces $U$. We will mention the cochain complexes
\begin{alignat*}{2}
C^nU&:=U^n&& \textup{ with differential }d=\textstyle\sum_{i=0}^{n+1}d^i;\\
N^nU&:=U^n/\textstyle\sum_{0< i\leq n}\im(d^i:U_{n-1}\to U_{n})&& \textup{ with differential }d=d^0;\\
%\Nop_nX&:=\textstyle\bigcap_{0\leq i< n}\ker(d_i:X_n\to X_{n-1})&& \textup{ with differential }d=d_n;\\
N^n_\subseteq U&:=\textstyle\bigcap_{0\leq i\leq n-1}\ker(s^i:U_n\to U_{n-1})&& \textup{ with differential }d=\textstyle\sum_{i=0}^{n+1}d^i.
\end{alignat*}
There are chain complex maps whose leftward composite is an isomorphism:
\[N^nU  \from C^nU\from N^n_\subseteq U,\]
and an explicit normalization map 
\[\textup{nml}=(1+d^ns^{n-1})\cdots (1+d^2s^1)(1+d^1s^0):C^nU\to N^n_\subseteq U\]
with properties dual to the simplicial version. The cohomology of any of these homotopy equivalent cochain complexes defines the \emph{cohomotopy} $\pi^*U$ of $U$.

Homotopy and cohomotopy correspond under dualization as follows.
If $V\in s\vect{}{}$, then $C^*\dual V=\dual C_*V$, and there is a natural isomorphism
\[H^*C^*\dual V=H^*\dual C_*V\to\dual H_* C_*V,\ \ \overline{\alpha}\mapsto\textup{``$\overline{v}\mapsto\alpha(v)$''}.\]

\SubsectionOrSection{Skeletal filtrations of almost free objects}\label{Skeletal filtrations}
Suppose that $X\in s\calc$ is almost free on $V_s\subseteq X_s$. Miller \citeBOX[p.~55]{MillerSullivanConjecture.pdf} defines a filtration of $X$ by almost free subobjects
\[\xymatrix@R=4mm@1{
0\ar@{ >->}[r]
&%r1c1
F_0X\ar@{ >->}[r]&%r1c2
F_1X\ar@{ >->}[r]&%r1c3
F_2X\ar@{ >->}[r]&%r1c3
\cdots \ar@{ >->}[r]&%r1c4
\colim F_mX= X
}\]
as follows.  For each $m\geq0$, write $F_mV_i$ for the subspace of $V_i$ consisting of degeneracies of elements of $V_j$ for $j\leq m$. Then write $F_mX$ for the subobject of $X$ which is almost free on the subobjects $F_mV_i$. The inclusions of these subobjects are  almost free maps, and the colimit is evidently $X$.
\begin{lem}\label{skeleton lemma}
For each $m$, $V_m$ has direct sum decomposition, evidently natural in maps of almost free objects preserving the chosen almost free subspaces:
\[V_m=(V_m\cap N_mX)\oplus(V_m\cap \textup{Deg}_mX),\]
with $ V_m\cap N_mX=\im\Bigl(V_m\overset{\textup{nml}}{\to}V_m\Bigr)$.
Moreover, the map 
\[(\textstyle\sum_{i=0}^{m-1}s_i):V_{m-1}^{\oplus m}\to F_{m-1}V_m\]
is injective.
\end{lem}
\begin{proof}
The second statement is implied by \cite[Fact 3.9]{MillerSullivanConjecture.pdf}. For the direct sum decomposition, it is enough to note that $\textup{nml}$ preserves $V_m$ (which is clear from its defining formula), as it is already known to be an idempotent on $X_m$ with image $N_mX$ and kernel the degenerate simplices $\textup{Deg}_mX$.
%it is clear that $\textup{nml}$ preserves $V_m$, and its purpose is to have image in $N_mX$ and kernel the degenerate elements in $C_mX$. Thus, it is an idempotent of $V_m$, with image in $V_m\cap N_mX$, and kernel the degenerate elements of $V_m$. It is also clear that $\im((\Id-\textup{nml})|_{V_m})\subseteq F_{m-1}V_m$. As $N_mX\cap F_{m-1}V_m=0$, the decomposition of $V_m$ provided by this idempotent is precisely that claimed.
\end{proof}

\SubsectionOrSection{Dold's theorem}
According to Dold \cite{DoldHomologySPs.pdf}, see also \citeBOX[Lemma 3.1]{ChingUnpublished}: 
\begin{thm}[(Dold's theorem)]
\label{Dold's theorem}
Suppose that $F:s\vect{+}{r}\to s\vect{+}{r}$ is a functor preserving weak equivalences, for example, the prolongation of an endofunctor of $\vect{+}{r}$. Then there is a functor $\calF:\vect{+}{r+1}\to\vect{+}{r+1}$ such that the following diagram commutes:
\[\xymatrix@R=4mm{
s\vect{+}{r}\ar[r]^-{F}
\ar[d]^-{\pi_*}
&%r1c1
s\vect{+}{r}\ar[d]^-{\pi_*}
\\%r1c2
\vect{+}{r+1}\ar[r]^-{\calF}
&
\vect{+}{r+1}
}\]
Moreover, if $F\cong {F}_2\circ {F}_1$, then $\calF\cong {\calF}_2\circ {\calF}_1$.
\end{thm}
\noindent The idea here is that the functor $\pi_*$ induces an equivalence between the homotopy category of $s\vect{+}{r}$ and $\vect{+}{r+1}$. In fact, the inverse equivalence can be lifted to a functor into $s\vect{+}{r}$, namely 
\[V\mapsto \Gamma V,\ \ \vect{+}{r+1}\to s\vect{+}{r},\]
where we view $V$ as a trivial chain complex. Then $\calF$ can be constructed as $\calF V:=\pi_*(F\Gamma V)$.
%given that the homotopy category of simplicial vector spaces is graded vector spaces, the derived functors of a weak equivalence preserving functor $F:s\calV\to s\calV$ are well defined, and there is a commuting diagram


\SubsectionOrSection{Homology and cohomology functors $H^{\calc}_*$ and $H_{\calc}^*$}
\label{chom ccoh}


In this thesis we will always define the $\calc$-\emph{homology} of $X\in s\calc$ by the formula:
\[H_*^{\calc}X:=\pi_*(Q^\calc B^\calc X)=H_*N_*(Q^\calc B^\calc X).\]
These homology functors are well defined, as the $Q^\calc\dashv K^\calc$ adjunction is a Quillen adjunction (that $K^\calc$ preserves fibrations and acyclic fibrations is immediate), and indeed we are free to use any cofibrant replacement in place of $B^\calc X$.

It is not always entirely appropriate to call these functors homology. Indeed, Andr\'e-Quillen homology, as defined by Quillen, is the left derived functors of the abelianization functor of \citeBOX[\S II.5]{QuillenHomAlg.pdf}, and it is not true in all of our examples that $Q^\calc$ models the abelianization functor. Goerss \citeBOX[\S4]{MR1089001} explains that this does occur when $\calc$ is the category of non-unital commutative algebras, but it does not occur when $\calc$ is the category of restricted Lie algebras \cite{MR2025911}.

When $\calc$ is monadic over $\vect{+}{r}$, we may view the groups $H_*^{\calc}X$ together as an object of $\vect{+}{r+1}$. That is, each homology group $H_s^\calc X$ retains the gradings of $X$, and a new homological grading is added (to the left of the existing homological gradings). We will sometimes avoid substituting into the asterisk, writing expressions such as $(H_*^\calc X)_{s_{r+1},\ldots,s_1}^t$ in place of $(H_{s_{r+1}}^\calc X)_{s_r,\ldots,s_1}^t$.

We define the \emph{$\calc$-cohomology} $H^*_\calc X$ of $X$ to be $\dual(H_*^\calc X)$, or equivalently the cohomotopy groups $\pi^*\dual(Q^\calc X)$ of the dual cosimplicial object.  As we dualize to obtain cohomology, the cohomological gradings and homological gradings are swapped, and $H^*_{\calc}X$ may be viewed as an object of $\vect{r+1}{+}$. 


\begin{lem}
\label{lemma on homology class repd by normalized generator}
Suppose that $X\in s\calc$ is almost free with generating subspaces $V_n\subseteq X_n$. Then any homology class in $H_n^{\calc}X\cong \pi_nQ^{\calc}X$ can be represented by the image in $Q^{\calc}X_n$ of an element of $V_n\cap N_nX$.
\end{lem}
\begin{proof}
Follows from Lemma \ref{skeleton lemma} --- simply represent the class in question by an element of $V_n$, and then apply the natural map $\textup{nml}$. %\tiny Consider the maps $\textup{nml}:C_nX\to N_nX$ and $\textup{nml}:C_nQ^\calc X\to N_nQ^\calc X$ of Lemma \ref{the map nml}. Choose a representative $w\in ZN_nQ^\calc X$. As the composite $V_n\subseteq X_n\epi Q^\calc X_n$ is an isomorphism, one can find some $v\in V_n$ such that the image of $v$ in $Q^\calc X_n$ is $w$. Although $v$ need not lie in $N_nX$, $\textup{nml}(v)$ must lie in $N_nX$, and maps to $\textup{nml}(w)\in ZN_nQ^\calc X$ under the quotient map.  However, as $w\in ZN_nQ^\calc X$ and $\textup{nml}$ acts as the identity on this subspace, $w=\textup{nml}(w)$, so $v$ is the element we seek.
\end{proof}
\noindent This lemma states that we may find representatives for any homology class in the \emph{subobject} $V_n\cap N_nX$ of $X_n$, while for other applications it will be preferable simply to pass to the \emph{quotient} $V_n$ of $X_n$, and trivially: %We will sometimes wish to work directly with the generating subspaces when working with (co)-homology, and will then apply the following easy observation:
\begin{lem}
\label{identify almost free indecs with gens}
Suppose that $X$ is almost free with generating subspaces $V_n\subseteq X_n$. Then the simplicial object $(Q^{\calc}X)_{n}$ may be identified with the collection of vector spaces $\{V_n\}$, using  the composite
\[V_{n}\overset{d_0}{\to}X_n\cong F^{\calc}V_{n-1}\overset{\epsilon}{\to}V_{n-1}\]
as the zeroth face map of $\{V_n\}$, and using the other structure maps of $X$, which preserve the $V_{n}$, as the other structure maps of $\{V_n\}$.
\end{lem}



\SubsectionOrSection{The action of $\Sigma_2$ on $V^{\otimes 2}$}
For any vector space $V\in \vect{}{}$, the tensor power $V^{\otimes2}:=V\otimes V$ has an action of $\Sigma_2$ given by the map $T$ interchanging the two factors. We will write $S_2V$ for the coinvariants and $S^2V$ for the invariants of this action, and $\Lambda^2V$ for the image in $S^2V$ of the \emph{trace map} $\trace=(1+T):S_2V\to S^2V$. Thus, we have written $\Lambda^2V$ as a \emph{subobject} of $S^2V$.
On the other hand, one may view $\Lambda^2V$ as $S_2V/\ker(\trace)$,  the \emph{quotient} of $S_2V$ by the subspace generated by elements of the form $v\otimes v$. %It is often convenient to define maps out of $\Lambda^2V$ by viewing it as a subobject of $S^2V$, and maps into $\Lambda^2V$ by viewing it as a quotient of $S_2V$ (itself a quotient of $V^{\otimes2}$).

For any $V\in \vect{}{}$ there is a natural map
\[S_2\dual V\to \dual S^2V,\ \ \alpha\otimes \beta\mapsto\textup{``$v\otimes w\mapsto \alpha(v)\beta(w)$''}.\]
It is an isomorphism when $V$ is finite-dimensional.

Suppose that $V$ and $W$ are $\Ftwo $-vector spaces, and $p:S_2V\to W$ is a linear map. A \emph{quadratic refinement of $p$} is a function $\sigma:V\to W$ satisfying, for $v_1,v_2\in V$ and $\alpha\in\Ftwo $:
\[\sigma(v_1+v_2)=\sigma(v_1)+\sigma(v_2)+p(v_1\otimes v_2)\textup{ and }\sigma(\alpha v_1)=\alpha^2\sigma(v_1).\]
In fact, the second condition is redundant (over $\Ftwo $), and these conditions are equivalent to the following condition. For any set $B$, define $\Lambda^2B$ to be the set of subsets of $B$ of cardinality exactly two. 
The equivalent condition is that, for every collection of vectors $v_b\in V$ and of coefficients $\alpha_b\in \Ftwo $ indexed by a set $B$, in which all but finitely many of the $\alpha_b$ are zero, the equation
\[\sigma\Bigl(\sum_{b\in B}\alpha_bv_b\Bigr)=\sum_{b\in B}\alpha_b^2\sigma(v_b)+\sum_{\{b,c\}\in \Lambda^2B}\alpha_b\alpha_cp(v_b\otimes v_c)\]
holds.
%\[v\otimes v
%=
%\sum_{b\in B}\alpha_b^2v_b+\sum_{\{b,c\}\in \Lambda^2B}\alpha_b\alpha_c\trace(v_b\otimes v_c)\]
%
%
%
%
%\begin{alignat*}{2}
%v\otimes v
%&=
%\sum_{b\in B}\alpha_b^2v_b+\sum_{\{b,c\}\in \Lambda^2B}\alpha_b\alpha_c\trace(v_b\otimes v_c)%
%\\
%f(v\otimes v)&:=
%\sum_{b\in B}\alpha_b^2\sigma(v_b)+\sum_{\{b,c\}\in \Lambda^2B}\alpha_b\alpha_cp(v_b\otimes v_c)%
%\\
%% Left hand side
%% Relation
%&=
%% Right hand side
%\textstyle\sum_{i<j}\alpha_i\alpha_jp(v_i\otimes v_j)+\textstyle\sum_{i}\alpha_i^2\sigma(v_i)%
%►% Comment
%&\qquad&\text{(ᾮ)}©
%\end{alignat*}
%
If $f:S^2V\to W$ is a linear map, the function
$v\mapsto f(v\otimes v)$
is a quadratic refinement of $\trace\circ f$, and indeed:
\begin{prop}
\label{propOnExtendingToInvariants}
For any linear map $p:S_2V\to W$, extensions of $p$ to a linear map $f:S^2V\to W$ are in natural bijection  with quadratic refinements of $p$.
\end{prop}
\begin{proof}
%It is enough to prove this result for finite-dimensional vector spaces $V$, as any vector space $V$ is the filtered colimit of its finite-dimensional subspaces, and the functors $S_2$ and $S^2$ commute with filtered colimits.
Suppose that $V$ has basis $\{v_b\ |\ b\in B\}$. Then $S^2V$ has basis the set 
\[\{\trace(v_b\otimes v_c)\ |\ \{b,c\}\in\Lambda^2B\}\cup\{v_b\otimes v_b\ |\ b\in B\}.\]
This is easy to check for $V$ finite dimensional, and extends to the infinite dimensional case as $S^2$ preserves filtered colimits, and we may calculate $V$ as the colimit
\[\colim_{B'\subseteq B}\Ftwo \langle B'\rangle=V.\]
In particular, an extension $f$ of $p$ is determined by the quadratic refinement $v\mapsto f(v\otimes v)$. Thus, as long as we can produce an extension $f$ with $\sigma(v)=f(v\otimes v)$ for any quadratic refinement $\sigma$ of $p$, we will have the natural construction we need.

What remains to prove is that the linear map $f$ defined on this basis by
\[\trace(v_b\otimes v_c)\mapsto p(v_b\otimes v_c),\quad v_b\otimes v_b\mapsto \sigma(v_b)\]
does in fact have the property that $f(v\otimes v)=\sigma(v)$ for \textup{all} $v\in V$. Indeed, if we write $v$ in terms of the chosen basis as $v=\sum_{b\in B}\alpha_bv_b$, then
\[v\otimes v
=
\sum_{b\in B}\alpha_b^2v_b+\sum_{\{b,c\}\in \Lambda^2B}\alpha_b\alpha_c\trace(v_b\otimes v_c),\]
and we can apply our definition of the linear map $f$ to this expansion directly, obtaining
\[f(v\otimes v):=
\sum_{b\in B}\alpha_b^2\sigma(v_b)+\sum_{\{b,c\}\in \Lambda^2B}\alpha_b\alpha_cp(v_b\otimes v_c)=\sigma(v).\qedhere\]
\end{proof}
\begin{cor}
There is a natural linear map $\sqrt{\DASH}:S^2V\to V$, the \emph{square root map}, uniquely determined by the requirements:
\[\sqrt{v_1\otimes v_2+v_2\otimes v_1}=0,\ \ \ \sqrt{v\otimes v}=v \textup{ \ for all $v_1,v_2,v\in V$}.\]
\end{cor}
\begin{proof}
This map is the unique extension of $0:S_2V\to V$ corresponding to the quadratic refinement $\Id:V\to V$ of $0$.
\end{proof}
\noindent The evocative square root symbol is doubly appropriate, as if $V$ is dual to a finite-dimensional vector space $U\in\vect{}{}$, the linear dual of the square root map,
\[\dual V\to \dual S^2V\overset{\cong }{\from}S_2\dual V\]
equals the \emph{squaring map} $U\to S_2U$, defined by $u\mapsto u\otimes u$.


\SubsectionOrSection{Lie algebras in characteristic 2}
\label{introtoLiealgssection}
As we work in characteristic 2, there is more than one available notion of a Lie algebra. An \emph{$S(\LieOperad)$-algebra} is a vector space $L$ equipped with a bracket $L\otimes L\to L$ satisfying the Jacobi identity and the (anti)-symmetry condition $[x,y]=[y,x]$. A \emph{Lie algebra} (or $\Lambda(\LieOperad)$-algebra) is a vector space $L$ equipped with a bracket $L\otimes L\to L$ satisfying the Jacobi identity and the alternating condition $[x,x]=0$. Finally, a \emph{restricted Lie algebra} \cite{CurtisSimplicialHtpy.pdf,6Author.pdf} (or $\Gamma(\LieOperad)$-algebra) is a Lie algebra equipped with a \emph{squaring} or \emph{restriction} function $\restn{(\DASH)}:L\to L$, satisfying the axioms
\[\restn{(x_1+x_2)}=\restn{x_1}+\restn{x_2}+[x_1,x_2]\textup{ \ and \ }[\restn{x_1},x_2]=[x_1,[x_1,x_2]].\]
The alternating condition implies the (anti)-symmetry condition, and these three types of Lie algebras form a hierarchy: a restricted Lie algebra is in particular a Lie algebra, and a Lie algebra is in particular an $S(\LieOperad)$-algebra.

We will  write $\liealgs$ for the category of ungraded Lie algebras, and $\restliealgs$ for the category of ungraded restricted Lie algebras.

Fresse \cite{FresseSimplicialAlgs.pdf} explains how to construct the monads $S(\LieOperad)$, $\Lambda(\LieOperad)$ and $\Gamma(\LieOperad)$ on $\vect{}{}$ which give rise to these structures, starting with the Lie operad $\LieOperad$. For $V\in\vect{}{}$, it is standard that the functor
\[S(\LieOperad):V\mapsto \bigoplus_{n\geq1}(\LieOperad(n)\otimes V^{\otimes n})_{\Sigma_n}\]
inherits the structure of a monad from the composition maps of $\LieOperad$. Fresse observes that the functor
\[\Gamma(\LieOperad):V\mapsto \bigoplus_{n\geq1}(\LieOperad(n)\otimes V^{\otimes n})^{\Sigma_n}\]
may also be equipped with a monad structure, such that the trace map $S(\LieOperad)\to \Gamma(\LieOperad)$
%\[\trace:S(\LieOperad)(V)\to \Gamma(\LieOperad)(V)\]
is a map of monads, and an intermediate monad may be defined by
\[\Lambda(\LieOperad):V\mapsto\im\bigl(\trace:S(\LieOperad)(V)\to \Gamma(\LieOperad)(V)\bigr).\]
These monads give rise to the three indicated forms of Lie algebras in characteristic 2. Now each of these functors supports a \emph{quadratic grading}:
\[\quadgrad{k}(\Gamma(\LieOperad)V):=(\LieOperad(k)\otimes V^{\otimes k})^{\Sigma_k},\]
and similarly for the other two monads.

 Now $\LieOperad(2)\cong\Ftwo $, so that
\[\quadgrad{2}(S(\LieOperad)V)\cong S_2V,\quad \quadgrad{2}(\Lambda(\LieOperad)V)\cong \Lambda^2V,\quad\textup{and}\quad\quadgrad{2}(\Gamma(\LieOperad)V)\cong S^2V.\]
%Each of these three monads is generated by a single binary operation, contained in quadratic grading 2. 
Moreover, one can view an $S(\LieOperad)$-algebra as a map $S_2L\to L$, a $\Lambda(\LieOperad)$-algebra as a map $\Lambda^2L\to L$, and a $\Gamma(\LieOperad)$-algebra as a map $S^2L\to L$, where in each case we demand the necessary compatibilities that these maps extend to the full monad. %Here, we used the analogous definitions $S_2V:=(V^{\otimes2})_{\Sigma_2}$, $S^2V:=(V^{\otimes2})_{\Sigma_2}$ and $\Lambda^2V:=\im(\trace:S_2V\to S^2V)$.
By pulling back along the natural maps
\[S_2V\to \Lambda^2 V\to S^2V\]
one can demote a restricted Lie algebra to a Lie algebra, or a Lie algebra to an $S(\LieOperad)$-algebra.

A \emph{restrictable ideal} in a Lie algebra $L$ is a Lie ideal $I$ of $L$, equipped with a restriction function $\restn{(\DASH)}:I\to I$, satisfying the following axioms, for $x_1,x_2\in I$ and $x_3\in L$:
\[\restn{(x_1+x_2)}=\restn{x_1}+\restn{x_2}+[x_1,x_2]\textup{ \ and \ }[\restn{x_1},x_3]=[x_1,[x_1,x_3]].\]
In fact, let $\mathsf{PRL}$ denote the category of \emph{partially restricted Lie algebras}, whose objects are pairs of vector spaces $(L_+,L_0)$, equipped with a Lie algebra structure on $L_+\oplus L_0$ in which $L_+$ is a restrictable ideal, and whose maps are Lie algebra maps preserving the decomposition and commuting with the partial restrictions. This category is monadic over $\vect{}{}\times\vect{}{}$, the category of pairs of vector spaces, and the value of monad $F^{\mathsf{PRL}}$  on $(V_+,V_0)$ is just an appropriately chosen subalgebra of $\Gamma(\LieOperad)(V_+\oplus V_0)$. We will refer to homogeneous elements of $L_+$ as \emph{restrictable}, and homogeneous elements of $L_0$ as \emph{non-restrictable}.

In \S\ref{The categories Ln} we will define various categories of graded partially restricted Lie algebras, where membership of the restrictable ideal is determined by the non-vanishing of certain gradings.


\SubsectionOrSection{Non-unital commutative algebras}
In this thesis we will always work with \emph{non-unital} commutative algebras, except when we specify otherwise. As for Lie algebras, there are three different notions of non-unital commutative algebra available in characteristic 2. A \emph{commutative algebra} (or $S(\CommOperad)$-algebra) is a vector space $A$ equipped with an associative commutative pairing $A\otimes A\to A$.  We will work with these often, and will write $\algs$ for the category of such algebras. In fact, we will so often discuss simplicial non-unital commutative algebras that we will refer to them simply as \emph{simplicial algebras}.

An \emph{exterior algebra} (or $\Lambda(\CommOperad)$-algebra) is a commutative algebra $A$ with the property that $x^2=0$ for all $x\in A$. A \emph{divided power algebra} (or $\Gamma(\CommOperad)$-algebra) is a commutative algebra $A$ equipped with \emph{divided power} operations, as described in \citeBOX[1.2.2]{FresseSimplicialAlgs.pdf} or \citeBOX[\S2]{MR1089001}. In characteristic 2, these operations are all determined by a single operation, the \emph{divided square} $\gamma_2:A\to A$, which satisfies
\[\gamma_2(xy)=x^2\gamma_2(y),\ \gamma_2(\lambda x)=\lambda^2\gamma_2(x)\textup{ and }\gamma_2(x+y)=\gamma_2(x)+\gamma_2(y)+xy.\]
Note that the second condition is in fact extraneous over $\Ftwo $, and that the last condition implies that a divided power algebra is exterior. Thus, $\gamma_2(xy)=x^2\gamma_2(y)=0$.

There is a notion of a \emph{divided power ideal} of a commutative algebra: an ideal $I$ of a commutative algebra $A$ equipped with a compatible divided power structure on $I$. In this case, $I$ is necessarily exterior, although for $x\in A$ and $y\in L$, $\gamma_2(xy)=x^2\gamma_2(y)$ need not be zero.

Again, Fresse \cite{FresseSimplicialAlgs.pdf} explains how to construct the monads $F^{\algs}:=S(\CommOperad)$, $\Lambda(\CommOperad)$ and $\Gamma(\CommOperad)$ on $\vect{}{}$ which give rise to these structures, using the commutative operad $\CommOperad$ instead of $\LieOperad$. Again, there is a quadratic grading definable on these three monads, and each monad is generated in degree 2, so that a commutative algebra may be thought of as a  map $S_2L\to L$, an exterior algebra as a map $\Lambda^2L\to L$, and a divided power algebra as a map $S^2L\to L$.

The coproduct $A\sqcup B$ in the category of non-unital commutative algebras is the direct sum $A\oplus (A\otimes B)\oplus B$, with the obvious product. Moreover, the \emph{smash coproduct} (to be defined in general in \S\ref{subseq:The smash coproduct}) is simply $A\smashcoprod B:=A\otimes B$. Indeed, coproducts and smash coproducts in each of the above three categories are given by these formulae.

















\SubsectionOrSection{First quadrant cohomotopy spectral sequences}
Suppose that $V_{p,q}$ is a bisimplicial vector space, ungraded for now. We will follow the standard conventions, those of \cite{MR2245560}, in defining the cohomotopy spectral sequence of $V$, which calculates the cohomotopy of the diagonal $\diag{V} $ of $V$. For  more detail, see \citeBOX[\S1.15]{MR2245560}.   

There is a double chain complex $C_{p,q}V:=C\uhor_p C\uver_q V=V_{p,q}$, where we have decorated the  functors $C\uver$ and $C\uhor$ in order to distinguish them from the functor $C_{**}$ being introduced, and to distinguish the coordinates --- we will always refer to $p$ as the \emph{horizontal} coordinate and $q$ as the \emph{vertical} coordinate. The total complex $TV$, along with a canonical increasing filtration, is defined by 
\[(TV)_n:=\bigoplus_{i=0}^{n}C_{i,n-i}V,\qquad F_p(TV)_n:=\bigoplus_{i=0}^{p}C_{i,n-i}V.\]
The dual total complex $\dual TV$ admits a decreasing filtration defined by
\[F^p\dual TV:=\ker\bigl(\dual TV\epi \dual F_{p-1} TV\bigr).\]
Correspondingly, $H^*(\dual TV)\cong \pi^*(\dual\diag{ V})$ is equipped with a decreasing filtration. This filtration is evidently \emph{finite} (eventually stabilizing in any given dimension), \emph{exhaustive} (having union $H^*(\dual TV)$) and \emph{Hausdorff} (having intersection zero). One defines
\[\Edown{0}{\pi^*(\dual\diag{ V})}{p,q}{}:=F^{p}\pi^{p+q}(\dual\diag{ V})/F^{p+1}\pi^{p+q}(\dual\diag{ V}).\]
Then, there is a spectral sequence with
\[\Edown{2}{V}{p,q}{}=\pi\dhor^{p}\pi\dver^{q}(\dual V),\]
and differential $d_r:\Edown{r}{V}{p,q}{}\to \Edown{r}{V}{p+r,q-r+1}{}$ so that $\Edown{r+1}{V}{}{}$ is the cohomology of the cochain complex $(\Edown{r}{V}{}{};d_r)$, and for each fixed $p$ and $q$,
\[\Edown{r}{V}{p,q}{}\textup{ stabilizes to }\Edown{\infty}{V}{p,q}{}\cong \Edown{0}{\pi^*(\dual\diag{ V})}{p,q}{}\textup{ as $r\rightarrow\infty$}.\]

Typically, $V$ will admit an \emph{augmentation} to a simplicial object $V_{-1}\in s\vect{}{}$, \emph{inducing a weak equivalence} $\diag{ V}\overset{\sim}{\to}V_{-1}$. All of our  augmentations are horizontal maps to a vertical object, i.e.\ an augmentation is a simplicial (in $q$) map:
\[d\uhor_0:V_{0,q}\to V_{-1,q}\textup{ coequalizing }d\uhor_0,d\uhor_1:V_{1,q}\to V_{0,q}.\]
In this case, we view the spectral sequence as a tool for the calculation of the cohomotopy $\pi^*(\dual V_{-1})$, via isomorphisms
\[\Edown{\infty}{V}{p,q}{}\cong \Edown{0}{\pi^*(\dual V_{-1})}{p,q}{}.\]


If $V$ is instead a bisimplicial \emph{graded} vector space $V\in ss\vect{c}{h}$, then we may regard $\Edown{r}{V}{}{}$ as an element of $\vect{h+2}{c}$. That is:
\[\Edown{r}{V}{p,q,s_h,\ldots,s_1}{t_c,\ldots,t_1}:=\Edown{r}{(V^{t_c,\ldots,t_1}_{s_h,\ldots,s_1})}{p,q}{}.\]
In our application of these conventions we will actually have $V\in \vect{+}{h}$, and will write $p=s_{h+2}$ and $q=s_{h+1}$. We will even sometimes have a quadratic grading on $V$, which will transfer to a further grading on the spectral sequence, so our spectral sequences will appear in the format
\[\quadgrad{k}\Edown{r}{V}{s_{h+2},\ldots,s_1}{t}:=\Edown{r}{(\quadgrad{k}V^{t}_{s_h,\ldots,s_1})}{s_{h+2},s_{h+1}}{}.\]


%\TODOCOMMENT{\tiny delete this old chapter}
%\begin{shaded}\tiny
%Suppose that $V^{s}_{t}$ is a cosimplicial simplicial (ungraded) vector space. Then there is a cochain-chain complex 
%\[CV^{s}_{t}:=C\dhor^{s}C\uver_{t}V=V^{s}_{t},\]
%with $s$ the horizontal and $t$ the vertical coordinate. The \emph{total complex} $TV$, is a chain complex with a canonical decreasing filtration, defined by 
%\[(TV)_n=\prod_{t-s=n}CV^{s}_{t},\qquad(F^mTV)_n=\prod_{\substack{t-s=n\\s\geq m}}CV^{s}_{t}.\]
%Correspondingly, $H_*( TV)$ is equipped with a decreasing filtration, which is exhaustive, but need not be either Hausdorff or finite, and one defines
%\[\Edownup{0}{}{H_*( TV)}{s}{t}:=F^{s}H_{t-s}( TV)/F^{s+1}   H_{t-s}( TV).\]
%Then, there is a spectral sequence with
%\[\Edownup{1}{}{V}{s}{t}=C\dhor^{s}\pi\uver_{t}V,\qquad\Edownup{2}{}{V}{s}{t}=\pi\dhor^{s}\pi\uver_{t}V,\]
%and differential $d_r:\Edownup{r}{}{V}{s}{t}\to \Edownup{r}{}{V}{s+r}{t+r-1}$ so that $\Edownup{r+1}{}{V}{}{}$ is the homology of the chain complex $(\Edownup{r}{}{V}{}{};d_r)$. From now on, we suppose that the spectral sequence admits a \emph{vanishing line of slope $\alpha$} on $\Edownup{2}{}{}{}{}$, i.e.\ there exists a constant $c$ such that:
%\[\Edownup{2}{}{V}{s}{t}=0\textup{ for }s>c+\alpha(t-s).\]
%Then the filtration on $H_*( TV)$ \emph{is} in fact Hausdorff and finite, and we can say that for each fixed $s$ and $t$:
%\[\Edownup{r}{}{V}{s}{t}\textup{ stabilizes to }\Edownup{\infty}{}{V}{s}{t}\cong \Edownup{0}{}{H_*( TV)}{s}{t}\textup{ as $r\rightarrow\infty$}.\]
%
%
%
%Now if we write
%\[(\calT_mV)_n:=(TV/F^{m+1} TV)_n\cong\prod_{\substack{t-s=n\\s\leq m}}CV^{s}_{t}, \]
%then the $\calT_mV$ form a tower of surjections of chain complexes with inverse limit $\calT_\infty V=TV$.  A result of Bousfield relates this to the totalization tower of $V$:
%\begin{lem}[{\cite[Lemma 2.2]{BousfieldHSSCS.pdf}}]
%There are natural chain maps $\phi_m:N_*\Tot_mV\to \calT_mV$ for $m\leq\infty$ which induce an isomorphism of towers $\pi_*\Tot_mV\to H_*\calT_mV$.
%\end{lem}
%\noindent Given a vanishing line, there is no $\lim^1$ in either tower, so that $\pi_*\Tot V\cong H_*\calT V$, and the spectral sequence truly calculates $\pi_{t-s} \Tot V$.
%
%Typically, $V$ will admit a \emph{coaugmentation} from a simplicial object $V_{-1}\in s\vect{}{}$.  We necessarily mean a horizontal map from a vertical object, i.e.\ a coaugmentation is a simplicial (in $t$) map:
%\[d\dhor^0:V_{-1,t}\to V_{0,t}\textup{ equalizing }d\dhor^0,d\dhor^1:V_{0,t}\to V_{1,t}.\]
%Suppose that this map \emph{induces a weak equivalence} $V_{-1}\overset{\sim}{\to}\Tot V$ (and still that $\Edownup{2}{}{V}{s}{t}$ admits a vanishing line).
%In this case, we view the spectral sequence as a tool for the calculation of the homotopy $\pi_*(V_{-1})$, via  isomorphisms
%\[\Edownup{\infty}{}{V}{s}{t}\cong \Edownup{0}{}{\pi_*(V_{-1})}{s}{t}.\]
%
%We can again add internal gradings, the richest case being that in which  $V\in cs\vect{c}{h}$ also has a quadratic grading:
%\[\quadgrad{k}\Edownup{r}{}{V}{s,s_c,\ldots,s_1}{t,t_h,\ldots,t_1}=\Edownup{r}{}{(\quadgrad{k}V^{s_c,\ldots,s_1}_{t_h,\ldots,t_1})}{s}{t}.\]
%\end{shaded}
%In fact, \textbf{something about comparing the two sseqs........ flip across fibrations.}
%
%
%\begin{shaded}\tiny
%\hfil
%\end{shaded}
%
%
%
%
%
%\begin{cor}
%Suppose that $X^\bullet$ is a cosimplicial object in $s\algs$. Then there is a map of spectral sequences, which is an isomorphism from $E_1$ (\textbf{que?}), from the homotopy spectral sequence of the $\Tot$ tower $\{I\Tot_mX^\bullet\}$ to the spectral sequence of the bicomplex underlying $IX^\bullet$. Moreover, there is an isomorphism $\pi_*I\Tot X^\bullet\to H_*TIX^\bullet$ (to the homology of the total complex of the bicomplex), which is compatible with the map of spectral sequences.
%\end{cor}
%\begin{cor}
%Suppose that $X$ is a connected element of $s\algs$. Then the \BKSS\ for $X$ is the spectral sequence of the bicomplex underlying the cosimplicial simplicial vector space $I((c(KQ)^\bullet X)^{\textup{rf}})$. This spectral sequence admits a vanishing line from $E_2$, and so converges strongly to its target, $\pi_{t-s}(I\hat X)$.
%\end{cor}

















\SubsectionOrSection{Second quadrant homotopy spectral sequences}\label{Towers, exact couples and coaugmented cosimplicial objects}
Suppose %\TODOCOMMENT{\tiny read} 
that
\[0=\calT_{-1}\epifrom \calT_0\epifrom \calT_1\epifrom \calT_2\epifrom \cdots \epifrom \calT_\infty\]
is a tower of surjections of chain complexes, with $\calT_\infty$ the inverse limit. Then $\calT_\infty$ has a canonical decreasing filtration:
\[F^m=F^m\calT_\infty:=\ker(\calT_{\infty}\to\calT_{m-1})\]
and we define, with the conventional suspensions:
\begin{gather*}
\Edownup{0}{}{}{s}{}:=\Sigma^s\ker\bigl(\calT_s\to\calT_{s-1}\bigr);\\
\Edownup{1}{}{}{s}{}:=H_*\Edownup{0}{}{}{s}{}=H_{*-s}\ker\bigl(\calT_s\to\calT_{s-1}\bigr).
\end{gather*}
From this data we may derive the following diagram, in which any pair of composable maps that consists of a monomorphism then an epimorphism is a short exact sequence of chain complexes:
\[\cdots\ \vcenter{\xymatrix@R=4mm{\textstyle
0
&
\Sigma^{-0}\Edownup{0}{}{}{0}{}
&
\Sigma^{-1}\Edownup{0}{}{}{1}{}
&
\Sigma^{-2}\Edownup{0}{}{}{2}{}
&
\Sigma^{-3}\Edownup{0}{}{}{3}{}
\\
\calT_\infty\ar@{ >->}[d]\ar@{->>}[u]
&
F^0\ar@{ >->}[d]\ar@{=}[l]\ar@{->>}[u]
&
F^1\ar@{ >->}[d]\ar@{ >->}[l]\ar@{->>}[u]
&
F^2\ar@{ >->}[d]\ar@{ >->}[l]\ar@{->>}[u]
&
F^3\ar@{ >->}[d]\ar@{ >->}[l]\ar@{->>}[u]
\\
\calT_\infty\ar@{=}[r]\ar@{->>}[d]&
\calT_\infty\ar@{=}[r]\ar@{->>}[d]&
\calT_\infty\ar@{=}[r]\ar@{->>}[d]&
\calT_\infty\ar@{=}[r]\ar@{->>}[d]&
\calT_\infty\ar@{->>}[d]
\\
0
&\calT_{-1}\ar@{=}[l]
&\calT_0\ar@{->>}[l]
&\calT_1\ar@{->>}[l]
&\calT_2\ar@{->>}[l]
\\
0\ar@{ >->}[u]
&
0\ar@{ >->}[u]
&
\Sigma^{-0}\Edownup{0}{}{}{0}{}\ar@{ >->}[u]
&
\Sigma^{-1}\Edownup{0}{}{}{1}{}\ar@{ >->}[u]
&
\Sigma^{-2}\Edownup{0}{}{}{2}{}\ar@{ >->}[u]
}}\!\!\!\!\!\cdots \]
Taking homology, each short exact sequence of chain complexes creates a long exact sequence, and we obtain two exact couples (c.f.\ \cite{limits_and_sseq.pdf} or \citeBOX[\S2.2]{mccleary.pdf}), which we juxtapose, using dotted maps to indicate boundary homomorphisms:
\[\cdots\ \vcenter{\xymatrix@R=4mm{\textstyle
0
&
\Sigma^{-0}H\Edownup{1}{}{}{0}{}\ar@{..>}[dr]
&
\Sigma^{-1}H\Edownup{1}{}{}{1}{}\ar@{..>}[dr]
&
\Sigma^{-2}H\Edownup{1}{}{}{2}{}\ar@{..>}[dr]
&
\Sigma^{-3}H\Edownup{1}{}{}{3}{}
\\
H\calT_\infty\ar[d]\ar[u]
&
HF^0\ar[d]\ar@{=}[l]\ar[u]
&
HF^1\ar[d]\ar[l]\ar[u]
&
HF^2\ar[d]\ar[l]\ar[u]
&
HF^3\ar[d]\ar[l]\ar[u]
\\
H\calT_\infty\ar@{=}[r]\ar[d]&
H\calT_\infty\ar@{=}[r]\ar[d]&
H\calT_\infty\ar@{=}[r]\ar[d]&
H\calT_\infty\ar@{=}[r]\ar[d]&
H\calT_\infty\ar[d]
\\
0
&H\calT_{-1}\ar@{=}[l]\ar@{..>}[dr]\ar@{..>}@/^2em/[uu]
&H\calT_0\ar[l]\ar@{..>}[dr]\ar@{..>}@/^2em/[uu]
&H\calT_1\ar[l]\ar@{..>}[dr]\ar@{..>}@/^2em/[uu]
&H\calT_2\ar[l]\ar@{..>}@/^2em/[uu]
\\
0\ar[u]
&
0\ar[u]
&
\Sigma^{-0}H\Edownup{1}{}{}{0}{}\ar[u]
&
\Sigma^{-1}H\Edownup{1}{}{}{1}{}\ar[u]
&
\Sigma^{-2}H\Edownup{1}{}{}{2}{}\ar[u]
}}\!\!\!\!\!\cdots \]
The vertical boundary homomorphisms $H\calT_m\to \Sigma HF^{m+1}$ in fact form a morphism of exact couples (c.f.\ \cite{limits_and_sseq.pdf}), as follows from Verdier's octahedral axiom (in the  homotopy category of  chain complexes c.f.\ \cite[Appendix A.1]{MR1388895}) or a diagram chase. Moreover, the two resulting spectral sequences have the same $E_1$-page, so that they are identical (see also \citeBOX[\S6]{limits_and_sseq.pdf}). %We have made an effort to note these opposite yet equal spectral sequences, as they form a model for more homotopical
This common spectral sequence is simply the spectral sequence of the decreasing filtration $F^m$ on the complex $\calT_{\infty}$ (c.f.\ \citeBOX[\S2.2]{mccleary.pdf}, \cite{BousfieldHSSCS.pdf}).
The intended target $H\calT_\infty$ has decreasing filtration
\[F^m(H\calT_\infty):=\im(HF^m\to H\calT_\infty)=\ker(H\calT_\infty\to H\calT_{m-1}),\]
and one writes $\Edownup{0}{}{H\calT_\infty}{s}{t}=F^sH_{t-s}\calT_\infty/F^{s+1}H_{t-s}\calT_\infty$. %\TODOCOMMENT[inline]{pro-triviality, bousfield 2.3}

One context in which we may make these constructions is when given any sequence of maps 
\[0=\mathbb{T}_{-1}\from \mathbb{T}_{0} \from  \mathbb{T}_{1}\from \cdots\]
in $s\vect{}{}$. Such a tower may be converted into a homotopy equivalent tower of surjections
\[0=\mathbb{T}'_{-1}\epifrom \mathbb{T}'_{0} \epifrom  \mathbb{T}'_{1}\epifrom \cdots\] and we may perform the above constructions with $\calT_m:=C_*\mathbb{T}'$. Homotopy equivalent towers will produce isomorphic spectral sequences from $E_1$. From this perspective, a straightforward way to give a map of spectral sequences that \emph{shifts filtration} is simply to give a map of such  towers with the corresponding shift.


Suppose now that $V$ is an object of $(s\vect{}{})^{\Delta_+}$, the category of coaugmented cosimplicial objects in the category of simplicial vector spaces. We think of the cosimplicial direction as \emph{horizontal} and the simplicial direction as \emph{vertical}, so that the \emph{coaugmentation} of $V$ is a (horizontal) map from a (vertical) simplicial object $V^{-1}\in s\vect{}{}$,i.e.\ a simplicial (in $t$) map:
\[d\dhor^0:V_{t}^{-1}\to V^{0}_{t}\textup{ equalizing }d\dhor^0,d\dhor^1:V^{0}_{t}\to V^{1}_{t}.\]

There is a cochain-chain complex 
\[CV^{s}_{t}:=C\dhor^{s}C\uver_{t}V=V^{s}_{t},\]
with $s$ the horizontal and $t$ the vertical coordinate. The \emph{total complex} $TV$ is a chain complex with a canonical decreasing filtration, defined by 
\[\textstyle(TV)_n=\prod_{t-s=n}CV^{s}_{t},\qquad d=d\dhor+d\uver,\qquad(F^mTV)_n=\prod_{\substack{t-s=n\\s\geq m}}CV^{s}_{t}.\]
On the other hand, we may write
\[(\calT_mV)_n:=(TV/F^{m+1} TV)_n\cong\textstyle\prod_{\substack{t-s=n\\s\leq m}}CV^{s}_{t}, \]
giving a tower $\calT_mV$ of surjections of chain complexes with inverse limit $\calT_\infty V=TV$, as in the earlier discussion. The two evident filtrations of $H_*(TV)=H_*(\calT_\infty V)$ coincide, by inspection of the distinguished triangles drawn vertically in the above diagram. The resulting spectral sequence satisfies
\begin{alignat*}{2}
\Edownup{0}{}{V}{s}{t}
&:=
C\dhor^sC\uver_tV,
&\qquad&d_0=d\uver;\\
\Edownup{1}{}{V}{s}{t}
&:=
C\dhor^s\pi\uver_tV,
&\qquad&d_1=d\dhor;\\
\Edownup{2}{}{V}{s}{t}
&:=
\pi\dhor^s\pi\uver_tV.
\end{alignat*}
%The target groups $H_{*}( TV)$ have two evident decreasing filtrations, which coincide by inspection of the long exact sequences appearing above. If one defines
%\[\Edownup{0}{}{H_*( TV)}{s}{t}:=F^{s}H_{t-s}( TV)/F^{s+1}   H_{t-s}( TV),\]
%then there is a spectral sequence with
%\[\Edownup{1}{}{V}{s}{t}=C\dhor^{s}\pi\uver_{t}V,\qquad\Edownup{2}{}{V}{s}{t}=\pi\dhor^{s}\pi\uver_{t}V,\]
The differential is of the form $d_r:\Edownup{r}{}{V}{s}{t}\to \Edownup{r}{}{V}{s+r}{t+r-1}$, and as ever, $\Edownup{r+1}{}{V}{}{}$ is the homology of the chain complex $(\Edownup{r}{}{V}{}{};d_r)$. We will work with this spectral sequence in detail, and will thus use the following description of the entries. Define:
\begin{alignat*}{2}
\EZ{}{r}{V}{s}{t}&:=\left\{x\in (F^sTV)_{t-s}\ \middle| dx\in (F^{s+r}TV)_{t-s-1}\right\};\\
\E{}{r}{V}{s}{t}&:=\EZ{}{r}{V}{s}{t}/\left(d\left(\EZ{}{r-1}{V}{s-r+1}{t-r+2}\right)+\EZ{}{r-1}{V}{s+1}{t+1}\right).
\end{alignat*}


The spectral sequence will sometimes admit a \emph{vanishing line of slope $\alpha$} on $\Edownup{2}{}{}{}{}$, i.e.\ there will exist a constant $c$ such that:
\[\Edownup{2}{}{V}{s}{t}=0\textup{ for }s>c+\alpha(t-s).\]
In this case, the filtration on $H_*( TV)$ is Hausdorff and finite, and  for each fixed $s$ and $t$:
\[\Edownup{r}{}{V}{s}{t}\textup{ stabilizes to }\Edownup{\infty}{}{V}{s}{t}\cong \Edownup{0}{}{H_*( TV)}{s}{t}\textup{ as $r\rightarrow\infty$}.\]
%\begin{shaded}\tiny
%\noindent Given a vanishing line, there is no $\lim^1$ in either tower, so that $\pi_*\Tot V\cong H_*\calT V$, and the spectral sequence truly calculates $\pi_{t-s} \Tot V$.
%
%Typically, $V$ will admit a \emph{coaugmentation} from a simplicial object $V_{-1}\in s\vect{}{}$.  We necessarily mean a horizontal map from a vertical object, i.e.\ a coaugmentation is a simplicial (in $t$) map:
%\[d\dhor^0:V_{-1,t}\to V_{0,t}\textup{ equalizing }d\dhor^0,d\dhor^1:V_{0,t}\to V_{1,t}.\]
%Suppose that this map \emph{induces a weak equivalence} $V_{-1}\overset{\sim}{\to}\Tot V$ (and still that $\Edownup{2}{}{V}{s}{t}$ admits a vanishing line).
%\end{shaded}


The coaugmentation  induces a map $V^{-1}\overset{\sim}{\to}\Tot V$ where $\Tot V$ is  the totalization of $V$ in the simplicial model category $s\vect{}{}$ \citeBOX[VII.5]{goerss-jardine.pdf}. 
Bousfield explains how this relates to the totalization tower \citeBOX[VII.5]{goerss-jardine.pdf} of $V$: %\TODOCOMMENT{\tiny enough said?}
\begin{lem}[{\cite[Lemma 2.2]{BousfieldHSSCS.pdf}}]
There are natural chain maps $N_*\Tot_mV\to \calT_mV$ for $m\leq\infty$ which induce an isomorphism of towers $\pi_*\Tot_mV\to H_*\calT_mV$. In particular $H_*(TV)\cong \pi_*\Tot V$.
\end{lem}
%When this map is a weak equivalence, we view the spectral sequence as a tool for the calculation of the homotopy $\pi_*(V^{-1})$, 
%via  isomorphisms\[\Edownup{\infty}{}{V}{s}{t}\cong \Edownup{0}{}{\pi_*(V^{-1})}{s}{t}.\]
Not only then do we have a tower under $\calT_\infty V\simeq C_*\Tot V$, but $\Tot V$ accepts the coaugmentation map from $V^{-1}$. 
Of course, the coaugmentation map need not be surjective, but if we factor it  as a composite
\[\xymatrix@R=4mm{
V^{-1}\ar@{ >->}[r]^-{\sim}
&%r1c1
r(V^{-1})\ar@{->>}[r]&%r1c2
\Tot V%r1c3
}\]
we may form the following diagram by demanding that the vertical composites be strict fiber sequences
\[\cdots\ \vcenter{\xymatrix@R=4mm{\textstyle
r(V^{-1})\ar@{ >->}[d]
&
\textup{Fib}^0\ar@{ >->}[d]\ar@{=}[l]
&
\textup{Fib}^1\ar@{ >->}[d]\ar@{ >->}[l]
&
\textup{Fib}^2\ar@{ >->}[d]\ar@{ >->}[l]
&
\textup{Fib}^3\ar@{ >->}[d]\ar@{ >->}[l]
\\
r(V^{-1})\ar@{=}[r]\ar@{->>}[d]&
r(V^{-1})\ar@{=}[r]\ar@{->>}[d]&
r(V^{-1})\ar@{=}[r]\ar@{->>}[d]&
r(V^{-1})\ar@{=}[r]\ar@{->>}[d]&
r(V^{-1})\ar@{->>}[d]
\\
0
&\Tot_{-1}\ar@{=}[l]
&\Tot_0\ar@{->>}[l]
&\Tot_1\ar@{->>}[l]
&\Tot_2\ar@{->>}[l]
}}\ \cdots \]
and applying the functor $C_*$.
%\[0=\calT_{-1}V\epifrom \calT_0V\epifrom \calT_1V\epifrom \calT_2V\epifrom \cdots \epifrom \calT_\infty V\from C_*V^{-1}.\]
%Of course, the coaugmentation map need not be surjective, but this will be of little consequence.

We will in general hope that $V^{-1}\to \Tot V$ will be a weak equivalence, and to investigate whether or not this is so, it will be helpful to be able to identify the fibers $\textup{Fib}^m$ up to homotopy.
For this we recall a useful relationship between cosimplicial objects and cubical diagrams, explained by Sinha in \cite[Theorem 6.5]{SinhaSpacesOfKnots.pdf}, and expanded on by Munson-Voli\'c \cite{CubicalHomotopyTheory.pdf}. We will only present that part of the theory that we need, and refer the reader to \cite{GoodwillieCalcII}, \cite{LuisGoodwillie.pdf} or \cite{CubicalHomotopyTheory.pdf} for the  theory of cubical diagrams and their homotopy total fibers. For $n\geq0$ let $[n]=\{0,\ldots,n\}$, and define $\calP[n]=\left\{S\subseteq [n]\right\}$ to be the poset category whose morphisms are the inclusions $S\subseteq S'$. Then an \emph{$(n+1)$-cube} in $s\vect{}{}$ is a functor $\calP[n]\to s\vect{}{}$.


Sinha describes a diagram of inclusions of categories
\[\xymatrix@!0@R=10mm@C=13mm{
\calP[-1]\ar[rr]^-{\tau}\ar_-{h_{-1}}[drrr]
&&%r1c1
\calP[0]\ar[rr]^-{\tau}\ar^(.65){\!h_{0}}[dr]
&&%r1c2
\calP[1]\ar[rr]^-{\tau}\ar_(.65){h_{1}\!}[dl]
&&%r1c3
\calP[2]\ar[rr]^-{\tau}\ar^-{h_{2}}[dlll]
&&%r1c4
\cdots \\
&&&\Delta_+\!\!
}\]
The augmented cosimplicial simplicial vector space $V:\Delta_+\to s\vect{}{}$ may be pulled back along $h_m$ to form an $(m+1)$-cubical diagram $h_m^*V$. After noting that $V$ is Reedy fibrant (c.f.\ \citeBOX[{X.4.9}]{YellowMonster}), Sinha explains that there are natural weak equivalences 
\[\textup{Fib}^{m+1}\sim\hofib(V^{-1}\to\textup{Tot}_mV) \overset{\smash{\sim}}{\to} \textup{hototfib}(h_m^*V)\]
under which the inclusion $\textup{Fib}^m\from \textup{Fib}^{m+1}$
%\[\hofib(V^{-1}\to\textup{Tot}_nV)\to \hofib(V^{-1}\to\textup{Tot}_{n-1}V)\]
is identified with the map
\[\textup{hototfib}(h_m^*V)\to \textup{hototfib}(\tau^*h_m^*V)=\textup{hototfib}(h_{m-1}^*V).\]
As $h_{-1}^*V$ is the 0-cube with value $V^{-1}$, the tower of homotopy total fibers is identified up to homotopy with the tower of the $\textup{Fib}^m$.

%Suppose that $V^{s}_{t}$ is a cosimplicial simplicial (ungraded) vector space. Then there is a cochain-chain complex 
%\[CV^{s}_{t}:=C\dhor^{s}C\uver_{t}V=V^{s}_{t},\]
%with $s$ the horizontal and $t$ the vertical coordinate. The \emph{total complex} $TV$ is a chain complex with a canonical decreasing filtration, defined by 
%\[\textstyle(TV)_n=\prod_{t-s=n}CV^{s}_{t},\qquad d=d\dhor+d\uver,\qquad(F^mTV)_n=\prod_{\substack{t-s=n\\s\geq m}}CV^{s}_{t}.\]
%On the other hand, we may write
%\[\textstyle(\calT_mV)_n:=(TV/F^{m+1} TV)_n\cong\prod_{\substack{t-s=n\\s\leq m}}CV^{s}_{t}, \]
%giving a tower $\calT_mV$ of surjections of chain complexes with inverse limit $\calT_\infty V=TV$.
%
%We can summarise this situation with a diagram, abbreviating $F^mTV$ to $F^m$ and $\calT_mV$ to $\calT_m$:
%\[\cdots\ \vcenter{\xymatrix@R=4mm{\textstyle
%0
%&
%\frac{F^0}{F^1}
%&
%\frac{F^1}{F^2}
%&
%\frac{F^2}{F^3}
%&
%\frac{F^3}{F^4}
%\\
%TV\ar@{ >->}[d]\ar@{->>}[u]
%&
%F^0\ar@{ >->}[d]\ar@{=}[l]\ar@{->>}[u]
%&
%F^1\ar@{ >->}[d]\ar@{ >->}[l]\ar@{->>}[u]
%&
%F^2\ar@{ >->}[d]\ar@{ >->}[l]\ar@{->>}[u]
%&
%F^3\ar@{ >->}[d]\ar@{ >->}[l]\ar@{->>}[u]
%\\
%TV\ar@{=}[r]\ar@{->>}[d]&
%TV\ar@{=}[r]\ar@{->>}[d]&
%TV\ar@{=}[r]\ar@{->>}[d]&
%TV\ar@{=}[r]\ar@{->>}[d]&
%TV\ar@{->>}[d]
%\\
%0
%&\calT_{-1}\ar@{=}[l]
%&\calT_0\ar@{->>}[l]
%&\calT_1\ar@{->>}[l]
%&\calT_2\ar@{->>}[l]
%\\
%0\ar@{ >->}[u]
%&
%0\ar@{ >->}[u]
%&
%\frac{F^0}{F^1}\ar@{ >->}[u]
%&
%\frac{F^1}{F^2}\ar@{ >->}[u]
%&
%\frac{F^2}{F^3}\ar@{ >->}[u]
%}}\ \cdots \]
%In this diagram, any pair of composable maps consisting of a monomorphism then an epimorphism is a short exact sequence of chain complexes. Taking homology, each such pair creates a long exact sequence, and we obtain two \emph{spread out exact couples}, which we juxtapose, using wavy maps to indicate boundary homomorphisms:
%\[\cdots\ \vcenter{\xymatrix@R=4mm{\textstyle
%0
%&
%H\frac{F^0}{F^1}\ar@{~>}[dr]
%&
%H\frac{F^1}{F^2}\ar@{~>}[dr]
%&
%H\frac{F^2}{F^3}\ar@{~>}[dr]
%&
%H\frac{F^3}{F^4}
%\\
%HTV\ar[d]\ar[u]
%&
%HF^0\ar[d]\ar@{=}[l]\ar[u]
%&
%HF^1\ar[d]\ar[l]\ar[u]
%&
%HF^2\ar[d]\ar[l]\ar[u]
%&
%HF^3\ar[d]\ar[l]\ar[u]
%\\
%HTV\ar@{=}[r]\ar[d]&
%HTV\ar@{=}[r]\ar[d]&
%HTV\ar@{=}[r]\ar[d]&
%HTV\ar@{=}[r]\ar[d]&
%HTV\ar[d]
%\\
%0
%&H\calT_{-1}\ar@{=}[l]\ar@{~>}[dr]\ar@{~>}@/^2em/[uu]
%&H\calT_0\ar[l]\ar@{~>}[dr]\ar@{~>}@/^2em/[uu]
%&H\calT_1\ar[l]\ar@{~>}[dr]\ar@{~>}@/^2em/[uu]
%&H\calT_2\ar[l]\ar@{~>}@/^2em/[uu]
%\\
%0\ar[u]
%&
%0\ar[u]
%&
%H\frac{F^0}{F^1}\ar[u]
%&
%H\frac{F^1}{F^2}\ar[u]
%&
%H\frac{F^2}{F^3}\ar[u]
%}}\ \cdots \]


\end{Conventions and notation} %return me to my home


\begin{Pi-algebras and cohomology algebras}

\SectionOrChapter{Homotopy operations and cohomology operations}
\label{Pi-algebras and cohomology algebras}
Let $\calc$ be a category of universal graded algebras, monadic over $\vect{+}{r}$. Our goal is to understand construct operations on the homotopy and cohomology of an object of $s\calc$. In \S\ref{homotopy and pialgs} and \S\ref{cohomology and Halgs}, we set out dual frameworks in which these operations can be organised, and in \ref{quadratic part section} and \ref{subseq:The smash coproduct}, we will describe some useful chain level operations that we will use to construct cohomology operations in \S\ref{sec:Constructing cohomology operations}.

\SubsectionOrSection{The spheres in $s\calc$ and their mapping cones}\label{spheres and cones}



%The forgetful functor $U^\calc:\calc\to \vect{+}{r}$ used to define the weak equivalences and fibrations in $s\calc$, allows us to consider any object $X\in s\calc$ as an object of $s\vect{+}{r}$. In particular, for any $X\in s\calc$ we may define homotopy groups
Using the forgetful functor $U^\calc:\calc\to \vect{+}{r}$, for any $X\in s\calc$ we may define the homotopy groups $\pi_*X$ of $X$,
%\[(\pi_*X)_{s_{r+1},\ldots,s_1}^{t}:=\pi_{s_{r+1}}(X^t_{s_r,\ldots,s_1}),\]
which we view together as an object  of $\vect{+}{r+1}$. By the definition of the model structure on $s\calc$, the functor $\pi_*:s\calc\to \vect{+}{r+1}$ is homotopical, which is to say that it inverts weak equivalences. For any set of indices $t\geq0$ and $s_{r+1},\ldots,s_1\geq0$, write:
\begin{alignat*}{2}
\mathbb{S}^{\calc,t}_{s_{r+1},\ldots,s_1}
&:=
F^\calc \mathbb{K}^{t}_{s_{r+1},\ldots,s_1};\textup{ and}
\\
% Left hand side
C\mathbb{S}^{\calc,t}_{s_{r+1},\ldots,s_1}
% Relation
&:=
% Right hand side
F^\calc C\mathbb{K}^{t}_{s_{r+1},\ldots,s_1}%
% Comment
\end{alignat*}
These are the \emph{spheres in $s\calc$} and \emph{cones on spheres in $s\calc$} respectively, and we write
\begin{alignat*}{2}
\spheres{\calc}&:=\bigl\{\mathbb{S}^{\calc,t}_{s_{r+1},\ldots,s_1}\,|\,t\geq0,\ s_{r+1},\ldots,s_1\geq0\bigr\} %\textup{ and}
%\\
%\cones{\calc}&:=\bigl\{C\mathbb{S}^{\calc,t}_{s_{r+1},\ldots,s_1}\,|\,t\geq0,\ s_{r+1},\ldots,s_1\geq0\bigr\}
\end{alignat*}
for the set of spheres in $\calc$. Note that we were very literal here --- the \emph{spheres} of $s\calc$ are precisely this family of objects, and not the family, say, of cofibrant objects weakly equivalent to some $\mathbb{S}^{\calc,t}_{s_{r+1},\ldots,s_1}$. For $S\in \spheres{\calc}$ we write $CS$ for the corresponding cone.

For any $S\in\spheres{\calc}$, there is an evident cofibration $\imath :S\to CS$. Indeed, for any sphere $S=\mathbb{S}^{\calc,t}_{s_{r+1},\ldots,s_1}$, $S$ contains  a distinguished normalized cycle, the \emph{fundamental cycle}:
\[z\in (ZN_{s_{r+1}}S)_{s_r,\ldots,s_1}^t,\]
and the cone $CS$ contains a distinguished normalized chain, the \emph{cone on $z$}:
\[h\in (N_{s_{r+1}+1}CS)_{s_r,\ldots,s_1}^t,\]
and $\imath$ is defined by the requirement that $\imath(z)=d_0h$.
For any $X\in s\calc$, by adjunction:
%\[\hom_{s\calc}(F^\calc\commentmathbb{K}^t_{n,s_r,\ldots,s_1},X)\cong (ZN_nX)_{s_r,\ldots,s_1}^t\textup{ and }\hom_{s\calc}(F^\calc C\commentmathbb{K}^t_{n,s_r,\ldots,s_1},X)\cong (N_{n+1}X)_{s_r,\ldots,s_1}^t,\]
%\begin{gather*}
%\hom_{s\calc}(F^\calc\commentmathbb{K}^t_{s_{r+1},\ldots,s_1},X)\cong (ZN_{s_{r+1}}X)_{s_r,\ldots,s_1}^t\makebox[0cm][l]{ and}\\
%\hom_{s\calc}(F^\calc C\commentmathbb{K}^t_{s_{r+1},\ldots,s_1},X)\cong (N_{s_{r+1}+1}X)_{s_r,\ldots,s_1}^t,
%\end{gather*}
\begin{alignat*}{2}
\hom_{s\calc}(\mathbb{S}^{\calc,t}_{s_{r+1},\ldots,s_1},X)&\cong (ZN_{s_{r+1}}X)_{s_r,\ldots,s_1}^t\makebox[0cm][l]{ and}\\
\hom_{s\calc}(C\mathbb{S}^{\calc,t}_{s_{r+1},\ldots,s_1},X)&\cong (N_{s_{r+1}+1}X)_{s_r,\ldots,s_1}^t,
\end{alignat*}
and indeed $i^*$ plays the same role as above, representing the differential of $N_*X$. Moreover, in the homotopy category corresponding to the above model category structure, $\mathbb{S}^{\calc,t}_{n,s_r,\ldots,s_1}$ represents $\pi_n(\DASH)^t_{s_r,\ldots,s_1}$ (c.f.\ \citeBOX[\S1]{MR1089001} or \citeBOX[\S3.1.1]{Blanc_Stover-Groth_SS.pdf}) which is why  we  refer to the objects $\mathbb{S}^{\calc,t}_{n,s_r,\ldots,s_1}$ as spheres.

\SubsectionOrSection{Homotopy groups and $\calc$-$\Pi$-algebras}\label{homotopy and pialgs}
By virtue of the algebraic structure possessed by $X$, the homotopy groups $\pi_*X$ possess certain natural algebraic structure, that of a $\calc$-$\Pi$-algebra. Indeed, as any given homotopy group is a representable functor on the homotopy category, natural $N$-ary operations on homotopy groups
\[(\pi_*X)_{s_{r+1}^{1},\ldots,s_{1}^{1}}^{t^1}\times\cdots \times(\pi_*X)_{s_{r+1}^{N},\ldots,s_{1}^{N}}^{t^N}\to (\pi_*X)_{s_{r+1},\ldots,s_{1}}^{t}\]
are in bijective correspondence with elements of the group
\[ \pi_*\Bigl(\mathbb{S}^{\calc,t^1}_{s_{r+1}^{1},\ldots,s_{1}^{1}}\sqcup\cdots\sqcup \mathbb{S}^{\calc,t^N}_{s_{r+1}^{N},\ldots,s_{1}^{N}}\Bigr)\makebox[0cm][l]{}_{s_{r+1},\ldots,s_{1}}^{t}.\]
Blanc and Stover \cite{Blanc_Stover-Groth_SS.pdf} define a new category of graded universal algebras, the category $\PA{\calc}$ of $\calc$-$\Pi$-algebras, monadic over $\vect{+}{r+1}$, whose objects are graded vector spaces $V\in\vect{+}{r+1}$ with a structure map 
\[V_{s_{r+1}^{1},\ldots,s_{1}^{1}}^{t^1}\times\cdots \times V_{s_{r+1}^{N},\ldots,s_{1}^{N}}^{t^N}\to V_{s_{r+1},\ldots,s_{1}}^{t}\]
for every such homotopy class, satisfying certain natural compatibilities.

It is a standard formalism to encode these compatibilities as follows. A \emph{model}  \cite{Blanc_Stover-Groth_SS.pdf} in $s\calc$ is 
an almost free object of $s\calc$ which is weakly equivalent to a coproduct of spheres (for example, $F^{\calc}\Gamma V$ for any $V\in\vect{+}{r+1}$ viewed as a chain complex with zero differential). A \emph{finite model} is a model in which this coproduct is finite. Let $\Pi$  be the $\vect{}{}$-enriched category with objects the finite models in $s\calc$, and morphisms
\[\hom_{\Pi}(M,M'):=\hom_{\textup{ho}(s\calc)}(M,M').\]
Then the category of $\calc$-$\Pi$-algebras  may be defined as the category of $\vect{}{}$-enriched functors $\Pi^\textup{op}\to \vect{}{}$ that send finite coproducts into products (where by $\vect{}{}$ we mean the category of ungraded $\Ftwo $-vector spaces). The category of $\calc$-$\Pi$-algebras is monadic over $\vect{+}{r+1}$, with  forgetful functor $U^{\PA{\calc}}A$ defined on a functor $A\in \PA{\calc}$ by:%$A$ under the forget Such a functor $A$ yields an object of $U^{\PA{\calc}}A$ of $\vect{+}{r+1}$ via
\[(U^{\PA{\calc}}A)^t_{s_{r+1},\ldots,s_1}:=A(\mathbb{S}^{\calc,t}_{s_{r+1},\ldots,s_{1}}).\]
Any operators advertised above on $U^{\PA{\calc}}A$ is then induced by the corresponding homotopy class of a coproduct of spheres, viewed as a map in $\Pi$. %The natural compatibilities referred to above are the conditions required that a collection of structure maps extends to a functor on $\Pi^\textup{op}$.

One obtains the free $\calc$-$\Pi$-algebra on a graded vector space $V\in \vect{+}{r+1}$ using Dold's theorem (\ref{Dold's theorem}). That is, one views $V$ as a chain complex in $\complexes\vect{+}{r}$ with zero differential, and applies the Dold-Kan correspondence and $\calc$-free functor, obtaining an object $F^\calc\Gamma V\in s\calc$. Then:
\[F^{\PA{\calc}}V=\pi_*(F^\calc\Gamma V).\]
Moreover, as $F^\calc $ is an augmented monad, so is $F^{\PA{\calc}}$, via the map
\[F^{\PA{\calc}}V=\pi_*(F^\calc\Gamma V)\overset{\pi_*\epsilon}{\to}\pi_*(\Gamma V)=H_*V=V,\]
and in particular, there is an adjunction $Q^{\PA{\calc}}\dashv K^{\PA{\calc}}$.
%For any simplicial object $A\in s\calc$, there is a natural map $\gamma:Q^{\PA{\calc}}\pi A\to \pi Q^{\calc}A$.

The theory above has the upshot that understanding the category $\PA{\calc}$ is equivalent to calculating the homotopy groups of the finite models. In many cases, this can be performed by calculating the homotopy of individual spheres, and then using a \emph{Hilton-Milnor theorem} (\S\ref{Homotopy operations for simplicial Lie algebras}) or \emph{K\"unneth theorem} (Proposition \ref{salgs homotopy kunneth}) to bootstrap up to a calculation on all finite models.



\begin{lem}
\label{Q of a model}
For any model $A$ in $s\calc$, the Hurewicz map $\pi_*A\to \pi_*Q^\calc A$ descends to an isomorphism
\[\gamma:Q^{\PA{\calc}}\pi_* A\to \pi_* Q^{\calc}A\cong H_*^{\calc}A.\]
%If $A$ is a model in $s\calc$ then $\gamma$ is an isomorphism.
\end{lem}
\begin{proof} $A$ is (cofibrant and) homotopic to a coproduct of spheres, and as such may be taken to be equal to a coproduct of spheres. As $\pi_* A$ is free on generators in correspondence with the sphere summands, $Q^{\PA{\calc}}\pi_* A$ is simply the vector space with basis their fundamental classes, which is isomorphic to $H_*^{\calc}A$.
\end{proof}
\SubsectionOrSection{Cohomology groups and $\calc$-$H^*$-algebras}\label{cohomology and Halgs}
It will in general be preferable for us to consider algebraic structure on cohomology, rather than coalgebraic structure on homology: algebra is in general a more familiar subject than coalgebra, and cohomology has the advantage that it consists of representable functors. 
Another advantage is that this theory is dual to that of homotopy groups and $\calC$-$\Pi$-algebras. On the other hand, using cohomology groups has the disadvantages associated with double-dualization.

For any set of indices $t\geq0$ and $s_{r+1},\ldots,s_1\geq0$, write:
\begin{alignat*}{2}
\mathbb{K}^{\calc,t}_{s_{r+1},\ldots,s_1}
&:=
K^\calc \mathbb{K}^{t}_{s_{r+1},\ldots,s_1};\textup{ and}
\\
% Left hand side
C\mathbb{K}^{\calc,t}_{s_{r+1},\ldots,s_1}
% Relation
&:=
% Right hand side
K^\calc C\mathbb{K}^{t}_{s_{r+1},\ldots,s_1}%
% Comment
\end{alignat*}
The $\mathbb{K}^{\calc,t}_{s_{r+1},\ldots,s_1}$ are the \emph{Eilenberg-Mac Lane objects in $s\calc$}. In the homotopy category of $s\calc$, the object $\mathbb{K}^{\calc,t}_{n,s_r,\ldots,s_1}$ represents the contravariant functor $H^n_{\calc}(\DASH)_t^{s_r,\ldots,s_1}:s\calc\to\vect{}{}$,  c.f.\ \cite[Proposition 4.3]{MR1089001}. 


%The forgetful functor $U^\calc:\calc\to \vect{+}{r}$ used to define the weak equivalences and fibrations in $s\calc$, allows us to consider any object $X\in s\calc$ as an object of $s\vect{+}{r}$. In particular, for any $X\in s\calc$ we may define homotopy groups
%We have defined the cohomology of $X\in s\calc$ to be the dual left-derived functors of the indecomposables functor $Q^\calc:\calc\to \vect{+}{r}$,
%\[(H^{s_{r+1}}_\calc X)^{s_r,\ldots,s_1}_t:=((H_{s_{r+1}}^\calc X)_{s_r,\ldots,s_1}^t)^*:=(\pi_{s_{r+1}}(Q^{\calc}B^\calc X)_{s_r,\ldots,s_1}^t)^*,\]
%which we view together as an object of $\vect{r+1}{+}$.


%In light of this fact, the homotopy groups of $X$ possess
By virtue of the algebraic structure possessed by $X$, the cohomology groups $H_\calc^*X$ possess certain natural algebraic structure, that of a $\calc$-$H^*$-algebra.  As for $\calc$-$\Pi$-algebras, natural $N$-ary operations on cohomology groups
\[(H^*_\calc X)^{s_{r+1}^{1},\ldots,s_{1}^{1}}_{t^1}\times\cdots \times(H^*_\calc X)^{s_{r+1}^{N},\ldots,s_{1}^{N}}_{t^N}\to (H^*_\calc X)^{s_{r+1},\ldots,s_{1}}_{t}\]
are in bijective correspondence with elements of the group
\[ H^*_\calc\Bigl(\mathbb{K}_{s_{r+1}^{1},\ldots,s_{1}^{1}}^{\calc,t^1}\times\cdots\times \mathbb{K}_{s_{r+1}^{N},\ldots,s_{1}^{N}}^{\calc,t^N}\Bigr)\makebox[0cm][l]{}^{s_{r+1},\ldots,s_{1}}_{t}.\]
The category of $\calc$-$H^*$-algebras, monadic over $\vect{r+1}{+}$, has objects graded vector spaces $V\in\vect{r+1}{+}$ with a structure map 
\[V^{s_{r+1}^{1},\ldots,s_{1}^{1}}_{t^1}\times\cdots \times V^{s_{r+1}^{N},\ldots,s_{1}^{N}}_{t^N}\to V^{s_{r+1},\ldots,s_{1}}_{t}\]
for every such cohomology class, satisfying certain natural compatibilities.

The formalism required to express these compatibilities is as follows. A \emph{generalized Eilenberg-Mac Lane object}, or \emph{GEM}, in $s\calc$ is an almost free object of $s\calc$ which is weakly equivalent to a product of Eilenberg-Mac Lane objects  $\mathbb{K}_{s_{r+1},\ldots,s_{1}}^{\calc,t}$. A \emph{finite GEM} is a GEM in which this product is finite. Let $\mathbb{K}$  be the $\vect{}{}$-enriched category with objects the finite GEMs in $s\calc$, and morphisms
\[\hom_{\mathbb{K}}(M,M'):=\hom_{\textup{ho}(s\calc)}(M,M').\]
Then the category of $\calc$-$H^*$-algebras  may be defined as the category of $\vect{}{}$-enriched functors $\mathbb{K}\to \vect{}{}$ that preserve finite products. The category of $\calc$-$H^*$-algebras is monadic over $\vect{r+1}{+}$, with forgetful functor defined  on a functor $h:\mathbb{K}\to \vect{}{}$  by:%$A$ under the forget Such a functor $A$ yields an object of $U^{\HA{\calc}}A$ of $\vect{r+1}{+}$ via
\[(U^{\HA{\calc}}h)_t^{s_{r+1},\ldots,s_1}:=h(\mathbb{K}_{s_{r+1},\ldots,s_{1}}^{\calc,t}),\]
and any of the structure maps advertised above is induced by the corresponding cohomology class. %The natural compatibilities referred to above are the conditions required that a collection of structure maps extends to a functor on $\Pi^\textup{op}$.

One obtains the free $\calc$-$H^*$-algebra on a graded vector space $V\in \vect{r+1}{+}$ \emph{of finite type} as follows. One views $\dual V$ as a chain complex in $\complexes\vect{+}{r}$ with zero differential, and applies the Dold-Kan correspondence and $K^\calc$, obtaining an object $K^\calc\Gamma \dual V\in s\calc$. Then:
\[F^{\HA{\calc}}V=H^*_{\calc}K^\calc\Gamma\dual V.\]
Moreover, $F^{\HA{\calc}}$ is an augmented monad: one applies $H^*_{\calc}$ to the natural collapse map $F^{\calc}\Gamma V\to K^{\calc}\Gamma V$, to obtain
\[K^{\HA{\calc}}V\cong H^*_{\calc} F^{\calc}\Gamma \dual V\from H^*_{\calc} K^{\calc}\Gamma \dual V=:F^{\HA{\calc}}V.\]
%\[F^{\HA{\calc}}V=\pi_*F^\calc\Gamma V\overset{\pi_*\epsilon}{\to}\pi_*\Gamma V=H_*V=V,\]
and in particular, there is an adjunction $Q^{\HA{\calc}}\dashv K^{\HA{\calc}}$.
%For any simplicial object $A\in s\calc$, there is a natural map $\gamma:Q^{\HA{\calc}}\pi A\to \pi Q^{\calc}A$.

These definitions simplify when we apply them to the dual of a vector space $U\in\vect{+}{r+1}$ of finite type:
\[F^{\HA{\calc}}(\dual U):= \pi^*\dual Q^{\calc}cK^\calc\Gamma \dual^{2}U\overset{\cong}{\from}\pi^*\dual Q^{\calc}cK^\calc\Gamma U\overset{\cong}{\to} \dual  \pi_*Q^{\calc}cK^\calc\Gamma U\]
suggesting that the functor $F^{\HA{\calc}}$ is altogether of the wrong variance. It is preferable to work with the functor
\[C^{\HC{\calc}}U:=\pi_*Q^{\calc}cK^\calc\Gamma U\]
discussed in \S\ref{homology and Hcoalgs}.



To dualize a paragraph from \S\ref{homotopy and pialgs}: the theory above has the upshot that understanding the category $\HA{\calc}$ is equivalent to calculating the cohomology groups of finite GEMs. In many cases, this can be performed by calculating the cohomology of individual Eilenberg-Mac Lane objects, and then using a \emph{Hilton-Milnor theorem} (\citeBOX[\S11]{MR1089001} and \S\ref{The example of simplicial commutative F2-algebras}) or \emph{K\"unneth theorem} (theorems \ref{Prop on cohomology of product of finite lie gems} and \ref{Koszul-dual Hilton-Milnor theorem}) to bootstrap up to a calculation on all finite GEMs.

\SubsectionOrSection{The reverse Adams spectral sequence}
\label{reverse Adams spectral sequence}
We will now give a description of Miller's reverse Adams spectral sequence \citeBOX[\S4]{MillerSullivanConjecture.pdf}, which was used by Goerss \cite[Chapter V]{MR1089001} to calculate the cohomology of Eilenberg-Mac Lane objects in $s\algs$.

Suppose that $X\in s\calC$, and consider the \emph{bisimplicial} object $Q^{\calc}(B^{\calc}_pX)_{q}\in ss\vect{+}{n}$. There is a first quadrant cohomotopy spectral sequence
\[\Edown{2}{\,Q^{\calc}B^{\calc}X}{p,q,s_n,\ldots,s_1}{t}=\pi\dhor^{p}\pi\dver^{q}(\dual Q^{\calc}B^{\calc}X)^{s_n,\ldots,s_1}_{t}\]
converging to $H^*_{\calc}X:=\pi^{p+q}\dual Q^{\calc}\diag{B^{\calc}X}$. For each fixed $p$, %Lemma \ref{Q of a model} shows 
%\[\pi\dver^{*}(\dual Q^{\calc}B^{\calc}_p X)=\dual \pi\uver_{*}(Q^{\calc}B^{\calc}_p X)=\dual Q^{\PA{\calc}}\pi\uver_{*}(B^{\calc}_p X)\]
\begin{alignat*}{2}
\pi^{*}(\dual Q^{\calc}B^{\calc}_p X)&\cong\dual \pi_{*}(Q^{\calc}B^{\calc}_p X)\\
&\cong\dual Q^{\PA{\calc}}\pi_{*}(B^{\calc}_p X)\\
&\cong\dual Q^{\PA{\calc}}B^{\PA{\calc}}_p \pi_{*}(X),{}
\end{alignat*}
where the second isomorphism is that of Lemma \ref{Q of a model}, so that 
\[\Edown{2}{\,Q^{\calc}B^{\calc}X}{p,q,s_n,\ldots,s_1}{t}\cong (H^{p}_{\PA{\calc}}\pi_*X)^{q,s_n,\ldots,s_1}_{t}.\]

When $\calc=\algs$, this is precisely the spectral sequence used by Goerss in \cite[Chapter V]{MR1089001} to calculate the cohomology of Eilenberg-Mac Lane objects in $s\algs$. In this thesis, we will use this spectral sequence only for certain low-dimensional calculations. Goerss equipped the reverse Adams spectral sequence with certain spectral sequence operations \citeBOX[\S14]{MR1089001}, work which can be framed using the external operations, due to Singer, reprised in \S\ref{Operations in composite functor spectral sequences}. %Indeed, these operations on the reverse Adams spectral sequence make for an interesting point of comparison for the operations we present in \S\ref{Operations in composite functor spectral sequences}. \textbf{some are probably the same.}%, which arise in a composite functor spectral sequence with the same target as the unstable Adams s. We will define various spectral sequences, and construct spectral sequence operations whic

In this thesis we study a Bousfield-Kan spectral sequence (BKSS), which is also known as an unstable Adams spectral sequence, for the category $\algs$. The operations defined in \S\ref{Operations on the Bousfield-Kan spectral sequence} for this spectral sequence make for another point of comparison. Loosely, we find that the operations on the \BKSS\ are in a sense Koszul dual to the operations on the reverse Adams spectral sequence.

\SubsectionOrSection{The smash coproduct}\label{subseq:The smash coproduct}
For $X_1$ and $X_2$ objects of any algebraic category, for example $\calc$, $\PA{\calc}$ or $\HA{\calc}$ (to be defined shortly), we define the \emph{smash coproduct} $X_1\smashcoprod X_2$ to be the kernel of the natural map $X_1\sqcup X_2\to X_1\times X_2$. When $X_1=X_2=X$, $X\smashcoprod X$ has a natural action of $\Sigma_2$, and we write $X\smashcoprod^{\Sigma_2} X$ for the subobject of invariant elements under this action.

When $X_1$ and $X_2$ are  objects of $s\calc$, taking this strict fiber is in fact homotopically correct, since the map $X_1\sqcup X_2\to X_1\times X_2$ is always a fibration, and indeed:
\begin{prop}
\label{smash coprod}
For $X_1$ and $X_2$ in $s\calc$, the natural $\calc$-$\Pi$-algebra map
\[\pi_*(X_1\times X_2)\to \pi_* X_1\times \pi_* X_2\]
 is an isomorphism. If $X_1$ and $X_2$ are models in $s\calc$, the natural $\calc$-$\Pi$-algebra map 
\[\pi_* X_1\sqcup \pi_* X_2\to\pi_*(X_1\sqcup X_2)\]
takes part in an isomorphism of short exact sequences:
\[\xymatrix@R=4mm{
0\ar[r]&%r1c1
\pi_* X_1\smashcoprod \pi_* X_2
\ar[r]\ar[d]_-{\cong}
%\ar[d]^-{i}_-{\cong}
&%r1c2
\pi_* X_1\sqcup \pi_* X_2
\ar[r]
\ar[d]_-{\cong}\ar[d]_-{\cong}
&%r1c3
\pi_* X_1\times \pi_* X_2
\ar[r]
\ar[d]_-{\cong}
&%r1c4
0\\%r1c5
0\ar[r]&%r1c1
\pi_* (X_1\smashcoprod  X_2)
\ar[r]
&%r1c2
\pi_* (X_1\sqcup X_2)
\ar[r]
&%r2c3
\pi_* (X_1\times X_2)
\ar[r]&0
}\]
\end{prop}
\begin{proof}
The first claim is easy: the forgetful functor is a right adjoint, and $\pi_*$ preserves products (of vector spaces). Consider the commuting diagram
\[\xymatrix@R=4mm{
0\ar[r]&%r1c1
\pi_* X_1\smashcoprod \pi_* X_2
\ar[r]
%\ar[d]^-{i}_-{\cong}
&%r1c2
\pi_* X_1\sqcup \pi_* X_2
\ar[r]
\ar[d]^-{i}
&%r1c3
\pi_* X_1\times \pi_* X_2
\ar[r]
\ar[d]_-{\cong}
&%r1c4
0\\%r1c5
&%r1c1
\pi_* (X_1\smashcoprod  X_2)
\ar[r]
&%r1c2
\pi_* (X_1\sqcup X_2)
\ar[r]
&%r2c3
\pi_* (X_1\times X_2)
}\]
in which the top row is a short exact sequence, and the bottom row is just a three term excerpt of the homotopy long exact sequence of the fiber sequence defining $X_1\smashcoprod X_2$. If $i$ were an isomorphism, the bottom row would also be short exact, and a simple diagram chase would show that $i$ restricts to the isomorphism we desire.


If $X_1$ and $X_2$ are models, the displayed map $i$ is an isomorphism, since both source and target represent the free $\calc$-$\Pi$-algebra on generators corresponding to the sphere summands of $X_1$ and $X_2$ taken together. 
\end{proof}

\SubsectionOrSection{Cofibrant replacement via the small object argument}\label{Cofibrant replacement via the small object argument}
The homotopy of an object $X$ of $s\calc$ was defined simply by application of the forgetful functor $U^{\calc}:\calc\to\vect{}{}$, a definition which is tautologically homotopically correct. On the other hand, in order to define the homology $H_*^{\calc}X$, as the left Quillen functor $Q^{\calc}$ does not preserve all weak equivalences, we must perform a cofibrant replacement before applying $Q^{\calc}$. While the comonadic bar construction $B^{\calc}$ described in \S\ref{ssec: quillen model and bar construction} suffices to define the groups $H_*^{\calc}X$, it lacks the structure that we will need at various points in this thesis.

Radulescu-Banu's innovation \cite{Radulescu-Banu.pdf} first explained that the cofibrant replacement functor $c:s\calc\to s\calc$ constructed by Quillen's \emph{small object argument}  \cite{QuillenHomAlg.pdf}, which by design already possesses a natural acyclic fibration $\epsilon:c\to\Id$, in fact admits the full structure of a comonad, with diagonal $\beta:c\to cc$. As explained by Blumberg and Riehl \cite[Remark 4.12]{BlumRiehlResolutions.pdf}:
\begin{prop}
\label{QcK is a comonad}
The endofunctor $Q^\calc cK^\calc$ of $s\vect{}{}$ admits the structure of a comonad, via the maps
\[Q^\calc cK^\calc\overset{\smash{Q^\calc (\beta)}}{\to}Q^\calc ccK^\calc \overset{\smash{Q^\calc c(\eta)}}{\to} Q^\calc cK^\calc Q^\calc cK^\calc\text{ \ and \ }Q^\calc cK^\calc\overset{\smash{Q^\calc (\epsilon)}}{\to}Q^\calc K^\calc\cong\Id,\]
where $\eta$ denotes the unit of the $Q^\calc\dashv K^\calc$ adjunction.
\end{prop}

The functor $c$ of the small object argument depends on the choice of sets of generating cofibrations and acyclic cofibrations. It will be important to our applications that included in this set are certain important cofibrations, namely the inclusion of $0$ into any sphere $\mathbb{S}^{\calc,t}_{s_{r+1},\ldots,s_1}$, and any of the cofibrations  defined in \S\ref{three cell complex} and \S\ref{two-cell complex for the deltas}. 

It will be helpful to have included these maps, because of the following facts about the small object argument functor $cX$. It is constructed as the colimit of a (transfinite) sequence of cofibrations:
\[\xymatrix@!0@R=10mm@C=13mm{
\makebox[0cm][r]{$0={}$}c_0X\ar[rr] \ar@{ >->}[drrr]
&&%r1c1
c_1X\ar@{ >->}[rr]\ar[dr]
&&%r1c2
c_2X\ar@{ >->}[rr]\ar[dl]
&&%r1c3
c_3X\ar@{ >->}[rr]\ar[dlll]
&&%r1c4
\cdots \\
&&&X\!
}\]
and given an element $f:A\to B$ of the chosen set of generating cofibrations and a commuting square
\[\xymatrix@R=4mm{
A\ar[r]\ar@{ >->}[d]^-{f}&%r1c1
c_nX\ar[d]\\%r1c2
B\ar[r]&%r2c1
X%r2c2
}\]
there is a canonical choice of map $B\to c_{n+1}X$ making
\[\xymatrix@R=4mm{
A\ar[r]\ar@{ >->}[dd]^-{f}&%r1c1
c_nX\ar[d]\\%r1c2
&
c_{n+1}X\ar[d]\\%r1c2
B\ar[r]\ar[ur]&%r2c1
X%r2c2
}\]
commute. Indeed, the map $c_nX\to c_{n+1}X$ is constructed by attaching a copy of $B$ along the image of $A$ in $c_nX$, for each such commuting square.

We will use this canonical lift later, so establish a little notation. There is a function
\[\hom_{s\algs}(\mathbb{S}^{\calc}_t,X)\to \hom_{s\algs}(\mathbb{S}^{\calc}_t,c_1X)\]
denoted $\alpha\mapsto \widetilde{\alpha}$, natural in $X\in s\calc$, and providing a section of
\[\hom_{s\algs}(\mathbb{S}^{\calc}_t,cX)\overset{\epsilon_*}{\to} \hom_{s\algs}(\mathbb{S}^{\calc}_t,X)\]
 where we define $\widetilde{\alpha}$ to be the canonical lift corresponding to the square
\[\xymatrix@R=4mm{
0\ar[r]\ar@{ >->}[d]&%r1c1
c_0X\makebox[0cm][l]{${}:=0$}\ar[d]\\%r1c2
\mathbb{S}^{\calc}_t\ar[r]^-{\alpha}&%r2c1
X%r2c2
}\]

Finally, we note that Radulescu-Banu's construction has a convenient consequence (albeit not crucial) for the construction of homotopy cofibers in $s\calc$. Quillen's small object argument actually provides a functorial factorization
\[\xymatrix@R=4mm{
X\ar@{ >->}[r]
&%r1c1
c^\textup{fac}(g)\ar@{->>}[r]^-{\sim}
& Y.
}\]
 of any map $g:X\to Y$, and in this notation, one might say that we have been writing $cX$ as shorthand for $c^\textup{fac}(0\to X)$. There is a commuting square
\[\xymatrix@R=4mm{
0\ar@{ >->}[r]\ar[d]
&%r1c1
cY\ar@{=}[d]
\\%r1c2
cX\ar[r]^{cg}&%r2c1
cY%r2c2
}\]
which by functoriality induces a commuting diagram (ignoring the dashed map):
\[\xymatrix@R=4mm{
0\ar@{ >->}[r]\ar[d]
&
ccY\ar@{->>}[r]^-{\epsilon_{cY}}
\ar[d]_-{m}
&%r1c1
cY\ar@{=}[d]
\\%r1c2
cX\ar@{ >->}[r]&%r2c1
c^\textup{fac}(cg)\ar@{->>}[r]_-{\sim}
&
cY\ar@{-->}[ul]|-{\beta}
%r2c2
}\]
Radulescu-Banu's diagonal $\beta$ is the dotted map in this diagram, and as it is a comonad diagonal $\epsilon_{cY}\circ \beta=\Id_{cY}$, so that $m\circ \beta$ is a section of the acyclic fibration $c^\textup{fac}(cg)\overset{\sim}{\epi} cY$.

We may define the homotopy cofiber as the pushout
\[\xymatrix@R=4mm{
cX\ar@{ >->}[r]\ar[d]&%r1c1
c^\textup{fac}(cg)\ar[d]\\%r1c2
0\ar[r]&%r2c1
\hocof(g)%r2c2
}\]
and by virtue of the construction just given:
\begin{prop}
\label{hocof no ziggy-zaggy}
There is a natural construction in $s\calc$ of the homotopy cofiber $\hocof(g)$ of a map $g:X\to Y$, implemented by  natural maps
\[cX\overset{cg}{\to} cY\to \hocof(g).\]
\end{prop}
\noindent This is in contrast to the standard situation, where there is at best a natural zig-zag, even from $cY$ to  $\hocof(g)$.
\SubsectionOrSection{Homology groups and $\calc$-$H_*$-coalgebras}\label{homology and Hcoalgs}
There is a commuting diagram
\[\xymatrix@R=4mm@C=20mm{
s\vect{+}{r}\ar[r]^-{Q^\calc cK^\calc }
\ar[d]^-{\pi_*}
&%r1c1
s\vect{+}{r}\ar[d]^-{\pi_*}
\\%r1c2
\vect{+}{r+1}\ar[r]^-{C^{\HC{\calc}}}
&
\vect{+}{r+1}
}\]
in which we are using Dold's theorem (\ref{Dold's theorem}) to \emph{define} $C^{\HC{\calc}}$, the \emph{cofree $\calc$-$H_*$-coalgebra comonad}.
By Proposition \ref{QcK is a comonad} and the naturality of Dold's theorem, this is a comonad on $\vect{+}{r+1}$. A $\calc$-$H_*$-coalgebra is simply a coalgebra over this monad, i.e.\ any $h\in\vect{+}{r+1}$ equipped with a coaction map $h\to C^{\HC{\calc}}h$ satisfying the standard compatibilities. The homology $H_*^\calc X$ of $X\in s\calc$ is a $\calc$-$H_*$-coalgebra using the map
\[\pi_*(Q^{\calc}cX)\overset{\pi_*(Q^{\calc}(\beta))}{\to}\pi_*(Q^{\calc}ccX)\overset{\pi_*(Q^{\calc}c(\eta))}{\to}\pi_*(Q^{\calc}cK^{\calc}Q^{\calc}cX)=C^{\HC{\calc}}(\pi_*(Q^{\calc}cX)).\]
If $X\sim K^{\calc}V$ for some $V\in s\vect{+}{r}$, then $H_*^{\calc}X\cong C^{\HC{\calc}}(\pi_*(V))$, and the coaction map of $H_*^{\calc}X$ is none other than the diagonal map of the comonad.

The comparison maps of \S\ref{cohomology and Halgs} give the dual of a $\calc$-$H_*$-algebra of finite type a $\calc$-$H^*$-algebra structure. 
\begin{prop}
\label{something about dualization}
If $V\in \vect{}{}$ and $X,X'\in \HC{\calc}$ are of finite type, then there are natural isomorphisms:
\begin{gather*}
\dual C^{\HC{\calc}}V\cong F^{\HA{\calc}}\dual V;\\
Q^{\HA{\calc}}\dual X\cong \dual \Prim^{\HC{\calc}} X;\\
\dual(X\times X')\cong \dual X\sqcup \dual X';\makebox[0cm][l]{}\\
\dual(X\smashprod X')\cong \dual X\smashcoprod \dual X',
\end{gather*}
where the primitives $\Prim^{\HC{\calc}}X$ are to be defined \S\ref{The Hurewicz map, primitives and homology completion}, and the smash product $X\smashprod X'$ in \S\ref{subseq:The smash product}.
\end{prop}
\SubsectionOrSection{The Hurewicz map, primitives and homology completion}\label{The Hurewicz map, primitives and homology completion}
For any $X\in s\calc$, there is a map $\pi_*X\to H_*^{\calc}X$, the \emph{Hurewicz map}, defined as the composite
\[\pi_*X\cong \pi_*(cX)\to \pi_*(Q^{\calc}cX).\]
Indeed, the Hurewicz map provides a coaugmentation of the comonad $C^{\HC{\calc}}$, the natural transformation $a:\Id\to C^{\HC{\calc}}$ of endofunctors of $\vect{+}{r+1}$ defined by
\[V\cong \pi_*(cK^{\calc}\Gamma  V)\to \pi_*(Q^{\calc}cK^{\calc}\Gamma  V)=F^{\PA{\calc}}V.\]
One reading of this observation is:
\begin{lem}
\label{hurewicz is a section}
If $X\in s\calc$ is in the image of $K^\calc$, then $Q^\calc X=U^{\calc}X$, and the Hurewicz map of $X$ is a section of the composite
\[H_*^{\calc}X:=\pi_*Q^{\calc}cX\overset{(Q\epsilon)_*}{\to}\pi_*Q^{\calc}X=\pi_*X.\]
\end{lem}
Given that the comonad $C^{\HC{\calc}}$ has a coaugmentation, we may define the \emph{primitives} of a $\calc$-$H_*$-coalgebra $H$ as the equaliser (of graded vector spaces):
\[\xymatrix@R=4mm@1{
\Prim^{\HC{\calc}}(H)\ar[r]
&%r1c1
H\ar@<.5ex>[r]^-{a}
\ar@<-.5ex>[r]_-{\textup{coact}}
&%r1c2
CH%r1c3
}.\]
%$\pi_*(V)\cong \pi_*(cKV)\to\pi_*(QcKV)$, in the sense that $\epsilon\circ a=1$ and $a\circ a=\Delta\circ a$. In particular, the monomorphism $a$ presents $W$ as a subspace of $CW$, and if $H$ is any $C$-coalgebra, we may define the \emph{primitives} in $H$ 
%``Include a bit about colimits, as in \verb|adamsops|''(?x?x?x?x?)
%\begin{prop}
%For $W$ a graded vector space, the map $a:W\to CW$ is an isomorphism onto the primitives $\Prim(CW)$ of the cofree $C$-coalgebra $CW$.
%\end{prop}
We will briefly defer the proof of:
\begin{prop}
\label{hurewicz}
The Hurewicz map $\pi_*X\to H_*^{\calc}X$ factors through $\Prim^{\HC{\calc}}(H_*^{\calc}X)$, and if $X$ is GEM, the resulting map $\pi_*X\to\Prim^{\HC{\calc}}(H_*^\calc X)$ is an isomorphism. In particular, for any $V\in\vect{+}{r+1}$, $\Prim^{\HC{\calc}}(C^{\HC{\calc}}V)\cong V$.
\end{prop}
%\begin{proof}
%Represent an element $\alpha\in\pi_nA$ by a map $\alpha:S^n_\calc\to A$. Then, $h(\alpha)$ is the image of the fundamental homology class $\imath\in H_n(S^n_\calc)$ under the map $\alpha_*$. As $\imath$ is primitive, so is $h(\alpha)$.
%\end{proof}

Radulescu-Banu \cite{Radulescu-Banu.pdf} has constructed a cosimplicial resolution $\calx^\bullet$ of an object $X\in s\calc$ by GEMs, and defined the \emph{homology completion of $X$} to be the totalization $X\hat{\ }:=\textup{Tot}(\calx^\bullet)$. 

This construction is the analogue of Bousfield and Kan's $R$-completion functor on simplicial sets \cite{BousKanSSeq.pdf}, a construction that has proven extremely useful in classical homotopy theory.
There is an additional difficulty, however,  in constructing the cosimplicial resolution $\calx^{\bullet}$, which is not present in the classical context: since not all simplicial algebras are cofibrant, the na\"{\i}ve cosimplicial resolution (with coaugmentation dashed)
\[\makebox[0cm][r]{\,$\qquad $}\vcenter{
\def\labelstyle{\scriptstyle}
\xymatrix@C=1.5cm@1{
X\,
\ar@{-->}[r]
&
\,K^{\calc}Q^{\calc}X\,
\ar[r];[]
&
\,(K^{\calc}Q^{\calc})^2X\,
\ar@<-.65ex>[l];[]
\ar@<+.65ex>[l];[]
\ar@<+.65ex>[r];[]
\ar@<-.65ex>[r];[]
&
\,(K^{\calc}Q^{\calc})^3X\,\makebox[0cm][l]{\,$\cdots. $}
\ar[l];[]
\ar@<-1.3ex>[l];[]
\ar@<+1.3ex>[l];[]
}}\]
fails to be homotopically correct, and  as $Q^{\calc}K^{\calc}=\Id$, fails to hold any interest whatsoever.

Radulescu-Banu's innovation was to explain that the cofibrant replacement functor $c:s\calc\to s\calc$ constructed by Quillen's small object argument \cite{QuillenHomAlg.pdf} admits a comonad diagonal $\beta:c\to cc$  (already used in \S\ref{homology and Hcoalgs}) and can thus be mixed into the cosimplicial resolution, making it homotopically correct. Specifically, the diagonal is needed in order  to define the coface maps in Radulescu-Banu's resolution, the coaugmented cosimplicial object
\[\makebox[0cm][r]{\,$\calx^\bullet:\qquad $}\vcenter{
\def\labelstyle{\scriptstyle}
\xymatrix@C=1.5cm@1{
cX\,
\ar@{-->}[r]
&
\,cK^{\calc}Q^{\calc}cX\,
\ar[r];[]
&
\,c(K^{\calc}Q^{\calc}c)^2X\,
\ar@<-.65ex>[l];[]
\ar@<+.65ex>[l];[]
\ar@<+.65ex>[r];[]
\ar@<-.65ex>[r];[]
&
\,c(K^{\calc}Q^{\calc}c)^3X\,\makebox[0cm][l]{\,$\cdots. $}
\ar[l];[]
\ar@<-1.3ex>[l];[]
\ar@<+1.3ex>[l];[]
}}\]
By an application of Dold's theorem (\ref{Dold's theorem}), if $X\to Y$ is a weak equivalence, so is $\calx^s\to\calY^s$ for each $s$. Both $\calx$ and $\calY$, being group-like,  are automatically Reedy fibrant (cf.\ \citeBOX[{X.4.9}]{YellowMonster}), so that the map of completions $X\hat{\ }\to Y\hat{\ }$ is a weak equivalence.
This construction is explained and generalized by Blumberg and Riehl \citeBOX[\S4]{BlumRiehlResolutions.pdf}: instead of simply using the unit and counit of the adjunction respectively, one uses the composites discussed in \S\ref{homology and Hcoalgs}:
\[c\overset{\beta}{\to}cc\overset{c\eta c}{\to}cK^{\calc}Q^{\calc}c\textup{\quad and\quad }Q^{\calc}cK^{\calc}\overset{Q\epsilon K}{\to}Q^{\calc}K^{\calc}\to \textup{id}.\]
Comments in \citeBOX[\S4]{BlumRiehlResolutions.pdf}  show that the coaugmented cosimplicial $\calc$-$H_*$-coalgebra $H_*^{\calc}\calx^\bullet$  is weakly equivalent to its coaugmentation $H_*^{\calc}X$ \emph{as a vector space}, which starts to explain  the title \emph{homology completion}. One says that $X$ is \emph{homology complete} when the map $cX\to X\hat{\ }:=\textup{Tot}(\calx^\bullet)$ is an equivalence.
In \S\ref{sec:connectivityAnalysis} we will specialize to the case when $\algcat$ is either the category $\algs$  of ungraded non-unital commutative algebras the category $\restliealgs$ of ungraded restricted Lie algebras, and prove that the completion $X\hat{\ } $ is weakly equivalent to $X$ when $X$ is connected. Analogous results for topological Quillen homology may be found in \cite{2015arXiv150206944C}.
%\todo[inline]{Comment on this theorem in context of Hess' \cite{2010arXiv1001.1556H} and Ching and Harper's \cite{2015arXiv150206944C}}

\begin{proof}[Proof of Proposition \ref{hurewicz}]
Note that the maps $d^0,d^1:\calx^{0}\to\calx^{1}$ induce respectively the coaugmentation and coaction maps for $\pi_*\calx^{0}=H_*^{\calc}X$ on homotopy, while $d^0:\calx^{-1}\to\calx^{0}$ induces the Hurewicz map. The very existence of this diagram then shows that the  Hurewicz map factors through the primitives. The following observation of \citeBOX[\S4]{BlumRiehlResolutions.pdf} completes the proof: that this cosimplicial object has extra codegeneracies  when $X=K^\calc V$.
\end{proof}

\SubsectionOrSection{The smash product of homology coalgebras}\label{subseq:The smash product}
For $X_1$ and $X_2$ objects of  $\HC{\calc}$, we define the \emph{smash product} $X_1\smashprod X_2$ to be the cokernel of the natural map $X_1\sqcup X_2\to X_1\times X_2$. 

The theory changes a little in form after passing from homotopy to homology, and we must introduce the \emph{left derived smash product} in $\calc$. For $A_1$ and $A_2$ in $s\calc$, the natural map $A_1\sqcup A_2\to A_1\times A_2$ is a surjection, and so in general very far from a cofibration. We define the \emph{left derived smash product} $A_1\Lsmashprod A_2$ to be the cofiber of this map.
In light of Proposition \ref{hocof no ziggy-zaggy}, there are natural maps
\[c(A_1\sqcup A_2)\to c(A_1\times A_2)\to A_1\Lsmashprod A_2.\]
As in \cite[Proposition 4.6]{MR1089001}, this \emph{cofiber sequence} induces  a homology long exact sequence.
%
%
%Using any a functorial cofibration/acyclic fibration factorization on $s\calc$, form a natural commuting diagram
%\[\xymatrix@R=4mm{
%c(A_1\sqcup A_2)\ar@{ >->}[r]\ar@{->>}[d]^-{\epsilon}_-{\sim}
%&%r1c1
%M\ar@{ ->>}[r]^-{\sim} &%r1c2
%c(A_1\sqcup A_2)\ar@{->>}[d]^-{\epsilon}_-{\sim}\\%r1c3
%A_1\sqcup A_2\ar[rr]&%r2c1
%&%r2c2
%A_1\times A_2%r2c3
%}\]
%Then $A_1\Lsmashprod A_2$ is defined as the pushout in $s\calc$:
%\[\xymatrix@R=4mm{
%c(A_1\sqcup A_2)\ar@{ >->}[r]\ar[d]&%r1c1
%M\ar[d]\\%r1c2
%0\ar[r]&%r2c1
%A_1\Lsmashprod A_2%r2c2
%}\]
%%There is a zigzag $c(A_1\sqcup A_2)\overset{\sim}{\from}M\to A_1\Lsmashprod A_2$. 
%If we use Quillen's small object argument to construct the functorial factorization through $M$  (c.f.\ \cite{QuillenHomAlg.pdf}) it happens that Radulescu-Banu's construction can be extended, not only to produce a diagonal $\psi:c\to cc$, but to produce a natural map $c(A_1\sqcup A_2)$ to a commuting diagram{}

The following result and its proof are dual to Proposition \ref{smash coprod} and its proof.
\begin{prop}
\label{smash prod}
For $X_1$ and $X_2$ in $s\calc$, the natural $\calc$-$H_*$-coalgebra map
\[H^\calc_*(X_1\sqcup X_2)\from H^\calc_* X_1\sqcup H^\calc_* X_2\]
 is an isomorphism. If $X_1$ and $X_2$ are GEMs in $s\calc$, the natural $\calc$-$H_*$-coalgebra map 
\[H^\calc_* X_1\times H^\calc_* X_2\from H^\calc_*(X_1\times X_2)\]
takes part in an isomorphism of short exact sequences:
\[\xymatrix@R=4mm{
0\ar[r];[]&%r1c1
H^\calc_* X_1\smashprod H^\calc_* X_2
\ar[r];[]\ar[d];[]^-{\cong}
%\ar[d];[]_-{i}_-{\cong}
&%r1c2
H^\calc_* X_1\times H^\calc_* X_2
\ar[r];[]
\ar[d];[]^-{\cong}\ar[d];[]^-{\cong}
&%r1c3
H^\calc_* X_1\sqcup H^\calc_* X_2
\ar[r];[]
\ar[d];[]^-{\cong}
&%r1c4
0\\%r1c5
0\ar[r];[]&%r1c1
H^\calc_* (X_1\Lsmashprod X_2)
\ar[r];[]
&%r1c2
H^\calc_* (X_1\times X_2)
\ar[r];[]
&%r2c3
H^\calc_* (X_1\sqcup X_2)
\ar[r];[]&0
}\]
\end{prop}





\SubsectionOrSection{The quadratic part of a $\calc$-expression}\label{quadratic part section}
In this thesis, we will often use a method of constructing cohomology operations used by Goerss in \citeBOX[\S5]{MR1089001}, and here we will set up a framework that can be applied to each case. Suppose that $\calc$ is an algebraic category, monadic over $\vect{}{}$, category of graded vector spaces. 



For $V\in\vect{}{}$, the diagonal map $\Delta:V\to V\oplus V$ of $V$ induces a \emph{diagonal map} $F^\calc V\to F^\calc (V\oplus V)\cong (F^\calc (V))^{\sqcup 2}$, and writing $i_1$ and $i_2$ for the two summand inclusions $F^\calc (V)\to (F^\calc (V))^{\sqcup 2}$, consider the map
\[(F^\calc(\Delta)+i_1+i_2):F^\calc V\to (F^\calc V)^{\sqcup2}.\]
This map  factors through $(F^\calc V)^{\smashcoprod 2}$, and is symmetric. We name this factoring the \emph{cross terms}:
\[\crossterms:F^\calc V\to (F^\calc V)\smashcoprod^{\Sigma_2} (F^\calc V),\]
as it measures the non-linearity in an expression in $F^\calc V$.  We will give an example in each of the categories $\algs$, $\liealgs$ and $\restliealgs$, in each case using subscripts to denote membership of the first or second copy of $V$:
\begin{alignat*}{2}
\algs:&\qquad&\crossterms(vw)&=(v_1+v_2)(w_1+w_2)+v_1w_1+v_2w_2=v_1w_2+w_1v_2;\\
\liealgs:&\qquad&\crossterms([v,w])&= [v_1+v_2,w_1+w_2]+[v_1,w_1]+[v_2,w_2]=[v_1,w_2]+[w_1,v_2];\\
\restliealgs:&\qquad&\crossterms(\restn{v})&= \restn{v_1}+\restn{v_2}+\restn{(v_1+v_2)}=[v_1,v_2].
\end{alignat*}


%
%For example, when $\calc=\algs$, then for $v,w\in V$, :
%\[\crossterms(vw)=(v_1+v_2)(w_1+w_2)+v_1w_1+v_2w_2=v_1w_2+w_1v_2.\]

For certain categories of interest to us we will define a \emph{decomposition map}, natural and symmetric in $X_1,X_2\in\calc$:
\[j_\calc:Q^\calc(X_1\smashcoprod X_2)\to Q^\calc(X_1)\otimes Q^\calc(X_2).\]

When $\calc=\algs$, $X_1\smashcoprod X_2\cong X_1\otimes X_2$, and $Q(X_1\smashcoprod X_2)\cong QX_1\otimes QX_2$, and we choose the identity map of this object as decomposition map $j_\algs$. In other words, the map $j_{\algs}$ is defined by $x_1x_2\mapsto x_1\otimes x_2$ whenever $x_1\in X_1$ and $x_2\in X_2$.
%%is defined by the requirement that the following diagram commutes:
%\[\xymatrix@R=4mm{
%X_1\smashcoprod X_2\ar@{->>}[r]\ar@{->>}[d]
%&%r1c1
%QX_1\otimes QX_2\\%r1c2
%Q(X_1\smashcoprod X_2)\ar[ur]_-{J_{\algs}}
%&%r2c1
%%r2c2
%}\]

When $\calc=\liealgs$ or $\calc=\restliealgs$, we define the decomposition map by 
\[j_{\calL(n)}:\restnRepeated{[x_1,\cdots ,x_a]}{r}\longmapsto\begin{cases}
x_1\otimes x_2,&\textup{if }r=0,\,a=2,\,z_1\in X_1,\,z_2\in X_2,
\\0,&\textup{otherwise,}
\end{cases}\]
where by $\restnRepeated{[x_1,\cdots ,x_a]}{r}$ we mean the $r$-fold restriction ($r=0$ when $\calc=\liealgs$) of some bracketing of various $x_1,\ldots,x_a$ from $X_1$ and $X_2$, with at least one $z_k$ must be from each of $X_1$ and $X_2$.  Any element of the smash coproduct may be written as a sum of such expressions, so there is at most one map $j_\algs$ satisfying this equation. That this map is well defined is less obvious, but nonetheless routine.


Finally, we define the \emph{quadratic part} map, $\quadratic_\calc$ for  the composite
\[\quadratic_\calc:\left(F^\calc V\overset{\crossterms}{\to}(F^\calc V)^{\smashcoprod 2}\to Q^\calc((F^\calc V)\smashcoprod^{\Sigma_2} (F^\calc V))\overset{j_\calc}{\to}S^2(Q^{\calc}F^\calc V)=S^2V\right).\]
\begin{lem}
Suppose $V\in\vect{}{}$. Then:
\begin{enumerate}
\item \makebox[\widthof{$\quadratic_\restliealgs$}][l]{$\quadratic_\algs$}\,is the 
\makebox[\widthof{projection}][l]{composite}
$\smash{\makebox[\widthof{$F^\restliealgs V$}][c]{$F^\algs V$}\epi \makebox[\widthof{$\Lambda^2V$}][c]{$S_2V$}\overset{\trace}{\to} S^2V}$;
\item \makebox[\widthof{$\quadratic_\restliealgs$}][l]{$\quadratic_\liealgs$}\,is the \makebox[\widthof{projection}][l]{composite} $\smash{\makebox[\widthof{$F^\restliealgs V$}][c]{$F^\liealgs V$}\epi \Lambda^2V\overset{\trace}{\to} S^2V}$;
\item \makebox[\widthof{$\quadratic_\restliealgs$}][l]{$\quadratic_\restliealgs$}\,is the \makebox[\widthof{projection}][l]{projection} $\makebox[\widthof{$F^\restliealgs V$}][c]{$F^\restliealgs V$}\epi S^2V$.
\end{enumerate}
\end{lem}
\begin{proof}
These are simple observations, and an example is more useful than a proof: consider the expression $e:=u+vw+xy^2\in F^{\algs}V$ where $u,v,w,x,y$ are in $V$. Then
\begin{align*}
\crossterms(e)&\phantom{:}=\smash{v_1w_2+w_1v_2+x_1y_2^2+y_1^2x_2,\ \textup{and}}\\
\smash{\quadratic_\calc(e)}&:=\smash{j_\algs(\crossterms(e))}\\
&\phantom{:}=\smash{v_1\otimes w_2+w_1\otimes v_2+x_1\otimes y_2^2+y_1^2\otimes x_2}\\
&\phantom{:}=\smash{v_1\otimes w_2+w_1\otimes v_2 \in S^2(Q^\algs F^{\algs}V)}.
\end{align*}
Parts \emph{(2)} and \emph{(3)} are checked later in a more general case, in Proposition \ref{quadpartcalc for various W and L}.
\end{proof}







In each category of interest to us, the following equation of maps $F^\calc F^\calc V\to S^2 V$ will always be satisfied:
\[\quadratic_\calc\circ\mu_V=\quadratic_\calc\circ \epsilon_{F^\calc V} +\quadratic_\calc\circ {F^\calc \epsilon_V},\]
where $\mu$ here stands for the multiplication map of the monad $U^\calc F^\calc $. This is to say that if  $f(g_i)$ is a $\calc$-expression in various $\calc$-expressions $g_i(v_{ij})$, we have 
\[\quadratic(fg_i)(v_{ij})=\quadratic(f\epsilon(g_i))(v_{ij})+\epsilon(f)(\quadratic(g_i)(v_{ij})),\]
another expression of homogeneity in the relations defining $\calc$. For an example when $\calc=\algs$, we specify an expression $f(g_1,g_2,g_3):=g_1g_2+g_3\in F^\calc F^\calc V$ in expressions $g_i:=v_{i1}v_{i2}+v_{i3}\in F^\calc V$ for each $i=1,2,3$. Then
\begin{alignat*}{2}
\textup{\small$\quadratic(fg_i)(v_{ij})={}$}
&
\textup{\small$\quadratic((v_{11}v_{12}+v_{13})(v_{21}v_{22}+v_{23})+(v_{31}v_{32}+v_{33}))=\trace(v_{13}\otimes v_{23}+v_{31}\otimes v_{32}),$}
\\
\textup{\small$\quadratic(f\epsilon(g_i))(v_{ij})={}$}
&
\textup{\small$\quadratic((v_{13})(v_{23})+(v_{33}))=\trace(v_{13}\otimes v_{23}),\textup{ and}$}
\\
\textup{\small$\epsilon(f)(\quadratic(g_i)(v_{ij}))={}$}
&
\textup{\small$\quadratic(v_{31}v_{32}+v_{33})=\trace(v_{31}\otimes v_{32}).$}
\end{alignat*}










\end{Pi-algebras and cohomology algebras}

\begin{Bousfield-Kan spectral sequence}
\SectionOrChapter{The Bousfield-Kan spectral sequence}
\label{Bousfield-Kan spectral sequence}

In this chapter, we will write $\algcat$ for any category of universal graded $\Ftwo $-algebras satisfying the standing assumptions of \S\ref{Universal algebras}. The \emph{Bousfield-Kan spectral sequence of $X\in s\algcat$} is the second quadrant homotopy spectral sequence (c.f.\ \S\ref{Towers, exact couples and coaugmented cosimplicial objects}) of the resolution $\calx\in cs\algcat$ of $X$.
The key objective of this thesis is to understand this spectral sequence when $\calc=\algs$.

Our first step, in \S\ref{Idnt E1 E2}, is to identify the $E_2$-page as appropriate derived functors.

We must also grasp the convergence target, which in this case is the homotopy of $\Tot \calx=:X\hat{\ }$. From \S\ref{sec:relnWithRB} to the end of this section, we will give a proof of Theorem \ref{completenesstheorem} --- that the completion $X\hat{\ } $ is weakly equivalent to $X$ when $\calC$ is either $\algs$ or $\restliealgs$ and $X$ is connected.

Theorem \ref{completenesstheorem} does not fully resolve the convergence question for a second quadrant spectral sequence, and we will prove in \S\ref{A vanishing line on the Bousfield-Kan} that if $X\in s\algs$ is a connected object with $H^*_\algs$ of finite type, the spectral sequence supports a vanishing line at $E_2$.




\SubsectionOrSection{Identification of $E_1$ and $E_2$}\label{Idnt E1 E2}
In light of \S\ref{Cofibrant replacement via the small object argument} and \S\ref{homology and Hcoalgs}, applying the functor $H_*^{\algcat}$ to $\calx$, one obtains precisely the monadic cobar resolution of $H_*^{\algcat}X$ obtained by repeated application of the monad on $\HC{\algcat}$ of the adjunction 
\[U^{\HC{\algcat}}:\HC{\algcat}\rightleftarrows \vect{}{1}:C^{\HC{\algcat}}.\]
In more detail, we have a map of coaugmented cosimplicial objects
\[\vcenter{
\def\labelstyle{\scriptstyle}
\xymatrix@C=1.5cm@1{
cX\,
\ar@{-->}[r]\ar@{->>}[d]
&
\,cK^{\calc}Q^{\calc}cX\,
\ar[r];[]\ar@{->>}[d]
&
\,c(K^{\calc}Qc^{\calc})^2X\,
\ar@<-.65ex>[l];[]
\ar@<+.65ex>[l];[]
\ar@<+.65ex>[r];[]
\ar@<-.65ex>[r];[]\ar@{->>}[d]
&
\,\cdots
\ar[l];[]
\ar@<-1.3ex>[l];[]
\ar@<+1.3ex>[l];[]\\
QcX\,
\ar@{-->}[r]
&
\,Q^{\calc}cK^{\calc}Q^{\calc}cX\,
\ar[r];[]
&
\,Q^{\calc}c(K^{\calc}Q^{\calc}c)^2X\,
\ar@<-.65ex>[l];[]
\ar@<+.65ex>[l];[]
\ar@<+.65ex>[r];[]
\ar@<-.65ex>[r];[]
&
\,\cdots
\ar[l];[]
\ar@<-1.3ex>[l];[]
\ar@<+1.3ex>[l];[]
}}\]
Write `$C$' for the monad $C^{\HC{\algcat}}U^{\HC{\algcat}}$ on $\HC{\algcat}$. Applying  $\pi_*$, we obtain a cosimplicial \emph{Hurewicz map}:
\[\vcenter{
\def\labelstyle{\scriptstyle}
\xymatrix@R=1.2cm@C=3.3cm@!0{
{\qquad\qquad\pi_*\calx^\bullet:\!\!\!\!\!\!\!\!\!\!\!\!\!\!}&
\pi_*X\,
\ar@{-->}[r]\ar@{->}[d]
&
\,\pi_*\calx^0\,
\ar[r];[]\ar[d]^-{\cong}
&
\,\pi_*\calx^1\,
\ar@<-.65ex>[l];[]
\ar@<+.65ex>[l];[]
\ar@<+.65ex>[r];[]
\ar@<-.65ex>[r];[]\ar[d]^-{\cong}
&
\,{\cdots\qquad\qquad}
\ar[l];[]
\ar@<-1.3ex>[l];[]
\ar@<+1.3ex>[l];[]
\\
{\qquad\Prim^{\HC{\calc}}(H_*^\algcat X^\bullet):\!\!\!\!\!\!\!\!\!\!\!\!\!\!}&
\Prim(H_*^\algcat X)\,
\ar@{-->}[r]
\ar@{ >->}[d]
&
\,\Prim(CH_*^\algcat X)\,
\ar[r];[]
\ar@{ >->}[d]
&
\,\Prim(C^2H_*^\algcat X)\,
\ar@<-.65ex>[l];[]
\ar@<+.65ex>[l];[]
\ar@<+.65ex>[r];[]
\ar@<-.65ex>[r];[]
\ar@{ >->}[d]
&
\,{\cdots\qquad\qquad}
\ar[l];[]
\ar@<-1.3ex>[l];[]
\ar@<+1.3ex>[l];[]
\\
{\qquad\qquad H_*^\algcat X^\bullet:\!\!\!\!\!\!\!\!\!\!\!\!\!\!}&
H_*^\algcat X\,
\ar@{-->}[r]
&
\,CH_*^\algcat X\,
\ar[r];[]
&
\,C^2H_*^\algcat X\,
\ar@<-.65ex>[l];[]
\ar@<+.65ex>[l];[]
\ar@<+.65ex>[r];[]
\ar@<-.65ex>[r];[]
&
\,{\cdots\qquad\qquad}
\ar[l];[]
\ar@<-1.3ex>[l];[]
\ar@<+1.3ex>[l];[]
}}\qquad\qquad\qquad
\]
The marked maps are isomorphisms since each $\calx^s$ for $s\geq0$ is a GEM, thanks to Proposition \ref{hurewicz}.
In particular, we see that:
\begin{alignat*}{2}
\Edownup{1}{}{\calx}{s}{t}&\cong (\Prim^{\HC{\calc}}(C^sH_*^{\algcat}X))_{t};\\
\Edownup{2}{}{\calx}{s}{t}&\cong((\mathbb{R}^s\Prim^{\HC{\calc}})H_*^{\algcat}X)_t.
\end{alignat*}
Corollaries \ref{finite type pres by FW0} and \ref{finite type pres by F lierest halg} and Proposition \ref{something about dualization} show that
\begin{thm}
\label{identify E2 with derived Q}
If $\algcat$ is either $\algs$ or $\restliealgs$, and $X$ is connected with $H^*_\calc X$  of finite type, then $H^*_\calc\calx^s$ is of finite type for each $s$, and:
\begin{alignat*}{2}
\Edownup{1}{}{\calx}{s}{t}&\cong (C^*\dual Q^\algcat B^\algcat H^*_{\algcat}X)^{s}_{t};\\
\Edownup{2}{}{\calx}{s}{t}&\cong (H^*_{\HA{\algcat}}H^*_{\algcat}X)^{s}_{t}.
\end{alignat*}
\end{thm}





\SubsectionOrSection{The Adams tower}\label{sec:derWRTab}\label{sec:relnWithRB}
Bousfield and Kan defined the \emph{Bousfield-Kan spectral sequence}, or \emph{unstable Adams spectral sequence}, of a simplicial set in two different ways. Their earlier approach \cite{BK_pairings.pdf} was to define the \emph{derivation of a functor with respect to a ring}. This approach constructs the \emph{Adams tower} over the simplicial set in question, and lends itself well to connectivity analyses. Their latter approach, \cite{BousKanSSeq.pdf}, to give a cosimplicial resolution of a simplicial set by simplicial $R$-modules, lends itself more to the analysis of the $E_2$-page, and is directly analogous to Radulescu-Banu's construction described in \S\ref{The Hurewicz map, primitives and homology completion}.

Since the release of \cite{BK_pairings.pdf} and \cite{BousKanSSeq.pdf}, the relationship between the two approaches has been clarified by the introduction of cubical homotopy theory \cite{GoodwillieCalcII}. In this section we will define the Adams tower of a simplicial algebra using a construction analogous to Bousfield and Kan in \cite{BK_pairings.pdf}, and then apply the theory of cubical diagrams to relate it to Radulescu-Banu's construction. 


For brevity, write $K:= K^{\calc}$ and $Q:=Q^{\calc}$. For any functor $F:s\algcat\to s\algcat$, we define the $r^\textup{th}$ derivation $\dupdown{r}{c}F$ of $F$ with respect to  homology. The definition is recursive, and again involves repeated application of the functor $c$: 
\begin{alignat*}{2}
(\dupdown{0}{c}F)(X)
&:=
F(cX)%
\\
(\dupdown{s}{c}F)(X)
&:=
\hofib\bigl((\dupdown{s-1}{c}F)(cX)\xrightarrow{(\dupdown{s-1}{c}F)(\eta_{cX})} (\dupdown{s-1}{c}F)(KQcX)\bigr)
\end{alignat*}
where $\eta$ is the unit of the adjunction $Q\dashv K$, i.e.\ the natural surjection onto indecomposables, and $\hofib$ is any fixed  functorial construction of the homotopy fiber. These functors fit into a tower via the following composite natural transformation:
\[\delta:\left((\dupdown{s}{c}F)(X)\to (\dupdown{s-1}{c}F)(cX)\overset{(\dupdown{s-1}{c}F)(\epsilon)}{\to} (\dupdown{s-1}{c}F)(X)\right).\]
We have thus constructed a tower
\[\xymatrix@R=4mm{
\cdots 
\ar[r]
&%r1c1
(\dupdown{2}{c}F)X
\ar[r]
&%r1c3
%r1c4
(\dupdown{1}{c}F)X
\ar[r]&%r1c5
(\dupdown{0}{c}F)X=FcX,
}\]%reverse me
which is natural in the object $X$ and the functor $F$.
The functors $\dupdown{r}{c}F$ are homotopical as long as $F$ preserves weak equivalences between cofibrant objects. Employing the shorthand
\[\dupdown{s}{c}X:=(\dupdown{s}{c}\Id )X,\]
we define \emph{the Adams tower of $X$} to be the tower
\[\xymatrix@R=4mm{
\cdots 
\ar[r]
&%r1c1
\dupdown{2}{c}X
\ar[r]
&%r1c3
%r1c4
\dupdown{1}{c}X
\ar[r]&%r1c5
\dupdown{0}{c}X=cX.
}\]

For example, $(\dupdown{2}{c}F)(X)$ is constructed by the following diagram in which every composable pair of parallel arrows is \emph{defined} to be a homotopy fiber sequence.
\[\def\labelstyle{\scriptstyle}
\xymatrix@!0@R=30pt@C=43pt{
(\dupdown{2}{c}F)(X)\ar[d]\\
(\dupdown{1}{c}F)(cX) \ar[rr]\ar[d]         &           &FcccX \ar[rr]\ar[d]         &           &   FcKQccX            \ar[d]  &                  \\
(\dupdown{1}{c}F)(KQcX) \ar[rr] &                     &  FccKQcX \ar[rr] &             & FcKQcKQcX
}\]
In general, $(\dupdown{n+1}{c}F)(X)$ is the homotopy total fiber of an $(n+1)$-cubical diagram:
\[(\dupdown{n+1}{c}F)(X):=\textup{hototfib} \bigl(({\plainD}_{n+1}^{\smash{\square}}F)X\bigr).\]
See \cite{GoodwillieCalcII}, \cite{LuisGoodwillie.pdf} or \cite{CubicalHomotopyTheory.pdf} for the general theory of cubical diagrams. Before defining the cubical diagram $({\plainD}_{n+1}^{\smash{\square}}F)X$, we set notation: for $n\geq0$ let $[n]=\{0,\ldots,n\}$, and define $\calP[n]=\left\{S\subseteq [n]\right\}$ to be the poset category whose morphisms are the inclusions $S\subseteq S'$. Then an $(n+1)$-cube in $s\algcat$ is a functor $\calP[n]\to s\algcat$, and the $(n+1)$-cubical diagram $({\plainD}_{n+1}^{\smash{\square}}F)X:\calP[n]\to s\algcat$ is the functor:
\[S\mapsto Fc(KQ)^{\chi_{n}}c(KQ)^{\chi_{n-1}}c\cdots c(KQ)^{\chi_0}cX\quad \textup{where}\quad \chi_i:=\begin{cases}
1,&\textup{if }i\in S;\\
0,&\textup{if }i\notin S,
%\\,&\textup{if }
\end{cases}
\]
where for $S\subseteq S'$, the map $(({\plainD}_{n+1}^{\smash{\square}}F)X)(S)\to (({\plainD}_{n+1}^{\smash{\square}}F)X)(S')$ is given by applying the counit $\eta:1\to KQ$ in those locations indexed by $S'\setminus S$.

%\SubsectionOrSection{Relationship between the Adams tower and Radulescu-Banu's resolution}\label{sec:relnWithRB}

Now Radulescu-Banu defines the homology completion of $X$ to be the totalization
\[X\hat{\ }:=\textup{Tot}(\calx^\bullet)=\holim (\textup{Tot}_n(\calx^\bullet)),\]
and the \BKSS\ to be the spectral sequence of the tower
\[\cdots \to\textup{Tot}_n(\calx^\bullet)\to \textup{Tot}_{n-1}(\calx^\bullet)\to\cdots \]
under $cX$. Our goal in this section is to prove
\begin{prop}
\label{towerIdentification}
There is a natural zig-zag of weak equivalences of towers \[\left\{\plainD_{n+1}X\right\}_n\simeq\left\{\hofib(cX\to\textup{Tot}_{n}(\calx^\bullet))\right\}_n.\] That is,  the $\textup{Tot}$ tower induces the Adams tower by taking homotopy fibers, and thus the spectral sequence of the $\textup{Tot}$ tower coincides with the spectral sequence of the Adams tower.
\end{prop}
\noindent As $\calx^{-1}$ equals $cX$, the tower $\hofib(\calx^{-1}\to\textup{Tot}_n\calx^{\bullet})$ appearing in \S\ref{Towers, exact couples and coaugmented cosimplicial objects} is one of the towers in Proposition \ref{towerIdentification}. This proposition explains the relevance of the Adams tower to the cosimplicial resolution, and thus its relevance to the \BKSS\ which was defined as the spectral sequence of this cosimplicial object.
\begin{proof}[Proof of Proposition \ref{towerIdentification}] 
It will suffice to construct a weak equivalence $h_n^*\calx^\bullet\to (\plainD_n^{\smash{\square}}\Id  )(X)$ of $(n+1)$-cubes. The $(n+1)$-cubical diagram $h_n^*\calx^\bullet$ is defined by
\[(h_n^*\calx^\bullet)(S):= c(KQc)^{\chi_{n}}(KQc)^{\chi_{n-1}}\cdots (KQc)^{\chi_0}X\quad \textup{where}\quad \chi_i:=\begin{cases}
1,&\textup{if }i\in S,\\
0,&\textup{if }i\notin S,
\end{cases}
\]
where we describe the map $(h_n^*\calx^\bullet)(S)\to (h_n^*\calx^\bullet)(S\sqcup\{i\})$, for $i\notin S$, as follows. Let $j$ be the smallest element of $S\sqcup\{n+1\}$ exceeding $i$, so that
\[(h_n^*\calx^\bullet)(S):= \begin{cases}
c(KQc)^{\chi_{n}}\cdots (KQc)^{\chi_{j+1}}(KQ\underline{c})(KQc)^{\chi_{i-1}}\cdots (KQc)^{\chi_0}X,&\textup{if }j\leq n;\\
\underline{c}(KQc)^{\chi_{i-1}}\cdots (KQc)^{\chi_0}X,&\textup{if }j=n+1.
%\\,&\textup{if }
\end{cases}
\]
In the expression for either case, we have distinguished one of the applications of $c$ with an underline, and the map to $(h_n^*\calx^\bullet)(S\sqcup\{i\})$ is induced by the composite $\underline{c}\to cc\to cKQc$ of the diagonal of the comonad $c$ with the unit of the monad $KQ$. 

We now define maps $(h_n^*\calx^\bullet)(S)\to (({\plainD}_{n+1}^{\smash{\square}}\Id )X)(S)$ for $S=\{j_0<j_1<\cdots<j_r\}\subseteq\{0,\ldots,n\}$. 
The only difference between the domain and codomain is that in $(({\plainD}_{n+1}^{\smash{\square}}\Id )X)(S)$, all $n+2$ applications of $c$ are present, whereas in $(h_n^*\calx^\bullet)(S)$, only $r+2$ appear. The map is then
\[\beta^{n-j_r}KQ\beta^{j_r-j_{r-1}-1}KQ\beta^{j_{r-1}-j_{r-2}-1}KQ\cdots KQ\beta^{j_{1}-j_0-1}KQ\beta^{j_0}X\]
which is to say that we apply the iterated diagonal the appropriate number of times in each $c$ appearing in the domain. As $\beta$ is coassociative, this definition is unambiguous, and the resulting maps assemble to a weak equivalence of $(n+1)$-cubes. 
\end{proof}



\SubsectionOrSection{Connectivity estimates and homology completion}
\label{sec:connectivityAnalysis}
In this section we will prove
the following conjecture of Radulescu-Banu:
\begin{thm}
\label{completenesstheorem}
If either $\algcat=\algs$ or $\algcat=\restliealgs$ and $X\in s\algcat$ is connected, then $X$ is naturally equivalent to its homology completion $X\hat{\ }$.
\end{thm}
This will follow from Proposition \ref{towerIdentification} and the following connectivity estimates in the Adams tower:
\begin{prop}
\label{convergenceProp}
Suppose that $\algcat$ is one of the categories $\algs$ or $\restliealgs$, that $X\in s\algcat$ is connected, and that  $t\geq1$ and $q\geq2$. Then there is some $f(q,t)\geq t$ such that the map $\pi_q(\dupdown{f(q,t)}{c}X)\to\pi_q(\dupdown{t}{c}X)$ is zero.
\end{prop}
%\noindent Note that Proposition \ref{towerIdentification} and \ref{convergenceProp} together imply Theorem \ref{completenesstheorem}:
\begin{proof}[Proof of Theorem \ref{completenesstheorem}]
The fiber sequences $\dupdown{n+1}{c}X\to cX\to \textup{Tot}_n\calx^\bullet$ fit together into a tower of fiber sequences. Taking homotopy limits, one obtains a fiber sequence
\[\holim (\dupdown{n}{c}X)\to cX\to X\hat{\ }.\]
We need to show that $\holim (\dupdown{n}{c}X)$ has zero homotopy groups.
%We may replace the tower $\dupdown{n}{c}X$ with a weakly equivalent tower of fibrations in $s\algcat$ whose set-theoretic inverse limit is the homotopy limit in question. 
Applying \cite[Proposition 6.14]{goerss-jardine.pdf}, there is a short exact sequence
\[0\to \textup{lim}^1\pi_{q+1}(\dupdown{n}{c}X)\to \pi_q(\holim (\dupdown{n}{c}X))\to \textup{lim}\,\pi_{q}(\dupdown{n}{c}X)\to 0.\]
Proposition \ref{convergenceProp} implies that for each $q$, the tower $\{\pi_{q}(\dupdown{n}{c}X)\}_n$ has zero inverse limit and satisfies the Mittag-Leffler condition (c.f.\ \citeBOX[p.~264]{YellowMonster}), so that the $\textup{lim}^1$ groups appearing also vanish.
\end{proof}
The application of the small object argument functor $c$ adds to the difficulty of proving the connectivity estimates of Proposition \ref{convergenceProp}. We circumvent the difficulty of working with $c$ by shifting to the standard bar construction $B^{\algCat}$ on $s\algCat$, which we abbreviation to $\barConstructionMightAbbreviate $.

We define recursively a somewhat less homotopical version $\caldup{s}F$ of the derivations $\dupdown{s}{}F$:
\begin{alignat*}{2}
(\caldup{0}F)(X)
&:=
F(X),%
\\
(\caldup{s}F)(X)
&:=
\ker((\caldup{s-1}F)(\barConstructionMightAbbreviate X)\xrightarrow{(\caldup{s-1}F)(\eta_{\barConstructionMightAbbreviate X})} (\caldup{s-1}F)(KQ\barConstructionMightAbbreviate X)).
\end{alignat*}
There are three differences between this definition and that of $\plainD_sF$: here, there is one fewer cofibrant replacement applied, we use $\barConstructionMightAbbreviate $ instead of  $c$, and we take \emph{strict} fibers, not homotopy fibers.
While these functors are not generally homotopical, we define \emph{the modified Adams tower of $X$} to be the tower
\[\xymatrix@R=4mm{
\cdots 
\ar[r]^{\delta}
&%r1c1
\caldup{2}X
\ar[r]^{\delta}
&%r1c3
%r1c4
\caldup{1}X
\ar[r]^{\delta}&%r1c5
\caldup{0}X\makebox[0cm][l]{${}=X$,}
}\]
where $\caldup{s}X$ is again shorthand for $(\caldup{s}\Id )X$, and the tower maps $\delta$ are defined as before.
\begin{prop}
\label{prop:modifiedAdamsTower}
There is a natural zig-zag of weak equivalences of towers between the Adams tower of $X$ and the modified Adams tower of $X$. In particular, the modified Adams tower is homotopical.
\end{prop}
\begin{proof}
Let $\mathsf{CR(s\algcat)}$ be the category of cofibrant replacement functors in $s\algcat$. That is, an object of $\mathsf{CR(s\algcat)}$ is a pair, $(f,\epsilon)$, such that $f:s\algcat\to s\algcat$ is a functor whose image consists only of cofibrant objects, and $\epsilon:f\Rightarrow \Id $ is a natural acyclic fibration. Morphisms in $\mathsf{CR(s\algcat)}$ are natural transformations which commute with the augmentations. For any $(f,\epsilon)\in\mathsf{CR(s\algcat)}$ we obtain an alternative definition of the derivations of a functor $F:s\algcat\to s\algcat$:
\begin{alignat*}{2}
(\plainD_{\smash{0}}^{\smash{f}}F)(X)
:=
F(fX),\quad %
(\plainD_{\smash{s}}^{\smash{f}}F)(X)
:=
\hofib((\plainD_{\smash{s-1}}^{\smash{f}}F)(fX)\to (\plainD_{\smash{s-1}}^{\smash{f}}F)(KQfX)).
\end{alignat*}
These functors are natural in $f$, so that a morphism in $\mathsf{CR(s\algcat)}$ induces a weak equivalence of towers. Our proposed zig-zag of towers is: 
\[\plainD_{\smash{s}}= \plainD_{\smash{s}}^{\smash{c}}\Id \overset{}{\from}\plainD_{\smash{s}}^{\smash{\barConstructionMightAbbreviate \circ c}}\Id \overset{}{\to}\plainD_{\smash{s}}^{\smash{\barConstructionMightAbbreviate }}\Id \overset{\gamma_s}{\from}\caldup{s}\barConstructionMightAbbreviate \overset{\caldup{s}\epsilon}{\to}\caldup{s}\Id =\caldup{s}\]
The maps with domain $\plainD_{\smash{s}}^{\smash{\barConstructionMightAbbreviate \circ c}}\Id $ are induced by the maps $\epsilon c:\barConstructionMightAbbreviate \circ c\to c$ and $\barConstructionMightAbbreviate \epsilon:\barConstructionMightAbbreviate \circ c\to \barConstructionMightAbbreviate $ and are evidently natural weak equivalences of towers. The map $\gamma_0:(\caldup{0}\barConstructionMightAbbreviate )X\to (\plainD_{\smash{0}}^{\smash{\barConstructionMightAbbreviate }}\Id )X$ is the identity of $\barConstructionMightAbbreviate X$, and the map $\caldup{0}\epsilon:(\caldup{0}\barConstructionMightAbbreviate )X\to (\plainD_{\smash{0}}^{\smash{\barConstructionMightAbbreviate }}\Id )X$ is $\epsilon:\barConstructionMightAbbreviate X\to X$. Thereafter, $\gamma_s$ and $\caldup{s}\epsilon$ are defined recursively:
\[\qquad \quad \xymatrix@R=4mm{
\makebox[0cm][r]{$(\caldup{s+1}\Id )X:=\,$}\makebox[5cm][l]{$\,\ker((\caldup{s}\Id )(\barConstructionMightAbbreviate X)\to (\caldup{s}\Id )(KQ\barConstructionMightAbbreviate X))$}\\
\makebox[0cm][r]{$(\caldup{s+1}\barConstructionMightAbbreviate )X:=\,$}\makebox[5cm][l]{$\,\ker((\caldup{s}\barConstructionMightAbbreviate )(\barConstructionMightAbbreviate X)\to (\caldup{s}\barConstructionMightAbbreviate )(KQ\barConstructionMightAbbreviate X))$}\ar[d]^-{\textup{incl.}}_-{}%m_2}
\ar[u]_-{\textup{induced by }(\caldup{s}\epsilon,\caldup{s}\epsilon)}^-{}%m_1}
\\%r1c1
\makebox[5cm][l]{$\,\hofib((\caldup{s}\barConstructionMightAbbreviate )(\barConstructionMightAbbreviate X)\to(\caldup{s}\barConstructionMightAbbreviate )(KQ\barConstructionMightAbbreviate X))$}\ar[d]^-{\textup{induced by }(\gamma_{s},\gamma_{s})}_-{}%m_3}
\\%r2c1
\makebox[0cm][r]{$(\plainD_{\smash{s+1}}^{\smash{\barConstructionMightAbbreviate }})X:=\,$}\makebox[5cm][l]{$\,\hofib((\plainD_{\smash{s}}^{\smash{\barConstructionMightAbbreviate }}\Id )(\barConstructionMightAbbreviate X)\to (\plainD_{\smash{s}}^{\smash{\barConstructionMightAbbreviate }}\Id )(KQ\barConstructionMightAbbreviate X))$}%r3c1
}\]
\noindent Lemma \ref{towerWithPowers} shows that the kernels taken are actually kernels of surjective maps, and by induction on $s$, the maps $\gamma_s$ and $\caldup{s}\epsilon$ are weak equivalences.
\end{proof}
The connectivity result will rely on the observation that any element in the $s^\textup{th}$ level of the modified tower maps down to an $(s+1)$-fold expression in $X$. In order to formalise this, when $\algcat=\algs$, we let $P^s:s\algcat\to s\algcat$ be the ``$s^\textup{th}$ power'' functor, the prolongation of the endofunctor $Y\mapsto Y^s$ of $\algcat$, where $Y^s=\textup{im}(\textup{mult}:Y^{\otimes s}\to Y)$. When $\algcat=\algs$, we define $P^s:=\Gamma^s$, the $s^\textup{th}$ term in the lower central series filtration (c.f.\ \cite{6Author.pdf}). Then we have:
\begin{lem}
\label{towerWithPowers}
Suppose that  either $\algcat=\algs$ or $\algcat=\restliealgs$. The functors $\caldup{r}$, $\caldup{r}\barConstructionMightAbbreviate $ and $\caldup{r}P^{s}$ preserve surjective maps and there is a commuting diagram of functors:
\[\xymatrix@R=4mm{
\cdots 
\ar[r]
&%r1c1
\caldup{r}
\ar[r]
\ar[d]
&%r1c2
\cdots \ar[r]
&%r1c4
\caldup{2}
\ar[r]
\ar[d]
&%r1c4
\caldup{1}
\ar[r]
\ar[d]&%r1c5
\caldup{0}
\ar@{=}[d]
\\%r1c6
\cdots
\ar@{ >->}[r]
&%r2c1
P^{r+1}
\ar@{ >->}[r]
&%r2c3
\cdots 
\ar@{ >->}[r]&%r2c4
P^3
\ar@{ >->}[r]
&P^2
\ar@{ >->}[r]
&%r2c5
\Id %r2c6
}\]
\end{lem}
\begin{proof}
As $\barConstructionMightAbbreviate$ and $P^s$ preserve surjections, we need only check the claims about $\caldup{r}X$ for $X\in s\algcat$, which is constructed as the subobject
\[\caldup{r}X:= \bigcap_{i=1}^{r}\ker\!\left(\barConstructionMightAbbreviate^{r-i}\eta \barConstructionMightAbbreviate^{i}:\barConstructionMightAbbreviate^{r}X\to \barConstructionMightAbbreviate^{r-i}KQ\barConstructionMightAbbreviate^{i}X\right)\]
of $\barConstructionMightAbbreviate^rX$. In dimension $n$, this is the following subset of $(\barConstructionMightAbbreviate^rX)_n:=(F^{n+1})^rX_n$:
\[(\caldup{r}X)_n:=\bigcap_{i=1}^{r}\ker\!\left(F^{(r-i)(n+1)}\eta F^{i(n+1)}:(F^{n+1})^rX_n\to (F^{n+1})^{r-i}KQ(F^{n+1})^iX_n\right).\]
%Where we have written $F$ for $F^{\algcat}$.

Whichever of $\algs$ or $\restliealgs$ we are working with, it is possible to construct monomial bases for $FV$ once a basis of $V$ has been chosen. For given $n$ and $r$, first choose a basis of $X_n$; build from it a monomial basis of $FX_n$; build from this a monomial basis of $F^2X_n$; etc. Continue until we have a monomial basis of $F^{r(n+1)}X_n=(b^rX)_n$. The effect of the map $F^{(r-i)(n+1)}\eta F^{i(n+1)}$ on monomials is either to annihilate them or leave them unchanged, depending on whether any non-trivial constructions were employed at the $((n+1)i)^\textup{th}$ stage.
Thus, the subset $(\caldup{r}X)_n$ has basis those iterated monomials in which some non-trivial construction was used in the $((n+1)i)^\textup{th}$ for $1\leq i\leq r$. The image of such a monomial in $X_n$ lies in $P^r$.


To see that $\caldup{r}$ preserves surjections: if $f:X\to Y$ is a surjection, choose a basis $B\sqcup B'$ of $X_n$ for  which $f$ maps the $B$ bijectively onto a basis of $Y_n$ and $B'$ to zero. We may continue this pattern at each stage of the construction of iterated monomial bases of $F^{r(n+1)}X_n$ and $F^{r(n+1)}Y_n$. That is, we may choose a  basis $C\sqcup C'$ of $F^{r(n+1)}X_n$  such that the monomials in $C$ only involve the elements of $B$ and map under $f$ bijectively onto a basis of $F^{r(n+1)}Y_n$, and such that each monomial in $C'$ involves some element of $B'$, and so vanishes under $f$. This pattern is further preserved in passing to the monomial bases discussed earlier for $(\caldup{r}X)_n$ and $(\caldup{r}Y)_n$, proving the claim that $\caldup{r}$ preserves surjections. 
%
%Thus, choosing a basis of. One simply uses the standard monomial basis
%The $r$ conditions on elements of $(\caldup{r}X)_n$ ensure that their image in $(\caldup{0}X)_n=X_n$ under the iterated tower map is a sum of $(r+1)$-fold products. This completes the construction of the tower of functors.
%
%[\textbf{Reformulate for generalizability:} is not it just that except for simplicial structure you do not see more than a vector space?] In order to prove the surjectivity statements, we must describe the iterated free construction $F^{r(n+1)}X_n$. A basis of $FX_n$ may be given by the \emph{monomials} in a basis of $X_n$, and $F^{r(n+1)}X_n$ has basis given by taking monomials iteratively, $r(n+1)$ times. The subset $(\caldup{r}X)_n$ has basis those iterated monomials in which the monomials formed in the $((n+1)i)^\textup{th}$ iteration have degree at least two for each $1\leq i\leq r$. This simple description of a basis of $(\caldup{r}X)_n$ shows that $\caldup{r}$ preserves surjections. Similar analysis applies to $\caldup{r}\barConstructionMightAbbreviate $ and $\caldup{r}P^s$.
\end{proof}
We are now able to state and prove the key connectivity result:
\begin{lem}
\label{connectivityOfDerivedPowers}
Suppose that $X\in s\algcat$ is connected, $t\geq1$ and $s\geq2$. If $\algcat=\algs$, then $(\caldup{t}P^{s})(X)$ is $(s-t)$-connected. If $\algcat=\restliealgs$, then $(\caldup{t}P^{s})(X)$ is  $(\log_2(s)+1-t)$-connected.
\end{lem}
\begin{proof}
We will prove this by induction on $t$. The induction step is simple: %Now let $t\geq2$ and suppose by induction that $(\caldup{t-1}P^{s})(B)$ is $(s-(t-1))$-connected for any connected $B$ and any $s\geq2$. Then 
by Lemma \ref{towerWithPowers}, there is a short exact sequence:
\[\xymatrix{
0\ar[r]&
(\caldup{t}P^{s})(X)\ar[r]&
(\caldup{t-1}P^{s})(\barConstructionMightAbbreviate X)\ar[r]&
(\caldup{t-1}P^{s})(Q\barConstructionMightAbbreviate X)\ar[r]&
0.
}\]
Now both $\barConstructionMightAbbreviate X$ and $Q\barConstructionMightAbbreviate X$ are connected, as they have $\pi_0(\barConstructionMightAbbreviate X)=\pi_*X$ is zero by assumption, and $\pi_0(Q\barConstructionMightAbbreviate X)=Q\pi_*X$. By induction we can bound the connectivity of $(\caldup{t-1}P^{s})(\barConstructionMightAbbreviate X)$ and $(\caldup{t-1}P^{s})(Q\barConstructionMightAbbreviate X)$, and the associated long exact sequence shows that $(\caldup{t}P^{s})(X)$ has a connectivity bound at most one degree lower.


For the base case, $t=1$, as $P^s(QX)=0$ for $s\geq2$:
\[(\caldup{1}P^{s})X:=\ker(P^{s}(\barConstructionMightAbbreviate X)\to P^{s}(Q\barConstructionMightAbbreviate X))=P^{s}(\barConstructionMightAbbreviate X).\]
When $\algcat=\restliealgs$, a modification \citeBOX[4.3]{6Author.pdf} of a theorem of Curtis \citeBOX[\S5]{Curtis_LCS.pdf} states that $P^{s}(\barConstructionMightAbbreviate X)$ is $\log_2(s)$-connected.
When $\algcat=\algs$, we must demonstrate then that $P^s(\barConstructionMightAbbreviate X)$ is $(s-1)$-connected. For this
 we use a truncation of Quillen's fundamental spectral sequence, as presented in \cite[Theorem 6.2]{MR1089001}: the filtration
\[P^s(\barConstructionMightAbbreviate X)\supset P^{s+1}(\barConstructionMightAbbreviate X)\supset P^{s+2}(\barConstructionMightAbbreviate X)\supset\cdots \]
of $P^s(\barConstructionMightAbbreviate X)$ yields a convergent spectral sequence  $\E{}{0}{P^s(\barConstructionMightAbbreviate X)}{p}{q}\implies \pi_q(P^s(\barConstructionMightAbbreviate a))$, with:
\[\E{}{0}{P^s(\barConstructionMightAbbreviate X)}{p}{q}=
\begin{cases}
N_q\bigl((Q^{\algs}\barConstructionMightAbbreviate X)^{\otimes p}_{\Sigma_p}\bigr),&\textup{if }p\geq s;\\
0,&\textup{if }p<s.
%\\,&\textup{if }
\end{cases}\]
%the filtration
%\[P^s(\barConstructionMightAbbreviate X)\supset P^{s+1}(\barConstructionMightAbbreviate X)\supset P^{s+2}(\barConstructionMightAbbreviate X)\supset\cdots \]
%of $P^s(\barConstructionMightAbbreviate X)$ yields a convergent spectral sequence  $E^0_{p,q}\implies \pi_q(P^s(\barConstructionMightAbbreviate a))$, with:
%\[E^0_{p,q}=N_q\bigl(P^{p}(\barConstructionMightAbbreviate X)/P^{p+1}(\barConstructionMightAbbreviate X)\bigr)\textup{ if $p\geq s$, and }E^0_{p,q}=0\textup{ if $p<s$.}\]
%
%Examination of the structure of $\barConstructionMightAbbreviate X$ reveals that $P^{p}(\barConstructionMightAbbreviate X)/P^{p+1}(\barConstructionMightAbbreviate X)\cong (Q\barConstructionMightAbbreviate X)^{\otimes p}_{\Sigma_p}$, and moreover,
As $\pi_0(Q^{\algs}\barConstructionMightAbbreviate X)=Q^{\algs}(\pi_0\barConstructionMightAbbreviate X)=Q^{\algs}(0)=0$, % $Q\barConstructionMightAbbreviate X$ is a connected simplicial vector space, 
the $t=1$ result follows from \cite[Satz 12.1]{DoldPuppeSuspension.pdf}: if $V$ is a connected simplicial vector space then $V^{\otimes p}_{\Sigma_p}$ is $(p-1)$-connected. 
\end{proof}
Before we can give the proof of Proposition \ref{convergenceProp}, we need the following \emph{twisting lemma}, analogous to that of \cite{BK_pairings.pdf}. Before stating it, we note that $(\caldup{s}\caldup{t})X$ and $\caldup{s+t}X$ are equal by construction.
\begin{lem}
\label{DsDt=Dt+s}
The maps $\caldup{i}\delta:\caldup{n}X\to \caldup{n-1}X$ are homotopic for $0\leq i< n$.
\end{lem}
\begin{proof}
We may reindex the twisting lemma as follows: the maps 
\[\caldup{s}\delta,\caldup{s-1}\delta:\caldup{s+t}X\to \caldup{s+t-1}X\]
are homotopic whenever $s,t\geq1$. Now $\caldup{s+t}X$ is constructed as the subalgebra
\[\caldup{s+t}X:= \bigcap_{i=1}^{s+t}\ker\!\left(\barConstructionMightAbbreviate^{s+t-i}\eta \barConstructionMightAbbreviate^{i}:\barConstructionMightAbbreviate^{s+t}X\to \barConstructionMightAbbreviate^{s+t-i}KQ\barConstructionMightAbbreviate^{i}X\right)\]
of the iterated bar construction $\barConstructionMightAbbreviate^{s+t}X$, and for $0\leq i<s+t$, $\caldup{i}\delta$ is the restriction of the map $\barConstructionMightAbbreviate^i\epsilon \barConstructionMightAbbreviate^{s+t-i-1}:\barConstructionMightAbbreviate^{s+t}X\to \barConstructionMightAbbreviate^{s+t-1}X$.
Proposition \ref{IteratedBarConstructionHomotopy} gives an explicit simplicial homotopy between the maps $\barConstructionMightAbbreviate^s\epsilon \barConstructionMightAbbreviate^{t-1}$ and $\barConstructionMightAbbreviate^{s-1}\epsilon \barConstructionMightAbbreviate^{t}$. Moreover, the naturality of the construction of Proposition \ref{IteratedBarConstructionHomotopy} implies that this homotopy does indeed restrict to a homotopy of maps $\caldup{s+t}X\to \caldup{s+t-1}X$.
\end{proof}


Now that we have the twisting lemma, Proposition \ref{convergenceProp} follows:
\begin{proof}[Proof of Proposition \ref{convergenceProp}]
By Proposition \ref{prop:modifiedAdamsTower}, it is enough to prove that for any $q\geq0$ and $t\geq1$, $\pi_q(\caldup{f(q,t)}X)\to\pi_q(\caldup{t}X)$ is zero for some $f(q,t)\geq t$.
Apply $\caldup{t}\DASH$ to the diagram of functors constructed in \ref{towerWithPowers} and apply the result to $X$ to obtain a commuting diagram of functors
\[\xymatrix@R=4mm{
\caldup{f(q,t)}X
\ar[r]^-{\caldup{t}\delta}
\ar[d]
&
\cdots \ar[r]^-{\caldup{t}\delta}
&%r1c4
\caldup{t+1}X
\ar[r]^-{\caldup{t}\delta}
\ar[d]&%r1c5
\caldup{t}X
\ar@{=}[d]
\\%r1c6
\caldup{t}P^{f(q,t)-t+1}X
\ar[r]
&%r2c3
\cdots 
\ar[r]&%r2c4
\caldup{t}P^2X
\ar[r]
&%r2c5
\caldup{t}P^1X%r2c6
}\]
By the twisting lemma, \ref{DsDt=Dt+s}, the composite along the top row is homotopic to the map of interest, and factors through $\caldup{t}P^{f(q,t)-t+1}X$. If we choose $f(q,t)=2t+q-1$ when $\algcat=\algs$ and $f(q,t)=2^{t+q-1}+t-1$ when $\algcat=\restliealgs$, then Lemma \ref{connectivityOfDerivedPowers} shows that $\caldup{t}P^{f(q,t)-t+1}X$  is $q$-connected.
\end{proof}

\SubsectionOrSection{Iterated simplicial bar constructions}\label{sec:ItSimpBar}


We will state and prove a useful result on iterated simplicial bar constructions, used in the proof of the twisting lemma. The result here applies in general in the category $\algcat$ of algebras over a monad. Establishing notation, for any simplicial object $X$ in $\algCat$, we will write 
\[\trip{d}{i,q}{X}:X_q\to X_{q-1}\textup{\ and\ }\trip{s}{i,q}{X}:X_q\to X_{q+1}\]
for the $i^\textup{th}$ face and degeneracy maps out of $X_q$. Suppose that $G$ and $G'$ are endofunctors of $\algCat$, that $\Phi:G\to G'$ is a natural transformation, and that $C,C'\in\algCat$ are objects. Write $[\Phi]:\hom_{\algCat}(C,C')\to\hom_{\algCat}(GC,G'C')$ for the operator sending $m:C\to C'$ to the diagonal composite in the commuting square
\[\xymatrix@R=4mm{
GC
\ar[r]^-{\Phi_C}
\ar[d]_-{Gm}
\ar[dr]|-{[\Phi]m}
&%r1c1
G'C
\ar[d]^-{G'm}
\\%r1c2
GC'
\ar[r]_-{\Phi_{C'}}
&%r2c1
G'C'
%r2c2
}\]
There is an (augmented) simplicial endofunctor, $\frakb\in s({\algCat}^{\algCat})$, derived from the unit and counit of the adjunction:
\[\vcenter{
\def\labelstyle{\scriptstyle}
\xymatrix@C=2cm@1{
{\ \Id\,}
&
\,(F^{\algCat})^1\,
\ar@{-->}[l]|(.65){\frakd_{0,0}}
\ar[r]|(.65){\fraks_{0,0}}
&
\,(F^{\algCat})^2\,
\ar@<-1ex>[l]|(.65){\frakd_{0,1}}
\ar@<+1ex>[l]|(.65){\frakd_{1,1}}
\ar@<+1ex>[r]|(.65){\fraks_{0,1}}
\ar@<-1ex>[r]|(.65){\fraks_{1,1}}
&
\,(F^{\algCat})^3\,\makebox[0cm][l]{\,$\cdots $}
\ar[l]|(.65){\frakd_{1,2}}
\ar@<-2ex>[l]|(.65){\frakd_{0,2}}
\ar@<+2ex>[l]|(.65){\frakd_{2,2}}
%\ar[r]|(.65){\fraks_{1,2}}
%\ar@<+2ex>[r]|(.65){\fraks_{0,2}}
%\ar@<-2ex>[r]|(.65){\fraks_{2,2}}
%&
%\,(F^{\algCat})^4\,\makebox[0cm][l]{\,$\cdots $}
%\ar@<-3ex>[l]|(.65){\frakd_{0,3}}
%\ar@<-1ex>[l]|(.65){\frakd_{1,3}}
%\ar@<+1ex>[l]|(.65){\frakd_{2,3}}
%\ar@<+3ex>[l]|(.65){\frakd_{3,3}}
}}\]
The simplicial bar construction $\barConstructionMightAbbreviate =B^{\algCat}$ on $s\algCat$ is the diagonal of the bisimplicial object obtained by levelwise application of $\frakb$. That is, for $X\in s\algCat$, $\barConstructionMightAbbreviate X$ is the simplicial object with $(\barConstructionMightAbbreviate X)_q:=(F^{\algCat})^{q+1}X_q$, and with
\[\trip{d}{i,q}{\barConstructionMightAbbreviate X}:=[\frakd_{i,q}]\trip{d}{i,q}{X}.\]
The augmentation $\epsilon:\barConstructionMightAbbreviate \to \Id $ is defined on level $q$ by 
\[\epsilon_q=\frakd_{0,0}\frakd_{0,1}\cdots \frakd_{0,q}:(F^{\algCat})^{q+1}\to \Id .\]
We can now construct the simplicial homotopy needed for the twisting lemma, \ref{DsDt=Dt+s}.
\begin{prop}
\label{IteratedBarConstructionHomotopy}
The natural transformations $\epsilon_\barConstructionMightAbbreviate $ and $\barConstructionMightAbbreviate \epsilon$ from $\barConstructionMightAbbreviate ^2:s\algcat\to s\algcat$ to $\barConstructionMightAbbreviate :s\algcat\to s\algcat$ are naturally simplicially homotopic.
\end{prop}

\begin{proof}
Write $K=\barConstructionMightAbbreviate ^{2}X$ and $L=\barConstructionMightAbbreviate X$ for the source and target of these maps respectively. Noting the formulae
\[[\frakd_{iq}]^2= [\frakd_{q+i,2q}\circ\frakd_{i,2q+1}]\ \textup{and}\  [\fraks_{iq}]^2= [\fraks_{q+i+2,2q+2}\circ\fraks_{i,2q+1}],\]
we can describe the simplicial structure maps in $K$ and $L$ as follows:
\begin{alignat*}{2}
\trip{d}{iq}{L}&=[\frakd_{iq}]\trip{d}{iq}{X}\\
\trip{s}{iq}{L}&=[\fraks_{iq}]\trip{s}{iq}{X}\\
\trip{d}{iq}{K}&=[\frakd_{q+i,2q}\circ\frakd_{i,2q+1}]\trip{d}{iq}{X}\\
\trip{s}{iq}{K}&=[\fraks_{q+i+2,2q+2}\circ\fraks_{i,2q+1}]\trip{s}{iq}{X}
\end{alignat*}
We can now state an explicit simplicial homotopy between the two maps of interest. Using precisely the notation of \citeBOX[\S5]{MaySimpObj.pdf}, we define $\trip{h}{jq}{}:K_q\to L_{q+1}$, for $0\leq j\leq q$, by the formula
\[\trip{h}{jq}{}:=[\frakd_{j+1,q+2}\circ\cdots \circ\frakd_{j+1,2q+1}]\trip{s}{jq}{X}.\]
We first check that these maps satisfy the defining identities for the notion of simplicial homotopy, numbered (1)-(5) as in \citeBOX[\S5]{MaySimpObj.pdf}. Each identity can be checked in two parts (a)-(b):
{\renewcommand{\circ}{\relax}
\begin{enumerate}\squishlist
\setlength{\parindent}{.25in}
\item We must check that $\trip{d}{i,q+1}{L}\circ \trip{h}{j,q}{}=\trip{h}{j-1,q-1}{}\circ \trip{d}{i,q}{K}$ whenever $0\leq i<j\leq q$, i.e.:
\begin{enumerate}\squishlist
\setlength{\parindent}{.25in}
\item[({\makebox[.51em][c]{a}})] $\trip{d}{i,q+1}{X}\circ \trip{s}{j,q}{X}=\trip{s}{j-1,q-1}{X}\circ \trip{d}{i,q}{X}$,\textup{ and}%$\trip{\frakd}{i,q+1}{}\circ \trip{\fraks}{j,q}{}=\trip{\fraks}{j-1,q-1}{}\circ \trip{\frakd}{i,q}{}$
\item[({\makebox[.51em][c]{b}})]
$\frakd_{i,q+1}\circ
\frakd_{j+1,q+2}\circ\cdots \circ\frakd_{j+1,2q+1}=
\frakd_{j,q+1}\circ\cdots \circ\frakd_{j,2q-1}\circ
\frakd_{q+i,2q}\circ\frakd_{i,2q+1}$.
\end{enumerate}
\item We must check that $\trip{d}{j+1,q+1}{L}\circ \trip{h}{j,q}{}=\trip{d}{j+1,q+1}{L}\circ \trip{h}{j+1,q}{}$ whenever $0\leq j\leq q-1$, i.e.:
\begin{enumerate}\squishlist
\setlength{\parindent}{.25in}
\item[({\makebox[.51em][c]{a}})] $\trip{d}{j+1,q+1}{X}\circ \trip{s}{j,q}{X}=\trip{d}{j+1,q+1}{X}\circ \trip{s}{j+1,q}{X}$,\textup{ and}%$\trip{\frakd}{j+1,q+1}{}\circ \trip{\fraks}{j,q}{}=\trip{\frakd}{j+1,q+1}{}\circ \trip{\fraks}{j+1,q}{}$
\item[({\makebox[.51em][c]{b}})]
$\frakd_{j+1,q+1}\circ
\frakd_{j+1,q+2}\circ\cdots \circ\frakd_{j+1,2q+1}=
\frakd_{j+1,q+1}\circ
\frakd_{j+2,q+2}\circ\cdots \circ\frakd_{j+2,2q+1}$.
\end{enumerate}
\item We must check that $\trip{d}{i,q+1}{L}\circ \trip{h}{j,q}{}=\trip{h}{j,q-1}{}\circ \trip{d}{i-1,q}{K}$ whenever $0\leq j<i-1\leq q$, i.e.:
\begin{enumerate}\squishlist
\setlength{\parindent}{.25in}
\item[({\makebox[.51em][c]{a}})] $\trip{d}{i,q+1}{X}\circ \trip{s}{j,q}{X}=\trip{s}{j,q-1}{X}\circ \trip{d}{i-1,q}{X}$,\textup{ and}%$\trip{\frakd}{i,q+1}{}\circ \trip{\fraks}{j,q}{}=\trip{\fraks}{j,q-1}{}\circ \trip{\frakd}{i-1,q}{}$
\item[({\makebox[.51em][c]{b}})] 
$\frakd_{i,q+1}\circ
\frakd_{j+1,q+2}\circ\cdots \circ\frakd_{j+1,2q+1}=
\frakd_{j+1,q+1}\circ\cdots \circ\frakd_{j+1,2q-1}\circ
\frakd_{q+i-1,2q}\circ\frakd_{i-1,2q+1}$.
\end{enumerate}
\item We must check that $\trip{s}{i,q+1}{L}\circ \trip{h}{j,q}{}=\trip{h}{j+1,q+1}{}\circ \trip{s}{i,q}{K}$ whenever $0\leq i\leq j\leq q$, i.e.:
\begin{enumerate}\squishlist
\setlength{\parindent}{.25in}
\item[({\makebox[.51em][c]{a}})] $\trip{s}{i,q+1}{X}\circ \trip{s}{j,q}{X}=\trip{s}{j+1,q+1}{X}\circ \trip{s}{i,q}{X}$,\textup{ and}%$\trip{\fraks}{i,q+1}{}\circ \trip{\fraks}{j,q}{}=\trip{\fraks}{j+1,q+1}{}\circ \trip{\fraks}{i,q}{}$
\item[({\makebox[.51em][c]{b}})] 
$\fraks_{i,q+1}\circ
\frakd_{j+1,q+2}\circ\cdots \circ\frakd_{j+1,2q+1}=
\frakd_{j+2,q+3}\circ\cdots \circ\frakd_{j+2,2q+3}\circ
\fraks_{q+i+2,2q+2}\circ\fraks_{i,2q+1}$.
\end{enumerate}
\item We must check that $\trip{s}{i,q+1}{L}\circ \trip{h}{j,q}{}=\trip{h}{j,q+1}{}\circ \trip{s}{i-1,q}{K}$ whenever $0\leq j<i\leq q+1$, i.e.:
\begin{enumerate}\squishlist
\setlength{\parindent}{.25in}
\item[({\makebox[.51em][c]{a}})] $\trip{s}{i,q+1}{X}\circ \trip{s}{j,q}{X}=\trip{s}{j,q+1}{X}\circ \trip{s}{i-1,q}{X}$,\textup{ and}%$\trip{\fraks}{i,q++1}{}\circ \trip{\fraks}{j,q}{}=\trip{\fraks}{j,q+1}{}\circ \trip{\fraks}{i-1,q}{}$
\item[({\makebox[.51em][c]{b}})] 
$\fraks_{i,q+1}\circ
\frakd_{j+1,q+2}\circ\cdots \circ\frakd_{j+1,2q+1}=
\frakd_{j+1,q+3}\circ\cdots \circ\frakd_{j+1,2q+3}\circ
\fraks_{q+i+1,2q+2}\circ\fraks_{i-1,2q+1}$.
\end{enumerate}
\end{enumerate}
\noindent Each of these equations follows from the simplicial identities, proving that the $h_{jq}$ form a homotopy. 
Finally, we check that this homotopy is indeed a homotopy between the two maps of interest:
\begin{alignat*}{2}
\trip{d}{0,q+1}{L}\trip{h}{0,q}{}
&=
[\frakd_{0,q+1}\frakd_{1,q+2}\circ\cdots \circ\frakd_{1,2q+1}](\trip{d}{0,q+1}{X}\trip{s}{0q}{X})%
\\
&=
[\frakd_{0,q+1}\frakd_{0,q+2}\circ\cdots \circ\frakd_{0,2q+1}]\trip{\textup{id}}{X_q}{}%
\end{alignat*}
is the action of  $\epsilon_{(\barConstructionMightAbbreviate X)}$ in level $q$, and similarly,
\begin{alignat*}{2}
\trip{d}{q+1,q+1}{L}\trip{h}{q,q}{}
&=
[\frakd_{q+1,q+1}\frakd_{q+1,q+2}\circ\cdots \circ\frakd_{q+1,2q+1}](\trip{d}{q+1,q+1}{X}\trip{s}{qq}{X})
\\
&=
[\frakd_{q+1,q+1}\frakd_{q+1,q+2}\circ\cdots \circ\frakd_{q+1,2q+1}]\trip{\textup{id}}{X_q}{}%
\end{alignat*}
is the action of $\barConstructionMightAbbreviate \epsilon_{X}$ in level $q$.
}%end "do not print \circ"
\end{proof}
\end{Bousfield-Kan spectral sequence}

\begin{Constructing homotopy operations}



\SectionOrChapter{Constructing homotopy operations}
\label{sec:Constructing homotopy operations}
\label{Constructing homotopy operations}
\SubsectionOrSection{Higher simplicial Eilenberg-Mac Lane maps}%[\textbf{$\Nabla$ because} it is the right symbol, should be an upper index since it decreases homological degree.]


In what follows, we will often have a natural map $F$ whose domain and codomain both support a switch map $T$, obtained by interchanging tensor factors. Furthermore, we will so often have use for the expression $TFT$, that we introduce the shorthand $\twist F:=TFT$. %Although this notation has the potential for ambiguity, we will be careful to avoid confusion.
Although this notation is potentially ambiguous, whenever we write $\sigma FG$, for functions $F$ and $G$, we mean $(\sigma F)G$, not $\sigma(FG)$.

Let $\{\DeltaUp^k\}$ be a higher simplicial Eilenberg-Mac Lane map \citeBOX[\S3]{DwyerHtpyOpsSimpComAlg.pdf}, i.e.\ a collection of maps
\[\DeltaUp^k:(CU\otimes CV)_{i+k}\to N(U\otimes V)_i\textup{\ \ defined for $0\leq k\leq i$}\]
natural in simplicial vector spaces $U$ and $V$, such that for $k\geq0$, the identity
\[(1+\twist) \DeltaUp^k=\oldphi^k+\begin{cases}
\DeltaUp^{k-1}\partial+\partial\DeltaUp^{k-1},&\textup{if }k\geq1,\\
\DeltaUp,&\textup{if }k=0,
%\\,&\textup{if }
\end{cases}
\]
holds on classes of simplicial dimension at least $2k$, and:
\begin{itemize}
\setlength{\parindent}{.25in}
\item $\DeltaUp:CU\otimes CV\to N(U\times V)$ is the Eilenberg-Zilber map, a chain homotopy equivalence inducing the identity in dimension zero; and
\item $\oldphi^k$ is the map $(CU\otimes CV)_{i+k}\to N(U\otimes V)_i$ which vanishes except on $U_k\otimes V_k$, where its value is just the projection $U_k\otimes V_k\to N(U\times V)_k$.
\end{itemize}
Note that as $\oldphi^0$ commutes with symmetry isomorphisms, so does $\DeltaUp$.





\SubsectionOrSection{External unary homotopy operations}\label{External unary homotopy operations}
In this section we recall the definition of certain homotopy operations with domain $\pi_*V$ for any $V\in s\vect{}{}$, implicit in \citeBOX[\S4]{DwyerHtpyOpsSimpComAlg.pdf} (or Cartan or Bousfield - explicit?) and explicit in 
\citeBOX[\S3]{MR1089001}, using the functions
\[a \mapsto \DeltaUp^{n-i}(a\otimes a),\ \ N_nV \to N_{n+i}(S_2V).\]
By postcomposing with the maps $S_2V\to \Lambda^{2}V\to S^2V$, we obtain functions from $N_nV$ to $N_{n+i}(\Lambda^2V)$ and $N_{n+i}(S^2V)$.
\begin{prop}[{\cite[Lemma 4.1]{DwyerHtpyOpsSimpComAlg.pdf}, \citeBOX[\S3]{MR1089001}}]
\label{extUnaryHomotOps}
These functions descend to  well defined homotopy operations:
\begin{alignat*}{2}
\delta_i^\textup{ext}:\pi_nV&\to \pi_{n+i}(S_2V),&\quad&(2\leq i\leq n),\\
\lambda_i^\textup{ext}:\pi_nV&\to \pi_{n+i}(\Lambda^2V),&\quad&(1\leq i\leq n),\\
\sigma_i^\textup{ext}:\pi_nV&\to \pi_{n+i}(S^2V),&\quad&(1\leq i\leq n).
\end{alignat*}
These operations are linear for $1\leq i< n$.
The function $N_nV \to N_{n}(S^2V)$ given by $\overline{a}\mapsto \overline{a\otimes a}$ yields a well defined homotopy operation $\sigma_0^\textup{ext}:\pi_nV\to \pi_{n}(S^2V)$. 
For all $n\geq0$, the map $\sigma_n^\textup{ext}:\pi_nV\to\pi_{2n}(S^2V)$ satisfies
\[\sigma_n^\textup{ext}(\overline{x}+\overline{y})=\sigma_n^\textup{ext}(\overline{x})+\sigma_n^\textup{ext}(\overline{y})+\overline{(1+T)\Nabla(x\otimes y)}\quad\text{for $x,y\in ZN_nV$}.\]
%For $2\leq i\leq n$, the function $N_nV \to N_{n+i}S_2V$ yields a well-defined homotopy operation $\sigma_i^\textup{ext}:\pi_nV\to \pi_{n+i}(S_2V)$. 
%For $1\leq i\leq n$, the function $N_nV \to N_{n+i}\Lambda^2V$ yields a well-defined homotopy operation $\sigma_i^\textup{ext}:\pi_nV\to \pi_{n+i}(\Lambda^2V)$. 
%%produces a well defined homotopy operation $\sigma_i^\textup{ext}:\pi_n(V)\to \pi_{n+i}(\Lambda^2V)$. 
%The function \[a\mapsto \oldphi^n(a\otimes a),\ \ N_nV \to N_{n}S^2V\] yields a well-defined homotopy operation $\sigma_0^\textup{ext}:\pi_nV\to \pi_{n}(S^2V)$.
\end{prop}
\begin{proof}
Although all of the operations  are defined in the cited references, we will be a little more explicit about the definition of $\sigma^\textup{ext}_0$, and the final equation of the proposition.

As described in \citeBOX[\S3]{MR1089001}, we might choose to define $\sigma^\textup{ext}_0$ using a universal example, for which the cycle 
\[z\otimes z\in ZN_n(S^2\mathbb{K}_n)\cong \Ftwo \]
is the only possible representative, demonstrating that the formula $\overline{a}\mapsto \overline{a\otimes a}$ yields the correct (well defined) operation. To check that $\sigma_0^\textup{ext}\pi_0V\to \pi_0S^2V$ satisfies the stated equation, we need only check that it holds on $z_1+z_2\in ZN_0(\mathbb{K}_0\oplus \mathbb{K}_0)\cong \Ftwo \oplus \Ftwo $. But
\[\sigma_0^\textup{ext}(z_1+z_2)-\sigma_0^\textup{ext}(z_1)-\sigma_0^\textup{ext}(z_2)=z_1\otimes z_2+z_2\otimes z_1=(1+T)\Nabla(z_1\otimes z_2),\]
as $\Nabla$ is the identity in dimension zero.

To explain the equation when $n\geq1$, as $\sigma_n^\textup{ext}(\overline{x}):=\overline{(1+T)\Nabla^0(x\otimes x)}$, we obtain %into the expression  $\sigma_n^\textup{ext}(x+y)-\sigma_n^\textup{ext}(x)-\sigma_n^\textup{ext}(y)$ to obtain
\[\sigma_n^\textup{ext}(\overline{x}+\overline{y})-\sigma_n^\textup{ext}(\overline{x})-\sigma_n^\textup{ext}(\overline{y})
=
\overline{(1+T)\Nabla^0(1+T)(x\otimes y)}\]
and using the symmetry  $T\bigl((1+T)(x\otimes y)\bigr)=(1+T)(x\otimes y)$, and the fact that $\oldphi^0$ vanishes on $(1+T)(x\otimes y)$:
%\begin{alignat*}{2}
%(1+T)\Nabla^0(1+T)(x\otimes y)
%&=
%(1+\omega\Nabla^0)(1+T)(x\otimes y)%
%\\
%&=
%(\Nabla+\oldphi^0)(1+T)(x\otimes y)%
%\\
%&=
%\Nabla(1+T)(x\otimes y)%
%\end{alignat*}
\[
(1+T)\Nabla^0(1+T)(x\otimes y)
=
(1+\omega\Nabla^0)(1+T)(x\otimes y)
=
\Nabla(1+T)(x\otimes y).\qedhere
\]
%where we have used the symmetry
%\[T\bigl((1+T)(x\otimes y)\bigr)=(T+T^2)(x\otimes y)=(1+T)(x\otimes y)\]
%in order to exchange $T\Nabla^0$ and $\omega\Nabla^0$ at the second equality.
\end{proof}

\SubsectionOrSection{External binary homotopy operations}\label{External binary homotopy operations}
We will now give an account of various natural external homotopy operations, most of which are binary operations, induced by the  Eilenberg-Mac Lane shuffle map $\nabla:N_*(V)\otimes N_*(V)\to N_*(V\otimes V)$, which is also known as the Eilenberg-Zilber map.
%The homotopy operations we will present will be in the form of a natural transformation of functors $s\vect{+}{n}\to \vect{+}{n+1}$, which will be induced by the Eilenberg-Mac Lane shuffle map $\nabla:N_*(V)\otimes N_*(V)\to N_*(V\otimes V)$. \textbf{Need to clarify what is meant by quad grading.}
\begin{prop}[(Various authors)]
\label{the top external homotopy operations}
There is a natural commuting diagram:
\[\xymatrix@R=4mm{
S_2(\pi_*V)\ar[r]^-{\smash{\widetilde{\nabla}}}
\ar[d]^-{\textup{proj}}
&%r1c1
\pi_*(S_2V)\ar[d]^-{\pi_*(\textup{proj})}
\\%r1c2
\Lambda^2(\pi_*V)\ar[r]^-{\widetilde{\nabla}}
\ar[d]^-{\textup{incl}}
&%r1c1
\pi_*(\Lambda^2V)\ar[d]^-{\pi_*(\textup{incl})}
\\%r1c2
S^2(\pi_*V)\ar[r]^-{\widetilde{\nabla}}&%r2c1
\pi_*(S^2V)%r2c2
}\]
For cycles $x,y\in ZN_*(V)$ and $z\in ZN_n(V)$, the upper horizontal is determined by:
\[\overline{x}\otimes\overline{y}\mapsto \overline{x\otimes y},\]
and the lower horizontal is determined by:
\[\overline{x}\otimes\overline{y}+\overline{y}\otimes \overline{x}\mapsto\overline{\nabla(x\otimes y+y\otimes x)}\textup{, and }\overline{z}\otimes\overline{z}\mapsto\sigma^\textup{ext}_n(\overline{z}).\]
\end{prop}
\begin{proof}
During this proof, write $\widetilde{\nabla}_U$, $\widetilde{\nabla}_M$ and $\widetilde{\nabla}_L$ for the upper, middle and lower horizontal maps. We must demonstrate: that $\widetilde{\nabla}_U$  is well defined; that
\[\ker(\pi_*(\textup{proj})\circ\widetilde{\nabla}_U)\supseteq\ker(\textup{proj}),\]
so that there is a unique map $\widetilde{\nabla}_M$ for which the upper square commutes; and that one may extend the composite $\pi_*(\trace)\circ\widetilde{\nabla}_U$ along the trace map $S_2(\pi_*A)\to S^2(\pi_*A)$ using the operations
$\sigma_n^\textup{ext}$.
A simple diagram chase would then reveal that the bottom square must also commute.
%in order that $\widetilde{\nabla}_U$ extends to  a unique map $\widetilde{\nabla}_M$ for which the diagram commutes.

As $\nabla$ is a chain  map, it produces a well defined map $(\pi_*V)^{\otimes2}\to \pi_*(V^{\otimes2})$, and the fact that $\nabla=\twist\nabla$ implies that this map descends to a well defined map $\nabla_U$.

The kernel of the projection $S_2(\pi_*V)\to \Lambda^2(\pi_*V)$ is spanned by classes of the form $\overline{x}\otimes \overline{x}$, and the image under $\pi_*(\textup{proj})\circ\widetilde{\nabla}_U$ of such a class may be represented by $x\otimes x\in \Lambda^2V$, which equals zero, proving the inclusion of kernels.

Finally, to extend the composite $\pi_*(\trace)\circ\widetilde{\nabla}_U$ to $S^2(\pi_*V)$, we simply need the operations $\sigma_n^\textup{ext}$ to satisfy the equations of Proposition \ref{propOnExtendingToInvariants}, which are part of Proposition \ref{extUnaryHomotOps}.
%
%To produce $\widetilde{\nabla}_L$ is to extend the composite $\pi_*(\trace)\circ\widetilde{\nabla}_U$, along the trace map $S_2(\pi_*A)\to S^2(\pi_*A)$.
%
%Since $\nabla$ is a symmetric chain map, the formula given for $\widetilde{\nabla}_L$ does indeed return homotopy classes in $\pi_*(S_2V)$.
%
% If these maps are well defined, the rest is clear. %, however, it is not clear that either of the maps $\widetilde{\nabla}$ is well defined. 
%
%%In fact, we only need to check that  $\widetilde{\nabla}_L$ is well defined: as $\pi_*(\textup{incl})$ is a monomorphism (by citation[Prop 5.6]{BousOpnsDerFun.pdf}), it will follow that the upper map is well defined. 
%
%By general observations on natural transformations with domain the endofunctor $S^2$ of $\vect{}{}$, in order to extend
%\[\pi_*(\trace)\circ\widetilde{\nabla}_U:S_2(\pi_*V)\to \pi_*(S^2V)\]
%to a natural map $\widetilde{\nabla}_L$, it suffices to specify a natural function $\restn{(\DASH)}:\pi_n(V)\to \pi_{2n}(S^2V)$ satisfying the equation
%\[\restn{(x+y)}=\restn{x}+\restn{y}+\pi_*(\trace)(\widetilde{\nabla}_U(x\otimes y)).\] 
% Then $\widetilde{\nabla}_L$ is the unique extension of $\pi_*(\trace)\circ\widetilde{\nabla}_U$  such that $\widetilde{\nabla}_L(\alpha\otimes \alpha)=\restn{\alpha}$ for $\alpha\in\pi_*\left(V\right)$.
%%\[p=\left(S_2\pi_*(V)\overset{\trace}{\to} S^2\pi_*(V)\overset{\widetilde{\nabla}'}{\to} \pi_*(S^2V)\right)\textup{ and }\restn{(\DASH)}=\left(\pi_*(V)\overset{v\mapsto v\otimes v}{\to} S^2\pi_*(V)\overset{\widetilde{\nabla}'}{\to} \pi_*(S^2V)\right).\]
%We define $\widetilde{\nabla}_L$ in this way, with $\restn{(\DASH)}$ the operation $\sigma_n:\pi_n(V)\to \pi_{2n}(S^2V)$, which satisfies this equation by Proposition \ref{extUnaryHomotOps}.
%%This operation does not depend on the choice of representative $x$. Indeed,
%%\[\nabla((x+dh)\otimes (x+dh))=\nabla(x\otimes x)+d\trace(x\otimes h)+d(h\otimes dh)\]
%%
%%
%%
%%When $n\geq0$, this map is defined on cycles by $z\mapsto \nabla(z\otimes z)$, and the required equation is simply the expansion
%%\[\Nabla((x+y)\otimes(x+y))=\Nabla(x\otimes x)+\Nabla(y\otimes y)+(1+T)\Nabla(x\otimes y).\]
\end{proof}

\SubsectionOrSection{Homotopy operations for simplicial commutative algebras}\label{Homotopy operations for simplicial commutative algebras}
Suppose that $A\in s \algs$ is a simplicial non-unital commutative algebra, with multiplication map $\mu:S_2A\to A$. Then by composition with the map $\pi_*(\mu):\pi_*(S_2A)\to \pi_*A$, one obtains unary operations:
\[\delta_i:=\pi_*(\mu)\circ\delta_i^\textup{ext}:\pi_nA\to \pi_{n+i}A,\textup{ defined when }2\leq i\leq n,\]
and a pairing\[\mu:=\pi_*(\mu)\circ\widetilde{\nabla}:S_2(\pi_*A)\to \pi_{*}A.\]
\begin{prop}[{\cite{DwyerHtpyOpsSimpComAlg.pdf}}]
\label{omnibus on htpy of simp algs}
These operations have the following properties:
\begin{enumerate}
\item the pairing $\mu$ gives $\pi_*A$ the structure of a non-unital commutative algebra;
\item the ideal $\bigoplus_{n\geq1}\pi_nA$ is an exterior algebra;
\item the ideal $\bigoplus_{n\geq2}\pi_nA$ is a divided power algebra, with divided square given by the \emph{top $\delta$-operation}, i.e.\ $x\mapsto \delta_nx\text{ for }x\in\pi_nA$;
\item the \emph{non-top operations}, $\delta_i:\pi_nA\to \pi_{n+i}A$ for $2\leq i<n$, are linear;
\item there holds the following \emph{Cartan formula}: for $x\in\pi_nA$, $y\in \pi_mA$ and $2\leq i\leq n$
\[\delta_i(xy)=\begin{cases}
y^2\delta_i(x),&\text{if }m=0;\\
0,&\text{otherwise};
\end{cases}
\]
\item \label{deltaademsunstable} the \emph{$\delta$-Adem relations} hold: if $\delta_i\delta_jx$ is defined, and $i<2j$, then
\[\delta_i\delta_jx:=\sum_{s=\lceil(i+1)/2\rceil}^{\lfloor(i+j)/3\rfloor}\binom{j+s-i-1}{ j-s}\delta_{i+j-s}\delta_sx.\]
\end{enumerate}
\end{prop}
A few comments are in order. Firstly, the proposition distinguishes between the \emph{top} and \emph{non-top} $\delta$-operations, as they have rather different behaviour --- this will be a recurring pattern. Secondly, it is not immediately obvious that the $\delta$-Adem relations make sense, in that it is not obvious that every term in the right hand side is defined. This does indeed happen, by Lemma \ref{lemOnAdemChangeInMDeltaPlain} (to follow). 

Now we may define an associative unital algebra, $\deltaalg$, as the algebra generated by $\delta_i$ for $i\geq2$, subject to relations
\[\delta_i\delta_j:=\sum_{s=\lceil(i+1)/2\rceil}^{\lfloor(i+j)/3\rfloor}{j+s-i-1\choose j-s}\delta_{i+j-s}\delta_s\textup{ when $i<2j$}.\]
We will say that a sequence $I=(i_\ell,\ldots,i_1)$ of integers $i_j\geq2$ is \emph{$\delta$-admissible} if $i_{j+1}\geq 2i_j$ for $1\leq j <\ell$. For any sequence $I=(i_\ell,\ldots,i_1)$, write $\delta_I$ for the composite $\delta_{i_\ell}\cdots \delta_{i_1}$. Then this relation evidently allows us to write any $\delta_I$ in $\deltaalg$ as a sum of composites $\delta_J$ in which $J$ is $\delta$-admissible. In fact, it follows from \cite[Proposition 2.7]{MR1089001} that the algebra $\deltaalg$ has an \emph{admissible basis}, consisting of those $\delta_I=\delta_{i_\ell}\cdots \delta_{i_{1}}$ with $I$ a $\delta$-admissible sequence. 

It then makes sense to make the following definition. Suppose that $I$ is any non-empty sequence of integers at least $2$, and $J$ is a $\delta$-admissible sequence. Then we will say that \emph{$I$ produces $J$ in $\deltaalg$}, denoted $\produces{I}{J}{\deltaalg}$ if, when $\delta_I$ is written in the $\delta$-admissible basis of $\deltaalg$, $\delta_J$ appears with non-zero coefficient. In this case, $J$ must be $\delta$-admissible, and $I$ must be $\delta$-inadmissible unless $J=I$.

Proposition \ref{omnibus on htpy of simp algs} does \emph{not} state that $\pi_*A$ is a left module over $\deltaalg$, due to the fact that the $\delta$-operations are not always defined (or even linear). We define
\[\minDimDelta(I):=\max\{(i_1),\,(i_2-i_1),\,(i_3-i_2-i_1),\,\ldots,\,(i_{\ell}-\cdots-i_1)\},
\]
following the convention that $\max(\emptyset)=-\infty$, for any sequence $I$ of integers $i_j\geq2$ (or more generally, for any sequence of non-negative integers).  The intent of this definition is that the composite $\delta_I$, by which we mean
\[\pi_{n}A\overset{\delta_{i_1}}{\to}\pi_{n+i_1}A\overset{\delta_{i_2}}{\to}\cdots \overset{\delta_{i_\ell}}{\to}\pi_{n+i_1+\cdots +i_\ell}A,\]
is defined if and only if $n\geq \minDimDelta(I)$.
Note that when $I$ is a non-empty \emph{$\delta$-admissible} sequence,
\[\minDimDelta(I)=i_{\ell}-i_{\ell-1}-\cdots -i_1=:e(I),\]
the \emph{Serre excess of $I$}. Moreover, if $I$ is $\delta$-admissible, then for any expression $\delta_{i_{\ell}}\cdots \delta_{i_{1}}x$ there is some $k$ with $0\leq k\leq \ell$ such that each of the $k$ the operations $\delta_{i_{\ell}}\cdots \delta_{i_{\ell-k+1}}$ are acting as top operations, and each of the remaining $\ell-k$ are acting as non-top operations.

The following lemma assures us that the $\delta$-Adem relations make sense even in \emph{(\ref{deltaademsunstable})}.
%\begin{lem}\label{lemOnAdemChangeInMDeltaPlain}
%Suppose that $i,j\geq2$ and $i<2j$, and $(i+1)/2\leq s\leq (i+j)/3$. Then $i+j-s\geq 2s$, $i+j-s\geq2$, $s\geq2$, so that the $\delta$-Adem relation writes $\delta_i\delta_j$ as a sum of $\delta$-admissible composites. Moreover,  $\minDimDelta(i,j)\geq \minDimDelta(i+j-s,s)$.
%\end{lem}
%\begin{proof}
%The only tricky inequality is $\minDimDelta(i,j)\geq \minDimDelta(i+j-s,s)$. Now the right hand side must equal $i+j-2s$, and the left hand side is at least $j$. The result follows, since $2s\geq i+1$.
%\end{proof}
\begin{lem}
\label{lemOnAdemChangeInMDeltaPlain}
If $\produces{I}{J}{\deltaalg}$, then $\minDimDelta(I)\geq\minDimDelta(J)$.
\end{lem}
\begin{proof}
It is enough to show this result when $I$ and $J$ are distinct and have length two, in light of the evident algorithm for expressing $\delta_I$ in terms of admissible composites. In the length two case it can be checked directly from the format of the $\delta$-Adem relation, and the inequality is in fact strict (unless $I$ is itself $\delta$-admissible).
\end{proof}
%\noindent Thus, even though $\pi_*$ is not a $\deltaalg$-module, any composite $\delta_I:\pi_nA\to \pi_mA$ that is defined may be written as a sum
Finally, one should note that these operations generate all of the operations in the category $\PA{\algs}$, and that all of the relations between the operations in $\PA{\algs}$ are implied by those presented here. Goerss \citeBOX[\S2]{MR1089001} presents this information as follows. First, he observes that there is a \emph{K\"unneth theorem} available:
\begin{prop}
\label{salgs homotopy kunneth}
Suppose that $A_1$ and $A_2$ are models in $s\algs$. Then $\pi_*(A_1\sqcup A_2)$, which is the coproduct of $\pi_*A_1$ and $\pi_*A_2$ in $\PA{\algs}$, may be calculated as the non-unital commutative algebra coproduct of $\pi_*A_1$ and $\pi_*A_2$.
\end{prop}
%This may be read as a theorem about coproducts of free $\algs$-$\Pi$-algebras, since $\pi_*(A_1\sqcup A_2)$ is the coproduct in $\PA{\algs}$ of $\pi_*(A_1)$ and $\pi_*(A_2)$, the theorem being that coproducts of free $\algs$-$\Pi$-algebras may be taken in the category $\algs$.
\noindent After giving the calculation on a single sphere, the homotopy of finite models will be determined by this theorem and the Cartan formula. The structure defining $\PA{\algs}$ is then well understood in light of:
\begin{prop}[{\cite[Proposition 2.7]{MR1089001}}]
\label{homotopy of comm alg sphere}
\noindent For $n\geq0$, let $\imath_n$ be the fundamental class in $\pi_n(\mathbb{S}^\algs_n)$. There  are isomorphisms of non-unital commutative algebras:
\begin{alignat*}{2}
\pi_*(\mathbb{S}^\algs_0)
&\cong
S(\CommOperad)[\imath_0]=F^{\algs}[\imath_0];%
\\
\pi_*(\mathbb{S}^\algs_n)
&\cong
\makebox[\widthof{$\Lambda$}][c]{$\Lambda$}(\CommOperad)[\delta_I(\imath_n)\ |\ \textup{$I$ is $\delta$-admissible, $e(I)\leq n$}]%
&\ \ &\makebox[\widthof{for $n\geq1$}][c]{for $n\geq1$;}\\
% Left hand side
\pi_*(\mathbb{S}^\algs_n)
% Relation
&\cong
% Right hand side
\makebox[\widthof{$\Lambda$}][c]{$\Gamma$}(\CommOperad)[\delta_I(\imath_n)\ |\ \textup{$I$ is $\delta$-admissible, $e(I)< n$}]%
% Comment
&\ \ &\makebox[\widthof{for $n\geq1$}][c]{for $n\geq2$.}%\text{for $n\geq2$.}
\end{alignat*}
\end{prop}

\SubsectionOrSection{Homotopy operations for simplicial Lie algebras}\label{Homotopy operations for simplicial Lie algebras}
Suppose that $L\in s \liealgs$ is a simplicial Lie algebra with bracket $[\,,]:\Lambda^2L\to L$. There are unary operations:
\[\lambda_i:=\pi_*([\,,])\circ\lambda_i^\textup{ext}:\pi_nL\to \pi_{n+i}L,\textup{ defined when }1\leq i\leq n,\]
which we write on the right as $x\mapsto x\lambda_i$, and a bracket
\[[\,,]:=\pi_*([\,,])\circ\widetilde{\nabla}:\Lambda^2(\pi_*L)\to \pi_{*}L.\]

Alternatively, one can suppose that $L\in s \restliealgs$ is a simplicial restricted Lie algebra with bracket $[\,,]:S^2L\to L$, and construct operations:
\begin{gather*}
\lambda_i:=\pi_*([\,,])\circ\sigma_i^\textup{ext}:\pi_nL\to \pi_{n+i}L,\textup{ defined when }0\leq i\leq n\textup{, and}\\
[\,,]:=\pi_*([\,,])\circ\widetilde{\nabla}:S^2(\pi_*L)\to \pi_{*}L.
\end{gather*}
\begin{prop}[{\cite{6Author.pdf}, \citeBOX[\S8]{CurtisSimplicialHtpy.pdf}}]
\label{omnibus on htpy of Lie algs}
For $L\in s \liealgs$, these operations satisfy:
\begin{enumerate}
\item \label{meep1} the bracket gives $\pi_*L$ the structure of a Lie algebra;
\item \label{meep2} the ideal $\bigoplus_{n\geq1}\pi_nL$ is a restricted Lie algebra, with restriction given by the \emph{top $\lambda$-operation}, i.e.\ $\restn{x}=\lambda_nx\text{ for }x\in\pi_nL$;
\item \label{meep3} the \emph{non-top} operations, $\lambda_i:\pi_nL\to \pi_{n+i}L$ for $i<n$, are linear;
\item \label{meep4} there holds the following \emph{Cartan formula}, for $x\in \pi_*L$, $y\in \pi_nL$ and $1\leq i\leq n$:
\[[x,y\lambda_i]=\begin{cases}
[y,[x,y]],&\text{if }i=n;\\
0,&\text{otherwise};
\end{cases}
\]
\item \label{meep5} the \emph{$\Lambda$-Adem relations} hold: if $x\lambda_j\lambda_i$ is defined, and $i>2j$, then
\[x\lambda_j\lambda_i=\sum_{k=0}^{(i-2j)/2-1}\binom{i-2j-2-k}{ k}x\lambda_{i-j-1-k}\lambda_{2j+1+k}.\]
\end{enumerate}
For $L\in s \restliealgs$, we may omit (\ref{meep1}), modify (\ref{meep2}) to state that the whole of $\pi_*L$ is restricted, and modify (\ref{meep3})-(\ref{meep5}) to include $\lambda_0$.
\end{prop}
Similar comments apply as for commutative algebras, for example, one needs Lemma \ref{lemOnAdemChangeInMLambdaPlain} (to follow) to understand why this unstable relation makes sense. 

The well known \emph{$\Lambda$-algebra} is the unital associative algebra generated by $\lambda_i$ for $i\geq0$, subject to relations \[\lambda_j\lambda_i=\sum_{k=0}^{(i-2j)/2-1}\binom{i-2j-2-k}{ k}\lambda_{i-j-1-k}\lambda_{2j+1+k}\textup{\ for $i>2j$.}\]
We say that a sequence $I=(i_\ell,\ldots,i_1)$ of non-negative integers is \emph{$\Lambda$-admissible} if $i_{j+1}\leq 2i_j$ for $1\leq j <\ell$. 
For any sequence $I=(i_\ell,\ldots,i_1)$, if we write $\lambda_I$ for the element $\lambda_{i_1}\cdots \lambda_{i_\ell}$ in $\Lambda$, then the $\Lambda$-algebra has the evident admissible basis, and we may make sense of the symbol $\produces{I}{J}{\Lambda}$. Note that the ordering of the generators in $\lambda_I$ and $\delta_I$ are opposite, for consistency with the fact that we write the $\lambda$-operations on $\pi_*L$ \emph{on the right}. Thus, we may think of $\lambda_I$ as the composite operator  
\[\pi_{n}L\overset{\lambda_{i_1}}{\to}\pi_{n+i_1}L\overset{\lambda_{i_2}}{\to}\cdots \overset{\lambda_{i_\ell}}{\to}\pi_{n+i_1+\cdots +i_\ell}L,\]
again defined only when $\minDimDelta(I)\leq n$, so that $\pi_*L$ is \emph{not} a right module over $\Lambda$. We will say, however, that it is an \emph{unstable partial right $\Lambda$-module}.
Note that when $I$ is a non-empty \emph{$\Lambda$-admissible} sequence,
$\minDimDelta(I)=i_1$, not the Serre excess, reflecting the observation that when $x\lambda_{i_1}\cdots \lambda_{i_\ell}$ is a $\Lambda$-admissible composite, the only operation that  can be a top (i.e.\ restriction) operation is $\lambda_{i_1}$.
The following lemma assures us that the $\Lambda$-Adem relations make sense in \emph{(\ref{meep5})}.
%\begin{lem}\label{lemOnAdemChangeInMDeltaPlain}
%Suppose that $i,j\geq2$ and $i<2j$, and $(i+1)/2\leq s\leq (i+j)/3$. Then $i+j-s\geq 2s$, $i+j-s\geq2$, $s\geq2$, so that the $\delta$-Adem relation writes $\delta_i\delta_j$ as a sum of $\delta$-admissible composites. Moreover,  $\minDimDelta(i,j)\geq \minDimDelta(i+j-s,s)$.
%\end{lem}
%\begin{proof}
%The only tricky inequality is $\minDimDelta(i,j)\geq \minDimDelta(i+j-s,s)$. Now the right hand side must equal $i+j-2s$, and the left hand side is at least $j$. The result follows, since $2s\geq i+1$.
%\end{proof}
\begin{lem}
\label{lemOnAdemChangeInMLambdaPlain}
If $\produces{I}{J}{\Lambda}$, then $\minDimDelta(I)\geq\minDimDelta(J)$, and $J$ does not contain zero unless $I$ does.
\end{lem}
These operations generate all of the operations in each of the categories $\PA{\liealgs}$ and $\PA{\restliealgs}$, and the relations presented here are sufficient. 


From  \cite[Theorem 8.8 and proof]{CurtisSimplicialHtpy.pdf} and \cite{6Author.pdf} we gather:
\begin{prop}
\label{Lie homotopy operations}
For $V\in\vect{}{1}$,   choose a homogeneous basis of $V$, and construct from it a monomial  basis $B$ of $\Lambda(\LieOperad)V$ (such as any choice of Hall basis). Then:
\begin{align*}
F^{\PA{\liealgs}}V&=\Ftwo \left\{\lambda_Ib\ \middle|\ \genfrac{}{}{0pt}{}{b\in B_t\textup{, $I$ $\Lambda$-admissible with }\minDimDelta(I)\leq t,}{\textup{$I$ does not contain 0}}\right\}\textup{ and;}
\\
F^{\PA{\restliealgs}}V&=\Ftwo \left\{\lambda_Ib\ \middle|\ \genfrac{}{}{0pt}{}{b\in B_t\textup{, $I$ $\Lambda$-admissible with }\minDimDelta(I)\leq t,}{\textup{$I$ does not contain 0 when $t=0$}}\right\}
\\
&=\Ftwo \left\{\lambda_I\bigl(\restnRepeated{b}{r}\bigr)\ \middle|\ \genfrac{}{}{0pt}{}{b\in B_t\textup{, $I$ $\Lambda$-admissible with }\minDimDelta(I)< t,}{\textup{$r\geq0$}}\right\}.
\end{align*}
\end{prop}
We will now emulate  Goerss' method of calculating the cohomology of GEMs in $s\algs$ (c.f.\ \S\ref{The example of simplicial commutative F2-algebras}, \cite{GoerssHiltonMilnor.pdf} and \citeBOX[\S11]{MR1089001}) by giving a Hilton-Milnor decomposition for the calculation of the free $\liealgs$-$\Pi$-algebra on a finite-dimensional object of $\vect{}{1}$, using \cite[Proposition 3.1]{Schles-SimpLieRing.pdf}. For any $i\geq0$, write $\Sigma^i\Ftwo \in\vect{}{1}$ for a one-dimensional vector space concentrated in homological dimension $i$.

For any finite collection of indices $i_1,\ldots,i_n\geq 0$, we would like to calculate:
\[F^{\PA{\liealgs}}(\Sigma^{i_1}\Ftwo\oplus\cdots \oplus\Sigma^{i_n}\Ftwo)=\pi_*F^\liealgs(\mathbb{K}_{i_1}\oplus\cdots \oplus \mathbb{K}_{i_n}).\]
We obtain a decomposition of $\pi_*F^\liealgs(\mathbb{K}_{i_1}\oplus\cdots \oplus \mathbb{K}_{i_n})$ as follows. For any monomial $b$ is the free Lie algebra on $\{x_1,\ldots,x_n\}$ and collection of $n$ vector spaces $A_{1},\ldots,A_n$, there is a corresponding tensor product $w_b(A_{1},\ldots,A_n)$. For example, one defines
\[w_{[[x_2,x_1],x_3]}:=A_2\otimes A_1\otimes A_3.\]
Moreover, for each monomial $b$ there is an evident function \[w_b(A_1,\ldots,A_n)\to F^\liealgs(A_1\oplus\cdots\oplus A_n),\]
given in our example by $a_2\otimes a_1\otimes a_3\mapsto [[a_2,a_1],a_3]$.

Iteration of the procedure described in \citeBOX[\S4.3]{Neisendorfer.pdf}, using the formula of \cite[Proposition 3.1]{Schles-SimpLieRing.pdf}, we obtain a Hall basis $B$ of the free Lie algebra on $\{x_1,\ldots,x_n\}$, with the property that the resulting map
\[\bigoplus_{b\in B}F^{\liealgs}w_b(A_1,\ldots,A_n)\to F^\liealgs(A_1\oplus\cdots\oplus A_n)\]
is an isomorphism, natural  in $A_1,\ldots,A_n$.

Returning to the calculation desired, we have an isomorphism in $s\vect{}{1}$:
\[\bigoplus_{b\in B}F^{\liealgs}w_b(\mathbb{K}_{i_1},\ldots,\mathbb{K}_{i_n})\to F^\liealgs(\mathbb{K}_{i_1}\oplus\cdots \oplus \mathbb{K}_{i_n}).\]
Moreover, if we follow \citeBOX[\S11]{MR1089001} by writing $j_k(b)$ for the number of appearances of $x_k$ in the monomial $b$, there is a homotopy equivalence
\[w_b(\mathbb{K}_{i_1},\ldots,\mathbb{K}_{i_n})\simeq \mathbb{K}_{\sum_{k=1}^n J_k(b)i_k}.\]
Thus, on homotopy there is a decomposition:
\[\bigoplus_{b\in B}F^{\PA{\liealgs}}\Sigma^{(\sum_{k=1}^n j_k(b)i_k)}\Ftwo\overset{\cong}{\to} F^{\PA{\liealgs}}(\Sigma^{i_1}\Ftwo\oplus\cdots \oplus\Sigma^{i_n}\Ftwo),\]
and evidently the fundamental class of a summand on the left maps to the corresponding Lie bracket of fundamental classes on the right. This proves the first part of Proposition \ref{Lie homotopy operations}.


\end{Constructing homotopy operations}
\begin{Constructing cohomology operations}
\SectionOrChapter{Constructing cohomology operations}
\label{sec:Constructing cohomology operations}
\label{Constructing cohomology operations}
\SubsectionOrSection{Higher cosimplicial Alexander-Whitney maps}
Let $\{D^k\}$ be a special cosimplicial Alexander-Whitney map \cite[Proposition 5.2]{turner_opns_and_sseqs_I.pdf}, i.e.\  maps
\[D^k:(CR\otimes CS)^{i+k}\to C(R\otimes S)^{i}\textup{\ \ for $i,k\geq0$},\]
natural in cosimplicial vector spaces $R,S$,
with the properties:
\begin{enumerate}
\setlength{\parindent}{.25in}
\item $dD^k+D^kd=(1+\twist)D^{k-1}$ for $k\geq1$;
\item $D^0$ is a chain homotopy equivalence inducing the identity in dimension zero;
\item the restriction of $D^k$ to $C^{i}R\otimes C^{j}S$ is zero unless $i\geq k$ and $j\geq k$; and
\item $D^k$ maps $C^{k}R\otimes C^{k}S$ identically onto $C^{k}(R\otimes S)$.
\end{enumerate}
It is a natural convention to define $D^k=0$ for whenever either $k<0$ or $i<0$, in which case the relation $dD^k+D^kd=(1+\twist)D^{k-1}$ holds for any $k$.

Maps dual to these are described in detail, under the name \emph{special cup-$k$ product}, by Singer in \cite[Definitions 1.91 and 1.94]{MR2245560}, and developed originally in \cite{DoldUber}. Indeed, we will use these maps later, and denote them
\[(D^k)^{\star}:C(U\otimes V)_{i}\to (CU\otimes CV)_{i+k}\textup{\ \ for $i,k\geq0$},\]
natural in $U,V\in s\vect{}{}$. The sense in which these maps are dual to the $D^k$ is as follows. Suppose that $U,V\in s\vect{}{}$. Then there is a commuting diagram (for $i,k\geq0$):
\[\xymatrix@R=4mm@C=17mm{
(C\dual U\otimes C\dual V)^{i+k}\ar[r]^-{D^k}
\ar[d]&%r1c1
C(\dual U\otimes\dual V)^{i}\ar[d]\\%r1c2
(\dual (CU\otimes CV))^{i+k}\ar[r]^-{((D^k)^{\star})^*}
&%r2c1
(\dual C(U\otimes V))^{i}%r2c2
}\]


\SubsectionOrSection{External unary cohomotopy operations}\label{External unary cohomotopy operations}
%\begin{shaded}\tiny
%In this section we recall the definition of certain homotopy operations with domain $\dual(\pi_*V)$ (or $\pi^*\dual V$) for any $V\in s\vect{}{}$, \textbf{reference}, using the function
%\[D^*_{n-i}\oldphi(\alpha\otimes \alpha)+D^*_{n-i+1}\oldphi(\alpha\otimes d\alpha),\ \ \dual(N_nV) \to \dual(N_{n+i}(S^2V)),\]
%where we regard the $D^*_k$ as maps into $\dual(N_*(S^2V))$, and the map $\oldphi$ is that described in \citeBOX[\S1.9]{MR2245560}. Namely, [\textbf{potentially redundant:} as the normalized complex is naturally a retract of the unnormalized complex, we may define] $\oldphi:\dual\left(N_mV\right)\otimes \dual\left(N_mV\right)\to \dual\left(N_m(V\otimes V)\right)$ to be
%\[\left(C_mV\overset{\alpha}{\to}\Ftwo \right)\otimes \left(C_mV\overset{\beta}{\to}\Ftwo \right)\mapsto\left(C_m(V\otimes V)\overset{\alpha\otimes\beta}{\to}\Ftwo \otimes \Ftwo \overset{\textup{mult}}{\to}\Ftwo \right).\]
%\begin{prop}[same refs, I guess]
%These functions descend to  well defined cohomotopy operations:
%\[\ExtCohOp^k:\dual(\pi_{n}V)\to \dual(\pi_{n+k}(S^2V)),\text{ \ zero unless $0\leq k\leq n$.}\]
%\end{prop}
%\begin{proof}
%One checks that these cochain operations induce well defined operations $\ExtCohProd$ and $\ExtCohOp^k$ by the same arguments as in \citeBOX[\S1.12]{MR2245560}.
%\end{proof}
%\end{shaded}
In this section we recall the definition of certain cohomotopy operations with domain $\pi^*U$ for any $U\in c\vect{}{}$, using the function
\[D^{n-i}(\alpha\otimes \alpha)+D^{n-i+1}(\alpha\otimes d\alpha),\ \ C^nU \to C^{n+i}(S_2U).\]
\begin{prop}
\label{external singly simplicial steenrods}
These functions descend to  well defined linear operations:
\[\ExtCohOp^k:\pi^{n}U\to \pi^{n+k}(S_2U),\text{ \ zero unless $0\leq k\leq n$.}\]
\end{prop}
\begin{proof}
One checks that these cochain operations induce well defined operations $\ExtCohProd$ and $\ExtCohOp^k$ by the same arguments as in \citeBOX[\S1.12]{MR2245560}.
\end{proof}
Now if $U=\dual V$ for some $V\in s\vect{}{}$, then we may use the natural transformation $S_2\dual\to\dual S^2$ to form the following composite, also denoted $\ExtCohOp^k$:
\[\pi^n\dual V\overset{\ExtCohOp^k}{\to}\pi^{n+k}S_2\dual V\to \pi^{n+k}\dual S^2V.\]
This will be part of the process we use to define cohomology operations.
%and under the natural isomorphism $\pi^*\dual\cong \dual\pi_*$, this operation is the operation defined before the shading.



\SubsectionOrSection{Linearly dual homotopy operations}\label{Linearly dual homotopy operations}
Whenever $V\in s\vect{}{}$ has $\pi_*V$ of finite type, the linear maps $\ExtCohOp^k:\pi^n\dual V\to\pi^{n+k}\dual S^2V$ induce dual operators
\[\pi_*(S^2V)\to \pi_{*-k}V.\]
Following \citeBOX[\S3]{MR1089001}, one can do much better than just this observation, giving a direct definition of homology operations, valid under any circumstances, whose duals are the $\ExtCohOp^k$.

At this point, we introduce a new notational convention. The cohomotopy operation $\ExtCohOp^k$ is of primary interest, and we prefer to allow it its standard symbol (albeit with the attached subscript). On the other hand, we are about to product a homotopy operation of which it is the dual. Our convention in this setting is to write $(\ExtCohOp^k)^\star$ for the homotopy operation, using a star, not an asterisk. That is, for \emph{any} $V\in s\vect{}{}$, there is an operation
\[(\ExtCohOp^k)^\star:\pi_*(S^2V)\to \pi_{*-k}V\]
such that the following diagram commutes:
\[\xymatrix@R=4mm@C=20mm{
\pi^{n+k}\dual S^2V&%r1c1
\pi^n\dual V\ar[l]_-{((\ExtCohOp^k)^\star)^*}
\ar[dl]^-{\ExtCohOp^k}
\\%r1c2
\pi^{n+k}S_2\dual V\ar[u]
}\]
%whose dual equals the operation of \S\ref{External unary cohomotopy operations}:
%\[((\ExtCohOp^k)^\star)^*=\ExtCohOp^k:\pi^{*-k}(V)\to \pi_*(S^2V)\]
%
%directly, without the double dualization as described below. %Setting $V=Q^{\calc}B^{\calc}X$ and precomposing with the map $(\psi_\calc)_*:\pi_*(V)\to\pi_{*-1}(S^2V)$ yields homology operations on $H_*^\calc$ whose duals are the cohomology operations of \S\ref{generic coh ops section}.

In order to define these precursor homotopy operations, Goerss \cite[Proposition 3.7]{MR1089001}
shows that any element of $x\in \pi_m(S^2V)$ can be written as a sum
\[x=\sum_j\pi_*(1+T)(y_j\otimes z_j)+\sum_{\smash{0\leq k\leq \lfloor m/2\rfloor}}\sigma_k(w_k),\]
where $w_k\in \pi_{m-k}V$ for $0\leq k\leq\lfloor m/2\rfloor$, and $y_j,z_j\in \pi_{*}V$, and that we may \emph{define}:
\[(\ExtCohOp^k)^\star(x):=w_k.\]
%That is, we are defining a map $(\ExtCohOp^k)^\star$ by this requirement, and the method of \citeBOX[\S3]{MR1089001} shows that its linear dual, $((\ExtCohOp^k)^*)^*$ equals $\ExtCohOp^k$. Of course, the chosen \emph{symbol} $(\ExtCohOp^k)^*$ \textbf{(use stars here!)} is then only appropriate in the finite-dimensional case, but this definition of a map $(\ExtCohOp^k)^*$ works in any setting.

One might only need to determine the operations $(\ExtCohOp^k)^\star$ for $k<m/2$, so that when $m$ is even we may ignore the dual of the top operation, $(\ExtCohOp^{m/2})^\star$. In this case, it is more convenient to rewrite the key equation as: 
\[x=\widetilde{\nabla}(v)+\sum_{\smash{0\leq k< m/2}}\sigma_k(w_k),\]
where $v\in(S^2(\pi_*V))_{m}$ and $w_k\in \pi_{m-k}V$ for $0\leq k<m/2$.


\SubsectionOrSection{External binary cohomotopy operations}
%\begin{shaded}\tiny
%Consider the cochain level pairing
%\[\alpha\otimes\beta\mapsto D_0^*\oldphi(\alpha\otimes\beta),\quad\dual(N_*V)\otimes \dual(N_*V)\to \dual(N_*(S^2V)).\]
%\begin{prop}
%This pairing induces a pairing on cohomotopy,
%\[\ExtCohProd:S_2(\dual(\pi_*V))\to \dual(\pi_*(S^2V)),\]
%with the property that $\ExtCohProd(\alpha\otimes \alpha)=\ExtCohOp^k\alpha$ for $\alpha\in\dual(\pi_kV)$.
%\end{prop}
%\end{shaded}
\begin{prop}
\label{the external cohomotopy pairing}
Suppose that $U\in c\vect{}{}$. Then there is a pairing
\[\ExtCohProd:S_2(\pi^*U)\to \pi^*(S_2U),\]
defined by $x\otimes y\mapsto D^0(x\otimes y)$,
with the property that $\ExtCohProd(\alpha\otimes \alpha)=\ExtCohOp^k\alpha$ for $\alpha\in \pi^kU$.
\end{prop}
Unsurprisingly, these bear relation to the homotopy operation $\widetilde{\nabla}:\pi_*S^2V\to S^2\pi_*V$, via a commuting diagram, for any $V\in s\vect{}{}$:
\[\xymatrix@R=4mm{
\dual \pi_* S^2 V\ar[r];[]_-{(\widetilde{\nabla})^*}
&\dual S^2\pi_*  V
\\%r1c2
\pi^* S_2\dual  V\ar[u]&
S_2\pi^* \dual  V\ar[l]_-{\mu_\textup{ext}}\ar[u]
%r2c2
}\]
We might have denoted $\widetilde{\nabla}$ by $\mu_\textup{ext}^\star$, but decided against the idea.

\SubsectionOrSection{Chain level structure for cohomology operations; the maps $\xi_\calc $ and $\psi_\calc$}\label{chain level structure}

We will now give generalisations of Goerss' constructions in \citeBOX[\S5]{MR1089001} to yield useful structure on the complexes calculating cohomology.  Suppose that $X\in s\calc$ is almost free, with $V_s\subseteq X_s$ the freely generating subspace. Then for each $s$, the functor 
\[\Hom_\calc(X_s,\DASH)\cong \Hom_{\vect{}{}}(V_s,U^\calc\DASH)\]
is naturally an $\Ftwo $-vector space. Writing $\phi_s=F^\calc(\Delta):X_s\to X_s\sqcup X_s$, the addition operation `$\star$' on $\Hom_\calc(X_s,\DASH)$ is given by $f\star g= (f\sqcup g)\circ\phi_s$. Now let
$\overline{\xi}_\calc=((d_0\sqcup d_0)\phi_s)\star(\phi_{s-1}d_0)$, the sum of maps in the $\Ftwo $-vector space $\textup{Hom}_\calc(X_s,X_{s-1}\sqcup X_{s-1})$. It is completely formal to check that $\overline{\xi}_\calc$ maps to zero in the group
\[\textup{Hom}_\calc(X_s,X_{s-1}\times X_{s-1})=\textup{Hom}_\calc(X_s,X_{s-1})\times \textup{Hom}_\calc(X_s,X_{s-1}),\]
and thus $\overline{\xi}_\calc$ factors through a unique map $\xi_\calc:X_s\to X_{s-1}\smashcoprod X_{s-1}$. Furthermore, $\xi_\calc$ enjoys the symmetry $\tau\xi_\calc=\xi_\calc$, and it is again formal to verify the analogue of \cite[Lemma 5.5]{MR1089001}:
\begin{lem}
\label{psi is basically the quadratic part}
When the equation $\quadratic_\calc\circ\mu_V=\quadratic_\calc\circ \epsilon_{F^\calc V} +\quadratic_\calc\circ {F^\calc \epsilon_V}$ of \S\ref{quadratic part section} is satisfied, the map $Q^\calc\xi_\calc$ induces a chain map of degree $-1$ on normalized complexes:
\[N_s(Q^{\calc}X)\to N_{s-1}((Q^{\calc}(X\smashcoprod X))^{\Sigma_2}).\]
The composite
\[\psi_\calc:\left(N_s(Q^{\calc}X)\overset{Q^\calc\xi_\calc}{\to} N_{s-1}((Q^{\calc}(X\smashcoprod X))^{\Sigma_2})\overset{j_\calc}{\to} N_{s-1}(S^2(Q^{\calc}X))\right),\]
is essentially $\quadratic_\calc$, in that if $v\in V_s\cap N_sX$ represents an element of $N_sQ^\calc X$, writing $d_0v=f(w_j)$ for $w_j\in V_{s-1}$, we have $\psi_\calc(v)=\quadratic_\calc(f)(w_j)\in S^2(V_{s-1})$.
\end{lem}
The typical use of this structure is to define cohomology operations using the external cohomotopy operations defined above, i.e.\ natural operations on $H^*_\calc X=\pi^*(\dual(Q^\calc B^\calc X))$ defined by the composites:
\begin{gather*}
H_\calc^{n_1}X\otimes H_\calc^{n_2}X\overset{\ExtCohProd}{\to} \pi^{n_1+n_2}\dual(S^2(Q^\calc B^\calc X))\overset{\psi_\calc^*}{\to} H_\calc^{n_1+n_2+1}X,\\
H_\calc^{n}X\overset{\ExtCohOp^k}{\to} \pi^{n+k}\dual(S^2(Q^\calc B^\calc X))\overset{\psi_\calc^*}{\to} H_\calc^{n+k+1}X.
\end{gather*}
These operations are the duals of natural homology co-operations, defined using the maps of \S\ref{Linearly dual homotopy operations} and \S\ref{External binary homotopy operations}:
\begin{gather*}
(S^2H^\calc_{*}X)_{n}\overset{\widetilde{\nabla}}{\from} \pi_{n}(S^2(Q^\calc B^\calc X))\overset{\psi_\calc^*}{\from} H^\calc_{n+1}X,\\
H^\calc_{n}(X)\overset{(\ExtCohOp^k)^{\star}}{\from} \pi_{n+k}(S^2(Q^\calc B^\calc X))\overset{\psi_\calc^*}{\from} H^\calc_{n+k+1}(X).
\end{gather*}

We will prove instead a somewhat more general result:
\begin{prop}
\label{general CohOpns given irreducibility}
Suppose that $\theta:F^\calc V\to GV$ is a natural transformation from $F^\calc$ to another endofunctor $G$ of $\vect{}{}$ satisfying the condition:
\[\smash{\theta\circ\mu_V=\theta\circ \epsilon_{F^\calc V} +\theta\circ {F^\calc \epsilon_V}:F^\calc F^\calc V\to G V.}\]
Write $\smash{\widetilde{\theta}}:Q^{\calc}X_s\to G(Q^{\calc}X_{s-1})$ for the following composite, (which depends on the almost free structure chosen):
\[\smash{Q^{\calc}X_s\overset{\cong}{\to} V_s\overset{d_0}{\to}F^\calc V_{s-1}\overset{\theta}{\to}GV_{s-1}\overset{\cong}{\to}G(Q^{\calc}X_{s-1})}\]
Then $d_0\circ\widetilde{\theta}=\widetilde{\theta}\circ(d_0+d_1)$, and $d_j\circ\widetilde{\theta}=\widetilde{\theta}\circ d_{j+1}$ for $j\geq1$, so that $\widetilde{\theta}$ restricts to degree $-1$ chain maps $N_sQ^\calc X\to N_{s-1}Q^\calc X$ and $C_sQ^\calc X\to C_{s-1}Q^\calc X$.
%For $v\in V_s$, write $d_0v=g(v_i)$ as an expression in certain $v_i\in V_{s-1}$. Then $\widetilde{\theta}(v):=\theta(g)(v_i)$. Here, we view the $g(v_i)$ as lying in $F^\calc V_{s-1}\cong X_{s-1}$, so that $\theta(g(v_i))\in G V_{s-1}\cong G(Q^\calc X_{s-1})$. Then $\widetilde{\theta}$ is a chain map, even descending to the subspace $N_*Q^\calc X$.
\end{prop}
\begin{proof}
In order to see that $d_j\circ\widetilde{\theta}=\widetilde{\theta}\circ d_{j+1}$ for $j\geq1$, we examine the diagram
\[{\xymatrix@R=4mm{
V_s\ar[r]^-{d_0}
\ar@{..>}[d]^-{d_{j+1}}
&%r1c1
F^\calc V_{s-1}\ar[r]^-{\theta}
\ar[d]^-{d_j}
&GV_{s-1}\ar[r]^-{\cong}
\ar@{..>}[d]^-{d_j}
&%r2c2
GQ^{\calc}X_{s-1}\ar[d]^-{GQ^{\calc}(d_j)}
\\%r1c3
V_{s-1}\ar[r]^-{d_0}
&%r2c1
F^\calc V_{s-2}\ar[r]^-{\theta}
&GV_{s-2}\ar[r]^-{\cong}
&%r2c2
GQ^{\calc}X_{s-2}
}}\]
The dotted vertical arrows are available since $X$ is almost free. That the left square commutes is a simplicial identity, and the center square commutes by naturality of $\theta$. In order to show that $d_0\circ\widetilde{\theta}=\widetilde{\theta}\circ(d_0+d_1)$, we use the following diagram, which commutes except for the leftmost square:
\[\xymatrix@R=4mm{
V_s\ar[r]^-{d_0}
\ar@{..>}[d]_-{d_{1}+\epsilon\circ d_0}
&%r1c1
F^\calc V_{s-1}\ar[r]^-{\theta}
\ar[d]^-{F^\calc (\epsilon\circ d_0)}
&GV_{s-1}\ar[r]^-{\cong}
\ar@{..>}[d]^-{G(\epsilon\circ d_0)}
&%r2c2
GQ^{\calc}X_{s-1}\ar[d]^-{GQ^{\calc}(d_0)}
\\%r1c3
V_{s-1}\ar[r]^-{d_0}
&%r2c1
F^\calc V_{s-2}\ar[r]^-{\theta}
&GV_{s-2}\ar[r]^-{\cong}
&%r2c2
GQ^{\calc}X_{s-2}
}\]
Now the rightmost two squares both commute, so to show that the outer rectangle commutes, it is enough to see that the two composites $V_{s}\to F^\calc V_{s-2}$ are coequalized by $\theta$. Using the simplicial identity $d_0 d_1=d_0d_0$, we are trying to show that $\theta d_0d_0+\theta d_0\epsilon d_0$ and $\theta F^\calc (\epsilon d_0) d_0$ are the same map from $V_s$ to $GV_{s-2}$. Even more, we will show that $\theta d_0+\theta d_0\epsilon$ and $\theta F^\calc (\epsilon d_0)$ are the same map $F^\calc V_{s-1}$ to $GV_{s-2}$.

Starting with an expression $f(v_i)$ in various $v_i\in V_{s-1}$, we calculate $\theta d_0 f(v_i)=\theta (fd_0v_i)$, $\theta d_0\epsilon f(v_i)=\theta (\epsilon (f)(d_0v_i))$ and $\theta F^\calc (\epsilon d_0)(f(v_i))=\theta(f)(\epsilon(d_0v_i))$. That these three terms add to zero was the requirement specified for $\theta$.
\end{proof}



\SubsectionOrSection{Cohomology operations for simplicial commutative algebras}
\label{The example of simplicial commutative F2-algebras}
Goerss \citeBOX[\S5]{MR1089001} defines cohomology operations, natural in $A\in s \algs$:
\begin{gather*}
P^i=\psi^*_{\algs}\circ\ExtCohOp^i:H^n_{\algs}A\to H_{\algs}^{n+i+1}A;\textup{ and}\\
[\,,]=\psi^*_{\algs}\circ\ExtCohProd :H_{\algs}^nA\otimes H_{\algs}^mA\to H_{\algs}^{n+m+1}A.
\end{gather*}
He also defines a natural operation $\beta:H_{\algs}^0A\to H_{\algs}^1A$. Note that as a result of the use of $\psi^*_{\algs}$, these operations have a grading shift.
\begin{prop}[{\citeBOX[\S5]{MR1089001}}]
\label{omnibus on coh of simp algs}These operations have the following properties:
\begin{enumerate}
\item the bracket gives $H^*_{\algs}A$ the structure of an $S(\LieOperad)$-algebra (with grading shift);
\item the operation $\beta$ acts as a restriction defined only in dimension zero, so that for $x,y\in H^0_{\algs}A$ and $z\in H^*_{\algs}A$:
\[\beta(x+y)=\beta(x)+\beta(y)=[x,y],\text{\ \ and \ }[\beta(x),z]=[x,[x,z]];\]
\item the self-bracket operation on $H^*_{\algs}A$ equals the \emph{top $P$-operation}:
\[P^nx=[x,x]\text{\ \ for $x\in H^n_{\algs}A$};\]
\item \label{P unstable vanishing} if $x\in H^n_{\algs}A$, then $P^ix=0$ unless $2\leq i\leq n$;
\item every $P$-operation is linear;
\item there holds the following \emph{Cartan formula}:  for all $x,y\in   H^*_{\algs}A$ and $i\geq0$,
\[[x,P^iy]=0;\]
\item \label{yeah H is a Pmodule}the \emph{$P$-Adem relations} hold: if $i\geq 2j$, then
\[P^iP^jx=\sum_{s=i-j+1}^{i+j-2}\binom{2s-i-1}{ s-j}P^{i+j-s}P^sx.\]
\end{enumerate}
\end{prop}
In this case, \emph{(\ref{yeah H is a Pmodule})} \emph{does} state that $H^*_{\algs}A$ is a left module over $\Palg$, the \emph{Steenrod algebra for commutative algebras over $\Ftwo $} of $\Palg$-algebra. This is the unital associative algebra generated by symbols $P^i$ for $i\geq0$, modulo the two sided ideal generated by $P^0$, $P^1$, and the evident \emph{$P$-Adem relations}.


A sequence $I=(i_\ell,\ldots,i_1)$ of integers $i_j\geq2$ is \emph{$P$-admissible} if $i_{j+1}<2i_j$ for $1\leq j <\ell$. For any sequence $I=(i_\ell,\ldots,i_1)$, write $P^I$ for the monomial $P^{i_\ell}\cdots P^{i_1}$ in $\Palg$. It follows from \cite[Theorem I]{MR1089001} that $\Palg$ has an \emph{admissible basis}, consisting of those $P^I=P^{i_\ell}\cdots P^{i_{1}}$ with $I$ a $P$-admissible sequence.  Again, we will say that \emph{$I$ produces $J$ in $\Palg$}, denoted $\produces{I}{J}{\Palg}$ if, when $P_I$ is written in the $P$-admissible basis of $\Palg$, $P_J$ appears with non-zero coefficient. In this case, $J$ must be $P$-admissible, and $I$ must be $P$-inadmissible unless $J=I$.

We define
\[\minDimP(I):=\max\{(i_1),\,(i_2-i_1-1),\,(i_3-i_2-i_1-2),\,\ldots,\,(i_{\ell}-\cdots-i_1-\ell+1)\},\]
for a rather different purpose than in \S\ref{Homotopy operations for simplicial commutative algebras} and \S\ref{Homotopy operations for simplicial Lie algebras}: although the composite 
\[P^I:\bigl(H_{\algs}^{n}A\overset{P^{i_1}}{\to}H_{\algs}^{n+i_1+1}A\overset{P^{i_2}}{\to}\cdots \overset{P^{i_\ell}}{\to}H_{\algs}^{n+i_1+\cdots +i_\ell+\ell}A\bigr)\]
is always defined, it is \emph{forced to be zero} by \emph{(\ref{P unstable vanishing})} alone except  when $n\geq\minDimP(I)$.

As in \S\ref{Homotopy operations for simplicial commutative algebras} and \S\ref{Homotopy operations for simplicial Lie algebras}, if a non-empty sequence $I$  is \emph{$P$-admissible}, we can identify which term is largest in the maximum defining $\minDimP(I)$ and calculate that $\minDimP(I)=i_1$. More explicitly, in a (non-vanishing) admissible expression $P^{i_\ell}\cdots P^{i_1}x$, for $x\in H^n_{\algs}A$, the only one of these $P$-operations that can be a top operation is $P^{i_{1}}$.
%Unlike in \S\ref{Homotopy operations for simplicial commutative algebras} and \S\ref{Homotopy operations for simplicial Lie algebras}, having a non-empty sequence $I$  be \emph{$P$-admissible} does not allow us identify which term is largest in the maximum defining $\minDimP(I)$. More explicitly, in an admissible expression $P^{i_\ell}\cdots P^{i_1}x$, for $x\in H^n_{\algs}A$, top $P$-operations may be interspersed with non-top $P$-operations.


%\begin{lem}\label{lemOnAdemChangeInMDeltaPlain}
%Suppose that $i,j\geq2$ and $i<2j$, and $(i+1)/2\leq s\leq (i+j)/3$. Then $i+j-s\geq 2s$, $i+j-s\geq2$, $s\geq2$, so that the $\delta$-Adem relation writes $\delta_i\delta_j$ as a sum of $\delta$-admissible composites. Moreover,  $\minDimDelta(i,j)\geq \minDimDelta(i+j-s,s)$.
%\end{lem}
%\begin{proof}
%The only tricky inequality is $\minDimDelta(i,j)\geq \minDimDelta(i+j-s,s)$. Now the right hand side must equal $i+j-2s$, and the left hand side is at least $j$. The result follows, since $2s\geq i+1$.
%\end{proof}

The following result shows that whenever an expression $P^Jx$ is forced to be zero by \emph{(\ref{P unstable vanishing})}  and we reduce $P^Jx$ to a sum of $P$-admissible composites, all of the summands are forced to be zero by \emph{(\ref{P unstable vanishing})}.
\begin{lem}
\label{lemOnAdemChangeInMP}
If $\produces{I}{J}{\Palg}$, then $\minDimP(J) \geq \minDimP(I)$, with strict inequality when $I$ and $J$ are distinct and of length two.
\end{lem}
The main theorem of Goerss' memoir is that these operations generate all of the operations in the category $\HA{\algs}$, and that all  the relations between them  are implied by those presented here. In \cite[Chapter V]{MR1089001}, Goerss shows that the listed operations completely capture the cohomology of an object $\mathbb{K}_n^{\algs}$. He proves a Hilton-Milnor theorem \cite{GoerssHiltonMilnor.pdf}, which he uses in \citeBOX[\S11]{MR1089001} to bootstrap up to a calculation of the cohomology of any GEM in $s\algs$, namely \cite[Theorem I]{MR1089001}. The result states that whenever $V\in \vect{1}{}$,  is a  vector space of finite type, not only is 
$F^{\algs}V$
generated by $V$ under the operations of Proposition \ref{omnibus on coh of simp algs}, it is as large as is conceivable given the relations presented. We will present, in Proposition \ref{partialgoerss}, a partial version of his result.

It is  interesting to observe that $\Palg$, the Steenrod algebra for commutative algebras, is
in fact \emph{Koszul dual} (c.f.\ \cite{PriddyKoszul.pdf}) to $\deltaalg$, the algebra which possesses an unstable partial left action on the homotopy of a simplicial algebra. Indeed, Goerss \emph{calculates} $\Palg$ as the Koszul dual of $\deltaalg$, using a reverse Adams spectral sequence due to Miller \cite{MillerSullivanConjecture.pdf} (c.f.\ \S\ref{reverse Adams spectral sequence}). We will explore this duality further when we consider the Bousfield-Kan spectral sequence.

\SubsectionOrSection{The categories $\calw(0)$ and $\calU(0)$}
Suppose that $A\in s\algs$ is \emph{connected}: $\pi_0A=0$. Then $H^0_{\algs}=Q^{\algs}\pi_0A=0$. It is advantageous to work with the cohomology of connected objects, as we do not need to worry about the operation $\beta$. If we say that $V\in \vect{1}{}$ is \emph{connected} when $V^{0}=0$, we may identify the full subcategory of $\vect{1}{}$ on the connected objects with $\vect{+}{}$.

Goerss' result proves that the monad
$F^{\HA{\algs}}$ on $\vect{1}{}$
preserves $\vect{+}{}$ (and indeed it is a general fact that no non-trivial natural cohomology operations decrease dimension).
We will write $\calw(0)$ for the category of connected $\algs$-$H^*$-algebras, so that the monad $F^{\calw(0)}$  is simply the restriction of $F^{\HA{\algs}}$ to $\vect{+}{}$. The way that we will report Goerss' result here is to explain how the monad $F^{\calw(0)}$ may be constructed on objects of $\vect{+}{}$ of finite type.

Let the \emph{category of unstable $\Palg$-modules}, denoted $\calU(0)$, be the category whose objects are $\vect{+}{0}$-graded $\Palg$-modules in which $P^i$ acts with grading $i+1$, and such that $P^i:V^n\to V^{n+i+1}$ is zero unless $i\leq n$. Recall that  we have \emph{already} imposed the $P$-Adem relations and set $P^0$ and $P^1$ to be zero in $\Palg$. 
\begin{prop}
The monad $F^{\calU(0)}$ may be defined by
\begin{alignat*}{2}
F^{\calU(0)}V
:={}&
(\Palg\otimes V)\,/\,\Ftwo \{P^I\otimes v\ |\ V\in V^n,\ \minDimP(I)>n \}%
\\
={}&
(\Palg\otimes V)\,/\,\Ftwo \{P^I\otimes v\ |\ V\in V^n,\ \minDimP(I)>n ,\ I\textup{ is $P$-admissible}\}.%
\end{alignat*}
\end{prop}
\begin{proof}
This follows from Goerss' \cite[Theorem I]{MR1089001} and Lemma \ref{lemOnAdemChangeInMP}.
\end{proof}
Now an object of $\calw(0)$ is in particular an object of $\calU(0)$. It is also a (degree shifted) $S(\LieOperad)$-algebra. Thus, there  is a natural map 
\[F^{\calU(0)}F^{S(\LieOperad)}V\to F^{\calw(0)}V.\]
This map is not an isomorphism, but it follows from \cite[Theorem I]{MR1089001} that it is surjective. Moreover, our final reading of Goerss' result is:
\begin{prop}
\label{partialgoerss}
For $V\in\vect{+}{}$ of finite type,  $F^{\calw(0)}V\in \vect{+}{}$ is the coequalizer:
\[\textup{coeq}\left(\xymatrix@R=4mm{
\Palg\otimes F^{S(\LieOperad)}V
\ar@<+.5ex>[r]^-{\smash{\textup{sb}_1}}
\ar@<-.5ex>[r]_-{\textup{sb}_2}
&%r1c1
\Palg\otimes F^{S(\LieOperad)}V
\ar@{->>}[r]
&
F^{\calu(0)}F^{S(\LieOperad)}V%r1c2V
%r1c3
}\right),\]
where the maps $\textup{sb}_1$ and $\textup{sb}_2$ are defined on  $\Palg\otimes(F^{S(\LieOperad)}V)^{m}$ by
\[\textup{sb}_1(P^I\otimes x)=P^I\otimes [x,x]\textup{ \ and \ }\textup{sb}_2(P^I\otimes x)=P^IP^m\otimes x.\]
Choose a homogeneous basis of $V$, construct from it a monomial  basis of $\Lambda(\LieOperad)V$ (such as any choice of Hall basis), and then lift these monomials in the evident way to a collection $B$ of elements of $S(\LieOperad)V$. Then a basis of $F^{\calw(0)}V$ is 
\[\left\{P^Ib\ \middle|\ b\in B,\ \minDimP(I)\leq|b|,\ \textup{$I$ is $P$-admissible}\right\}.\]
\end{prop}
\begin{cor}
\label{finite type pres by FW0}
Suppose that $V\in\vect{+}{}$ is of finite type. Then so is $F^{\calw(0)}V$.
\end{cor}

The following observations will be useful for the calculation of the cohomology of objects of $\calw(0)$.
\begin{lem}
\label{quad gradings have a chance on U0 W0}
The monads $F^{\calU(0)}$ and $F^{\calw(0)}$ on $\vect{+}{}$ may be promoted to monads on the category $\quadgrad{}\vect{+}{}$, by insisting that quadratic gradings add under brackets and double under $P$-operations.
\end{lem}
\noindent It is typical to think of $V\in\vect{+}{}$ as an object of $\quadgrad{}\vect{+}{}$ concentrated in quadratic grading one when considering $F^{\calw(0)}V$.


%An object of $\calw(0)$ is simultaneously an $S(\LieOperad)$-algebra and an unstable left $\Palg$-module.
%
%
%\begin{prop}[{\citeBOX[\S5]{MR1089001}}]\label{omnibus on coh of simp algs}These operations have the following properties:
%\begin{enumerate}
%\item the bracket gives $H^*_{\algs}A$ the structure of an $S(\LieOperad)$-algebra (with grading shift);
%\item the operation $\beta$ acts as a restriction defined only in dimension zero, so that for $x,y\in H^0_{\algs}A$ and $z\in H^*_{\algs}A$:
%\[\beta(x+y)=\beta(x)+\beta(y)=[x,y],\text{\ \ and \ }[\beta(x),z]=[x,[x,z]];\]
%\item the self-square operation on $H^*_{\algs}A$ equals the top $P$-operation:\[P^nx=[x,x]\text{\ \ for $x\in H^n_{\algs}A$};\]
%\item \label{P unstable vanishing} if $x\in H^n_{\algs}A$, then $P^ix=0$ unless $2\leq i\leq n$;
%\item every $P$-operation is linear;
%\item there holds the following \emph{Cartan formula}:  for all $x,y\in   H^n_{\algs}A$ and $i\geq0$,
%\[[x,P^iy]=0\]
%\item \label{yeah H is a Pmodule}the \emph{$P$-Adem relations} hold: if $i\geq 2j$, then
%\[P^iP^jx=\sum_{s=i-j+1}^{i+j-2}{2s-i-1\choose s-j}P^{i+j-s}P^sx.\]
%\end{enumerate}
%\end{prop}
%
%
%
%
%
%Now let $\calw(0)$ be the category of connected objects in Goerss' category `$\calw$' \cite[Definition G]{MR1089001}. Explicitly, $\calw(0)$ has objects $\vect{+}{0}$-graded vector spaces $X$, equipped with:
%\begin{enumerate}
%\item a commutative, bilinear product $[\,,]:X^n\otimes X^m\to X^{n+m+1}$ satisfying the Jacobi identity; and
%\item linear maps $P^i:X^n\to X^{n+i+1}$ such that
%\begin{enumerate}
%\item $P^i=0$ unless $2\leq i\leq n$, and $P^nx=[x,x]$;
%\item $[x,P^iy]=0$ for all $x$, $y$ and $i$; and
%\item satisfying the $P$-Adem relation:
%\[P^iP^j=\sum_{s=i-j+1}^{i+j-2}{2s-i-1\choose s-j}P^{i+j-s}P^s\textup{ whenever $i\geq 2j$.}\]
%\end{enumerate}
%\end{enumerate}
%Then Goerss's results \cite[Theorem H]{MR1089001}, restricted to the full subcategory of connected objects in $s\algs$ (i.e.\ objects $A$ with $\pi_0A=0$, or equivalently $H^0_{\algs}A=0$), state that the operations $[\,,]$ and $P^i$ give $H^*A$ the structure of an element of $\calw(0)$, and moreover, that these are the only natural operations on the cohomology of connected objects of $s\algs$ \cite[Theorem I]{MR1089001}. That is, Goerss shows that the above relations are satisfied by the operations defined, and then calculates the cohomology of an abelian object in $s\algs$; neither of these two steps is straightforward.
%
%\textbf{used:}[Let $\calU(0)$ be the category whose objects are $\vect{+}{0}$-graded vector spaces $V$ equipped with linear left $P$-operations $P^i:V^n\to V^{n+i+1}$, which are zero unless  $2\leq i\leq n$, and satisfy the $P$-Adem relation as above.]
%
%Any object $X$ of $\calw(0)$ is in particular an object of $\calU(0)$. It is \emph{not} true, however, that $X$ is a Lie algebra. Indeed, the relations above do not imply that $[x,x]=0$ for any $x\in X$ (see \citeBOX[p.~18]{MR1089001}). Rather, $X$ is an algebra over the monad $S(\LieOperad)$ discussed in \S\ref{sec on Lie algs and homotopy ops}. We will describe the construction of the monads $F^{\calw(0)}$ and $F^{\calU(0)}$ in \S\ref{Construction of the monads}.
%\begin{shaded}\tiny
%Denote by $\Palg$ the algebra arising in Goerss' work, defined as a quotient of a free unital associative algebra:
%\[\Palg=F^{\textup{ass}}(P_0,P_1,P_2,\ldots)/\left(P_0=P_1=0,\ \textup{$P$-Adem relation}\right)\]
%This associative algebra can be $\vect{+}{}$-graded, by thinking of the generator $P^i$ as having grading $i+1$ (\textbf{or} even in $\vect{1}{1}$, taking $P^i\in(\Palg)^{i+1}_1$). For any $x\in\Palg$, there is a homomorphism, \emph{right multiplication} by $x$:
%\[\textup{m}_x:\Palg\to \Palg,\quad y\mapsto yx.\]
%Moreover, $\Palg$ has an \emph{admissible basis}, consisting of those monomials
%\[P^I:=P^{i_\ell}P^{i_{\ell-1}}\cdots P^{i_1}\]
%for which $I=(i_\ell,\ldots,i_1)$ is a sequence of integers such that each $i_j\geq2$, and  each comparison $i_{j+1}< 2i_j$ holds. Moreover, $\Palg$ has a decreasing filtration $\Palg\supset F^1\Palg\supset F^2\Palg\supset\cdots $, where
%\[F^n\Palg=\Ftwo \left\{P^I\,:\,\textup{$I$ non-empty and admissible, }i_j+\cdots +i_1\geq n+j\textup{ for some $j\geq 1$}\right\}.\]
%By examining the $P$-Adem relation, one can determine that this is a filtration by left ideals. For a vector space $V\in\vect{+}{0}$, we have
%\[F^{\calU(0)}V=\Palg\otimes V/\bigoplus_{i\geq1}(F^n\Palg\otimes V^n)\]
%Next, writing $S(\LieOperad)$ for the monad on $\vect{+}{0}$ with cohomological grading shifted, we may construct the monad $F^{\calw(0)}$ (the restriction of the monad $U$ in \citeBOX[p.~18]{MR1089001}), by coequalizing two maps $\Palg\otimes S(\LieOperad)V\to F^{\calU(0)} (S(\LieOperad)V)$. The first map is the composite:
%\[\xymatrix@R=4mm{
%\Palg\otimes S(\LieOperad)V\ar[r]^-{\Id\otimes \textup{frob}}
%&%r1c1
%\Palg\otimes S(\LieOperad)V\ar@{->>}[r]&%r1c2
%F^{\calU(0)}(S(\LieOperad)V).%r1c3
%}\]
%The second map restricts to each subset $\Palg\otimes (S(\LieOperad)V)^r$ to the composite:
%\[\xymatrix@R=4mm{
%\Palg\otimes (S(\LieOperad)V)^r\ar[r]^-{\textup{m}_{P^r}\otimes\Id}
%&
%\Palg\otimes (S(\LieOperad)V)^r\ar@{^{(}->}[r]
%&%r1c1
%\Palg\otimes S(\LieOperad)V\ar@{->>}[r]&%r1c2
%F^{\calU(0)}(S(\LieOperad)V).%r1c3
%}\]
%One can check, using \citeBOX[p.~18]{MR1089001} the following is a coequalizer in $\vect{+}{0}$:
%\[\xymatrix@R=4mm{
%\Palg\otimes S(\LieOperad)V\ar@<.5ex>[r]\ar@<-.5ex>[r]
%&%r1c1
%F^{\calU(0)}(S(\LieOperad)V)\ar@{->>}[r]
%&%r1c2
%F^{\calw(0)}V%r1c3
%}\]
%The monad $F^{\calw(0)}$ possesses a quadratic grading, as do the $\calw(n)$ for $n\geq1$. One views $V\subseteq F^{\calw(0)}V$ as lying in quadratic grading 1, and demands that quadratic grading adds under taking brackets and doubles when applying $P$-operations.
%\end{shaded}
An object of $\calw(0)$ is in particular an object of $\calU(0)$, and (as all of the $P$-operations are linear), we can define a functor $Q^{\calU(0)}:\calw(0)\to\vect{+}{0}$ which takes the quotient by the image of the $P$-operations. Moreover:
\begin{lem}
\label{Kill P ops gives lie alg}
For $X\in\calw(0)$, the vector space $Q^{\calU(0)}X\in\vect{+}{}$ inherits a (grading shifted) Lie algebra structure from the bracket of $X$, yielding a factorization:% of indecomposable functors, $Q^{\calw(n)}=Q^{\calL(n)}\circ Q^{\calU(n)}$
\[Q^{\calw(0)}=Q^{\Lambda(\LieOperad)}\circ Q^{\calU(0)}:\left(\calw(0)\to \Lambda(\LieOperad)\to \vect{+}{0}\right).\]
Moreover the composite $Q^{\calU(0)}\circ F^{\calw(0)}$ equals the free construction $F^{\calL(0)}$.
\end{lem}
\begin{proof}
One checks that the bracket is well defined in the quotient, and that taking the quotient by the top $P$-operation imposes the relation $[x,x]=0$, to create a $\Lambda(\LieOperad)$-algebra from the pre-existing $S(\LieOperad)$-algebra structure. The final claim follows from Proposition \ref{partialgoerss}.
\end{proof}
\SubsectionOrSection{Cohomology operations for simplicial (restricted) Lie algebras}
\label{section: Cohomology operations for simplicial (restricted) Lie algebras}
A standard definition for the cohomology of a simplicial Lie algebra $L\in s\liealgs$ or $s\restliealgs$ is presented in \cite{PriddySimplicialLie.pdf} as follows. Let $UL$ be the simplicial primitively Hopf algebra obtained by applying the universal enveloping algebra functor levelwise, or the \emph{restricted} universal enveloping algebra functor when working in $s\restliealgs$. %\textbf{[Priddy works with restricted Lie algebras. Cutting out the restriction cuts out $Sq^1$.]} 
Applying the Eilenberg-Mac Lane suspension functor (\citeBOX[\S2.3]{PriddySimplicialLie.pdf}, \citeBOX[\S5]{MillerSullivanConjecture.pdf} or \citeBOX[p.~87]{MaySimpObj.pdf}), one defines (using a subscript $\bar{W}$ to avoid confusion):
\[H_{\bar{W}}^*L:=\begin{cases}
\pi^*\dual\bar{W}UL,&\textup{if }*>0;\\
0,&\textup{if }*=0.
%\\,&\textup{if }
\end{cases}\]
We discuss universal enveloping algebra functors in Appendix \ref{The partially restricted universal enveloping algebra}.
The suspension $\bar{W}$ destroys the associative algebra structure but leaves a simplicial cocommutative coalgebra structure on $\bar{W}UL$. Homotopy operations for simplicial cocommutative coalgebras are well known, being the mode of definition of the cup product and Steenrod operations on the mod two cohomology of a simplicial set. These operations can be constructed using  propositions \ref{external singly simplicial steenrods} and \ref{the external cohomotopy pairing}:
\begin{gather*}
\Sq^k:=\Delta^*\circ\ExtCohOp^k:\bigl(\pi^n\dual (\bar{W}UL)\overset{\ExtCohOp^k}{\to}\pi^{n+k}\dual S^2(\bar{W}UL)\overset{\Delta^*}{\to}\pi^{n+k}\dual (\bar{W}UL)\bigr);\\
\mu:=\Delta^*\circ\ExtCohProd:\bigl(S_2(\pi^*\dual(\bar{W}UL))^{n}\overset{\ExtCohProd}{\to} \pi^nS_2\dual(\bar{W}UL)\to \pi^n\dual S^2(\bar{W}UL)\overset{\Delta^*}{\to}\pi^n\dual (\bar{W}UL)\bigr).
\end{gather*}
The operations here make $H^*_{\bar{W}}L$ a module over the \emph{homogeneous Steenrod algebra}, which is the usual mod 2 Steenrod algebra `with $\Sq^0$ set to zero'. That is, the \emph{homogeneous Steenrod algebra} is the unital associative algebra $\LieSteen$ generated by symbols $\Sq^j$ for $j\geq1$, subject to the \emph{homogeneous $\Sq$-Adem relation}:
\[\Sq^i\Sq^j=\sum_{k=1}^{\lfloor i/2\rfloor}\binom{j-k-1}{ i-2k}\Sq^{i+j-k}\Sq^k\textup{\ for $i<2j$.}\]
We will only ever work with the \emph{homogeneous} Steenrod algebra and the \emph{homogeneous} $\Sq$-Adem relation, and so may omit the word \emph{homogeneous} if we desire.

This algebra is Koszul dual to the \emph{opposite} of the $\Lambda$-algebra (c.f.\ \ref{Homotopy operations for simplicial Lie algebras}). There is an index shift in this duality, so that $\Sq^i$ corresponds to $\lambda_{i-1}$ for $i\geq1$ \citeBOX[\S7.1]{PriddyKoszul.pdf}.

In \cite{PriddySimplicialLie.pdf}, Priddy concentrates on simplicial \emph{restricted} Lie algebras $L$, and works out all of the natural operations on $H^*_{\bar{W}}$ and the relations between them. Moreover, he gives a spectral sequence argument showing that the two notions of cohomology are isomorphic, \emph{with a shift in degree} arising from the use of $\bar{W}$: $H_{\bar{W}}^nL\cong H^{n-1}_{\restliealgs}L$ for $n\geq1$.

For our purposes it is better to work in the framework set out in this thesis, giving an alternative definition of the operations needed. This alternative definition will fit more readily into the spectral sequence arguments we intend to make. 

For now, let $\calc$ stand either for $\restliealgs$ or $\liealgs$. Our definition of the cohomology operations is:
\begin{gather*}
\Sq^k:=\psi^*_{\calc}\circ\ExtCohOp^{k-1}:H_\calc^{n}L\overset{\ExtCohOp^{k-1}}{\to} \pi^{n+k-1}\dual(S^2(Q^\calc B^\calc L))\overset{\psi_\calc^*}{\to} H_\calc^{n+k}L,\\
\mu:=\psi^*_{\calc}\circ\ExtCohProd:(S_2 H_\calc^{*}L)^{n}\overset{\ExtCohProd}{\to} \pi^{n}\dual(S^2(Q^\calc B^\calc L))\overset{\psi_\calc^*}{\to} H_\calc^{n+1}L.
\end{gather*}
We will check the properties of these operations using a spectral sequence argument similar to Priddy's, although we will need to a richer construction of the spectral sequence in order to extract information about the operations. This work will be deferred until Appendix \ref{appendix on Lie coh ops}, and will prove:
\begin{prop}
\label{all the lie steenrod ops are the same}
%Under the isomorphisms $H^*_\calc L\cong H^{*+1}_{\bar{W}}L$ for $n\geq0$, The operations
There are commuting diagrams:
\[\xymatrix@R=4mm@C=17mm{
H^n_\calc L
\ar[d]^-{\cong}
\ar[r]^-{\psi^*_{\calc}\circ\ExtCohOp^{k-1}}&%r1c1
H^{n+k}_\calc L
\ar[d]^-{\cong}
&
H^{n_1}_\calc L\otimes H^{n_2}_\calc L
\ar[d]^-{\cong}
\ar[r]^-{\psi^*_{\calc}\circ\ExtCohProd}&%r1c1
H^{n_1+n_2+1}_\calc L
\ar[d]^-{\cong}
\\%r1c2
H^{n+1}_{\bar{W}} L
\ar[r]^-{\Delta^*\circ\ExtCohOp^k}&%r2c1
H^{n+k+1}_{\bar{W}} L%r2c2
&
H^{n_1+1}_{\bar{W}} L\otimes H^{n_2+1}_{\bar{W}} L
\ar[r]^-{\Delta^*\circ\ExtCohProd}&%r2c1
H^{n_1+n_2+2}_{\bar{W}} L%r2c2
}\]
That is, the two definitions of $\Sq^k$ coincide, as do the two definitions of $\mu$.
\end{prop}

Given the suspension, one expects the notion of \emph{top Steenrod operation} to be different to that in other settings. We say that $\Sq^{n+1}:H^n_\calc L\to H^{2n+1}_\calc L$ is the top operation.
\begin{prop}[{\citeBOX[\S5.3]{PriddySimplicialLie.pdf}}]
\label{omnibus on coh of simp lie algs}These operations have the following properties:
\begin{enumerate}
\item the product $\mu$ gives $H^*_{\calc}L$ the structure of a commutative algebra (with grading shift);
\item the squaring operation on $H^*_{\calc}L$ equals the \emph{top Steenrod operation}:
\[\Sq^{n+1}x=x^{2}\text{\ \ for $x\in H^n_{\calc}L$};\]
\item \label{Sq unstable vanishing} if $x\in H^n_{\calc}L$, then $\Sq^ix=0$ unless $1\leq i\leq n+1$;
\item every Steenrod operation is linear;
\item the \emph{Cartan formula} holds:  for all $x,y\in   H^*_{\calc}L$ and $i\geq0$,
\[\Sqh^i(xy)=\textstyle\sum_{k=1}^{i-1}(\Sq^kx)(\Sq^{i-k}y);\]
\item \label{yeah H is a Stmodule} the  homogeneous $\Sq$-Adem relations hold, making $H^{*}_\calc L$ a left $\LieSteen$-module.
\end{enumerate}
\end{prop}
This fact follows from Proposition \ref{the point of the appendix}. We will also use the following calculation:
\begin{prop}
\label{prop on sq1 on lie not rest}
If $\calc=\liealgs$ (as opposed to $\restliealgs$), then $\Sq^1=0$. In particular, for $x\in H^0_{\liealgs}X$, $x^2=0$.
\end{prop}
\begin{proof}
It is enough to prove this for the universal example $\imath_n\in H^*_{\liealgs}\mathbb{K}_n^{\liealgs}$. The reverse Adams spectral sequence (\S\ref{reverse Adams spectral sequence}) is of the form
\[E_2^{p,q}=(H^{p}_{\PA{\liealgs}}\mathbb{K}_{0,n}^{\PA{\liealgs}})^{q}\implies H^{p+q}_{\liealgs}\mathbb{K}_n^{\liealgs}.\]
The point now is that $\mathbb{K}_{0,n}^{\PA{\liealgs}}$, which is just a constant object in $s\PA{\liealgs}$ with value a one-dimensional Lie algebra in internal dimension $n$, is actually free as an object of $\PA{\liealgs}$ below internal dimension $n+1$. This is simply because there is no $\lambda_0$ operation defined in $\PA{\liealgs}$. One may thus construct a pas-\`a-pas resolution of $\mathbb{K}_{0,n}^{\PA{\liealgs}}$ (a process described in \cite{Andre-StepByStep}) which in positive simplicial dimension is concentrated in internal dimension at least $n+1$, implying that $E_2^{p,q}=0$ when $p\geq1$ and $q\leq n$. Moreover, $E_2^{0,q}=0$ unless $q=n$, showing that $H^{n+1}_{\liealgs}\mathbb{K}_n^{\liealgs}=0$. This group contains $\Sq^1\imath_n$.
\end{proof}
A sequence $I=(i_\ell,\ldots,i_1)$ of integers $i_j\geq1$ is \emph{$\Sq$-admissible} if $i_{j+1}\geq 2i_j$ for $1\leq j <\ell$. For any sequence $I=(i_\ell,\ldots,i_1)$, write $\Sq^I$ for the monomial $\Sq^{i_\ell}\cdots \Sq^{i_1}$. The homogeneous Steenrod algebra has the expected admissible basis, and we say that \emph{$I$ produces $J$ in $\LieSteen$}, denoted $\produces{I}{J}{\Sq}$ if $\Sq^J$ appears in the $\Sq$-admissible expansion of $\Sq^I$.

We use the function $\minDimDelta$ defined in \S\ref{Homotopy operations for simplicial commutative algebras}, this time  noting that the composite 
\[\Sq^I:\bigl(H_{\algs}^{n}A\overset{\Sq^{i_1}}{\to}H_{\algs}^{n+i_1}A\overset{\Sq^{i_2}}{\to}\cdots \overset{\Sq^{i_\ell}}{\to}H_{\algs}^{n+i_1+\cdots +i_\ell}A\bigr)\]
is \emph{forced to be zero} by \emph{(\ref{Sq unstable vanishing})} alone except  when $n\geq\minDimDelta(I)-1$.


If a non-empty sequence $I$  is \emph{$\Sq$-admissible}, we have 
\[\minDimDelta(I)=e(I)=i_\ell-i_{\ell-1}-\cdots -i_1,\]
the Serre excess of $I$. We now have enough notation available to describe the category $\HA{\restliealgs}$, using Priddy's calculations. The results are similar to those in \S\ref{Homotopy operations for simplicial commutative algebras} on the category $\PA{\algs}$.
Firstly, there is again  a \emph{K\"unneth theorem}:
\begin{prop}
\label{Prop on cohomology of product of finite lie gems}
Suppose that $K_1$ and $K_2$ are finite GEMs in $s\restliealgs$. Then $H^*_\restliealgs K_1$ and $H^*_\restliealgs K_2$ in $\HA{\restliealgs}$ are of finite type, and their coproduct
$H^*_\restliealgs(K_1\times K_2)$ in $\HA{\restliealgs}$ may be calculated as the non-unital (grading shifted) commutative algebra coproduct of $H^*_\restliealgs K_1$ and $H^*_\restliealgs K_2$.
\end{prop}
\begin{proof}
We rely on Proposition \ref{all the lie steenrod ops are the same} and the following calculation:
\begin{alignat*}{2}
H^*_{\bar{W}}(K_1\times K_2)
&:=
\pi^*\dual \bar{W}U(K_1\times K_2)%
\\
&\phantom{:}\cong
\pi^*\dual \bar{W}(UK_1\otimes UK_2)%
\\
&\phantom{:}\cong
\pi^*\dual (\bar{W}UK_1\otimes \bar{W}UK_2)%
\\
&\phantom{:}\cong
\dual (\pi_*\bar{W}UK_1\otimes \pi_*\bar{W}UK_2)%
\\
&\phantom{:}\supseteq
% Right hand side
H^*_{\bar{W}}K_1\otimes H^*_{\bar{W}}K_2.
\end{alignat*}
This containment in in fact an equality when $H^*_{\bar{W}}K_1 $ and $H^*_{\bar{W}}K_2$ are both of finite type, in which case $H^*_{\bar{W}}(K_1\times K_2)$ is also of finite type, and the isomorphism is proved. Thus, by induction on the total number of factors $\mathbb{K}_n^{\restliealgs}$ of $K_1$ and $K_2$, we only need to check that $H^*_{\bar{W}}\mathbb{K}_n^{\restliealgs}$ is of finite type for any $n\geq0$. This is implied by Priddy's calculation, Proposition \ref{calc of restliecoh on single EMobject}.
\end{proof}
\noindent After giving the calculation on a single Eilenberg-Mac Lane object, the cohomology of finite GEMs, and thus the category $\HA{\restliealgs}$ is determined by Proposition \ref{Prop on cohomology of product of finite lie gems} and the Cartan formula. The structure defining $\PA{\restliealgs}$ is then well understood in light of:
\begin{prop}[{\citeBOX[6.1]{PriddySimplicialLie.pdf}}]
\label{calc of restliecoh on single EMobject}
For $n\geq0$, let $\imath$ be the fundamental class in $H^n_{\restliealgs}(\mathbb{K}_n^{\restliealgs})$. Then, as non-unital (degree shifted) commutative algebras:
\[H^n_{\restliealgs}(\mathbb{K}_n^{\restliealgs})\cong S(\CommOperad)\bigl[\Sq^I\imath\,\bigm|\,\textup{$I$ is $\Sq$-admissible, $e(I)\leq n$}\bigr].\]
\end{prop}

\begin{cor}
\label{finite type pres by F lierest halg}
Suppose that $V\in\vect{+}{}$ is of finite type. Then so is $F^{\HA{\restliealgs}}V$. That is, the restriction of the monad $F^{\HA{\restliealgs}}V$ on $\vect{1}{}$ to $\vect{+}{}$ preserves objects of finite type.
\end{cor}

The case of simplicial Lie algebras mimics that of simplicial commutative algebras: for Lie algebras, the homogeneous Steenrod algebra acts on cohomology, and is Koszul dual to the opposite of the $\Lambda$-algebra, which possesses an unstable partial left action on  homotopy.
Further material on the cohomology of Lie algebras is deferred to Appendix \ref{appendix on Lie coh ops}.
%\TODOCOMMENT{$\liealgs$?}

\end{Constructing cohomology operations}






\begin{homotopy operations for PRLs}

\SectionOrChapter{Homotopy operations for partially restricted Lie algebras}\label{sec on Lie algs and homotopy ops}
\label{homotopy operations for PRLs}
\SubsectionOrSection{The categories $\calL(n)$ of partially restricted Lie algebras}\label{The categories Ln}
%[\textbf{Old version} of spiel on PRLie algs commented out here.%
%Let $\LieOperad$ be the Lie operad in characteristic 2. As explained in [Fresse], the monad $S(\LieOperad)$ on $\vect{}{}$ defined by
%\[S(\LieOperad)(V)=\bigoplus_{n\geq1}(\LieOperad(n)\otimes V^{\otimes n})_{\Sigma_n}\]
%does not return the free Lie algebra on $V$ in the traditional sense. Rather, an $S(\LieOperad)$-algebra is a vector space $L$ equipped with a bracket $L\otimes L\to L$ satisfying the Jacobi identity and the antisymmetry condition $[x,y]=-[y,x]$. This condition does \emph{not} imply the alternating condition $[x,x]=0$ (which is impossible to encode operadically). We will refer to structures of this type as $S(\scrL)$-algebras, to distinguish them from Lie algebras, which we demand satisfy the alternating condition.
%
%Fresse constructs a monad $\Gamma(\LieOperad)$ defined by
%\[\Gamma(\LieOperad)(V)=\bigoplus_{n\geq1}(\LieOperad(n)\otimes V^{\otimes n})^{\Sigma_n},\]
%which represents the free restricted [See Curtis or 6A or something] Lie algebra. He further constructs a norm map $\textup{Tr}:S(\LieOperad)\to \Gamma(\LieOperad)$, and defines a monad $\Lambda(\LieOperad)$ by the formula
%\[\Lambda(\LieOperad)V=\im (\textup{Tr}:S(\LieOperad)V\to \Gamma(\LieOperad)V).\]
%Then $\Lambda(\LieOperad)(V)$ is the free Lie algebra on $V$, subject to the standard axiom, $[x,x]=0$.
%
%We will be interested in \emph{partially restricted Lie algebras}. That is, Lie algebras $L$ equipped with a decomposition $L=L_0\oplus L_1$ such that $L_1$ is a Lie ideal and a map $\restn{(\DASH)}:L_1\to L_1$ satisfying the axioms one would expect for a restriction map. Namely, for any $a,b\in L_1$ and $c\in L$, $\restn{(a+b)}=\restn{a}+\restn{b}+[a,b]$, and $[\restn{a},c]=[a,[a,c]]$. There is a free functor from $\vect{}{}\times \vect{}{}$ to partially restricted Lie algebras, sending $(V_0,V_1)$ to the algebra generated by restrictable elements $V_1$ and non-restrictable elements $V_0$. This free functor may be described as follows. There is a natural $\Sigma_n$-equivariant decomposition: %The expressions for $S(\LieOperad)(V_0\oplus V_1)$ and $\Gamma(\LieOperad)(V_0\oplus V_1)$ both contain the terms:
%\[\LieOperad(n)\otimes (V_0\oplus V_1)^{\otimes n}=\Bigl(\LieOperad(n)\otimes (V_0)^{\otimes n}\Bigr)\oplus\Bigl(\bigoplus(\LieOperad(n)\otimes V_{i_1}\otimes V_{i_2}\otimes\cdots\otimes V_{i_n})\Bigr)\]
%where the direct sum is taken over all sequences $(i_1,\ldots,i_n)\in\{0,1\}^n$ other than $(0,\ldots,0)$.
%%and this decomposition is of $\Sigma_n$-modules.
%The free partially restricted Lie algebra is the subspace of $\Gamma(\LieOperad(V_0\oplus V_1))$ given by
%\[\im \Bigl(S(\LieOperad)(V_0)\overset{\textup{Tr}}{\to}\Gamma(\LieOperad)(V_0\oplus V_1)\Bigr)\oplus\bigoplus_{n\geq1} \Bigl(\bigoplus(\LieOperad(n)\otimes V_{i_1}\otimes V_{i_2}\otimes\cdots\otimes V_{i_n})\Bigr)^{\Sigma_n}\]
%END commented out old material
For each $n\geq0$, we will be interested in certain categories of Lie algebras monadic over $\vect{+}{n}$, with a grading shift. Broadly, a $\vect{+}{n}$-graded Lie algebra is an element $L\in\vect{+}{n}$ with a structure map $\Lambda^2 L\to L$ which shifts gradings as follows
\[L^{t}_{s_n,\ldots,s_1}\otimes L^{t'}_{s'_n,\ldots,s'_1}\to L^{t+t'+1}_{s_n+s'_n,\ldots,s_1+s'_1}.\]
If we wished to be precise we could  view the Lie operad as an operad in $\vect{+}{n}$, such that
\[\LieOperad(r)=(\LieOperad(r))^{r-1}_{0,\ldots,0},\]
and then a $\vect{+}{n}$-graded Lie algebra would be an algebra over the corresponding monad $\Lambda(\LieOperad)$ on $\vect{+}{n}$. In our context, the Lie operad arises as the Koszul dual of the commutative operad, through the constructions in \citeBOX[\S5]{MR1089001}, and the use of the operadic bar construction (c.f.\ \citeBOX[\S3]{FresseKoszulDuality.pdf}) explains the shift. See \citeBOX[\S5.3.4]{FresseKoszulDuality.pdf} for a discussion of Koszul duality of operads in positive characteristic. From this point forward we will simply think of an object of $\calL(n)$ as a vector space $L\in\vect{+}{n}$ with a map $\Lambda^2L\to L$ shifting degrees as described.

A $\vect{+}{n}$-graded \emph{partially restricted} Lie algebra is to be a graded Lie algebra such that only certain graded parts admit a restriction operation. Specifically, there is to be defined a restriction operation
\[\restn{(\DASH)}:L^t_{s_n,\ldots,s_1}\to L^{2t+1}_{2s_n,\ldots,2s_1}\]
whenever not all of $s_n,\ldots,s_{1}$ are zero. We will denote the category of such objects $\calL(n)$. It is monadic over $\vect{+}{n}$, with an adjunction
\[F^{\calL(n)}:\vect{+}{n}\rightleftarrows \calL(n):U^{\calL(n)}.\]
The monad of this adjunction may be constructed as an appropriately chosen submonad of $\Gamma(\LieOperad):\vect{+}{n}\to \vect{+}{n}$ (with $\LieOperad$ shifted as above), containing $\Lambda(\LieOperad)$. As such, the free construction $F^{\calL(n)}V$ admits a quadratic grading, which we denote $\quadgrad{k}F^{\calL(n)}V$, as in \S\ref{introtoLiealgssection}.

\SubsectionOrSection{Homotopy operations for $s\calL(n)$}\label{Homotopy operations for sLn}
We will now state exactly how much of the structure given in \S\ref{Homotopy operations for simplicial Lie algebras} carries over to our new setting. If $L\in s\calL(n)$, we have a structure map $[\,,]:F^{\calL(n)}L\to L$, and more specifically, a structure map
\[[\,,]:\quadgrad{2}F^{\calL(n)}L\to L,\textup{\ \ with\ \ }\Lambda^2L\subseteq \Sigma^{-1}\quadgrad{2}F^{\calL(n)}L\subseteq S^2L,\]
where the desuspension acts in the cohomological degree $t$.

Only certain of the external homotopy operations $\pi_*V\to \pi_* S^2V$ defined in \S\ref{External unary homotopy operations} factor through $\pi_*\Sigma^{-1}\quadgrad{2}F^{\calL(n)}V$, and similarly for the operations of \S\ref{External binary homotopy operations}.  One readily checks that the operations that factor in this way are:
%\begin{gather*}
%%\sigma_i^\textup{ext}:\pi_nV\to \pi_{n+i}(\Sigma^{-1}\quadgrad{2}F^{\calL(n)}V)\textup{\ when $0\leq i\leq n$ but $i,s_1,\ldots,s_n$ are not all zero};\\
%\widetilde{\nabla}: \Sigma^{-1}\quadgrad{2}F^{\calL(n+1)}(\pi_*V)\to \pi_*(\Sigma^{-1}\quadgrad{2}F^{\calL(n)}V).
%\end{gather*}
\[\sigma_i^\textup{ext}:\pi_nV\to \pi_{n+i}(\Sigma^{-1}\quadgrad{2}F^{\calL(n)}V)\]
defined only when $0\leq i\leq n$ but $i,s_1,\ldots,s_n$ are not all zero, and
\[\widetilde{\nabla}: \quadgrad{2}F^{\calL(n+1)}(\pi_*V)\to \pi_*(\quadgrad{2}F^{\calL(n)}V).\]
The resulting operations on $\pi_*L$, for $L\in s\calL(n)$, 
%\begin{gather*}
%\lambda_i:=\pi_*([\,,])\circ\sigma_i^\textup{ext}:\pi_nL\to \pi_{n+i}L\textup{\ when $0\leq i\leq n$ but $i,s_1,\ldots,s_n$ are not all zero;}\\
%[\,,]:=\pi_*([\,,])\circ\widetilde{\nabla}: \quadgrad{2}F^{\calL(n+1)}(\pi_*L)\to \pi_*L.
%\end{gather*}
are right $\lambda$-operations
\[(\DASH)\lambda_i:\Bigl((\pi_{s_{n+1}}L)_{s_n,\ldots,s_1}^t\overset{\sigma_i}{\to}(\pi_{s_{n+1}+i}(\Sigma^{-1}\quadgrad{2}F^{\calL(n)}L))_{2s_n,\ldots,2s_1}^{2t}\overset{\pi_*([\,,\,])}{\to}(\pi_{s_{n+1}+i}L)_{2s_n,\ldots,2s_1}^{2t+1}\Bigr)\]
defined whenever $0\leq i\leq s_{n+1}$ and not all of $i,s_n,\ldots,s_1$ equal zero, and a bracket:
%\[[\,,]:\Bigl((\pi_{s_{n+1}}L)_{s_n,\ldots,s_1}^t\otimes (\pi_{s'_{n+1}}L)_{s'_n,\ldots,s'_1}^{t'} \overset{\widetilde{\nabla}}{\to}(\pi_{s_{n+1}+i}(\Sigma^{-1}\quadgrad{2}F^{\calL(n)}L))_{2s_n,\ldots,2s_1}^{2t}\overset{\pi_*([\,,\,])}{\to}(\pi_{s_{n+1}+i}L)_{2s_n,\ldots,2s_1}^{2t+1}\Bigr)\]
%\begin{alignat*}{2}
%[\,,]:\Bigl((\pi_{s_{n+1}}L)_{s_n,\ldots,s_1}^t\otimes (\pi_{s'_{n+1}}L)_{s'_n,\ldots,s'_1}^{t'} \overset{\widetilde{\nabla}}{\to}&(\pi_{s_{n+1}+s'_{n+1}}(\Sigma^{-1}\quadgrad{2}F^{\calL(n)}L))_{s_n+s'_n,\ldots,s_1+s'_1}^{t+t'}\\
%&\overset{\pi_*([\,,\,])}{\to}(\pi_{s_{n+1}+s'_{n+1}}L)_{s_n+s'_n,\ldots,s_1+s'_1}^{t+t'+1}\Bigr)
%\end{alignat*}
\begin{alignat*}{2}
[\,,]:\Bigl(&(\pi_*L)_{s_{n+1},\ldots,s_1}^t\otimes (\pi_*L)_{s'_{n+1},\ldots,s'_1}^{t'} 
\overset{}{\to}
(\quadgrad{2}F^{\calL(n+1)}\pi_*L)_{s_{n+1}+s'_{n+1},\ldots,s_1+s'_1}^{t+t'+1}
\\
&\qquad\qquad\overset{\widetilde{\nabla}}{\to}(\pi_*\quadgrad{2}F^{\calL(n)}L)_{s_{n+1}+s'_{n+1},\ldots,s_1+s'_1}^{t+t'} \overset{\pi_*([\,,\,])}{\to}(\pi_*L)_{s_{n+1}+s'_{n+1},\ldots,s_1+s'_1}^{t+t'+1}\Bigr).
\end{alignat*}
We have written the bracket as a map from $(\pi_*L)^{\otimes 2}$ to clarify the degree shift, but nevertheless, the top $\lambda$-operation, whenever it is defined, is a restriction for this bracket whenever.
Indeed, this set of natural operations satisfies the evident modification of Proposition \ref{omnibus on htpy of Lie algs}.


%
%\textbf{(much of this now moved...)} We will now give an account of the natural operations on the homotopy of a simplicial object of $\calL(n)$.
%
%The homotopy operations we will present will be in the form of a natural transformation of functors $s\vect{+}{n}\to \vect{+}{n+1}$, which will be induced by the Eilenberg-Mac Lane shuffle map $\nabla:N_*(V)\otimes N_*(V)\to N_*(V\otimes V)$. \textbf{Need to clarify what is meant by quad grading.}
%\begin{prop}[(Various authors)]
%\label{the top homotopy operations for Lie algebras}
%There is a natural commuting diagram:
%\[\xymatrix@R=4mm{
%\quadgrad{2}F^{\calL(n+1)}(\pi_*V)\ar@{-->}[r]^-{\widetilde{\nabla}}
%\ar[d]^-{\textup{incl}}
%&%r1c1
%\pi_*(\quadgrad{2}F^{\calL(n)}V)\ar[d]^-{\pi_*(\textup{incl})}
%\\%r1c2
%S^2(\pi_*V)\ar[r]^-{\widetilde{\nabla}}&%r2c1
%\pi_*(S^2V)%r2c2
%}\]
%%\[\widetilde{\nabla}:\quadgrad{2}F^{\calL(n+1)}(\pi_*V)\to \pi_*(\quadgrad{2}F^{\calL(n)}V)\]
%with horizontal maps defined by the formulae (for cycles $x,y,z\in ZN_*(V)$):
%\[\widetilde{\nabla}(\overline{x}\otimes \overline{y}+\overline{y}\otimes \overline{x})=\overline{\nabla(x\otimes y+y\otimes x)}\textup{, and }\widetilde{\nabla}(\overline{x}\otimes\overline{x})=\overline{\nabla(x\otimes x)}.\]
%\end{prop}
%\begin{proof}
%Since $\nabla$ is symmetric, the formulae given for $\widetilde{\nabla}$ do indeed return homotopy classes in either $\pi_*(\quadgrad{2}F^{\calL(n)}V)$ or $\pi_*(S^2V)$, however, it is not clear that either of the maps $\widetilde{\nabla}$ is well defined. We will only need to check that the lower map is well defined: as $\pi_*(\textup{incl})$ is a monomorphism (by \cite[Proposition 5.6]{BousOpnsDerFun.pdf}), it will follow that the upper map is well defined. 
%%Although it is not clear that $\widetilde{\nabla}$ is well defined, $\widetilde{\nabla}$ will prove to be the unique lift in a diagram: %We will construct it as the unique lift (dotted) of an analogous map $\widetilde{\nabla}'$:
%%\[\xymatrix@R=4mm{
%%\quadgrad{2}F^{\calL(n+1)}(\pi_*V)\ar@{-->}[r]^-{\widetilde{\nabla}}
%%\ar[d]^-{\textup{incl}}
%%&%r1c1
%%\pi_*(\quadgrad{2}F^{\calL(n)}V)\ar[d]^-{\pi_*(\textup{incl})}
%%\\%r1c2
%%S^2(\pi_*V)\ar[r]^-{\widetilde{\nabla}'}&%r2c1
%%\pi_*(S^2V)%r2c2
%%}\]
%%in which the map $\pi_*(\textup{incl})$ is a monomorphism (by \cite[Prop 5.6]{BousOpnsDerFun.pdf}). Now $\widetilde{\nabla}'$ will be defined by the same formulae as $\widetilde{\nabla}$. 
%\begin{shaded}\tiny
%By general observations on natural transformations out of the endofunctor $S^2$ of $\vect{}{}$, to construct a natural map $\widetilde{\nabla}:S^2(\pi_*V)\to\pi_*(S^2V)$ is to give a natural linear map $p:S_2(\pi_*(V))\to \pi_*(S^2V)$ and a natural function $\restn{(\DASH)}:\pi_n(V)\to \pi_{2n}(S^2V)$ satisfying $\restn{(x+y)}=\restn{x}+\restn{y}+p(x\otimes y)$. Then $\widetilde{\nabla}$ is the unique natural map such that $\widetilde{\nabla}\circ\trace=p$ and $\widetilde{\nabla}(a)=p(a\otimes a)$ for $a\in\pi_*\left(V\right)$.
%%\[p=\left(S_2\pi_*(V)\overset{\trace}{\to} S^2\pi_*(V)\overset{\widetilde{\nabla}'}{\to} \pi_*(S^2V)\right)\textup{ and }\restn{(\DASH)}=\left(\pi_*(V)\overset{v\mapsto v\otimes v}{\to} S^2\pi_*(V)\overset{\widetilde{\nabla}'}{\to} \pi_*(S^2V)\right).\]
%We define $\widetilde{\nabla}$ in this way, with
%\[p=\pi_*(\trace)\circ\nabla:\left(S_2(\pi_*(V))\overset{\nabla}{\to}\pi_*(S_2(V))\overset{\pi_*(\trace)}{\to}\pi_*(S^2(V))\right)\]
%and with $\restn{(\DASH)}$ the operation $\sigma_n:\pi_n(V)\to \pi_{2n}(S^2V)$ detailed in \citeBOX[\S3]{MR1089001}.
%\end{shaded}
%%In general, to construct a natural map $g$ from $S^2$ to any other endofunctor $G$ of the category of $\Ftwo $-vector spaces, it is enough to define a natural linear map $p:S_2V\to GV$ and natural function $\restn{(\DASH)}:V\to GV$ satisfying $\restn{(x+y)}=\restn{(x)}+\restn{(y)}+p(x\otimes y)$. Then there is a unique natural map $g:S^2\to G$ such that we have
%%\[p=\left(S_2V\overset{\trace}{\to} S^2V\to GV\right)\textup{ and }\restn{(\DASH)}=\left(V\overset{v\mapsto v\otimes v}{\to} S^2V\to GV\right).\]
%%The map $\widetilde{\nabla}'$ is exactly obtained by this method, using 
%%
%%\[p([x]\otimes [y])=[\nabla(x\otimes y+y\otimes x)],\textup{ and }\restn{([x])}=[\nabla(x\otimes x)].\qedhere\]
%\end{proof}
%It is classical that via this map, the homotopy of a simplicial object $L\in s\calL(n)$ obtains the structure of an object of $\calL(n+1)$:
%\begin{prop}[(Curtis, 6-A)]
%\label{prop on top operations}
%If $L\in s\calL(n)$ has structure map $\rho:F^{\calL(n)}L\to L$, then $\pi_*(L)\in\calL(n+1)$, with structure map extending $\pi_*(\rho)\circ\widetilde{\nabla}:\quadgrad{2}F^{\calL(n+1)}(\pi_*L)\to \pi_*L$.
%\end{prop}
%\begin{proof}
%This theorem follows from the analogous theorem for fully restricted simplicial Lie algebras $L$, once the operations $\widetilde{\nabla}$ have been identified as the operations of \citeBOX[\S8.5]{CurtisSimplicialHtpy.pdf}. In the proposed $\calL(n+1)$-structure, for $x,y
%\in N_*L$ the bracket of $\overline{x},\overline{y}\in\pi_*L$ is given by
%\[\overline{\rho(\nabla(x\otimes y+y\otimes x))}=\overline{\rho(\trace(\nabla(x\otimes y)))}=\overline{\textup{br}(\nabla(x\otimes y))},\]
%and if $n\geq1$, the restriction of an element $\overline{x}\in\pi_nL$ is given by
%\[\overline{\rho\nabla(x\otimes x)}=\overline{\rho\trace(\nabla^0(x\otimes x))}=\overline{\textup{br}(\nabla^0(x\otimes x))},\]
%and these two formulae coincide with those of \cite{CurtisSimplicialHtpy.pdf}. The point here was that both before and after taking homotopy groups, brackets are defined using the trace. We should also check that in the proposed structure, the proposed restriction $\pi_0L\to \pi_0L$ is given on the chain level by the restriction $L_0\to L_0$, as in \cite{CurtisSimplicialHtpy.pdf}. As $\nabla^0:N_0L\otimes N_0L\to N_0(L\otimes L)$ is the identity, for $\overline{x}\in\pi_*L$:
%\[(\pi_*(\rho)\circ\widetilde{\nabla})(\overline{x})=\overline{\rho(\nabla(x\otimes x))}=\overline{\rho(x\otimes x)}=\overline{\restn{x}}.\qedhere\]
%\end{proof}
%We will now turn to some further operations defined on the homotopy of an object of $s\calL(n)$.
%\begin{prop}[{\cite{CurtisSimplicialHtpy.pdf}, \cite{6Author.pdf}}]
%\label{linear operations on homotopy of lie alg}
%If $L\in s\calL(n)$ has structure map $\rho:F^{\calL(n)}L\to L$, then there are right operations
%\[(\DASH)\lambda_i:(\pi_{s_{n+1}}L)_{s_n,\ldots,s_1}^t\to (\pi_{s_{n+1}+i}L)_{2s_n,\ldots,2s_1}^{2t+1}\]
%defined whenever $0\leq i\leq s_{n+1}$ and not all of $i,s_n,\ldots,s_1$ equal zero by the composite
%\[(\pi_{s_{n+1}}L)_{s_n,\ldots,s_1}^t\overset{\sigma_i}{\to}(\pi_{s_{n+1}+i}(\quadgrad{2}F^{\calL(n)}L))_{2s_n,\ldots,2s_1}^{2t}\overset{\pi_*(\rho)}{\to}(\pi_{s_{n+1}+i}(L))_{2s_n,\ldots,2s_1}^{2t+1}.\]
%The operation $\lambda_i$ is linear when $i<s_{n+1}$, while the top operation $\lambda_{s_{n+1}}$ equals the restriction defined in Proposition \ref{prop on top operations}. The operations satisfy the Adem relations for the $\Lambda$-algebra. That is, whenever $i>2j$, and $x$ is a homogeneous element of $\pi_*L$ such that $x\lambda_j\lambda_i$ is defined:
%\[x\lambda_j\lambda_i=\sum_{k=0}^{(i-2j)/2-1}{i-2j-2-k\choose k}x\lambda_{i-j-1-k}\lambda_{2j+1+k}.\]
%\end{prop}
%\begin{proof}
%As in the previous proposition, we only need to identify the operations defined with those given by Curtis in the case of a (fully) restricted Lie algebra. This is verified when $i\geq1$ using [Dwyer, 4.4], and when $i=0$ using the fact that $\sigma_0:\pi_*L\to \pi_*(S^2L)$ is the map $\overline{x}\mapsto\overline{x\otimes x}$.
%\end{proof}
\SubsectionOrSection{The category $\calU(n+1)$ of unstable partial right $\Lambda$-modules}
For $n\geq0$, let $\calU(n+1)$ denote \emph{the category  of unstable partial right $\Lambda$-modules}, the algebraic category whose objects are vector spaces $V\in \vect{+}{n+1}$ equipped with  \emph{linear} right $\lambda$-operations
\[(\DASH)\lambda_i:V_{s_{n+1},s_n,\ldots,s_1}^t\to V_{s_{n+1}+i,2s_n,\ldots,2s_1}^{2t+1}\]
defined whenever $0\leq i< s_{n+1}$ and not all of $i,s_{n},\ldots,s_{1}$ are zero, satisfying the unstable $\Lambda$-Adem relations of Proposition \ref{omnibus on htpy of Lie algs}\emph{(\ref{meep5})}.

We have shown that an object of $\PA{\calL(n)}$ is in particular an object of $\calU(n+1)$, indeed, the $\calU(n+1)$-structure on $\pi_*L$ consists solely of its \emph{non-top} $\lambda$-operations.
%\begin{prop}
%\label{compatibilities between U and L in W}
%For $L\in s\calL(n)$, the homotopy groups $\pi_*(L)$ are simultaneously an object of $\calL(n+1)$ and of $\calU(n+1)$. The $\calU(n+1)$-operations are annihilated by the bracket. That is, for $i<s_{n+1}$, $x\in\pi_{s_{n+1}}L$ and $y\in \pi_*L$ such that $x\lambda_i$ is defined, we have $[x\lambda_i,y]=0$.
%\end{prop}
\SubsectionOrSection{The category $\calw(n+1)$ of $\calL(n)$-$\Pi$-algebras}
\label{wn+1 or pialgs section}
For $n\geq0$, let $\calw(n+1)$ denote \emph{the category of $\calL(n)$-$\Pi$-algebras}. By the following proposition, this is the algebraic category whose objects are $\vect{+}{n+1}$-graded vector spaces which are simultaneously an object of $\calU(n+1)$ and of $\calL(n+1)$, such that the compatibilities of Proposition \ref{omnibus on htpy of Lie algs} are satisfied. Explained another way, an object of $\calw(n+1)$ is such a vector space with all of the $\lambda$-operations, both top and non-top, and the bracket described in \S\ref{Homotopy operations for sLn}, subject to the compatibilities of Proposition \ref{omnibus on htpy of Lie algs}.

This category has a number of useful properties, following from the calculations of \cite{6Author.pdf}, primarily:
\begin{prop}
\label{prop on Wnplus1 being the pialgs for Wn}
The category $\calw(n+1)$ is isomorphic to the category $\PA{\calL(n)}$ of $\calL(n)$-$\Pi$-algebras, so that the operations defined above exhaust the set of natural operations on the homotopy of simplicial objects of $\calL(n)$.
The monad $F^{\calw(n+1)}$ on $\vect{+}{n+1}$ factors as a composite $F^{\calU(n+1)}\circ F^{\calL(n+1)}$, with monad structure arising from a distributive law \cite{BeckDistLaws} of monads on $\vect{+}{n+1}$.
\end{prop}
\begin{proof}
All of these facts are easy to prove after observing that, for $W\in s\vect{}{}$ a coproduct of spheres, $\pi_*(F^{\calL(n)}W)$ embeds in $\pi_*({\Gamma(\LieOperad)}W)$, which, along with $\pi_*({\Lambda(\LieOperad)}W)$, is described in Proposition \ref{Lie homotopy operations} (although by $\Lambda(\LieOperad)$ and $\Gamma(\LieOperad)$ we mean the shifted monads of \S\ref{The categories Ln}). In order to make this observation, let write $W_\textbf{0}$ for $\bigoplus_{t\geq1}W_{0,\ldots,0}^t$, the non-restrictable part of $W$. This is actually a sub-coproduct of $W$, the coproduct of those summands of $W$ which lie in homological dimension $(0,\ldots,0)$. There is a commuting diagram of simplicial vector spaces, containing two short exact sequences:
\[\xymatrix@R=-2mm{
0
\ar[r]
&%r1c1
F^{\calL(n)}W
\ar[r]^{\alpha}
&%r1c2
{\Gamma(\LieOperad)}W
\ar@{->>}[dr]^-(.4){\rho\gamma}\ar@{->>}[dd]^-{\gamma}
&%r1c3
%\Fr{\RestLie{n}}W/\Fr{\PRLie{n}}W
%\ar[r]
%\ar[d]^-{\cong}
%&%r1c4
\\%r1c5
&&&\frac{\displaystyle {\Gamma(\LieOperad)}W_\textbf{0}}{\displaystyle {\Lambda(\LieOperad)}W_\textbf{0}}
\ar[r]
&
0
\\
0
\ar[r]
&%r1c1
{\Lambda(\LieOperad)}W_\textbf{0}
\ar[r]^{\beta}
&%r1c2
{\Gamma(\LieOperad)}W_\textbf{0}
\ar@{->>}[ru]^-(.4){\rho}&%r1c3
%\Fr{\RestLie{n}}W^\textup{z}/ F^{\Lambda(\LieOperad)}W^\textup{z}
%\ar[r]
%&%r1c4
%0
}\]
where $\alpha$ and $\beta$ are inclusions and $\gamma$ and $\rho$ are epimorphisms.
On homotopy groups: $\beta_*$ is injective (its source and target are well understood), so that $\rho_*$ is surjective. Thus $\gamma_*$ is surjective (after all, $\gamma$ is an isomorphism in those internal degrees in which its codomain in non-zero), implying that $(\rho\gamma)_*$ is surjective. This implies that $\alpha_*$ is injective, as hoped.
%\textbf{change gamma to other symbol?}
\end{proof}
%It will be useful in what follows that Proposition \ref{prop on Wnplus1 being the pialgs for Wn} states that $\calw(n+1)$ is monadic over $\vect{+}{n+1}$, not just over a category of graded sets.
The following two lemmas are the direct analogues of lemmas \ref{quad gradings have a chance on U0 W0} and \ref{Kill P ops gives lie alg}:
\begin{lem}
\label{quad gradings have a chance on Un Wn}
For $n\geq0$, the monads $F^{\calU(n+1)}$ and $F^{\calw(n+1)}$ on $\vect{+}{n+1}$ may be promoted to monads on the category $\quadgrad{}\vect{+}{n+1}$, by insisting that quadratic gradings add under brackets and double under $\lambda$-operations. The same holds for $\call(n)$ for $n\geq0$.
\end{lem}
\SubsectionOrSection{The factorization $Q^{\calL(n)}\circ Q^{\calU(n)}$ of $Q^{\calw(n)}$}
For $n\geq0$, we define
\[Q^{\calU(n+1)}:=\Bigl(\calw(n+1)\overset{\textup{forget}}{\to}\calU(n+1)\overset{Q^{\calU(n+1)}}{\to}\vect{+}{n+1}\Bigr).\]
That is, for $X\in\calw(n+1)$ we may take the quotient by the image of the \emph{non-top} $\lambda$-operations (which are linear, so that this operation is well defined). In fact, it is not hard to see that, for $X\in\calw(n+1)$, $Q^{\calU(n+1)}X$ retains the structure of an object of $\calL(n+1)$, so that we may view $Q^{\calU(n+1)}$ as a functor $\calw(n+1)\to \calL(n+1)$.
\begin{lem}
\label{Kill lambda ops gives lie alg}
For $n\geq0$ and $X\in\calw(n+1)$, $X$ is in particular an object of $\calL(n+1)$, and the vector space $Q^{\calU(n+1)}X$ retains this structure, yielding a factorization:% of indecomposable functors, $Q^{\calw(n)}=Q^{\calL(n)}\circ Q^{\calU(n)}$
\[Q^{\calw(n+1)}=Q^{\calL(n+1)}\circ Q^{\calU(n+1)}:\left(\calw(n+1)\to \calL(n+1)\to \vect{+}{n+1}\right).\]
Moreover the composite $Q^{\calU(n+1)}\circ F^{\calw(n+1)}$ equals the free construction $F^{\calL(n+1)}$.
\end{lem}
\begin{proof}
Similar to the proof of Lemma \ref{Kill P ops gives lie alg}, using the observation from the proof of Proposition \ref{prop on Wnplus1 being the pialgs for Wn} that $\pi_*(F^{\calL(n)}W)\subseteq\pi_*(F^{\Gamma(\LieOperad)}W)$.
\end{proof}
This differs from the definition of $Q^{\calU(0)}\to \calL(0)$, in which one takes the quotient by \emph{all} the $P$-operations.
Indeed, the category $\calw(0)$ differs from the categories $\calw(n+1)$ (for $n\geq0$) in a number of ways. Indeed, if $X\in\calw(0)$ and $Y\in \calw(n+1) $:
\begin{enumerate}
\item $Y$ is a  Lie algebra, while $X$ is only an $S(\LieOperad)$-algebra;
\item the $P$-operations on $X$ are always defined and vanish when out of range, while the $\lambda$ operations are simply undefined when out of range;
\item the top $P$-operation is the self square and thus is linear, while the top $\lambda$-operation is the restriction and thus a quadratic refinement of the bracket.
\end{enumerate}
Nevertheless, these two regimes share at least the following common ground:
\begin{cor}
\label{indec functors common ground 0 and n}
For all $n\geq0$, there are algebraic categories $\calw(n)$ and $\calU(n)$, a forgetful functor $\calw(n)\to\calU(n)$, and a functor $Q^{\calU(n)}:\calw(n)\to \calL(n)$, such that
\[Q^{\calw(n)}=Q^{\calL(n)}\circ Q^{\calU(n)}\textup{\ \ and\ \ }Q^{\calU(n)}\circ F^{\calw(n)}=F^{\calL(n)}.\]
\end{cor}


\SubsectionOrSection{Decomposition maps for $\calL(n)$ and $\calw(n)$}
Here we will introduce decomposition maps for the categories $\calL(n)$ and $\calw(n)$, and prove a certain compatibility between them. The definitions are simple enough, and the reader can verify that each is well defined. For any  $n\geq0$, the following formulae define decomposition maps $j_\calc:Q^\calc(X\smashcoprod Y)\to Q^\calc X\otimes Q^\calc Y$:
\begin{alignat*}{2}
%\calw(0):&\quad &
j_{\calw(0)}:P^{i_\ell}\cdots P^{i_1}[z_1,\cdots ,z_a]&\longmapsto\begin{cases}
z_1\otimes z_2,&\textup{if }\ell=0,\,a=2,\,z_1\in X,\,z_2\in Y,
\\0,&\textup{otherwise.}
\end{cases}
\\
%\calw(n+1):&&
j_{\calw(n+1)}:[z_1,\cdots ,z_a]\lambda_{i_1}\cdots \lambda_{i_\ell}&\longmapsto\begin{cases}
z_1\otimes z_2,&\textup{if }\ell=0,\,a=2,\,z_1\in X,\,z_2\in Y,
\\0,&\textup{otherwise.}
\end{cases}
\\
%\calL(n):&&
j_{\calL(n)}:\restnRepeated{[z_1,\cdots ,z_a]}{r}&\longmapsto\begin{cases}
z_1\otimes z_2,&\textup{if }r=0,\,a=2,\,z_1\in X,\,z_2\in Y,
\\0,&\textup{otherwise.}
\end{cases}
\end{alignat*}
\begin{prop}
\label{quadpartcalc for various W and L}
Suppose $V\in\vect{+}{n}$. Then:
\begin{enumerate}
\item \makebox[\widthof{$\quadratic_\restliealgs$}][l]{$\quadratic_{\calL(n)}$} is the composite $F^{\calL(n)} V\epi \quadgrad{2}F^{\calL(n)} V\subseteq \Sigma S^2V$;
\item \makebox[\widthof{$\quadratic_\restliealgs$}][l]{$\quadratic_{\calw(n)}$} is the composite $F^{\calw(n)} V\epi F^{\calL(n)} V\epi \quadgrad{2}F^{\calL(n)} V\subseteq \Sigma S^2V$.\end{enumerate}
\end{prop}
\begin{proof}
Consider the case $\calw(n+1)$ for $n\geq0$. As $\quadratic_\calc$ vanishes except on quadratic grading 2, one only checks terms $[x,y]$, $\restn{x}$, and $x\lambda_i$ (a $\calU(n+1)$-operation, not the restriction):
\begin{alignat*}{2}
\quadratic_{\calw(n+1)}([x,y])&=j_{\calw(n+1)}([{x_1}+{{x_2}},{y_1}+{{y_2}}]+[{x_1},{y_1}]+[{{x_2}},{{y_2}}])\\
&=j_{\calw(n+1)}([{x_1},{{y_2}}]+[{{x_2}},{y_1}])=x\otimes y+y\otimes x,
\end{alignat*}
which is precisely the representation of $[x,y]$ in $\quadgrad{2}F^{\calL(n+1)}V\subseteq \Sigma (V^{\otimes2})$. Similarly, $\quadratic_{\calw(n+1)}(x\lambda_i)$ vanishes (as $\calU(n+1)$-operations are linear), while $\quadratic_{\calw(n+1)}(\restn{x})$ equals $x\otimes x$ as desired. The other cases, including the case of $\calw(0)$, are barely any different.
\end{proof}





\end{homotopy operations for PRLs}


\begin{Cohomology Operations for W and U}
\SectionOrChapter{Operations on $\calw(n)$- and $\calU(n)$-cohomology}
\label{Cohomology Operations for W and U}

%\SubsectionOrSection{Cohomology for objects of $\calw$}
%We will use the standard simplicial bar construction as our simplicial resolution of $X\in\calw$, writing $B^\calw_sX=U^{s+1}X$. Then the cohomology of $X$, $H^{s}_t(X)\in\vect{1}{+}$, is the cohomology of the cochain complex
%\[C^{s}_{t}=\Hom(N_s(Q^\calw B_{\bullet}X)^{t},\Ftwo ).\]
%We will frequently use the identification $Q^\calw B_{s}X=U^{s}X$.


\SubsectionOrSection{Vertical $\delta$-operations on $H^*_{\calw(0)}$ and $H^*_{\calU(0)}$}
For $V\in \vect{+}{0}$, we will define natural homomorphisms
\[\theta_i:(F^{\calw(0)}V)^{t+i+1}\to V^{t}, \textup{ for $2\leq i<t$}.\]
Indeed, there are natural homomorphisms into the quadratic grading 2 part of $F^{\calw(0)}V$:
\begin{align*}
P^i:V^{t}&\to \quadgrad{2}(F^{\calw(0)}V)^{t+i+1}, \textup{\quad  for $2\leq i<t$}\\
[\,,]:(S_2V)^{t}&\to \quadgrad{2}(F^{\calw(0)}V)^{t+1},
\end{align*}
and for given $m\geq1$, the degree $m$, quadratic grading 2 part $\quadgrad{2}(F^{\calw(0)}V)^m$ decomposes as
%\[UV^{m,(2)}=\left(\bigoplus_{n_1+n_2=m-1}\im ([\,,]_{n_1,n_2})\right) \oplus\left(\bigoplus_{2\leq i< (m-1)/2}\im (P^i_{m-i-1})\right)\]
%
\[\quadgrad{2}(F^{\calw(0)}V)^{m}=%\left(\bigoplus_{n_1+n_2=m-1}
\im \bigl((S_2V)^{m-1}\overset{[,]}{\to} \quadgrad{2}F^{\calw(0)}V\bigr)%\right) 
\oplus\bigoplus_{\!\!\!\!\!\!2\leq i< (m-1)/2\!\!\!\!\!\!}\im \bigl(V^{m-i-1}\overset{P^i}{\to}\quadgrad{2}F^{\calw(0)}V\bigr).\]
Moreover, each map $P^i:V^t\to \quadgrad{2}(F^{\calw(0)}V)^{t+i+1}$ appearing in this decomposition is an isomorphism onto its image, so that for $2\leq i <t$ we may construct $\theta_i$ as the composite
\[\theta_i:\left((F^{\calw(0)}V)^{t+i+1}\overset{\textup{proj}}{\makebox[.06ex][l]{$\to$}\to} (\quadgrad{2}F^{\calw(0)}V)^{t+i+1}\overset{\textup{proj}}{\makebox[.06ex][l]{$\to$}\to} \im (P^i)\overset{(P^i)^{-1}}{\to}V^{t}\right).\]
Here we have projected onto the quadratic filtration 2 part, and then further onto the relevant summand in its natural decomposition. Note that although $P^t:V^t\to \quadgrad{2}(F^{\calw(0)}V)^{2t+1}$ is a non-trivial linear map when $t\geq2$, its image is entangled with the image of the bracket, and we are not able to split it off. Thus we are not able to improve on the bounds $2\leq i< t$.
\begin{prop}
\label{prop on thetaistar}
There is a map $\theta_i^\star:V_t\to (C^{\HC{\algs}}V)_{t+i+1}$, natural in $V\in\vect{}{+}$, such that the following diagram commutes:
\[\xymatrix@R=4mm@C=20mm{
\dual((C^{\HC{\algs}}V)_{t+i+1})&%r1c1
\dual V\ar[l];[]^-{(\theta_i^\star)^*}
%\ar[dl]^-{\ExtCohOp^k}
\\%r1c2
(F^{\calw(0)}\dual V)^{t+i+1}\ar[u]\ar[ur]_-{\theta_i}
%&
%\dual V\ar[u]
}\]
\end{prop}
\begin{proof}
When $V$ is of finite type, as $F^{\calw(0)}$ preserves vector spaces of finite type, we may simply define $\theta_i^\star$ to be the dual of $\theta_i$. This is natural on vector spaces of finite type, and any vector space is the filtered colimit of such.
\end{proof}


\begin{prop}
\label{operations on goerss homology}
Suppose that $X\in s\calw(0)$ is almost free\todo{enhance?},  %\TODOCOMMENT{s=0?}, 
so that we may identify $H^*_{\calw(0)}X$ with $\pi^*\dual Q^{\calw(0)}X$. Then for $2\leq i <t$, the chain map $\widetilde{\theta_i}$ of Proposition \ref{general CohOpns given irreducibility} induces a linear operation
\[\deltav_i:(H^*_{\calw(0)}X)^{s}_t\to (H^*_{\calw(0)}X)^{s+1}_{t+i+1}.\] 
These operations are natural in maps preserving the generating subspaces, and satisfy the unstable $\delta$-Adem relation of Proposition \ref{omnibus on htpy of simp algs}\textit{(\ref{deltaademsunstable})}.
\end{prop}
\noindent For \emph{any} $X\in s\calw(0)$, the bar construction $Q^{\calw(0)}B^{\calw(0)}X$ calculating  $H^*_{\calw(0)}X$ has a natural almost free structure, so that Proposition \ref{operations on goerss homology} constructs natural operations on $H^*_{\calw(0)}X$ for all $X\in s\calw(0)$.
%Now suppose that $X\in\calw(0)$, and consider the bar construction $Q^{\calw(0)}B^{\calw(0)}X$ whose cohomotopy is $H^*_{\calw(0)}X$. Until the end of \S\ref{Cohomology Operations for W and U}, we will write $Q\mathbb{B}$ for this simplicial object, write $F$ for $F^{\calw(0)}$, and identify $Q\mathbb{B}_s$ with $F^sX$.
\begin{proof}[Proof of Proposition \ref{operations on goerss homology}]
The conditions of Proposition \ref{general CohOpns given irreducibility} are satisfied with $\theta=\theta_i$ and $G$ is the identity functor. The condition \[\smash{\theta\circ\mu_V=\theta\circ \epsilon_{F^\calc V} +\theta\circ {F^\calc \epsilon_V}:F^\calc F^\calc V\to V}\]
just states that given an iterated expression in $F^\calc F^\calc V$, the two obvious ways to produce a summand of the form $P^iv$ under the map $F^\calc F^\calc V\to F^\calc V$ are the only ways, due to the homogeneity of the $P$-Adem relations.

It just remains to prove the $\delta$-Adem relations, which we will do using the technique of \cite{PriddyKoszul.pdf}, the point being that the algebra of $\delta$-operations is Koszul dual to $\Palg$. For this, we define a map $\theta_{ij}$, whenever $i<2j$, $2\leq j<t$ and $2\leq i<t+j+1$:
\[\theta_{ij}:\left((F^{\calw(0)}V)^{t+i+j+2}\overset{\textup{proj}}{\makebox[.06ex][l]{$\to$}\to} (\quadgrad{4}F^{\calw(0)}V)^{t+i+j+2}\overset{\textup{proj}}{\makebox[.06ex][l]{$\to$}\to} \im (P^{i,j})\overset{(P^{i,j})^{-1}}{\to}V^{t}\right),\]
where we have split off the image of 
$P^{i,j}=P^iP^j$ %:V^n\to (F^{\calw(0)}^{(4)}V)^{n+i+j+2}\]
as before. This is possible since neither $P^j$ nor $P^i$ are entangled with the bracket in these ranges. We may identify $Q^{\calw(0)}X_s$ with $V_s$, at the cost of replacing $d_0$ with $\epsilon\circ d_0$, as in Lemma \ref{identify almost free indecs with gens}.  Define $\widetilde{\theta_{ij}}$ to be the composite $V_{s+1}\overset{d_0}{\to}FV_s\overset{\theta_{ij}}{\to}V_s$. This will be the nullhomotopy giving the $\delta$-Adem relation. %It'll feel easier to think of it as a nullhomotopy of the un-normalized complex. 
As in the proof of Proposition \ref{general CohOpns given irreducibility}, we have $d_k\circ\widetilde{\theta_{ij}}=\widetilde{\theta_{ij}}\circ d_{k+1}$ for $k\geq1$, and $\widetilde{\theta_{ij}}$ has nullhomotopy the sum
\[\epsilon d_0\widetilde{\theta_{ij}}+\widetilde{\theta_{ij}}(\epsilon d_0+d_1)=(\epsilon d_0\theta_{ij}+\theta_{ij}d_0\epsilon+\theta_{ij}d_0)d_0.\]
The $\delta$-Adem relation will follow from
\[\textstyle\theta_{ij}d_0=\Bigl(\epsilon d_0\theta_{ij}+\theta_{ij}d_0\epsilon+\sum_{\produces{(\alpha,\beta)}{(i,j)}{\Palg}}\theta_\beta d_0\theta_\alpha\Bigr):FV_{s+1}\to V_s,\]
%where the sum is taken over all the combinations of $\alpha,\beta\geq2$ such that $P^iP^j$ appears when $P^\alpha P^\beta$ is written in the $P$-admissible basis. 
This identity states the following: if $V\in\vect{+}{0}$, and $f(g_k)$ is a nested $\calw(0)$-expression with $g_k\in F^{\calw(0)}V$ and $f(g_k)\in F^{\calw(0)}F^{\calw(0)}V$, then if we write $d_0:F^{\calw(0)}F^{\calw(0)}V\to F^{\calw(0)}V$ for the monad product map, there are only three ways that one may obtain expressions of the form $P^iP^jv$ in $d_0(f(g_k))$: for some $k$, $g_k=P^iP^jv$, and $f $ adds no further operations to this term; $f=P^iP^jg_{k}$ for some $k$ for which $g_{k}=v$ is a unit expression; or for some $k$, $g_k=P^\beta v$, and  $f$ has $P^\alpha(g_k)$ as a summand. In this last case, after applying $d_0$, we may need to rearrange the composite $P^{\alpha}P^{\beta}v$ using the $P$-Adem relations, and we sum over those $(\alpha,\beta)$ producing a summand $P^iP^jv$.

This shows that the proposed nullhomotopy equals  
\[\textstyle\sum_{\produces{(\alpha,\beta)}{(i,j)}{\Palg}}\widetilde{\theta_\beta}\circ\widetilde{\theta_\alpha},\]
 and as Goerss \cite{MR1089001} \emph{constructs} the $P$-algebra as the Koszul dual, in the sense of \cite{PriddyKoszul.pdf}, of the $\Delta$-algebra, so that $\produces{(\alpha,\beta)}{(i,j)}{\Palg}$ if and only if $\produces{(i,j)}{(\alpha,\beta)}{\Delta}$, so that the nullhomotopy equals the desired sum:
\[\textstyle\sum_{\produces{(i,j)}{(\alpha,\beta)}{\Delta}}\widetilde{\theta_\beta}\circ\widetilde{\theta_\alpha}.\qedhere\]
%$P^{\alpha }P^{\beta}$ produces a summand $P^iP^j$ if and only if $\delta_i\delta_j$ produces a summand $\delta_{\alpha }\delta_{\beta}$. This proves the result. \textbf{Move Koszul observation earlier.}
\end{proof}
The same constructions work in the category $s\calU(0)$ of simplicial unstable $\Palg$-\emph{modules}, the only difference being that when we define $\theta_i$, we need not worry about Lie algebra structures, and we can define a map
\[\theta_i:(F^{\calU(0)}V)^{t+i+1}\to V^t\]
whenever $2\leq i\leq t$, so that there is one more operation available on $H^*_{\calU(0)}$ than on $H^*_{\calw(0)}$. It will be useful to encode this structure in a definition. Write $\calMv(1)$ for the algebraic category whose objects are vector spaces $M\in\vect{1}{+}$ with left $\delta$-operations
\[\deltav_i:M^{s}_t\to M^{s+1}_{t+i+1},\textup{ defined whenever $2\leq i\leq t$,}\]
satisfying unstable $\delta$-Adem relations analogous to those of \emph{(\ref{deltaademsunstable})} in Proposition \ref{omnibus on htpy of simp algs}.
\begin{prop}
\label{operations on untable P homology}
Suppose that $X\in s\calU(0)$ is almost free. Then the chain maps $\widetilde{\theta_i}$ of Proposition \ref{general CohOpns given irreducibility} give $H^*_{\calU(0)}X$ the structure of an object of $\calMv(1)$, natural in maps preserving the generating subspaces. In fact, $\calMv(1)$ is the category of $\calU(0)$-$H^*$-algebras.
\end{prop}
\noindent See \S\ref{Koszul Cx section} for further discussion of this fact, and Proposition \ref{the cokoszul complex is free} for a restatement. It is \textbf{not true} that $H^*_{\calw(0)}$ is an object of $\calMv(1)$, a fact that we emphasize because $H^*_{\calw(n)}$ \emph{will} be an object of $\calMv(n+1)$ for $n\geq 1$ (under definitions made in \S\ref{section: vertical Koszul operations n positive}).

In order to give a basis for a free object in $\calMv(1)$, for a sequence $I=(i_\ell,\ldots,i_1)$ of integers $i_j\geq2$, we use the function
\[\minDimP(I):=\max\{(i_1),\,(i_2-i_1-1),\,(i_3-i_2-i_1-2),\,\ldots,\,(i_{\ell}-\cdots-i_1-\ell+1)\},
\]
of \S\ref{The example of simplicial commutative F2-algebras},
following the convention that $\max(\emptyset)=-\infty$, and the notion of  $\delta$-admissibility from \S\ref{Homotopy operations for simplicial commutative algebras}: each $i_j\geq2$ and  $i_{j+1}\geq 2i_j$ for $1\leq j <\ell$.
\begin{lem}
\label{basis of element of M(0)}
%For $V\in\vect{1}{+}$ with homogeneous basis $B$, a basis of $F^{\calMv(1)}V$ consists of the classes $\delta_Ib$ where $b\in B\cap V^s_t$ is a basis element and $I$ is a $\delta$-admissible sequence with $t\geq\minDimP(I)$.
For $V\in\vect{1}{+}$ with homogeneous basis $B$, a basis of $F^{\calMv(1)}V$ is
\[\left\{\deltav_Ib\ \middle|\ b\in B^{s}_t\textup{, $I$ $\delta$-admissible with }\minDimP(I)\leq t\right\}.\]
\end{lem}
We will often apply such results as these when $V$ is concentrated in degrees $V_t^0$. At this point we introduce a notational abuse, identifying $\vect{}{+}$ with  the full subcategory of $\vect{1}{+}$ with objects concentrated in these degrees. The effect of this will be that we will be able to write $F^{\calMv(1)}V$ for $V\in \vect{}{+}$. In this setting, we may restrict Lemma \ref{basis of element of M(0)} to:
\begin{cor}
\label{basis of element of M(0) in deg 0}
For $V\in\vect{}{+}$ with homogeneous basis $B$, a basis of $F^{\calMv(1)}V\in\vect{1}{+}$ is
\[\left\{\deltav_Ib\ \middle|\ b\in B_t\textup{, $I$ $\delta$-admissible with }\minDimP(I)\leq t\right\}.\]
\end{cor}


\SubsectionOrSection{Vertical Steenrod operations for $H^*_{\calw(n)}$ and $H^*_{\calU(n)}$ when $n\geq1$}
\label{section: vertical Koszul operations n positive}
For $V\in \vect{+}{n}$, we will define natural homomorphisms
\[\theta^i:(F^{\calw(n)}V)^{2t+1}_{s_n+i-1,2s_{n-1},\ldots,2s_1}\to V^{t}_{s_n,\ldots,s_1},\]
which are defined for all $i,s_1,\ldots,s_n\geq0$ and $t\geq1$, but are zero except when $1\leq i \leq s_n$ and not all of $i-1,s_{n-1},\ldots,s_1$ are zero.
These are rather easier to define than in the $n=0$ case above, as the monad $F^{\calw(n)}$ is a simple composite $F^{\calu(n)}F^{\call(n)}$ of monads.
Indeed, there are natural monomorphisms
\[(\DASH)\lambda_{i-1}:V^{t}_{s_n,\ldots,s_1}\to (\quadgrad{2}F^{\calw(n)}V)^{2t+1}_{s_n+i-1,2s_{n-1},\ldots,2s_1}\]
defined only when   $1\leq i\leq n$ and $i-1,s_{n-1},\ldots,s_1$ are not all zero, and an inclusion
\[\textup{incl}:\quadgrad{2}F^{\calL(n)}V\to \quadgrad{2}F^{\calw(n)}V.\]
As in the $n=0$ case, the images of the listed maps are linearly independent and span the quadratic grading 2 part of $F^{\calw(n)}V$. We define $\theta^i$ to be zero unless $1\leq i\leq n$ and $i-1,s_{n-1},\ldots,s_1$ are not all zero, in which case we define it as the composite: %project onto the quadratic grading 2 part, then further onto the summand corresponding to the image of $\lambda_{i-1}$, and finally use the inverse of the map $\lambda_{i-1}$:
\[\theta^i:\Bigl((F^{\calw(n)}V)^{2t+1}_{s_n+i-1,2s_{n-1},\ldots,2s_1}\overset{\textup{proj}\circ\textup{proj}}{\makebox[.06ex][l]{$\to$}\to} \im (\lambda_{i-1})\overset{(\lambda_{i-1})^{-1}}{\to}V^{t}_{s_{n,\ldots,s_1}}\Bigr).\]
One can give exactly the same definitions for the free construction in $\calU(n)$, producing functions $\theta^i:F^{\calU(n)}V\to V$ which are zero under the same conditions as for $\calw(n)$.
Write $\calMv(n+1)$ for the algebraic category whose objects are vector spaces $M\in\vect{n+1}{+}$ with left Steenrod operations
\[\Sqv^i:M^{s_{n+1},\ldots,s_1}_t\to M^{s_{n+1}+1,s_n+i-1,2s_{n-1},\ldots,2s_1}_{2t+1},\]
which are zero except when $1\leq i \leq s_n$ and not all of $i-1,s_{n-1},\ldots,s_1$ are zero, and which
satisfy the homogeneous $\Sq$-Adem relations. Note that in an object of $\calMv(2)$, $\Sqv^1=0$.

In the present case, $n\geq1$, there is no disparity between the unstableness conditions on $\calw(n)$- and $\calu(n)$-cohomology, so that the analogue of propositions \ref{operations on goerss homology} and \ref{operations on untable P homology} is:
\begin{prop}
\label{vertical steenrod operations prop}
Suppose that $X\in s\calc$ is almost free, where $\calc$ stands for either $\calw(n)$ or $\calU(n)$ with $n\geq1$. Then the chain maps $\smash{\widetilde{\theta^i}}$ of Proposition \ref{general CohOpns given irreducibility} give $H^*_{\calc}X$ the structure of an object of $\calMv(n+1)$, natural in maps of almost free objects preserving the generating subspaces. Again, $\calMv(n+1)$ is the category of $\calU(n)$-$H^*$-algebras.
\end{prop}

In order to give a basis for a free object in $\calMv(n+1)$, for a sequence $I=(i_\ell,\ldots,i_1)$ of integers $i_j\geq1$, we define
\[\minDimSq(I):=\max\{(i_1),\,(i_2-i_1+1),\,(i_3-i_2-i_1+2),\,\ldots,\,(i_{\ell}-\cdots-i_1+(\ell-1))\}.
\]
Recall that $I$ is \emph{$\Sq$-admissible} if each $i_j\geq1$ and $i_{j+1}\geq 2i_j$ for $1\leq j <\ell$.
\begin{lem}
\label{basis of element of M(n+1)}
For $V\in\vect{n+1}{+}$ with homogeneous basis $B$, a basis of $F^{\calMv(n+1)}V$ is
\[\left\{\Sq\dver^Jb\ \middle|\ \genfrac{}{}{0pt}{}{b\in B^{s_{n+1},\ldots,s_1}_t\textup{, $J$ $\Sq$-admissible with }\minDimSq(J)\leq s_n,}{\textup{if }s_{n-1}\!=\!\cdots\!=\!s_1\!=\!0\textup{ then $J$ does not contain 1}}\right\}.\]
\end{lem}
Performing the same abuse of notation as in Corollary \ref{basis of element of M(0) in deg 0}:
\begin{cor}
\label{basis of element of M(n+1) in deg 0}
For $V\in\vect{n}{+}$ with homogeneous basis $B$, a basis of $F^{\calMv(n+1)}V\in\vect{n+1}{+}$ is
\[\left\{\Sq\dver^Jb\ \middle|\ \genfrac{}{}{0pt}{}{b\in B^{s_{n},\ldots,s_1}_t\textup{, $J$ $\Sq$-admissible with }\minDimSq(J)\leq s_n,}{\textup{if }s_{n-1}\!=\!\cdots\!=\!s_1\!=\!0\textup{ then $J$ does not contain 1}}\right\}.\]
\end{cor}

\SubsectionOrSection{Horizontal Steenrod operations and a product for $H^*_{\calw(n)}$}\label{Horizontal Steenrod operations and a product for HWn}
For any $n\geq 0$, we will construct operations on the homology $H^*_{\calw(n)}$ arising from the Lie structure.

Indeed, suppose that $X\in s\calw(n)$ is almost free. Then $Q^{\calU(n)}X\in s\calL(n)$ is also almost free, on essentially the same generating subspaces. Thus, the cohomotopy of $Q^{\calw(n)}X=Q^{\calL(n)}Q^{\calU(n)}X$ is an instance of simplicial partially restricted Lie algebra cohomology. Cohomology operations of this type are discussed in \S\ref{section: Cohomology operations for simplicial (restricted) Lie algebras} and Appendix \ref{appendix on Lie coh ops}. In the present context, we have two equivalent definitions, one using $\psi_{\calL(n)}$ and one using $\psi_{\calw(n)}$. Until Appendix \ref{appendix on Lie coh ops}, we will use $\psi_{\calw(n)}$, defining operations
\begin{gather*}
\Sqh^j:\left(H_{\calw(n)}^{n}(X)\overset{\ExtCohOp^{j-1}}{\to} \pi^{n+j-1}\dual S^2Q^{\calw(n)}X\overset{\psi_{\calw(n)}^*}{\to} H_{\calw(n)}^{n+j}(X)\right)\textup{ and}\\
\mu:\left(H_{\calw(n)}^{n_1}(X)\otimes H_{\calw(n)}^{n_2}(X)\overset{\ExtCohProd}{\to} \pi^{n_1+n_2}\dual S^2Q^{\calw(n)}X\overset{\psi_{\calw(n)}^*}{\to} H_{\calw(n)}^{n_1+n_2+1}(X)\right).
\end{gather*}
In more detail:
\begin{prop}
\label{Wn Halg omnibus}
Fix $n\geq0$. For $X\in s\calw(n)$, there are natural operations
\begin{gather*}
\Sqh^j:(H^*_{\calw(n)}X)_t^{s_{n+1},\ldots,s_1}\to (H^*_{\calw(n)}X)_{2t+1}^{s_{n+1}+j,2s_{n},\ldots,2s_1},\\
\mu:(H^*_{\calw(n)}X)_t^{s_{n+1},\ldots,s_1}\otimes (H^*_{\calw(n)}X)_q^{p_{n+1},\ldots,p_1}\to (H^*_{\calw(n)}X)_{t+q+1}^{s_{n+1}+p_{n+1}+1,s_{n}+p_{n},\ldots,s_1+p_1}
\end{gather*}
with the following properties
\begin{enumerate}
\item the product $\mu$ gives $H^*_{\calc}L$ the structure of a (grading shifted) $S(\CommOperad)$-algebra;
\item the squaring operation on $H^*_{\calc}L$ equals the \emph{top Steenrod operation}:
\[\Sq^{s_{n+1}+1}x=x^{2}\text{\ \ for $x\in (H^*_{\calw(n)}X)_t^{s_{n+1},\ldots,s_1}$};\]
\item \label{Sq unstable vanishing II} if $x\in (H^*_{\calw(n)}X)_t^{s_{n+1},\ldots,s_1}$, then $\Sq^ix=0$ unless $1\leq i\leq s_{n+1}+1$ and not all of $i-1,s_{n},\ldots,s_1$ equal zero;
\item \label{Sq unstable vanishing III} if $n=0$ then $\Sq^1\equiv 0$, and $\Sq^2x=0$ for $x\in (H^*_{\calw(0)}X)_t^{s_1}$ with $t\geq 2$;
\item every Steenrod operation is linear;
\item the \emph{Cartan formula} holds:  for all $x,y\in   H^*_{\calc}L$ and $i\geq0$,
\[\Sqh^i(xy)=\textstyle\sum_{k=1}^{i-1}(\Sq^kx)(\Sq^{i-k}y);\]
\item \label{yeah H is a Stmodule II} the  homogeneous $\Sq$-Adem relations hold, making $H^{*}_\calc L$ a left $\LieSteen$-module.
\end{enumerate}
\end{prop}
\begin{proof}
Almost everything here follows from Proposition \ref{the point of the appendix}, which demonstrates that the operations we are discussing here coincide with those defined on $H^*_{\bar{W}}$, as in Proposition \ref{all the lie steenrod ops are the same}. The same technique used to prove Proposition \ref{prop on sq1 on lie not rest} proves the new part of \emph{(\ref{Sq unstable vanishing II})}. For \emph{(\ref{Sq unstable vanishing III})},  when $n=0$, \emph{(\ref{Sq unstable vanishing II})} shows that $\Sq^1=0$. On the other hand, to see why $\Sq^2=0$ when $t\geq2$ is more difficult, especially since we have not determined the category of $\calw(0)$-$\Pi$-algebras. Nonetheless, as in the proof of Proposition \ref{prop on sq1 on lie not rest}, we will prove this for the universal example $\imath^s_t\in H^*_{\calw(0)}\mathbb{K}_s^{\calw(0),t}$. The reverse Adams spectral sequence (\S\ref{reverse Adams spectral sequence}) can be equipped with a quadratic grading if we view the generator of $\mathbb{K}_s^{\calw(0),t}$ as lying in quadratic grading one, and is of the form
\[\quadgrad{k}E_2^{p,q}=\quadgrad{k}(H^{p}_{\PA{\calw(0)}}\mathbb{K}_{0,s}^{\PA{\calw(0)},t})^{q}_{T}\implies \quadgrad{k}(H^{p+q}_{\calw(0)}\mathbb{K}_s^{\calw(0),t})_T.\]
As $\Sq^2\imath_t^s\in \quadgrad{2}(H^{s+2}_{\calw(0)}\mathbb{K}_s^{\calw(0),t})_{2t+1}$, we need to determine
\[
\quadgrad{2}(H^{2}_{\PA{\calw(0)}}\mathbb{K}_{0,s}^{\PA{\calw(0)},t})^{s}_{2t+1}\textup{ and }
\quadgrad{2}(H^{1}_{\PA{\calw(0)}}\mathbb{K}_{0,s}^{\PA{\calw(0)},t})^{s+1}_{2t+1},
\] 
and as in the proof of Proposition \ref{prop on sq1 on lie not rest}, we need to see how far $\Ftwo\{\imath^t_s\}$ is from being free in $\PA{\calw(0)}$. Fortunately, we only need to answer this question in quadratic grading two, and 
\[F^{\PA{\calw(0)}}(\Ftwo\{\imath^t_s\})=\pi_*(\mathbb{S}_s^{\calw(0),t})=\pi_*(F^{\calw(0)} \mathbb{K}^{t}_s).\]
Now if $t\geq2$, $\quadgrad{2}F^{\calw(0)}V$ naturally decomposes as 
\[\quadgrad{2}F^{\calw(0)}V=\quadgrad{2}F^{S(\LieOperad)}V\oplus P^2V\oplus\cdots \oplus P^{t-1}V,\]
and
we calculate, by Proposition \ref{homotopy of comm alg sphere}:
\[\quadgrad{2}F^{\PA{\calw(0)}}(\Ftwo\{\imath^t_s\})^{2t+1}=\Ftwo\{\lambda_2\imath^t_s,\lambda_3\imath^t_s,\ldots,\lambda_s\imath^t_s\}\]
 That is, there are two missing $\lambda$-operations, $\lambda_0$ and $\lambda_1$, in the functor $F^{S(\LieOperad)}$, and the presence of the operations $P^2,\ldots,P^{t-1}$ do not effect $\quadgrad{2}F^{\PA{\calw(0)}}(\Ftwo\{\imath^t_s\})$ in internal dimension $2t+1$. We now have enough information to proceed as in the proof of Proposition \ref{prop on sq1 on lie not rest}, since $\lambda_k\imath^t_s\in \quadgrad{2}F^{\PA{\calw(0)}}(\Ftwo\{\imath^t_s\})^{2t+1}_{s+k}$ for $2\leq k\leq s$.
\end{proof}



For $n\geq0$, write $\calMh(n+1)$ for the algebraic category whose objects are vector spaces $M\in\vect{n+1}{+}$ with left Steenrod operations and a commutative pairing satisfying the conditions of Proposition \ref{Wn Halg omnibus}. We have simply shown that $H^*_{\calw(n)}$ takes values in $\calMh(n+1)$ --- it is certainly not true that $\calMh(n+1)$ is the category of $\calw(n)$-$H^*$-algebras, as we have also seen that $H^*_{\calw(n)}$ takes values in $\calMv(n+1)$.

Note that the unstableness condition implies that $x^2=0$ whenever $x\in M_t^{0,\ldots,0}$. Indeed
\begin{prop}
\label{basis of free horizontal operations algebra}
Suppose that $n\geq1$. For $V\in\vect{n+1}{+}$ with homogeneous basis $B$. Then $F^{\calMh(n+1)}V$ is the quotient of the non-unital commutative algebra
\[S(\CommOperad) \left[\Sqh^Jb\ \middle|\ \genfrac{}{}{0pt}{}{b\in B^{s_{n+1},\ldots,s_1}_t\textup{, $J$ $\Sq$-admissible with }\excess(J)\leq s_{n+1},}{\textup{if }s_{n}=\cdots=s_1=0\textup{ then $J$ does not contain 1}}\right]\]
by the relation $b^2=0$ if $b\in B_t^{0,\ldots,0}$. Here, $e(J):=j_\ell-j_{\ell-1}-\cdots j_1$ is the Serre excess of $J$.
\end{prop}
\begin{proof}
By \citeBOX[6.1]{PriddySimplicialLie.pdf}, the true free object is a quotient of what we propose. It is in fact equal to what we propose, because the two-sided ideal in the homogeneous Steenrod algebra $\LieSteen$ generated by $\Sqh^1$ is spanned by those admissible sequences ending in $\Sqh^1$, so that forcing $\Sq^1_h=0$ in the relevant degrees has no unintended consequences. %Another way to express this fact is to say that left multiplication by $\Sq^1_h$ annihilates the augmentation ideal in $\LieSteen$.
\end{proof}
\begin{cor}
\label{basis of free horizontal operations algebra restricted}
Suppose that $n\geq1$. For $V\in\vect{n}{+}$ with homogeneous basis $B$. Then $F^{\calMh(n+1)}V\in\vect{n+1}{+}$ is the non-unital commutative algebra coproduct
\[S(\CommOperad) \left[b\ \middle|\ \genfrac{}{}{0pt}{}{b\in B^{s_{n},\ldots,s_1}_t}{s_{n},\ldots,s_1\textup{ not all zero}}\right]\sqcup\Lambda(\CommOperad) \left[b\ \middle|\ b\in B^{0,\ldots,0}_t\right].\]
\end{cor}
\SubsectionOrSection{Relations between the horizontal and vertical operations}
\label{Relations between the horizontal and vertical operations}
%With such a broad range of available operations, 
It will be helpful to be able to reduce %We are interested in calculating the composite $\delta_i\Sqh^jx$, or $\delta_i(xy)$, for $x,y\in H^*_{\calw(n)}X$ \textbf{but only for $n=0$}, so that 
expressions in the various available operations to a standard format, namely:
\[\prod_i \Sqh^{J_{i}}\deltav_{I_{i}}x_i\textup{ when $n=0$, or }\prod_i \Sqh^{J_{i}}\Sqv^{I_{i}}x_i\textup{ when $n\geq1$}.\]
This is possible, thanks to:
%\TODOCOMMENT{\tiny low dim cases}
\begin{prop}
\label{rearrange horiz and vert ops}
Suppose that $x,y\in H^*_{\smash{\calw(0)}}X$. If $\Sqh^jx\in (H^*_{\smash{\calw(0)}}X)^{s}_{t}$, then $\deltav_i\Sqh^{j}x=0$ for  $2\leq i<t$, and if $xy\in (H^*_{\smash{\calw(0)}}X)^{s}_{t}$, then $\deltav_i(xy)=0$ for  $2\leq i<t$.

Suppose that $n\geq1$ and $x,y\in H^*_{\smash{\calw(n)}}X$. Then $\Sqv^{i}\Sqh^{j}x=0$ and $\Sqv^i(xy)=0$.
%The expressions $\deltav_i\Sqh^jx$ and $\deltav_i(xy)$ for $x,y\in H^*_{\calw(0)}X$ are zero whenever defined. 
\end{prop}
\begin{proof}
For the case $n=0$, suppose that $X\in s\calw(0)$ is almost free on generating subspaces $V_s$. It is enough to prove that the composite
\[N_{s+1}(Q^{\calw(0)}X_{s+2})^{t+i+1}
\overset{\smash{\widetilde{\theta_i}}}{\to}
N_s(Q^{\calw(0)}X_{s+1})^{t}
\overset{{\psi_{\calw(n)}}}{\to}
N_{s-1}(S^2(Q^{\calw(0)}X_{s}))^{t-1}
\]
is nullhomotopic, using a similar method to that used in the proof of Proposition \ref{operations on goerss homology}. For any $V\in \vect{+}{0}$, there is a natural composite
\[(S_2V)^{t-1}\underset{\smash{\alpha}}{\overset{[\,,\,]}{\to}} (\quadgrad{2}F^{\calw(0)}V)^{t}\underset{\smash{\beta}}{\overset{P^i}{\mono}} (F^{(4)}_{\calw(0)}V)^{t+i+1},\]
whose maps we have labeled $\alpha$ and $\beta$ for convenience.
The map $\beta|_{\im(\alpha)}$ is not a monomorphism when $i=t-1$ is even, as in this case, for any $v\in V^{i/2}$,
\[P^i[v,v]=P^{i}P^{i/2}v=\textstyle\sum_{k=i/2+1}^{3i/2-2}\binom{2(k-i/2)-1}{ k-i/2}P^{3i/2-k}P^kv=0,\]
as each expression $P^kv$ in the sum vanishes by the unstableness condition.
However, the $\ker(\beta|_{\im(\alpha)})$ is contained in $\ker(\quadratic_{\calw(0)})$, 
and $\im(\beta\circ\alpha)$ does naturally split off as a direct summand of $(F^{(4)}_{\calw(0)}V)^{t+i+1}$.
 We write $h_i$ for the composite:
\[h_{i}:\Bigl((F^{\calw(0)}V)^{t+i+1}\overset{\textup{proj}}{\epi}
(\im(\beta\circ\alpha))^{t+i+1}\overset{\beta^{-1}}{\to}\frac{\im(\alpha)}{\ker(\beta)\cap\im(\alpha)}\overset{\quadratic_{\calw(0)}}{\to}(S^2V)^{t-1}\Bigr).\]
Identifying $Q^{\calw(0)}X_s$ with $V_s$ as in the proof of Proposition \ref{operations on goerss homology}, the nullhomotopy associated with the composite $\widetilde{h_i}:(V_{s+1}\overset{d_0}{\to}FV_s\overset{h_i}{\to}V_s)$ is the sum
\[(\epsilon d_0h_i+h_id_0\epsilon+h_id_0)d_0,\]
and the relation we seek will follow from the identity
\[h_id_0=\Bigl(\epsilon d_0h_i+h_id_0\epsilon+\quadratic_{\calw(0)} d_0\theta_i\Bigr):FV_{s+1}\to V_s,\]
as then $\psi_{\calw(0)}\smash{\widetilde{\theta}_i}=\quadratic_{\calw(0)} d_0\theta_id_0=d_0\widetilde{h_i}+\widetilde{h_i}d_0$. This identity states the following: if $V\in\vect{+}{0}$, and $f(g_k)\in F^{\calw(0)}F^{\calw(0)}V$ is a nested $\calw(0)$-expression in various expressions $g_k\in F^{\calw(0)}V$, then if we write $d_0:F^{\calw(0)}F^{\calw(0)}V\to F^{\calw(0)}V$ for the monad product map, there are only three ways that one may obtain summands of the form $P^i[v_1,v_2]$ in $d_0(f(g_k))\in (F^{\calw(0)}V)^{t+i+1}$: for some $k$, $g_k=P^i[v_1,v_2]$, and $f $ adds no further operations to this term; $f=P^i[g_{k_1},g_{k_2}]$, where $g_{k_1}=v_1$ and $g_{k_2}=v_2$ are unit expressions; or for some $k$, $g_k=[v_1,v_2]$, and  $f$ has $P^i(g_k)$ as a summand.

For the case $n\geq0$, the proof only becomes easier, the main difference being that in the corresponding composite:
\[(\quadgrad{2}F^{\call(n)})^{t-1}\underset{\smash{\alpha}}{{\to}} (\quadgrad{2}F^{\calw(0)}V)^{t}\underset{\smash{\beta}}{\overset{\lambda_{i-1}}{\mono}} (F^{(4)}_{\calw(0)}V)^{t+i-1},\]
both $\alpha$ and $\beta|_{\im(\alpha)}$ are monomorphisms.
\end{proof}

\SubsectionOrSection{The categories $\calMhv(n+1)$}
For $n\geq1$, let $\calMhv(n+1)$ be the following algebraic category, monadic over $\vect{n+1}{+}$. An object of $\calMhv(n+1)$ is a vector space $V\in \vect{n+1}{+}$ which is simultaneously an object of $\calMv(n+1)$ and of $\calMh(n+1)$, and in which for all $x,y\in V$:
\[\Sqv^i(xy)=0\textup{ and }\Sqv^i(\Sqh^j(x))=0.\]
%the following equations are satisfied. When $n=0$, we require, for all homogeneous elements $x,y\in V$ such that the following expressions are defined:
%\[\deltav_i(xy)=0\textup{ and }\deltav_i(\Sqh^j(x))=0.\]
%When $n\geq1$, we require, 
By Proposition \ref{rearrange horiz and vert ops}, for $n\geq1$ and $X\in\calw(n)$, $H^*_{\calw(n)}X$ is naturally an object of $\calMhv(n+1)$. The corresponding fact would \emph{not} be true for $n=0$, so we do not define a category $\calMhv(1)$.

For any $n\geq1$, the monad $F^{\calMhv(n+1)}$ factors as $F^{\calMh(n+1)}F^{\calMv(n+1)}$, with the evident distributive law of monads, and combining Corollary \ref{basis of element of M(n+1) in deg 0} and Proposition \ref{basis of free horizontal operations algebra}:
\begin{cor}
\label{calMhv(n+1) description}
For $n\geq1$ and $V\in\vect{n}{+}$ with homogeneous basis $B$,  $F^{\calMhv(n+1)}V$ is the quotient of the non-unital commutative algebra
%\[S(\CommOperad) \left[\Sqh^J\Sq\dver^Ib\ \middle|\ \genfrac{}{}{0pt}{}{b\in B^{s_{n},\ldots,s_1}_t\textup{, $I,J$ $\Sq$-admissible with }\minDimSq(I)\leq s_n,\ \excess(J)\leq \ell I}{\textup{if }s_{n-1}\!=\!\cdots\!=\!s_1\!=\!0\textup{ then $I$ does not contain 1}}\right].\]
%
%
%
\[S(\CommOperad) \left[\Sqh^J\Sq\dver^Ib\ \middle|\ \genfrac{}{}{0pt}{}{{\displaystyle\genfrac{}{}{0pt}{}{b\in B^{s_{n},\ldots,s_1}_t\textup{, $I,J$ $\Sq$-admissible with }\minDimSq(I)\leq s_n,\ \excess(J)\leq \ell I}{\textup{if }s_{n-1}\!=\!\cdots\!=\!s_1\!=\!0\textup{ then $I$ does not contain 1}}}}{
\textup{if }s_{n}\!=\!\cdots\!=\!s_1\!=\!0\textup{ then $J$ does not contain 1}
}
\right]\]
by the relation $b^2=0$ if $b\in B_t^{0,\ldots,0}$.
\end{cor}


Although we do not define $\calMhv(1)$, it will be useful to have a description of the composite $F^{\calMh(1)}F^{\calMv(1)}$.
Combining Corollary \ref{basis of element of M(0) in deg 0} and Proposition \ref{basis of free horizontal operations algebra}:

\begin{cor}
\label{calMhv(1) description}
For $V\in\vect{}{+}$ with homogeneous basis $B$, $F^{\calMh(1)}F^{\calMv(1)} V$ is isomorphic to the non-unital commutative algebra coproduct
\[S(\CommOperad) \left[\Sqh^J\deltav_Ib\ \middle|\ \genfrac{}{}{0pt}{}{b\in B_t\textup{, $I$ non-empty, $\delta$-admissible with }\minDimP(I)\leq t,}{\textup{$J$ $\Sq$-admissible with }\excess(J)\leq \ell I,\textup{ and }1\notin J}\right]\sqcup \Lambda(\CommOperad)\left[b\ \middle|\ b\in B\right].\]
For elements $b_1,\ldots,b_N$ of $B$ with $b_k\in B_{t_k}$ and appropriate sequences $I_k,J_k$, we have
\[\textstyle\prod_{k=1}^N \Sqh^{J_k}\deltav_{I_k}b_k\in \left(F^{\calMh(1)}F^{\calMv(1)}V\right)^{-1+\sum_k(nJ_k+\ell I_k+1)}_{-1+\sum_k(2^{\ell J_k+\ell I_k}(t_k+1))}.\]
\end{cor}



\SubsectionOrSection{Compressing sequences of Steenrod operations}
The following theorem creates a model for the convergence of a spectral sequence which we will discuss in \S\ref{Operations in composite functor spectral sequences}. One should think of $F^{\calMhv(n+1)}V$ as the $E_\infty$-page of  a first quadrant cohomotopy spectral sequence and $F^{\calMh(n)}V$ as the cohomotopy of the total complex.
\begin{thm}
\label{thm on compressing seqs of steenrod ops}
Suppose that $n\geq1$ and $V\in \vect{n}{+}$. Then there is a decreasing filtration on $F^{\calMh(n)}V$, the \emph{target filtration}, and an isomorphism
\[ f:(F^{\calMhv(n+1)}V)^{s_{n+1},\ldots,s_1}_t\overset{\cong}{\to} \Edown{0}{F^{\calMh(n)}V}{s_{n+1},\ldots,s_1}{t},\]
%E_0^{s_{n+1}}(F^{\calMh(n)}V)^{s_n,\ldots,s_1}_t,\]
defined by requiring that
$f(\Sqv^Iv)=\Sqh^Iv$ for $v\in V$, that $f(w_1w_2)=f(w_1)f(w_1)$ for $w_1,w_2\in F^{\calMhv(n+1)}V$,
and that
\[f(\Sqh^jw)=\Sqh^{j+s_n}f(w)\textup{ for }w\in (F^{\calMhv(n+1)}V)^{s_{n+1},\ldots,s_1}_t.\]
\end{thm}
\begin{proof}
The proposed map $f$ is not a well defined map to $F^{\calMh(n)}V$ since the Adem relations between the $\Sqh$ are not preserved by the proposed map $f$. Write $W(V)$ for the quotient of $S(\CommOperad)[\LieSteen\otimes F^{\calMv(n+1)}V]$ by the \emph{horizontal} unstableness relations and Cartan formula, so that $F^{\calMhv(n+1)}V$ is obtained from $W(V)$ by taking the quotient by the two-sided ideal generated by the \emph{horizontal} Adem relations. Then may define a map $\overline{f}:W(V)\to F^{\calMh(n)}V$ by requiring the same of $\overline{f}$ as of $f$. There is a decreasing filtration on $W(V)$, given by 
\[F^pW(V)=\bigoplus_{s_{n+1}\geq p}\bigoplus_{s_n,\ldots,s_1\geq0}\bigoplus_{t\geq1}W(V)^{s_{n+1},\ldots,s_1}_t\]
 and we define the \emph{target filtration} on the target by $F^p(F^{\calMh(n)}V):=\overline{f}(F^pW(V))$.

The map $\overline{f}$ fails to descend to a well-defined map  $F^{\calMhv(n+1)}V\to F^{\calMh(n)}V$, because it does not annihilate the Adem relations. However, we will show that it does send them into higher filtration, so that $\overline{f}$ induces a well defined map $f$ as advertised: if $w\in W(V)^{s_{n+1},\ldots,s_1}_t$ and $i<2j$, then
%\[f(\Sqh^i\Sqh^jw):=\Sqh^{i+2s_n}\Sqh^{j+s_n}f(w),\]
%and such an assignment does not preserve the Adem relations. Indeed, the Adem relation arising when $i<2j$:
%\[\Sqh^i\Sqh^jw-\sum_{k=0}^{\lfloor i/2\rfloor}{j-k-1\choose i-2k}\Sqh^{i+j-k}\Sqh^{k}w\]
%is sent to
%%\[\Sqh^{i+2s_n}\Sqh^{j+s_n}f(w)-\sum_{k=0}^{\lfloor i/2\rfloor}{j-k-1\choose i-2k}\Sqh^{i+j-k+2s_n}\Sqh^{k+s_n}f(w)\]
\begin{alignat*}{2}
\overline{f}\Bigl(\Sqh^i&\Sqh^jw-\textstyle\sum_{k=0}^{\lfloor i/2\rfloor}\binom{j-k-1}{ i-2k}\Sqh^{i+j-k}\Sqh^{k}w\Bigr)\\
{}:={}&\Sqh^{i+2s_n}\Sqh^{j+s_n}\overline{f}(w)-\textstyle\sum_{k=0}^{\lfloor i/2\rfloor}\binom{j-k-1}{ i-2k}\Sqh^{i+j-k+2s_n}\Sqh^{k+s_n}\overline{f}(w)\\
{}={}&
\Sqh^{i+2s_n}\Sqh^{j+s_n}\overline{f}(w)-\textstyle\sum_{k=s_n}^{\lfloor (i+2s_n)/2\rfloor}\binom{j+s_n-k-1}{ i+2s_n-2k}\Sqh^{(i+2s_n)+(j+s_n)-k}\Sqh^{k}\overline{f}(w)\\
{}={}&\textstyle\sum_{k=0}^{s_n-1}\binom{j+s_n-k-1}{ i+2s_n-2k}\Sqh^{(i+2s_n)+(j+s_n)-k}\Sqh^{k}\overline{f}(w)\\
{}=\makebox[0cm][r]{$:$}{}&\textstyle\sum_{k=0}^{s_n-1}\binom{j+s_n-k-1}{ i+2s_n-2k}\overline{f}(\Sqh^{i+j+2(s_n-k)+1}\Sqv^{k}w),
\end{alignat*}
which is in filtration $s_{n+1}+i+j+2(s_n+1-k)>s_{n+1}+i+j$ (the second equation holds by simply shifting the dummy variable $k$, the third by an Adem relation in the codomain).
%This quantity is non-zero, but is written as the image under $f$ of an element of filtration 

What remains is to show that $f$ is an isomorphism as in the theorem statement, which we approach simply by choosing a set of multiplicative generators for both the domain and codomain. The domain is generated by those expressions $\Sqh^I\Sqv^Jv$, for $v\in V^{s_{n},\ldots,s_1}_t$ running through a basis of $V$, and appropriate $\Sq$-admissible sequences $J$ and $I$. The codomain is generated by expressions $\Sqh^Kv$, for $v\in V^{s_{n},\ldots,s_1}_t$ running through a basis of $V$, and appropriate $\Sq$-admissible sequences $K$. It is a  combinatorial  exercise in the properties of admissible sequences to show that these sets of generators are put in bijection by $f$, and this bijection sends polynomial generators to polynomial generators and exterior generators to exterior generators.
\end{proof}
\end{Cohomology Operations for W and U}

\begin{Koszul complexes}

\SectionOrChapter{Koszul complexes calculating $\calU(n)$-homology}
\label{Koszul Cx section}\label{Koszul complexes}
We will now discuss the Koszul resolutions that one may use to calculate $H_*^{\calU(n)}X$ for $X$ a (non-simplicial)  object of $\calU(n)$ or $\calw(n)$ of finite type, using Priddy's technique \cite{PriddyKoszul.pdf}, adapted to an unstable context, as in \cite{CurtisSimplicialHtpy.pdf} and \cite[Chapter V]{MR1089001}.

\SubsectionOrSection{The Koszul complex and co-Koszul complex}
\label{The Koszul complex and co-Koszul complex}
Write $N_*^\div $ and $C_*$ for the chain complexes $N_*^\div Q^{\calU(n)}B^{\calU(n)}X$ and $C_* Q^{\calU(n)}B^{\calU(n)}X$.  We will use to the convenient bar notation after which the bar construction is named, c.f.\ \citeBOX[\S7]{grpsHPin.pdf}. Suppose that $n=0$, then the vector space  $N_s^\div$ is spanned by
\[\left[P^{i_{k_s+\cdots +k_1}}\cdots P^{i_{k_{s-1}+\cdots +k_1+1}}
\middle|\cdots 
\middle|P^{i_{k_2+k_1}}\cdots P^{i_{k_1+1}}
\middle|P^{i_{k_1}}\cdots P^{i_1}\right]
x,\]
where $x\in X$, the expressions in each of the $r$ spaces are $P$-admissible, and no bar is empty (so that $k_j>0$ for $1\leq j\leq s$). Such an expression represents an element of the repeated free construction $Q^{\calU(n)}B^{\calU(n)}_sX\cong (F^{\calU(0)})^{s}X$, with the requirement that no bar be empty reflecting having taken the quotient by degenerate simplices. In particular, this expression equals zero unless $\minDimP(i_{k_s+\cdots k_1},\ldots,i_1)\leq |x|$.

When $n\geq1$, the vector space $N_*^\div$ is spanned by expressions \[x\left[\lambda_{i_1}\cdots \lambda_{i_{k_1}} 
\middle|\lambda_{i_{k_1+1}}\cdots \lambda_{i_{k_2+k_1}}
\middle|\cdots\middle|\lambda_{i_{k_{s-1}+\cdots +k_1+1}}\cdots \lambda_{i_{k_s+\cdots +k_1}}\right],
\]
again without empty bars, and subject to an admissibility condition. However, these expressions are only defined when every $\lambda$-operation appearing is \emph{defined}. That is, if $x\in X^{t}_{s_n,\ldots,s_1}$, then we require $\minDimDelta(i_{k_s+\cdots k_1},\ldots,i_1)\leq s_n$, and for no $\lambda_0$ to appear if $s_{n-1}=\cdots =s_1=0$.

Each of these complexes admits an increasing filtration, the length filtration, with $F_\ell N^\div_s$ generated by those terms in which $i_{k_s+\cdots +k_1}\leq \ell$ for each term, which is to say that there are at most $\ell$ generators appearing in the $s$ free constructions in $C_s= Q^{\calU(n)}B^{\calU(n)}_sX\cong (F^{\calU(n)})^sX$. Note that $F_{s-1}N^\div_s=0$. %Note further that $d(F_sN_s^-)\subseteq F_{s-1}N_{s-1}^-$, which is easily checked on the isomorphic filtered complex $N_*^\div$.

Write $E^r_{\ell,s}$ for the spectral sequence of the filtered complex $N^\div_*X$, so that $E^0_{\ell,p}$ is the associated graded complex. As $F_{s-1} N^\div_s=0$, $E^r_{\ell,s}=0$ for $\ell<s$, and $E^0_{s,s}$ is the subspace $F_sN^\div_s$ of $N^\div$. %The $d_0$-differential vanishes on $E^0_{s,s}$ since $d(F_sN_s^-)\subseteq F_{s-1}N_{s-1}^-$ (\textbf{?x?x?x?x?}). We conclude that $E^1_{s,s}=F_s\Nop_s$, with $d^1$-differential $E^1_{s,s}\to E^1_{s-1,s-1}$ identified with the differential of $\Nop_*$. On the other hand, 
Priddy \cite[Proof of Theorem 5.3]{PriddyKoszul.pdf} shows that $E^1_{\ell,s}=0$  for $\ell>s$. Thus, the groups
\[K_s^{\calU(n)}X:=E^1_{s,s}, \textup{ equipped with $d^1:E^1_{s,s}\to E^1_{s-1,s-1}$}\]
form a subcomplex of $N^\div_*$, the \emph{Koszul complex}, whose inclusion is a homotopy equivalence. Note that $E^1_{ss}$ is the preimage of $F_{s-1}N^\div_{s-1}$ under
\[d:F_{s}N^\div_{s}\to F_{s}N^\div_{s-1}.\]


Rather than determining these groups directly, Priddy's theory works with their duals, $K^*_{\calU(n)}X$, which form a cochain complex with homology $H^*_{\calU(n)}X$. In fact, Priddy's theory shows that the cochain complex $K^*_{\calU(n)}X$, the \emph{co-Koszul complex}, is actually a differential unstable left module over the same operations as its cohomology $H^*_{\calU(n)}X$, and indeed that this (partial) module is \emph{free}. More precisely,  $K_0^{\calU(n)}X= X$, and $K^*_{\calU(n)}X$ is free on the  subspace $X^*$:
\begin{prop}
\label{the cokoszul complex is free}
Suppose that $n\geq0$, and $X$ is an object of $\calU(n)$ of finite type. The chain maps $\widetilde{\theta^i}$ when $n\geq1$ and $\widetilde{\theta_i}$ when $n=0$, on $C_*Q^{\calU(n)}B^{\calU(n)}X$ restrict to the subcomplex $K_*^{\calU(n)}X$, and induce an $\calMv(n+1)$-structure on $K^*_{\calU(n)}X$ which commutes with the differentials. The inclusion $\dual X\cong K^0_{\calU(n)}X\subseteq K^*_{\calU(n)}X$ induces an isomorphism $F^{\calMv(n+1)}(\dual X)\to K^*_{\calU(n)}X$. Moreover, this $\calMv(n+1)$-structure on $K^*_{\calU(n)}$ induces the $\calMv(n+1)$-structure on $H^*_{\calU(n)}X$ of propositions \ref{operations on untable P homology} and \ref{vertical steenrod operations prop}.
\end{prop}


%\SubsectionOrSection{The co-Koszul complex for $H_*^{\calU(0)}$}
Although it is easier to calculate the co-Koszul complex, we will need to understand the Koszul complex itself in order to calculate the $\calw(n+1)$-structure of $H_*^{\calU(n)}$. For this, we will introduce a little notation:
\begin{prop}
\label{propDerivedIndTrivialUobject n=0}
Suppose that $X\in\calU(0)$ has homogeneous basis $B$. Then $K_*^{\calU(0)}X$ has basis
\[\left\{\deltavstar_Ib\ \middle|\ b\in B^t\textup{, $I$ $\delta$-admissible with }\minDimP(I)\leq t\right\},\]
where we define, for any $x \in X^t$ and $I$ $\delta$-admissible with $\minDimP(I)\leq t$:
\[\deltavstar_Ix:=
%\cAnCel{\left[P^{i_\ell}\middle|\cdots\middle|P^{i_1} \right]x}+
\sum_{\produces{K}{I}{\deltaalg}}\left[P^{k_\ell} \middle|\cdots\middle|P^{k_1} \right]x\textup{ for $x\in X^t$}.\]
%Supposing that $X$ is loc fin, let $B^*\subseteq X^*$ represent the basis dual to $B$, and denote by $b^*\in B^*$ the element dual to $b\in B$. Then $K^*_{\calU(0)}X$ has basis $\{\delta_I(b^*)\}$, and this basis is dual to the basis just presented.
%
If $X$ is of finite type, there is a basis $\{\deltav_I(b^*)\}$ of $K^*_{\calU(0)}X$ constructed using Proposition \ref{the cokoszul complex is free}, Corollary \ref{basis of element of M(0) in deg 0} and the basis $\{b^*\}$ of $\dual X$ dual to $B$. The bases $\{\deltav_I(b^*)\}$ and $\{\deltavstar_Ib\}$ are dual.

The differential of $K^{\calU(0)}_*X$ is given by the formula:
\[d(\deltavstar_Ix)=\sum_{ \substack{\produces{K}{I}{\deltaalg}\\(k_{\ell},\ldots,k_2){\,\deltaalg\textup{-admis.}}}}\!\!\!\!\!\!\!\!\!\!\!\! \deltavstar_{(k_{\ell},\ldots,k_2)}(P^{k_1}x),\]
summing over those $K=(k_{\ell},\ldots,k_1)$ such that $(k_{\ell},\ldots,k_2)$ is $\delta$-admissible, and yet $\produces{K}{I}{\deltaalg}$.
\end{prop}
Note that the sum defining $\deltavstar_Ix$ is finite, simply because the $\Delta$-algebra is graded by the sum of indices.
We may assign $\deltavstar_Ix=0$ for $x\in X^t$ and $\minDimP(I)>t$, if we wish, since:
\begin{lem}
\label{lemOnAdemChangeIn minDimP}
If $\produces{I}{J}{\deltaalg}$, then $\minDimDelta(I)\geq\minDimDelta(J)$. This inequality is strict if $I$ and $J$ have length 2.
\end{lem}
\noindent In fact, this lemma ensures that the sum defining $\deltavstar_Ix$ is finite. That is, any $K$ with $\produces{K}{I}{\Delta}$ must have  $\minDimDelta(K)\geq\minDimDelta(I)$
 $\delta$-Adem relations only decrease $\minDimP$. Indeed, we may further restrict the two sums appearing in this proposition be requiring that $\minDimP(K)\leq t$ in each case, but there is no need. Dually, in the co-Koszul complex, the operations $\deltav_i$ are \emph{undefined} when out of range.
\begin{proof}[Proof of Proposition \ref{propDerivedIndTrivialUobject n=0}]
Firstly, we may assume that $X$ is  of finite type, as any object of $\calU(0)$ is the union of its  subobjects of finite type, and the functor $K_*^{\calU(0)}$ preserves unions. It is enough to check that $\deltavstar_Ib$ is in fact a member of $\Nop_*$, not just of $C_*$, as then the collection $\deltavstar_Ib$ will evidently be the dual basis to the $\deltav_I(b^*)$: in the sum defining $\deltavstar_Ib$, the only $\delta$-admissible sequence $K$ appearing is $K=I$.  % the basis element $\delta_J(b^*_2)\in K^*$ pairs non-trivially with $\delta(I,b_1)\in K_*$ if and only if $I=J$ and $b_1=b_2$. 

Using \cite[Lemma 3.2]{PriddyKoszul.pdf}, to check that $\deltavstar_Ib\in\Nop_*$, we only need to check that $d(\deltavstar_Ib)\in F_{s-1}C_{s-1}$.
%To check this membership condition is to check that $\deltavstar_Ib$ pairs to zero with $\im(d^*:\dual(F_sC_{s-1})\to\dual(F_sC_s))$. Priddy's proof shows that $\dual(F_sC_s)$ is spanned by functionals $[(P^{k_s})^*|\cdots |(P^{k_1})^*]f$, for $f\in \dual X$.  This functional evaluates to zero on $\deltavstar_Ib$ unless $f(b)=1$ and $\produces{K}{I}{\deltaalg}$. However, the image of $d^*$, as determined by Priddy, is spanned by the space of `$\delta$-Adem relations' (see \cite[Theorem 2.5 and proof]{PriddyKoszul.pdf}). These evaluate to zero on any $\deltavstar_Ib$, tautologically: in reducing a $\delta$-relation to a sum of $\delta$-admissible expressions using $\delta$-relations, one obtains zero.
To check this membership condition is to check that $\deltavstar_Ib$ pairs to zero with $\im(d^*:\dual(F_sN_{s-1})\to\dual(F_sN_s))$. Priddy's proof shows that $\dual(F_sN_s)$ is spanned by functionals $[(P^{k_s})^*|\cdots |(P^{k_1})^*]b^*$, which pair with the $\deltavstar_Ic$ according to: %Viewing $\delta(I,c)$ as an element of the double dual,
\[\left([(P^{k_s})^*|\cdots |(P^{k_1})^*]b^*\right)\left(\deltavstar_Ic\right)= b^*(c)\cdot\left(\textup{$\delta_I$ coeff.\ of $\delta_K\in\deltaalg$ written in admissibles}\right).\]
% $\delta(I,c)$ is the functional which sends, and this functional evaluates to zero on $\delta(I,b)$ unless $b=b_2$ and $\produces{K}{I}{\deltaalg}$. 
However, the image of $d^*$, as determined by Priddy, is spanned by the space of `$\delta$-Adem relations' (see \cite[Theorem 2.5 and proof]{PriddyKoszul.pdf}), and these tautologically evaluate to zero on any $\deltavstar_Ib$.
\end{proof}
The same analysis applies in the $n\geq1$ case. Although we write the bar construction on the right, we end up with a left action of the homogeneous Steenrod algebra, as the homogeneous Steenrod algebra is Koszul dual to the \emph{opposite} of the $\Lambda$-algebra with an index shift.
\begin{prop}
\label{propDerivedIndTrivialUobject n at least 1}
Suppose that $n\geq1$ and $X\in\calU(n)$ has homogeneous basis $B$. Then $K_*^{\calU(n)}X$ has basis
\[\left\{\Sqvstar{J}b\ \middle|\ \genfrac{}{}{0pt}{}{b\in B_{s_{n},\ldots,s_1}^t\textup{, $J$ $\Sq$-admissible with }\minDimSq(J)\leq s_n,}{\textup{if }s_{n-1}\!=\!\cdots\!=\!s_1\!=\!0\textup{ then $J$ does not contain 1}}\right\}.\]
where we define
\[\Sqvstar{J}b:=
%\cAnCel{\left[P^{i_\ell}\middle|\cdots\middle|P^{i_1} \right]b}+
\sum_{\produces{K}{J}{\Sq}}b\left[\lambda_{k_1-1} \middle|\cdots\middle|\lambda_{k_\ell-1} \right]\]
whenever $J$ and $b$ satisfy the conditions on $\minDimSq(J)$ and on the appearance of 1 in $J$.
If $X$ is  of finite type, this basis is dual to the $\{\Sqv^Jb^*\}$ basis of $K^*_{\calU(n)}X$   constructed using Proposition \ref{the cokoszul complex is free}, Corollary \ref{basis of element of M(n+1) in deg 0} and the basis $\{b^*\}$  of $\dual X$ dual to $B$. The differential of $K^{\calU(n)}_*X$ is given by the formula:
\[d(\Sqvstar{J}x)=\sum_{ \substack{\produces{K}{J}{\Sq}\\(k_{\ell},\ldots,k_2){\,\Sq\textup{-admis.}}}}\!\!\!\!\!\!\!\!\!\!\!\! \Sqvstar{(k_{\ell},\ldots,k_2)}(x\lambda_{k_1-1}),\]
summing over $K=(k_{\ell},\ldots,k_1)$ such that $(k_{\ell},\ldots,k_2)$ is $\Sq$-admissible %, $K$ does not contain 1 if $s_{n-1}=\cdots s_1=0$, 
and yet $\produces{K}{J}{\Sq}$.
\end{prop}
\noindent As part of the omitted analysis, we would use \ref{lemOnAdemChangeInMLambdaPlain}, and the fact that the $\Lambda$-algebra and the homogeneous Steenrod algebra are Koszul dual, to show:
\begin{lem}
If $I$ and $J$ are sequences of non-negative integers (of any length), such that $\produces{J}{I}{\Sq}$, then $\minDimSq(J)\leq\minDimSq(I)$, and if $1$ appears in $J$, it must also appear in $I$.
\end{lem}
\noindent Thus, all the terms appearing in the above definition of $\Sqvstar{J}x$ are indeed \emph{defined}.
%[\textbf{On well definedness of the term $\Sqvstar{J}b$:} {Search for }``Lemma \ref{lemOnAdemChangeInM} demonstrates that if $I$ and $J$ are sequences of'' in PRLieAlgs.
%In this case, the condition $\produces{K}{J}{\Sq}$ forces \textbf{[look above, ineq \& non-zero]}, so that every term appearing in the sum is automatically non-zero. On the other hand, the class $\Sqvstar{J}b$ is only defined under certain conditions. This adds up to \textbf{further evidence} that the Steenrod operations are all defined, just with some zero.]
\SubsectionOrSection{The $\calw(n+1)$-structure on $H_*^{\calU(n)}X$}\label{section on structure on homology of koszul cx}
Suppose that $X\in\calw(n)$ for some $n\geq0$. The form of the bases of $K_*^{\calU(n)}X$ given in propositions \ref{propDerivedIndTrivialUobject n=0} and \ref{propDerivedIndTrivialUobject n at least 1} imply:
\begin{cor}
\label{cycles in the Koszul complex are normalized cycles}
The Koszul complex $K_*^{\calU(n)}X$ is naturally a subcomplex of $\Nop Q^{\calU(n)}B^{\calU(n)}X$.
\end{cor}
\noindent  There are  thus two monomorphic quasi-isomorphisms of chain complexes with homology is $H_*^{\calU(n)}X$, and we denote their composite $\jmath$:
\[\jmath:\left(K_*^{\calU(n)}X\subseteq \Nop_*Q^{\calU(n)}B^{\calU(n)}X\subseteq \Nop_*Q^{\calU(n)}B^{\calw(n)}X\right).\]
The key upshot of Corollary \ref{cycles in the Koszul complex are normalized cycles} is that cycles in the Koszul complex map to \emph{normalized} cycles under $\jmath$.

Now $H^{\calU(n)}_*X$ is an object of $\calw(n+1)$, since it can be calculated as the homotopy of $Q^{\calU(n)}B^{\calw(n)}X\in s\calL(n)$, and this  structure will be needed for the composite functor spectral sequences discussed in \S\ref{Comp funct sseqs}. We will go some way to calculating this structure in this section. Our method will be to take cycles in the Koszul complex, map them into the large complex using $\jmath$, perform the operations in question, and then homotope the outcome back into the Koszul complex.

We will need a little notation for elements of the various bar constructions. We will label the $s+1$ free constructions in $B^{\calw(n)}_{s}X$ with subscripts in angle brackets: 
\[B^{\calw(n)}_{s}X= F^{\calw(n)}_{\langle -1\rangle}F^{\calw(n)}_{\langle 0\rangle}\cdots F^{\calw(n)}_{\langle s-1\rangle}X\]%corrected
so that we can then indicate in which free construction operations are being performed. For example, when $n=0$ and $x,y\in X$, $B_2^{\calw(0)}X$ contains an element
\[[P^i_{\langle 0\rangle}x,P^j_{\langle 1\rangle}y]_{\langle -1\rangle}:=[\eta P^i\eta^2 x,\eta^2P^j\eta y]\]%corrected
where we write $\eta:\Id\to F^{\calw(n)}$ for the unit of the monad on $\vect{+}{n}$ (omitting the forgetful functor). That is: we apply $P^j$, not to $y\in X$, but rather to $\eta y$, the corresponding generator of $F^{\calw(n)}_{\langle 1\rangle}X$; we  apply $P^i$ to $\eta^2 x$, a generator of $F^{\calw(n)}_{\langle 0\rangle}F^{\calw(n)}_{\langle 1\rangle}X$; the bracket is taken in the outermost free construction in $B_2^{\calw(0)}X:=F^{\calw(n)}_{\langle -1\rangle}F^{\calw(n)}_{\langle 0\rangle}F^{\calw(n)}_{\langle 1\rangle}X$.%corrected

With this notation in hand, the map $\jmath$ is induced by the assignment
\begin{alignat*}{2}
[P^{i_s}|\cdots |P^{i_1}]x&\longmapsto P_{\langle 0\rangle}^{i_s}P_{\langle 1\rangle}^{i_s-1}\cdots P_{\langle s-1\rangle}^{i_1}x &\quad&(\textup{if }n=0),\\
x[\lambda_{i_1}|\cdots |\lambda_{i_s}]&\longmapsto x\lambda_{i_1\langle s-1\rangle}\cdots \lambda_{i_{s-1}\langle1\rangle}\lambda_{i_s\langle 0\rangle} &\quad&(\textup{if }n\geq1).%both corrected
\end{alignat*}
Before making calculations, we recall the formulae of \citeBOX[\S8]{CurtisSimplicialHtpy.pdf} for the Lie algebra homotopy operations discussed in \S\ref{Homotopy operations for simplicial Lie algebras}. Let $\Shuffles{p}{q}$ be the set of $(p,q)$-shuffles, that is, pairs $(\alpha,\beta)$ where $\alpha=(\alpha_{p-1},\ldots,\alpha_0)$ and $\beta=(\beta_{q-1},\ldots,\beta_0)$ are disjoint monotonically decreasing sequences that together partition the set $\{0,\ldots,p+q-1\}$. Let $s_{\alpha}$ denote the iterated degeneracy operator $s_{\alpha_{p-1}}\cdots s_{\alpha_0}$.
Finally, let $\HalfShuffles{i}{i}$ denote the subset of $\Shuffles{i}{i}$ consisting of those shuffles $(\alpha,\beta)\in\Shuffles{i}{i}$ such that $\beta_{i-1}=2i-1$. The formulae of \citeBOX[\S8]{CurtisSimplicialHtpy.pdf}, for $z\in ZK_p(X)$ and $w\in ZK_q(X)$ cycles representing classes $\overline{z},\overline{w}\in H_*^{\calU(n)}X$, are as follows:
\begin{alignat*}{3}
[\overline{z},\overline{w}]\textup{ is represented by }&&\sum_{(\alpha,\beta)\in\Shuffles{p}{q}}[s_\beta(\jmath z), s_\alpha(\jmath w)]_{\langle -1\rangle}&\in Q^{\calU(n)}B^{\calw(n)}_{p+q}X;\\
\overline{z}\lambda_i\textup{ is represented by }&&\sum_{(\alpha,\beta)\in\HalfShuffles{i}{i}}[s_\beta(\jmath z), s_\alpha(\jmath z)]_{\langle -1\rangle}&\in Q^{\calU(n)}B^{\calw(n)}_{p+i}X,&\quad&(0<i\leq p);\\
\overline{z}\lambda_0\textup{ is represented by }&&\restnwithsubscript{(z)}{\langle -1\rangle}&\in Q^{\calU(n)}B^{\calw(n)}_{p}X,&&(\textup{when defined}).
\end{alignat*}%all three corrected
It will be important to understand these sums. Suppose that $z\in ZK_p^{\calU(n)}X$ (for $n\geq1$). Then  $z$ may be written as a sum of terms of the form $x\lambda_{i_1\langle p-1\rangle}\cdots \lambda_{i_p\langle 0\rangle}$, and
\begin{lem}
\label{what degens do to iterated operations}
If $(\alpha,\beta)\in\Shuffles{p}{q}$, then $s_\beta(x\lambda_{i_1\langle p-1\rangle}\cdots \lambda_{i_p\langle 0\rangle})=x\lambda_{i_1\langle \alpha_{p-1}\rangle}\cdots \lambda_{i_p\langle \alpha_0\rangle}$.
\end{lem}
We will also need the following consequence of the simplicial identities:

\begin{lem}
\label{LemmaOnSimplicialRelations}
Choose $i\geq1$ and $\alpha=(\alpha_{p-1},\ldots,\alpha_0)$ with $\alpha_{p-1}>\cdots >\alpha_0\geq0$.
\begin{enumerate}[i)]\squishlist
\setlength{\parindent}{.25in}
\item[i)] If neither $i-1$ nor $i-2$ appear in $\alpha$, then  $d_{i-1}s_\alpha=s_{\alpha'}d_{i'}$ for some $\alpha'$ and $i'$.
\item[ii)] If exactly one of $i-1$ and $i-2$ appears in $\alpha$, then  $d_{i-1}s_\alpha$ does not depend on which of $i-1$ and $i-2$ appeared in $\alpha$.
\end{enumerate}
\end{lem}

\begin{prop}
\label{LieBracketsTrivial}
The $\PA{\calL(n)}$ bracket $H_p^{\calU(n)}X\otimes H_q^{\calU(n)}X\to H_{p+q}^{\calU(n)}X$ vanishes except when $p=q=0$. 
The Lie algebra structure on $H_0^{\calU(n)}X$ is induced by that on $X$: if $z,w\in X$ represent $\overline{z},\overline{w}\in H_0^{\calU(n)}X$, then $[\overline{x},\overline{y}]$ is represented by the cycle $[x,y]\in ZC_0(Q^{\calU(n)}B^{\calw(n)}X)$.
\end{prop}
\noindent This theorem shows that $H_*^{\calU(X)}$ is trivial in positive dimensions \emph{as a Lie algebra}, but nonetheless, the \emph{restriction} need not be trivial (c.f.\ propositions \ref{QkTrivial} and \ref{Q0ZeroByPriddyAlg}).
\begin{proof}
We will give the proof for $n\geq1$, but it works the same way for $n=0$. In fact, when $n=0$ we can ignore all discussion of top and non-top operations.

Use the abbreviation $\mathbb{B}:=Q^{\calU(n)}B^{\calw(n)}X\in s\calL(n)$. Then $\mathbb{B}$ is almost free on the subspaces $V_s=F^{\calw(n)}_{\langle 0\rangle}\cdots F^{\calw(n)}_{\langle s-1\rangle}X$. Choose representatives $z\in ZK_p^{\calU(n)}X$ and $w\in ZK_q^{\calU(n)}X$. For any $(\alpha,\beta)\in\Shuffles{p}{q}$, the elements $s_{0}s_\beta(\jmath z)$ and $s_{0}s_\alpha(\jmath w)$ of $\mathbb{B}_{p+q+1}$ both lie in $V_{p+q+1}$, and it is only a minor abuse of notation to define:
\[a:=\sum_{(\alpha,\beta)\in\Shuffles{p}{q}}[s_{0}s_\beta(\jmath z), s_{0}s_\alpha(\jmath w)]_{\langle 0\rangle}\in C_{p+q+1}\mathbb{B}.\]
What we mean here is that the bracket of the elements $s_{0}s_\beta(\jmath z)$ and $s_{0}s_\alpha(\jmath w)$ of
\[ V_{p+q+1}=F^{\calw(n)}_{\langle 0\rangle}\cdots F^{\calw(n)}_{\langle s-1\rangle}X\subseteq F^{\calw(n)}_{\langle -1\rangle}F^{\calw(n)}_{\langle 0\rangle}\cdots F^{\calw(n)}_{\langle s-1\rangle}X=B^{\calw(n)}_{s}X\]
is taken in the free construction $F^{\calw(n)}_{\langle 0\rangle}$.

Using the simplicial identity $d_0s_0=\Id$, we have $d_{0}a=\sum [s_\beta(\jmath z), s_\alpha(\jmath w)]_{\langle -1\rangle}$, the representative given for $[\overline{z},\overline{w}]$. Moreover, we will find that $d_ia=0$ for $i>0$, except when $p=q=0$, in which case $d_1a=[x,y]$. Thus, in either case, $a$ is the required homotopy in $C_*\mathbb{B}$.

Using the simplicial identity $d_1s_0=\Id$, we have $d_{1}a=\sum [s_\beta(\jmath z), s_\alpha(\jmath w)]_{\langle 0\rangle}$. Now for every pair $(\alpha,\beta)$ indexing this sum, unless $p=q=0$, one of $\alpha$ or $\beta$, say $\beta$, will contain $0$. Then by Lemma \ref{what degens do to iterated operations}, every summand in $s_{\alpha}(\jmath z)$ is in the image of some \emph{non-top} $\lambda_{i\langle 0\rangle}$, and as $[x\lambda_i,y]=0$ whenever $\lambda_i$ is not a top operation, the entire expression vanishes in the construction $F^{\calw(n)}_{\langle 0\rangle}$.

What remains is to show that $d_{i}a=0$ for $2\leq i\leq p+q+1$. As $d_is_0=s_0d_{i-1}$ for $i\geq2$:
\[d_{i}a=\sum [s_{0}d_{i-1}s_\beta(\jmath z), s_{0}d_{i-1}s_\alpha(\jmath w)]_{\langle 0\rangle}.\]
For this, we will define an involution $\rho_i$ of the set $\Shuffles{p}{q}$ indexing the sum, for $2\leq i\leq p+q+1$.
If $\alpha$ and $\beta$ do not each contain exactly one of $i-1$ and $i-2$, then $\rho_i$ fixes $(\alpha,\beta)$. Otherwise, $\rho_i$ interchanges the positions of $i-1$ and $i-2$ in $(\alpha,\beta)$. To avoid confusion, we note that $\rho_{p+q+1}$ is the identity, as neither $\alpha$ nor $\beta$ ever contain $p+q$.

If $(\alpha,\beta)$ is a fixed point of $\rho_i$, then one of $\alpha$ and $\beta$, say $\alpha$, contains neither of $i$ and $i-1$. Then by Lemma \ref{LemmaOnSimplicialRelations}(i), $d_{i-1}s_\alpha(\jmath w)=s_{\alpha'}d_{i'}(\jmath w)=0$, as $\jmath w\in Z\Nop_*\mathbb{B}$. Thus, the summands corresponding to fixed points vanish.
On the other hand, given a shuffle $(\alpha,\beta)$ which is not fixed by $\rho_i$, Lemma \ref{LemmaOnSimplicialRelations}(ii) shows that the summand corresponding to $(\alpha,\beta)$ equals the summand corresponding to $\rho_i(\alpha,\beta)$, so these two summands cancel with each other.
\end{proof}

In order to state our calculation of $\lambda_0$ of $H^{\calU(0)}_*X$ for $X\in\calw(0)$, define
\[\aD{t}:=\left\{I\ \middle|\ \textup{$I$ a non-empty $\delta$-admissible sequence with }\minDimP(I)\leq t\right\}.\]
\begin{lem}
\label{lemma on the func TOPt}
There is an injective function $\TOP_t:\aD{t}\to \aD{t}$ given by
\[I=(i_{\ell},\ldots,i_1)\overset{\smash{\TOP_t}}{\longmapsto }(t+nI+\ell,i_\ell,\ldots,i_1).\]
\end{lem}
\begin{proof}
This is indeed a well defined injective endomorphism of the set $\aD{t}$, in that it preserves admissibility and the condition $\minDimP(I)\leq t$. The claim about $\minDimP(I)$ holds by definition. For $\delta$-admissibility, as $\minDimP(I)\leq t$,
\[i_{\ell}
\leq
\ell-1+i_{\ell-1}+\cdots +i_1+t\]
which (even) implies the (strict) inequality
\[2i_{\ell}< \ell+i_{\ell}+\cdots +i_1+t.\qedhere\]
%
%\begin{alignat*}{2}
%i_{\ell}
%&\leq
%\ell-1+i_{\ell-1}+\cdots +i_1+t&\qquad&\textup{(as $\minDimP(I)\leq t$)}%
%\\
%% Left hand side
%2i_{\ell}
%% Relation
%&\leq
%% Right hand side
%\ell-1+i_{\ell}+\cdots +i_1+t\\
%&< \ell+i_{\ell}+\cdots +i_1+t%
%\end{alignat*}
%as required. 
\end{proof}
%The strict inequality here is interesting: suppose that $I=(i_\ell,\ldots,i_1)\in \aD{t}$ and 
%\[(i_{j+1},\ldots,i_1)=\TOP_t(i_{j},\ldots,i_1)\textup{ where $1\leq j<i-1$.}\]
%Thanks to this strict inequality, it need not be true that $I=\TOP_t^{i-1-j}(i_{j+1},\ldots,i_1)$. %This is in contrast to the situation for standard unstable algebras over the Steenmrod algebra. Suppose that $x\in H^n(X;\Ftwo )$, for $X$ a topological space, and $I$ is a $\Sq$-admissible sequence with excess $e(I)\leq $
%Define:
%\[\aDirr{t}:= \aD{t}\setminus\im\bigl(\TOP_t:\aD{t}\to\aD{t}\bigr),\]
%the set of sequences in $\aD{t}$ not in the image of $\TOP_t$, so that we may decompose $\aD{t}$ as the disjoint union
%\[\aD{t}=\bigsqcup_{\smash{I\in \aDirr{t}}}\left\{I,\TOP_t I,\TOP_t^2I,\ldots\right\}.\]
\begin{prop}
\label{QkTrivial}
Suppose that $n\geq 0$, $z\in (ZK^{\calU(n)}_{s_{n+1}}X)_{s_n,\ldots,s_1}^t$ and $1\leq k\leq s_{n+1}$, so that $z\lambda_k$ is defined. Then $z\lambda_k=0$ unless $n=0$ and $k=1$.

When $n=0$, $k=1$ and $s_1\geq1$, $\lambda_1$ may be defined at the level of the Koszul complex as follows. The generic cycle $z\in (ZK_{s_1}^{\calU(0)}X)^{t}$  may be written as a sum
\[z=\textstyle\sum_{j}\deltavstar_{I_j}x_j, \text{\ with\ }x_j\in X^{t_j}\text{ and }I_j\in \aD{t_j}\text{ of length ${s_1}$}.\]
Then $\overline{z}\lambda_1$ is represented by the cycle
\[\textstyle\sum_{j} \deltavstar_{(\TOP_{t_j}I_j)}x_j\in (ZK^{\calU(0)}_{{s_1}+1}X)^{2t+1}.\]
\end{prop}
%\begin{prop}
%Suppose that $z\in (ZK^{\calU(n)}_{s_{n+1}}X)_{s_n,\ldots,s_1}^t$. If $n\geq1$, $\overline{z}\lambda_k=0$ for $1\leq k\leq {s_{n+1}}$. If $n=0$, writing $s=s_1$ we have $\overline{z}\lambda_k=0$ for $2\leq k\leq {s}$, and when $s\geq1$, $\lambda_1$ may be defined at the level of the Koszul complex as follows. The generic cycle $z\in (ZK_s^{\calU(0)}X)^{t}$  may be written as a sum
%\[z=\textstyle\sum_{j}\deltavstar_{I_j}x_j, \text{\ with\ }x_j\in X^{t_j}\text{ and }I_j\in \aD{t_j}\text{ of length $s$}.\]
%Then $v\lambda_1$ is represented by
%\[\textstyle\sum_{j} \deltavstar_{(\TOP_{t_j}I_j)}x_j\in (ZK^{\calU(0)}_{s+1}X)^{2t+1}.\]
%\end{prop}
\begin{proof}
We will first prepare for the calculation of $\lambda_1$ in case $n=0$, abbreviating $s_1$ to $s$.  Note that each $\TOP_{t_j}$ appends the same integer, $t$, to $I_j$.
Write $e$ for the proposed representative $\textstyle\sum_{j} \deltavstar_{(\TOP_{t_j}I_j)}x_j$ of $\overline{z}\lambda_1$. Our first claim is that $e=P^{t}_{\langle 0\rangle}s_0(\jmath z)$, since
\[\!\!\!\!\sum_{j,\,\produces{K}{(\TOP_{t_j}I_j)}{\deltaalg}}\!\!\!\!\left[P^{k_{s+1}} \middle|\cdots\middle|P^{k_1} \right]x_j
 =\!\!\!\!\sum_{j,\,\produces{H}{I_j}{\deltaalg}}\!\!\!\!\left[P^t\middle|P^{h_{s}} \middle|\cdots\middle|P^{h_1} \right]x_j.\]
The first of these two sums a priori contains more terms. However, the extra terms all vanish, by the unstableness condition.
%This $e$ does in fact lie in the Koszul complex: the sequences $(t,i_{s}^{(j)},\ldots,i_1^{(j)})$ are $\delta$-admissible, since
%\[t=(i_{s}+1)+|P(i_{s-1},\ldots,i_1)x^{(j)}|\geq2i_{s}+1,\]
%since we have demanded that $\minDimP(i_{s},\ldots,i_1)\leq|x^{(j)}|$. Then, as long as
%\[\jmath e:=\!\!\!\!\sum_{j,\,\produces{K}{(t,i_{s}^{(j)},\ldots,i_1^{(j)})}{\deltaalg}}\!\!\!\!\left[P^{k_{s+1}} \middle|\cdots\middle|P^{k_1} \right]x^{(j)}\in \Nop_{s+1}Q^{\calU(n)}B^{\calw(n)}X^{2t+1}\]
%$\jmath e$ in $\Nop_{s+1}Q^{\calU(n)}B^{\calw(n)}X^{2t+1}$ is homotopic to the standard representative of $\overline{z}\lambda_1$, the calculation for $\lambda_1$ is complete. This chain is given by the formula
%\[\!\!\!\!\sum_{j,\,\produces{K}{(t,i_{s}^{(j)},\ldots,i_1^{(j)})}{\deltaalg}}\!\!\!\!\left[P^{k_{s+1}} \middle|\cdots\middle|P^{k_1} \right]x^{(j)}
% =\!\!\!\!\sum_{j,\,\produces{H}{(i_{s}^{(j)},\ldots,i_1^{(j)})}{\deltaalg}}\!\!\!\!\left[P^t\middle|P^{h_{s}} \middle|\cdots\middle|P^{h_1} \right]x^{(j)}=P^{t}_{\langle 0\rangle}s_0(\jmath z),\]
More precisely:
%\[\produces{(k_{s_1+1},\ldots,k_1)}{(t,i_{s}^{(j)},\ldots,i_1^{(j)})}{\deltaalg}\textup{ iff }k_{s_1+1}=t\textup{ and }\produces{(k_{s_1},\ldots,k_1)}{(i_{s}^{(j)},\ldots,i_1^{(j)})}{\deltaalg}\]
if $\produces{(k_{s+1},\ldots,k_1)}{\TOP_{t_j}I_j}{\deltaalg}$ and $k_{s+1}\neq t$, then $\minDimP(k_{s+1},\ldots,k_1)>t_j$, so that $\left[P^{k_{s+1}} \middle|\cdots\middle|P^{k_1} \right]x_j=0$.
%\[\textup{if }\produces{(k_{s+1},\ldots,k_1)}{\TOP_{t_j}I_j}{\deltaalg}\textup{ and }k_{s+1}\neq t\textup{, then }\minDimP(k_{s+1},\ldots,k_1)>t_j,\textup{ so that }\left[P^{k_{s+1}} \middle|\cdots\middle|P^{k_1} \right]x_j=0\]
To understand this observation, as $\delta$-Adem relations cannot increase $\minDimP$ (Lemma \ref{lemOnAdemChangeIn minDimP}), we may reduce to the case where $(k_s,\ldots,k_1)$ is already $\delta$-admissible, $t\neq k_{s+1}$, and $\produces{(k_{s+1},k_{s})}{(t,k_{s+1}+k_{s}-t)}{\deltaalg}$, where::
\begin{alignat*}{2}
\minDimP(k_{s+1},\ldots, k_1)&\geq \minDimP(k_{s+1},k_s)-(k_{s-1}+1)-\cdots -(k_{1}+1)\\
&> \minDimP(t,k_{s+1}+k_s-t)-(k_{s-1}+1)-\cdots -(k_{1}+1)\\
&\geq 2t-(k_{s+1}+\cdots +k_1+s)\\
&= 2t-(t+i_{s}+\cdots +i_1+s)=t_j.
\end{alignat*}
where: the two non-strict inequalities are by definition of $\minDimP$; the strict inequality follows from Lemma \ref{lemOnAdemChangeIn minDimP}; the first equation holds as $\deltaalg$ is graded by the sum of the indices; and the second equation holds as $t$ is the dimension of $\deltavstar_{I_j}x_j$. %, so that the maximum $\minDimP(t,i^{(j)}_{s},\ldots)$ is attained at its final argument.

With this in hand, we return the general case, $1\leq k\leq p$ and $n\geq0$, our goal being to produce a nullhomotopy, except when $n=0$ and $k=1$, when we need a homotopy to $P^t_{\langle 0\rangle}s_0(\jmath z)$. We proceed as in the previous proof, defining
\[a:=\sum_{(\alpha,\beta)\in\HalfShuffles{k}{k}}[s_{0}s_\beta(\jmath z), s_{0}s_\alpha(\jmath z)]_{\langle 0\rangle}\in C_{p+k+1}\mathbb{B}_{2s_n,\ldots,2s_1}^{2t+1}.\]
Then $d_0a$ is the representative for $\overline{z}\lambda_k$, and $d_1a=0$ as in the previous proof (and there is no analogue here of the special case $p=q=0$). Now consider the same involutions $\rho_i$ as in the previous proof, now acting on $\Shuffles{k}{k}$. When $2\leq i< 2k$, $\rho_i$ preserves $\HalfShuffles{k}{k}$. When $2k<i\leq p+k+1$, $\rho_i$ is the identity, so preserves $\HalfShuffles{k}{k}$ trivially. Thus, $d_ia=0$ for all $2\leq i\leq p+k+1$ with $i\neq2k$, as the summands corresponding to fixed points vanish, and the cancellations still all occur within the smaller sum
\[d_{i}a=\sum_{(\alpha,\beta)\in\HalfShuffles{k}{k}} [s_{0}d_{i-1}s_\beta(\jmath z), s_{0}d_{i-1}s_\alpha(\jmath z)]_{\langle 0\rangle}.\]
To address the question of $d_{2k}$, we define an alternative involution $\widetilde{\rho}_{2k}$ of $\Shuffles{k}{k}$ as follows.  If $\alpha$ and $\beta$ do not each contain exactly one of $2k-2$ and $2k-1$, then $\widetilde{\rho}_{2k}$ fixes $(\alpha,\beta)$. Otherwise, we define $\widetilde{\rho}_{2k}(\alpha,\beta):=\rho_{2k}(\beta,\alpha)$, which is to say that $\widetilde{\rho}_{2k}$ swaps \emph{everything but} $2k-2$ and $2k-1$.

Now the summands in this formula exhibit a symmetry not present in the previous proof: $z$ is repeated. This symmetry, along with Lemma \ref{LemmaOnSimplicialRelations}(ii), shows that all the summands corresponding to shuffles not fixed by $\widetilde{\rho}_{2k}$ cancel out. When $k>1$, the fixed points of $\widetilde{\rho}_{2k}$ are only those shuffles in which one of $\alpha$ and $\beta$ contains neither $2k-2$ nor $2k-1$, and the corresponding summands vanish, by \ref{LemmaOnSimplicialRelations}(i), as in previous arguments. When $k=1$, however, $\widetilde{\rho}_{2k}$ has an \emph{extra} fixed point, the shuffle $((0),(1))$, which fails to differ from its image under $\widetilde{\rho}_{2k}$. In this case, then:
%\[d_2a=[s_{0}d_1s_1(\jmath z), s_{0}d_1s_0(\jmath z)]_{\langle 0\rangle}=[s_{0}(\jmath z), s_{0}(\jmath z)]_{\langle 0\rangle}\]
\begin{alignat*}{2}
d_2a
&=
[s_{0}d_1s_1(\jmath z), s_{0}d_1s_0(\jmath z)]_{\langle 0\rangle}%
\\
&=
[s_{0}(\jmath z), s_{0}(\jmath z)]_{\langle 0\rangle}\\
% Left hand side
% Relation
&=
% Right hand side
\begin{cases}
0,&\textup{if }n\geq1,\\
P^{t}_{\langle 0\rangle}s_0(\jmath z),&\textup{if }n=0.
\end{cases}% Comment
\end{alignat*}
That is, if $n\geq1$, this self-bracket vanishes (an object of $\calw(n)$ for $n\geq1$ is a Lie algebra), while if $n=0$, the self-bracket is equal to the top $P$-operation, in this case $P^t$.

In sum, we have shown that $d_0a=0$ represents $\overline{z}\lambda_i$, and that $d_ia=0$ whenever $1\leq i\leq p+k+1$, except when $k=1$, $i=2$ and $n=0$, in which case $d_2a=e$, as hoped.
\end{proof}
\begin{prop}
\label{Q0ZeroByPriddyAlg}
Suppose that $n\geq1$, and $z\in (ZK^{\calU(n)}_{s_{n+1}}X)_{s_n,\ldots,s_1}^t$ where not all of $s_n,\ldots,s_1$ equal zero. If $s_{n+1}=0$ then $\overline{z}\lambda_0$ is represented by $z\lambda_{s_n}\in X_{2s_n,\ldots,2s_1}^{2t+1}$. 

%Suppose instead that $s_{n+1}>0$,
%and write a cycle $z\in (ZK^{\calU(n)}_{s_{n+1}}X)_{s_n,\ldots,s_1}^t$  as
%\[z=\textstyle\sum_{j}\Sqvstar{I_j}x_j, \text{\ for various\ }x_j\in X\text{ and $\Sq$-admissible sequences }I_j=(i_{j,s_{n+1}},\ldots,i_{j,1}).\]

Suppose instead that $s_{n+1}>0$, and consider a cycle
\[z=\textstyle\sum_{j}\Sqvstar{I_j}x_j\in (ZK^{\calU(n)}_{s_{n+1}}X)_{s_n,\ldots,s_1}^t,\]
for various $x_j\in X$  and $\Sq$-admissible sequences $I_j=(i_{j,s_{n+1}},\ldots,i_{j,1})$.
Suppose further that for each summation index $j$, $x_j\lambda_{i-1}=0$ whenever $i\geq i_{j,1}$. Then $\overline{z}\lambda_0=0$.
\end{prop}
\begin{proof}
Write $p:=s_{n+1}$. The same homotopy $a$ as in the previous cases shows that $\widetilde{z}\lambda_0$ is represented by $\restnwithsubscript{(z)}{\langle 0\rangle}=z\lambda_{s_n\langle 0\rangle}$ when $p>0$, and by $z\lambda_{s_n}\in X$ when $p=0$, so that we may restrict to the case $p>0$. Then, $z\lambda_{s_n\langle 0\rangle}$ is the image of the following element  of $ZF_{p+1}N^\div_{p}Q^{\calU(n)}B^{\calU(n)}X$:
%\[E:=\sum_{j,\,\produces{K}{I_j}{\Sq}}x_j\left[\lambda_{k_1-1} \middle|\cdots\middle| \lambda_{k_{p-1}-1} \middle|\lambda_{k_{p}-1}\lambda_{s_n} \right]\]
\begin{alignat*}{2}
E
&=
\sum_{j,\,\produces{K}{I_j}{\Sq}}x_j\left[\lambda_{k_1-1} \middle|\cdots\middle| \lambda_{k_{p-1}-1} \middle|\lambda_{k_{p}-1}\lambda_{s_n} \right]%
\\
% Left hand side
% Relation
&=
% Right hand side
\sum_{j,\,\produces{K}{I_j}{\Sq}}\sum_{\produces{(\alpha,\beta)}{(s_n+1,k_p)}{\Sq}}x_j\left[\lambda_{k_1-1} \middle|\cdots\middle| \lambda_{k_{p-1}-1} \middle|\lambda_{\beta-1}\lambda_{\alpha-1} \right],%
\end{alignat*}
where the second equation holds by the Koszul duality of the $\Lambda$-algebra and  the homogeneous Steenrod algebra.
As homogeneous $\Sq$-Adem relations move $\Sq$-inadmissible sequences towards $\Sq$-admissibility, when $p\geq2$ we have $k_1\geq i_{j,1}$ in each summand, and when $p=1$ we have $\beta\geq i_{j,1}$ in each summand.

Dualizing Priddy's work, namely \cite[Proof of Theorem 5.3]{PriddyKoszul.pdf}, gives a sequence of homotopies which move this cycle into $F_pN_p^\div$. Indeed, given an expression
\[e=y\left[\lambda_{g_1-1} \middle|\cdots \middle|\lambda_{g_{r-2}-1}\middle|
\lambda_{g_{r-1}-1}\lambda_{g_r-1}\middle|\lambda_{g_{r+1}-1}\middle|\cdots\middle|\lambda_{g_{p+1}-1}\right]\in F_{p+1}N_p^\div,\]
%where $g_{r}-1\leq 2(g_{r-1}-1)$ 
(with the composite $\lambda_{g_{r-1}-1}\lambda_{g_r-1}$  $\Lambda$-admissible), define:
\[\Gamma(e):=\begin{cases}
y\left[\lambda_{g_1-1} \middle|\cdots \middle|
\lambda_{g_{r-1}-1}\middle|\lambda_{g_r-1}\middle|\cdots\middle|\lambda_{g_{p+1}-1}\right],&\textup{if }(g_{p+1},\ldots,g_{r})\textup{ is $\Sq$-admissible};\\
0,&\textup{otherwise}.
\end{cases}\]
If we further define $\Gamma$ to be zero on $F_pN_p^\div$, then $\Gamma:F_{p+1}N^\div_p\to F_{p+1}N^\div_{p+1}$ may be used as a chain homotopy to compress $E\in ZF_{p+1}N^\div_p$ into $ZF_pN^\div_p$:
\[(\Id+d\Gamma)^uE\textup{ stabilizes to an element of } ZF_pN_p^\div\textup{ as $u\to\infty$.}\]
As we repeatedly  apply $(\Id+d\Gamma)$ to this $e$, because $a_1\geq b_1$ whenever $\produces{(b_2,b_1)}{(a_2,a_1)}{\Lambda}$, the very leftmost $\lambda$-operation in any of the expressions that appear is $\lambda_{m-1}$ for some $m\geq g_1$, and every term in $(\Id+d\Gamma)^ue\in ZF_pN_p^\div$ will be of the form $y\lambda_{m-1}[{}\cdots{}]$ for some $m\geq g_1$.

Applying these observations in the very specific circumstances of this proposition, along with the earlier observation that in the sum defining the cycle $E$ we always have $k_1\geq i_{j,1}$  (or $\beta\geq i_{j,1}$ if $1$), one derives that $(\Id+d\Gamma)^uE=0$, so that $E$ is nullhomotopic. 
%
%
%Priddy's process returns a (large) sum of terms of the form
%\[x^{(j)}\lambda_{\kappa_0-1}\left[\lambda_{\kappa_1-1} \middle|\cdots\middle| \lambda_{\kappa_{p-1}-1} \middle|\lambda_{\kappa_p-1}\right]\textup{ where }\kappa_0\geq i_1^{(j)}.\]
%Under the very specific circumstances of this proposition, each of these terms vanishes.
\end{proof}
\end{Koszul complexes}


\begin{second quadrant homotopy sseq operations}

\SectionOrChapter{Operations on second quadrant homotopy spectral sequences}
\label{second quadrant homotopy sseq operations}

%\[\pi\uver_{t}( X)\overset{\delta_i^\textup{ext}}{\to} \pi\uver_{t+i}(S_2 X).\]
%On the other hand, the operation $\hExtCohOp^k$ equals the composite:
%\[\pi\dhor^s\pi\uver_t X
%\overset{\ExtCohOp^k}{\to} 
%\pi\dhor^{s+k}S_2(\pi\uver_t X)
%\overset{\pi\dhor^{s+k}(\widetilde{\nabla})}{\to}
%\pi\dhor^{s+k}\pi\uver_{2t}S_2 X
%\]

In this chapter we will produce various external operations on second quadrant homotopy spectral sequences. That is, for $X\in cs\vect{}{}$, we will produce numerous operations
from $\E{}{r}{X}{}{}$ to each of $\E{}{r}{S_2X}{}{}$ and $\E{}{r}{\Lambda^2 X}{}{}$. This approach leaves open a number of possibilities. If $X\in cs\algs$, then the structure map $S_2X\to X$ induces a spectral sequence map $\E{}{r}{S^2X}{}{}\to \E{}{r}{X}{}{}$, and so the external operations induce internal operations on $\E{}{r}{X}{}{}$. If $X$ is a Lie algebra  we may apply the analogous technique. 

In \S\ref{Operations on the Bousfield-Kan spectral sequence}, we will use these external operations in another way to produce operations on the \BKSS\ of a commutative algebra or Lie algebra --- the construction will involve a shift in filtration, which is conceivable given that  Radulescu-Banu's resolution  is a resolution by GEMs.




A number of authors have written on spectral sequence operations in a variety of settings. Singer's work \cite{MR2245560} on first quadrant cohomology spectral sequences has already been used in this thesis, and has been extended by Turner \cite{turner_opns_and_sseqs_I.pdf}. 
 Perhaps the closest recent examples are due to Hackney \cite{MR3019742} and \cite{MR3171258}, who works out the operations available on the homotopy spectral sequence of a cosimplicial $E_\infty$- or $E_n$-space respectively, using Bousfield-Kan's universal examples \cite{BK_pairings.pdf}. We will be working with cosimplicial simplicial \emph{vector spaces}, and so a direct approach, mirroring Dwyer's work in second quadrant cohomotopy spectral sequences \cite{DwyerHigherDividedSquares.pdf}, is available. 

Dwyer's work makes an interesting point of comparison with ours. In both cases: Steenrod operations and higher divided powers (as in \cite{DwyerHigherDividedSquares.pdf} and \S\ref{Homotopy operations for simplicial commutative algebras}) are produced on the spectral sequence; one set of operations is not present in homotopy;  the other set of operations is present in homotopy, but fewer operations are present there than at $E_2$; differentials are constructed between the two varieties to simultaneously rectify these disparities. Between Dwyer's theory and the theory presented here, the roles of the two types of operations are interchanged.

%%%
%%%first quadrant homotopy --- Stover's van Kampen\\
%%%first quadrant cohomology --- Serre sseq, our GSSeqs\\
%%%second quadrant homotopy --- Adams or Bousfield-Kan spectral sequences\\
%%%second quadrant cohomology --- Dwyer's work Eilenberg-Moore
%%%
%%%
%%%
%%%
%%%
%%%for example, Turner discusses 
%%%\TODOCOMMENT[inline]{
%%%Mention Dwyer's work on second quadrant cohomology spectral sequences for comparison; do some general throatclearing, including some content from the offcut: }
%%%\begin{shaded}\tiny
%%%We produce, for $X$ a cosimplicial simplicial vector space, an external commutative pairing
%%%\[\mu_\textup{ext}:(E_rX)^s_{t}\otimes (E_rX)_{t'}^{s'}\to (E_r(S_2X))^{s+s'}_{t+t'}\]
%%%and further operations:
%%%\[\delta_i^\textup{ext}:(E_rX)^s_{t}\to (E_r(S_2X))^{s}_{t+i}\textup{\qquad (defined only when $2\leq i<t-(r-2)$)}\]
%%%\[\Sq^j_\textup{ext}:(E_rX)^s_{t}\to (E_r(S_2X))^{s+j}_{2t}\textup{\qquad (defined up to indeterminacy)}\]
%%%Of course, if $X$ is a mixed simplicial algebra, then postcomposition with the product $X\otimes_{\Sigma_2}X\to X$ produces internal operations on the spectral sequence of $[E_rX]$. The resulting theory lives opposite Bill Dwyer's theory [?] of operations in the spectral sequence of a cosimplicial simplicial coalgebra. The two theories are not equivalent --- although linear dualization interchanges (finite type) algebras and coalgebras, the symmetry is broken by the choice of filtration direction. The operations $\mu_\textup{raw}$, $\delta_i^\textup{raw}$ and $\Sq^j_\textup{raw}$ may be of independent interest, and we will summarise their properties in \S\ref{The case of a cosimplicial simplicial commutative algebra}.
%%%
%%%In the present context, however, the operations of theorem XYZ are all equal to zero on $E_2$, simply because the Bousfield-Kan spectral sequence is derived from a cosimplicial object which is levelwise an Eilenberg-Mac Lane object. However, we will be able to use the external operations in section XYZ, shifting them one filtration higher in order to obtain non-trivial operations at on the Bousfield-Kan $E_2$.
%%%
%%%\textbf{similar para:} Let $X$ be a mixed simplicial commutative non-unital $\Ftwo $-algebra. I'd prefer to construct spectral sequence operations externally, as maps $E_rX\to E_r(X\otimes_{\Sigma_2}X)$, which may sometimes be multi-valued. I'll also assume that $X$ admits a coaugmentation $X^{-1}_\bullet\to X^\bullet_\bullet$ from a simplicial algebra $X^{-1}_\bullet$, and that this coaugmentation induces an isomorphism on total homology.
%%%\end{shaded}


\SubsectionOrSection{Maps of mixed simplicial vector spaces}

For mixed simplicial vector spaces $X,Y\in cs\vect{}{}$, we will write $C(X\otimes Y)$ for the double complex associated with the levelwise tensor product of $C(X\otimes Y)$, so that $C(X\otimes Y)_t^s=X_t^s\otimes Y_t^s$. We will write $C(X\otimes\dver Y)$ for the double complex with $C(X\otimes\dver Y)_t^s=\bigoplus_{t'+t''=t}X_{t'}^s\otimes X_{t''}^s$. The following vector space maps are given by prolonging $D^k$, $\Nabla$, $\Nabla_k$ and $\oldphi_k$ \emph{wherever these maps are defined}, and by zero elsewhere:
\begin{align*}
{D}^k:(CX\otimes CY)^{s+k}_{t}&\to C(X\otimes\dver Y)^{s}_{t}&\quad&\textup{(zero unless $0\leq k\leq s$)}\\
{\Nabla}:C(X\otimes\dver  Y)^{s}_{t}&\to C(X\otimes Y)^{s}_{t}\\
{\Nabla}_k:C(X\otimes\dver Y)^{s}_{t+k}&\to C(X\otimes Y)^{s}_{t}&\quad&\textup{(zero unless $0\leq k\leq t$)}\\
{\oldphi}_k:C(X\otimes\dver Y)^{s}_{t+k}&\to C(X\otimes Y)^{s}_{t}&\quad&\textup{(zero unless $k= t\geq0$)}
%{\rho}:C(X\otimes X)^{s}_{t}&\to C(S_2X)^{s}_{t}&\quad&\textup{(proj onto coinvariants)}
\end{align*}
We have essentially just committed to regarding $\Nabla_k$ as zero where it is not defined. This is certainly not a natural convention, and we hope that the ends justify the means. It results in such unpleasant statements as:
\begin{lem}
\label{unpleasant formula}
Suppose that $z\in C(X\otimes\dver Y)^{s}_{t}$. Then
\[(d\Nabla_k+\Nabla_kd)z=((1+\twist)\Nabla_{k+1}+\oldphi_{k+1} )z\]
whenever $k\geq0$ and $t$ does not equal either of $2k$ and $2k+1$.
\end{lem}
\noindent 

As discussed earlier, we will write $T$ for any symmetry isomorphism, write ``$\twist F$'' as shorthand for the function $TFT$, and  whenever we write $\twist FG$, we will mean $(\twist F)G$. We will also use the notation
\[X^{\otimes2}\overset{\rho}{\to}S_2X\textup{\ \ and\ \ }X^{\otimes2}\overset{\rho'}{\to}\Lambda^2
%\qquad X^{\otimes2}\overset{\rho''}{\to}S^2X,
\]
for the projection onto coinvariants and further onto the exterior quotient. Until \S\ref{External spectral sequence operations delta}, the operations that we will produce into each of $\Edownup{r}{}{S_2X}{}{}$ and $\Edownup{r}{}{\Lambda^2X}{}{}$ will be essentially the same.
\SubsectionOrSection{An external spectral sequence pairing $\mu_\mathrm{ext}$}
The easiest of our constructions is that of an external product, using 
the chain-level formula
\[x\otimes y\mapsto\rho\Nabla D^0(x\otimes y).\]
Indeed, both $\Nabla$ and $D^0$ are chain maps respecting filtration, and thus:
\begin{prop}
\label{prop on basic product}
The map $\rho\Nabla D^0(x\otimes y):CX\otimes CX\to CX$ induces a pairing
\[\mu_\textup{ext}:\Edownup{r}{}{X}{s}{t}\otimes \Edownup{r}{}{X}{s'}{t'}\to \Edownup{r}{}{S_2X}{s+s'}{t+t'}\]
for each $r$, satisfying the Liebniz formula. For $r\geq2$, this map descends to the symmetric quotient $S_2\Edownup{r}{}{X}{}{}$. Under the identifications $\Edownup{2}{}{X}{s}{t}\cong \pi\dhor^s\pi\uver_t X$ and $\Edownup{2}{}{S_2X}{s}{t}\cong \pi\dhor^s\pi\uver_t S_2X$, $\mu_\textup{ext}$ corresponds to the composite
\[S_2\pi\dhor^*\pi\uver_* X
\overset{\ExtCohProd}{\to} 
\pi\dhor^*(S_2\pi\uver_* X)
\overset{\pi\dhor^{*}(\widetilde{\nabla})}{\to}
\pi\dhor^{*}\pi\uver_{*}S_2 X.\]
\end{prop}
\noindent We will write $x\times y$ for $\mu_\textup{ext}(x\otimes y)$.
%\begin{shaded}\tiny
%\begin{proof}
%As both $\Nabla$ and $D^0$ are chain maps respecting filtration, proving that the operations are well defined.
%\end{proof}
%As both $\Nabla$ and $D^0$ are chain maps, $\mu_\textup{ext}$ satisfies the Liebniz formula. 
%
%As an example of the type of result one expects, we have the following. The homotopy of a simplicial algebra is exterior in positive dimension. If $X\in cs\algs$, then the algebra structure on $\E{}{2}{X}{}{}$ (induced by $\mu_\textup{ext}$ and the structure map $S_2X\to X$) need not be exterior, and yet we discover universal differentials hitting the external self-square:
%\begin{prop}
%Suppose that $X\in (s\vect{}{})^{\Delta_+}$ and $x\in \Edownup{r}{}{X}{s}{t}$ is a permanent cycle detecting a class $\overline{w}\in \pi_{t-s}((S_2X)^{-1})$ with $t-s>0$. Then $x\times x$ is eventually hit by a differential. That is, $x\times x\in \Edownup{r}{}{S_2X}{2s}{2t}$ is a permanent cycle which represents zero on $\Edownup{\infty}{}{S_2X}{}{}$.
%\end{prop}
%\begin{proof}
%Let $n=t-s$. As $x$ detects $w$, $x$ can be written as $x^0+dy$, where $y\in (TX)_{n+1}$ and $x^0=d\dhor^0w\in X^{0}_{n}$. Of course, $dx^0=0$. Now $x\times x$ is represented by
%\[\rho\Nabla D^0(x^0\otimes x^0)+\rho\Nabla D^0(dy\otimes x^0+x^0\otimes dy+dy\otimes dy).\]
%The first term here vanishes, as
%\[\rho\Nabla D^0(x^0\otimes x^0)=\rho(\oldphi_0 D^0(x^0\otimes x^0)+\Nabla_0 D^0(x^0\otimes x^0)+\Nabla_0 TD^0(x^0\otimes x^0)),\]
%and $\oldphi_0D^0(x^0\otimes x^0)=0$ as $n\geq1$, and $TD^0(x^0\otimes x^0)=D^0(x^0\otimes x^0)$ since $\{D^k\}$ is special. The remaining term may be expressed as the boundary
%\[d(\rho\Nabla(D^0(y\otimes dy)+D^1(dy\otimes x^0))).\qedhere\]
%\end{proof}
%\end{shaded}
\SubsectionOrSection{External spectral sequence operations $\Sq_\mathrm{ext}^i$}
\label{External spectral sequence operations Sq}
Consider the chain-level map:
\[\textup{SQ}^{i,s}:x\mapsto \rho\Nabla (D^{s-i}(x\otimes x)+D^{s-i+1}(x\otimes dx)).\]
We will use these maps to define external Steenrod squares, the behaviour of which is rather different on $E_1$ than on later pages. Thus, we will state two separate propositions that we will prove together.

\begin{prop}
\label{prop on e1 steens 1}
For $r\geq1$, the chain level operation $\textup{SQ}^{i,s}$ defines an \emph{external Steenrod operation} with indeterminacy vanishing by $E_{2r-2}$:
\[\Sq^i_\textup{ext}:\Edownup{r}{}{X}{s}{t}\to \Edownup{r}{}{S_2X}{s+i}{2t}.\]
Suppose that $r\geq 2$ and $x\in\Edownup{r}{}{X}{s}{t}$. $\Sq^i_\textup{ext}x=0$ unless $\min\{t,r\}\leq i\leq s$,  and this vanishing occurs without indeterminacy.
In any case, $\Sq^i_\textup{ext}x$ survives to $\Edownup{2r-1}{}{S_2X}{s+i}{2t}$, and the following equation in $\Edownup{2r-1}{}{S_2X}{s+i+2r-1}{2t+2r-2}$ holds (without indeterminacy): \[d_{2r-1}(\Sq^i_\textup{ext}x)=\Sq^{i+r-1}_\textup{ext}(d_rx).\]
%which is an equation in    neither of the two expressions here have any determinacy. %If $i<r$ and $t \geq i+1$ then $\Sq^i_\textup{ext}x=0$, and in fact $\Sq^i_\textup{ext}x$ may be taken to have no indeterminacy.

The \emph{top operation} $\Sq^s_\textup{ext}x$ is equal to the product-square $x\times x$, and in particular has no indeterminacy. 
As for the only potentially non-zero $\Sq^0_\textup{ext}$ operation:
\[\Sq^0_\textup{ext}:\Edownup{r}{}{X}{s}{0}\to \Edownup{r}{}{S_2X}{s}{0}\textup{ is induced by $X\overset{\textup{squaring}}{\to}S_2X$}.\]

At $E_2$, there is no indeterminacy, and the operation $\hExtCohOp^k$ corresponds to the composite:
\[\pi\dhor^s\pi\uver_t X
\overset{\ExtCohOp^i}{\to} 
\pi\dhor^{s+i}S_2(\pi\uver_t X)
\overset{\pi\dhor^{s+i}(\widetilde{\nabla})}{\to}
\pi\dhor^{s+i}\pi\uver_{2t}S_2 X.
\]

The condition $\min\{t,r\}\leq i\leq s$ may be replaced with $\min\{t+1,r\}\leq i\leq s$ after composing with $\Edownup{r}{}{S_2X}{s+i}{2t}\to \Edownup{r}{}{\Lambda^2X}{s+i}{2t}$.
\end{prop}

\begin{prop}
\label{prop on e1 steens 2}
%Fix $i,s\geq0$ and set $r=1$. The chain level operation $\textup{SQ}^{i,s}$ defines an 
At $E_1$, the external Steenrod operations have no indeterminacy, and $d_{1}$ commutes with $\Sq^i_\textup{ext}x $ for each $i$. Suppose that $x\in \Edownup{1}{}{X}{s}{t}$.
The top operation $\Sq^s_\textup{ext}x$ \emph{need not} equal the product-square $x\times x$ on $E_1$, and
$\Sq^{s+1}_\textup{ext}x$ \emph{need not} vanish, instead equalling $x\times d_1x$ on $E_1$.
At least for $i>s+1$, $\Sq^i_\textup{ext}x=0$. $\Sqh^1x$ is zero whenever $t\geq1$. $\Sqh^0x=0$ for all $t$.
\end{prop}








\begin{proof}[Proof of propositions \ref{prop on e1 steens 1} and \ref{prop on e1 steens 2}]
Choose a representative $x\in \EZdownup{r}{}{X}{s}{t}$ of the class of interest. We readily check that $\textup{SQ}^{i,s}x$ has filtration at least $s+i$:
\begin{align*}
\filt(\rho\Nabla (D^{s-i}(x\otimes x))&\geq s+s-(s-i)=s+i,\\
\filt(\rho\Nabla (D^{s-i+1}(x\otimes dx))&\geq s+(s+r)-(s-i+1)=s+i+(r-1).
\end{align*}
Thus, we may view $\textup{SQ}^{i,s}x$ as an element of $\EZdownup{0}{}{S_2X}{s+i}{2t}$. 
A straightforward calculation shows that \[d(\textup{SQ}^{i,s}x)=\rho\Nabla D^{s-i+1}(dx\otimes dx),\] so that, as $x\in \EZdownup{r}{}{X}{s}{t}$, we calculate:
\[\filt(\textup{SQ}^{i,s}x)\geq (s+r)+(s+r)-(s-i+1)=(s+i)+(2r-1),\]
so that $\EZdownup{2r-1}{}{S_2X}{s+i}{2t}$. This demonstrates the survival property, along with the formula commuting the $\Sq^i_\textup{ext}$ with spectral sequence differentials.

To examine the indeterminacy, as a representative of a class in $\Edownup{r}{}{X}{s}{t}$,  $x$ is only determined up to boundaries of $y\in \EZdownup{r-1}{}{X}{s-r+1}{t-r+2}  $ and elements of $\Edownup{r-1}{}{X}{s+1}{t+1}$. The latter are irrelevant, as their effect on the value of $\textup{SQ}^{i,s}$ is restricted to high filtration because $\{D^k\}$ is special.
We calculate, using analogues of \cite[(1.111) and (1.112)]{MR2245560}, that
\[\textup{BC}:=\rho\Nabla [D^{s-i-1}(y\otimes y)+D^{s-i}(y\otimes dy)+D^{s-i+1}(dy\otimes x)]\]
has boundary
\begin{alignat*}{2}
d(\textup{BC})
&=
\rho\Nabla [D^{s-i-1}(dy\otimes y+y\otimes dy)+D^{s-i}(dy\otimes dy)+D^{s-i+1}(dy\otimes dx)]%
\\
&\ \ \ +
\rho\Nabla [0+D^{s-i-1}(y\otimes dy+dy\otimes y)+D^{s-i}(dy\otimes x+x\otimes dy)]%
\\
&=
\rho\Nabla [D^{s-i}(dy\otimes x+x\otimes dy+dy\otimes dy)
+D^{s-i+1}(dy\otimes dx)]\\
&=\textup{SQ}^{i,s}x-\textup{SQ}^{i,s}(x+dy).%
\end{alignat*} 
That is, $\textup{BC}$ is a bounding chain for this difference, and
\[\filt(\textup{BC})\geq 2(s-r+1)-(s-i-1)=(s+i)-(2r-3),\]
so that $\Sq^i_\textup{ext}x$ has indeterminacy vanishing by $\Edownup{2r-2}{}{S_2X}{s+i}{2t}$ as claimed. When $i=s$, this result may be improved to $\filt(\textup{BC})\geq 2s-(r-1)$,  as in this case the lowest filtration summand in fact vanishes --- this is one explanation of why the top square has no indeterminacy.

When $i\geq s+2$, we have $\textup{SQ}^{i}x=0$, and even with $i=s+1$:
\[\textup{SQ}^{s+1,s}x=\rho\nabla D^{0}(x\otimes dx)\in F^{2s+r},\]
so that $\Sq^{s+1}_\textup{ext}x$ vanishes when $r\geq2$,  and $\Sq^{s+1}_\textup{ext}x=x\times d_1x$  when $r=1$, without indeterminacy in both cases. 

We must also check that $\Sq^i_\textup{ext}x$ vanishes (without indeterminacy) when $i<\min\{t,r\}$. For this we use the filtration preserving operations $\textup{DEL}_i$ to be defined in \S\ref{External spectral sequence operations delta}. Suppose   Proposition \ref{dvsDEL} states that
\[d(\textup{DEL}_{t-i+1}(x))+ \textup{DEL}_{t-i+1}(dx)=
\textup{SQ}^{i,s}(x)\]
as long as $2\leq t-i+1\leq t+1$ (which is satisfied whenever $i<t$). Moreover, if $i<r$, then 
\[ \textup{DEL}_{t-i+1}(dx)\in F^{s+r}\subset F^{s+i+1}\textup{ and }\textup{DEL}_{t-i+1}(x)\in F^{s},\]
so that this equation states that $\textup{SQ}^{i,s}(x)=0$ in $\E{}{r}{X}{s+i}{2t}$, without indeterminacy.


For the statement about the top operation, one calculates that
\[\textup{SQ}^{s,s}x-\rho\Nabla D^0(x\otimes x)=\rho\nabla D^{1}(x\otimes dx)\in F^{2s+r-1},\] which exceeds filtration $2s$ when $r\geq2$.



For the statement about $\Sq_\textup{ext}^0x$ when $t=0$, using the specialness assumption:
\[\textup{SQ}^{0,s}x-\rho\Nabla D^s(x\otimes x)=\rho\Nabla D^{s+1}(x\otimes dx) \in F^{s+1},\]
so that (using the assumption that $\{D^k\}$ is special):
\begin{alignat*}{2}
\textup{SQ}^{0,s}x
&\equiv
\rho\Nabla D^s(x^s_t\otimes x^s_t)&&\textup{(mod $F^{s+1}$)}%
\\
&\quad =
\rho(\oldphi_0+(1{+}\twist)\Nabla_0)(x^s_t\otimes\dver x^s_t)%
&\quad&\text{($x^s_t\otimes\dver x^s_t\in C(X\otimes\dver X)_{2t}^{s}$)}\\
&\quad =
\rho\oldphi_0(x^s_t\otimes\dver x^s_t)\in C(X\otimes_{\Sigma_2} X)_{t}^{s}.
\end{alignat*}

The statements about $\Edownup{r}{}{\Lambda^2X}{s+i}{2t}$ follow similarly, replacing  $\textup{DEL}_i$ with $\textup{LAM}_i$.\end{proof}
%%%%%
%%%%%Finally, $\Sq^1_\textup{ext}x\in\Edownup{2}{}{S_2X}{}{}$ vanishes for $x\in \Edownup{2}{}{X}{s}{t}$ with $t\geq2$ since it is the image of a $d_1$, namely $d_1(\delta^\textup{ext}_tx)$, where $\delta_t^\textup{ext}x$ will be defined in Proposition
%%%%%
%%%%%
%%%%%Proposition \ref{dvsDEL} shows that the $E_0$-operation $\delta_{t+1}^\textup{ext}$ provides a nullhomotopy of $\Sq_\textup{ext}^0$, but only in dimensions where it is defined.
%%%%%
%%%%%It may seem that there is an extra non-zero Steenrod operation coming from the chain level operation $\textup{SQ}^{s+1,s}$, on $E_r$, however, this operation lives as high as filtration $2s+r$ (which exceeds the expected $2s+1$ when $r\geq2$), and equals the operation $x\mapsto x\times d_rx$. Note that at $E_1$, this operation is in fact in the expected filtration.
%\begin{prop}
%The chain-level operation $\textup{SQ}^{s+1,s}$ defines an operation $\Sq^{s+1}_\textup{over}:E_r_{-s,t}(X)\to E_r_{-2s-r,2t+r-1}(X\otimes_{\Sigma_2}X)$, with indeterminacy vanishing by $E_{?x?x?}$. For $x\in E_r_{-s,t}(X)$, $\Sq^{s+1}_\textup{over}(x)$ survives to $E_{?x?x?}$, and moreover, $d_{r}(\Sq^{s+1}_\textup{over}x)=\Sq^{s+r}(d_rx)=d_rx\times d_rx$ modulo indeterminacy.
%\end{prop}
%\begin{proof}
%$\textup{Sq}^{s+1,s}x=\rho\Nabla D^0(x\otimes dx)\in F_{2s+r}$ has boundary $\rho\Nabla D^0(dx\otimes dx)\in F_{2s+2r}$, so we are getting a $d_r$.
%\end{proof}

\SubsectionOrSection{External spectral sequence operations $\delta^\mathrm{ext}_i$}
\label{External spectral sequence operations delta}
For any $k$, positive or otherwise, write $\mathbb{D}_k:(C(X)\otimes C(Y))_i\to C(X\otimes Y)_{i-k}$ for the map:
\[\mathbb{D}_r(z)= \sum_{\alpha-\beta=r}\Nabla_\alpha\twist^\alpha D^\beta(z).
%=\begin{cases}
%\sum_{j\geq0}\Nabla_{j}\twist^{j}D^{j-r},&\textup{if }r\leq0;\\
%\sum_{j\geq0}\Nabla_{j+r}\twist^{j+r}D^{j},&\textup{if }r\geq0.
%%\\,&\textup{if }
%\end{cases}
\]
\begin{lem}
\label{DkIsNiceToFiltration}
If $x\in F_s$ and $y\in F_{s'}$, then \[\mathbb{D}_k(x\otimes  y)\in F_s\cap F_{s'}\textup{ and } \mathbb{D}_k(x\otimes  y)\in F_{\lceil k/2\rceil}.\]
\end{lem}
\begin{proof}
We may assume that $x$ and $y$ are each homogeneous, with $x\in X^{s}_{t}$ and $y\in Y^{s'}_{t'}$. As $\{D^k\}$ is special, $D^{\beta}(x\otimes y)=0$ unless $\beta\leq \min\{s,s'\}$, in which case
\[\filt(D^{\beta}(x\otimes y))\geq s+s'-\beta\geq s+s'-\min\{s,s'\}=\max\{s,s'\}\geq \lceil (s+s')/2\rceil.\qedhere\]
\end{proof}
\begin{lem}
\label{boundaryVsBBD}
For all $k$, positive or otherwise, the equation:
\[(d\mathbb{D}_k+\mathbb{D}_kd)(z)= ((1+\twist)\mathbb{D}_{k+1}+\Nabla\twist D^{-k-1})(z)\]
holds when $z\in (CX\otimes CX)_{t-s}$ with $t-s>2(k+1)$. When $t-s= 2(k+1)$, 
\[(d\mathbb{D}_k+\mathbb{D}_kd)(z)= ((1+\twist)\mathbb{D}_{k+1}+\Nabla\twist D^{-k-1}+\textstyle\sum_{\alpha} \oldphi_{\alpha}\twist^{\alpha+1} D^{\alpha-k-1})(z).\]
\end{lem}
\begin{proof}
\newcommand{\twolinesum}[2]{\mathop{\sum_{\mathclap{#1}}}_{\mathclap{#2}}}
\newcommand{\onelinesum}[1]{\sum_{\mathclap{#1}}}
We may assume that $ z$ is homogeneous, $z\in (CX\otimes CX)_{t}^{s}$ with $n=t-s\geq2(k+1)$. Choose $\alpha$ and $\beta$ such that $\alpha-\beta=k$. Then $(\twist^\alpha D^\beta(z))\in C(X\otimes\dver Y)^{s-\beta}_{t}$.

We will need to apply Lemma \ref{unpleasant formula} to calculate, for $\alpha-\beta=k$:
\[(d\Nabla_\alpha+\Nabla_\alpha d)(\twist^\alpha D^\beta(z))=((1+\twist)\Nabla_{\alpha+1}+\oldphi_{\alpha+1} )(\twist^\alpha D^\beta(z)),\]
but Lemma \ref{unpleasant formula} does not apply when $t=2\alpha+e$ for $e\in\{0,1\}$. Fortunately, in that case $D^\beta(z)$ is zero, so the equation holds by default: after all, if $t=2\alpha+e$, our assumed inequality on $n$ implies:
\[\beta=\frac{t-e}{2}-k\geq \frac{t-e}{2}-\frac{t-s-2}{2}=\frac{s+2-e}{2}>\frac{s}{2}.\]

After these observations and under our conventions on the $\Nabla_\alpha$ and $D^\beta$, all but one of the following manipulations is totally formal:%. The only sticking point is that $d\Nabla_{-1}+\Nabla_{-1}d$ equals $0$, as opposed to $(1{+}\twist)\Nabla_0+\oldphi_0$ --- we rectify this anomaly by adding in the term $((1+\twist)\Nabla_{0}+\oldphi_{0})\twist D^{-k-1}(z)$:
%
% Now $D^\beta(z)$ is zero unless $\beta\leq 2s$, so that the term can be ignored unless $\alpha=\beta+k\leq 2s+k$. As $t\geq 2(2s+k+1)\geq 2(\alpha+1)$, we can apply the rule.
\begin{alignat*}{2}
(d\mathbb{D}_k+\mathbb{D}_kd)(z)
:={}&
\onelinesum{\alpha-\beta=k}\left(d\Nabla_\alpha\twist^\alpha D^\beta+\Nabla_\alpha\twist^\alpha D^\beta d\right)(z)%
\\
={}&
\onelinesum{\alpha-\beta=k}\left((d\Nabla_\alpha+
\Nabla_\alpha d)\twist^\alpha D^\beta+
\Nabla_\alpha\twist^\alpha (dD^\beta+
D^\beta d)\right)(z)%
\\
={}&
\onelinesum{\alpha-\beta=k,\ \alpha\geq0}((1{+}\twist)\Nabla_{\alpha+1}+\oldphi_{\alpha+1})\twist^\alpha D^\beta(z)+ \onelinesum{\alpha-\beta=k} \Nabla_\alpha\twist^\alpha(1{+}\twist) D^{\beta-1}(z)%
\\
={}&
\onelinesum{\alpha-\beta=k+1,\ \alpha\geq1}((1{+}\twist)\Nabla_{\alpha}+\oldphi_{\alpha})\twist^{\alpha-1} D^\beta(z)+ \onelinesum{\alpha-\beta=k+1} \Nabla_\alpha\twist^\alpha(1{+}\twist) D^{\beta}(z)%
%\\
%={}&
%\onelinesum{\alpha-\beta=k+1} \left(((1+\twist)\Nabla_{\alpha}+ \phi_{\alpha})\twist\twist^{\alpha} D^\beta+\Nabla_\alpha(1+\twist)\twist^\alpha D^{\beta}\right)(z) + {}%
%\\
%&\ \ \ \ \ \ \ \ \ \ \ +
%((1+\twist)\Nabla_{0}+\phi_{0})\twist D^{-k-1}(z)
\end{alignat*}
Using the identity $(1+\twist)\Nabla_{0}+\oldphi_{0}=\Nabla$ for the first equation, and the observation that $(1+\twist)F\twist G+F(1+\twist)G=(1+\twist)(FG)$  for the second (with $F=\Nabla_\alpha$ and $G=\twist^\alpha D^\beta$):
%\[(d\mathbb{D}_k+\mathbb{D}_kd)(z)-\Nabla\twist D^{-k-1}(z)=\onelinesum{\alpha-\beta=k+1} \left(((1+\twist)\Nabla_{\alpha}+ \phi_{\alpha})\twist\twist^{\alpha} D^\beta+\Nabla_\alpha(1+\twist)\twist^{\alpha}D^{\beta}\right)(z)\]
\begin{alignat*}{2}
(d\mathbb{D}_k+\mathbb{D}_kd)(z)-\Nabla\twist D^{-k-1}(z)
&=
\onelinesum{\alpha-\beta=k+1} \left(((1+\twist)\Nabla_{\alpha}+ \oldphi_{\alpha})\twist\twist^{\alpha} D^\beta+\Nabla_\alpha(1+\twist)\twist^{\alpha}D^{\beta}\right)(z)%
\\
&=
\onelinesum{\alpha-\beta=k+1} \left((1+\twist)(\Nabla_\alpha\twist^\alpha D^\beta)+ \oldphi_{\alpha}\twist^{\alpha+1} D^\beta\right)(z)%
\\
% Left hand side
% Relation
&=
% Right hand side
(1+\twist)\mathbb{D}_{k+1}(z)+\onelinesum{\alpha} \textstyle\oldphi_{\alpha}\twist^{\alpha+1} D^{\alpha-k-1}(z)%
\end{alignat*}
When the strict inequality $t-s>2(k+1)$ holds, due to the application of $\oldphi_{\alpha}$, each summand $\oldphi_{\alpha}\twist^{\alpha+1} D^{\alpha-k-1}(z)$ is zero unless $t=2\alpha$, but in that case, $s=2\alpha-n<2(\alpha-k-1)$, and then $D^{\alpha-k-1}(z)$ vanishes as $\{D^k\}$ is special.
%
%On the other hand, $(1+\twist)F\twist G+F(1+\twist)G=(1+\twist)(FG)$ holds for any $F$ and $G$, and in particular for $F=\Nabla_\alpha$ and $G=\twist^\alpha D^\beta$, this quantity differs from
%\[(1+\twist)\mathbb{D}_{k+1}(z)=\sum_{\alpha-\beta=k+1}(1+\twist)\Nabla_\alpha\twist^\alpha D^\beta(z)\]
%only by the terms
%\[\sum_{\mathclap{\alpha-\beta=k+1}} \phi_{\alpha}\twist^{\alpha-1}D^{\beta}(z)=\sum_{\mathclap{\alpha-\beta=k+1}} \phi_{\alpha}\twist^{\alpha-1}D^{\alpha-k-1}(z_{2\alpha}^{2\alpha-n})\]
%but the cosimplicial degree $2\alpha-n$ is less than $2(\alpha-k-1)$, and $\{D^k\}$ is special, so that each term $D^{\alpha-k-1}(z_{2\alpha}^{2\alpha-n})$ vanishes.
\end{proof}

















We will be able to define (sometimes multi-valued) operations (for $r\geq 0$)
\begin{gather*}
\delta^\textup{ext}_i:\Edownup{r}{}{X}{s}{t}\to \Edownup{r}{}{S_2X}{s}{t+i}\textup{\ \ \ for $2\leq i\leq\max\{n,t-(r-1)\}$}\\
\lambda^\textup{ext}_i:\Edownup{r}{}{X}{s}{t}\to \Edownup{r}{}{\Lambda^2X}{s}{t+i}\textup{\ \ \ for $1\leq i\leq\max\{n,t-(r-1)\}$}
\end{gather*}
using the chain-level maps $\textup{DEL}_{i}:C_*X\to C_*S_2X$ and $\textup{LAM}_{i}:C_*X\to C_*\Lambda^2X$:
%\[x\mapsto \textup{DEL}_{i}(x):=\rho(\mathbb{D}_{n-i}(x\otimes x)+\mathbb{D}_{n-i-1}(dx\otimes x)),\]
%\[x\mapsto \textup{LAM}_{i}(x):=\rho'(\mathbb{D}_{n-i}(x\otimes x)+\mathbb{D}_{n-i-1}(dx\otimes x)),\]
\begin{alignat*}{2}
% Left hand side
\textup{DEL}_{i}(x)&:=\makebox[\widthof{$\rho'$}][l]{$\rho$}(\mathbb{D}_{n-i}(x\otimes x)+\mathbb{D}_{n-i-1}(dx\otimes x)),\\
\textup{LAM}_{i}(x)&:=\makebox[\widthof{$\rho'$}][l]{$\rho'$}(\mathbb{D}_{n-i}(x\otimes x)+\mathbb{D}_{n-i-1}(dx\otimes x)),
\end{alignat*}
where we write $n:=t-s$ in each formula. Except when $i<2$, we  can work just with $\textup{DEL}$, as in \S\ref{External unary homotopy operations}. Lemma \ref{DkIsNiceToFiltration} shows immediately that these maps  preserve filtration. 
To proceed, we will need a formula for the boundary of $\textup{DEL}_i(x)$:
\begin{prop}
\label{dvsDEL}
For $2\leq i\leq t+1$ and $x\in \EZdownup{0}{}{X}{s}{t}$:
\[d(\textup{DEL}_i(x))+ \textup{DEL}_i(dx)=\textup{SQ}^{t-i+1,s}(x)=\begin{cases}
\textup{SQ}^{t-i+1,s}(x),&\textup{if }n+1\leq i \leq t+1;\\
\rho\Nabla D^0(x\otimes dx),&\textup{if }i=n;\\
0,&\textup{if }i< n.
\end{cases}\]

The same equations hold for $\textup{LAM}_i$ in the extended range $1\leq i\leq t+1$.
\end{prop}
\begin{proof}
We may apply Lemma \ref{boundaryVsBBD} to calculate $d\mathbb{D}_{n-i}(x\otimes x)$ and $d\mathbb{D}_{n-i-1}(dx\otimes x)$, since 
\[|x\otimes x|=2n> 2(n-i+1)\textup{ \ and \ }|dx\otimes x|=2n-1> 2(n-i-1+1) \textup{ \ when $i\geq2$}.\]
Note that the first inequality is an equation when  $i=1$, which will explain the lack of $\delta^\textup{ext}_1$. We can work around this difficulty when defining $\lambda^\textup{ext}_1$, but the reversal of the second inequality when $i=0$ is problematic for the definition of $\lambda^\textup{ext}_0$. Returning to $\textup{DEL}_i$ for $i\geq2$:
\begin{alignat*}{2}
d(\textup{DEL}_i(x))+{}& \textup{DEL}_i(dx)
=
\smash{\rho d\Bigl(\mathbb{D}_{n-i}(x\otimes x)+\mathbb{D}_{n-i-1}(dx\otimes x)\Bigr)}+\rho\mathbb{D}_{n-i-1}(dx\otimes dx)%
\\
&=
\rho\Bigl\{d\mathbb{D}_{n-i}(x\otimes x)\Bigr\}+\rho\Bigl\{d\mathbb{D}_{n-i-1}(dx\otimes x))+\mathbb{D}_{n-i-1}(d(dx\otimes x))\Bigr\}%
\\
&=
\rho\Bigl\{\mathbb{D}_{n-i}d(x\otimes x)+(1+\twist)\mathbb{D}_{n-i+1}(x\otimes x)+\Nabla\twist D^{i-n-1}(x\otimes x)\Bigr\}
\\
&\ \ \ \ +\rho\Bigl\{(1+\twist)\mathbb{D}_{n-i}(dx\otimes x)+\Nabla\twist D^{i-n}(dx\otimes x)\Bigr\},
\end{alignat*}
where we used braces to indicate the two applications of Lemma \ref{boundaryVsBBD}.
Everything cancels except for $\rho\Nabla(D^{i-n-1}(x\otimes x)+D^{i-n}(x\otimes dx))$ which equals $\textup{SQ}^{t-i+1,s}(x)$. We have studied this expression above, explaining the three cases.

If $i=1$, Lemma \ref{boundaryVsBBD} yields an extra term,
%\[\textstyle\sum_{\alpha} \oldphi_{\alpha}\twist^{\alpha+1} D^{\alpha-n}(x\otimes x),\]
and if we write $x$ as the sum $\sum_T x_T^{T-n}$ of its homogeneous parts, as $\{D^k\}$ is special, this term is:
\[\rho(\textstyle\sum_{\alpha} \oldphi_{\alpha}\twist^{\alpha+1} D^{\alpha-n}(x\otimes x))=\rho(\textstyle\sum_{T} x_T^{T-n}\otimes x_T^{T-n})\in S_2X.\]
Although this term need not vanish, its image in $\Lambda^2X$ certainly does, so that $\textup{LAM}_1$ satisfies the desired equation.
\end{proof}
Suppose now that $x\in \Edownup{r}{}{X}{s}{t}$. In light of the above calculation, when $n<i\leq t+1$, the purpose of $\delta^\textup{ext}_i(x)$ will be to support a $d_{t-i+1}$-differential to $\Sq_\textup{ext}^{t-i+1}(x)$. Thus, we would not expect to be able to define $\delta^\textup{ext}_i(x)$ when $t-i+1<r$; indeed, the following result will construct $\delta^\textup{ext}_i(x)$ whenever $i\leq t-(r-1)$. Moreover, the Steenrod operation $\Sq^{t-i+1}(x)$ has indeterminacy vanishing by $\Edownup{2(r-1)}{}{S_2X}{s+t-i+1}{2t}$, and one should expect that whenever $t-i+1<2(r-1)$, $\delta^\textup{ext}_i(x)$ will be multi-valued, but that the set of values for $\delta^\textup{ext}_i(x)$ will map onto the set of values for $\Sq_\textup{ext}^{t-i+1}(x)$ under $d_{t-i+1}$. 
We are not saying that the indeterminacy of $\delta^\textup{ext}_i(x)$ must vanish by a certain page, but rather that one expects the  multiple values of $\delta^\textup{ext}_i(x)$ to all fail to be permanent cycles together. Note that when $r\leq 2$, there is no indeterminacy whatsoever in either set of operations.
\begin{prop}
\label{Prop on delta external}
Suppose $r\geq0$. The chain-level map $\textup{DEL}_i$ produces potentially multi-valued operations $\delta^\textup{ext}_i:\Edownup{r}{}{X}{s}{t}\to \Edownup{r}{}{S_2X}{s}{t+i}$ whenever $2\leq i\leq \max\{n,t-(r-1)\}$. This operation is single-valued whenever $2\leq i\leq \max\{n,t+1-2(r-1)\}$, and at $E_1$ may be identified with the operations of \S\ref{External unary homotopy operations}:
\[\pi\uver_{t}( X^s)\overset{\delta_i^\textup{ext}}{\to} \pi\uver_{t+i}S_2(X^s).\]

Suppose that $x\in \E{}{r}{X}{s}{t}$ and $2\leq i\leq \max\{t-s,t-(r-1)\}$, so that $\delta^\textup{ext}_ix $ is defined. Then $\delta^\textup{ext}_i d_r x$ is defined, and 
\[d_r\delta^\textup{ext}_i(x)+ \delta^\textup{ext}_i(d_rx)=\begin{cases}
\Sq_\textup{ext}^{t-i+1}(x),&\textup{if $i>t-s$ and $r =t-i+1$};\\
\mu^\textup{ext}(x\otimes d_rx),&\textup{if $i=t-s$,  $s=0$ and $r\geq2$};\\
%\Sq_\textup{ext}^{t-i+1}(x),&\textup{if $t-s<i =t-(r-1)$}\\
0,&\textup{otherwise.}
\end{cases}\]
If $i\leq \max\{n,t+1-2(r-1)\}$, so that $\delta^\textup{ext}_i x$ is single-valued, then so is $\delta^\textup{ext}_i d_r x$, and this equation holds exactly. Otherwise, this equation holds modulo the indeterminacy of the right hand side. When $i>t-s$ and $r =t-i+1$ the set of values of the left hand side coincides with the set of values of the right hand side.  In particular, if $x\in \E{}{\infty}{X}{s}{t}$ and $t-s>0$, then $x\times x=0\in\E{}{\infty}{S_2X}{2s+1}{2t+1}$.

The same conclusions hold for $\textup{LAM}_i$, producing operations $\lambda_i^\textup{ext}$, and the inequality $2\leq i$ can be replaced with $1\leq i$ in this case.
\end{prop}
\begin{proof}
Suppose that $x\in \EZdownup{r}{}{X}{s}{t}$. Then Proposition \ref{dvsDEL} shows that $d\textup{DEL}_i(x)\in F^{s+r}C(S_2X)$ as long as $i\leq \max\{n,t-(r-1)\}$, so that $\textup{DEL}_i(x)\in \EZdownup{r}{}{X}{s}{t+i}$. 
Proposition \ref{Prop on delta external} then provides the formula for 
$d_r\delta^\textup{ext}_i(x)+ \delta^\textup{ext}_i(d_rx)$ (modulo indeterminacy).

Let us begin with the operations $\delta^\textup{ext}_i:\Edownup{1}{}{X}{s}{t}\to \Edownup{1}{}{S_2X}{s}{t+i}$. Due to the assumption that $\{D^k\}$ is special, for any $x,y\in F^{s}CX$, $\mathbb{D}_k(x\otimes  y)\equiv \nabla_k\twist^k D^0(x\otimes y) $ modulo $F^{2s+1}X$, and due to Lemma \ref{DkIsNiceToFiltration}, we can ignore the horizontal component of the differential in calculating $dx$ applied when defining $\textup{DEL}_ix$. The resulting operations have leading term which is almost identical to the definition of the operations $\delta_i^\textup{ext}$ of \S\ref{External unary homotopy operations}. The only difference is the $\twist$ operator that appears, but this does not affect the resulting operation, by \cite[Lemma 4.1]{DwyerHtpyOpsSimpComAlg.pdf}.

We will now examine, for $r\geq1$, the extent to which $\textup{DEL}_{i}$ induces a well defined operation with domain $\Edownup{r}{}{X}{s}{t}$, which is to examine the difference $\textup{DEL}_i(x)-\textup{DEL}_i(x+dy)$ for $y\in \EZdownup{r-1}{}{X}{s-r+1}{t-r+2}$.
We may assume that $s>0$, as the operation on $E_1$ has just been shown to be well defined, and $\Edownup{r}{}{X}{0}{t}\subseteq \Edownup{1}{}{X}{0}{t}$ for $r\geq1$.
This implies that $t-i\geq1$ whenever $i=n$. 
By Lemma \ref{boundaryVsBBD}:
\begin{alignat*}{4}
d\mathbb{D}_{n-i+1}(y\otimes y)&=
\mathbb{D}_{n-i+1}d(y\otimes y)&&+
{(1{+}\twist)\mathbb{D}_{n-i+2}(y\otimes y)}&&+
\Nabla\twist D^{-(n-i+2)}(y\otimes y)\\
d\mathbb{D}_{n-i}(dy\otimes y)&=
\mathbb{D}_{n-i}(dy\otimes dy)&&+
(1{+}\twist)\mathbb{D}_{n-i+1}(dy\otimes y)&&+
\Nabla\twist D^{-(n-i+1)}(dy\otimes y)\\
d\mathbb{D}_{n-i-1}(x\otimes dy)&=
{\mathbb{D}_{n-i-1}(dx\otimes dy)}&&+
(1{+}\twist)\mathbb{D}_{n-i}(x\otimes dy)&&+
\Nabla\twist D^{-(n-i)}(x\otimes dy)
\end{alignat*}
(As in the proof of Proposition \ref{dvsDEL}, there are extra terms which appear when $i=1$, but they are annihilated by the application of $\rho'$.)
Write $H:=\rho(\mathbb{D}_{n-i-1}(x\otimes dy)+\mathbb{D}_{n-i}(dy\otimes y)+\mathbb{D}_{n-i+1}(y\otimes y))$. By Lemma \ref{DkIsNiceToFiltration}, $H\in F^{s-r+1}$, and the above calculations show that% $H$ has boundary
%\small
%%\[(\textup{DEL}_i(x)-\textup{DEL}_i(x+dy))+\rho\Nabla(D^{i-n-2}(y\otimes y)+D^{i-n-1}(y\cdot dy)+D^{i-n}(dy\cdot x))+\rho\mathbb{D}_{n-i-1}(dx\cdot dy).\]
%\[dH=\textup{DEL}_i(x)-\textup{DEL}_i(x+dy)+T_1+T_2+T_3,\]
%\normalsize
%where $\textup{DEL}_i(x)-\textup{DEL}_i(x+dy)\in F^s$ is the quantity of interest, and
%\begin+{align*}
%\textup{DEL}_i(x)-\textup{DEL}_i(x+&dy)=dH+T_1+T_2+T_3\textup{, \ where}
%\\
%T_1:=\rho\Nabla(D^{i-n-2}(y\otimes y))&\begin{cases}
%\in F^{s+(t-i)-2(r-2)},&\textup{if }n+2\leq i\leq t-1;\\
%=0,&\textup{if }i\leq n+1.
%%\\,&\textup{if }
%\end{cases}
%\\
%T_2:=\rho\Nabla(D^{i-n-1}(y\otimes dy))&\begin{cases}
%\in F^{s+(t-i)-(r-2)},&\textup{if }n+1\leq i\leq t-1;\\
%=0,&\textup{if }i\leq n.
%%\\,&\textup{if }
%\end{cases}
%\\
%T_3:=\rho\Nabla(D^{i-n}(dy\otimes x))&
%\ \left\{\begin{array}{ll}
%\smash{\in F^{s+(t-i)}},&\smash{\textup{if }n\leq i\leq t-1;}\\
%\smash{=0,}&\smash{\textup{if }i\leq n-1.}
%\end{array}\right.
%\begin{cases}
%\smash{\in F^{s+(t-i)}},&\smash{\textup{if }n\leq i\leq t-1;}\\[-.5em]
%\smash{=0,}&\smash{\textup{if }i\leq n-1.}
%%\\,&\textup{if }
%\end{cases}
%\\
%\end{align*}
%\begin{align*}
%T_1:={}&\rho\Nabla(D^{i-n-2}(y\otimes y))&&\in F^{s+(t-i)-2(r-2)}\textup{\ \ equals zero if }i\leq n+1.\\
%T_2:={}&\rho\Nabla(D^{i-n-1}(y\otimes dy))&&\in F^{s+(t-i)-(r-2)}\textup{\ \ equals zero if }i\leq n.\\
%T_3:={}&\rho\Nabla(D^{i-n}(dy\otimes x))&&\in F^{s+(t-i)}\textup{\ \ equals zero if }i\leq n-1.\\
%\end{align*}
\begin{align*}
dH=\textup{DEL}_i(x)&-\textup{DEL}_i(x+dy)+T_1+T_2+T_3,\textup{ where:}\\
T_1:={}&\rho\Nabla(D^{i-n-2}(y\otimes y))\in F^{s+(t-i)-2(r-2)}\textup{\ \ equals zero when }i\leq n+1;\\
T_2:={}&\rho\Nabla(D^{i-n-1}(y\otimes dy))\in F^{s+(t-i)-(r-2)}\textup{\ \ equals zero when }i\leq n;\\
T_3:={}&\rho\Nabla(D^{i-n}(dy\otimes x))\in F^{s+(t-i)}\textup{\ \ equals zero when }i\leq n-1.
\end{align*}
%Now if $i=n$, except when $s=0$ (so that $i=n=t$), $T_3\in F^{s+1}$, and can be ignored. 
If $i\leq n-1$, then $T_1=T_2=T_3=0$. If $i=n$ then $T_3$ need not vanish, but as $t-1\geq1$, $T_3\in F^{s+1}$. This shows that $\delta_i^\textup{ext}$ is single-valued if $i\leq n$.
If $i\geq n+1$ then $T_1+T_2$  can only be assured to lie in filtration above $s$ when $i\leq t+1-2(r-1)$. This proves that the operations are single-valued when we claim. 
%\textbf{The following is an observation that I thought might improve this result, but which does not  seem to - the idea was to leave out the third row of stuff. Maybe it could still be good if we leave out some number of the summands from the third row, but it sounds pretty dicey. Should try it in order to extend rande of definition. Also should check whether one should expect the operations to be defined all the way up to $i\leq t-r+1$. ``Now suppose that $x\in F_s$, and $y\in F_{s-r+1}$. Let us look at $\mathbb{D}_{n-i+1}(dy\cdot y)$. As $dy$ has filtration $s$, each term $\Nabla_{\alpha}\twist^{\alpha}D^\beta(dy\cdot y)$ with $\beta>s$ vanishes, and each term with $\beta<s$ has filtration higher than $dy$. Thus, modulo $F_{s+1}$, $\mathbb{D}_{n-i+1}(dy\cdot y)\equiv\Nabla_{t-i+1}\twist^{t-i+1}D^s((dy)^s_{t}\cdot y^{s}_{t+1})$.''}

Even when $s=0$ or $i\leq n$, we may summarise the situation as follows. There is some $H\in F^{s-r+1}$ such that
%\[dH=\textup{DEL}_i(x)-\textup{DEL}_i(x+dy)+T, \textup{ where } dT=\textup{SQ}^{i,s}(x)-\textup{SQ}^{i,s}(x+dy)\]
\[
dH
=
\textup{DEL}_i(x)-\textup{DEL}_i(x+dy)+\textup{BC},%
\]
where $\textup{BC}:=T_1+T_2+T_3$ is an example of the bounding chain appearing in the proof of propositions \ref{prop on e1 steens 1} and \ref{prop on e1 steens 2}, so that
modulo $F^{s+t-i+2}$:
\[d(\textup{DEL}_i(x))-d(\textup{DEL}_i(x+dy))
\equiv
\textup{SQ}^{t-i+1,s}(x)-\textup{SQ}^{t-i+1,s}(x+dy),\]
%
%\[dT
%\equiv\begin{cases}
%\textup{SQ}^{t-i+1,s}(x)-\textup{SQ}^{t-i+1,s}(x+dy),&\textup{if }n+1\leq i\leq t+1;\\
%0,&\textup{if }2\leq i\leq n.
%%\\,&\textup{if }
%\end{cases}
%\]
%(In the case that $s=0$ and $i=t$, we use the observation that $dT_3=0$ $\cdots \cdots \cdots \cdots \cdots $)
%In particular, if $n+1\leq i\leq t$, the modulo filtration exceeding $s+t-i+1$:
%\[d\textup{DEL}_i(x+dy)-d\textup{DEL}_i(x)=\textup{SQ}^{t-i+1,s}(x)-\textup{SQ}^{t-i+1,s}(x+dy),\]
so that the multiple values of $\delta_i^\textup{ext}$ map \emph{onto} those of $\Sq_\textup{ext}^{t-i+1}$ under $d_{t-i+1}$.
\end{proof}
%\begin{cor}
%\label{cor on meaning of equation for ddeltai}
%Suppose that $x\in \E{}{r}{X}{s}{t}$ and $2\leq i\leq \max\{t-s,t-(r-1)\}$, so that $\delta^\textup{ext}_ix $ is defined. Then $\delta^\textup{ext}_i d_r x$ is defined, and 
%\[d_r\delta^\textup{ext}_i(x)+ \delta^\textup{ext}_i(d_rx)=\begin{cases}
%\mu^\textup{ext}(x\otimes d_rx),&\textup{if $i=t-s$ and  $s=0$};\\
%\Sq_\textup{ext}^{t-i+1}(x),&\textup{if $t-s<i =t-(r-1)$}\\
%0,&\textup{otherwise.}
%\end{cases}\]
%If $i\leq \max\{n,t+1-2(r-1)\}$, so that $\delta^\textup{ext}_i x$ is single-valued, then so is $\delta^\textup{ext}_i d_r x$, and this equation holds exactly. Otherwise, this equation holds modulo appropriate indeterminacy.
%\end{cor}
%
%\begin{cor}[ver2]
%Suppose that $x\in \E{}{r}{X}{s}{t}$ and $2\leq i\leq \max\{t-s,t-(r-1)\}$, so that $\delta^\textup{ext}_ix $ is defined.
%\[d_r\delta^\textup{ext}_i(x)=\begin{cases}
%\delta^\textup{ext}_i(d_rx),&\textup{if $i< t-s$ or $i<t-(r-1)$};\\
%\delta^\textup{ext}_i(d_rx),&\textup{if $i=t-s$ and $s>0$};\\
%\delta^\textup{ext}_i(d_rx)+\mu^\textup{ext}(x\otimes d_rx),&\textup{if $i=t-s$ and  $s=0$};\\
%\delta^\textup{ext}_i(d_rx)+\Sq_\textup{ext}^{t-i+1}(x),&\textup{if $i>t-s$ and $i=t-(r-1)$}.
%\end{cases}\]
%\end{cor}
%

\begin{prop}
\label{Prop on einfty ops}
Suppose that $X\in (s\vect{}{})^{\Delta_+}$, i.e.\ that $X$ admits a coaugmentation from some $X^{-1}\in s\vect{}{}$. For $2\leq i\leq t-s$, the operations $\delta^\textup{ext}_i:\E{}{\infty}{X}{s}{t}\to \E{}{\infty}{S^2X}{s+1}{t+i+1}$ agree with the homotopy operations $\delta^\textup{ext}_i:\pi_{t-s}(X^{-1})\to \pi_{t-s+i}(S^2(X^{-1}))$. Similarly, the external pairing at $S_2\E{}{\infty}{X}{}{}\to \E{}{\infty}{S_2X}{}{}$ agrees with  $\widetilde{\nabla}:S_2\pi_*(X^{-1})\to \pi_*(S_2(X^{-1}))$.

The same conclusions hold for the $\lambda_i$ for $1\leq i\leq t-s$.
\end{prop}
\begin{proof}
We will only prove the statement about $\delta_i^\textup{ext}$, as the statement about products is easier and more standard. We only need to show that the following diagram commutes whenever $2\leq i\leq n$:
\[\xymatrix@R=4mm@C=21mm{
ZC_{n}(X)
\ar[r]^-{\textup{DEL}_i}
\ar[d];[]^-{d\dhor^0}
&%r1c1
ZC_{n}(S_2X)\ar[d];[]^-{d\dhor^0}
\\%r1c3
ZC_{n}(X^{-1})
\ar[r]_-{z\mapsto \rho(\nabla_{n-i}(z\otimes z))}
&%r1c1
ZC_{n}(S_2(X^{-1}))
}\]
We calculate
\begin{alignat*}{3}
\textup{DEL}_i(d\dhor^{0}z)
:={}&
\rho\mathbb{D}_{n-i}(d\dhor^{0}z\otimes d\dhor^{0}z)+\rho\mathbb{D}_{n-i-1}(d\dhor^{0}z\otimes d(d\dhor^{0}z))%
\\
={}&
\rho\nabla_{n-i}\twist^{n-i}D^0(d\dhor^{0}z\otimes d\dhor^{0}z)+\rho\mathbb{D}_{n-i-1}(d\dhor^{0}z\otimes 0)\\
={}&
\rho\nabla_{n-i}(d\dhor^{0}z\otimes\dver d\dhor^{0}z)=d\dhor^{0}(\rho\nabla_{n-i}(z\otimes z)),
\end{alignat*}
where we have used the assumption that $\{D^k\}$ is special in both the second and third equations, and $d(d\dhor^{0}z)=0$ since $z\in ZC_n(X^{-1})$ is a (vertical) cycle, and $d\dhor^{0}$ equalizes $d\dhor^{0}$ and $d\dhor^{1}$.
\end{proof}


\SubsectionOrSection{Internal operations on $\E{}{r}{X}{}{}$ for $X\in cs\calc$}
\label{The case of a cosimplicial simplicial commutative algebra}
Suppose that $X\in cs\algs$. We may define operations:
\begin{gather*}
\delta_i:\left(\Edownup{r}{}{X}{s}{t} \smash{{}\overset{\delta_i^\textup{ext}}{\to}{}} \Edownup{r}{}{\quadgrad{2}F^\calc X}{s}{t+i}\overset{\mu_*}{\to} 
\Edownup{r}{}{X}{s}{t+i}\right),
\\
\Sq^j:\left(\Edownup{r}{}{X}{s}{t}   \smash{{}\overset{\Sq^j_\textup{ext}}{\to}{}} \Edownup{r}{}{\quadgrad{2}F^\calc X}{s+j}{2t}\overset{\mu_*}{\to} 
\Edownup{r}{}{X}{s+j}{2t}\right),\\
\mu:\left(\Edownup{r}{}{X}{s}{t}\otimes \Edownup{r}{}{X}{s'}{t'}\smash{{}\overset{\mu_\textup{ext}}{\to}{}} \Edownup{r}{}{\quadgrad{2}F^\calc X}{s+s'}{t+t'}\overset{\mu_*}{\to} 
\Edownup{r}{}{X}{s+s'}{t+t'}\right),
\end{gather*}
with the $\delta_i$ potentially multi-valued functions,  defined when $2\leq i\leq \max\{n,t-(r-1)\}$, and single valued whenever $2\leq i\leq\min\{n+1,t+1-2(r-1)\}$, and the $\Sq^j$ potentially multi-valued functions with indeterminacy vanishing by $E_{2r-2}$, and which equal zero unless $\min\{t,r\}\leq j\leq s$. 

These operations will not be used in rest of this thesis, as they will equal zero in the case of interest to us, namely when $X\in cs\calc$ is a GEM in each cosimplicial level. Nonetheless, we hope they are of some independent interest. 

Numerous properties of these operations follow directly from the earlier results, namely propositions \ref{prop on basic product}, \ref{prop on e1 steens 1}, \ref{prop on e1 steens 2} \ref{Prop on delta external} and \ref{Prop on einfty ops}. In addition, we have


\begin{prop}
\label{final generic prop for basic sseq ops}
The operations $\delta_i:\Edownup{1}{}{X}{s}{t}\to \Edownup{1}{}{X}{s}{t+i}$
are (the restriction of) the homotopy operations of \S\ref{Homotopy operations for simplicial commutative algebras} applied to the homotopy of the simplicial algebra $X^{s}$. Moreover, for each $s$, $\pi\uver_*X^s$ is a graded commutative algebra (again, c.f.\ \S\ref{Homotopy operations for simplicial commutative algebras}), and the operations $\mu$ and $\Sq^j$ on $E_2$ are the standard operations on the cohomotopy of the cosimplicial commutative algebra $\pi\uver_*X^s$. As such, the operations $\Sq^j$ make $\Edownup{2}{}{X}{}{}$ is an unstable left module over the homogeneous Steenrod algebra, and satisfy the evident unstableness condition and the Cartan formula.

If $x\in \Edownup{1}{}{X}{s}{t}$ and $2\leq i \leq 2t$ (so that the $\delta^\textup{ext}_i$ operation that follows is defined), then $\delta_i\Sq^jx=0\in \Edownup{1}{}{X}{s+j}{2t+i}$. If also $y\in \Edownup{1}{}{X}{s'}{t'}$, and $2\leq i <t+t'$, then $\delta_i(xy)=0$.
\end{prop}
\begin{proof}[Proof of Proposition \ref{final generic prop for basic sseq ops}]
Everything here is straightforward, and we will present the calculation 
$\delta_i\Sq^jx=0$ as an example.
Suppose that $x\in \EZdownup{1}{}{X}{s}{t}$ and $2\leq i\leq 2t$. This condition implies that $t>0$, and for our current purpose we can assume that $x\in X^s_t$, so that $d\dver x=0$.  Then $\Sq^j_\textup{ext}x$ is represented by the image of $D^{s-j}(x\otimes x)+D^{s-j+1}(x\otimes d\dhor x)$ under the composite
\[N^{s+j}(N_tX\otimes N_tX)\overset{N^{s+j}(\widetilde{\Nabla})}{\to}N^{s+j}N_{2t}(X\otimes_{\Sigma_2} X)\overset{\mu}{\to}N^{s+j}N_{2t}(X)\overset{\delta_i}{\to}N^{s+j}N_{2t+i}(X),\]
and Proposition \ref{omnibus on htpy of simp algs} states that the final $\delta$-operation annihilates products of positive dimensional classes, so that this composite is zero.
\end{proof}

As a final note here, suppose that $X\in s\liealgs$. We may define operations:
\begin{gather*}
\lambda_i:\left(\Edownup{r}{}{X}{s}{t} \smash{{}\overset{\lambda_i^\textup{ext}}{\to}{}} \Edownup{r}{}{\quadgrad{2}F^\calc X}{s}{t+i}\smash{{}\overset{[,]_*}{\to} {}}
\Edownup{r}{}{X}{s}{t+i}\right),
\\
P^{j-1}:\left(\Edownup{r}{}{X}{s}{t}   \smash{{}\overset{\Sq^j_\textup{ext}}{\to}{}} \Edownup{r}{}{\quadgrad{2}F^\calc X}{s+j}{2t}\smash{{}\overset{[,]_*}{\to} {}}
\Edownup{r}{}{X}{s+j}{2t}\right),\\
[\,,]:\left(\Edownup{r}{}{X}{s}{t}\otimes \Edownup{r}{}{X}{s'}{t'}\smash{{}\overset{\mu_\textup{ext}}{\to}{}} \Edownup{r}{}{\quadgrad{2}F^\calc X}{s+s'}{t+t'}\smash{{}\overset{[,]_*}{\to} {}}
\Edownup{r}{}{X}{s+s'}{t+t'}\right),
\end{gather*}
with the $\lambda_i$ potentially multi-valued functions,  defined when $1\leq i\leq \max\{n,t-(r-1)\}$ %(resp.\ $0\leq i\leq \max\{n,t-(r-1)\}$), 
and single valued whenever $1\leq i\leq\min\{n+1,t+1-2(r-1)\}$.
It should be possible to state versions of all of the above results in this case. The author guesses that the operations $P^{k}$ will form an unstable left action of the $P$-algebra (the Steenrod algebra for commutative $\Ftwo$-algebras, as in \S\ref{The example of simplicial commutative F2-algebras}) but has not worked out the details.

%As $\Edownup{1}{}{X}{s}{t}$ is obtained from $E_0$ by taking homotopy in the simplicial direction, there are $\delta^\textup{ext}$-operations on $E_1$:
%\[\delta^\textup{ext}_i:\Edownup{1}{}{X}{s}{t} \to \Edownup{1}{}{S_2X}{s}{t+i}\textup{ for }2\leq i\leq t.\]
%These are unstable with respect to internal degree $t$, not with respect to total degree $n=t-s$, are linear \emph{except for the top operation} \citeBOX[4.2]{DwyerHtpyOpsSimpComAlg.pdf}, and commute with the $d_1$ differential. Thus, each of these operations except the top operation induces an operation on $E_2$:
%\[\delta^\textup{ext}_i:\Edownup{2}{}{X}{s}{t} \to \Edownup{2}{}{S_2X}{s}{t+i}\textup{ for }2\leq i< t.\]
%\begin{prop}
%On $E_2$, for $i<t$, the spectral sequence operations $\delta^\textup{ext}_i$ defined by $\textup{DEL}_i$ equal the operations $\delta^\textup{alg}_i$. The same holds on $E_1$ when $2\leq i\leq t$. In particular, the $\delta^\textup{ext}$-operations on the spectral sequence satisfy the $\delta$-Adem relations \textbf{if there is a commutative algebra structure around}.
%\end{prop}
%\begin{proof}
%The operations $\delta^\textup{alg}_i$ are constructed by taking a representative $z\in C_{-s,t}X$ such that $\partial_\textup{sim}z=0$, and forming $\rho\Nabla_{t-i}(z\otimes\dver z)$, where $z\otimes\dver z$ denotes the element of $C(X\otimes\dver X)_{-s,2t}$.
%
%Let us calculate the leading term of $\textup{DEL}_{i}(x)$ whenever $2\leq i\leq t$. Write $x=x^s_t+x^{>s}_{>t}$, where $x^s_t\in X^s_t$, and $x^{>s}_{>t}$ lies in filtration $s+1$. Then the leading term will be $\rho\Nabla_{t-i}\twist^{t-i}D^s (x^s_t\otimes x^s_t)$. Moreover, as $\{D^k\}$ is \emph{special}, this equals $\rho{\Nabla}_{t-i}(x^s_t\otimes\dver x^s_t)\in C(X\otimes_{\Sigma_2} X)_{-s,t+i}$, where $(x^s_t\otimes\dver x^s_t)$ denotes the element of $C(X\otimes\dver X)_{-s,2t}$. Thus `$x^s_t\mapsto \twist^{t-i}D^s(x^s_t\otimes x^s_t)$' implements the assignment `$z\mapsto z\otimes\dver z$' discussed above, and we are getting a representative for $\delta^\textup{alg}_i(x)$.
%\end{proof}
%
%
%
%
%\begin{thm}
%By postcomposition with the multiplication $S_2X\to X$, the operations $\Sq^j_\textup{ext}$, $\delta_i^\textup{ext}$ and $\mu_\textup{ext}$ induce operations $\Sq^j_\textup{raw}$, $\delta_i^\textup{raw}$, $\mu_\textup{raw}$ on $E_r(X)$. The Steenrod operations vanish whenever $i<r$, after all:
%\begin{enumerate}\squishlist
%\setlength{\parindent}{.25in}
%\item When $i<t-s$, $d_r\delta^\textup{raw}_i(x)=\delta^\textup{raw}_i(d_rx)$.
%\item When $i=t-s$, $d_r\delta^\textup{raw}_i(x)=\begin{cases}
%\delta^\textup{raw}_i(d_rx),&\textup{if }s>0;\\
%\delta^\textup{raw}_i(d_rx)+x\times d_rx,&\textup{if }s=0.
%%\\,&\textup{if }
%\end{cases}$
%\item When $i>t-s$, $d_r\delta^\textup{raw}_i(x)=\begin{cases}
%\delta^\textup{raw}_i(d_rx),&\textup{if }r<t-i+1;\\
%\delta^\textup{raw}_i(d_rx)+\Sq_\textup{raw}^{t-i+1}(x),&\textup{if }r=t-i+1.
%\end{cases}$
%\end{enumerate}
%Broadly speaking then, the $\delta^\textup{raw}_i$ operations for $i>t-s$ support a differential hitting either a Steenrod operation or another $\delta^\textup{raw}_i$ operation, and all Steenrod operations applied to permanent cycles are hit in this way. If $X$ admits a coaugmentation from a simplicial algebra $X$, then the $\delta^\textup{raw}$-operations with $i\leq n$ at $E_\infty$ agree with the $\delta^\textup{raw}$-operations on the target of the spectral sequence.
%
%The $\delta^\textup{raw}$-operations (resp.\ Steenrod operations)  at $E_2$ agree with those following from the observation that $E_1$ is the normalization of a cosimplicial $\textup{Com}$-$\Pi$-algebra (resp.\ algebra). As such, the $\delta_i^\textup{raw}$ and $\Sq^j_\textup{raw}$ satisfy the expected Adem relations. Finally, at $E_1$, we have the relation $\delta_i^\textup{raw}\Sq^j_\textup{raw}x=0$ (whenever $\delta_i^\textup{raw}$ is non-top). \TODOCOMMENT[inline]{Probably more detail needed}, for example, when is $\Sq_\textup{raw}^0=0$ or $\Sq_\textup{raw}^1=0$?
%
%[from above]: For $x\in E_r^{-s,t}(X)$, $\Sq^i_\textup{ext}(x)$ survives to $E_{2r-1}$, and moreover, $d_{2r-1}(\Sq^i_\textup{ext}x)=\Sq^{i+r-1}_\textup{ext}(d_rx)$ modulo indeterminacy. The top operation, $\Sq^s_\textup{ext}(x)$, is equal to the product-square $\mu_\textup{ext} (x\otimes x)$, and $\Sq^i_\textup{ext}(x)=0$ for $i>s$. The bottom operation, $\Sq^0_\textup{ext}(x)$, is zero whenever $t>0$, and $\Sq^1_\textup{ext}(x)=0$ for all $t\geq2$. If $X$ is a mixed simplicial algebra, postcomposition with the product $S_2X\to X$ yields the same squares at $E_2$ as those arising from the fact that $E_2$ is the cohomotopy of a cosimplicial algebra.
%\end{thm}

\end{second quadrant homotopy sseq operations}

\begin{Operations on the Bousfield-Kan spectral sequence}
\SectionOrChapter{Operations in the Bousfield-Kan spectral sequence}
\label{Operations on the Bousfield-Kan spectral sequence}
In this chapter
we will define operations on the \BKSS\ for an object $X\in s\calc$ whenever $\calc$ is any of the categories $\algs$, $\liealgs$ or $\restliealgs$. We will always write $\calx$ for Radulescu-Banu's resolution of $X\in s\algcat$. %As we have developed homology operations for the category $\calw(0)$, but not for the categories $\HA{\liealgs}$


\SubsectionOrSection{An alternate definition of the Adams tower}
\label{An alternate definition of the Adams tower}
We will now give an alternate definition of the Adams tower of \S\ref{sec:derWRTab}, given in \cite{BK_pairings_products.pdf}, which is more suited for the definition of spectral sequence operations in our setting.

For $Z\in {\vect{}{}}^{\Delta_+}$, the category of coaugmented cosimplicial vector spaces, Bousfield and Kan write $VZ$ for a ``path-like construction'' \citeBOX[\S3.1]{BK_pairings_products.pdf} on $Z$ obtained by shifting $Z$ down and forgetting the $0^\textup{th}$ coface and codegeneracy. That is, $(VZ)^s:=(VZ)^{s+1}$, and:
\begin{align*}
\bigl((VZ)^s\overset{d^i}{\to} (VZ)^{s+1}\bigr)&:=\bigl(Z^{s+1}\overset{d^{i+1}}{\to} Z^{s+2}\bigr)\\
\bigl((VZ)^s\overset{s^i}{\to} (VZ)^{s-1}\bigr)&:=\bigl(Z^{s+1}\overset{s^{i+1}}{\to} Z^{s}\bigr)
\end{align*}
In fact, the unused coface $d^0$ induces a map $v:Z\to VZ$.

For $Y\in s\vect{}{}$, the standard simplicial path fibration (c.f.\ \citeBOX[p.~82]{BousKanSSeq.pdf}) produces $\Lambda Y\in s\vect{}{}$ by shifting down and restricting to a kernel:
\[\Lambda Y_s=\ker\bigl(d_{s+1}\cdots d_1:Y_{s+1}\to Y_0\bigr).\]
We forget the $0^\textup{th}$ face and degeneracy as before. This time the unused face map $d_0$ induces a fibration $\lambda:\Lambda Y\to Y$, and $\Lambda Y$ is  contractible.


Each of these constructions can be prolonged to an endofunctor of $(s\calc)^{\Delta_+}$, endofunctors which are  necessary for a key construction of Bousfield and Kan \cite{BK_pairings_products.pdf,BK_pairings.pdf}. Define an endofunctor $\Dendo^1$ of the category $(s\calc)^{\Delta_+}$ of augmented cosimplicial objects in $s\calc$, using the pullback (for $W\in (s\calc)^{\Delta_+}$):
\[\xymatrix@R=4mm{
\Dendo^1 W \ar[r]
\ar[d]^-{\delta}
&%r1c1
\Lambda VW \ar[d]^-{\lambda}
\\%r1c2
W \ar[r]^-{v}
&%r2c1
VW %r2c2
}\]
Then one can form a tower in $(s\calc)^{\Delta_+}$, (writing $\Dendo^n:=\Dendo^1\circ\cdots \Dendo^1$):
\[\xymatrix@R=4mm{
\cdots 
\ar[r]
&%r1c1
\Dendo^2W
\ar[r]
&%r1c3
%r1c4
\Dendo^1W
\ar[r]&%r1c5
\Dendo^0W
\ar@{=}[r]
&
W.
}\]
Restricting to augmentations, there is a tower of fiber sequences in $s\calc$:
\[\xymatrix@R=4mm{
\cdots 
\ar[r]
&%r1c1
(\Dendo^2W)^{-1}
\ar[r]\ar[d]^-{d^0}
&%r1c3
%r1c4
(\Dendo^1W)^{-1}
\ar[d]^-{d^0}\ar[r]&%r1c5
(\Dendo^0W)^{-1}\ar[d]^-{d^0}
\ar@{=}[r]
&
W^{-1}
\\
&
(\Dendo^2W)^{0}
&
(\Dendo^1W)^{0}
&
(\Dendo^0W)^{0}
}\]
%whose homotopy long exact sequences fit together to form an exact couple [reference], below which we may insert some dotted arrows to be defined:
%\[\xymatrix@R=4mm{
%\cdots 
%\ar[r]
%&%r1c1
%\pi_{t-2}((D^2W)^{-1})
%\ar[r]\ar[d]^-{d^0}
%&%r1c3
%%r1c4
%\pi_{t-1}((D^1W)^{-1})
%\ar[d]^-{d^0}\ar[r]&%r1c5
%\pi_{t}((D^0W)^{-1})\ar[d]^-{d^0}
%\ar@{=}[r]
%&
%\pi_{t}(W^{-1})
%\\
%&
%\pi_{t-2}((D^2W)^{0})\ar@{-->}[ul]^-{\partial}
%&
%\pi_{t-1}((D^1W)^{0})\ar@{-->}[ul]^-{\partial}
%&
%\pi_{t-0}((D^0W)^{0})\ar@{-->}[ul]^-{\partial}
%\\\cdots&
%N_\subseteq^2\pi_tW^\bullet \ar[u]_-{\partial_\textup{it}}^-{\cong} \ar@{..>}[l]_-{d^1} \ar@{ >->}[d]_-{\sim}&
%N_\subseteq^1\pi_tW^\bullet \ar[u]_-{\partial_\textup{it}}^-{\cong} \ar@{..>}[l]_-{d^1} \ar@{ >->}[d]_-{\sim}&
%N_\subseteq^0\pi_tW^\bullet \ar[u]_-{\partial_\textup{it}}^-{\cong} \ar@{..>}[l]_-{d^1} \ar@{ >->}[d]_-{\sim}&
%\\
%\cdots&
%C^2\pi_tW  \ar@{..>}[l]_-{d^1}\ar@{-->}[dl]_-{\partial}&
%C^1\pi_tW  \ar@{..>}[l]_-{d^1}\ar@{-->}[dl]_-{\partial}&
%C^0\pi_tW  \ar@{..>}[l]_-{d^1}\ar@{-->}[dl]_-{\partial}&
%%\pi_t W^{-1} \ar[l]_-{d}
%\\
%\cdots\ar[r]&
%H_t(F^2TW)\ar[r]\ar[u]&
%H_t(F^1TW)\ar[r]\ar[u]&
%H_t(F^0TW)\ar[r]\ar[u]&
%}\]
%
%
%\[\xymatrix@R=4mm{
%\cdots&
%C^2\pi_tW  \ar@{..>}[l]_-{d^1}&
%C^1\pi_tW  \ar@{..>}[l]_-{d^1}&
%C^0\pi_tW  \ar@{..>}[l]_-{d^1}
%%\pi_t W^{-1} \ar[l]_-{d}
%\\
%&
%H_{t-2}(\calK_2)\ar[d]\ar@{=}[u]
%&
%H_{t-1}(\calK_1)\ar[d]\ar@{=}[u]
%&
%H_{t}(\calK_0)\ar[d]\ar@{=}[u]
%\\
%\cdots\ar[r]&
%H_{t-2}(\calT_2W)\ar[r]\ar@{-->}[ul]_-{\partial}
%&
%H_{t-1}(\calT_1W)\ar[r]\ar@{-->}[ul]_-{\partial}
%&
%H_t(\calT_0W)\ar[r]\ar@{-->}[ul]_-{\partial}
%&0
%}\]
Bousfield and Kan note \citeBOX[\S3.3 and \S4.2]{BK_pairings_products.pdf} that this tower \emph{equals} the Adams tower $R_nX$ when $W=\calx$ is Radulescu-Banu's resolution of $X\in s\algcat$. They also explicitly perform the resulting identification of the $E_1$-page of the spectral sequence of this tower with $N^s_\subseteq\pi_tW$, using iterates of the connecting map
\[\pi_t(W^{s})=\pi_t(VW^{s-1})\overset{\partial_\textup{conn} }{\to}\pi_{t-1}(\Dendo^1W)^{s-1}\]
of the fiber sequence $(\Dendo^1W)^{s-1}\to W^{s-1}\to VW^{s-1}$, which has the property:
\begin{prop}[{\cite[Proposition 5.2]{BK_pairings_products.pdf}}]
\label{BK D1 is awesome}
The following composite involving the connecting map $\partial_\textup{conn}$ induces an isomorphism of cochain complexes:
\[N^s_\subseteq \pi_t W\subseteq N^{s-1}_\subseteq \pi_t VW \overset{\partial_\textup{conn}}{\to}N^{s-1}_\subseteq \pi_{t-1} (\Dendo^1W).\]
\end{prop}
\noindent Note that the inclusion in this theorem can be strict --- the subspace $N^s_\subseteq \pi_t W$ of $C^s\pi_t W^s$ is defined by the vanishing of the maps $s^0,\ldots,s^{s-1}:\pi_t W^s\to \pi_t W^{s-1}$, while $ N^{s-1}_\subseteq \pi_t (VW)^{s-1}$ is defined by the vanishing only of $s^1,\ldots,s^{s-1}:\pi_t W^s\to \pi_t W^{s-1}$, as $s^0$ is forgotten in passing to $VW$.

If we declare the spectral sequence an object $W\in cs\algcat$ to be the spectral sequence of the tower
\[\xymatrix@R=4mm{
\cdots 
\ar[r]
&%r1c1
(\Dendo^2W)^{-1}
\ar[r]
&%r1c3
%r1c4
(\Dendo^1W)^{-1}
\ar[r]&%r1c5
(\Dendo^0W)^{-1}
}\]
then the spectral sequence of $\Dendo^1\calx$ maps to the spectral sequence of $\calx$, with a filtration shift, via the map of towers:
\[\xymatrix@R=4mm{
\cdots 
\ar[r]
&%r1c1
(\Dendo^2\Dendo^1\calx)^{-1}
\ar[r]\ar[d]^{=}
&%r1c1
(\Dendo^1\Dendo^1\calx)^{-1}
\ar[r]\ar[d]^{=}
&%r1c3
%r1c4
(\Dendo^0\Dendo^1\calx)^{-1}
\ar[d]^{=}
\\
\cdots 
\ar[r]
&%r1c1
(\Dendo^3\calx)^{-1}
\ar[r]
&%r1c1
(\Dendo^2\calx)^{-1}
\ar[r]
&%r1c3
%r1c4
(\Dendo^1\calx)^{-1}
\ar[r]&%r1c5
(\Dendo^0\calx)^{-1}
}\]
That is, there are spectral sequence maps which at $E_1$ are isomorphisms  of the form
\[\Edownup{1}{}{\Dendo^1\calx}{s}{t}\overset{\cong}{\to} \Edownup{1}{}{\calx}{s+1}{t+1}.\]
Under Bousfield and Kan's identification of $E_1$, this isomorphism is the inverse of the composite of Proposition \ref{BK D1 is awesome}.

A reasonable goal is to create a factorization
\[\xymatrix@R=4mm{
&%r1c1
\Dendo^1\calx\ar[d]^-{\delta}
\\%r1c2
\quadgrad{2}F^{\calc}\calx\ar@{..>}[ur]
\ar[r]
&%r2c1
\calx%r2c2
}\]
%\[\quadgrad{2}F^{\calc}\calx\to \calx\]
of the structure map of $\calx$ through $\delta$. This will be possible up to a zig-zag whenever $\calx^s_t=(c(K^{\calc}Q^{\calc}c)^{s+1}X)_t$ is a  Radulescu-Banu resolution of $X\in s\calc$. 
%
%
%Consequently, the connecting map induces an equivalence
%\[C^s_\subseteq \pi_t (\Dendo^1\calx)\to  C^s_\subseteq \pi_{t-1} (V\calx),\]
%and 
%
%
%Bousfield and Kan explain that the composite
%\[N^\fraks\pi_\frakt(D^1X)\overset{\partial_\textup{conn}}{\from}N^\fraks\pi_{\frakt+1}(VX)\supset N^{\fraks+1}\pi_{\frakt+1}(X)\]
%is an isomorphism, where we note that the condition to lie in $N^{s+1}(O)\subseteq O^{s+1}$ is stricter than the condition to lie in $N^s(VO)\subseteq O^{s+1}$, for $O$ a cosimplicial module. Note that this composite is a cochain map, the reason being that the extra ``$d^0$'' in the coboundary out of $N^{\fraks+1}\pi_{\frakt+1}(X)$ that does not appear in the coboundary out of $N^{\fraks}\pi_{\frakt+1}(VX)$ becomes null after application of $\partial_\textup{conn}$, which is to say that the composite
%\[\pi_{\frakt+1}(X^{\fraks+1})\overset{d^0}{\to}\pi_{\frakt+1}(X^{\fraks+2})\overset{\partial_\textup{conn}}{\to}\pi_\frakt((D^1X)^{\fraks+1})\]
%is zero, which is clear, as we are just looking at adjacent maps in the homotopy long exact sequence.




%\begin{shaded}\tiny
%For $Z\in c\vect{}{}$, Bousfield and Kan write $VZ$ for a ``path-like construction'' \citeBOX[\S3.1]{BK_pairings_products.pdf} on $Z$ obtained by shifting $Z$ down and forgetting the 0th coface and codegeneracy. That is, $(VZ)^s:=(VZ)^{s+1}$, and:
%\begin{align*}
%\bigl((VZ)^s\overset{d^i}{\to} (VZ)^{s+1}\bigr)&:=\bigl(Z^{s+1}\overset{d^{i+1}}{\to} Z^{s+2}\bigr)\\
%\bigl((VZ)^s\overset{s^i}{\to} (VZ)^{s-1}\bigr)&:=\bigl(Z^{s+1}\overset{d^{i+1}}{\to} Z^{s}\bigr)
%\end{align*}
%In fact, the unused coface $d^0$ induces a map $v:Z\to VZ$.
%
%For $Y\in s\vect{}{}$, the standard simplicial path fibration (c.f.\ \citeBOX[p.~82]{BousKanSSeq.pdf}) produces $\Lambda Y\in s\vect{}{}$ by shifting down and restricting to a kernel:
%\[\Lambda Y_s=\ker\bigl(d_{s+1}\cdots d_1:Y_{s+1}\to Y_0\bigr).\]
%We forget the $0^\textup{th}$ face and degeneracy as before. This time the unused face map $d_0$ induces a fibration $\lambda:\Lambda Y\to Y$, and $\Lambda Y$ is  contractible.
%
%
%
%Each of these constructions can be prolonged to an endofunctor of $cs\vect{}{}$. 
%Now it is possible to recall a key construction of Bousfield and Kan \cite{BK_pairings_products.pdf,BK_pairings.pdf}. Define $D^1\calx \in cs\algs$ to be the pullback:
%\[\xymatrix@R=4mm{
%D^1\calx \ar[r]
%\ar[d]^-{\delta}
%&%r1c1
%\Lambda V\calx \ar[d]^-{\lambda}
%\\%r1c2
%\calx \ar[r]^-{v}
%&%r2c1
%V\calx %r2c2
%}\]
%The overwhelming virtue of this construction is the following observation of Bousfield and Kan \cite[Proposition 5.2]{BK_pairings_products.pdf}:
%\begin{prop}
%\label{BK D1 is awesome}
%The connecting map
%\[\pi_t(\calx^{s})=\pi_t(V\calx^{s-1})\overset{\partial }{\to}\pi_{t-1}(D^1\calx)^s\]
%in the homotopy long exact sequence of the fiber sequence $(D^1\calx)^{s-1}\to \calx^{s-1}\to V\calx^{s-1}$ induces an isomorphism of cochain complexes
%\[N^s_\subseteq \pi_t \calx\subseteq N^{s-1}_\subseteq \pi_t V\calx \to  N^{s-1}_\subseteq \pi_{t-1} (D^1\calx).\]
%\end{prop}
%Consequently, the connecting map induces an equivalence
%\[C^s_\subseteq \pi_t (D^1\calx)\to  C^s_\subseteq \pi_{t-1} (V\calx),\]
%and 
%\end{shaded}
\SubsectionOrSection{A modification of the functor $\Dendo^1$}
In this case, not only does
\[(V\calx)^s_t=(c(K^{\calc}Q^{\calc}c)^{s+2}X)_t\in cs\calc\]
have cosimplicial and simplicial structure maps,
but there is a cosimplicial simplicial algebra structure on the object $\overline{V}\calx$ obtained by omitting the leftmost replacement $c$:
\[(\overline{V}\calx)^s_t=((K^{\calc}Q^{\calc}c)^{s+2}X)_t\in cs\calc.\]
That is, we do not \emph{need} the outermost cofibrant replacement in order to define $\overline{V}\calx$, as in passing from $\calx$ to $V\calx$ one discards $d^0$. Of course, there is a $cs\calc$-map $\epsilon:V\calx \to \overline{V}\calx $ which is a weak equivalence in each cosimplicial level. Finally, the composite
\[\overline{v}:=\epsilon\circ v:\left(\calx\overset{v}{\to}V\calx\overset{\epsilon}{\epi} \overline{V}\calx\right)\]
is in each cosimplicial level a fibration in $s\calc$, as it is defined in cosimplicial degree $s$ by the formula \[\overline{v}=\eta:c(K^{\calc}Q^{\calc}c)^{s+2}X\to K^{\calc}Q^{\calc}c(K^{\calc}Q^{\calc}c)^{s+2}X.\]
 The object $\overline{V}\calx$ has two key advantages: $\overline{v}$ is a fibration in each cosimplicial level, and $\overline{V}\calx$ is a trivial object in $s\calc$ (i.e.\ it is in the image of $K^{\calc}$). This second property implies that  $\overline{V}\calx$ is an abelian group object in $s\calc$ in each cosimplicial level, as every vector space is a group object, and $K^{\calc}$ is a right adjoint. In other words, since all the structure maps in $\overline{V}\calx$ are trivial, they commute  with vector space addition. We write
\[\textup{add}:\overline{V}\calx\times \overline{V}\calx\to \overline{V}\calx\]
for the group operation. Under the identifications arising from propositions \ref{something about dualization} and \ref{smash prod}, the map $\textup{add}$ induces the expected abelian group and cogroup structures on $H_*^\calc\overline{V}\calx$ and $H^*_\calc\overline{V}\calx$:
\begin{gather*}
H_*^\calc\overline{V}\calx\times H_*^\calc\overline{V}\calx\to H_*^\calc\overline{V}\calx;\\
H^*_\calc\overline{V}\calx\sqcup H^*_\calc\overline{V}\calx\from H^*_\calc\overline{V}\calx.
\end{gather*}

The observation that $\overline{v}$ is a fibration leads us to define $\overline{\Dendo}^1\calx $ to be the strict fiber
\[\xymatrix@R=4mm{
\overline{\Dendo}^1\calx \ar[r]
\ar[d]^-{\delta}
&%r1c1
0\ar[d]\\
\calx \ar@{->>}[r]^-{\overline{v}}
&%r2c1
\overline{V}\calx %r2c2
}\]
There is a commuting diagram in $cs\calc$ (in which double-headed arrows denote maps which are fibrations in $s\calc$ in each cosimplicial level):
\[\xymatrix@R=4mm@C=15mm{
%0\ar[r]\ar[d]
%&%r1c1
0\ar[d]
\ar[r]
&%r1c2
\Lambda \overline{V}\calx\ar@{->>}[d]^-{\lambda}
&%r1c3
\Lambda V\calx\ar[l]^-{\Lambda(\epsilon)}
\ar@{->>}[d]^-{\lambda}
\\%r1c4
%0\ar[r]
%\ar[dr]_-{\mu}
%&%r2c1
\overline{V}\calx\ar@{=}[r]
&%r2c2
\overline{V}\calx&%r2c3
V\calx\ar@{->>}[l]^-{\epsilon}
\\%r2c4
%X\smashcoprod X\ar[r]^-{\mu}\ar[u]
%&%r3c1
\calx\ar@{->>}[u]_-{\overline{v}}
\ar@{=}[r]
&%r3c2
\calx\ar@{->>}[u]_-{\overline{v}}&%r3c3
\calx\ar[u]_-{v}\ar@{=}[l]%r3c4
\\
\makebox[0cm][r]{pullbacks:\qquad}%X\smashcoprod X\ar[r]^-{\overline{\mu}}
%&
\overline{\Dendo}^1\calx\ar[r]^-{\textup{$E_1$-eq}}
&
\smash{\widetilde{\Dendo}^1\calx}&
\ar[l]_-{\textup{$E_1$-eq}}
\Dendo^1\calx
}\]
producing a zig-zag of $E_1$-equivalences from $\overline{\Dendo}^1\calx$ to $\Dendo^1\calx$.  In each cosimplicial level, each of the objects in the top row is contractible, yielding homotopy long exact sequences, and the resulting connecting homomorphisms commute:
\[\mathclap{\xymatrix@R=4mm{
\pi_{t}(\overline{\Dendo}^1\calx)&%r1c3
\pi_{t+1}(\overline{V}\calx)\ar[l]_-{\partial_\textup{conn}}
\\%r1c4
\pi_{t}(\Dendo^1\calx)\ar@{-}[u]^-{\textup{zig-zag}}_-{\cong}
&%r2c3
\pi_{t+1}(V\calx)\ar[u]_-{\cong}
\ar[l]_-{\partial_\textup{conn}}
}}\]
so that there are isomorphisms of spectral sequences (starting from $E_1$):
\[\xymatrix@R=4mm{
\E{}{r}{\overline{\Dendo}^1\calx}{s}{t}\ar[r]^-{\cong}\ar[d]^{\cong}&%r1c1
\E{}{r}{\widetilde{\Dendo}^1\calx}{s}{t}\ar[d]^{\cong}&%r1c2
\E{}{r}{\widetilde{\Dendo}^1\calx}{s}{t}\ar[d]^{\cong}\ar[l]_-{\cong}\\%r1c3
\E{}{r}{\calx}{s+1}{t+1}\ar@{=}[r]&
\E{}{r}{\calx}{s+1}{t+1}&%r1c2
\E{}{r}{\calx}{s+1}{t+1}\ar@{=}[l]
}\]
%\begin{shaded}\tiny
%Bousfield and Kan explain that the composite
%\[N^\fraks\pi_\frakt(\Dendo^1X)\overset{\partial_\textup{conn}}{\from}N^\fraks\pi_{\frakt+1}(VX)\supset N^{\fraks+1}\pi_{\frakt+1}(X)\]
%is an isomorphism, where we note that the condition to lie in $N^{s+1}(O)\subseteq O^{s+1}$ is stricter than the condition to lie in $N^s(VO)\subseteq O^{s+1}$, for $O$ a cosimplicial module. Note that this composite is a cochain map, the reason being that the extra ``$d^0$'' in the coboundary out of $N^{\fraks+1}\pi_{\frakt+1}(X)$ that does not appear in the coboundary out of $N^{\fraks}\pi_{\frakt+1}(VX)$ becomes null after application of $\partial_\textup{conn}$, which is to say that the composite
%\[\pi_{\frakt+1}(X^{\fraks+1})\overset{d^0}{\to}\pi_{\frakt+1}(X^{\fraks+2})\overset{\partial_\textup{conn}}{\to}\pi_\frakt((\Dendo^1X)^{\fraks+1})\]
%is zero, which is clear, as we are just looking at adjacent maps in the homotopy long exact sequence.
%\end{shaded}
%
\SubsectionOrSection{Definition of the spectral sequence operations}
\label{Definition of the spectral sequence operations}
Whichever of the three categories of interest $\calc$ we are working in, there is a factorization
\[\vcenter{\xymatrix@R=4mm{
&%r1c1
\overline{\Dendo}^1\calx\ar[d]^-{\delta}
\\%r1c2
\quadgrad{2}F^{\calc}\calx\ar@{..>}[ur]
\ar[r]
&%r2c1
\calx%r2c2
}}
\textup{\qquad\quad induced by\qquad}
\vcenter{\xymatrix@R=4mm{
&\overline{\Dendo}^1\calx \ar[r]
\ar[d]^-{\delta}
&%r1c1
0\ar[d]\\
\quadgrad{2}F^{\calc}\calx\ar@{..>}[ur]
\ar[r]
&
\calx \ar@{->>}[r]^-{\overline{v}}
&%r2c1
\overline{V}\calx %r2c2
}}\]
where the composite $\quadgrad{2}F^{\calc}\calx\to \overline{V}\calx $ must vanish as it factors through the structure map $\quadgrad{2}F^{\calc}\overline{V}\calx\to \overline{V}\calx $, which is zero since $\overline{V}\calx$ is a  trivial object. We denote the resulting map of spectral sequences
\[L:\Edownup{r}{}{\quadgrad{2}F^{\calc}\calx}{s}{t}\to \Edownup{r}{}{\calx}{s+1}{t+1}.\]
Thanks to the isomorphisms
\[S_2V\cong \quadgrad{2}F^{\algs}V,\qquad \Lambda^2V\cong \quadgrad{2}F^{\liealgs}V,\qquad S^2V\cong \quadgrad{2}F^{\restliealgs}V,\]
and the various external spectral sequence operations from $\E{}{r}{V}{}{}$ to each of $\E{}{r}{S_2V}{}{}$, $\E{}{r}{\Lambda^2V}{}{}$ and $\E{}{r}{S^2V}{}{}$, we are now able to define numerous spectral sequence operations on $\Edownup{r}{}{\calx}{s}{t}$ in each case. When $\calc=\algs$, we define:
\begin{gather*}
\deltav_i:\left(\Edownup{r}{}{\calx}{s}{t} \smash{{}\overset{\delta_i^\textup{ext}}{\to}{}} \Edownup{r}{}{\quadgrad{2}F^\calc \calx}{s}{t+i}\overset{L}{\to} 
\Edownup{r}{}{\calx}{s+1}{t+i+1}\right),
\\
\Sqh^j:\left(\Edownup{r}{}{\calx}{s}{t}   \smash{{}\overset{\Sq^{j-1}_\textup{ext}}{\to}{}} \Edownup{r}{}{\quadgrad{2}F^\calc \calx}{s+j-1}{2t}\overset{L}{\to} 
\Edownup{r}{}{\calx}{s+j}{2t+1}\right),\\
\mu:\left(\Edownup{r}{}{\calx}{s}{t}\otimes \Edownup{r}{}{\calx}{s'}{t'}\smash{{}\overset{\mu_\textup{ext}}{\to}{}} \Edownup{r}{}{\quadgrad{2}F^\calc \calx}{s+s'}{t+t'}\overset{L}{\to} 
\Edownup{r}{}{\calx}{s+s'+1}{t+t'+1}\right),
\end{gather*}
with the $\deltav_i$ potentially multi-valued functions,  defined when $2\leq i\leq \max\{n,t-(r-1)\}$, and single-valued whenever $i\leq\min\{n+1,t+1-2(r-1)\}$, and the $\Sq^j$ potentially multi-valued functions with indeterminacy vanishing by $E_{2r-2}$, and which equal zero unless $\min\{t,r\}< j\leq s+1$.  All of the functions that are  defined on $E_2$ are single valued, so it makes sense to state 
\begin{prop}
\label{adams operations are right for comm}
When $\calc=\algs$,
under the identification $\E{}{2}{\calx}{s}{t}\cong H^*_{\calw(0)}H^*_{\algs}X$, the operations just defined coincide with the  ${\calw(0)}$-cohomology operations defined in \S\ref{Cohomology Operations for W and U}.
\end{prop}
\noindent We will prove this result in \S\ref{proof of prop: adams operations are right for comm}. It implies that from $E_2$ the operations just defined have the properties cataloged in propositions \ref{operations on goerss homology}, \ref{Wn Halg omnibus} and \ref{rearrange horiz and vert ops} --- the $\deltav_i$ satisfy the $\delta$-Adem relations, the $\Sqh^j$ and $\mu$ satisfy the properties of such operations on Lie algebra cohomology, and there is a commutation relation between the $\deltav$ and the $\Sqh$ and $\mu$. These relations persist to relations on the higher pages (modulo appropriate indeterminacy), but evidently do not hold on $E_1$.


The following results follow from propositions \ref{prop on basic product}, \ref{prop on e1 steens 1}, \ref{prop on e1 steens 2}, \ref{Prop on delta external} and \ref{Prop on einfty ops} respectively, for $X\in s\algs$ and $\calx\in cs\algs$ its Radulescu-Banu resolution:
\begin{cor}[(of Proposition \ref{prop on basic product})]
\label{prop on basic product composed with lift}
The pairing $\mu$ satisfies the Liebniz formula. For $r\geq2$, $\mu$ descends to the symmetric quotient $S_2\Edownup{r}{}{\calx}{}{}$.
\end{cor}


%\begin{shaded}\tiny
%\begin{cor}[(of Proposition \ref{prop on e1 steens 1})]
%%\label{prop on e1 steens 1 composed with lift}
%The operations $\Sqh^i:\Edownup{r}{}{\calx}{s}{t}\to \Edownup{r}{}{S_2\calx}{s+i}{2t+1}$ have indeterminacy vanishing by $\Edownup{2r-2}{}{S_2\calx}{s+i}{2t+1}$, and  no indeterminacy at $E_2$.
%Suppose that $x\in \Edownup{r}{}{\calx}{s}{t}$ and $r\geq2$. $\Sqh^ix$ survives to $\Edownup{2r-1}{}{S_2\calx}{s+i}{2t+1}$, and modulo indeterminacy: \[d_{2r-1}(\Sqh^ix)=\Sqh^{i+r-1}(d_rx).\]
%The notion of \emph{top operation} has shifted: $\Sqh^{s+1}x$ is the top operation, and equals to the product-square $x\times x$, and in particular, it has no indeterminacy. For $i>s+1$, $\Sqh^ix=0$. Whenever $t\geq2$, $\Sqh^2x=0$. $\Sqh^1x$ is zero whenever $t>0$. $\Sqh^0x=0$ for all $t$.
%\end{cor}
%
%
%\begin{cor}[(of Proposition \ref{prop on e1 steens 2})]
%%\label{prop on e1 steens 2 composed with lift}
%%Fix $i,s\geq0$ and set $r=1$. The chain level operation $\textup{SQ}^{i,s}$ defines an 
%At $E_1$, the operations $\Sqh^i:\Edownup{r}{}{\calx}{s}{t}\to \Edownup{r}{}{S_2\calx}{s+i}{2t+1}$ have no indeterminacy, and satisfy  $d_{1}(\Sqh^ix)=\Sqh^{i}(d_1x)$. Suppose that $x\in \Edownup{1}{}{\calx}{s}{t}$.
%The top operation $\Sqh^{s+1}x$ \emph{need not} equal the product-square $x\times x$ on $E_1$, and
%$\Sqh^{s+2}x$ \emph{need not} vanish, instead equalling $x\times d_1x$ on $E_1$.
%At least for $i>s+2$, $\Sqh^ix=0$. The results for the operations $\Sqh^0x$ and $\Sqh^{1}$ are unchanged, but there are no results to report for $\Sqh^{2}$ on $E_1$.
%\end{cor}
%\end{shaded}


\begin{cor}[(of Proposition \ref{prop on e1 steens 1})]
\label{prop on e1 steens 1 composed with lift}
The operations $\Sqh^i:\Edownup{r}{}{\calx}{s}{t}\to \Edownup{r}{}{S_2\calx}{s+i}{2t+1}$ have indeterminacy vanishing by $\Edownup{2r-2}{}{\calx}{s+i}{2t+1}$, and  no indeterminacy at $E_2$.
Suppose that $r\geq 2$ and $x\in\Edownup{r}{}{\calX}{s}{t}$. %Unless $\min\{t,r\}\leq i-1\leq s$, $\Sqh^ix=0$ (and is well defined without indeterminacy).
$\Sqh^ix=0$ unless $\min\{t,r\}< i\leq s+1$,  and this vanishing occurs without indeterminacy.
In any case, $\Sqh^ix$ survives to $\Edownup{2r-1}{}{\calX}{s+i}{2t+1}$, and the following equation in $\Edownup{2r-1}{}{\calX}{s+i+2r-1}{2t+2r-1}$ holds (without indeterminacy): \[d_{2r-1}(\Sqh^ix)=\Sqh^{i+r-1}(d_rx).\]

The notion of \emph{top operation} has shifted: $\Sqh^{s+1}x$ is the top operation, it equals the product-square $x\times x$, and in particular,  has no indeterminacy.
Whenever $t\geq2$, $\Sqh^2x=0$. $\Sqh^1x$ is zero whenever $t\geq1$. $\Sqh^0x=0$ for all $t$.
\end{cor}

\begin{cor}[(of Proposition \ref{prop on e1 steens 2})]
\label{prop on e1 steens 2 composed with lift}
%Fix $i,s\geq0$ and set $r=1$. The chain level operation $\textup{SQ}^{i,s}$ defines an 
At $E_1$, the operations $\Sqh^i:\Edownup{1}{}{\calx}{s}{t}\to \Edownup{1}{}{S_2\calx}{s+i}{2t+1}$ have no indeterminacy, and $d_{1}$ commutes with $\Sqh^ix $ for each $i$. Suppose that $x\in \Edownup{1}{}{\calx}{s}{t}$.
The top operation $\Sqh^{s+1}x$ \emph{need not} equal the product-square $x\times x$ on $E_1$, and
$\Sqh^{s+2}x$ \emph{need not} vanish, instead equalling $x\times d_1x$ on $E_1$.
At least for $i>s+2$, $\Sqh^ix=0$. The results for the operations $\Sqh^0x$ and $\Sqh^{1}$ are unchanged (so that they both vanish for $t\geq1$), but there are no results to report for $\Sqh^{2}$ on $E_1$.
\end{cor}


\begin{cor}[(of Proposition \ref{Prop on delta external})]
\label{cor on delta external composed with lift}
Suppose $r\geq1$, $x\in \E{}{r}{\calx}{s}{t}$ and $2\leq i\leq \max\{t-s,t-(r-1)\}$, so that $\deltav_ix $ is defined. Then $\deltav_i d_r x$ is defined, and 
\[d_r\deltav_i(x)+ \deltav_i(d_rx)=\begin{cases}
\Sqh^{t-i+2}(x),&\textup{if $i>t-s$ and $r =t-i+1$};\\
\mu(x\otimes d_rx),&\textup{if $i=t-s$,  $s=0$ and $r\geq2$};\\
%\Sqh^{t-i+1}(x),&\textup{if $t-s<i =t-(r-1)$}\\
0,&\textup{otherwise.}
\end{cases}\]
If $i\leq \max\{n,t+1-2(r-1)\}$, so that $\deltav_i x$ is single-valued, then so is $\deltav_i d_r x$, and this equation holds exactly. Otherwise, this equation holds modulo the indeterminacy of the right hand side. When $i>t-s$ and $r =t-i+1$ the set of values of the left hand side coincides with the set of values of the right hand side.
\end{cor}


\begin{cor}[(of Proposition \ref{Prop on einfty ops})]
\label{Prop on einfty ops composed with lift}
For $2\leq i\leq t-s$, the operations $\deltav_i:\E{}{\infty}{\calx}{s}{t}\to \E{}{\infty}{\calx}{s+1}{t+i+1}$ agree with the homotopy operations $\delta_i:\pi_{t-s}X\to \pi_{t-s+i}X$ on the target of the spectral sequence. Similarly, the product at $\E{}{\infty}{\calx}{}{}$ agrees with the product on the target.
\end{cor}



When $\calc=\liealgs$ or $\calc=\restliealgs$ there are operations
\begin{gather*}
\lambdav_i:\left(\Edownup{r}{}{X}{s}{t} \smash{{}\overset{\lambda_i^\textup{ext}}{\to}{}} \Edownup{r}{}{\quadgrad{2}F^\calc X}{s}{t+i}\smash{{}\overset{L}{\to} {}}
\Edownup{r}{}{X}{s+1}{t+i+1}\right),
\\
\Ph^{j}:\left(\Edownup{r}{}{X}{s}{t}   \smash{{}\overset{\Sq^j_\textup{ext}}{\to}{}} \Edownup{r}{}{\quadgrad{2}F^\calc X}{s+j}{2t}\smash{{}\overset{L}{\to} {}}
\Edownup{r}{}{X}{s+j+1}{2t+1}\right),\\
[\,,]:\left(\Edownup{r}{}{X}{s}{t}\otimes \Edownup{r}{}{X}{s'}{t'}\smash{{}\overset{\mu_\textup{ext}}{\to}{}} \Edownup{r}{}{\quadgrad{2}F^\calc X}{s+s'}{t+t'}\smash{{}\overset{L}{\to} {}}
\Edownup{r}{}{X}{s+s'+1}{t+t'+1}\right),
\end{gather*}
with the $\lambdav_i$ potentially multi-valued functions,  defined when $1\leq i\leq \max\{n,t-(r-1)\}$, and single valued whenever $i\leq\min\{n+1,t+1-2(r-1)\}$, and the $\Ph^j$ potentially multi-valued functions with indeterminacy vanishing by $E_{2r-2}$, and which equal zero unless $\min\{t,r\}\leq j\leq s$. We will not be able to prove a version of Proposition \ref{adams operations are right for comm} in the present work, since we have not derived a version of \S\ref{Cohomology Operations for W and U} for the categories $s\liealgs$ and $s\restliealgs$. Nonetheless, these operations will satisfy analogues of corollaries \ref{prop on basic product composed with lift}-\ref{Prop on einfty ops composed with lift}.

The purpose of rest of this chapter is to give the necessary constructions to  prove Proposition \ref{adams operations are right for comm}, so that in the following, we will work only in the category $\calc=\algs$. However, the constructions, including of the following two- and three-cell complexes, generalize to the categories of Lie algebras.



%\begin{enumerate}\squishlist
%\setlength{\parindent}{.25in}
%\item When $i<t-s$ or $i=t-s$ and $s>0$, $d_r\deltav_i(x)=\deltav_i(d_rx)$.
%\item When $i=t-s$, $d_r\deltav_i(x)=\begin{cases}
%\deltav_i(d_rx),&\textup{if }s>0;\\
%\deltav_i(d_rx)+x\times d_rx,&\textup{if }s=0.
%%\\,&\textup{if }
%\end{cases}$
%\item When $i>t-s$, $d_r\deltav_i(x)=\begin{cases}
%\deltav_i(d_rx),&\textup{if }r<t-i+1;\\
%\deltav_i(d_rx)+\Sq^{t-i+2}(x),&\textup{if }r=t-i+1.
%\end{cases}$
%\end{enumerate}
%
%\begin{enumerate}\squishlist
%\setlength{\parindent}{.25in}
%\item if $x\in \E{}{r}{\calx}{s}{t}$, $2\leq i< \max\{t-s,t-(r-1)\}$ then \[d_r\deltav_i(x)=\deltav_i(d_rx);\]
%\item if $x\in \E{}{r}{\calx}{s}{t}$, $s=0$, $i=t$ and $r=1$ then \[d_r\deltav_t(x)=\deltav_t(d_rx)+\mu(x\otimes d_rx);\]
%\item if $x\in \E{}{r}{\calx}{s}{t}$, $s>0$, $i>t-s$ and $r=t-i+1$ then \[d_r\deltav_t(x)=\deltav_t(d_rx)+\Sq^{t-i+2}(x).\]
%\end{enumerate}
%\[d_r\deltav_i(x)=\begin{cases}
%\deltav_i(d_rx),&\textup{if $2\leq i< \max\{t-s,t-(r-1)\}$};\\
%\deltav_i(d_rx),&\textup{if $i< t-s$ and $i<t-(r-1)$};\\
%\deltav_i(d_rx),&\textup{if $s>0$ and $i=t-s$};\\
%\deltav_i(d_rx)+\mu(x\otimes d_rx),&\textup{if $s=0$ and $i=t-s$};\\
%\deltav_i(d_rx)+\Sq^{t-i+2}(x),&\textup{if $s>0$, $i>t-s$ and $i=t-(r-1)$}.
%\end{cases}\]
%
%Suppose that $x\in \E{}{r}{\calx}{s}{t}$ and $2\leq i\leq \max\{t-s,t-(r-1)\}$, so that $\deltav_ix $ is defined.
%\[d_r\deltav_i(x)=\begin{cases}
%\deltav_i(d_rx),&\textup{if $i< t-s$ or $i<t-(r-1)$};\\
%\deltav_i(d_rx),&\textup{if $i=t-s$ and $s>0$};\\
%\deltav_i(d_rx)+\mu(x\otimes d_rx),&\textup{if $i=t-s$ and  $s=0$};\\
%\deltav_i(d_rx)+\Sq^{t-i+2}(x),&\textup{if $i>t-s$ and $i=t-(r-1)$}.
%\end{cases}\]
%
%
%\begin{cor}
%\label{cor on meaning of equation for ddeltai vert}
%Suppose that $x\in \E{}{r}{\calx}{s}{t}$ and $2\leq i\leq \max\{t-s,t-(r-1)\}$, so that $\deltav_ix $ is defined.
%\[d_r\deltav_i(x)=\begin{cases}
%\deltav_i(d_rx),&\textup{if $i< t-s$ or $i<t-(r-1)$};\\
%\deltav_i(d_rx),&\textup{if $i=t-s$ and $s>0$};\\
%\deltav_i(d_rx)+\mu(x\otimes d_rx),&\textup{if $i=t-s$ and  $s=0$};\\
%\deltav_i(d_rx)+\Sqv^{t-i+2}(x),&\textup{if $i>t-s$ and $i=t-(r-1)$}.
%\end{cases}\]
%\end{cor}
%
%\begin{thm}\hfil
%\begin{enumerate}\squishlist
%\setlength{\parindent}{.25in}
%\item When $i<t-s$, $d_r\deltav_i(x)=\deltav_i(d_rx)$.
%\item When $i=t-s$, $d_r\deltav_i(x)=\begin{cases}
%\deltav_i(d_rx),&\textup{if }s>0;\\
%\deltav_i(d_rx)+x\times d_rx,&\textup{if }s=0.
%%\\,&\textup{if }
%\end{cases}$
%\item When $i>t-s$, $d_r\deltav_i(x)=\begin{cases}
%\deltav_i(d_rx),&\textup{if }r<t-i+1;\\
%\deltav_i(d_rx)+\Sq^{t-i+2}(x),&\textup{if }r=t-i+1.
%\end{cases}$
%\end{enumerate}
%The operations $\deltav_i:E^\infty_{-s,t}\to E^\infty_{-s-1,t+i+1}$, for $i\leq t-s$, agree with the homotopy operations $\deltav_i:\pi_{t-s}\to \pi_{t-s+i}$ on the target of the spectral sequence, and similarly for the product at $E^\infty$.
%
%The $\deltav_i$ and $\Sq^j$ satisfy the expected Adem relations, and the relation $\deltav_i\Sq^jx=0$ (when?). The top Steenrod operation, $\Sq^{s+1}$, equals the squaring operation under the product. (more: identify every ingredient in \citeBOX[{?2.1}]{DwyerHtpyOpsSimpComAlg.pdf}, and in \citeBOX[{?5.3}]{PriddySimplicialLie.pdf}, including what happens on the axes.)
%
%For $x\in E^r_{-s,t}(\calx )$, $\Sq^i(x)$ survives to $E^{2r-1}$, and moreover, $d_{2r-1}(\Sq^ix)=\Sq^{i+r-1}(d_rx)$ modulo indeterminacy.  [The bottom operation, $\Sq^1(x)$, is zero whenever $t>0$\textbf{ silly if in connected case, should just say `is zero'}, or really just say exactly when everything is zero, page by page]. In particular, all Steenrod operations commute with $d_1$, in that $d_1\Sq^ix=\Sq^id_1x$.
%\end{thm}










\SubsectionOrSection{A chain-level construction $\xi_\textup{res}^\star$ inducing $\xi_{\HA{\calc}}$}\label{sec xires}
Let $\calc=\algs$. In \S\ref{chain level structure}, we defined
\[\xi_{\HA{\calc}}:B^{s+1}_{\HA{\calc}}H^*_{\calc}X\to B^{s}_{\HA{\calc}}H^*_{\calc}X\smashcoprod B^{s}_{\HA{\calc}}H^*_{\calc}X,\]
and in \S\ref{Horizontal Steenrod operations and a product for HWn} we used this map to define Steenrod operations and a product on $H^{*}_{\HA{\calc}}H^*_{\calc}X$. 
We will now construct, at the level of Radulescu-Banu's resolution, a map
\[\xi_\textup{res}^\star:\calx^s\Lsmashprod \calx^s\to V\calx^s,\]
which, under the isomorphisms of Theorem \ref{identify E2 with derived Q} and Proposition \ref{smash prod}, induces the map $\xi_{\HA{\calc}}$ on cohomology:
%\[\bigl(B^{s}_\calc H^*_\calc X\smashcoprod B^{s}_\calc H^*_\calc X\cong H^*_{\calc}(\calx^s\Lsmashprod \calx^s)\overset{\xi_{\HA{\calc}}}{\from} H^*_{\calc}(V\calx^s)\cong B^{s+1}_\calc H^*_\calc X\bigr): (\xi_\textup{res}^\star)^*\]
\[\xymatrix@R=4mm@C=20mm{
H^*_{\calc}(\calx^s\Lsmashprod \calx^s)&\ar[l]_-{(\xi_\textup{res}^\star)^*}
H^*_{\calc}(V\calx^s)\\%r1c2
B^{s}_\calc H^*_\calc X\smashcoprod B^{s}_\calc H^*_\calc X\ar[u]^-{\cong}&%r2c1
B^{s+1}_\calc H^*_\calc X\ar[l]_-{\xi_{\HA{\calc}}}\ar[u]^-{\cong}%r2c2
}\]


 %This map could \emph{not} be a cosimplicial map, as it  $(H_*^{\calc}X^\bullet)^{\times2}\to H_*^{\calc}VX^\bullet$, as it must increase cosimplicial degree by one. However, it \emph{does} induce a chain map on the \cAnCel{normalised?} complexes of primitives, as we might expect from \ref{psi is basically the quadratic part}.
We will need to abbreviate a little for the sake of compactness. Fix a cosimplicial degree $s$. Write $\calx$ for $\calx^s$, $V$ for $V\calx^s$, $\overline{V}$ for $\overline{V}\calx^s$, dots for categorical products, and superscripts for categorical self-products. There is a diagram
\[\newcommand{\Times}{{\cdot}}
\mathclap{\xymatrix@R=4mm{
\calx\Lsmashprod \calx
\ar@{-->}@/^1em/[rrrrd]^-(.9){\xi_\textup{res}^\star}
\\%r3c5
\ar[u]
\ar@/^1.3em/[rrrr]^-(.1){\overline{\xi}_\textup{res}^\star}
c(\calx^2)\ar[r]_-(.7){c(\epsilon,\textup{id})\circ\beta}
\ar@{..>}[dr]_-(.8){(\epsilon,\textup{id})}
&%r4c1
c(\calx^2\Times c(\calx^2))\ar[r]_-{c(\overline{v}^2\Times c(\textup{add}\circ\epsilon^2))}
\ar@{..>}[d]^-{\epsilon}
%\ar[ddd]
&%r4c2
c(\overline{V}^2\Times \calx)
\ar[r]_-{c(\textup{add}\Times \overline{v})}
\ar@{..>}[d]^-{\epsilon}
%\ar[ddd]
&%r4c3
c(\overline{V}\Times \overline{V})\ar[r]_-{c(\textup{add})}
\ar@{..>}[d]^-{\epsilon}
%\ar[dd]^-{\beta}
&%r4c4
V\ar@{..>}[d]^-{\epsilon}
\\%r4c5
c(\calx\sqcup \calx)\ar[u]
&\calx^2\Times c(\calx^2)
\ar@{..>}[r]^-{\overline{v}^2\Times c(\textup{add}\circ\epsilon^2)}
&
\overline{V}^2\Times \calx
\ar@{..>}[r]^-{\textup{add}\Times \overline{v}}
&
\overline{V}\Times \overline{V}\ar@{..>}[r]^-{\textup{add}}
&
\overline{V}
}}\]
in which we define $\overline{\xi}_\textup{res}^\star$ to be the composite of the horizontal solid arrows. The sub-diagram consisting of solid and dotted arrows strictly commutes, and we will define   $\xi_\textup{res}^\star$ to be the unique map up to homotopy such that the full diagram homotopy commutes, after showing that the composite $c(\calx\sqcup\calx)\to V$ is null.
The maps defined here need a little clarification, during which we will resume writing cosimplicial degrees:
\begin{gather*}
c(\epsilon,Id)\circ\beta:\bigl( c(\calx^s)^2\overset{\beta}{\to}cc(\calx^s)^2\overset{c(\Id,\Id)}{\to}c(c(\calx^s)^2\cdot c(\calx^s)^2)\overset{c(\epsilon\cdot \Id)}{\to}c((\calx^s)^2\cdot c(\calx^s)^2) \bigr)\\
c(\textup{add}\circ\epsilon^2):\bigl(c((\calx^s)^2)\overset{c(\epsilon^2)}{\to}c((\overline{V}\calx^{s-1})^2)\overset{c(\textup{add})}{\to} c(\overline{V}\calx^{s-1})=\calx^{s} \bigr).
\end{gather*}
Fortunately, the fact that the diagram (without the dashed arrow) commutes is obvious: the small triangle commutes by counitality of $\beta$, and the three squares commute by naturality of $\epsilon:c\to\Id$.
\begin{prop}
\label{res xi induces xi}
The map $\overline{\xi}_\textup{res}^\star$ induces the map $\overline{\xi}_{\HA{\calc}}$ on cohomology, and descends to a map ${\xi}_\textup{res}^\star:\calx\Lsmashprod\calx\to V $ as suggested by the dashed arrow above. This map induces the map ${\xi}_{\HA{\calc}}^\star$ on homology.
\end{prop}
\begin{proof}
Under the isomorphisms of propositions \ref{something about dualization} and \ref{smash prod}, if we apply $\pi^*(\dual Q^\calc(\DASH))$ to the solid maps in this diagram, we obtain (abbreviating $H^*_\calc$ to $H$):
\[\newcommand{\Times}{{\sqcup}}
\mathclap{\xymatrix@R=4mm{
(H\calx)^{\smashcoprod 2}
\ar@{->}@/_1em/[rrrrd];[]_-(.1){\xi_{\HA{\calc}}}
\\%r3c5
\ar@{->>}[u];[]
\ar@/_1.3em/[rrrr];[]_-(.91){\overline{\xi}_{\HA{\calc}}}
(H\calx)^{\sqcup2}\ar[r];[]^-{(\textup{id},\textup{id})}
&%r4c1
(H\calx)^{\sqcup2}\Times(H\calx)^{\sqcup2}\ar[r];[]^-{d_0\Times d_0\Times \phi_s}
%\ar[ddd];[]
&%r4c2
(HV)^{\sqcup2}\Times H\calx
\ar[r];[]^-{\phi_{s+1}\Times d_0}
%\ar[ddd];[]
&%r4c3
HV\Times HV\ar[r];[]^-{\phi_{s+1}}
%\ar[dd];[]_-{\beta}
&%r4c4
HV\\%r4c5
H\calx\times H\calx\ar@{ >->}[u];[]
}}\]
One observes that the horizontal composite is the very definition of $\overline{\xi}_{\HA{\calc}}$. We know from \S\ref{chain level structure} that $\overline{\xi}_{\HA{\calc}}$ factors through the smash coproduct, which is how we were able to fill in the dashed arrow on cohomology.

In order to obtain a map $\xi^\star_\textup{res}$, it is enough to check that the composite $c(\calx\sqcup \calx)\to  V$ is null. However, as $V$ is a GEM, a map into $V$ is null if and only if it is zero on cohomology. We have just stated that the map $\overline{\xi}_{\HA{\calc}}$ factors through $(H\calx)^{\smashcoprod 2}$, which is to say that the composite $HV\to H\calx\times H\calx$ is zero.
\end{proof}
This map $\overline{\xi}_\textup{res}^\star$ is very rich, but it will be important to note that postcomposition with $\epsilon$ destroys much of that richness. That is, reading off the dotted portion of the above commuting diagram:
\begin{lem}
\label{lemma: epsilon is destructive}
The map $\epsilon\circ \overline{\xi}_\textup{res}^\star$ equals the following sum in $\hom_{s\vect{}{}}  (c(\calx^2), \overline{V})$:
\[\bigl(\overline{v}\circ c(\textup{add})\circ c(\epsilon^2)\bigr)+\bigl(\overline{v}\circ\pi_1\circ\epsilon\bigr)+\bigl( \overline{v}\circ\pi_2\circ\epsilon\bigr),\]
where the $\pi_i$ are the two projections $\calx^{\times2}\to \calx$.
\end{lem}

\SubsectionOrSection{A three-cell complex with non-trivial bracket}\label{three cell complex}
Let $\calc=\algs$, and fix $t,t'\geq1$. There is a map $\mathbb{S}^{\calc}_{t+t'}\to \mathbb{S}^{\calc}_t\sqcup \mathbb{S}^{\calc}_{t'}$ sending the fundamental class $z_{t+t'}$ to the shuffle product of the two fundamental classes in the codomain:
\[z_{t+t'}\mapsto \mu(\nabla(z_t\otimes z_{t'})),\]
where $\mu$ is the structural pairing in $\algs$. Consider the complex $J_{t,t'}$ formed as the pushout:
\[\xymatrix@R=4mm{
\mathbb{S}^{\calc}_{t+t'}\ar[r]
\ar@{ >->}[d]
&%r1c1
\mathbb{S}^{\calc}_t\sqcup \mathbb{S}^{\calc}_{t'}\ar@{ >->}[d]
\\%r1c2
C\mathbb{S}^{\calc}_{t+t'}\ar[r]
&%r2c1
J_{t,t'}%r2c2
}\]
The left vertical is evidently almost free (and thus a cofibration), and thus constructed map $\mathbb{S}^{\calc}_t\sqcup \mathbb{S}^{\calc}_{t'}\to J_{t,t'}$ is an almost free map.
The generating subspace $V_{t+t'+1}\subseteq (J_{t,t'})_{t+t'+1}$ has a $t+t'+1$-dimensional generator $h_{t,t'}$, the image of the cone class $h$ in $(\mathbb{S}^{\calc}_{t+t'})_{t+t'+1}$ (c.f.\ \S\ref{The Dold-Kan correspondence}). Moreover, the object $J_{t,t'}$ is cofibrant, and $h_{t,t'}$ becomes a cycle in $Q^\calc J_{t,t'}$: $d_ih_{t,t'}=0$ for $i\geq1$, and $d_0h_{t,t'}:=z_{t+t'}$, which we  have identified with the decomposable element $\mu(\nabla(z_t\otimes z_{t'}))$ in passing to the pushout.

The homology long exact sequence shows that $H_*J_{t,t'}$ is three-dimensional, containing classes $ z_t$, $ z_{t'}$ and $h_{t,t'}$. Moreover, there is a co-operation $\Delta$ on $H_{*}^{\calc}$ dual to the $S(\LieOperad)$ structure map on cohomology, and we prove:
\begin{prop}
\label{prop on three cell}
Under $\Delta:H_*^{\calc}J_{t,t'}\to (S^2H_*^{\calc}J_{t,t'})_{*-1}$, $h_{t,t'}\mapsto\imath_t\otimes\imath_{t'}+\imath_{t'}\otimes\imath_t$. All other co-operations on $H_*^{\calc}J_{t,t'}$ are zero.
\end{prop}
\begin{proof}
The representative $g$ has the property that $d_0(g)=\mu(\Nabla(\imath_t\otimes\imath_{t'}))$ and $d_i(g)=0$ for $i>0$. By Lemma \ref{psi is basically the quadratic part} and the description of $\quadratic_\calc$ in \S\ref{quadratic part section}:
\[\psi_\calc(g)=
\quadratic_\calc(\mu(\Nabla(\imath_t\otimes\imath_{t'})))=
\trace(\Nabla(\imath_t\otimes\imath_{t'})))\in (S^2Q^\calc J_{t,t'})_{t+t'}.\qedhere\]
\end{proof}
%Now suppose that $L$ is a (not necessarily zero-square) algebra, and that $\alpha,\beta$ are homotopy classes in $\pi_n (L)$ and $\pi_m (L)$ respectively. Then we will be able to use the complex $J_{n,m}$ to construct a homology class $g_{\alpha,\beta}\in H_{n+m+1}(L)$ whose diagonal is $\alpha\otimes\beta+\beta\otimes\alpha$ [probably just when $L$ is zero square, using the map $L\times L\to L$. Else, get a class in $H(L^2)$ with diagonal $\overline{\alpha}\otimes\overline{\overline{\beta}}+ \overline{\overline{\beta}}\otimes\overline{\alpha}$]. Firstly, note that there is a commuting diagram
%\[\xymatrix@R=4mm{
%0\ar[r]
%\ar@{ >->}[d]
%&%r1c1
%0\ar[d]
%\\%r1c2
%S^n\sqcup S^m\ar[r]^-{\alpha\sqcup\beta}
%&%r2c1
%L^{\times2}%r2c2
%}\]
%which furnishes $c_1(L^{\times2})$ with a map from $S^n\sqcup S^m$. Next, there is a diagram %\newdir{ >}{{}*!/-7pt/\dir{>}}
%\[\xymatrix@R=4mm{
%S^{n+m}\ar@{->}[r]_-{\textup{EZ}^\mu(\imath_n,\imath_m)}
%\ar@{ >->}[d]
%&%r1c1
%S^n\sqcup S^m\ar@{ >->}[d]
%\ar[r]
%&c_1(L^{\times2})\ar[d]
%\\%r1c2
%CS^{n+m}\ar[r]\ar@/_1em/[rr]_-{0}
%&%r2c1
%C{n,m}\ar@{..>}[r]
%%r2c2
%&L^{\times2}}\]
%where the outer rectangle commutes since in $L^{\times2}$, $(a,0)\cdot(0,b)=(0,0)$. This gives a map $C_{n,m}\to c_2(L^{\times2})$ whose restriction to the subcomplex $S^n\sqcup S^m$ gives the homotopy classes $\alpha$ and $\beta$. The image $g_{\alpha,\beta}$ of the class $g\in(C_{n,m})_{n+m+1}$ is then an explicit representative for the homology class sought.

\SubsectionOrSection{A chain level construction of $j^\star_{\HA{\calc}}$}\label{chain level construction of j}
Let $\calc=\algs$. We can use the cofibration just defined to construct, at the chain level, the image under 
\[j^\star_{\HA{\calc}}:\Prim^{\HC{\calc}}_t(HY)\otimes \Prim^{\HC{\calc}}_{t'}(HZ)\to \Prim^{\HC{\calc}}_{t+t'+1}(HY\smashprod HZ)\]
of a tensor product $\alpha\otimes\beta$ of \emph{spherical} homology classes. Abbreviating $H_*^\calc$ to $H$ and $\Prim^{\HC{\calc}}$ to $\Prim$:
\begin{prop}
\label{prop on F and J}
There is a function
\[\overline{F}:\hom_{s\calc}(\mathbb{S}^{\calc}_t,Y)\times\hom_{s\calc}(\mathbb{S}^{\calc}_{t'},Z)\to \hom_{s\calc}(J_{t,t'},c(Y\times Z)),\]natural in $Y,Z\in s\calc$,
such that the function
\[F:\hom_{s\calc}(\mathbb{S}^{\calc}_t,Y)\times\hom_{s\calc}(\mathbb{S}^{\calc}_{t'},Z)\to \pi_{t+t'+1}(Q^\calc  c(Y\times Z))=:H^\calc_{t+t'+1}(Y\times Z)\]
defined by $F(\alpha,\beta):=H^\calc_*(\overline{F}(\alpha,\beta))(h_{t,t'})$ makes the following diagram commute:
\[\mathclap{\xymatrix@R=4mm{
%(H(Y\times Z)^{\otimes2})_{t+t'}\ar[d]^-{\textup{id}+\tau}
%&
%\ar@/_2em/@<-3ex>[dd]_-{(\textup{id}+T)\circ \textup{hur}^{\otimes2}}
%\ar[ddr]|(.34){(\textup{id}+T)\circ \textup{hur}^{\otimes2}}
%\ar[l]_-{\textup{hur}^{\otimes2}}
%\hom_{s\algs}(F^{\calc}\commentmathbb{K}_t,Y)\times\hom_{s\algs}(F^{\calc}\commentmathbb{K}_{t'},Z)
{\genfrac{}{}{0pt}{}{\hom_{s\calc}(\mathbb{S}^{\calc}_t,Y)\times}{\ \hom_{s\calc}(\mathbb{S}^{\calc}_{t'},Z)\phantom{\times}\!}\!\!}
\ar[d]^-{F}
\ar[r]^-{\textup{hur}^{\otimes2}}
&%r1c1
\ar[ddl]_(.34){\Id+T}
\Prim(HY)_t\otimes\Prim(HZ)_{t'}\ar[r]^-{j^\star_{\HA{\calc}}}
&%r1c2
\Prim(HY\smashprod HZ)_{t+t'+1}\ar@{ >->}[d]
\\%r1c5
%(H(Y\times Z)^{\otimes2})_{t+t'}
%&
H_{t+t'+1}(Y\times Z)\ar[r]%\ar[l]_-{\Delta}
\ar[d]^-{\Delta}
&%r2c1
(HY\times HZ)_{t+t'+1}\ar@{->>}[r]
\ar[d]^-{\Delta}
&%r2c2
(HY\smashprod  HZ)_{t+t'+1}
\\
%&
(S^2H(Y\times Z))_{t+t'}\ar[r]
&(S^2(HY\times HZ))_{t+t'}
}}\]
\end{prop}
\noindent The south-westerly arrow in this diagram is composite of the tensor product of the maps
\[\Prim(H^\calc_*Y)_t\subseteq H^\calc_tY\to H^\calc_t(Y\times Z)\ \textup{and}\ \Prim(H^\calc_*Z)_t\subseteq H^\calc_tZ\to H^\calc_t(Y\times Z)\]
followed by $(\Id+T):(H^\calc_*(Y\times Z))^{\otimes 2}\to S^2H^\calc_*(Y\times Z)$.
%The $\calc$-$H_*$-coalgebra $HY\times HZ$ is calculated as the direct sum $HY\oplus HZ$, and
%\[(HY\oplus HZ)^{\otimes 2}=(HY)^{\otimes 2}\oplus (HY\otimes HZ\oplus HZ\otimes HY)\oplus (HZ)^{\otimes 2}.\]
\begin{proof}
The value of $\overline{F}$ on $(\alpha,\beta)$ is defined as follows. Construct canonical lifts (c.f.\ \S\ref{Cofibrant replacement via the small object argument}):
\[\vcenter{\xymatrix@R=4mm{
%0\ar[r]
&%r1c1
c_1(Y\times Z)\ar[d]
\\%r1c2
\mathbb{S}^{\calc}_t\ar[ru]^-{\widetilde{(\alpha,0)}}\ar[r]_-{(\alpha,0)}
&%r2c1
Y\times Z%r2c2
}}
\textup{\quad and\quad }
\vcenter{\xymatrix@R=4mm{
%0\ar[r]
&%r1c1
c_1(Y\times Z)\ar[d]
\\%r1c2
\mathbb{S}^{\calc}_t\ar[ru]^-{\widetilde{(0,\beta)}}\ar[r]_-{(0,\beta)}
&%r2c1
Y\times Z%r2c2
}}\]
There is a commuting diagram (without the dashed arrow):
\[\xymatrix@R=4mm@C=15mm{
\mathbb{S}^{\calc}_{t+t'}\ar@{->}[r]^-{\mu(\nabla(z_t\otimes z_{t'}))}
\ar@{ >->}[d]
&%r1c1
\mathbb{S}^{\calc}_t\sqcup \mathbb{S}^{\calc}_{t'}\ar@{ >->}[d]
\ar[r]^{\widetilde{(\alpha,0)}\sqcup\widetilde{(0,\beta)}}
&c_1(Y\times Z)\ar[d]
\\%r1c2
C\mathbb{S}^{\calc}_{t+t'}\ar[r]\ar@/_1em/[rr]_-{0}
&%r2c1
J_{t,t'}\ar@{-->}[r]
%r2c2
&Y\times Z}\]
The reason that  the zero map $C\mathbb{S}^{\calc}_{t+t'}\to Y\times Z$ makes the outer square commute is that the composite $\mathbb{S}^{\calc}_{t+t'}\to Y\times Z$ vanishes, as it sends $z_{t+t'}$ to $\mu(\nabla((\alpha,0)\otimes (0,\beta)))=0\in Y\times Z$.

Corresponding to the right square is a map $J_{t,t'}\to c_2(Y\times Z)$, and the composite with the cofibration $c_2(Y\times Z)\to c(Y\times Z)$ is $\overline{F}(\alpha,\beta)$. This function $\overline{F}$ is evidently natural in $Y$ and $Z$, and so then is $F$.

The diagram consists of a square, a triangle and a hexagon.
The square commutes as the horizontal arrows are maps in $\HC{\calc}$, and we can see that the triangle commutes because we understand the $\HC{\calc}$ structure of $H_*(J_{t,t'})$ (and $H^\calc_*(\overline{F}(\alpha,\beta))$ is a map of $\calc$-$H_*$-coalgebras). As all of the maps in the hexagon are natural, we may check that it commutes on the universal example alone: 
\[(\imath_t,\imath_{t'})\in\hom_{s\calc}(\mathbb{S}^{\calc}_t,\mathbb{S}^{\calc}_t)\times\hom_{s\calc}(\mathbb{S}^{\calc}_{t'},\mathbb{S}^{\calc}_{t'}).\]
 That is, it is enough to check that the following hexagon, with a one element set at the top left entry, commutes:
\[\mathclap{\xymatrix@R=4mm@C=17mm{
\{(\imath_t,\imath_{t'})\}\ar[d]^-{F}
\ar[r]^-{\textup{hur}^{\otimes2}}
&%r1c1
\Prim_t(H\mathbb{S}^{\calc}_t)\otimes\Prim_{t'}(H\mathbb{S}^{\calc}_{t'})\ar[r]^-{j^\star_{\HA{\calc}}}
&%r1c2
\Prim_{t+t'+1}(H\mathbb{S}^{\calc}_t\smashprod H\mathbb{S}^{\calc}_{t'})\ar@{ >->}[d]^-{\textup{inc}}
\\%r1c5
H_{t+t'+1}(\mathbb{S}^{\calc}_t\times \mathbb{S}^{\calc}_{t'})\ar[r]^-{r}
&%r2c1
(H\mathbb{S}^{\calc}_t\times H\mathbb{S}^{\calc}_{t'})_{t+t'+1}\ar@{->>}[r]^-{\textup{proj}}
&%r2c2
(H\mathbb{S}^{\calc}_t\smashprod  H\mathbb{S}^{\calc}_{t'})_{t+t'+1}
}}\]
In this diagram, $j^\star_{\HA{\calc}}$ and $\textup{inc}$ are isomorphisms of 1-dimensional vector spaces, so it is enough to check that $r(F(\imath_t,\imath_{t'}))$ does not lie in the kernel of $\textup{proj}$, i.e.:
\[r(F(\imath_t,\imath_{t'}))\notin(H\mathbb{S}^{\calc}_t\sqcup H\mathbb{S}^{\calc}_{t'})_{t+t'+1} = H_{t+t'+1}\mathbb{S}^{\calc}_t\oplus H_{t+t'+1}\mathbb{S}^{\calc}_{t'}=0,\]
yet $\Delta(r(F(\imath_t,\imath_{t'})))=\imath_t \otimes\imath_{t'}+\imath_{t'}\otimes\imath_{t}\neq 0$, using the commuting square and triangle already established.
\end{proof}
We record here a useful calculation:
\begin{lem}
\label{handy lemma for conn hom}
For $\alpha:\mathbb{S}^{\calc}_t\to \calx^s$ and  $\beta:\mathbb{S}^{\calc}_{t'}\to \calx^s$, the composite 
\[\mathbb{S}^{\calc}_t\sqcup \mathbb{S}^{\calc}_{t'}\to J_{t,t'}\overset{\overline{F}(\alpha,\beta)}{\to}c(\calx^s \times \calx^s )\overset{c(\textup{add}\circ(\epsilon^2))}{\to}\calx^s \]
equals $\widetilde{\epsilon\alpha}\sqcup \widetilde{\epsilon\beta}$. In particular, $(c(\textup{add}\circ(\epsilon^2))\circ \overline{F}(\alpha,\beta))(\mu\nabla(\imath_t\otimes \imath_{t'}))=\overline{\mu}(\nabla(\widetilde{\epsilon\alpha} \otimes\widetilde{\epsilon\beta}))$.% under the composite $J_{t,t'}\to X$.
\end{lem}
\begin{proof}
We may calculate the restrictions to the two summands individually, and by symmetry, we need only consider:
%Using naturality of $\overline{F}$ and the fact that $h=c(\textup{add}\circ(\epsilon\times\epsilon))$:
\newcommand{\Times}{{\cdot}}
\[
\xymatrix@R=4mm{
\mathbb{S}^{\calc}_t\ar@{ >->}[r]&
J_{t,t'}\ar[r]^-{\overline{F}(\alpha,\beta)}
&%r2c1
c(\calx ^s\Times \calx ^s)\ar[r]^-{c(\epsilon\times\epsilon)}
&%r2c2
c(\overline{V}\calx ^{s-1}\Times \overline{V}\calx ^{s-1})\ar[r]^-{c(\textup{add})}
&%r2c3
c(\overline{V}\calx ^{s-1})\makebox[0cm][l]{\,$=\calx ^s.$}\phantom{=}%r2c4
}\]
The composite $J_{t,t'}\to c(\overline{V}\calx ^{s-1}\Times \overline{V}\calx ^{s-1})$ equals $\overline{F}(\epsilon\alpha,\epsilon\beta)$, due to the naturality of $\overline{F}$. By definition of $\overline{F}$, the composite $\mathbb{S}^{\calc}_t\to c(\overline{V}\calx ^{s-1}\Times \overline{V}\calx ^{s-1})$ equals $\widetilde{(\epsilon\alpha,0)}$. The naturality of the operation $\alpha\mapsto\widetilde{\alpha} $ of \S\ref{Cofibrant replacement via the small object argument} finishes the proof.
\end{proof}
\SubsectionOrSection{A two-cell complex with non-trivial $P^i$ operation}\label{two-cell complex for the deltas}
Let $\calc=\algs$. In this section, we give a construction of a two-cell complex whose cohomology has a $P^i$ connecting the two cells.
Fix $t,i$ with  $2\leq i \leq t$. There is a map $\mathbb{S}^{\calc}_{t+i}\to \mathbb{S}^{\calc}_t$ defined by
\[z_{t+i}\mapsto \mu(\nabla_{t-i}(z_t\otimes z_t)),\]
where $\mu$ is the structural pairing in $\calc$. Consider the complex $\Theta_{t,i}$ formed as the pushout:
\[\xymatrix@R=4mm{
\mathbb{S}^{\calc}_{t+i}\ar[r]
\ar@{ >->}[d]
&%r1c1
\mathbb{S}^{\calc}_t\ar@{ >->}[d]
\\%r1c2
C\mathbb{S}^{\calc}_{t+i}\ar[r]
&%r2c1
\Theta_{t,i}%r2c2
}\]
By the same observations as made in \S\ref{three cell complex}, this map is a cofibration, and $H_*^{\calc}\Theta_{t,i}$ has cohomology spanned by $\imath_t$ and $h_{t,i}$ in dimension $t+i+1$. For dimension reasons, $\imath_t$ is primitive. On the other hand:
\begin{prop}
\label{prop on two cell delta}
In $H^*_{\calc}\Theta_{t,i}$, $P^i\imath_t^*=h_{t,i}^*$.
\end{prop}
\begin{proof}
We will calculate the action of $(P^j)^*$ and $\Delta$ on $h_{t,i}$. By the same methods  as in the proof of Proposition \ref{prop on three cell}:
\[\psi_\calc(g)=\trace(\Nabla_{t-i}(\imath_t\otimes\imath_t)))\in (S^2Q^\calc \Theta_{t,i})_{t+i},\]
which represents $\sigma_i^\textup{ext}\imath_t$. so that the defining equation
\[(\psi_\calc)_*(h_{t,i})=\textstyle\sum_{j}\pi_*(1+T)(y_j\otimes z_j)+\sum_k\sigma_k((P^k)^*h_{t,i})\]
degenerates to $\sigma_i((P^k)^*h_{t,i})=\sigma_i^\textup{ext}\imath_t$.
\end{proof}

\SubsectionOrSection{A chain level construction of $\theta^\star_i$}
Let $\calc=\algs$, and recall the functions $\theta^\star_i$ of Proposition \ref{prop on thetaistar}.
\begin{prop}
\label{propOnKoszulDelta}
For $2\leq i\leq t$, there is a function
\[\overline{G}:\hom_{s\vect{}{}}(\mathbb{K}_t,W)\to \hom_{s\calc}(\Theta_{t,i},cK^{\calc}W),\]
natural in $W\in s\vect{}{}$, and satisfying $\overline{G}(\alpha)(\imath_t)=\widetilde{\alpha}$, such that the function
\[G:\hom_{s\vect{}{}}(\mathbb{K}_t,W)\to \pi_{t+i+1}(Q^{\calc}cK^{\calc}W)=:H_{t+i+1}(K^{\calc}W)\]
defined by $G(\alpha):=H_*(\overline{G}(\alpha))(h)$ equals the following composite whenever $2\leq i<t$:
\[\hom_{s\vect{}{}}(\mathbb{K}_t,W)\epi \pi_t W\overset{\textup{hur}}{\to} H_{t}(K^{\calc}W) \overset{\theta_i^\star}{\to}H_{t+i+1}(K^{\calc}W).\]
\end{prop}
%\begin{shaded}\tiny
%\[{{\xymatrix@R=4mm{
%\hom_{s\vect{}{}}(S^t,W)\ar[d]^-{G}
%\ar@{->>}[r]
%&%r1c1
%\pi_t W\ar[d]^-{\textup{hur}}
%\\
%H_{t+i+1}(KW)\ar@{=}[r]^-{\theta_i^*}
%&
%H_{t}(KW)
%}}} \cAnCel{{\xymatrix@R=4mm{
%\hom_{s\vect{}{}}(S^t,W)\ar[d]^-{G}
%\ar[r]^-{\textup{hur}}
%&%r1c1
%\pi_t W\ar[d]^-{\check P^i\otimes(\DASH)\makebox[0cm][l]{\,\textbf{?x?x?x?x when $i=t$}}}
%\\
%H_{t+i+1}(KW)\ar@{=}[r]
%&%r2c1
%(C\pi V)_{t+i+1}
%}}}\]
%\end{shaded}
\begin{proof}
The value of $\overline{G}$ on $\alpha$ is defined as follows. There is a commuting diagram %(since in $Y\times Z$, products of the form $(a,0)(0,b)$ vanish):
\[\xymatrix@R=4mm@C=15mm{
\mathbb{S}^\calc_{t+i}\ar@{->}[r]_-{\mu\Nabla_{t-i}(\imath_t\otimes\imath_{t})}
\ar@{ >->}[d]
&%r1c1
\mathbb{S}^\calc_t\ar@{ >->}[d]
\ar[r]^{\widetilde{\alpha}}
&c_1(KW)\ar[d]
\\%r1c2
C\mathbb{S}^\calc_{t+i}\ar[r]\ar@/_1em/[rr]_-{0}
&%r2c1
\Theta_{t,i}\ar@{-->}[r]
%r2c2
&KW}\]
Corresponding to the right square is a map $\Theta_{t,i}\to c_2(KW)$, and the composite with the cofibration $c_2(KW)\to c(KW)$ is $\overline{G}(\alpha)$. This function $\overline{G}$ is evidently natural in $W$, and so then is $G$.

The statement about $G$ is natural in $W\in s\vect{}{}$, so may be checked on the universal example $\imath_t\in \hom_{s\vect{}{}}(\mathbb{K}_t,\mathbb{K}_t)$. As $\overline{G}(\imath)$ is a map of $\calc$-$H_*$-coalgebras: $(P^i)^*G(\imath_t)=\textup{hur}(\imath_t)$; $(P^j)^*G(\imath_t)=0$ for all $j\neq i$ with $2\leq j\leq (t+i)/2$; and $\Delta G(\imath_t)$ vanishes unless $i=t$, in which case it equals $\imath\otimes \imath$. By construction, $G(\imath_t)$ lies in quadratic grading $2$ of the cofree construction:
\[G(\imath_t)\in\quadgrad{2}H_{t+i+1}(\mathbb{K}^{\calc}_t)=\quadgrad{2}C^{\HC{\calc}}\{\imath_t\}.\]
These conditions suffice to identify $G(\imath_t)$.
%and it is clear that the only element of quadratic grading $2$
%
%Under the homology coalgebra map $\overline{G}(\alpha)_*:H\Theta_{t,i}\to HKW$, we have $h\mapsto G(\alpha)$ and $\imath_t\mapsto\textup{hur}(\alpha)$. \textbf{Hopefully we can figure out enough about cofree homology coalgebras and their right $P$-action to see that $G(\alpha)$ can only equal the Koszul operation on $\textup{hur}(\alpha)$.} In the non-top case, we have zero coproduct and only $(\DASH)P^i$ firing, while in the top case we have coproduct $\textup{hur}(\alpha)\otimes \textup{hur}(\alpha)$ instead.
\end{proof}





\SubsectionOrSection{Proof of Proposition \ref{adams operations are right for comm}} 
\label{proof of prop: adams operations are right for comm}
Let $\calc=\algs$. Proposition \ref{adams operations are right for comm} follows immediately from the following two commutative diagrams. In each, the bottom row is that used to define the cohomology operations on the derived functors with which the $E_2$-page can be identified, and the top composite is that used to define the spectral sequence operations (after applying $N\dhor^*$ and using the inverse of the composite of Proposition \ref{BK D1 is awesome}).
\begin{prop}
\label{prop for product compat}
There is a commuting diagram (writing $\frakt=t+t'$):
\[\mathclap{\xymatrix@R=4mm{
\pi\uver_t(\calx^s )\otimes \pi\uver_{t'}(\calx^s )\ar[dd]^-{\textup{hur}^{\otimes 2}}
\ar[r]^-{\widetilde{\nabla}}
&%r1c1
\pi\uver_{\frakt}(\quadgrad{2}F^{\calc}\calx^s )\ar[r]^-{\overline{\mu}}
&%r1c2
\pi\uver_{\frakt}(\overline{\Dendo}^1\calx^s )&%r1c3
\pi\uver_{\frakt+1}(\overline{V}\calx^s )
\ar[l]_-{\partial_\textup{conn}}
\\%r1c4
&%r2c1
&%r2c2
\pi\uver_{\frakt}(\Dendo^1\calx^s )\ar@{-}[u]^-{\textup{zig-zag}}_-{\cong}
&%r2c3
\pi\uver_{\frakt+1}(V\calx^s )\ar[u]_-{\cong}
\ar[l]_-{\partial_\textup{conn}}
\ar[d]^-{\cong}
\\%r2c4
\Prim_t(H\calx^s )\otimes \Prim_{t'}(H\calx^s )\ar[r]^-{j^\star_{\HA{\calc}}}
&%r3c1
\Prim_{\frakt+1}(H\calx^s \smashprod H\calx^s )\ar[rr]^-{\xi^\star_{\HA{\calc}}}
&%r3c2
&%r3c3
\Prim_{\frakt+1}(HV\calx^s )%r3c4
}}\]
\end{prop}
\begin{prop}
\label{prop for delta compat}
Whenever $2\leq i<t$ there is a commuting diagram
\[\xymatrix@R=4mm{
\pi\uver_t\calx^s \ar[r]^-{\delta_i^\textup{ext}}
\ar[d]^-{\textup{hur}}
&%r1c1
\pi\uver_{t+i}(\quadgrad{2}F^{\calc}\calx^s )\ar[r]^-{\overline{\mu}}
&
\pi\uver_{t+i}(\overline{\Dendo}^1\calx^s )
\\%r1c2
H^\calc_t\calx^s \ar[r]^-{\theta^\star_i}&%r2c1
H^\calc_{t+i+1}\calx^s \ar@{=}[r]
&
\pi\uver_{t+i+1}(\overline{V}\calx^s )
\ar[u]^-{\partial_\textup{conn}}
}\]
\end{prop}
In the following proofs, we will write $\calx$ for $\calx^s$, $V$ for $V\calx^s$, $\overline{V}$ for $\overline{V}\calx^s$, $\overline{\Dendo}^1\calx$ for $\overline{\Dendo}^1\calx^s$, $H$ for $H_*^{\calc}$, $\pi_*$ for $\pi\uver_*$, $Q$ for $Q^\calc$ and $\Prim$ for $\Prim^{\HC{\calc}}$.
\begin{proof}[Proof of Proposition \ref{prop for product compat}]
It will help to modify and augment this diagram a little. Indeed, for each cardinality one subset $\{(\alpha,\beta)\}\subseteq \hom_{s\calc}(\mathbb{S}^\calc_t,\calx )\times\hom_{s\calc}(\mathbb{S}^\calc_{t'},\calx )$, there is a diagram:
\[\mathclap{\xymatrix@R=4mm{
\{(\alpha,\beta)\}
%\pi_t(X)\otimes \pi_{t'}(X)
\ar@{~>}[dd]_-{\makebox[0cm][r]{\,\scriptsize$j^\star_{\HA{\calc}}$\,}\circ(\textup{hur}^{\otimes 2})}
\ar[r]^-{\widetilde{\nabla}}
\ar@{..>}[ddddr]^-(.2){F}
&%r1c1
\pi_{\frakt}(\quadgrad{2}F^{\calc}\calx )\ar[r]^-{\overline{\mu}_*}
&%r1c2
\pi_{\frakt}(\overline{\Dendo}^1\calx )&%r1c3
\pi_{\frakt+1}(\overline{V}\calx )\ar[l]_-{\partial_\textup{conn}}
\ar@{~}[dd]^-{\textup{zig-zag}}_-{\cong}
\\%r1c4
&%r2c1
&%r2c2
\pi_{\frakt+1}(Q\overline{V}\calx )\ar@{..>}[ur]_-{=}&%r2c3
%\pi_{\frakt+1}(\calx ^{\fraks+1})\ar[u]_-{\cong}
%\ar[d]^-{\cong}
\\%r2c4
%\Prim_t(H\calx ^{\fraks})\otimes \Prim_{t'}(H\calx ^{\fraks})\ar[r]^-{j}
%r3c1
\Prim_{\frakt+1}(H\calx \smashprod H\calx )
\ar@{~>}[rrr]^-{\xi^\star_{\HA{\calc}}}
\ar@{ >->}[d]
%\ar@{ >->}[dr]
&
&%r3c2
&%r3c3
\Prim_{\frakt+1}(HV\calx )\ar@{ >->}[d]
\\%r3c4
(H\calx \smashprod H\calx )_{\frakt+1}\ar@{=}[r]
&\pi_{\frakt+1}Q(\textup{cof})
\ar[r]^-{\pi_*(Q\xi_\textup{res})}
&
\pi_{\frakt+1}QV\calx \ar@{=}[r]
\ar@{..>}[uu]_-(.75){\pi_*(Q\epsilon)}
&
H_{\frakt+1}V\calx 
\\
(H\calx \times H\calx )_{\frakt+1}\ar@{=}[r]\ar@{->>}[u]
&\pi_{\frakt+1}Qc(\calx \times \calx )\ar@{->>}[u]
\ar@{..>}[ur]_-(.6){\pi_*(Q\overline{\xi}_\textup{res})}
}}\]
Although all of the arrows in this modified diagram have already been defined, we've decorated some of them for emphasis.  It will be enough to check that for each $(\alpha,\beta)$, this modified diagram commutes, since the collection of such $(\alpha,\beta)$ will exhaust all of the pure tensors in $\pi_t(\calx )\otimes \pi_{t'}(\calx )$. What we need to prove is that the large rectangle consisting of wavy and solid arrows commutes.

The composite of the dotted maps equals the composite of the wavy maps, by results above.  That is, Proposition \ref{prop on F and J} states that the two composites $\{(\alpha,\beta)\}\to (H\calx \smashprod H\calx )_{\frakt+1}$ are equal. The content of Proposition \ref{res xi induces xi} is that the small triangle and square at the bottom of the diagram each commute, and the two composites $\Prim_{\frakt+1}(H\calx \smashprod H\calx )\to H_{\frakt+1}V\calx $ are equal. Finally, the two composites $\Prim_{\frakt+1}(HV\calx )\to \pi_{\frakt+1}(\overline{V}\calx )$ are equal, by Lemma \ref{hurewicz is a section}.

Thus the image of $(\alpha,\beta)$ under either the wavy or the dotted composite equals the image of $h_{t,t'}\in\pi_{\frakt+1} (QJ_{t,t'})$ under the composite
\[QJ_{t,t'}\overset{Q\overline{F}(\alpha,\beta)}{\to}Qc(\calx \times \calx )\overset{Q\overline{\xi}_\textup{res}}{\to}QV\calx \overset{Q\epsilon}{\to}Q\overline{V}\calx =\overline{V}\calx ,\]
which, by Lemma \ref{lemma: epsilon is destructive},  decomposes as the sum of the three maps $\overline{v}\circ c(\textup{add})\circ c(\epsilon^2)\circ\overline{F}(\alpha,\beta)$ and $\overline{v}\circ\pi_i\circ\epsilon\circ\overline{F}(\alpha,\beta)$ for $i=1$ and $2$.
%Indeed, write `$\textup{dot}$' for the composite of the dotted maps, and write $h_{t,t'}\in QJ_{t,t'}$ for the a representative of $h_{t,t'}\in\pi_{\frakt+1}(QJ_{t,t'})$. A representative for $\textup{dot}(\alpha,\beta)$ is given by the image of $h_{t,t'}$ under the composite
%\[QJ_{t,t'}\overset{Q\overline{F}(\alpha,\beta)}{\to}Qc(X^\fraks\times X^\fraks)\overset{Q\overline{\xi}_\textup{res}}{\to}QX^{\fraks+1}\overset{Q\epsilon}{\to}QX^{\fraks+1}_\textup{fl}=X^{\fraks+1}_\textup{fl},\]
%which is the sum of the images of $h_{t,t'}$ under the three composites
%\[\xymatrix@R=4mm{
%&
%&%r1c1
%QcA\ar@/^.55em/[dr]^-{Q\epsilon}
%&%r1c2
%&%r1c3
%&%r1c4
%\\%r1c5
%QJ_{t,t'}
%\ar[r]^-{Q\overline{F}(\alpha,\beta)}
%&Qc(A^2)\ar[rr]^-{Qh}
%\ar@/^.5em/[ur]^-{Qc(\pi_1)}
%\ar@/_.5em/[dr]_-{Qc(\pi_2)}
%&%r2c1
%&%r2c2
%QA\ar[r]^-{Q\eta}
%&%r2c3
%QK_+\makebox[0cm][l]{$\,=K_+$}%\ar@{=}[r]
%\\%r2c5
%&
%&%r3c1
%QcA\ar@/_.55em/[ur]_-{Q\epsilon}&%r3c2
%&%r3c3
%}\]
The composite $\pi_1\circ \epsilon\circ\overline{F}(\alpha,\beta):J_{t,t'}\to \calx $, by construction of $\overline{F}$, is the (dashed) map out of the pushout in the diagram:
\[\xymatrix@R=4mm@C=15mm{
\mathbb{S}^\calc_{\frakt}\ar@{->}[r]_-{\mu\nabla(\imath_t\otimes\imath_{t'})}
\ar@{ >->}[d]
&%r1c1
\mathbb{S}^\calc_t\sqcup \mathbb{S}^\calc_{t'}\ar@{ >->}[d]
\ar[dr]^{\alpha\sqcup0}
\\%r1c2
C\mathbb{S}^\calc_{\frakt}\ar[r]\ar@/_1em/[rr]_-{0}
&%r2c1
J_{t,t'}\ar@{-->}[r]
%r2c2
&\calx }\]
Now $h_{t,t'}$ is in the image of the map $C\mathbb{S}^\calc_{\frakt}\to J_{t,t'}$, and so maps to zero under the dashed map to $\calx $. Similarly, the composite $\pi_2\circ \epsilon\circ\overline{F}(\alpha,\beta)$ vanishes on $h_{t,t'}$. Thus, the image of $(\alpha,\beta)$ under the dotted composite is represented by
\[A:=(\overline{v}\circ c(\textup{add}\circ \epsilon^2)\circ\overline{F}(\alpha,\beta))(h_{t,t'}).\]
Consider
%We must show that $\partial_\textup{conn}(\textup{dot}(\alpha,\beta))=\overline{\mu}_*(\nabla(\alpha\otimes\beta))$.
%All of the previous discussion together has shown $\textup{dot}(\alpha,\beta)$ is the image in $Q\overline{V}X=\overline{V}X$ of the cycle $h_{t,t'}\in QJ_{t,t'}$ under the top row of 
the following commuting diagram:
\[\xymatrix@R=4mm@C=12mm{
\makebox[0cm][r]{$h_{t,t'}\in\,$}QJ_{t,t'}\ar[r]^-{Q\overline{F}(\alpha,\beta)}
&%r1c2
Qc(\calx \times\calx )\ar[r]^-{Qc(\textup{add}\circ\epsilon^2)}
&%r1c3
Q\calx \ar[r]^-{Q\overline{v}}
&%r1c4
Q\overline{V}\calx \makebox[0cm][l]{$\,\ni A$}\\%r1c5
\makebox[0cm][r]{$h_{t,t'}\in\,$}
J_{t,t'}\ar[r]^-{\overline{F}(\alpha,\beta)}
\ar@{->>}[u]
\ar[d]_-{d_0}
&%r1c2
c(\calx \times \calx )\ar[r]^-{c(\textup{add}\circ\epsilon^2)}
\ar@{->>}[u]
\ar[d]_-{d_0}
&%r1c3
\calx \ar@{->>}[r]^-{\overline{v}}
\ar@{->>}[u]
\ar[d]^-{d_0}
&%r1c4
\overline{V}\calx \makebox[0cm][l]{$\,\ni A$}\ar@{=}[u]
\\%r2c5
\makebox[0cm][r]{$\mu\nabla(\imath_t\otimes\imath_{t'})\in\,$}J_{t,t'}\ar[r]^-{\overline{F}(\alpha,\beta)}
&%r1c2
c(\calx \times \calx )\ar[r]^-{c(\textup{add}\circ\epsilon^2)}
&%r1c3
\calx \makebox[0cm][l]{$\,\ni\overline{\mu}(\nabla(\widetilde{\epsilon\alpha} \otimes\widetilde{\epsilon\beta}))$}
&%r1c4
%\\%r3c5
%\makebox[0cm][r]{$\mu\nabla(\imath_t\otimes\imath_{t'})\in\,$}\mathbb{S}^\algs_t\sqcup \mathbb{S}^\algs_{t'}\ar[urr]_-{\widetilde{\epsilon\alpha} \sqcup \widetilde{\epsilon\beta}}
%\ar@{ >->}[u]
%&%r4c2
%&%r4c3
%&%r4c4
%%r4c5
}\]
The element $h_{t,t'}\in N\uver_{\frakt+1}(J_{t,t'})$ may been used to populate the whole diagram as shown. To understand the images of $h_{t,t'}$ at either end of the bottom row, note that $d_0h_{t,t'}=\mu\nabla(\imath_t\otimes \imath_{t'})$ by construction, and Lemma \ref{handy lemma for conn hom} states that under the maps of bottom row, $\mu\nabla(\imath_t\otimes \imath_{t'})$ maps to $\overline{\mu}(\nabla(\widetilde{\epsilon\alpha} \otimes\widetilde{\epsilon\beta}))$.

%As $\alpha\sim \widetilde{\epsilon\alpha}$ and $\beta\sim \widetilde{\epsilon\beta}$, it is enough to prove that
The data in the bottom right corner of this diagram demonstrates that $\partial_\textup{conn}\overline{A}\in\pi_\frakt(\overline{\Dendo}^1\calx )$ is represented by ${\overline{\mu}(\nabla(\widetilde{\epsilon\alpha} \otimes\widetilde{\epsilon\beta}))}$,
which suffices, as $\alpha\sim \widetilde{\epsilon\alpha}$ and $\beta\sim \widetilde{\epsilon\beta}$.
%%%%%%
%%%%%%
%%%%%%By the expression $(\overline{v}\circ c(\textup{add}\circ \epsilon^2)\circ\overline{F}(\alpha,\beta))(h_{t,t'})$ we have meant the image of $h_{t,t'}$ under the composite
%%%%%%\[J_{t,t'}\to QJ_{t,t'} \to Qc(X^2)\to QX\to Q\overline{V}X=\overline{V}X.\]
%%%%%%The diagram shows that this is the image under $\overline{v}$ of the element $E=c(\textup{add}\circ\epsilon^2)\overline{F}(\alpha,\beta)(h_{t,t'})\in N_{\frakt+1}X$. Thus, the value of the required boundary homomorphism is represented by $d_0E$, which necessarily lies in the subspace $\overline{D}^1X$ of $X$.
%%%%%%
%%%%%%
%%%%%%
%%%%%%Moreover, at the right of this diagram we see exactly the maps needed in order to construct the connecting homomorphism for the short exact sequence defining $\overline{D}^1X^\fraks$.
%%%%%%%\[0\to\overline{D}^1X^\fraks\to X^\fraks\overset{\overline{v}}{\to} \overline{V}X^\fraks\to 0,\]
%%%%%%This confirms that $\partial_\textup{conn}(\textup{dot}(\alpha,\beta))$ is the image under $\widetilde{\epsilon\alpha} \sqcup \widetilde{\epsilon\beta}$ of $\mu\nabla(\imath_t\otimes\imath_{t'})$.
\end{proof}



\begin{proof}[Proof of Proposition \ref{prop for delta compat}]
Choose a representative $\alpha\in\hom_{s\calc}(\mathbb{S}^\calc_t, \calx )$. Then, setting $W=\overline{V}\calx^{s-1}$ in Proposition \ref{propOnKoszulDelta}, we obtain a map $\overline{G}(\epsilon\alpha):\Theta_{t,i}\to \calx $ such that $(\theta_i^\star\circ\textup{hur})(\alpha)$ is represented by
%\[\overline{G}(\epsilon\alpha)(h_{t,i})\in N_{t+i+1}(Q\calx ),\]
%or equivalently,
\[(\overline{v}\circ \overline{G}(\epsilon\alpha))(h_{t,i})
\in N\uver_{t+i+1}(\overline{V}\calx).\]
We populate the following commuting diagram using the element $h_{t,i}\in N\uver_{t+i+1}(\Theta_{t,i})$:
%\[\xymatrix@R=10mm@C=30mm@!0{
%\makebox[0cm][r]{$h_{t,i}\in\,$}(\Theta_{t,i})_{t+i+1}\ar[d]_-{d_0}\ar[r]^-{\overline{G}(\epsilon\alpha)}&%r1c1
%\calx _{t+i+1}\ar[d]_-{d_0}
%\ar[r]^-{\overline{v}}
%&%r1c2
%(\overline{V})_{t+i+1}
%\makebox[0cm][l]{$\,\ni (\eta\circ \overline{G}(\epsilon\alpha))(h_{t,i})$}
%\\%r1c3
%\makebox[0cm][r]{$\mu\Nabla_{t-i}(\imath_t\otimes \imath_t)\in\,$}(\Theta_{t,i})_{t+i}\ar[r]^-{\overline{G}(\epsilon\alpha)}
%&%r2c1
%\calx _{t+i}\makebox[0cm][l]{$\,\ni\mu\Nabla_{t-i}(\widetilde{\epsilon\alpha}\otimes \widetilde{\epsilon\alpha})$}
%&%r2c2
%}\]
%
\[\xymatrix@R=10mm@C=30mm@!0{
\makebox[0cm][r]{$h_{t,i}\in\,$}N\uver_{t+i+1}\Theta_{t,i}\ar[d]_-{d_0}\ar[r]^-{\overline{G}(\epsilon\alpha)}&%r1c1
N\uver_{t+i+1}\calx \ar[d]_-{d_0}
\ar[r]^-{\overline{v}}
&%r1c2
N\uver_{t+i+1}(\overline{V}\calx)
\makebox[0cm][l]{$\,\ni (\eta\circ \overline{G}(\epsilon\alpha))(h_{t,i})$}
\\%r1c3
\makebox[0cm][r]{$\mu\Nabla_{t-i}(\imath_t\otimes \imath_t)\in\,$}ZN\uver_{t+i}\Theta_{t,i}\ar[r]^-{\overline{G}(\epsilon\alpha)}
&%r2c1
ZN\uver_{t+i}\calx \makebox[0cm][l]{$\,\ni\mu\Nabla_{t-i}(\widetilde{\epsilon\alpha}\otimes \widetilde{\epsilon\alpha})$}
&%r2c2
}\]
Here, the value of $d_0h_{t,i}$ is known by definition of $\Theta_{t,i}$, 
and the fact that $\overline{G}(\epsilon\alpha)(\imath_t)=\widetilde{\epsilon\alpha}$ allows us to calculate $(\overline{G}(\epsilon\alpha)\circ d_0)(h_{t,i})$. Finally, in order to calculate $\partial_\textup{conn}(\theta_i^\star\circ\textup{hur})(\alpha)$, we find a preimage under $N\uver_{t+i+1}\calx \overset{\overline{v}}{\to}N\uver_{t+i+1}\overline{V}\calx$ of the representative $(\eta\circ \overline{G}(\epsilon\alpha))(h_{t,i})$, and then apply the differential $d_0$. We may use the preimage $\overline{G}(\epsilon\alpha)$, which maps to $\mu\Nabla_{t-i}(\widetilde{\epsilon\alpha}\otimes \widetilde{\epsilon\alpha})\in N\uver_{t+i}\overline{\Dendo}^1\calx $ under $d_0$. This is homotopic to $\mu\Nabla_{t-i}(\alpha\otimes \alpha)$, which represents $\overline{\mu}\delta^\textup{ext}_i(\alpha)$.
\end{proof}

\end{Operations on the Bousfield-Kan spectral sequence}




\begin{Comp funct sseqs}

\SectionOrChapter{Composite functor spectral sequences}
\label{Comp funct sseqs}
It will be important for us to identify the derived functors $H^*_{\calw(0)}X:=\dual(\mathbb{L}_*Q^{\calw(0)}X)$ for $X\in\calw(0)$, in order to determine the  $E_2$-page of the \BKSS\ for a connected simplicial commutative algebra. More generally, we will now present a spectral sequence whose goal is to calculate $H^*_{\calw(n)}X$ for $X\in\calw(n)$. This will be a \CFSS\ analogous to Miller's spectral sequence in \citeBOX[\S2]{MillerSullivanConjecture.pdf}. The factorization of $Q^{\calw(n)}$ we will use is of course 
\[Q^{\calw(n)}=\left(\calw(n)\overset{Q^{\calU(n)}}{\to}\calL(n)\overset{Q^{\calL(n)}}{\to}\vect{+}{n}\right)\]
There is an added challenge in this context --- indeed, the available factorization of $Q^{\calw(n)}$ is through a non-abelian category. Thus, the standard technology for \CFSSs\ does not apply, and we must use Blanc and Stover's methods \cite{Blanc_Stover-Groth_SS.pdf}. They observe that the left derived functors $\mathbb{L}_*Q^{\calU(n)}X$ are calculated as the homotopy groups of a simplicial object in $\calL(n)$, namely $Q^{\calU(n)}B^{\calw(n)}X$, and as such, they have the structure of a $\calL(n)\textup{-$\Pi$-algebra}$. That is, they form an object of $\calw(n+1)$.  After verifying that the functor $Q^{\calU(n)}$ satisfies the requisite acyclicity condition (indeed it preserves free objects), one may apply \cite[Theorem 4.4]{Blanc_Stover-Groth_SS.pdf}: there is a spectral sequence, with $E_r\in\vect{+}{n+2}$,
\[\E{2}{\BSW}{X}{t}{s_{n+2},\ldots,s_1}=((H_*^{\calw(n+1)})(\mathbb{L}_*Q^{\calU(n)})X)_{s_{n+2},\ldots,s_1}^t\implies ((H_*^{\calw(n)})X)_{s_{n+2}+s_{n+1},s_n,\ldots,s_1}^t\]
If $U^{\calw,\calU}:\calw\to\calU$ is the forgetful functor, resulting from the fact that an object of $\calw(n)$ is in particular an object of $\calU(n)$:
\begin{prop}
\label{what is LQU, dude?}
For $X\in s\calw(n)$, the groups $\mathbb{L}_*Q^{\calU(n)}X$ are isomorphic to $H_*^{\calU(n)}U^{\calw}_{\calU}X$, the $\calU(n)$-homology of the object of $s\calU(n)$ underlying $X$.
\end{prop}
\begin{proof}
We may take $X$ to be almost free in $s\calw(n)$, and calculate $\mathbb{L}_*Q^{\calU(n)}X$ simply as $\pi_*Q^{\calU(n)}X$. Then $X$, viewed as an object of $s\calU(n)$, is levelwise free, but potentially not almost free. We need to show then that $\pi_*Q^{\calU(n)}X$ does indeed calculate $H_*^{\calU(n)}X$ whenever $X\in s\calU(n)$ is \emph{levelwise} free, which is to say that the map $Q^{\calU(n)}B^{\calU(n)}X\to Q^{\calU(n)}X$ is a weak equivalence in $s\vect{}{}$. For this, $Q^{\calU(n)}B^{\calU(n)}X$ is the diagonal of the bisimplicial vector space $Q^{\calU(n)}B_q^{\calU(n)}X_p$, and we use the spectral sequence arising from filtering by $p$. As $X$ is levelwise free, the $E^1$-page is isomorphic to the chain complex $N_p(Q^{\calU(n)}X)$, concentrated in $q=0$.
\end{proof}
We will prefer to work with the dual spectral sequence, which has $E_r\in\vect{n+2}{+}$:
\[\E{\BSW}{2}{X}{s_{n+2},\ldots,s_1}{t}=((H^*_{\calw(n+1)})(H_*^{\calU(n)})X)^{s_{n+2},\ldots,s_1}_t\implies \Edown{0}{H^*_{\calw(n)}X}{s_{n+2},\ldots,s_1}{t}\]
These two spectral sequences are respectively the homotopy and cohomotopy spectral sequences of a certain object of $ss\vect{+}{n}$, with which we will need to work directly. Indeed, in \S\ref{The Blanc-Stover comonad sect}, we will define a comonad $\BSW$ on $s\calL(n)$, and, for  $X\in\calw(n)$,
we will use the object
\[Q^{\calL(n)}\BSWres L\in ss\vect{+}{n}\textup{ where } L:=Q^{\calU(n)}{ B^{\calw(n)}X}\in s\calL(n).\]
The identification of $E_2^\BSW$ follows from Lemma \ref{Q of a model} and propositions \ref{what is LQU, dude?} and \ref{Id of E2 grothen}.
% and define $L=Q^{\calU(n)}B^{\calw(n)}X$. Let $\BSWres L$ be the bisimplicial object obtained from the simplicial bar construction using the comonad $\BSW$. We are interested in the cohomotopy spectral sequence of the bisimplicial object
%\[Q^{\calL(n)}\BSWres L.\]
%Now $L$ is an almost free object of $\calL(n)$, with $\pi_*L=H_*^{\calU(n)}X$. Thus $\pi_*(\BSWres L)$ is an almost free resolution of $H_*^{\calU(n)}X$ in $s\calw(n+1)$. Moreover, $\pi_*(Q^{\calL(n)}\BSWres L)=Q^{\calw(n+1)}\pi_*(\BSWres L)$, so that
%\[E^2_{**}=H_*^{\calw(n+1)}H_*^{\calU(n)}X.\]



Before we do, we will recall Blanc and Stover's constructions, and imbue them with certain extra structure that will be reflected in the spectral sequence.

\SubsectionOrSection{The Blanc-Stover comonad in categories monadic over $\Ftwo $-vector spaces}\label{The Blanc-Stover comonad sect}
Fix an algebraic category $\calc$, monadic over a category of graded $\Ftwo $-vector spaces $\vect{}{}$. % --- although we intend to work over a category of graded vector spaces, but for now we opt for notational simplicity \textbf{do we really need to do that?}). 
As we are working over a category of vector spaces, rather than a category of graded sets, we can find further structure on  the following comonad on $s\calc$ defined by Blanc and Stover. While they  use the notation `$W$' in \cite{Blanc_Stover-Groth_SS.pdf} and `$\scrV$' in \cite{StoverVanKampen.pdf}, we will use the symbol `$\BSW$' to avoid notational confusion.
%\begin{shaded}\tiny
%It will help to fix a little notation. First, for $n\geq0$, let $\commentmathbb{K}^n\in \complexes \vect{}{}$ and $C\commentmathbb{K}^n\in \complexes \vect{}{}$ be the chain complexes
%\[(\commentmathbb{K}^n)_j=\begin{cases}
%\Ftwo \langle z\rangle,&\textup{if }j=n;\\
%0,&\textup{otherwise},
%\end{cases}\qquad 
%(C\commentmathbb{K}^n)_j=\begin{cases}
%\Ftwo \langle h\rangle,&\textup{if }j=n+1;\\
%\Ftwo \langle dh\rangle,&\textup{if }j=n;\\
%0,&\textup{otherwise},
%\end{cases}
%\]
%%with $d:(C\commentmathbb{K}^n)_{n+1}\to (C\commentmathbb{K}^n)_{n}$ the identity of $\Ftwo $.
%There is an evident inclusion $\imath:\commentmathbb{K}\to C\commentmathbb{K}$. For any $V\in\complexes\vect{}{}$, we have
%\[\hom_{\complexes\vect{}{}}(\commentmathbb{K}^n,V)\cong ZN_nV,\textup{ and }\hom_{\complexes\vect{}{}}(C\commentmathbb{K}^n,V)\cong N_{n+1}V,\]
%and the differential $d:N_{n+1}V\to ZN_nV$ corresponds to $\imath^*$ under these isomorphisms. If $N:s\vect{}{}\rightleftarrows \complexes \vect{}{}
%:\Gamma$ are the inverse equivalences of the Dold-Kan correspondence, define
%\[S^n_{\calc}:=F^\calc \Gamma(\commentmathbb{K}^n)\textup{ and }CS^n_{\calc}:=F^\calc \Gamma(C\commentmathbb{K}^n).\]
%Corresponding to the above structure, $S^n_{\calc}$ contains an element $z$ in dimension $n$, $CS^n_{\calc}$ contains an element $h$ in dimension $n+1$, there is a map $\imath:S^n_{\calc}\to CS^n_{\calc}$ sending $z$ to $d_0h$. For $L\in s\calc$, $\imath^*$ represents the differential under the isomorphisms
%\[\hom_{s\calc}(S_\calc^n,L)\cong ZN_nL,\textup{ and }\hom_{s\calc}(CS_\calc^n,L)\cong N_{n+1}L.\]
%\end{shaded}
In our context, Blanc-Stover's comonad $\BSW$, applied to $L\in s\calc$, is the pushout
%\[\xymatrix@R=4mm{
%\bigsqcup_{n\geq0}\bigsqcup_{N_{n+1}L}S_\calc^n
%\ar[r]\ar[d]^-{\bigsqcup\bigsqcup\imath}
%&%r1c1
%\bigsqcup_{n\geq0}\bigsqcup_{ZN_{n}L}S_\calc^n\ar[d]\\%r1c2
%\bigsqcup_{n\geq0}\bigsqcup_{N_{n+1}L}CS_\calc^n
%\ar[r]&%r2c1
%\BSW L%r2c2
%}\]
\[\xymatrix@R=4mm{
\coprod_{\substack{S\in\spheres{\calc}\\y:CS\rightarrow L}}S_{ y\circ\imath}
\ar[r]\ar[d]%^-{\bigsqcup\bigsqcup\imath}
&%r1c1
\coprod_{\substack{S\in\spheres{\calc}\\x:S\rightarrow L}}S_{ x}\ar[d]\\%r1c2
\coprod_{\substack{S\in\spheres{\calc}\\y:CS\rightarrow L}}CS_{ y}
\ar[r]&%r2c1
\BSW L%r2c2
}\]
The subscripts are just notation to distinguish multiple copies of $S$ and $CS$ for each sphere $S\in\spheres{\calc}$.
The top horizontal map sends the sphere $S_{ y\circ\imath}$ isomorphically onto \emph{itself}. The left vertical map is the coproduct of copies of the inclusion $\imath:S\to CS$. The effect of taking this pushout is to modify the coproduct $S_{ x}$ of spheres by attaching the  cone on $S_{ x}$  once for each nullhomotopy of $x\in L$.

It will be useful to write $h_y$ for the image in $N_*\BSW L$ of $h\in N_*CS_{ y}$, and similarly, $z_x$ for image in $ZN_*\BSW L$ of $z\in ZN_*S_{ x}$. Indeed, recalling the discussion in \S\ref{spheres and cones}, the data of $S\in\spheres{\calc}$ with a map $S\to L$ is equivalent to the data of a homogeneous normalized cycle of $L$, and similarly, $S\in\spheres{\calc}$ with a map $CS\to L$ is equivalent to a homogeneous normalized chain of $L$ which is \emph{not in dimension zero}. From this viewpoint, if we write $\homog (ZN_*L)$ for the homogeneous normalized cycles and $\homog (N_{\geq1}L)$ for the homogeneous normalized chains of $L$ not in dimension zero, the pushout may be written as
\[\xymatrix@R=4mm{
\coprod_{y\in \homog (N_{\geq1}L)}S_{ dy}
\ar[r]\ar[d]%^-{\bigsqcup\bigsqcup\imath}
&%r1c1
\coprod_{x\in \homog (ZN_*L)}S_{ x}\ar[d]\\%r1c2
\coprod_{y\in \homog (N_{\geq1}L)}CS_{ y}
\ar[r]&%r2c1
\BSW L%r2c2
}\]



We will now show that $\BSW L$ is homotopy equivalent to a coproduct of spheres. Indeed,  let \[\homog(BN_*L)=\im \bigl(d:\homog(N_{\geq1}L)\to \homog(ZN_*L)\bigr),\]
 and choose a section $f$ of the surjection $d:\homog(N_{\geq1}L)\epi \homog(BN_*L)$. Then $\BSW L$ contains a contractible subobject, the pushout
\[\xymatrix@R=4mm{
\coprod_{x\in \homog(BN_*L)}S_{ x}
\ar[r]\ar[d]%^-{\bigsqcup\bigsqcup\imath}
&%r1c1
\coprod_{x\in \homog(BN_*L)}S_{ x}\ar[d]\\%r1c2
\coprod_{x\in \homog(BN_*L)}CS_{ f(x)}
\ar[r]&%r2c1
C_0%r2c2
}\]
whose inclusion is a cofibration.  Then
\[\BSW L/C_0 \cong \left(\bigsqcup_{y\in \homog(N_{\geq1}L)\setminus\im (f)}CS_{ y}/S\right) \sqcup\left(\bigsqcup_{x\in \homog(ZN_*L)\setminus \homog(BN_*L)}S_{ x}\right)\]
where we have written `$A/B$' for the pushout of a cofibration $B\to A$ along the map $B\to0$, using the cofibrations $C_0\to \BSW L$ and $\imath:S\to CS_{ y}$.
As $CS/S$ is \emph{isomorphic} to the  sphere of one dimension higher than $S$ (consider the construction of \S\ref{The Dold-Kan correspondence}), this shows that $\BSW L$ is homotopic to a coproduct of spheres.

The promised comonad structure maps $\epsilon:\BSW L\to L$ and $\Delta:\BSW L\to \BSW^2L$ are determined by:
\[\epsilon(h_x)=x,\ \epsilon(z_y)=y,\ \Delta(h_x)=h_{h_x},\textup{ and }\Delta(z_y)=z_{z_y}\textup{ for $x\in N_{n+1}L$ and $y\in ZN_nL$.}\]
We would like to find a sub\emph{space} of $\pi_*(\BSW L)$ which freely generates it as a $\calc\textup{-$\Pi$-algebra}$. Even better, we have the following rendition of an observation used in  \cite[Proof of theorem 4.2]{Blanc_Stover-Groth_SS.pdf}. We give the proof since we will need to be explicit about some parts of it in what follows.
\begin{prop}
\label{Id of E2 grothen}
For $L\in s\calc$, $\pi_*(B^{\BSW}L)$ is an almost free (monadic over $\vect{}{}$) simplicial $\calc$-$\Pi$-algebra weakly equivalent to $\pi_*L$.
\end{prop}
\noindent This differs from the observation in \cite[Proof of theorem 4.2]{Blanc_Stover-Groth_SS.pdf}, in that we show that all the structure maps of $\pi_*(B^{\BSW}L)\in s\PA{\calc}$ except for $d_0$ preserve \emph{vector spaces} of generators, rather than \emph{sets} of generators.
\begin{proof}
That the augmentation to $\pi_*L$ is a weak equivalence follows from Stover's result \citeBOX[2.7]{StoverVanKampen.pdf}. The only change from Blanc-Stover is that $\pi_*(B^{\BSW}L)$ is almost free over the category $\vect{}{}$, rather than the category of pointed sets.

During this proof, for any set $A$ we will write $\Ftwo \{A\}$ for the vector space generated by the symbols $\underline{a}$ for $a\in A$. Suppose that $M\in s\calL(n)$. There is a natural map $d_*:\Ftwo \{\homog( N_{\geq1}M)\} \to \Ftwo \{\homog( ZN_*M)\} $, and a natural monomorphism $\alpha:\ker(d_*)\to \pi_*(\BSW M)$, defined by
\[\alpha(\underline{x_1}-\underline{x_0})=\overline{h_{x_1}-h_{x_0}},\textup{ for $x_1,x_2\in N_{\geq1}M$ with $dx_1=dx_2$}.\]
Moreover, there is a natural map $\beta:\Ftwo \{\homog( ZN_*M)\}\to\pi_*(\BSW M)$ (which is not monomorphic) defined by
\[\beta(\underline{x})=\overline{z_{x}},\textup{ for $x\in \homog( ZN_*M)$}.\]
From the above expression for $\BSW M/C_0$, one sees that $\im (\alpha)$ and $\im (\beta)$ are linearly independent subspaces of $\pi_*(\BSW M)$, and that $\pi_*(\BSW M)$ is free on $\im (\alpha)\oplus\im (\beta)$. Moreover, if $M\to M'$ is a map in $s\calL(n)$, then the generating subspaces are preserved by the induced map $\pi_*\BSW M\to \pi_*\BSW M'$.

Applying this analysis to $\pi_*B^\BSW L\in s\PA{\calc}$, every face and degeneracy map except for $s_0$ and $d_0$ preserves the generators.
In order to check that $s_0$ preserves generators, we must see that the comonad diagonal of $\BSW$ sends the subspaces $\im (\alpha_L)$ and $\im (\beta_L)$ into the subspaces $\im (\alpha_{\BSW L})$ and $\im (\beta_{\BSW L})$. That $\im (\beta_L)$ maps into $\im (\beta_{\BSW L})$ is immediate. For $\im (\alpha_L)$, the image of $h_{x_1}-h_{x_0}$ under the diagonal is $h_{h_{x_1}}-h_{h_{x_0}}$, which is in $\im(\alpha_{\BSW L})$, since $dh_{x_1}=z_{dx_1}=z_{dx_0}=dh_{x_0}$.
\end{proof}



%There is a natural map $\Ftwo [d]:\Ftwo [N_{n+1}X]\to \Ftwo [ZN_nX]$, and a natural monomorphism $\alpha:\ker(\Ftwo [d])\to\pi_*(\BSW X)$. Indeed, writing $\underline{z}$ for the generator of $\Ftwo [N_{n+1}X]$ corresponding to $z\in N_{{n+1}X}$, the group $\ker(\Ftwo [d])$ is generated by differences $\underline{z}-\underline{z}'$ for $z,z'\in N_{n+1}X$ having $dz=dz'$, and we define $\alpha(\underline{z}-\underline{z}')$ to be the homotopy class of the difference of the cones corresponding to $z$ and $z'$, whose boundaries have been identified in the colimit. Moreover, there is a natural map $\beta:\Ftwo [ZN_nX]\to\pi_*(\BSW X)$ (which is not monomorphic), which sends a generator $z\in ZN_nX$ to the homotopy class in $\BSW X$ of the corresponding summand in the top right corner of the colimit diagram. From the above expression for $\BSW X/C_0$, one sees that $\im (\alpha)$ and $\im (\beta)$ are linearly independent subspaces of $\pi_*(\BSW X)$, and $\pi_*(\BSW X)$ is free on $\im (\alpha)\oplus\im (\beta)$. From this description it is clear that the generating subspaces are preserved by maps in the image of $\BSW $.
%
%Moreover, the diagonal of the comonad also preserves the subspaces $\im (\alpha)$ and $\im (\beta)$. That $\im (\alpha)$ is preserved is evident from the definitions. For $\im (\beta)$, note that $\im (\beta)\subseteq\pi_*(\BSW X)$ is spanned by terms $D_h-D_{h'}$ for $h,h':CS\to X$ satisfying $h\imath=h'\imath$. The diagonal applied to $D_h-D_{h'}$ may be written $D_{D_h}-D_{D_{h'}}$, and one notes that $D_h\imath=S_{h\imath}=S_{h'\imath}=D_{h'}\imath$.
%
%\textbf{Move this para: }Blanc and Stover explain that $\BSW $ actually has the structure of a comonad on $s\calL(n)$. As such, for any simplicial Lie algebra $L\in s\calL(n)$, there is a bisimplicial object $B^\BSW L$, the bar construction using the comonad $\BSW $, given by $B_{s_2}^\BSW L_{s_1}=(\BSW^{s_2+1}L)_{s_1}$. Flesh out the point here...
%
%
%Now the utility of the comonad structure lies in resolving an object $L\in s\calL(n)$ by the comonadic bar construction, $B^\BSW(L)$. This discussion shows that, except for $d_0$, all the maps in $B^\BSW L$ preserve the subspaces of generators of $\pi_*(B^\BSW L)$, which is to say that this object of $s\calw(n+1)$ is almost free. Moreover, according to [Stover 2.6], the augmentation map $\pi_*(B^\BSW L)\to \pi_*(L)$ is a weak equivalence in $s\calw(n+1)$. Thus, this Blanc-Stover $\BSW $-construction provides an almost-free replacement of $\pi_*(L)$ in $s\calw(n+1)$.





\SubsectionOrSection{A chain-level diagonal on the $\BSW $ construction}
\label{Subsection: Chain level diagonal}
We have seen, for $M\in s\calc$, that $\pi_*(\BSW M)$ is a free object in $\PA{\calc}$. As such, there is a diagonal
\[\phi_{\PA{\calc}}:\pi_*(\BSW M)\to \pi_*(\BSW M)\sqcup \pi_*(\BSW M).\]
In this section, we will describe how $\phi_{\PA{\calc}}$ is the map on homotopy induced by a morphism $\phi_\BSW :\BSW M\to \BSW M\sqcup \BSW M$ in $s\calc$, and construct a map $\xi_{\BSW}$ related to the map $\xi_{\PA{\calc}}$ of \S\ref{chain level structure}. % (so that $i\circ\phi_{\PA{\calc}}=\pi_*(\phi_{\BSW})$, where $i$ is the natural isomorphism $\pi_*(\BSW L)\sqcup\pi_*(\BSW L)\to \pi_*(\BSW L\sqcup \BSW L)$).

In order to construct a map $\phi_\BSW $, each $S\in\spheres{\calc}$ equals $S=F^\calc\mathbb{K}$ for some $\mathbb{K}$ as in \S\ref{The Dold-Kan correspondence} (with indices omitted), and we construct a commuting diagram:
\[\xymatrix@R=4mm{
S\ar[r]^-{\phi_1}
\ar[d]^-{\imath}
&%r1c1
S\sqcup S\ar[d]^-{\imath\sqcup\imath}
\\%r1c2
CS\ar[r]^-{\phi_2}
&%r2c1
CS\sqcup CS%r2c2
}\qquad \textup{\raisebox{-6mm}{by applying $F^\calc$ to }}\qquad \xymatrix@R=4mm{
\mathbb{K}\ar[r]^-{\Delta}
\ar[d]^-{\imath}
&%r1c1
\mathbb{K}\oplus \mathbb{K}\ar[d]^-{\imath\sqcup\imath}
\\%r1c2
C\mathbb{K}\ar[r]^-{\Delta}
&%r2c1
C\mathbb{K}\oplus C\mathbb{K}%r2c2
}\]
The maps $\phi_1$ and $\phi_2$ can then be applied respectively to all of the sphere and cone classes appearing in $\BSW M$.
%In fact, this diagram can even be formed in $s\vect{+}{n}$, before application of $F^\calc :\vect{+}{n}\to\calc$, and $\phi_1$ and $\phi_2$ are just the levelwise application of the diagonal map on a simplicial vector space. 
To understand the effect of $\phi_{\BSW }$ on homotopy, it is enough to identify where the generators of $\pi_*(\BSW M)$ are sent in $\pi_*(\BSW M)\sqcup\pi_*(\BSW M)$, which is easy. The theory of this map mimics that presented in \S\ref{chain level structure}, as intended, and we list some of its properties here, with proofs omitted.
\begin{lem}
$\BSW M$ is naturally a (strict) commutative cogroup object, having comultiplication map $\phi_{\BSW }$, counit map $0:\BSW M\to 0$, and inverse map $\Id:\BSW M\to \BSW M$. In particular, $\hom(\BSW M,\DASH)$ takes values in $\ensuremath{\Ftwo }$-vector spaces.
\end{lem}
%\begin{proof}
%There are a few axioms to check, for example, left counitality: that the composite 
%\[\xymatrix@R=4mm@1{
%\BSW L\ar[r]^-{\phi_{\BSW }}
%&%r1c1
%\BSW L\sqcup \BSW L\ar[r]^-{\Id\sqcup0}
%&%r1c2
%\BSW L%r1c3
%}\] is the identity. This follows since $(\Id\sqcup0)\phi_1$ is the identity of $S^n$ and $(\Id\sqcup0)\phi_2$ is the identity of $CS^n$. The other axioms follow similarly.
%\end{proof}
Writing $\star$ for the group operation on $\hom_{s\calc}(\BSW M,M')$, we have the following:
\begin{lem}
For maps $f,g:\BSW M\to M'$ we have 
\[Q^{\calc}(f\star g)=(Q^{\calc}f+Q^{\calc}g):Q^{\calc}(\BSW M)\to Q^{\calc}M'.\]
\end{lem}
\begin{proof}
It is enough to check that $Q^{\calc}(\phi_\BSW ):Q^{\calc}(\BSW M)\to Q^{\calc}(\BSW M\sqcup \BSW M)$ equals the diagonal map $Q^{\calc}(\BSW M)\to Q^{\calc}(\BSW M)\oplus Q^{\calc}(\BSW M)$. For this, $Q^{\calc}$ converts all the colimits involved in the construction of $\BSW M$ to direct sums of simplicial vector spaces, and $Q^{\calc}\phi_1$ and $Q^{\calc}\phi_2$ are both precisely the diagonal map.
\end{proof}

Now let $\overline{\xi}_{\BSW }$ denote the following composite:
%\[W^2X\overset{\phi_{\BSW }}{\to}(W^2X)^{\sqcup2}\overset{\phi_{\BSW }\sqcup \epsilon}{\to}(W^2X)^{\sqcup2}\sqcup(WX) \overset{\epsilon^{\sqcup2}\sqcup\phi_{ch}}{\to}((WX)^{\sqcup2})^{\sqcup2}\overset{\textup{fold}}{\to}(WX)^{\sqcup2}\]
\[\makebox[0mm][r]{$\overline{\xi}_{\BSW }:\ $}\BSW^2M\overset{\phi_{\BSW }}{\to}(\BSW^2M)^{\sqcup2}\overset{a\sqcup b}{\to}(\BSW M)^{\sqcup2}\]
where $a,b:\BSW^2M\to(\BSW M)^{\sqcup2}$ are the composites
\[\xymatrix@R=0mm{
\makebox[0mm][r]{$a:\ $}\BSW^2M\ar[r]^-{\phi_{\BSW }}
&%r1c1
(\BSW^2M)^{\sqcup2}\ar[r]^-{\epsilon^{\sqcup2}}
&%r1c2
(\BSW M)^{\sqcup2}\\
\makebox[0mm][r]{$b:\ $}\BSW^2M\ar[r]^-{\epsilon}
&%r1c1
(\BSW M)\ar[r]^-{\phi_{\BSW }}
&%r1c2
(\BSW M)^{\sqcup2}
}\]
Thanks to Lemma \ref{zerocomposite}, $\overline{\xi}_{\BSW }$ factors through the smash coproduct, defining a natural map
\[\xi_{\BSW }:\BSW^2M\to (\BSW M)^{\smashcoprod 2}.\]
\begin{lem}
\label{zerocomposite}
The composite 
$\BSW^2M\overset{\overline{\xi}_{\BSW }}{\to}(\BSW M)^{\sqcup2}\to(\BSW M)^{\times2}$ is zero.
\end{lem}
\begin{proof}
This follows from the observation that both composites $(\Id\sqcup0)\overline{\xi}_{\BSW }$ and $(0\sqcup\Id)\overline{\xi}_{\BSW }$ equal $\epsilon:\BSW^2M\to \BSW M$.
\end{proof}
The desired property for $\xi_{\BSW}$ is then the following lemma (involving the natural isomorphism $i$ of Proposition \ref{smash coprod}, and the almost free structure given in Proposition \ref{Id of E2 grothen}). 
\begin{lem}
\label{LemmaOn xi}
For $L\in s\calc$, we have $(i\circ \xi_{\PA{\calc}})=\pi_*(\xi_{\BSW})$, i.e.\ a commuting diagram:
\[\xymatrix@R=4mm@C=17mm{
\pi_*(B^\BSW_s L)\ar[r]^-{\pi_*(\xi_\BSW)}
\ar[dr]_-{\xi_{\PA{\calc}}}
&%r1c1
\pi_*((B^\BSW_{s-1} L)^{\smashcoprod2})\\%r1c2
&%r2c1
(\pi_*(B^\BSW_{s-1} L))^{\smashcoprod2}\ar[u]^-{i}_-{\cong}
%r2c2
}\]
\end{lem}
\begin{proof}
In view of the short exact sequences of Proposition \ref{smash coprod}, this is equivalent to $(i\circ \overline{\xi}_{\PA{\calc}})=\pi_*(\overline{\xi}_{\BSW}):\pi_*(\BSW^{s+1}L)\to\pi_*((\BSW^{s}L)^{\sqcup 2})$, which holds as $(i\circ\phi_{\PA{\calc}})=\pi_*(\phi_{\BSW})$.
\end{proof}
\begin{lem}
$(d_i)^{\smashcoprod 2}\xi_\BSW =\xi_\BSW d_{i+1}$ for $i\geq1$, and $(d_0)^{\smashcoprod 2}\xi_\BSW = (\xi_\BSW d_{0})\star(\xi_\BSW d_{1})$, so that the map $Q^{\calc}\xi_\BSW $ induces a degree $(-1,0)$ bicomplex map:
\[N_*N_*(Q^{\calc}B^\BSW_{\bullet}L)_{s_{n+2},s_{n+1}}\to
  N_*N_*(Q^{\calc}((B^\BSW_{\bullet}L)^{\smashcoprod 2}))_{s_2-1,s_1}.\]
\end{lem}
%\textbf{Maybe define $\psi_\BSW$} here, as opposed to defining it in a couple of pages, and give a description of it.
As in \S\ref{chain level structure}, we will use the composite double complex map
\[\psi_\BSW:=j_{\calL(n)}\circ Q^{\calL(n)}\xi_\BSW:N\uhor_*N\uver_*(Q^{\calL(n)}\BSWres L)_{s_{n+2},\ldots,s_1}^{t+1}\to N\uhor_*N\uver_*(S^2(Q^{\calL(n)}\BSWres L))_{s_{n+2}-1,s_{n+1},\ldots,s_1}^{t}\]
in what follows.


\SubsectionOrSection{Quadratic grading}
\label{Quadratic grading}
We will say that an object $X\in\calc$, where $\calc$ is any of $\calw(n)$, $\calu(n)$ or $\call(n)$, is \emph{quadratically graded} if the underlying vector space of $X$ is equipped with a quadratic grading such that the action map
$F^{\calc}X\to X$ 
preserves quadratic gradings (i.e.\ is a map in $\quadgrad{}\vect{+}{r}$).
Recall that $F^{\calc}$ is in fact a monad on $\quadgrad{}\vect{+}{n}$, by lemmas \ref{quad gradings have a chance on U0 W0} and \ref{quad gradings have a chance on Un Wn}. There are evident categories of quadratically graded objects in these three categories, which we write as $\quadgrad{}\calw(n)$, $\quadgrad{}\calu(n)$ or $\quadgrad{}\call(n)$, and the various homology and cohomology functors can be enriched to functors
\[H_*^{\calc}:s(\quadgrad{}\calc)\to \quadgrad{}\vect{+}{n+1}\textup{ \ and \ }H^*_{\calc}:s(\quadgrad{}\calc)\to \quadgrad{}\vect{n+1}{+}.\]
Similarly, the categories $\calMv(n+1)$ and $\calMh(n+1)$, in which $H^*_{\calw(n)}$ takes values, can both be enriched in this way, and if $X\in\quadgrad{}\calw(n)$ then 
$H^*_{\calw(n)}X$ is an object of $\quadgrad{}\calMv(n+1)$ and $\quadgrad{}\calMh(n+1)$, and 
$H_*^{\calu(n)}X$ is an object of $\quadgrad{}\calw(n+1)$.

Thus, if $X\in\quadgrad{}\calw(n)$ then the \CFSS
\[\E{\BSW}{2}{X}{s_{n+2},\ldots,s_1}{t}=((H^*_{\calw(n+1)})(H_*^{\calU(n)})X)^{s_{n+2},\ldots,s_1}_t\implies \Edown{0}{H^*_{\calw(n)}X}{s_{n+2},\ldots,s_1}{t}\]
has both $E_2$ and target quadratically graded. Because all of the cohomology and homotopy operations constructed in \S\S\ref{Constructing homotopy operations}-\ref{Cohomology Operations for W and U} are formed at the chain level using quadratic operations, it is not hard to check
\begin{prop}
\label{prop: cfsseq is quad graded}
If $X\in\quadgrad{}\calw(n)$ then the \CFSS\ is quadratically graded:
\[\quadgrad{k}\E{\BSW}{2}{X}{s_{n+2},\ldots,s_1}{t}=\quadgrad{k}(H^*_{\calw(n+1)})(H_*^{\calU(n)})X)^{s_{n+2},\ldots,s_1}_t\implies \quadgrad{k}\Edown{0}{H^*_{\calw(n)}X}{s_{n+2},\ldots,s_1}{t}.\]
\end{prop}


\SubsectionOrSection{The edge homomorphism and edge composite}
For $X\in s\calw(n)$, the spectral sequence
%\[(E_2)^{s_{n+2},\ldots,s_1}_t=(H^*_{\calw(n+1)}(H_*^{\calU(n)}X))^{s_{n+2},\ldots,s_1}_t\implies (H^*_{\calw(n)}X)^{s_{n+2}+s_{n+1},s_n,\ldots,s_1}_t\]
\[\E{\BSW}{2}{X}{s_{n+2},\ldots,s_1}{t}=((H^*_{\calw(n+1)})(H_*^{\calU(n)})X)^{s_{n+2},\ldots,s_1}_t\implies \Edown{0}{H^*_{\calw(n)}X}{s_{n+2},\ldots,s_1}{t}\]
has \emph{edge homomorphism}
%\[(H^*_{\calw(n)}X)^{s_{n+1},\ldots,s_{1}}_t\epi E_0^{0}(H^*_{\calw(n)}X)^{s_{n+1},\ldots,s_{1}}_t\cong (E_\infty)^{0,s_{n+1},\ldots,s_1}_t\subseteq (E_2)^{0,s_{n+1},\ldots,s_1}_t\]
\[(H^*_{\calw(n)}X)^{s_{n+1},\ldots,s_{1}}_t\epi
\Edown{0}{H^*_{\calw(n)}X}{0,s_{n+1},\ldots,s_1}{t}
\cong 
\E{\BSW}{\infty}{X}{0,s_{n+1},\ldots,s_1}{t}
\subseteq 
\E{\BSW}{2}{X}{0,s_{n+1},\ldots,s_1}{t}
\]
which we may compose with the inclusion
%\[(E_2)^{0,s_{n+1},\ldots,s_1}_t=(\dual(Q^{\calw(n+1)}H_*^{\calU(n)}X))^{s_{n+1},\ldots,s_1}_t\subseteq (H^*_{\calU(n)}X)^{s_{n+1},\ldots,s_1}_t.\]
\[\E{\BSW}{2}{X}{0,s_{n+1},\ldots,s_1}{t}
=
(\dual(Q^{\calw(n+1)}H_*^{\calU(n)}X))^{s_{n+1},\ldots,s_1}_t\subseteq (H^*_{\calU(n)}X)^{s_{n+1},\ldots,s_1}_t.\]
to form the \emph{edge composite}:
\[(H^*_{\calw(n)}X)^{s_{n+1},\ldots,s_{1}}_t\to (H^*_{\calU(n)}X)^{s_{n+1},\ldots,s_{1}}_t\]
%One may then ask whether this composite commutes with the operations defined in \S\ref{Cohomology Operations for W and U}.
\begin{prop}
\label{edgehomproposition}
Suppose that $n\geq1$. Then the edge composite commutes with the vertical Steenrod operations of Proposition \ref{vertical steenrod operations prop}: % (\textbf{do not need these conditions}):
\[\xymatrix@R=4mm{
(H^*_{\calw(n)}X)^{s_{n+1},\ldots,s_1}_t\ar[r]^-{\Sqv^i}
\ar[d]^-{\textup{edge comp.}}
&%r1c1
(H^*_{\calw(n)}X)^{s_{n+1}+1,s_n+i-1,2s_{n-1},\ldots,2s_1}_{2t+1}\ar[d]^-{\textup{edge comp.}}
\\%r1c2
(H^*_{\calU(n)}X)^{s_{n+1},\ldots,s_1}_t\ar[r]^-{\Sqv^i}
&%r1c1
(H^*_{\calU(n)}X)^{s_{n+1}+1,s_n+i-1,2s_{n-1},\ldots,2s_1}_{2t+1}
}\]
Setting $n=0$, suppose that $2\leq i <t$. The same composite commutes with the  $\deltav$-operations of propositions \ref{operations on goerss homology} and \ref{operations on untable P homology}:
\[\xymatrix@R=4mm{
(H^*_{\calw(0)}X)^{s}_t\ar[r]^-{\deltav_i}
\ar[d]^-{\textup{edge comp.}}
&%r1c1
(H^*_{\calw(0)}X)^{s+1}_{t+i+1}\ar[d]^-{\textup{edge comp.}}
\\%r1c2
(H^*_{\calU(0)}X)^{s}_t\ar[r]^-{\deltav_i}
&%r1c1
(H^*_{\calU(0)}X)^{s+1}_{t+i+1}
}\]
\end{prop}
\begin{proof}For this proof, we will suppress the `$(n)$' notation, as the proof is the same for all $n\geq0$. We will also suppress all internal gradings, and write $*$ for the grading $s_{n+1}$. The edge composite is dual to
\[H_*^{\calw}X:=\pi_*(Q^{\calw}|B^{\calw}X|)\overset{d\uhor_0}{\from}\pi\uhor_0\pi\uver_*(Q^{\calL}B^{\BSW} Q^{\calU}|B^{\calw}X|)\overset{z_{\DASH}}{\from}\pi_*(Q^{\calU}|B^{\calw}X|)\cong H_*^{\calU}X.\]
Abbreviating further by setting $D:=Q^{\calU}|B^{\calw}X|$ and $C:=Q^{\calL}B^{\BSW} Q^{\calU}|B^{\calw}X|$, the map $z_{\DASH}$ sends the class of $x\in ZN_*D$ to $\overline{z_x}\in \pi\uhor_0\pi\uver_*C$. This assignment does not produce a well defined map $\pi_*D\to N\uhor_0\pi\uver_*C$, as if $y\in ZN_* D$ represents the same class as $x$, $\overline{z_y}$ need not equal $\overline{z_x}$ in $N\uhor_0\pi\uver_*C$: we only know that $\overline{z_{x-y}}=0\in N\uhor_0\pi\uver_*C$.
%SPELLING OUT THAE NULLHOMOTOPY: However, $x-y=d\uver_0w$ for some $w\in N_{n+1}D$, and $h_w\in N^\textup{h}_0N\uver_{n+1}C$ satisfies $d\uver_0h_w=z_{x-y}$, so that $[z_{x-y}]=0\in N_0^\textup{h}\pi\uver_*C$. 
Fortunately, the element $\overline{z_{z_{x}-z_{y}}-z_{z_{x-y}}}\in N\uhor_1\pi\uver_*C$ provides a homotopy between $\overline{z_{x-y}}$ and $\overline{z_{x}-z_{y}}$ in $N\uhor_0\pi\uver_*C$:
%
% there is a (horizontal) homotopy between the two, which we will present as the element $z_{z_{x}-z_{y}}\in C_1^\textup{h}\pi\uver_*C$ in the unnormalized complex:
\[d\uhor_0\!\left(\overline{z_{z_{x}-z_{y}}-z_{z_{x-y}}}\right)=\overline{z_x-z_y}-\overline{z_{x-y}}\textup{, \  and \ }d\uhor_1\!\left(\overline{z_{z_{x}-z_{y}}-z_{z_{x-y}}}\right)=\overline{z_{{x}-{y}}-z_{{x}-{y}}}=0,\]
so that the map $z_\DASH$ is well defined.

We may model the final isomorphism as follows. Write $U^{\calw}_{\calU}:\calw\to\calU$ for the forgetful functor. For any $V\in\vect{+}{n}$, there is a natural inclusion $F^{\calU}V\to U^{\calw}_{\calU} F^{\calw}V$ in the category  $\calU$, adjoint to the inclusion $V\to F^{\calw}V$. This morphism yields an inclusion of bar constructions, a weak equivalence $|B^{\calU}U^{\calw}_{\calU}X|\to U^{\calw}_{\calU}|B^{\calw}X|$ in $s\calU$. Suppressing the forgetful functors, for $X\in\calw$, we have a weak equivalence $Q^{\calU}|B^{\calU}X|\to Q^{\calU}|B^{\calw}X|$ inducing the isomorphism. Our conclusion is then that the entire composite $H_*^{\calU}X\to H_*^{\calw}X$ is the map on homotopy induced by the composite 
\[Q^{\calU}|B^{\calU}X|\to Q^{\calU}|B^{\calw}X| \epi Q^{\calw}|B^{\calw}X|,\]
and the operations we are considering are easily understood in relation to this map.
\end{proof}

\SubsectionOrSection{An equivalent reverse Adams spectral sequence}
It happens that the \CFSS\ recently defined actually coincides with an instance of Miller's reverse Adams spectral sequence used Goerss   \cite[Chapter V]{MR1089001} (c.f.\ \S\ref{reverse Adams spectral sequence}). This seems to the author to be somewhat of a coincidence, as in \cite{MR1089001}, the reverse Adams spectral sequence appears for quite different reasons than in the present work.
We continue using Blanc and Stover's resolution, for two reasons. Firstly, that resolution more closely reflects out intention in constructing the spectral sequence in question, and secondly, the techniques we use here may be generalizable to other contexts in which the Blanc-Stover resolution is used.

\begin{prop}
The \CFSS\ applied to $X\in s\calw(n)$ coincides with the reverse Adams spectral sequence applied to  $L:=Q^{\calU(n)}{ B^{\calw(n)}X}\in s\calL(n)$.
\end{prop}
\noindent Before proving this fact, we should remove any confusion about the convergence targets of these spectral sequences. Indeed, the reverse Adams  spectral sequence has target
\[\pi^*\dual Q^{\calL(n)}B^{\calL(n)}L\cong \pi^*\dual Q^{\calL(n)}L=\pi^*\dual Q^{\calw(n)}B^{\calw(n)}X=:H^*_{\calw(n)}X,\]
where the isomorphism follows from the same acyclicity condition needed to define the \CFSS. Thus the targets coincide, as hoped.
\begin{proof}
We will use the Dwyer-Kan-Stover $E^2$ model structure on the category $ss\calc$, which originated in \cite{DKS.pdf} for bisimplicial sets, and is reinterpreted for objects of $ss\calc$ in \citeBOX[\S4.1.1]{Blanc_Stover-Groth_SS.pdf}.

Viewing $L$ as a constant object in $ss\calL(n)$, each of $B^{\calL(n)}_pL_q$ and $B^{\BSW}L$ admits an $E^2$-weak equivalence to $L$. Moreover, each  is cofibrant. Indeed, $B^{\BSW}L$ is cofibrant by construction, while we must check  that  $B^{\calL(n)}_pL_q$ is M-free, in the sense of \citeBOX[\S4.1.1]{Blanc_Stover-Groth_SS.pdf}.

For this, we use Lemma \ref{skeleton lemma}. That is, for each $q$, the horizontal simplicial object $(B^{\calL(n)}_pL_q)_{p,q}:=B^{\calL(n)}_pL_q$ has an obvious structure of almost free simplicial (in $p$) object, and the generating subspaces are preserved by the vertical simplicial maps. Thus, Lemma \ref{skeleton lemma} yields decompositions
\[V_p=\im\Bigl(V_{p-1}\overset{s\uhor_0}{\to}V_p\Bigr)\oplus\cdots \oplus \im\Bigl(V_{p-1}\overset{s\uhor_{p-1}}{\to}V_p\Bigr)\oplus \Bigl(V_p\cap N\uhor_pB^{\calL(n)}_pL_q\Bigr).\]
To show that $V_p$ is M-free, we need to decompose each $V_p$ into a coproduct of objects $\mathbb{K}_{s_{n+1},\ldots,s_1}^t\in s\vect{+}{n}$, up to homotopy, and ensure that the degeneracies are induced up to homotopy by sphere inclusions. The decompositions of $V_p$ just provided make this a simple task.
Suppose that $V_{p-1}$ already has chosen decomposition as a sum of objects $\mathbb{K}_{s_{n+1},\ldots,s_1}^t$ up to homotopy. Then if we choose such a decomposition of $V_p\cap N\uhor_p B^{\calL(n)}_pL_q$, and use the $p$ inclusions $s\uhor_i:V_{p-1}\to V_p$ to induce decompositions of the other summands of $V_p$ using the decomposition of $V_{p-1}$, we have the decomposition up to homotopy that we need.

Now, by factoring the map $0\to L$ by a cofibration followed by an acyclic fibration $B\to L$ in the $E_2$ model structure, we can form the solid maps in a diagram in which each object `$B$' is cofibrant:
\[\xymatrix@R=4mm{
B\ar@{->>}[dr]_-{\sim}
&%r1c1
B^{\calL(n)}_pL_q\ar[d]^-{\sim}\ar@{..>}[l]_-{\sim}\\%r1c2
B^{\BSW}L\ar[r]_-{\sim}\ar@{..>}[u]_-{\sim}&%r2c1
L%r2c2
}\]
By the lifting axiom (of cofibrations against acyclic fibrations) we can find the dotted maps, weak equivalences making the diagram commute. The theory presented in \cite{DKS.pdf} then explains that the three resulting spectral sequences  coincide. The spectral sequence arising from $B^{\calL(n)}_pL_q$ is the reverse Adams spectral sequence of $L$ in $s\calL(n)$, and that arising from $B^{\BSW}L$ is the \CFSS\ of $X\in s\calw(n)$.
\end{proof}
\end{Comp funct sseqs}



\begin{Operations in composite functor spectral sequences}
\SectionOrChapter{Operations in composite functor spectral sequences}
\label{Operations in composite functor spectral sequences}
Singer \cite{MR2245560} developed a useful theory of products and Steenrod operations in the first quadrant cohomology spectral sequence arising from a bisimplicial cocommutative coalgebra. Goerss used this theory in \cite[\S14]{MR1089001} in his calculation of the category $\HA{\algs}$.
 In the applications we have in mind, the bisimplicial object \[Q^{\calL(n)}\BSWres Q^{\calU(n)}{ B^{\calw(n)}X}\] 
will \emph{not} be a coalgebra. Instead the situation will resemble more the situation of \S\ref{chain level structure}, where there was a linear map $\psi_\calc:Q^{\calc}X^s\to S^2(Q^{\calc}X^{s-1})$ for any almost free object $X\in s\calc$, but certainly not a coalgebra map.


The lack of an underlying coalgebra structure will not stop us from applying Singer's techniques after we make the appropriate modifications. The idea is to externalize Singer's operations, so that for every bisimplicial vector space $V$, there are various external operations of type:
\[\Edown{r}{V}{**}{}\to \Edown{r'}{S^2V}{**}{}\ (r'\geq r)\textup{ \ and \ }S_2\Edown{r}{V}{**}{}\to \Edown{r}{S^2V}{**}{},\]
(which we will discuss shortly) compatible at $E_\infty$ with external operations of type:
\[H^*(\dual(TV))\to H^*(\dual(TS^2V))\textup{ and }S_2H^*(\dual(TV))\to H^*(\dual(TS^2V)).\]
When $V$ is  in fact a bisimplicial cocommutative coalgebra, one recovers Singer's theory by composing with the map of spectral sequences induced by the coproduct:
\[\Edown{r}{S^2V}{**}{}\to \Edown{r}{V}{**}{}.\]
%When $V$ is a cocommutative coalgebra, the diagonal map $V\to S^2V$ induces a map $E_r^{**}(S^2V)\to E_r^{**}(V)$, postcomposition with which yields Singer's operations, however we will not have access to such structure in what follows. Thus, we will briefly reprise Singer's results in this \emph{external} language, in order that we have them available.

An (external) \emph{operation with indeterminacy disappearing by $E_{r'}$} is a function
\[\Edown{r}{V}{**}{}\to \Edown{r}{S^2V}{**}{}/\EBdown{r,r'}{S^2V}{**}{},\]
where $\EBdown{r,r'}{S^2V}{**}{}\subseteq \Edown{r}{S^2V}{**}{}$ is the subgroup consisting of those elements which survive to $\Edown{r}{S^2V}{**}{}$ and represent zero there. Later in this thesis we will use the term \emph{multi-valued function} 
$\Edown{r}{V}{**}{}\to \Edown{r}{S^2V}{**}{}$, a clear abuse of terminology that should cause little offense.

Some of the operations $\Edown{r}{V}{**}{}\to \Edown{r'}{S^2V}{**}{}$ to be defined shortly are between different spectral sequence pages. 
Such an operation is equivalent to an external operation with indeterminacy $r'$ which also satisfies a survival property.

\SubsectionOrSection{External spectral sequence operations of Singer}\label{singer ext ops sect}
We now summarise some key aspects of Singer's work in \cite{MR2245560}, in particular theorems 2.15, 2.16, 2.17 and 2.22, and Proposition 2.21. Fix $V\in ss\vect{}{}$ with a (horizontal) augmentation $d\dhor^0:V\to V_{-1}$. The key construction is that of chain level operations:
\[S^k:\dual (TV)\to \dual (TS^2V)\]
inducing external operations as in the bottom row of the following diagrams:
%If $V$ admits an augmentation $d\dhor^0:V\to V_{-1}$, then $\ExtCohOp^k$ and $\ExtCohProd$ commute with the induced maps on cohomotopy, so that there are commuting diagrams %$\dual(d\dhor^0)$
\[\xymatrix@R=4mm{
\pi^{m}(\dual(V_{-1}))\ar[r]^-{\ExtCohOp^k}
\ar[d]%^-{\lambda^*}
&%r1c1
\pi^{m+k}(\dual(S^2V_{-1}))\ar[d]%^-{\dual(S^2\lambda)}
&%r1c3
S_2\pi^{*}(\dual(V_{-1}))\ar[r]^-{\ExtCohProd}
\ar[d]%^-{\lambda^*}
&%r1c1
\pi^{*}(\dual(S^2V_{-1}))\ar[d]%^-{\dual(S^2\lambda)}
\\%r1c5
H^{m}(\dual(TV))\ar[r]^-{\ExtCohOp^k}
&%r2c1
H^{m+k}(\dual(TS^2V))
&%r2c3
S_2H^{*}(\dual(TV))\ar[r]^-{\ExtCohProd}
&%r2c1
H^{*}(\dual(TS^2V))
}\]
The top rows are the operations arising from the singly (vertically) simplicial object $V_{-1}$, as in \S\ref{External unary cohomotopy operations}. Singer studies the effect of $S^k$ on filtration in detail, determining that it induces the following operations.
For all $p,q\geq0$ and all $r\geq2$, there are well-defined vector space homomorphisms:
\begin{alignat*}{2}
\ExtCohOp^k:\Edown{r}{V}{p,q}{}&\to \Edown{r}{S^2V}{p,q+k}{},&\qquad&\text{if }0\leq k \leq q;\\
\ExtCohOp^k:\Edown{r}{V}{p,q}{}&\to \Edown{r+k-2}{S^2V}{p+k-q,2q}{},&\qquad&\text{if }q\leq k\leq q+r-2;\\
\ExtCohOp^k:\Edown{r}{V}{p,q}{}&\to \Edown{2r-2}{S^2V}{p+k-q,2q}{},&\qquad&\text{if }q+r-2\leq k;
\end{alignat*}
which commute with the differentials (in the appropriate, somewhat complicated sense, c.f.\ \cite[Theorem 2.17]{MR2245560}), and an external (not `exterior') commutative product operation which satisfies the Liebniz rule:
\[\ExtCohProd:\Edown{r}{V}{p_1,q_1}{}\otimes \Edown{r}{V}{p_2,q_2}{}\to \Edown{r}{S^2V}{p_1+p_2,q_1+q_2}{}.\]
Note that the second and third operations are from $E_r\to E_{r'}$, sometimes with $r'\geq r$, which is to say that these operations have indeterminacy disappearing by $E_{r'}$, and the implied survival property.

Those operations with domain $\Edown{2}{V}{}{}$ have no indeterminacy, and we reindex them as follows: %are of the form $\Edown{2}{V}{**}{}\to \Edown{2}{S^2V}{**}{}$.
\begin{alignat*}{3}
\vExtCohOp^k&=\,\ExtCohOp^k&:\,&\Edown{2}{V}{p,q}{}\to \Edown{2}{S^2V}{p,q+k}{},&\qquad&\text{if }0\leq k \leq q,\\
\vExtCohOp^k&=0,&&&&\text{if }k > q,\\
\hExtCohOp^k&=\ExtCohOp^{q+k}&:\,&\Edown{2}{V}{p,q}{}\to \Edown{2}{S^2V}{p+k,2q}{},&\qquad&\text{if }0\leq k.
\end{alignat*}
Under the identification $\Edown{2}{V}{**}{}=\pi\dhor^{*}\pi\dver^{*}(\dual V)$, the operations $\vExtCohOp^k$ % :\pi\dhor^{p}\pi\dver^{q}(\dual V)\to \pi\dhor^{p}\pi\dver^{q+k}(\dual S^2V)$ 
are  obtained by applying $\pi\dhor^p$ to the linear maps of \S\ref{External unary cohomotopy operations}:
\[\pi\dver^{q}(\dual V)\overset{\ExtCohOp^k}{\to} \pi\dver^{q+k}(S_2\dual V)\to \pi\dver^{q+k}(\dual S^2V).\]
On the other hand, the operation $\hExtCohOp^k$ equals the composite:
\[\pi\dhor^p\pi\dver^q\dual V
\overset{\ExtCohOp^k}{\to} 
\pi\dhor^{p+k}(S_2\pi\dver^*\dual V)^{2q}
\overset{\pi\dhor^{p+k}(\ExtCohProd)}{\to}
\pi\dhor^{p+k}\pi\dver^{2q}S_2\dual V
\to 
\pi\dhor^{p+k}\pi\dver^{2q}\dual S^2 V,
\]
 and the pairing $\ExtCohProd:\Edown{2}{V}{**}{}\to \Edown{2}{S^2V}{**}{}$ equals:
\[S^2\pi\dhor^*\pi\dver^*\dual V
\overset{\ExtCohProd}{\to} 
\pi\dhor^*(S_2\pi\dver^*\dual V)
\overset{\pi\dhor^{*}(\ExtCohProd)}{\to}
\pi\dhor^{*}\pi\dver^{*}S_2\dual V
\to 
\pi\dhor^{*}\pi\dver^{*}\dual S^2 V.
\]

These operations on $E_2$ determine the operations at each $E_r$, $r>2$. The operations $\ExtCohOp^k$ commute with differentials as appropriate. Finally, the $\ExtCohOp^k$ stabilize to well defined maps on $E_\infty$, and there is a commuting diagram
\[\xymatrix@R=4mm{
\Edown{\infty}{V}{p,q}{}\ar[r]^-{\ExtCohOp^k}\ar[d]^-{\cong}
&%r1c1
\Edown{\infty}{S^2V}{p,q+k}{}
\ar[d]^-{\cong}
\\%r1c2
\Edown{0}{H^*(\dual(TV))}{p,q}{}\ar[r]^-{\ExtCohOp^k}&%r2c1
\Edown{0}{H^*(\dual(TS^2V))}{p,q+k}{}%r2c2
}\]
whenever $0\leq k\leq q$, and a commuting diagram
\[\xymatrix@R=4mm{
\Edown{\infty}{V}{p,q}{}\ar[r]^-{\ExtCohOp^k}\ar[d]^-{\cong}
&%r1c1
\Edown{\infty}{S^2V}{p+k-q,2q}{}
\ar[d]^-{\cong}
\\%r1c2
\Edown{0}{H^*(\dual(TV))}{p,q}{}\ar[r]^-{\ExtCohOp^k}&%r2c1
\Edown{0}{H^*(\dual(TS^2V))}{p+k-q,2q}{}%r2c2
}\]
whenever $q\leq k$ (which summarises also Singer's computation of how the $\ExtCohOp^k$ interact with the filtration on cohomology).
%Moreover, for $r\geq2$ and $p,q,k\geq0$ given, if $\alpha\in E_r^{p,q}$, then $\Sq_\textup{ext}^k\alpha$ and $\Sq_\textup{ext}^kd_r\alpha$ both survive to $E_t$, where
%\[t=\begin{cases}
%r,&\textup{if }0\leq k \leq q-r+1;\\
%2r+k-q-1,&\textup{if }q-r+1\leq k\leq q;
%\\2r-1,&\textup{if }q\leq k,
%\end{cases}
%\]
%and in $E_t$ we have the relation $d_t\Sq^k_textup{ext}\alpha=\Sq^k_textup{ext}d_r\alpha$.




%Moreover, if the bisimplicial object $V$ admits an augmentation $\lambda:V_{0\bullet}\to U_{\bullet}$, (\textbf{didn't I} do away with $\bullet$?) then %such that the natural map $\lambda_*:CV\to CU$ is a weak equivalence,
%Singer proves the following relationship between the spectral sequence operations and the operations of \S\ref{generic coh ops section}:
%\begin{thm}[Proposition 2.1]
%The $\ExtCohOp^k$ and $\ExtCohProd$ commute with $\lambda^*$, so that there are commuting diagrams
%\[\xymatrix@R=4mm{
%H^{m}(\dual(CU))\ar[r]^-{\ExtCohOp^k}
%\ar[d]^-{\lambda^*}
%&%r1c1
%H^{m+k}(\dual(CS^2U))\ar[d]^-{\dual(S^2\lambda)}
%&%r1c3
%H^{m_1}(\dual(CU))\otimes H^{m_2}(\dual(CU))\ar[r]^-{\ExtCohProd}
%\ar[d]^-{\lambda^*}
%&%r1c1
%H^{m_1+m_2}(\dual(CS^2U))\ar[d]^-{\dual(S^2\lambda)}
%\\%r1c5
%H^{m}(\dual(CV))\ar[r]^-{\ExtCohOp^k}
%&%r2c1
%H^{m+k}(\dual(CS^2V))
%&%r2c3
%H^{m_1}(\dual(CV))\otimes H^{m_2}(\dual(CV))\ar[r]^-{\ExtCohProd}
%&%r2c1
%H^{m_1+m_2}(\dual(CS^2V))
%}\]
%\end{thm}
%
%Singer separates the operations $\ExtCohOp^k:E^{p,q}_2\to E_{2}$ according to the pattern of change in the spectral sequence coordinates of their output. Indeed, he defines
%\begin{alignat*}{2}
%\vExtCohOp^k=\ExtCohOp^k:E^{p,q}_2(V)&\to E^{p,q+k}_2(S^2V)&\qquad&\text{if }0\leq k \leq q,\\
%\hExtCohOp^k=\ExtCohOp^{q+k}:E^{p,q}_2(V)&\to E^{p+k,2q}_2(S^2V)&\qquad&\text{if }0\leq k \leq p,
%\end{alignat*}
%and gives explicit formulae for these $E_2$ operations, and for the product at $E_2$ [Singer 2.23].








\SubsectionOrSection{Application to composite functor spectral sequences}
\label{Application to composite functor spectral sequences}


In order to use Singer's constructions in the present work, we will use the map of double complexes:
\[\psi_\BSW=j_{\calL(n)}\circ Q^{\calL(n)}\xi_\BSW:N_{p+1}N_q(Q^{\calL(n)}\BSWres L)_{s_n,\ldots,s_1}^{t+1}\to N_{p}N_q(S^2(Q^{\calL(n)}\BSWres L))_{s_n,\ldots,s_1}^{t}\]
to define a spectral sequence map
\[\Edownup{2}{}{S^2(Q^{\calL(n)}\BSWres L)}{p,q,s_n,\ldots,s_1}{t} \overset{(\psi_\BSW)^*}{\to}  \Edownup{2}{\BSW}{X}{p+1,q,s_n,\ldots,s_1}{t+1}.\]
We  then define internal spectral sequence operations
\begin{alignat*}{2}
%\Sq^k=\psi_\BSW^*\circ\ExtCohOp^{k-1}:(E^{p,q}_r)^{s_n,\ldots,s_1}_t&\to (E^{p+1,q+k-1}_r)^{2s_n,\ldots,2s_1}_{2t+1}&\qquad&\text{if }0\leq k-1 \leq q,\\
%
%
\Sq^k=\psi_\BSW^*\circ\ExtCohOp^{k-1}:
\Edownup{r}{\BSW}{X}{p,q,s_n,\ldots,s_1}{t}
&\to 
\Edownup{r}{\BSW}{X}{p+1,q+k-1,2s_n,\ldots,2s_1}{2t+1}
&&(0\leq k-1 \leq q),\\
\Sq^k=\psi_\BSW^*\circ\ExtCohOp^{k-1}:
\Edownup{r}{\BSW}{X}{p,q,s_n,\ldots,s_1}{t}
&\to 
\Edownup{r+k-q-1}{\BSW}{X}{p+k-q,2q,2s_n,\ldots,2s_1}{2t+1}
&&(q\leq k-1\leq q+r-2),\\
\Sq^k=\psi_\BSW^*\circ\ExtCohOp^{k-1}:
\Edownup{r}{\BSW}{X}{p,q,s_n,\ldots,s_1}{t}
&\to 
\Edownup{2r-2}{\BSW}{X}{p+k-q,2q,2s_n,\ldots,2s_1}{2t+1}
&\ &(q+r-2\leq k-1).\\
%
%\Sq^k=\psi_\BSW^*\circ\ExtCohOp^{k-1}:(E^{p,q}_r)^{s_n,\ldots,s_1}_t&\to (E^{p+k-q,2q}_{r+k-q-1})^{2s_n,\ldots,2s_1}_{2t+1}&\qquad&\text{if }q\leq k-1\leq q+r-2,\\
%\Sq^k=\psi_\BSW^*\circ\ExtCohOp^{k-1}:(E^{p,q}_r)^{s_n,\ldots,s_1}_t&\to (E^{p+k-q,2q}_{2r-2})^{2s_n,\ldots,2s_1}_{2t+1}&\qquad&\text{if }q+r-2\leq k-1,
\end{alignat*}
which at $E_2$ we may write (dropping internal degrees) as:
\begin{alignat*}{2}
\Sqv^k=\psi_\BSW^*\circ\vExtCohOp^{k-1}= \psi_\BSW^*\circ\ExtCohOp^{k-1}:E^{p,q}_2&\to E^{p+1,q+k-1}_2&\qquad&\text{if }0\leq k-1 \leq q,\\
\Sqh^k=\psi_\BSW^*\circ\hExtCohOp^{k-1}= \psi_\BSW^*\circ\ExtCohOp^{q+k-1}:E^{p,q}_2&\to E^{p+k,2q}_2&\qquad&\text{if }0\leq k-1 \leq p.
\end{alignat*}
Similarly, we   define a pairing:
%\[\mu=\psi_\BSW^*\circ\ExtCohProd:(E^{p,q}_r)^{s_n,\ldots,s_1}_t\otimes (E^{p',q'}_r)^{s'_n,\ldots,s'_1}_{t'}\to (E^{p+p'+1,q+q'}_r)^{s_n+s'_n,\ldots,s_1+s'_1}_{t+t'+1}.\]
\[\mu=\psi_\BSW^*\circ\ExtCohProd: \Edownup{r}{\BSW}{X}{p,q,s_n,\ldots,s_1}{t} \otimes \Edownup{r}{\BSW}{X}{p',q',s'_n,\ldots,s'_1}{t'}\to \Edownup{r}{\BSW}{X}{p+p'+1,q+q',s_n+s'_n,\ldots,s_1+s'_1}{t+t'+1}\]

The reader might now guess the key results:
\begin{thm}
\label{E2CompFuncLieOperationsID}
At $E_2\cong H^*_{\calw(n+1)}H_*^{\calU(n)}X$, the operations $\Sqh^k$ and $\mu$ defined here are equal to the $\calMh(n+2)$-operations of the same name defined on $\calw(n+1)$-cohomology in
\S\ref{Horizontal Steenrod operations and a product for HWn}.
\end{thm}
\begin{thm}
\label{E2CompFuncKosOperationsID}
At $E_2\cong H^*_{\calw(n+1)}H_*^{\calU(n)}X$, the operations $\Sqv^k$ defined here are equal to the $\calMv(n+2)$-operations of the same name defined on $\calw(n+1)$-cohomology in
\S\ref{section: vertical Koszul operations n positive}.
\end{thm}
\begin{thm}
\label{EInftyCompFuncOperationsID}
At $E_\infty \cong \Edown{0}{H^*_{\calw(n)}X}{**}{}$, the operations $\Sq^k$ are compatible with the
$\calMh(n+1)$-operations of the same name defined on $\calw(n)$-cohomology in
\S\ref{Horizontal Steenrod operations and a product for HWn}.
%
% operations $\Sqh^k$ defined on the target $H^*_{\calw(n)}(X)$, as defined in [\S?].
\end{thm} 
\SubsectionOrSection{Proofs of theorems \ref{E2CompFuncLieOperationsID}-\ref{EInftyCompFuncOperationsID}}

\begin{proof}[Proof of Theorem \ref{E2CompFuncLieOperationsID}]
This proof relies on a commuting diagram, in which we employ the  notation $L=Q^{\calU(n)} B^{\calw(n)}X \in s\calL(n)$, and abbreviate using $\calL=\calL(n)$ and $\calw=\calw(n+1)$.
\[\xymatrix@C=13mm@R=4mm{
(N\uhor_p\pi\uver_*Q^{\calL}\BSWres L)^{\otimes2}&%r1c1
(N\uhor_pQ^{\calw}\pi\uver_*\BSWres L)^{\otimes2}
\ar[l]_-{N\uhor_*(\gamma)^{\otimes 2}}
\\%r1c2
N\uhor_{p+k-1}((\pi\uver_*Q^{\calL}\BSWres L)^{\otimes2})
\ar[u]^-{(D\dhor^{p-k+1})^\star}
&%r2c1
N\uhor_{p+k-1}((Q^{\calw}\pi\uver_*\BSWres L)^{\otimes2})
\ar[u]_-{(D\dhor^{p-k+1})^\star}
\ar[l]_-{N\uhor_*(\gamma^{\otimes 2})}
\\%r2c2
N\uhor_{p+k-1}\pi\uver_*((Q^{\calL}\BSWres L)^{\otimes2})
\ar[u]^-{N\uhor_*(D\dver^0)^\star}
&%r3c1
\\%r3c2
N\uhor_{p+k}\pi\uver_* Q^{\calL}\BSWres L
\ar[u]^-{\pi\uver_*(\psi_\BSW)}
&%r4c1
N\uhor_{p+k}Q^{\calw}\pi\uver_* \BSWres L
\ar[uu]_-{\psi_{\calw}}
\ar[l]_-{N\uhor_*(\gamma)}
%r4c2
}\]
All of the horizontal maps are the isomorphisms of Lemma \ref{Q of a model}. By \cite[Theorem 2.23]{MR2245560} (summarized in \S\ref{singer ext ops sect}),  the left hand vertical composite is that used to define the horizontal operations $\Sqh^{k}$ on $E_2$. On the other hand, the right vertical was used in \S\ref{Horizontal Steenrod operations and a product for HWn} to define the $\calMh(n+2)$-operations on the $\calw(n+1)$-cohomology groups with which the $E_2$-page can be identified. Thus, if the diagram commutes, we are done. If we replace the maps $(D\dhor^j)^\star$ in the top square with $(D\dhor^0)^\star$, the same proof applies for $\mu$.

What remains is to prove that the bottom square commutes. It may be expanded into the eight maps in the outer square of the following larger commuting diagram:
\[
%\def\objectstyle{\scriptstyle}
\xymatrix@R=6mm@C=15mm@!0{
&%r2c1
\pi\uver_*((Q^{\calL}\BSW^{p+k}L)^{\otimes2}) \ar[rrr]_-{(D\dver^0)^\star}
&%r2c2
&%r2c3
&%r2c4
(\pi\uver_* Q^{\calL}\BSW^{p+k}L)^{\otimes2}&%r2c5
&%r2c6
&%r2c7
(Q^{\calw}\pi\uver_*\BSW^{p+k}L)^{\otimes2} \ar[lll]^-{\gamma^{\otimes2}}\\\\%r2c8
&%r3c1
\pi\uver_* Q^{\calL}((\BSW^{p+k}L)^{\smashcoprod 2})\ar[uu]^-{\pi\uver_*(j_{\calL})}
&%r3c2
&%r3c3
&%r3c4
Q^{\calw}\pi\uver_*((\BSW^{p+k}L)^{\smashcoprod 2})\ar[lll]_-{\gamma}
&%r3c5
&%r3c6
&%r3c7
Q^{\calw}((\pi\uver_* \BSW^{p+k}L)^{\smashcoprod 2})
\ar[uu]_-{j_{\calw}}
\ar[lll]_-{Q^{\calw}(i)}
\\\\
&%r3c1
\pi\uver_* Q^{\calL}\BSW^{p+k+1}L
\ar[uu]^-{\pi\uver_* Q^{\calL}(\xi_{\BSW})}
&%r3c2
&%r3c3
&%r3c4
Q^{\calw}\pi\uver_*\BSW^{p+k+1}L\ar[lll]_-{\gamma}
\ar[uu]_-{Q^{\calw}\pi\uver_*(\xi_{\BSW})}
&%r3c5
&%r3c6
&%r3c7
Q^{\calw}\pi\uver_* \BSW^{p+k+1}L
\ar[uu]_-{Q^{\calw}(\xi_{\calw})}
\ar[lll]_-{=}
}\]
The bottom left square commutes by naturality of $\gamma$, while the bottom right square is an instance of Lemma \ref{LemmaOn xi}. What remains is to check that the hexagon commutes. For notational convenience, write $A=\BSW^{p+k}L\in s\calL$, $\textup{br}:A^{\otimes 2}\to A^{\smashcoprod 2}$ for the $\call$-bracket, and
$\textup{br}:(\pi\uver_* A)^{\otimes 2}\to (\pi\uver_* A)^{\smashcoprod 2}$ for the $\calw$-bracket on homotopy.

The source in the hexagon is then $Q^{\calw}((\pi\uver_*A)^{\smashcoprod2})$, the smash product being the coproduct in  $\calw=\PA{\calL}$ of two copies of $\pi\uver_*A$. Any element of $Q^{\calw}((\pi\uver_*A)^{\smashcoprod2})$ can be represented by a sum
$\textstyle\sum_k\textup{br}(\overline{x_k}\otimes \overline{y_k})+E$,
with the $x_k$ (resp.\ $y_k$) representatives of elements $\overline{x_k}$ in the first (resp.\ second) copy of $\pi\uver_* A$ and, $E$ a sum of at least three-fold  brackets of elements in the two copies. This extra term $E$ is annihilated by both $j_{\calw}$ and $\pi\uver_*(j)\circ\gamma\circ Q^{\calw}(i)$, so can be ignored. One calculates: 
\[\smash{(\gamma^{\otimes 2}\circ j_\calw)(\textstyle\sum_k\textup{br}(\overline{x_k}\otimes \overline{y_k})+E)=\textstyle\sum_k \overline{x_k}\otimes \overline{y_k}.}\]
On the other hand,
% Now any element of $Q^{\calw(n+1)}((\pi\uver_* A)^{\smashcoprod 2})$ is the class of an expression $\textup{br}(\overline{x}\otimes\overline{y})+E$ where $\overline{x},\overline{y}\in\pi\uver_* A$ are represented by $x,y\in N\uver_*A$, and $E\in(\pi\uver_* A)^{\smashcoprod 2}$ is a sum of terms which are $\lambda$-operations applied to 3-fold brackets of elements of $\pi\uver_* A$. 
the map $Q^{\calw}(i)$ is induced by the Eilenberg-Mac Lane map shuffle map $\nabla_\textup{\!v}$ as in Proposition \ref{the top external homotopy operations}, and
\[\textstyle \smash{
\sum_k\textup{br}(\overline{x_k}\otimes\overline{y_k})\overset{\gamma\circ Q^{\calw}(i)}{\longmapsto}\overline{\sum_k\textup{br}(\nabla_\textup{\!v}(x_k\otimes y_k))}\overset{\pi\uver_*(j_\call)}{\longmapsto}\overline{\sum_k(\nabla_\textup{\!v}(x_k\otimes y_k))}\overset{(D\dver^0)^\star}{\longmapsto}\sum_k\overline{x_k}\otimes \overline{y_k}.
}\]
The last mapping follows from the fact that $(D\dver^0)^\star\circ\nabla_\textup{\!v}=\Id$, as $\{D^k\}$ is special.
\end{proof}

\begin{proof}[Proof of Theorem \ref{E2CompFuncKosOperationsID}]
We again employ the  notation $L=Q^{\calU(n)} B^{\calw(n)}X \in s\calL(n)$, and abbreviate using $\calL=\calL(n)$ and $\calw=\calw(n+1)$. Further, write $\mathbb{B}$ for the object $B_m^{\calw}\pi\uver_*L\in s\calw$. Write $V_m$ for the subspace $(F^{\calw})^{m}\subseteq \mathbb{B}$ of generators, and $V'_m:=V_m\cap N\uhor_m\mathbb{B}$. For each $m\geq0$, write $F_m\mathbb{B}$ for the $m$-skeleton of $\mathbb{B}$ (c.f.\ \S\ref{Skeletal filtrations}), which is almost free on  subspaces $F_mV_m\subseteq V_m$.

We must identify the operations $\Sqv^i=\psi_\BSW^*\circ\vExtCohOp^{i-1}$ with the $\calw$-cohomology operations $\Sqv^i$ defined in \S\ref{section: vertical Koszul operations n positive} using the maps $\theta^i$. However, the $\theta^i$ are defined on the bar construction, while $\psi_\BSW^*$ is defined on the Blanc-Stover resolution. In order to make the comparison, we will need to choose a sufficiently explicit weak equivalence of resolutions of $\pi\uver_* L$ in $s\calw$
\[\chi:\mathbb{B}\to \pi\uver_*(\BSWres L).\]
%We will use Miller's techniques \citeBOX[p.~55]{MillerSullivanConjecture.pdf} to define such a $\chi$ explicitly. 


%$F_mV_i$ for the subspace of $V_i$ consisting of degeneracies of elements of $V_j$ for $j\leq m$. Then write $F_m\mathbb{B}$ for the subobject of $\mathbb{B}$ which is almost free on the subspaces $F_mV_i$.

%Now for each $m$, $V_m$ splits into a direct sum
%$V_m=V'_m\oplus F_{m-1}V_m$,
%where $V'_m\subseteq N_m\uhor\mathbb{B}$. To make this classical insight explicit, consider the endomorphism $c:\mathbb{B}_m\to \mathbb{B}_m$ introduced in the proof of Lemma \ref{lemma on homology class repd by normalized generator}, an idempotent with image $N_m^\textup{h}\mathbb{B}$, preserving the subspace $V_m$. As $\im((\Id-c)|_{V_m})\subseteq F_{m-1}V_m$, we may define $V'_m=c(V_m)$. \textbf{(read this paragraph sometime)}

In order to define $\chi$, we recursively define its restriction to the skeleta $F_m\mathbb{B}$. Lemma \ref{skeleton lemma} implies that in order to extend a (horizontal simplicial) map $\chi_{m-1}:F_{m-1}\mathbb{B}\to \pi\uver_*(\BSWres L)$ to a map $\chi_m:F_m\mathbb{B}\to \pi\uver_*(\BSWres L)$, we need only to specify the values of $\chi_m$ on $V'_m$. That is, we only need to choose a lift in the diagram
\[\xymatrix@R=4mm{
V'_m\ar@{-->}[r]^-{\chi_m}
\ar[d]^-{d\uhor_0}
&%r1c1
N\uhor_m\pi\uver_*(\BSWres L)\ar[d]^-{d\uhor_0}
\\%r1c2
ZN\uhor_{m-1}\mathbb{B}\ar[r]^-{\chi_{m-1}}
&%r2c1
ZN\uhor_{m-1}\pi\uver_*(\BSWres L)%r2c2
}\]
However, in order to actually carry out this process, we will need to record some chain level information, and we will  construct maps into $N\uver_*B^{\BSW}L$, rather than just $\pi\uver_*B^{\BSW}L$. 

It is best to view the domain and codomain of the proposed map $\chi$ as augmented (horizontal) simplicial objects, and start by defining $\chi_{-1}$ to be the identity of $\pi\uver_*L$. Then for $m\geq0$, we will recursively construct functions $\overline{\chi}_m:V'_m\to ZN\uver_*\BSWres_m L$, with the property that $\im(\overline{\chi}_m)$ is contained in the span of the classes $z_{w}$ for $w\in ZN\uver_*\BSWres_{m-1}L$, so that there is a commuting diagram:
\[\xymatrix@R=4mm{
V'_m\ar[r]^-{\overline{\chi}_m}
\ar[d]_-{\chi_m}
&%r1c1
ZN\uver_*\BSWres_m L\ar@{->>}[d]
\\%r1c2
N\uhor_m\pi\uver_*\BSWres L\ar@{ >->}[r]
&%r2c1
\pi\uver_*\BSWres_m L%r2c2
}\] In order to do this, one may choose a basis of $V'_m$, and then for each basis element $v\in V'_m$, choose a $\calw$-expression $e$ for $d\uhor_0v$, so that \[d\uhor_0v=e(s\uhor_{\alpha_j}w_j) \in ZN\uhor_{m-1}\mathbb{B}\textup{ is a $\calw$-expression in various }s\uhor_{\alpha_j}w_j\in V_{m-1},\]
with $w_j\in V'_{n_j}$ for integers $n_j\leq m-1$ and degeneracy operators $s_{\alpha_j}:V'_{n_j}\to V_{m-1}$. Then,
from the cycles $s\uhor_{\alpha_j}\overline{\chi}_{n_j}(w_j)\in ZN\uver_*\BSWres_{m-1}L$,  form a cycle
\[e^\textup{rep}(s\uhor_{\alpha_j}\overline{\chi}_{n_j}(w_j))\in ZN\uver_*(\BSWres_{m-1}L),\]
using the explicit formulae of \citeBOX[\S8]{CurtisSimplicialHtpy.pdf} (which is a \emph{normalized} cycle, as  these formulae preserve the normalized subcomplex), so that 
%\[\overline{e^\textup{rep}(s\uhor_{\alpha_j}\overline{\chi}_{n_j}(w_j))}=e\left(\overline{s\uhor_{\alpha_j}\overline{\chi}_{n_j}(w_j)}\right)\in Z N\uhor_{m-1}\pi\uver_*(\BSWres L).\]
\begin{alignat*}{2}
\overline{e^\textup{rep}(s\uhor_{\alpha_j}\overline{\chi}_{n_j}(w_j))}
&=
e\left(\overline{s\uhor_{\alpha_j}\overline{\chi}_{n_j}(w_j)}\right)\in \pi\uver_*(\BSWres_{m-1} L)%
\\
&=
e\left({s\uhor_{\alpha_j}\chi_{n_j}(w_j)}\right)\\
&=\chi_{m-1}(d\uhor_{0}v) \in Z N\uhor_{m-1}\pi\uver_*(\BSWres L).%
\end{alignat*}
%where the final observation, that $\overline{e^\textup{rep}}$ is a \emph{normalized cycle}, is due to the fact that the formulae of \citeBOX[\S8]{CurtisSimplicialHtpy.pdf}  preserve the normalized subcomplex.
Our definition of $\overline{\chi}_m(v)$ is
\[\overline{\chi}_m(v):=z_{e^\textup{rep}(s\uhor_{\alpha_j}\overline{\chi}_{n_j}(w_j))}\in ZN\uver_*\BSWres_{m}L.\]
To check that the class of $\overline{\chi}_m(v)$ in $\pi\uver_*(\BSWres_mL)$ is in fact in $N\uhor_m\pi\uver_*(\BSWres L)$,  for $1\leq i \leq m$ (c.f.\ \cite[Lemma 2.7]{StoverVanKampen.pdf}):
\[d\uhor_i\overline{\chi}_m(v) =z_{d\uhor_{i-1}e^\textup{rep}(s\uhor_{\alpha_j}\overline{\chi}_{n_j}(w_j))}, \textup{ and }\overline{d\uhor_{i-1}e^\textup{rep}(s\uhor_{\alpha_j}\overline{\chi}_{n_j}(w_j))}=d\uhor_{i-1}\chi_{m-1}(d\uhor_0v)=0.\]
%where, by construction, 
%\[e^\textup{rep}(s\uhor_{\alpha_j}\overline{\chi}_{n_j}(w_j))\in ZN\uver_*(\BSWres_{m-1} L)\textup{ represents }\chi_{m-1}(d_{0}\uhor(v))\in ZN_{m-1}^\textup{h}\pi\uver_*(\BSWres  L).\]
By construction of the comonad $\BSW$, $d\uhor_i\overline{\chi}_m(v)$ must itself be null. 
%As such, the subscript $d^\textup{h}_{i-1}e^\textup{rep}(s^\textup{h}_{\alpha_j}\overline{\chi}_{n_j}(w_j))\in ZN\uver_*(\BSWres_{m-2} L)$ represents $0\in\pi\uver_*(\BSWres_{m-2}L)$. Choosing a nullhomotopy $H\in N\uver_{*+1}(\BSWres_{m-2}L)$, $h_H$ is a null-homotopy of $d_i^\textup{h}\overline{\chi}_m(v)$, showing that the class of  $\overline{\chi}_m(v)$ lies in $N^\textup{h}_m\pi\uver_*(\BSWres L)$.
%\[e^\textup{rep}(s^\textup{h}_{\alpha_j}\overline{\chi}_{n_j}(w_j))=\chi_{m-1}(d_{0}^\textup{h}(v))\in ZN\uver_*\]
Thus $\overline{\chi}_m$ does induce a map $\chi_m:V'_m\to N\uhor_m\pi\uver_*(\BSWres L)$, completing the construction of $\chi$.

Recall that the operations of \S\ref{section: vertical Koszul operations n positive} are the maps induced on cohomology by the degree -1 endomorphism $\theta^i$ of the chain complex $N\uhor_*(Q^{\calw} B^{\calw}\pi\uver_*L)$:
\[\theta^i:N\uhor_{p+1}(Q^{\calw} B^{\calw}\pi\uver_*L)^{2t+1}_{q+i-1,2s_n,\ldots,2s_1}\to N\uhor_{p}(Q^{\calw} B^{\calw}\pi\uver_*L)^{t}_{q,s_n,\ldots,s_1}.\]
If we write $V=Q^{\calL}\BSWres L$ for the double complex yielding the spectral sequence, our goal is to identify these operations with the spectral sequence operations
\[\psi_\BSW^*\circ\vExtCohOp^{i-1}:\Bigl(\Edownup{2}{}{V}{p,q}{}\overset{\vExtCohOp^{i-1}}{\to} \Edownup{2}{}{S^2V}{p,q+i-1}{}\overset{\psi_\BSW^*}{\to}\Edownup{2}{}{V}{p+1,q+i-1}{}\Bigr)\]
using the equivalence $Q^{\calw}\chi$ in $s\vect{}{}$ induced by $\chi$ and the isomorphism $\gamma$:%a weak equivalence $Q^{\calw}(\chi)$ in , which we denote:
\[Q^{\calw}\chi:\Bigl(Q^{\calw}B^{\calw} \pi\uver_*L\overset{Q^{\calw}\chi}{\to} Q^{\calw}\pi\uver_*(\BSWres )\overset{\gamma}{\to}\pi\uver_*(V)\Bigr).\]
%Now, $\gamma_i$ induces a map on $\calw$-homology, whose dual we would like to identify with the composite:

The composite $\psi_\BSW^*\circ\vExtCohOp^{i-1}$ has been identified as the dual of the composite in the bottom row of
%\[\left(E_{p,q}^2(V)\overset{(\vExtCohOp^{i-1})^*}{\from} E_{p,q+i-1}^2(S^2V)\overset{\psi_\BSW}{\from}E_{p+1,q+i-1}^2(V)\right),\]
%as Singer gives the formula on $meep^2$ \textbf{and it is easy} for vertical operations, and we talked about the dual operations $(\vExtCohOp^{i-1})^*$ like this in []. 
\[\xymatrix@R=4mm{
N\uhor_{p+1}(Q^{\calw}B^{\calw}\pi\uver_*L)_{q+i-1}\ar[rr]^-{\theta^i}
\ar[d]^-{Q^{\calw}\chi}
&%r1c1
&%r1c2
N\uhor_p(Q^{\calw}B^{\calw}\pi\uver_*L)_q\ar[d]^-{Q^{\calw}\chi}
\\%r1c3
N\uhor_{p+1}(\pi\uver_*V)_{q+i-1}\ar[r]^-{\psi_\BSW}
&%r2c1
N\uhor_{p}(\pi\uver_*S^2V)_{q+i-1}\ar[r]^-{(\vExtCohOp^{i-1})^\star}
&%r2c2
N\uhor_p(\pi\uver_*V)_q%r2c3
}\]
so that it is enough to prove that this diagram commutes for $1\leq i\leq q$.

Given the equations in $\S\ref{Linearly dual homotopy operations}$ defining the operations $(\vExtCohOp^{i-1})^\star$, it will suffice to show that the composite
\[(Q^{\calw}B^{\calw}_{p+1}\pi\uver_*L)_{q+i-1}\overset{Q^{\calw}\chi}{\to}(\pi\uver_*Q^{\calL}\BSWres_{p+1}L)_{q+i-1}\overset{\psi_\BSW}{\to}(\pi\uver_*S^2Q^{\calL}\BSWres_pL)_{q+i-1}\]
equals the sum of the composite
\[\begin{split}
&(Q^{\calw}B^{\calw}_{p+1}\pi\uver_* L)_{q+i-1}\overset{\psi_{\calw}}{\to}(S^2Q^{\calw}B^{\calw}_{p}\pi\uver_* L)_{q+i-1}\\
 &\qquad\qquad \overset{S^2(Q^{\calw}\chi)}{\to}
(S^2\pi\uver_*Q^{\call}\BSWres_pL)_{q+i-1}\overset{\widetilde{\nabla}}{\to}
(\pi\uver_*S^2Q^{\calL}\BSWres_{p}L)_{q+i-1}
\end{split}\]
and those composites, for $1\leq i \leq q$,
\[(Q^{\calw}B^{\calw}_{p+1}\pi\uver_* L)_{q+i-1}\overset{\theta^i}{\to} (Q^{\calw}B^{\calw}_{p}\pi\uver_* L)_q\overset{Q^{\calw}\chi}{\to} (\pi\uver_*Q^{\call}\BSWres_{p}L)_q\overset{\sigma_{i-1}}{\to} (\pi\uver_*S^2Q^{\calL}\BSWres_{p}L)_{q+i-1},\]
that are actually defined (these fail to be defined when $i=1$ in internal degrees satisfying $s_n=\cdots =s_1=0$). \todo{`non-zero when $i=1$'?}


By Lemma \ref{lemma on homology class repd by normalized generator}, we may represent any homology class of interest by an element $E=\sum_kv_k$, where the $v_k\in V'_{p+1}$ are elements of the basis chosen while defining $\chi$. We wrote each $v_k$ as a $\calw$-expression $e_k$ in various $u_{kj}\in V_{p}$:
\[v_k:=e_{k}(u_{kj})\in (V'_{p+1})_{q+i-1}\subseteq F^{\calw}V_p,\]
so that $d\uhor_0v=e_k(u_{kj})$, and  defined  $\chi(v_k)$ by the formula 
\[\chi(v_k)=z_{e_{k}^\textup{rep}\chi(u_{kj})}.\]
That each $\chi(u_{kj})$ is a sum of the classes $z_a$ implies that 
\[\psi_\BSW(\chi(v_k))=\quadratic_{\calL}(e_{k}^\textup{rep})(\chi(u_{kj})).\]

Taking $\quadratic_{\call}(e_{k}^\textup{rep})$ extracts the part of $e_{k}^\textup{rep}$ corresponding to the quadratic grading 2 part of $e_{k}$, in $\quadgrad{2}F^{\calw}$. That is, we may write $e_{k}\in F^{\calw}V_p$ as
\[e_{k}=\quadratic_{\calw}(e_k)(u_{kj})+\textstyle\sum_{1\leq i\leq q} \lambda_{i-1}(\theta^ie_k)(u_{kj}) + w\in F^{\calw}V_p,\]
where $w\in F^{\calw}V_p$ is the quadratic grading$\makebox[0cm][l]{}\neq2$ part of $e_{k}$, if we view $\quadratic_{\calw}(e_k)\in S^2V_p$ as an element of $F^{\calw}V_p$ via the inclusion $F^{\calL(n+1)}V_p\to F^{\calw(n+1)}V_p$, and then
%\[\quadratic_{\calL}(f_{p+1}^\textup{rep})(\chi(u_k ))=\quadratic_{\calL} (\textup{!}(\widetilde{\nabla}(y)(\chi(u_k))))+\sum_j \quadratic_{\calL}(\lambda^\textup{rep}_{n_j})(\chi(x_j))\in \pi_*(S^2(Q^{\calL}G_pL))\]
\begin{alignat*}{2}
\psi_\BSW(\chi(v_k))&=\quadratic_{\calL} \left(\widetilde{\nabla}(\quadratic_{\calw}(e_k))(\chi(u_{kj}))+\sum \sigma_{i-1}(\theta^ie_k)(\chi(u_{kj}))\right)\\
&=\widetilde{\nabla}(\quadratic_{\calw}(e_k))(\chi(u_{kj}))+\sum \sigma_{i-1}(\theta^ie_k)(\chi(u_{kj}))\\
&=\left(\widetilde{\nabla}\circ S^2(Q^{\calw}\chi)\circ \psi_{\calw}+\sum\sigma_{i-1}\circ Q^{\calw}\chi\circ\theta^i\right)(v_k).
\end{alignat*}
We were able to discard the application of $\quadratic_{\calL}$ as its argument already has quadratic grading 2. This formula is exactly what we needed to check in order to use the equations in $\S\ref{Linearly dual homotopy operations}$.
\end{proof}

\begin{proof}[Proof of Theorem \ref{EInftyCompFuncOperationsID}]
Write $L=Q^{\calU(n)} B^{\calw(n)}X \in s\calL(n)$,  $\calL=\calL(n)$ and $\calw=\calw(n)$ (not $\calw(n+1)$). We only need to show that the diagram of chain complexes
\[\xymatrix@R=4mm{
T_{m}(Q^{\calL}\BSWres L)
\ar[r]^-{\psi_\BSW}
\ar[d]^-{\epsilon}_-{d\uhor_0}
&%r1c1
T_{m-1}(S^2(Q^{\calL}\BSWres L))\ar[d]^-{\epsilon}_-{d\uhor_0}
\\%r1c3
N_{m}(Q^{\calL}L)
\ar[r]^-{\psi_{\calw}}
&%r1c1
N_{m-1}(S^2(Q^{\calL} L))
}\]
commutes up to homotopy (recall that $\psi_\BSW$ reduces filtration by one).  The augmentation maps $d\uhor_0$ are induced by the augmentation of $\BSW$:
\[\epsilon:\bigl(N\uhor_0N\uver_*(Q^{\call}\BSWres L)= N\uver_*Q^{\calL}\BSW L\overset{\epsilon}{\to} N\uver_*Q^{\calL}L\bigr). \]%=N\uver_*Q^{\calL}Q^{\calU}B^{\calw}X\]
We may understand $ N\uver_*Q^{\calL}\BSW L$ using the pushout square of chain complexes (obtained by applying $N\uver_*\circ Q^{\call} $ to that defining $\BSW$):
\[\xymatrix@R=4mm{
\bigoplus_{y\in \homog (N_{\geq1}L)}\Ftwo\{z_{dy}\}
\ar[r]\ar[d]%^-{\bigsqcup\bigsqcup\imath}
&%r1c1
\bigoplus_{x\in \homog (ZN_*L)}\Ftwo\{z_{x}\}\ar[d]\\%r1c2
\bigoplus_{y\in \homog (N_{\geq1}L)}\Ftwo\{h_y,z_{dy}\}
\ar[r]&%r2c1
N\uver_*Q^{\call}\BSW L%r2c2
}\]
which shows that $N\uver_*Q^{\call}\BSW L$ is the following complex (with differential $h_{y}\mapsto z_{d\uver_0y}$):
\[N\uver_*Q^{\call}\BSW L=\textstyle\bigoplus_{y\in \homog (N_{\geq1}L)}\Ftwo\{h_y\}\oplus \bigoplus_{x\in \homog (ZN_*L)}\Ftwo\{z_{x}\}\]
We will use the notation
\[L_t:=Q^{\calU}B^{\calw}_tX \cong F^{\calL}_{\langle -1\rangle}F^{\calw}_{\langle 0\rangle}\cdots F^{\calw}_{\langle t-1\rangle}X_t\]
so that we may write the basis elements of $N_mQ^{\calL}\BSW Q^{\calU}B^{\calw}X$ in the form
\[z_{f^{\langle -1\rangle}(g^{\smash{\langle} 0\smash{\rangle}}_{i_1}(h^{\smash{\langle} 1\smash{\rangle}}_{i_1i_2}))} \textup{ and }h_{f^{\langle -1\rangle}(g^{\smash{\langle} 0\smash{\rangle}}_{i_1}(h^{\smash{\langle} 1\smash{\rangle}}_{i_1i_2}))},\]
where the $h_{i_1i_2} $ are various elements of $ F^{\calw}_{\langle 1\rangle}\cdots F^{\calw}_{\langle m-1\rangle}X_m$, each $g_{i_1}$ is a $\calw$-expression $g_{i_1}(h_{i_1i_2})$ in certain of the $h_{i_1i_2}$, and finally, $f$ is some $\calL$-expression in the various $g_{i_1}$. % .   $f$ is an operator in $F^{\calL}$, while $g$, $h$, etc.\ are operators in $F^{\calw}$.
For brevity we will write $k_{f^{\langle -1\rangle}(g^{\smash{\langle} 0\smash{\rangle}}_{i_1}(h^{\smash{\langle} 1\smash{\rangle}}_{i_1i_2}))}$ for either of $z_{f^{\langle -1\rangle}(g^{\smash{\langle} 0\smash{\rangle}}_{i_1}(h^{\smash{\langle} 1\smash{\rangle}}_{i_1i_2}))}$ and $h_{f^{\langle -1\rangle}(g^{\smash{\langle} 0\smash{\rangle}}_{i_1}(h^{\smash{\langle} 1\smash{\rangle}}_{i_1i_2}))}$.

A chain homotopy $\Phi:T_m(Q^{\calL}\BSWres L)\to N_m(S^2(Q^{\calw}B^{\calw}X))$ is constructed as follows. Let $\Phi$ be zero except on 
$N\uhor_0N\uver_m (Q^{\call}\BSWres L)=N\uver_mQ^{\calL}\BSW Q^{\calU}B^{\calw}X$,
where it is defined by
%We then define a homotopy 
%\[\Phi:T_*(Q^{\calL}\BSWres L)\to N_*(S^2(Q^{\calw}B^{\calw}X))\]
\[\smash{k_{f^{\langle -1\rangle}(g^{\smash{\langle} 0\smash{\rangle}}_{i_1}(h^{\smash{\langle} 1\smash{\rangle}}_{i_1i_2}))}}\mapsto {\quadratic_\call}(f)(g^{\smash{\langle} 0\smash{\rangle}}_{i_1}(h^{\smash{\langle} 1\smash{\rangle}}_{i_1i_2})).\]
This definition makes sense (and yields a non-trivial map) because $f\relax $ is an operator in $Q^{\calU}F^{\calw}=F^{\calL}$.
The chain map $d\Phi+\Phi d$ is a sum of three terms:
\begin{enumerate}\squishlist
\setlength{\parindent}{.25in}
\item[(a)] $d\circ\Phi:N\uhor_0N\uver_m (Q^{\call}\BSWres L)\overset{\Phi}{\to} N_m(S^2(Q^{\call}L))\overset{d}{\to} N_{m-1}(S^2(Q^{\call}L))$ 
\item[(b)] $\Phi\circ d\uver:N\uhor_0N\uver_{m} (Q^{\call}\BSWres L)\overset{d\uver}{\to} N\uhor_0N\uver_{m-1}(Q^{\call}\BSWres L)\overset{\Phi}{\to} N_{m-1}(S^2(Q^{\call}L))$
\item[(c)] $\Phi\circ d\uhor:N\uhor_1N\uver_{m-1}(Q^{\call}\BSWres L)\overset{d\uhor}{\to} N\uhor_0N\uver_{m-1}(Q^{\call}\BSWres L)\overset{\Phi}{\to} N_{m-1}(S^2(Q^{\call}L))$
\end{enumerate}
We calculate
\begin{alignat*}{2}
(d\circ\Phi)(k_{f\relax (g^{\smash{\langle} 0\smash{\rangle}}_{i_1}(h^{\smash{\langle} 1\smash{\rangle}}_{i_1i_2}))})&=d( {\quadratic_\call}(f)(g^{\smash{\langle} 0\smash{\rangle}}_{i_1}(h^{\smash{\langle} 1\smash{\rangle}}_{i_1i_2})))\\
&={\quadratic_\call}(f)(g\relax _{i_1}(h^{\smash{\langle} 0\smash{\rangle}}_{i_1i_2}))\\
&={\quadratic_\call}(f)(\epsilon(g\relax _{i_1})(h^{\smash{\langle} 0\smash{\rangle}}_{i_1i_2})),
\end{alignat*}
(the last equation holds as we calculate in $S^2(Q^{\call}L)$), and 
\begin{alignat*}{2}
(\Phi\circ d\uver)(k_{f\relax (g^{\smash{\langle} 0\smash{\rangle}}_{i_1}(h^{\smash{\langle} 1\smash{\rangle}}_{i_1i_2}))})&=\begin{cases}
\Phi( k_{f\relax (g\relax _{i_1})(h^{\smash{\langle} 0\smash{\rangle}}_{i_1i_2}))}),&\textup{if `$k$' stands for `$h$'},\\
\Phi(0),&\textup{if `$k$' stands for `$z$'},
%\\,&\textup{if }
\end{cases}\\
&={\quadratic_\call}(f(g_{i_1}))(h^{\smash{\langle} 0\smash{\rangle}}_{i_1i_2})\quad\textup{(in either case).}
\end{alignat*}
By the equation of \S\ref{quadratic part section}, the sum of these two terms is ${\quadratic_\call}(\epsilon f(g_{i_1}))(h^{\smash{\langle} 0\smash{\rangle}}_{i_1i_2})$, which is exactly the formula for $(\psi_{\calw}\circ\epsilon)(k_{f\relax (g^{\smash{\langle} 0\smash{\rangle}}_{i_1}(h^{\smash{\langle} 1\smash{\rangle}}_{i_1i_2}))})$.

It remains to show that $\Phi\circ d\uhor$ coincides with $\epsilon^{\otimes 2}\circ\psi_{\calw}$. These two maps are only non-zero on the graded part $N\uhor_1N\uver_{m-1}(Q^{\call}\BSWres L)\subseteq Q^{\calL}\BSW^2 L$ of  $T_{m}(Q^{\calL}\BSWres L)$, and an element therein is a linear combination
\[K:=\textstyle\sum_j k_{e_j\bigl(\smash{s_{\alpha_{ji_0}} k_{f\relax _{ji_0}g^{\smash{\langle} 0\smash{\rangle}}_{ji_0i_1}h^{\smash{\langle} 1\smash{\rangle}}_{ji_0i_1i_2}}}\bigr)}\]
which satisfies the equation $d\uhor_1(K)=0$, i.e.:
%\[d\uhor_1(K)=\sum_j k_{e_j\left(s_{\alpha_{ji_0}} {f\relax _{ji_0}g^{\smash{\langle} 0\smash{\rangle}}_{ji_0i_1}h^{\smash{\langle} 1\smash{\rangle}}_{ji_0i_1i_2}}\right)}=0.\]
%We may rewrite this equation as
\[d\uhor_1(K)=\textstyle\sum_j \smash{k_{e_j(f_{ji_0})\left(s_{\alpha_{ji_0}} {g^{\smash{\langle} 0\smash{\rangle}}_{ji_0i_1}h^{\smash{\langle} 1\smash{\rangle}}_{ji_0i_1i_2}}\right)}}=0\textup{ in }N_{m-1}Q^{\calL}\BSW L.\]
%This equation is very strong, as it holds in the free vector space on the symbols $k_{\cdots}$. Indeed, there is a map %, $\textstyle\bigoplus_{y\in \homog (N_{\geq1}L)}\Ftwo\{h_y\}\oplus \bigoplus_{x\in \homog (ZN_*L)}\Ftwo\{z_{x}\}$
%
%
%Since the symbol $k$ is used in this condition, it is \emph{very} strong, implying that the element
%\[\textstyle\sum_j \underline{e_j(f_{ji_0})\bigl(s_{\alpha_{ji_0}} {g^{\smash{\langle} 0\smash{\rangle}}_{ji_0i_1}h^{\smash{\langle} 1\smash{\rangle}}_{ji_0i_1i_2}}\bigr)}=0\]
%in the free vector space $\Ftwo \{\homog( N_{*-1}L)\} \oplus \Ftwo \{\homog( ZN_{*-1}L)\}$.
There is a map 
\[N_{m-1}Q^{\calL}\BSW L \subseteq \Ftwo \{\homog( N_{m-1}L)\} \oplus \Ftwo \{\homog( ZN_{m-1}L)\}\to N_{m-1}S^2Q^{\calL}L\]
defined on generators using the function
\[N_{m-1}L\subseteq F^{\calL}(F^{\calw})^{m-1}X \smash{{}\overset{\quadratic_\call}{\to}{}}S^2(F^{\calw})^{m-1}X\cong S^2Q^{\calL}L.\]
This map sends $d\uhor_1(K)=0$ to %As the sum above is zero in the free vector space, its image under this map is certainly zero, so that there holds the following equation in $N_{*-1}(S^2(Q^{\call}L))$:
%\[\sum_j {q(e\relax _j(f\relax _{ji_0}))\left(s_{\alpha_{ji}} g^{\smash{\langle} 0\smash{\rangle}}_{ji_1}h^{\smash{\langle} 1\smash{\rangle}}_{ji_1i_2}\right)}=0\in N_{*-1}((QX)^{\otimes2}).\]
\[
\textstyle\sum_j {\quadratic_\call}(e_j(f_{ji_0}))\smash{\left(s_{\alpha_{ji_0}} {g^{\smash{\langle} 0\smash{\rangle}}_{ji_0i_1}h^{\smash{\langle} 1\smash{\rangle}}_{ji_0i_1i_2}}\right)}=0, \]
which by the equation of \S\ref{quadratic part section}, gives an equation in $N_{m-1}S^2Q^{\calL}L$:
\[\textstyle\sum_j {\quadratic_\call}(e_j(\epsilon (f_{ji_0})))\smash{\left(s_{\alpha_{ji_0}} {g^{\smash{\langle} 0\smash{\rangle}}_{ji_0i_1}h^{\smash{\langle} 1\smash{\rangle}}_{ji_0i_1i_2}}\right)}
=
\textstyle\sum_j {\quadratic_\call}(\epsilon (e_j)(f_{ji_0}))\smash{\left(s_{\alpha_{ji_0}} {g^{\smash{\langle} 0\smash{\rangle}}_{ji_0i_1}h^{\smash{\langle} 1\smash{\rangle}}_{ji_0i_1i_2}}\right)}.
\]
The proof is completed upon noting that the left hand side of this equation equals $(\epsilon^{\otimes 2}\circ \psi_\BSW)(K)$, while the right hand side equals $(\Phi\circ d\uhor)(K)$. We calculate:
\begin{alignat*}{2}
(\epsilon^{\otimes 2}\circ \psi_\BSW)(K)
&=
\epsilon^{\otimes 2}\left(\textstyle\sum_j {\quadratic_\call}(e_j)\left(s_{\alpha_{ji_0}} k_{f\relax _{ji_0}g^{\smash{\langle} 0\smash{\rangle}}_{ji_0i_1}h^{\smash{\langle} 1\smash{\rangle}}_{ji_0i_1i_2}}\right)\right)%
\\
&=
\textstyle\sum_j {\quadratic_\call}(e_j)\left(s_{\alpha_{ji_0}} \epsilon(f\relax _{ji_0})g^{\smash{\langle} 0\smash{\rangle}}_{ji_0i_1}h^{\smash{\langle} 1\smash{\rangle}}_{ji_0i_1i_2}\right)%
\\
&=
\textstyle\sum_j {\quadratic_\call}(e_j(\epsilon (f_{ji_0})))\left(s_{\alpha_{ji_0}} g^{\smash{\langle} 0\smash{\rangle}}_{ji_0i_1}h^{\smash{\langle} 1\smash{\rangle}}_{ji_0i_1i_2}\right)=\textup{LHS},
\end{alignat*}
and
\begin{alignat*}{2}
d\uhor(K)
&=
\textstyle\sum_j e_j\left(s_{\alpha_{ji_0}} k_{f\relax _{ji_0}g^{\smash{\langle} 0\smash{\rangle}}_{ji_0i_1}h^{\smash{\langle} 1\smash{\rangle}}_{ji_0i_1i_2}}\right)%
\\
&=
\textstyle\sum_j \epsilon (e_j)\left(s_{\alpha_{ji_0}} k_{f\relax _{ji_0}g^{\smash{\langle} 0\smash{\rangle}}_{ji_0i_1}h^{\smash{\langle} 1\smash{\rangle}}_{ji_0i_1i_2}}\right)%
&\qquad&\text{(in $N\uver_{m-1}Q^{\call}\BSW L)$}\\
\Phi(d\uhor(K))&=
\textstyle\sum_j \epsilon (e_j)\left( {\quadratic_\call}(f_{ji_0})(s_{\alpha_{ji_0}}g^{\smash{\langle} 0\smash{\rangle}}_{ji_0i_1}h^{\smash{\langle} 1\smash{\rangle}}_{ji_0i_1i_2})\right)%
&\qquad&\text{(relevant $s_{\alpha_{ji_0}}$ are $\Id$)}\\
&=
\textstyle\sum_j {\quadratic_\call}(\epsilon (e_j)(f_{ji_0}))\left(s_{\alpha_{ji_0}} {g^{\smash{\langle} 0\smash{\rangle}}_{ji_0i_1}h^{\smash{\langle} 1\smash{\rangle}}_{ji_0i_1i_2}}\right) ={}\makebox[0cm][l]{\textup{RHS}}.
\end{alignat*}
To explain further the third equation, note that $N\uver_{m-1}Q^{\call}\BSW L$ is spanned by the classes $k_{\cdots }$, and none of  their degeneracies. Thus, all of the degeneracies $s_{\alpha_{ji_0}}$ appearing in the second line that have not already been annihilated during the application of $\epsilon$ must be the identity. Thus, they can be carried harmlessly through to the end of the calculation, as shown.
\end{proof}













\end{Operations in composite functor spectral sequences}







\begin{Calculations of HWn}
\SectionOrChapter{Calculations of ${\calw(n)}$-cohomology}
\label{Calculations of HWn}
In this section, we will calculate the value of $H^*_{\calw(n)}X$ for certain  objects $X$ of $\calw(n)$ of finite type. In each subsection, we will write $V_{(n)}=\dual X$, so that $X$ has underlying vector space dual to $V_{{(n)}}\in\vect{n}{+}$. In fact, we will reinstate the upper asterisk for linear dualization, writing $\dual V_{(n)}:=V^*_{(n)}$, and recursively define:
\[V_{(k+1)}:=H^*_{\calU(k)}V^*_{(k)}\textup{ and }V^*_{(k+1)}:=H_*^{\calU(k)}V^*_{(k)}\textup{ for $k\geq n$}.\]
In this way, for each $k\geq n$, $V^*_{(k+1)}$ is an object of $\calw(k+1)$, vector space dual to $V_{(k+1)}\in\vect{k+1}{+}$, which itself has the structure of an object of $\calMv(k+1)$.
%We will calculate $V^*_{(k+1)}$ as an object of $\calw(k+1)$ at each stage. 
Having all of this data will allow us to draw conclusions about $H^*_{\calw(n)}V^*_{(n)}$, using, for each $k\geq n$, the $(k+1)^{\textup{st}}$ composite functor spectral sequence:
\[\E{(k+1)}{2}{}{s_{k+2},\ldots,s_1}{t}:=(H^*_{\calw(k+1)}V^*_{(k+1)})^{s_{k+2},\ldots,s_1}_{t}\implies (H^*_{\calw(k)}V^*_{(k)})^{s_{k+2}+s_{k+1},s_k,\ldots,s_1}_{t}.\]
The $1^{\textup{st}}$ \CFSS, which calculates $H^*_{\calw(0)}$ from $H^*_{\calw(1)}$, will appear in \S\ref{Calculations of HW0}.



\SubsectionOrSection{When $X\in\calw(n)$ is one-dimensional and $n\geq1$}
\label{one-dimensional, n geq1}
Let $X=V^*_{(n)}\in\calw(n)$ be a one dimensional object of $\calw(n)$, %dual to a vector space $V_{(n)}=(V_{(n)})^S_T=\langle v\rangle\in\vect{n}{+}$, where $T\geq1$ and $S\geq0$ (\textbf{better:} 
dual to a one-dimensional vector space $V_{(n)}\in\vect{n}{+}$, with non-zero element $v\in(V_{(n)})^{S_n,\ldots,S_1}_T$. Write $v^*\in X^T_{S_n,\ldots,S_1}$ for the non-zero element of $X$. As every $\calw(n)$-operation changes degrees, $X$ is necessarily trivial. We distinguish two cases: when $v$ is \emph{restrictable} and when $v$ is \emph{not restrictable}. Recall that $v$ is said to be restrictable when $\restn{v}$ is defined, i.e.\ when $S_n,\ldots,S_1$ are not all zero.
\begin{prop}
\label{iterative calc of the Vk all trivial}
For each $k\geq n$,
\[V_{(k)}=F^{\calMv(k)}F^{\calMv(k-1)}\cdots F^{\calMv(n+1)}V_{(n)},\]
and $V^*_{(k)}$ is a trivial object of $\calw(k)$.
\end{prop}
\begin{proof}
The proof is by induction, with the case $k=n$ simply our standing assumptions. If the statement holds for $V_{(k)}$, then Proposition \ref{propDerivedIndTrivialUobject n at least 1} shows that the Koszul complex calculating $V_{(k+1)}$ has zero differentials, as $V_{(k)}$ has trivial $\calw(k)$-structure, so that $V_{(k+1)}=F^{\calMv(k+1)}V_{(k)}$. This has trivial $\calw(k+1)$-structure, by the results of \S\ref{section on structure on homology of koszul cx}.
\end{proof}
Our next step is to calculate, for $k\geq n$, the groups:
\[\E{(k+1)}{2}{}{s_{k+2},0,s_k,\ldots,s_1}{t}:=(H^*_{\calw(k+1)}V^*_{(k+1)})^{s_{k+2},0,s_k,\ldots,s_1}_{t}\cong (H^{s_{k+2}}_{\calL(k)}V^*_{(k)})^{s_k,\ldots,s_1}_{t}.\]
The isomorphism shown here follows from the observation that in dimension $s_{k+1}=0$, an object of $\calw(k+1)$ is nothing more than an object of $\calL(k)$. More precisely, consider the functor $\DASH_\textbf{0}:\vect{+}{k+1}\to \vect{+}{k}$ given by
\[(Y_\textbf{0})^t_{s_k,\ldots,s_1}:=Y^t_{0,s_k,\ldots,s_1}.\]
Then $\DASH_\textbf{0}$ induces a functor $\DASH_\textbf{0}:\calw(k+1)\to \calL(k)$, such that, for all $Y\in\calw(k+1)$:
\[(F^{\calw(k+1)}(Y))_\textbf{0}\cong F^{\calL(k)}(Y_\textbf{0})\textup{ and }(Q^{\calw(k+1)}Y)_\textbf{0}\cong Q^{\calL(k)}(Y_\textbf{0}),\]
so that $(Q^{\calw(k+1)}B^{\calw(k+1)}Y)_{\textbf{0}} \cong(Q^{\calL(k)}B^{\calL(k)}Y_{\textbf{0}})$ for any $Y\in s\calw(k+1)$, and thus:
\begin{prop}
Suppose that $Y\in s\calw(k+1)$, where $k\geq0$. Then
\[(H^*_{\calw(k+1)}Y)^{s_{k+2},0,s_k,\ldots,s_1}_{t}\cong (H^{s_{k+2}}_{\calL(k)}Y_\textup{\textbf{0}})^{s_k,\ldots,s_1}_t.\]
\end{prop}
Returning to the calculation at hand, we may identify a part of the $E_2$-page with the Chevalley-Eilenberg-May complex of Appendix \ref{The Chevalley-Eilenberg-May complex}:
\begin{prop}
\label{calculation in internal dimension zero}
For each $k\geq n$, there is an isomorphism of commutative algebras:
\[\E{(k+1)}{2}{}{s_{k+2},0,s_k,\ldots,s_1}{t} \cong(\dual\UEAX(V^*_{(k)}))^{s_{k+2},s_k,\ldots,s_1}_t.\]
%where the argument $V_{(k)}^*=(V_{(k+1)}^*)_\textbf{\textup{0}}$ of the Chevalley-Eilenberg-May complex functor $\UEAX$ is a trivial object of $\calL(k)$. 
When $v\in V_{(n)}$ is restrictable,  $\dual\UEAX(V^*_{(k)})=S(\CommOperad)[V_{(k)}]$, the free non-unital commutative algebra. When $v\in V_{(n)}$ is not restrictable, $V_{(k)}=\Ftwo\{v\}$ is one-dimensional, and $\dual\UEAX(V^*_{(k)})$ is the one-dimensional exterior algebra $\Lambda(\CommOperad)[v]$. In either case, for each individual value of the grading $t$, the group \[\bigoplus_{s_{k+2},s_k,\ldots,s_1}\E{(k+1)}{2}{}{s_{k+2},0,s_k,\ldots,s_1}{t}\] is finite-dimensional.
\end{prop}
\begin{proof}
The only further observation necessary to prove this isomorphism is that if $v\in V_{(n)}$ is restrictable, every element of the trivial partially restricted Lie algebra $V_{(k)}$ is in restrictable degree, and that if $v\in V_{(n)}$ is not restrictable, each $V_{(k)}$ is one-dimensional, concentrated in non-restrictable degree. For the finiteness property, one simply notes that the $V_{(k)}$ have such a property, and that there is a degree shift in the algebra structure.
\end{proof}
%\begin{thm}
%There is an isomorphism $F^{\calMh(2)}F^{\calMv(2)}V_{(1)}\to H^*_{\calw(1)}V_{(1)}^* $, under which the filtration on $H^*_{\calw(1)}V_{(1)}^*$ arising from the composite functor spectral sequence coincides with the filtration on $F^{\calMh(2)}F^{\calMv(2)}V_{(1)}$ given in Theorem \ref{thm on compressing seqs of steenrod ops}.
%\end{thm}

Consider the diagram:
\[\qquad\xymatrix@R=1.5mm{
&
H^*_{\calw({n+1})}H_*^{\calU(n)}V^*_{(n)}\ar@{=>}[dl]^-{g_{n+1}}\ar@{=}[d]&
H^*_{\calw({n+2})}H_*^{\calU({n+1})}V^*_{({n+1})}\ar@{=>}[dl]^-{g_{n+2}}\ar@{=}[d]&
%H^*_{\calw({n+3})}H_*^{\calU({n+2})}V^*_{({n+2})}\ar@{=>}[dl]^-{g_{n+3}}\ar@{=}[d]&
%H^*_{\calw(5)}H_*^{\calU({n+3})}V^*_{({n+3})}\ar@{=>}[dl]^-{g_5}\ar@{=}[d]
\\
 H^*_{\calw(n)}V^*_{(n)}
&H^*_{\calw({n+1})}V^*_{({n+1})}
&H^*_{\calw({n+2})}V^*_{({n+2})}
%&H^*_{\calw({n+3})}V^*_{({n+3})}
%&H^*_{\calw(5)}V^*_{(5)}
\\
&&\makebox[5cm][r]{\,$\smash{\cdots }$}
\\
 F^{\calMhv({n+1})}V_{(n)}\ar[uu]_-{\rho_n}\ar@{=}[d]
&F^{\calMhv({n+2})}V_{({n+1})}\ar[uu]_-{\rho_{n+1}}\ar@{=}[d]\ar@{=>}[ld]^-{f_{n+1}}
&F^{\calMhv({n+3})}V_{({n+2})}\ar[uu]_-{\rho_{n+2}}\ar@{=}[d]\ar@{=>}[ld]^-{f_{n+2}}
%&F^{\calMhv(5)}F^{\calMh(5)}V_{({n+3})}\ar[uu]_-{\rho_{n+3}}\ar@{=}[d]\ar@{=>}[ld]
%&F^{\calMhv(6)}F^{\calMh(6)}V_{(5)}\ar[uu]_-{\rho_5}\ar@{=}[d]\ar@{=>}[ld]
\\
 F^{\calMh({n+1})}V_{({n+1})}
&F^{\calMh({n+2})}V_{({n+2})}
&F^{\calMh({n+3})}V_{({n+3})}
%&F^{\calMh(5)}V_{(5)}
%&F^{\calMh(6)}V_{(6)}
}\]
For each $k\geq n$, the map $\rho_k$ is induced by the inclusion $V_{(k)}\cong H^{0}_{\calw(k)}V^*_{(k)}\subseteq H^*_{\calw(k)}V^*_{(k)}$ (which exists as $V_{(k)}$ is trivial) and the $F^{\calMhv(k+1)}$-operations defined on $H^*_{\calw(k)}V_{(k)}$. (Note that $\rho_{k}$ is a graded map, since the effect of these operations on dimensions is the same in its domain and codomain.)

The double arrow $g_{k+1}$, representing the convergence of the $(k+1)^{\textup{st}}$ \CFSS\ $\E{(k+1)}{2}{}{}{}\implies H^*_{\calw(k)}V^*_{(k)}$, is in truth shorthand for the function
%\[g_{k+1}:\left(E_{\infty,(k+1)}\to E_0(H^*_{\calw(k)}V^*_{(k)})\right)\]
\[\E{(k+1)}{\infty}{}{s_{k+2},\ldots,s_1}{t}\to \E{}{0}{H^*_{\calw(k)}V^*_{(k)}}{s_{k+2},\ldots,s_1}{t} \]
so that $g_{k+1}$ may only be defined on the permanent cycles within $\E{(k+1)}{2}{}{}{}$, %=H^*_{\calw(k+1)}H_*^{\calU(k)}V^*_{(k)}$,
 and lands in the associated graded of $H^*_{\calw(k)}V^*_{(k)}$.

Similarly, we employ the double arrow $f_{k+1}$ as shorthand for the function of Theorem \ref{thm on compressing seqs of steenrod ops}, which is defined on the entirety of $F^{\calMh(k+2)}F^{\calMv(k+2)}V_{(k+1)}$, but whose true codomain is the graded object $E_0(F^{\calMh(k+1)}V_{(k+1)})$ associated with the target filtration defined in Theorem \ref{thm on compressing seqs of steenrod ops}.
\begin{thm}
\label{thm on collapsing of most sseqs}
For each $k\geq n$, $\im(\rho_{k+1})$ consists of permanent cycles and $\rho_k$ preserves the target filtrations, so that it is possible to form the composites $g_{k+1}\circ \rho_{k+1}$ and $E_0(\rho_{k})\circ f_{k+1}$. These composites are equal, and moreover, $\rho_k$ is an isomorphism. In particular, for $k\geq n$, the  $(k+1)^{\textup{st}}$ \CFSS\ collapses at $E_2$.
\end{thm}
Before giving the proof, we remark that in some dimensions, $\rho_k$ is already known to be an isomorphism:
\begin{prop}
\label{isomorphism rho k in some dims}
For $k\geq n$, $\rho_k$ is an isomorphism in dimension $s_k=0$:
\[\rho_k:(F^{\calMh(k+1)}F^{\calMv(k+1)}V_{(k)})^{s_{k+1},0,s_{k-1},\ldots,s_1}_{t} \overset{\cong}{\to}(H^*_{\calw(k)}V^*_{(k)})^{s_{k+1},0,s_{k-1},\ldots,s_1}_{t}.\]
\end{prop}
\begin{proof}
In this dimension, $\rho_k$ factors as
\begin{alignat*}{2}
(F^{\calMh(k+1)}F^{\calMv(k+1)}V_{(k)})^{s_{k+1},0,s_{k-1},\ldots,s_1}_{t}
&=
(F^{\calMh(k+1)} F^{\calMv(k+1)}V_{(k)}^{\textbf{\textup{0}}})^{s_{k+1},0,s_{k-1},\ldots,s_1}_{t}\\
&=(F^{\calMh(k+1)}V_{(k)}^{\textbf{\textup{0}}})^{s_{k+1},0,s_{k-1},\ldots,s_1}_{t}\\
%&\cong (\dual\UEAX(\dual(V_{(k)}^{\textbf{\textup{0}}})))^{s_{k+1},0,s_{k-1},\ldots,s_1}_{t}\\
&\cong (\dual\UEAX((V^*_{(k)})_{\textbf{\textup{0}}}))^{s_{k+1},0,s_{k-1},\ldots,s_1}_{t}\\
&\cong (H^*_{\calw(k)}V^*_{(k)})^{s_{k+1},0,s_{k-1},\ldots,s_1}_{t}
\end{alignat*}
Here, we are viewing $V_{(k)}^{\textbf{\textup{0}}}$, the subspace of $V_{(k)}$ in degree $s_k=0$, as an object of $\vect{k+1}{+}$ in order to apply $F^{\calMv(k+1)}$. The inclusion $F^{\calMh(k+1)}F^{\calMv(k+1)}V_{(k)}^{\textbf{\textup{0}}}\subseteq F^{\calMh(k+1)}F^{\calMv(k+1)}V_{(k)}$ restricts to the identity in degree $s_k=0$, explaining the first equation. The second equation is similar: any non-trivial $\calMv(k+1)$-operation lands outside degree $s_k=0$. The first isomorphism follows from Corollary \ref{basis of free horizontal operations algebra restricted}, % --- as $V_{(k)}^{\textbf{\textup{0}}}$ is concentrated in dimension $s_{k+1}=0$, the proposition 
which ensures that $F^{\calMh(k+1)}V_{(k)}^{\textbf{\textup{0}}}$ is a quotient of the polynomial algebra on $V_{(k)}^{\textbf{\textup{0}}}$, and indeed, the same quotient as $\dual\UEAX(\dual(V_{(k)}^{\textbf{\textup{0}}}))$. The second isomorphism is Proposition \ref{calculation in internal dimension zero}, since $(V^*_{(k)})_{\textbf{\textup{0}}}=V^*_{(k-1)}$.
\end{proof}
\begin{proof}[Proof of Theorem \ref{thm on collapsing of most sseqs}]
For each $k\geq n$, we will use the diagram
\[\xymatrix@R=4mm{
%&&H_{\calw(k+1)}^{0}H_*^{\calU(k)}V_{(k)}^*\\
%V_{(k)}\ar@{..>}[r]^-{i_0}\ar@{..>}[rd]_-{j_0}&
H^*_{\calw(k)}V^*_{(k)}\ar@{..>}@/^1.22em/[rr]|-{\textup{\,edge composite\,}}
&H^*_{\calw(k+1)}H_*^{\calU(k)}V^*_{(k)}\ar@{=>}[l]^-{g_{k+1}}%\ar@{..>}@<.35ex>[r]^-{}%\ar[u]_-{\textup{proj onto $*=0$}}
&H^*_{\calU(k)}V^*_{(k)}\ar@{..>}[l]^-{i_1}
&V_{(k)}\ar@{..>}[l]|-{\,i_2\,}\ar@{..>}[ld]^-{j_2}\ar@{..>}@/_1.72em/[lll]|-{\,i_0\,}
\\
%V_{(k)}\ar@{..>}[r]\ar@{..>}[u]^-{\cong}&
F^{\calMh(k+1)}F^{\calMv(k+1)}V_{(k)}\ar[u]^-{\rho_k}
&W(F^{\calMv(k+1)}V_{(k)})\ar[u]\ar@{->>}[l]_-{\overline{f}_{k+1}}\ar[u]_-{\rho_{k+1}}
&F^{\calMv(k+1)}V_{(k)}\ar@{..>}[l]_-{j_1}\ar@{=}[u]%\ar@{..>}@/^1.02em/[ll]|-{\,j_3\,}
%&V_{(k)}\ar@{..>}[l]\ar@{..>}[u]_-{=}\ar@{..>}@/^1.52em/[lll]|-{\textup{inc}}
%\\
% F^{\calMh(k+1)}F^{\calMv(k+1)}V_{(k)}
%&F^{\calMh(k+2)}V_{(k+2)}
}\]
where $W(F^{\calMv(k+1)}V_{(k)})$ is the object introduced in the proof of Theorem \ref{thm on compressing seqs of steenrod ops}, so that there is a quotient map
\[W(F^{\calMv(k+1)}V_{(k)})\epi F^{\calMh(k+2)}F^{\calMv(k+2)}F^{\calMv(k+1)}V_{(k)}.\]
%of which $F^{\calMh(k+2)}F^{\calMv(k+2)}F^{\calMv(k+1)}V_{(k)}$ is a quotient. 
Here, the maps $j_1,j_2$ are the evident inclusions of generators, while the maps $i_0,i_1,i_2$ are the inclusions arising because $V_{(k)}$ is trivial.

 We may define 
\[c:=(\rho_k\circ \overline{f}_{k+1}\circ j_1):F^{\calMv(k+1)}V_{(k)}\to H^*_{\calw(k)}V^*_{(k)},\]
without the need to pass to any associated graded objects. By construction of $\overline{f}_{k+1}$,  $c$ is induced by the inclusion $i_0$ and the $\calMv(k+1)$-structure of $H^*_{\calw(k)}V^*_{(k)}$.

The edge composite is the composite of a surjection, a monomorphism $m_1$, and an isomorphism $m_2$ (with inverse $i_1$):
\[H^*_{\calw(k)}V^*_{(k)}\epi 
\E{}{0}{H^*_{\calw(k)}V^*_{(k)}}{\textbf{0}}{}
\cong
\E{(k+1)}{\infty}{}{\textbf{0}}{}
%E_{\infty,(k+1)}^{0,*}
\overset{m_1}{\to} \E{(k+1)}{2}{}{\textbf{0}}{}
%E_{2,(k+1)}^{0,*}
\overset{m_2}{\to}H^*_{\calU(k)}V^*_{(k)}.\]
Moreover, $c$ is a section of the edge composite, since both maps are compatible with $\calMv(k+1)$-structures (Proposition \ref{edgehomproposition}), and their composite is the identity on $V_{(k)}\subseteq  F^{\calMv(k+1)}V_{(k)}$.
In particular, the edge composite is a surjection, so that $m_{1}$ is an isomorphism. That is, every class in $\im(i_1)$ is a permanent cycle. Singer's work (c.f.\ \S\ref{singer ext ops sect}) then shows that $\im(\rho_{k+1})$ consists of permanent cycles, as permanent cycles are preserved by the $F^{\calMhv(k+2)}$-operations on $\E{(k+1)}{2}{}{}{}$.

%The claim that $g_{k+1}\circ \rho_{k+1}=E_0(\rho_{k})\circ f_{k+1}$ is best proven together with the claim that $\rho_k$ preserves filtration.

Any section of $H^*_{\calw(k)}V^*_{(k)}\epi \E{(k+1)}{\infty}{}{\textbf{0}}{}\cong \E{(k+1)}{2}{}{\textbf{0}}{} $ will realize, up to filtration, the restriction of $g_{k+1}$ to $\E{(k+1)}{\infty}{}{\textbf{0}}{} \subset \E{(k+1)}{\infty}{}{}{}$, so we choose
\[E_{2,(k+1)}^{0,*}\overset{m_2}{\to} H^*_{\calU(k)}V^*_{(k)}\cong F^{\calMv(k+1)}V_{(k)}\overset{c}{\to}H^*_{\calw(k)}V^*_{(k)}.\]
In particular, $g_{k+1}\circ \rho_{k+1}\circ j_1=g_{k+1}\circ i_1=c\circ m_2\circ i_1=c$, up to filtration. More precisely, $g_{k+1}\circ \rho_{k+1}\circ j_1$ equals the composite
\[
F^{\calMv(k+1)}V_{(k)}\overset{c}{\to} H^*_{\calw(k)}V^*_{(k)}\epi \E{}{0}{H^*_{\calw(k)}V^*_{(k)}}{\textbf{0}}{}.
\]
Now the target filtrations on the domain and codomain of $\rho_k$ are induced by the filtrations on the domain and codomain of $\rho_{k+1}$ by cohomological dimension $s_{k+2}$, and $\rho_{k+1}$ is a graded map. Thus, for any $w\in F^pW(F^{\calMv(k+1)}V_{(k)})$, we must see that $\rho_k(\overline{f}_{k+1}(w))$ coincides with $g_{k+1}(\rho_{k+1}(w))$ modulo $F^{p+1}H^*_{\calw(k)}V^*_{(k)}$, as this will prove both that $\rho_k$ preserves target filtrations and that  $g_{k+1}\circ \rho_{k+1}=E_0(\rho_{k})\circ f_{k+1}$. However, this coincidence follows from the fact that $c=g_{k+1}\circ \rho_{k+1}\circ j_1$, as $W(F^{\calMv(k+1)}V_{(k)})$ is generated by $\im(j_1)$ under $F^{\calMhv(k+2)}$-operations, and the definition of $\overline{f}_{k+1}$ is modeled on the interaction of $g_{k+1}$ with these operations, as studied by Singer (c.f.\ \S\ref{singer ext ops sect}).

What remains is to show that the maps $\rho_k$ are isomorphisms. Suppose that 
\[x_{(k)}\in \E{(k)}{2}{}{{s_{k+1}^{k},\ldots,s_1^{k}}}{t}=(H^*_{\calw(k)}V^*_{(k)})^{s_{k+1}^{k},\ldots,s_1^{k}}_t.\]
Now $x_{(k)}$ is detected by some permanent cycle $x_{(k+1)}\in \E{(k+1)}{2}{}{}{}$, which is detected by some permanent cycle $x_{(k+2)}\in \E{(k+2)}{2}{}{}{}$, and so on, giving a sequence of elements
\begin{gather*}
x_{(r)}\in \E{(r)}{2}{}{s_{r+1}^{r},\ldots,s_1^{r}}{t}=(H^*_{\calw(r)}V^*_{(r)})^{s_{r+1}^{r},\ldots,s_1^{r}}_t\textup{ for $r\geq k$,}\\
\textup{where }s^{r}_{r+1}+s^{r}_{r}=s^{r-1}_{r}\textup{ and }s^{r}_{i}=s^{r-1}_{i}\textup{  for $1\leq i\leq r-1$ and $r>k$}.
\end{gather*}
We will say that $x_{(k)}$ has iterated filtration at least $(s^{k+1}_{k+2},s^{k+2}_{k+3},s^{k+3}_{k+4},\ldots)$ whenever a sequence of such classes $x_{(r)}$ exists, and partially order the set of possible iterated filtrations lexicographically. Then $x_{(r)}$ only determines $x_{(k)}$ modulo elements of $E_{2,(k)}$ of higher iterated filtration.

Simply because these gradings are always non-negative, it is inevitable that $s_r^r=0$ for some $r\geq k$, so that by Proposition \ref{isomorphism rho k in some dims}, $x_{(r)}=\rho_ry_{(r)}$ for some $y_{(r)}\in F^{\calMh(r+1)}F^{\calMv(r+1)}V_{(r)}$. Moreover, one only needs to examine finitely many sequences of gradings, each of the form
\[(s^r_{r+1},0,s^{r}_{r-1},\ldots, s^{r}_{k+1},s^k_{k},\ldots,s^k_1)\textup{ where }s^k_{k+1}=s^r_{r+1}+s^{r}_{r-1}+s^{r}_{r-2}+\cdots +s^{r}_{k+1}.\]
This, along with Proposition \ref{calculation in internal dimension zero}, shows that $(H^*_{\calw(k)}V^*_{(k)})^{s_{k+1}^{k},\ldots,s_1^{k}}_t$ is finite dimensional for each given value of $t$.



By the commutativity established above, $x_{(k)}\equiv\rho_{k}f_{k+1}\cdots f_{r-1}f_r(y_{(r)})$, modulo higher iterated filtration. As this congruence holds in a group which is finite dimensional for each given $t$, this establishes the surjectivity of $\rho_k$, and that
%
%
%
%{calculation in internal dimension zero},
%\[x_{(r)}\in (E_{2,(r)})^{s_{r+1}^{r},0,s_{r-1}^{r},\ldots,s_1^{r}}_{t}\cong\UEAX(V_{(r-1)}^*)^*,\]
%which is by definition generated by products of elements of 
%\[V_{(r-1)}\cong H^{0}_{\calw(r)}H_0^{\calU(r-1)}V_{(r-1)}\subseteq E_{2,(r)}.\]
%Such products are all contained in the image of $\rho_r$, so that they are all permanent cycles. But even more, we have seen that elements in the image of some $\rho_r$ are permanent cycles that detect elements in the image of $\rho_{r-1}$ whenever $r>1$. That is, \textbf{(filtration confusing)}, there is a sequence of permanent cycles (well defined up to higher filtration each time), each detecting the next. One of these is $x_{(k)}$, if we desire. Thus each $\rho_k$ is surjective, and 
every one of the spectral sequences is degenerate. Thus, we have shown that all of the maps $g_k$ are in fact isomorphisms, or rather that in the following commuting square, for any $k\geq n$,  $g_{k+1}$ is an isomorphism:
\[\xymatrix@R=4mm{
\E{}{0}{H^*_{\calw(k)}V^*_{(k)}}{}{}
&H^*_{\calw(k+1)}H_*^{\calU(k)}V^*_{(k)}\ar@{->}[l]^-{\cong}_-{{g}_{k+1}}
\\
%V_{(k)}\ar@{..>}[r]\ar@{..>}[u]^-{\cong}&
\E{}{0}{F^{\calMh(k+1)}V_{(k+1)}}{}{}\ar@{->>}[u]^-{E_0(\rho_k)}
&F^{\calMh(k+2)}F^{\calMv(k+2)}F^{\calMv(k+1)}V_{(k)}\ar@{->}[l]^-{\cong}_-{{f}_{k+1}}\ar@{->>}[u]_-{\rho_{k+1}}
}\]
For each $k$, $\rho_k$ is injective if and only if $E_0(\rho_k)$ is injective. This holds by repeated application of the snake lemma, using the fact that $\rho_k$ is surjective, and the observation that for any given value of the grading $t$, the group $(F^{\calMh(k+1)}V_{(k+1)})_t$ is finite dimensional, so that the filtrations of both the domain and codomain of $\rho_k$ are eventually zero in each degree $t$. More specifically,
\[\rho_k:(F^{\calMh(k+1)}V_{(k+1)})_t^{s_{k+1},\ldots,s_1}\to (H^*_{\calw(k)}V^*_{(k)})_t^{s_{k+1},\ldots,s_1}\]
is injective if and only if
%\[E_0(\rho_k):E_0^{s'_{k+2}}(F^{\calMh(k+1)}V_{(k+1)})_t^{s'_{k+1},s_k,\ldots,s_1}\to E_0^{s'_{k+2}}(H^*_{\calw(k)}V^*_{(k)})_t^{s'_{k+1},s_k,\ldots,s_1}\]
\[E_0(\rho_k):
\E{}{0}{F^{\calMh(k+1)}V_{(k+1)}}{s'_{k+2},s'_{k+1},s_k,\ldots,s_1}{t}
\to \E{}{0}{H^*_{\calw(k)}V^*_{(k)}}{s'_{k+2},s'_{k+1},s_k,\ldots,s_1}{t}\]
%E_0^{s'_{k+2}}(H^*_{\calw(k)}V^*_{(k)})_t^{s'_{k+1},s_k,\ldots,s_1}\]
is injective whenever $s'_{k+2}+s'_{k+1}=s_{k+1}$. As in the argument for surjectivity, in order to check that all the $\rho_k$ are injective, we now only need to check that every map
\[(F^{\calMh(r+1)}F^{\calMv(r+1)}V_{(r)})^{s_{r+1}^{r},0,s_{r-1}^{r},\ldots,s_1^{r}}_{t} \overset{\rho_r}{\to}(H^*_{\calw(r)}V^*_{(r)})^{s_{r+1}^{r},0,s_{r-1}^{r},\ldots,s_1^{r}}_{t}\]
is injective, which is part of Proposition \ref{isomorphism rho k in some dims}.
%%%%%%However, in this degree, $\rho_r$ is an isomorphism, as it factors as
%%%%%%\begin{alignat*}{2}
%%%%%%(F^{\calMh(r+1)}F^{\calMv(r+1)}V_{(r)})^{s_{r+1}^{r},0,s_{r-1}^{r},\ldots,s_1^{r}}_{t}
%%%%%%&=
%%%%%%(F^{\calMh(r+1)}F^{\calMv(r+1)}V_{(r-1)})^{s_{r+1}^{r},0,s_{r-1}^{r},\ldots,s_1^{r}}_{t}\\
%%%%%%&=(F^{\calMh(r+1)}V_{(r-1)})^{s_{r+1}^{r},0,s_{r-1}^{r},\ldots,s_1^{r}}_{t}\\
%%%%%%&\cong (\UEAX(V_{(r-1)}^*)^*)^{s_{r+1}^{r},0,s_{r-1}^{r},\ldots,s_1^{r}}_{t}.
%%%%%%\end{alignat*}
%%%%%%Here, we have written $V_{(r-1)}$ for the subspace of $V_{(r)}$ in degree $s_r=0$, which we view as an object of $\vect{r+1}{+}$. Then the inclusion $F^{\calMh(r+1)}F^{\calMv(r+1)}V_{(r-1)}\subseteq F^{\calMh(r+1)}F^{\calMv(r+1)}V_{(r)}$ restricts to the identity in degree $s_r=0$, explaining the first equality. The second equality is similar: any non-trivial $\calMv(r+1)$-operation lands outside degree $s_r=0$. The final isomorphism follows from Proposition \ref{basis of free horizontal operations algebra} --- as $V_{(r-1)}$ is concentrated in dimension $s_{n+1}=0$, the proposition ensures that $F^{\calMh(r+1)}V_{(r-1)}$ is a quotient of the polynomial algebra on $V_{(r-1)}$, and indeed, the same quotient as $\UEAX(V_{(r-1)}^*)^*$.
%
%, note that on $V_{(r-1)}$, there are no non-trivial horizontal Steenrod operations except for the top Steenrod operation, the squaring operation. That is, for any $v\in (V_{(r-1)})^{0,0,s_{r-1},\ldots, s_1}_t$, suppose that $\Sqh^{i_\ell}\cdots\Sqh^{i_1}$ is an admissible sequence such that $\Sqh^{i_\ell}\cdots\Sqh^{i_1}v$ is not forced to be zero \textbf{(did we ever use the word \emph{excess}, or give a basis of $\calMh$?)}. Then we must have $i_j=2^j$ for each $j$, and  $\Sqh^{i_\ell}\cdots\Sqh^{i_1}v=v^{2^\ell}$. Note that the axiom $\Sqh^1v=0$ applies iff $s_{r-1}=\cdots =s_1=0$, i.e.\ iff $v$ is not restrictable, and so is an exterior generator in $\UEAX(V_{(r-1)}^*)^*$. That is, $F^{\calMh(r+1)}V_{(r-1)}$ and $\UEAX(V_{(r-1)}^*)^*$ are generated by $V_{(r-1)}$ using the same operations (commutative products), and these operations satisfy the same relations (only that classes in $(V_{(r-1)})_t^{0,\ldots,0}$ are exterior), proving the isomorphism. See also Proposition \ref{basis of free horizontal operations algebra}.
\end{proof}
\SubsectionOrSection{A K\"unneth theorem for $\calw(n)$-cohomology}
This is an opportune moment to prove:
\begin{thm}
\label{Koszul-dual Hilton-Milnor theorem}
Suppose that $X,Y\in\calw(n)$ are  of finite type, with $n\geq0$. Then 
\[H^*_{\calw(n)}(X\times Y)\cong H^*_{\calw(n)}(X)\sqcup H^*_{\calw(n)}(Y),\]
where the coproduct is of non-unital commutative algebras.
\end{thm}
\begin{proof}
This follows from the K\"unneth Theorem (\ref{Prop on cohomology of product of finite lie gems}) adapted to $s\call(k)$, and the observation that $H_*^{\calu(k)}(Z\times Z')\cong H_*^{\calu(k)}Z\times H_*^{\calu(k)}Z'$,
using the techniques of the proof of Theorem \ref{thm on collapsing of most sseqs}.
%Basically, $H_{*}^{\calU(k)}$ is additive, and $H^*_{\calL(k)}$ turns $\oplus$ into $\otimes $. We do induction on iterated filtration. Importantly, all of the spectral sequences involved are multiplicative. 
\end{proof}
Theorems \ref{thm on collapsing of most sseqs} and \ref{Koszul-dual Hilton-Milnor theorem} together imply:
\begin{cor}
\label{yeah it's Wn-coh-algs}
For $n\geq1$, the category $\calMhv(n+1)$ is the category $\HA{\calw(n)}$ of $\calw(n)$-$H^*$-algebras.
\end{cor}
\SubsectionOrSection{A two-dimensional example in $\calw(2)$}
In this section, we suppose that $T\geq1$, and let $X=V^*_{(2)}\in\calw(2)$ be the two-dimensional object of $\calw(2)$ spanned by non-zero classes 
\[v_{0}^{*}\in (V^*_{(2)})^{T}_{0,1}\textup{ and }v_{1}^{*}\in (V^*_{(2)})^{2T+1}_{0,2}\]
such that $v^*_{1}=v^*_0\lambda_{0}=\restn{(v^*_0)}$, and with all other operations trivial. 
%Note that this non-trivial operation may be written $v^*_{1}=\restn{(v^*_0)}$, as $\lambda_0$ is fact a top operation.
\begin{prop}
\label{2d example in w2}
For all $k\geq2$, $V^*_{(k)}$ is two-dimensional, spanned by
\[v_{0}^{*}\in (V^*_{(k)})^{T}_{0,\ldots,0,1}\textup{ and }v_{1}^{*}\in (V^*_{(k)})^{2T+1}_{0,\ldots,0,2},\]
with  $v^*_{1}=\restn{(v^*_0)}$ the only non-trivial operation.
\end{prop}
\begin{proof}
An induction as in the proof of Proposition \ref{iterative calc of the Vk all trivial}, using the fact that at each stage, the only non-trivial $\lambda$-operation is a \emph{top} operation, and thus does not yield a differential in $K_*^{\calU(k)}V_{(k)}$. One also uses propositions \ref{LieBracketsTrivial}, \ref{QkTrivial} and \ref{Q0ZeroByPriddyAlg} to calculate the $\calw(k+1)$-structure of $V_{(k+1)}$ at each stage.
\end{proof}
\begin{prop}
For each $k\geq2$,
\[(H^*_{\calw(k+1)}V^*_{(k+1)})^{s_{k+2},0,s_k,\ldots,s_1}_{t}\cong (H^{s_{k+2}}_{\calL(k)}V^*_{(k)})^{s_k,\ldots,s_1}_t\cong (S(\CommOperad)[v_1^{2}]\sqcup \Lambda(\CommOperad)[v_0])_t^{s_{k+2},s_{k},\ldots,s_1}.\]
%The only non-zero classes in these groups are %
These groups are zero unless $s_k=\cdots =s_2=0$.
\end{prop}
\begin{proof}
One performs this calculation in the Chevalley-Eilenberg-May complex $\dual\UEAX(V^*_{(k)})$, which by Proposition \ref{CEM for trivial lie bracket} is the differential graded algebra $\Ftwo [v_0,v_1]$ with differential
\[d(v_0)=(\dualrestn{v_0})^2=0,\ \ d(v_1)=(\dualrestn{v_1})^2=v_0^2.\qedhere\]
\end{proof}
By a greatly simplified version of the proof of Theorem \ref{thm on collapsing of most sseqs}:
\begin{cor}
\label{statement of result on 2d w2 example}
For each $k\geq2$, $(E_{2,(k+1)})^{s_{k+2},s_{k+1},\ldots,s_1}_{t}$ is zero unless $s_{k+1}=\cdots s_2=0$, so that the spectral sequence $E_{2,(k+1)}\implies H^*_{\calw(k)}V^*_{(k)}$ collapses, and in particular,
\[H^*_{\calw(2)}V^*_{(2)}\cong S(\CommOperad) [v_1^{2}]\sqcup \Lambda(\CommOperad)[v_0].\]
\end{cor}

%\SubsectionOrSection{An infinite-dimensional example in $\calw(1)$}
%In this section, we suppose that $S,T\geq1$, and let $X=V^*_{(1)}\in\calw(1)$ be the infinite dimensional object of $\calw(1)$ spanned by non-zero classes
%\[v_{j}^{*}\in (V^*_{(1)})^{2^{j}(T+1)-1}_{S+j}\textup{ for $j\geq 0$},\] such that $v_{j+1}^*=v^*_j\lambda_{1}$ for $j\geq 0$, and all other operations are trivial. %(We must have $S\geq1$ for this definition to make sense: $v_0\lambda_1$ must be defined.)
%\begin{prop}\label{calc of koszul complex in inf dim example}
%The Koszul complex $K_*^{\calU(1)}V_{(1)}^*$ has basis
%\[\left\{\SMEHq_\textup{v}(J,v_j^*)\ \middle|\ \textup{$j\geq0$, $J$ is $\Sq$-admissible, $\minDimSq(J)\leq S+j$ and $1\notin J$}\right\} \]
%%\ \genfrac{}{}{0pt}{}{\textup{$J$ $\Sq$-admissible with }\minDimSq(J)\leq S+k,}{\textup{$J$ does not contain 1}}\right\},\]
%and all differentials zero except for:
%\[\SMEHq_\textup{v}((i_\ell,\ldots,i_2,2),v^*_{j})\longmapsto \SMEHq_\textup{v}((i_\ell,\ldots,i_2),v^*_{j+1}),\]
%so that $V^*_{(2)}:=H_*^{\calU(1)}V_{(1)}^{*}$ is the following subquotient of $K_*^{\calU(1)}V_{(1)}^*$:
%\[\frac{{\Ftwo\left\{\SMEHq_\textup{v}(J,v_j^*)\ \middle|\ \makebox[\widthof{$\textup{$j\geq0$}$}][c]{$\textup{$j\geq0$}$}\textup{, $J$ is $\Sq$-admissible, $\minDimSq(J)\leq S+j$ and}\makebox[\widthof{$\textup{ $1,2,3\notin J$}$}][c]{$\textup{ $1,2\notin J$}$}\right\}}}{{\Ftwo\left\{\SMEHq_\textup{v}(J,v_j^*)\ \middle|\ \makebox[\widthof{$\textup{$j\geq0$}$}][c]{$\textup{$j\geq1$}$}\textup{, $J$ is $\Sq$-admissible, $\minDimSq(J)\leq S+j$ and}\makebox[\widthof{$\textup{ $1,2,3\notin J$}$}][c]{$\textup{ $1,2,3\notin J$}$}\right\}}}.\]
%Equivalently, $V_{(2)}$ is the subquotient of $F^{\calMv(2)}V_{(1)}$ in which we restrict to the sub-$\calMv(2)$-object generated by the elements%\textbf{{(Can some of these be 0? yes if $S=1$, at least. How does this effect the calculation here?)}}
%\[\{v_0,\Sqv^2v_{1},\Sqv^3v_{1},\Sqv^2v_{2},\Sqv^3v_{2},\Sqv^2v_{3},\Sqv^3v_{3},\ldots\}\]
%and in which we set $\Sqv^2 v_{j}$ to zero for all $j\geq0$.  All of the $\Sqv^3v_{j}$ are non-zero, except when $j=S=1$. If $S\geq2$, then $V_{(2)}^{*}$ is trivial as an object of $\calw(2)$. %, and
%%\[V_{(k)}=F^{\calMv(k)}F^{\calMv(k-1)}\cdots F^{\calMv(3)}V_{(2)}.\]
%If $S=1$, then $V^*_{(2)}$ supports a single non-zero operation:
%\[(V_{(2)}^{*})_{0,1}^{T}\ni v_0^*\overset{\lambda_0}{\longmapsto}v_1^*\in (V_{(2)}^{*})_{0,2}^{2T+1}.\]
%% $\SMEHqv(\emptyset,v_0^*)\lambda_0=\SMEHqv(\emptyset,v_1^*)$. \textbf{fill in details on higher $V_{(k)}$}
%\end{prop}
%\begin{proof}
%The basis for the Koszul complex is just a reading of [above], but we must think a little about the differentials. As $\lambda_1$ is the only non-zero operation:
%\[d(\Smehqv(J,v^*_j))=\sum_{ \substack{\produces{(k_{\ell},\ldots,k_2,2)}{J}{\Sq}\\(k_{\ell},\ldots,k_2){\,\Sq\textup{-admis.}}}}\!\!\!\!\!\!\!\!\!\!\!\! \Smehqv((k_{\ell},\ldots,k_2),v_{j+1}^*).\]
%Consider a sequence $(k_\ell,\ldots,k_2,2)$ corresponding to a summand of this formula. Supposing that $\ell\geq2$ and $\Sqv^{k_2}\Sqv^2$ is not $\Sq$-admissible, it follows that $k_2$ is either 3 or 2, so that $\Sqv^{k_2}\Sqv^2$ is either zero or $\Sqv^{3}\Sqv^1$. As $J$ does not contain 1, and the two-sided ideal in $\LieSteen$ generated by $\Sqh^1$ is spanned by those admissible sequences ending in $\Sqh^1$, it cannot happen that $\produces{(k_{\ell},\ldots,k_2,2)}{J}{\Sq}$. Thus, the only summand appearing is that in which $(k_{\ell},\ldots,k_2,2)=J$, confirming our description of the differential. Taking the homology of this differential provides the formula for $V_{(2)}^*$, and dualizing provides that for $V_{(2)}$.
%
%In order to determine $V_{(2)}^*$ as an object of $\calw(2)$, note first that \ref{LieBracketsTrivial} and \ref{QkTrivial} show that all operations are zero except perhaps for $\lambda_0$. %Now $(\Sqmehv(J,w^*))\lambda_0$ is evidently zero, 
%Consider the operation $\lambda_0$ applied to a cycle of the form $\Smehqv(J,v_{j}^*)\in K_{*}^{\calU(1)}V^*_{(1)}$ with $J\neq\emptyset$ (so that $1,2\notin J$). As $J$ ends in an integer no less than 3, and as $\lambda_1$ is the only non-zero operation in $V_{(1)}^*$, the second part of Proposition \ref{Q0ZeroByPriddyAlg} implies that $\Smehqv(J,v_{j}^*)\lambda_0=0$.
%
%In the case $J=\emptyset$, Proposition \ref{Q0ZeroByPriddyAlg} states that $(\Smehqv(\emptyset,v_{j}^*))\lambda_0\in V_{(2)}^*$ is represented by $(\Smehqv(\emptyset,v_{j}^*\lambda_{S+j}))$, which is zero unless $j=0$ and $S=1$. Thus the only non-zero operation on $V_{(2)}^*$ is $v_0^*\lambda_0=v_1^*$ in the case $S=1$.
%\end{proof}
%
%\begin{prop}
%The spectral sequence $H^*_{\calw(2)}V_{(2)}^*\implies H^*_{\calw(1)}V_{(1)}^*$ collapses.
%\end{prop}
%\begin{proof}
%Initially, we assume that $S\geq2$. The first point is to observe that the generators $v_0$ and $\Sqv^3v_j$ ($j\geq1$) of $V_{(2)}$ are all permanent cycles in
%$(H^{*}_{\calw(2)}V_{(2)})^{0**}_{*}$. For $v_0\in (E_2)^{00S}_{T}$, this is obvious. It is less obvious for $\Sqv^3v_j$ ($j\geq1$), whose only opportunity to support a differential is
%\[\Sqv^3v_j\in (E_2)^{0,1,2+S+j}_{2^{j+1}(T+1)-1}\overset{d_2}{\to}(E_2)^{2,0,2+S+j}_{2^{j+1}(T+1)-1}.\]
%Fortunately, this target group is zero, due to the constraint that $s_2=0$. To see this, note that this group is spanned by the products of three classes in $(E_{2})^{00*}_*$, namely:
%\[v_{j_1}v_{j_2}v_{j_3}\in (E_{2})^{2,0,3S+j_1+j_2+j_3}_{(2^{j_1}+2^{j_2}+2^{j_3})(T+1)-1},\]
%and if this target group is non-zero, these indices must coincide.
%In order that $2^{j+1}$ equals $2^{j_{1}}+2^{j_{2}}+2^{j_{3}}$ it must happen that $j_1,j_2,j_3$ equal $j,j-1,j-1$ (in some order), but then $2+S+j=3S+j_1+j_2+j_3$ implies that $S+j=2$. This is impossible, as $S\geq2$ and $j\geq1$.
%
%Next, we can derive that $\Sqv^Jv_j$ is a permanent cycle for all $J,j$ such that $J$ is $\Sq$-admissible, with final entry 3 when $j>0$. For this, we will use  Proposition \ref{edgehomproposition}, that there is a commuting diagram:%as $V_{(2)}^*$ is a trivial object, there is an isomorphism $V_{(2)}\cong H^{0}_{\calw(2)}V_{(2)}^*$. In light of this isomorphism, Proposition \ref{edgehomproposition} shows that the edge homomorphism itself is compatible with the action of the vertical Steenrod operations, in the sense that
%\[\xymatrix@R=4mm{
%(H^*_{\calw(1)}V_{(1)}^*)^{s_{2},s_1}_t\ar[r]^-{\Sqv^i}
%\ar[d]^-{\textup{edge hom}}
%&%r1c1
%(H^*_{\calw(1)}V_{(1)}^*)^{s_{2}+1,s_1+i-1}_{2t+1}\ar[d]^-{\textup{edge hom}}
%\\%r1c2
%(E_{2})^{0,s_{2},s_1}_t\ar@{=}[d]%\ar[r]^-{\Sqv^i}
%&%r1c1
%(E_2)^{0,s_{2}+1,s_1+i-1}_{2t+1}\ar@{=}[d]\\
%(V_{(2)})^{s_{2},s_1}_t\ar[r]^-{\Sqv^i}
%&%r1c1
%(V_{(2)})^{s_{2}+1,s_1+i-1}_{2t+1}
%}\]
%Now as we have shown that the classes $v_0$ and $\Sqv^{3}v_j$  ($j\geq1$) are all permanent cycles, they are in the image of the edge homomorphism. Then this diagram shows that all of $V_{(2)}$ is in the image of the edge homomorphism, so that every element  of $V_{(2)}$ is a permanent cycle. Finally, as $V_{(2)}$ is a trivial object of $\calw(2)$, $E_{2}$ is (freely) generated by $V_{(2)}$ under the $\calmh(3)$- and $\calmv(3)$-operations, by Theorem \ref{thm on collapsing of most sseqs}. As we understand how these operations interact with the differential, we have shown that the spectral sequence collapses.
%
%When $S=1$ we must modify this proof a little. Indeed, in this case, $E_{2}=H^*_{\calw(2)}V_{(2)}^*$ is no longer generated by $V_{(2)}$ under the $\calmh(3)$- and $\calmv(3)$-operations. Rather, propositions \ref{2d example in w2} and \ref{Koszul-dual Hilton-Milnor theorem}, Theorem \ref{thm on collapsing of most sseqs} and Corollary \ref{statement of result on 2d w2 example} show that 
%%\[H^*_{\calw(2)}V_{(2)}^*\cong H^*_{\calw(2)}(\Ftwo \{v_0^*,v_1^*\}\oplus (V'_{(2)})^*)\cong \Ftwo [v_1^{2}]\otimes E(v_0)\otimes \calmh(3)\calmv(3)V'_{(2)} ,\]
%\begin{alignat*}{2}
%H^*_{\calw(2)}V_{(2)}^*&=H^*_{\calw(2)}(\Ftwo \{v_0^*,v_1^*\}\oplus (V'_{(2)})^*)\\
%&\cong \Ftwo [v_1^{2}]\otimes E(v_0)\otimes \calmh(3)\calmv(3)V'_{(2)}
%\end{alignat*}
%where $(V'_{(2)})^*$ is the trivial object of $\calw(2)$ dual to $V'_{(2)}$, the subquotient of $F^{\calMv(2)}V_{(1)}$ in which we restrict to the sub-$\calMv(2)$-object generated by the elements
%\[\{\Sqv^2v_{2},\Sqv^3v_{2},\Sqv^2v_{3},\Sqv^3v_{3},\ldots\}\]
%an in which we set $\Sqv^2 v_{j}$ to zero for all $j\geq2$. This object is chosen so that $V_{(2)}^*$ decomposes as a direct sum $\Ftwo \{v_0^*,v_1^*\}\oplus (V'_{(2)})^*$ (of objects of $\calw(2)$), in light of the description given in  Proposition \ref{calc of koszul complex in inf dim example}.
%
%Now we may proceed as before, except that in the case $S=1$, the class $v_1^{2}$ becomes relevant, as $v_1$ is no longer present in $H^*_{\calw(2)}V_{(2)}^*$, while $\Sqv^{3}v_1$ becomes zero. That $v_1^2\in(H^*_{\calw(2)}V_{(2)}^*)^{1,0,4}_{4T-3}$ cannot support differentials is obvious, while for $j\geq2$, the same degree argument as before shows that $\Sqv^3v_j$ is a permanent cycle. As before, an argument with the edge homomorphism shows that every element of $V'_{(2)}$ is a permanent cycle, and we have found enough permanent cycles to generate $E_2$ under  the $\calmh(3)$- and $\calmv(3)$-operations.
%
% \textbf{Obviously we need better notation for the nested vertical constructions.}
%\end{proof}
%
%
%






\SubsectionOrSection{An infinite-dimensional example in $\calw(1)$}
\label{sec: infinite-dimensional example}
In this section, we suppose that $S,T\geq1$, and let $X=V^*_{(1)}\in\calw(1)$ be the infinite dimensional object of $\calw(1)$ spanned by non-zero classes
\[v_{j}^{*}\in (V^*_{(1)})^{2^{j}(T+1)-1}_{S+j}\textup{ for $j\geq 0$},\] such that $v_{j+1}^*=v^*_j\lambda_{1}$ for $j\geq 0$, and all other operations are trivial. %(We must have $S\geq1$ for this definition to make sense: $v_0\lambda_1$ must be defined.)
\begin{prop}
\label{calc of koszul complex in inf dim example}
The Koszul complex $K_*^{\calU(1)}V^*_{(1)}$ has basis
\[\left\{\Sqvstar{J}(v_j^*)\ \middle|\ \textup{$j\geq0$, $J$ is $\Sq$-admissible, $\minDimSq(J)\leq S+j$ and $1\notin J$}\right\} \]
%\ \genfrac{}{}{0pt}{}{\textup{$J$ $\Sq$-admissible with }\minDimSq(J)\leq S+k,}{\textup{$J$ does not contain 1}}\right\},\]
and all differentials zero except for:
\[\Sqvstar{(i_\ell,\ldots,i_2,2)}(v^*_{j})\longmapsto \Sqvstar{(i_\ell,\ldots,i_2)}(v^*_{j+1}).\]
\end{prop}
\begin{proof}
The basis for the Koszul complex is just a reading of [above], but we must think a little about the differentials. As $\lambda_1$ is the only non-zero operation:
\[d(\Sqvstar{J}(v^*_j))=\sum_{ \substack{\produces{(k_{\ell},\ldots,k_2,2)}{J}{\Sq}\\(k_{\ell},\ldots,k_2){\,\Sq\textup{-admis.}}}}\!\!\!\!\!\!\!\!\!\!\!\! \Sqvstar{(k_{\ell},\ldots,k_2)}(v_{j+1}^*).\]
Consider a sequence $(k_\ell,\ldots,k_2,2)$ corresponding to a summand of this formula. Supposing that $\ell\geq2$ and $\Sqv^{k_2}\Sqv^2$ is not $\Sq$-admissible, it follows that $k_2$ is either 3 or 2, so that $\Sqv^{k_2}\Sqv^2$ is either zero or $\Sqv^{3}\Sqv^1$. As $J$ does not contain 1, and the two-sided ideal in $\LieSteen$ generated by $\Sqh^1$ is spanned by those admissible sequences ending in $\Sqh^1$, it cannot happen that $\produces{(k_{\ell},\ldots,k_2,2)}{J}{\Sq}$. Thus, the only summand appearing is that in which $(k_{\ell},\ldots,k_2,2)=J$, confirming our description of the differential.
\end{proof}
\begin{prop}
\label{Sg1 calc of V2}
When $S\geq2$, $V^*_{(2)}:=H_*^{\calU(1)}V^*_{(1)}$ is the subquotient
\[\frac{{\Ftwo\left\{\Sqvstar{J}(v_j^*)\ \middle|\ \makebox[\widthof{$\textup{$j\geq0$}$}][c]{$\textup{$j\geq0$}$}\textup{, $J$ is $\Sq$-admissible, $\minDimSq(J)\leq S+j$ and}\makebox[\widthof{$\textup{ $1,2,3\notin J$}$}][c]{$\textup{ $1,2\notin J$}$}\right\}}}{{\Ftwo\left\{\Sqvstar{J}(v_j^*)\ \middle|\ \makebox[\widthof{$\textup{$j\geq0$}$}][c]{$\textup{$j\geq1$}$}\textup{, $J$ is $\Sq$-admissible, $\minDimSq(J)\leq S+j$ and}\makebox[\widthof{$\textup{ $1,2,3\notin J$}$}][c]{$\textup{ $1,2,3\notin J$}$}\right\}}}\]
 of $K_*^{\calU(1)}V^*_{(1)}$. Equivalently, $V_{(2)}$ is the subquotient of $F^{\calMv(2)}V_{(1)}$ in which we restrict to the sub-$\calMv(2)$-object generated by the elements%\textbf{{(Can some of these be 0? yes if $S=1$, at least. How does this effect the calculation here?)}}
\[\{v_0,\Sqv^2v_{1},\Sqv^3v_{1},\Sqv^2v_{2},\Sqv^3v_{2},\Sqv^2v_{3},\Sqv^3v_{3},\ldots\}\]
and in which we set $\Sqv^2 v_{j}$ to zero for all $j\geq0$.   As an object of $\calw(2)$, $V^*_{(2)}$ is trivial.
\end{prop}
\begin{prop}
\label{Se1 calc of V2}
When $S=1$, $V^*_{(2)}:=H_*^{\calU(1)}V^*_{(1)}$ is the subquotient
\[\frac{{\Ftwo\left\{\Sqvstar{J}(v_j^*)\ \middle|\ \makebox[\widthof{$\textup{$j\geq2$}$}][c]{$\textup{$j\geq0$}$}\textup{, $J$ is $\Sq$-admissible, $\minDimSq(J)\leq S+j$ and}\makebox[\widthof{$\textup{ $1,2,3\notin J$}$}][c]{$\textup{ $1,2\notin J$}$}\right\}}}{{\Ftwo\left\{\Sqvstar{J}(v_j^*)\ \middle|\ \makebox[\widthof{$\textup{$j\geq2$}$}][c]{$\textup{$j\geq2$}$}\textup{, $J$ is $\Sq$-admissible, $\minDimSq(J)\leq S+j$ and}\makebox[\widthof{$\textup{ $1,2,3\notin J$}$}][c]{$\textup{ $1,2,3\notin J$}$}\right\}}}\]
 of $K_*^{\calU(1)}V^*_{(1)}$. Equivalently, $V_{(2)}$ is the subquotient of $F^{\calMv(2)}V_{(1)}$ in which we restrict to the sub-$\calMv(2)$-object generated by the elements%\textbf{{(Can some of these be 0? yes if $S=1$, at least. How does this effect the calculation here?)}}
\[\{v_0,v_1,\Sqv^2v_{2},\Sqv^3v_{2},\Sqv^2v_{3},\Sqv^3v_{3},\ldots\}\]
and in which we set $\Sqv^2 v_{j}$ to zero for all $j\geq1$.  As an object of $\calw(2)$, $V^*_{(2)}$ admits a single non-zero operation, $\lambda_0:v_0^*\longmapsto v_1^*$, and so decomposes as the direct sum of
$\Ftwo \{v_0^*,v_1^*\}$ with a trivial object $\dual(V'_{(2)})$, dual to $V'_{(2)}$, the subquotient of $F^{\calMv(2)}V_{(1)}$ in which we restrict to the sub-$\calMv(2)$-object generated by
$\{\Sqv^2v_{2},\Sqv^3v_{2},\Sqv^2v_{3},\Sqv^3v_{3},\ldots\}$ and set $\Sqv^2 v_{j}$ to zero for all $j\geq2$.
\end{prop}
%
%\begin{prop}
%If $S\geq2$, then $V_{(2)}^{*}$ is trivial as an object of $\calw(2)$. %, and
%%\[V_{(k)}=F^{\calMv(k)}F^{\calMv(k-1)}\cdots F^{\calMv(3)}V_{(2)}.\]
%If $S=1$, then $V^*_{(2)}$ supports a single non-zero operation:
%\[(V_{(2)}^{*})_{0,1}^{T}\ni v_0^*\overset{\lambda_0}{\longmapsto}v_1^*\in (V_{(2)}^{*})_{0,2}^{2T+1}.\]
%% $\Smehqv(\emptyset,v_0^*)\lambda_0=\Smehqv(\emptyset,v_1^*)$. \textbf{fill in details on higher $V_{(k)}$}
%\end{prop}
\begin{proof}[Proof of propositions \ref{Sg1 calc of V2} and \ref{Se1 calc of V2}]
For any $S\geq1$, taking the homology of this differential provides the formula for $V^*_{(2)}$, and dualizing provides that for $V_{(2)}$.
In order to determine $V^*_{(2)}$ as an object of $\calw(2)$, note first that \ref{LieBracketsTrivial} and \ref{QkTrivial} show that all operations are zero except perhaps for $\lambda_0$. %Now $(\Sqmehv(J,w^*))\lambda_0$ is evidently zero, 
Consider the operation $\lambda_0$ applied to a cycle of the form $\Sqvstar{J}(v_{j}^*)\in K_{*}^{\calU(1)}V^*_{(1)}$ with $J\neq\emptyset$ (so that $1,2\notin J$). As $J$ ends in an integer no less than 3, and as $\lambda_1$ is the only non-zero operation in $V^*_{(1)}$, the second part of Proposition \ref{Q0ZeroByPriddyAlg} implies that $\Sqvstar{J}(v_{j}^*)\lambda_0=0$.

In the case $J=\emptyset$, Proposition \ref{Q0ZeroByPriddyAlg} states that $(\Sqvstar{\emptyset}(v_{j}^*))\lambda_0\in V^*_{(2)}$ is represented by $(\Sqvstar{\emptyset}(v_{j}^*\lambda_{S+j}))$, which is zero unless $j=0$ and $S=1$. Thus the only non-zero operation on $V^*_{(2)}$ is $v_0^*\lambda_0=v_1^*$ in the case $S=1$.
\end{proof}
\begin{thm}
\label{W2 to W1 collapse}
The spectral sequence $H^*_{\calw(2)}V^*_{(2)}\implies H^*_{\calw(1)}V^*_{(1)}$ collapses, with
\[\E{(2)}{2}{}{}{}=H^*_{\calw(2)}V^*_{(2)}\cong\begin{cases}
F^{\calmh(3)}F^{\calmv(3)}V_{(2)},&\textup{if }S\geq2;\\
F^{\calmh(3)}F^{\calmv(3)}V'_{(2)}\sqcup S(\CommOperad) [v_1^{2}]\sqcup \Lambda(\CommOperad)[v_0],&\textup{if }S=1.
%\\,&\textup{if }
\end{cases}
\]
\end{thm}
\begin{proof}
The calculations of $\E{(2)}{2}{}{}{}$ follow from  theorems \ref{thm on collapsing of most sseqs} and \ref{Koszul-dual Hilton-Milnor theorem}, propositions \ref{2d example in w2}, \ref{Sg1 calc of V2} and \ref{Se1 calc of V2}  and Corollary \ref{statement of result on 2d w2 example}. What remains is to prove the collapsing result in each case.

Suppose that $S\geq2$. The first point is to observe that the generators $v_0$ and $\Sqv^3v_j$ ($j\geq1$) of $V_{(2)}$ under $\calMv(2)$-operations are all permanent cycles in
$(H^*_{\calw(2)}V_{(2)})^{0**}_{*}$. For $v_0\in \E{(2)}{2}{}{00S}{T}$, this is obvious. It is less obvious for $\Sqv^3v_j$ ($j\geq1$), which has only one opportunity to support a differential:
\[\Sqv^3v_j\in \E{(2)}{2}{}{0,1,2+S+j}{2^{j+1}(T+1)-1} \overset{d_2}{\to}\E{(2)}{2}{}{2,0,2+S+j}{2^{j+1}(T+1)-1}.\]
Fortunately, this target group is zero, due to the constraint that $s_2=0$. To see this, note that this group is spanned by three-fold products of classes in $\E{(2)}{2}{}{00*}{*}$, namely:
\[v_{j_1}v_{j_2}v_{j_3}\in \E{(2)}{2}{}{2,0,3S+j_1+j_2+j_3}{(2^{j_1}+2^{j_2}+2^{j_3})(T+1)-1},\]
and if this target group is non-zero, these indices must coincide.
In order that $2^{j+1}$ equals $2^{j_{1}}+2^{j_{2}}+2^{j_{3}}$ it must happen that $j_1,j_2,j_3$ equal $j,j-1,j-1$ (in some order), but then $2+S+j=3S+j_1+j_2+j_3$ implies that $S+j=2$. This is impossible, as $S\geq2$ and $j\geq1$.

Next, we can derive that $\Sqv^Jv_j$ is a permanent cycle for all $\Sq$-admissible $J$ and $j\geq0$ such that $J$ has final entry 3 when $j>0$. For this, we will use  Proposition \ref{edgehomproposition}, that there is a commuting diagram:%as $V_{(2)}^*$ is a trivial object, there is an isomorphism $V_{(2)}\cong H^{0}_{\calw(2)}V_{(2)}^*$. In light of this isomorphism, Proposition \ref{edgehomproposition} shows that the edge homomorphism itself is compatible with the action of the vertical Steenrod operations, in the sense that
\[\xymatrix@R=4mm{
(H^*_{\calw(1)}V^*_{(1)})^{s_{2},s_1}_t\ar[r]^-{\Sqv^i}
\ar[d]^-{\textup{edge hom}}
&%r1c1
(H^*_{\calw(1)}V^*_{(1)})^{s_{2}+1,s_1+i-1}_{2t+1}\ar[d]^-{\textup{edge hom}}
\\%r1c2
\E{(2)}{2}{}{0,s_{2},s_1}{t}\ar@{=}[d]%\ar[r]^-{\Sqv^i}
&%r1c1
\E{(2)}{2}{}{0,s_{2}+1,s_1+i-1}{2t+1}\ar@{=}[d]\\
(V_{(2)})^{s_{2},s_1}_t\ar[r]^-{\Sqv^i}
&%r1c1
(V_{(2)})^{s_{2}+1,s_1+i-1}_{2t+1}
}\]
As we have shown that the classes $v_0$ and $\Sqv^{3}v_j$  ($j\geq1$) are all permanent cycles, they are in the image of the edge homomorphism. Then this diagram shows that all of $V_{(2)}$ is in the image of the edge homomorphism, so that every element  of $V_{(2)}$ is a permanent cycle. Finally, as $E_{2}$ is (freely) generated by $V_{(2)}$ under the $\calMhv(3)$-operations, and we understand how these operations interact with the differential, this shows that the spectral sequence collapses.

%%When $S=1$ we must modify this proof a little, as $E_{2}=H^*_{\calw(2)}V_{(2)}^*$ is no longer generated by $V_{(2)}$ under the $\calmh(3)$- and $\calmv(3)$-operations. Rather, propositions \ref{2d example in w2} and \ref{Koszul-dual Hilton-Milnor theorem}, Theorem \ref{thm on collapsing of most sseqs} and corollary \ref{statement of result on 2d w2 example} show that 
%%%\[H^*_{\calw(2)}V_{(2)}^*\cong H^*_{\calw(2)}(\Ftwo \{v_0^*,v_1^*\}\oplus (V'_{(2)})^*)\cong \Ftwo [v_1^{2}]\otimes E(v_0)\otimes F^{\calmh(3)}F^{\calmv(3)}V'_{(2)} ,\]
%%\begin{alignat*}{2}
%%H^*_{\calw(2)}V_{(2)}^*&=H^*_{\calw(2)}(\Ftwo \{v_0^*,v_1^*\}\oplus (V'_{(2)})^*)\\
%%&\cong \Ftwo [v_1^{2}]\otimes E(v_0)\otimes F^{\calmh(3)}F^{\calmv(3)}V'_{(2)}.
%%\end{alignat*}
%where $(V'_{(2)})^*$ is the trivial object of $\calw(2)$ dual to $V'_{(2)}$, the subquotient of $F^{\calMv(2)}V_{(1)}$ in which we restrict to the sub-$\calMv(2)$-object generated by the elements
%\[\{\Sqv^2v_{2},\Sqv^3v_{2},\Sqv^2v_{3},\Sqv^3v_{3},\ldots\}\]
%an in which we set $\Sqv^2 v_{j}$ to zero for all $j\geq2$. This object is chosen so that $V_{(2)}^*$ decomposes as a direct sum $\Ftwo \{v_0^*,v_1^*\}\oplus (V'_{(2)})^*$ (of objects of $\calw(2)$), in light of the description given in  Proposition \ref{calc of koszul complex in inf dim example}.
Suppose instead that $S=1$. Then rather that having generators $v_0$ and $\Sqv^3v_j$ ($j\geq1$) as before, $E_2$ has generators $v_0$, $v_1^{2}$ and $\Sqv^3v_j$ ($j\geq2$). Note that $\Sqv^{3}v_1=0$ when $S=1$.
% by the class $v_1^{2}$  (and we no longer need to consider $\Sqv^{3}v_1$, as it is zero when $S=1$). 
That $v_1^2\in\E{(2)}{2}{}{1,0,4}{4T-3}$ cannot support differentials is obvious, while for $j\geq2$, the same degree argument as before shows that $\Sqv^3v_j$ is also a permanent cycle. The same argument with the edge homomorphism shows that every element of $V'_{(2)}$ is a permanent cycle, so $E_2$ is again generated by permanent cycles under the $\calMhv(3)$-operations, completing the proof.
% \TODOCOMMENT[inline]{Obviously we need better notation for the nested vertical constructions, and to explain what we really mean by the symbols --- this object is isomorphic to a subobject of a different object. It doesn't mean that the object actually has the generators you might hope.}
\end{proof}
\begin{cor}
\label{the actual statement of what H1V1 is}
If $S\geq2$, then $H^*_{\calW(1)}V^*_{(1)}$ is isomorphic, as a vector space in $\vect{2}{+}$, to the subquotient of $F^{\calMh(2)}F^{\calMv(2)}V_{(1)}$ generated by the elements
\[\{v_0,\Sqv^2v_{1},\Sqv^3v_{1},\Sqv^2v_{2},\Sqv^3v_{2},\Sqv^2v_{3},\Sqv^3v_{3},\ldots\}\]
and subject to relations generated by $\Sqv^2 v_{j}=0$ for all $j\geq0$. Under $\calMhv(2)$-operations, $H^*_{\calw(1)}V^*_{(1)}$ is generated by $v_0$, $\Sqv^3 v_1$, $\Sqv^3 v_2$, etc.

If $S=1$, then $H^*_{\calw(1)}V^*_{(1)}\in \vect{2}{+}$ is isomorphic, as a vector space in $\vect{2}{+}$, to the commutative algebra coproduct
\[\mathrm{subquo}\sqcup S(\CommOperad) [v_1^{2}]\sqcup \Lambda(\CommOperad)[v_0],\] where $v_1^2\in(H^*_{\calw(1)}V^*_{(1)})^{1,2^2}_{2^2(T+1)-1}$, and $\mathrm{subquo}$ is the subquotient of $F^{\calMh(2)}F^{\calMv(2)}V_{(1)}$ generated by the elements
\[\{\Sqv^2v_{2},\Sqv^3v_{2},\Sqv^2v_{3},\Sqv^3v_{3},\ldots\}\]
and subject to relations generated by 
%$\Sqv^2v_1^{2}=0$ and 
$\Sqv^2 v_{j}=0$ for all $j\geq2$. Under this isomorphism, $H^*_{\calw(1)}V^*_{(1)}$ is generated by $v_0$, $ v_1^2$, $\Sqv^3 v_2$, $\Sqv^3 v_3$, et cetera, under the $\calMhv(2)$-operations.
\end{cor}
\begin{proof}
Suppose first that $S\geq2$. Consider the elements
\[v_0\in (V_{(2)})^{0,S}_{T},\ \ \ \Sqv^j v_0\in (V_{(2)})^{1,S+j-1}_{2(T+1)-1}\  \textup{($j\geq2$)},\ \ \textup{and}\ \ \ \Sqv^3v_i\in (V_{(2)})^{1,S+i+2}_{2^{i+1}(T+1)-1}\ \textup{($i\geq1$)}.\]
These elements span $(V_{(2)})^{0,*}_*$ and $(V_{(2)})^{1,*}_*$, and can all be distinguished by their internal degrees, so the restrictions 
\[(H^*_{\calw(1)}V^*_{(1)})^{0,*}_*\epi \E{(2)}{2}{}{0,0,*}{*}=(V_{(2)})^{0,*}_*,\ \ (H^*_{\calw(1)}V^*_{(1)})^{1,*}_*\epi \E{(2)}{2}{}{0,1,*}{*}=(V_{(2)})^{1,*}_*\]
 of the edge composite (c.f.\ Proposition \ref{edgehomproposition}) are isomorphisms. We write
\[h:(V_{(2)})^{0,*}_*\oplus (V_{(2)})^{1,*}_*\to H^*_{\calw(1)}V^*_{(1)}\]
for the injection obtained by adding their inverse maps. Use the basis of $V_{(2)}$ arising from propositions \ref{Sg1 calc of V2} and \ref{basis of element of M(n+1) in deg 0} to extend $h$ to a vector space map $H:V_{(2)}\to H^*_{\calw(1)}V^*_{(1)}$ by the rule $H(\Sqv^jx)=\Sqv^j H(x)$. Although $H$ is not a map in $\calMv(2)$, it does induce the vector space isomorphism required for the proposition.

Suppose instead that $S=1$. The same argument produces a map $\mathrm{subquo}\to H^*_{\calw(1)}V^*_{(1)}$. The difference is that we must find candidates for $v_1^2$ and $v_0$ in $H^*_{\calw(1)}V^*_{(1)}$. We send $v_0$ to the unique non-zero element of $(H^*_{\calw(1)}V^*_{(1)})^{0,1}_{T}$ and $v_1^{2}$ to the unique non-zero element of $(H^*_{\calw(1)}V^*_{(1)})^{1,4}_{4(T+1)-1}$.
\end{proof}






\SubsectionOrSection{The Bousfield-Kan $E_2$-page for a sphere}
\label{Calculations of HW0}
Let $X=V^*_{(0)}\in\calw(0)$ be a one dimensional object of $\calw(0)$, dual to a one-dimensional vector space $V_{(0)}\in\vect{0}{+}$, with non-zero element $\imath\in(V_{(0)})_T$. Write $\imath^*\in X^T$ for the non-zero element of $X$.

As every $\calw(0)$-operation changes degrees, $X$ is necessarily trivial. Moreover, it is quadratically graded, by setting $\imath^*\in \quadgrad{1}X^T$. By Proposition \ref{prop: cfsseq is quad graded}, the first \CFSS\ will admit a quadratic grading.


Recall the function $\TOP_T:\aDT\to \aDT$ of \S\ref{section on structure on homology of koszul cx}. In view of the strict inequality derived during the proof of Lemma \ref{lemma on the func TOPt}, it need not be true that $I=\TOP_T^{i-1-j}(i_{j+1},\ldots,i_1)$ whenever  $I=(i_\ell,\ldots,i_1)\in \aDT$ satisfies $(i_{j+1},\ldots,i_1)=\TOP_T(i_{j},\ldots,i_1)$. Nevertheless, we may use $\TOP_T$ to decompose $\aDT$.
Define:
\[\aDTirr:= \aDT\setminus\im(\TOP_T:\aDT\to\aDT),\]
the set of sequences in $\aDT$ not in the image of $\TOP_T$, so that we may decompose $\aDT$ as the disjoint union
\[\aDT=\bigsqcup_{\smash{I\in \aDTirr}}\left\{I,\TOP_T I,\TOP_T^2I,\ldots\right\}.\]







\begin{prop}
\label{calc of V1 from W0 sphere}
$V^*_{(1)}:=H_*^{\calU(0)}V^*_{(0)}$ has basis $\{\imath^*\}\sqcup\left\{\deltavstar_I\imath^*\ \middle|\ I\in\aDT\right\}$ and all $\calw(1)$-operations trivial except for $\lambda_1$, which is defined (only when $\ell(I)\geq1$) by
\[\deltavstar_I\imath^*\overset{\smash{\lambda_1}}{\longmapsto }\deltavstar_{\TOP_T I}\imath^*.\]
Thus, as an object of $\calw(1)$, $V^*_{(1)}$ decomposes as a direct sum
\[\Ftwo \{\imath^*\}\oplus\bigoplus_{\smash{I\in \aDTirr}}\Ftwo \left\{\deltavstar_I(\imath^*\lambda_1^j)\ |\ j\geq0\right\}.\]
\end{prop}
\begin{proof}
The basis of the Koszul complex was described in Proposition \ref{propDerivedIndTrivialUobject n=0}, and the Koszul differential is zero as $X$ is trivial. The $\lambda$-operations were calculated in Proposition \ref{QkTrivial}.
\end{proof}
Now we have put considerable effort into calculating $H^*_{\calw(1)}$ of each summand in this decomposition: Theorem \ref{thm on collapsing of most sseqs} proves that
\[H^*_{\calw(1)}(\Ftwo \{\imath^*\})\cong F^{\calmh(2)}F^{\calmv(2)}(\Ftwo \{\imath\})\cong \Lambda(\CommOperad)(\imath),\]
while propositions \ref{Sg1 calc of V2} and \ref{Se1 calc of V2} calculate
\[H^*_{\calw(1)}\left(\Ftwo \left\{(\deltavstar_I\imath^*)\lambda_1^j\ |\ j\geq0\right\}\right)\textup{ for $I\in\aDTirr$.}\]
With a view to calculating the first \CFSS,
we catalogue a collection of generators of $\E{(1)}{2}{}{}{}$ under the $\calMhv(2)$-operations. The \emph{fundamental class} $\imath\in 
\quadgrad{1}\E{(1)}{2}{}{0,0}{T}$ is an exterior generator (arising in Theorem \ref{thm on collapsing of most sseqs}). %, as are the classes $\deltav_k\imath\in (E_2)^{0,1,(2)}_{d+k+1}$ for $2\leq k\leq d$ (denoted $v_0$ in Theorem \ref{W2 to W1 collapse} \textbf{or the corollary to that: write it)).
Moreover, for all $I\in\aDTirr$, there are further generators, arising in Corollary \ref{the actual statement of what H1V1 is}:
\begin{alignat}{2}
\label{first generator}\deltav_I\imath&\in \quadgrad{2^{\ell I}}\E{(1)}{2}{}{0,\ell I}{T+nI+\ell I}&\qquad&\\
\Sqv^{3}\deltav_{\TOP_T^j\!I}\imath&\in 
\quadgrad{2^{1+\ell I+j}}\E{(1)}{2}{}{1,2+\ell I+j}{2^{j+1}(T+nI+\ell I+1)-1}
%(E_2)^{1,2+\ell I+j,(2^{1+\ell I+j})}_{2^{j+1}(T+nI+\ell I+1)-1}
&\qquad&\textup{(when $j\geq1$, but not $j=\ell I=1$),}\\
(\deltav_{\TOP_T I}\imath)^{2}&\in 
%(E_2)^{1,4,(2^{3})}_{2^{2}(T+nI+\ell I+1)-1}
\quadgrad{2^{3}}\E{(1)}{2}{}{1,4}{2^{2}(T+nI+\ell I+1)-1}
&\qquad&\textup{(when $\ell I=1$)},
\label{last generator}
\end{alignat}
where they are referred to  as $v_0$, $\Sqv^3v_j$ and $v_1^2$ respectively.
Note that this final generator, $(\deltav_{\TOP_T I}\imath)^{2}$, has the same degrees as the generator $\Sqv^{3}\deltav_{\TOP_T^j\!I}\imath$ that is \emph{missing} when $j=\ell I=1$.
\begin{thm}
The first \CFSS\ collapses at $E_2$:
\[\E{(1)}{2}{}{}{}=H^*_{\calw(1)}V^*_{(1)}\implies H^*_{\calw(0)}V^*_{(0)}.\]
\end{thm}
\begin{proof}
The fundamental class is a permanent cycle, so to prove that the spectral sequence collapses, it is enough to show that no classes
%\[x\in (E_2)^{0,\ell I,(2^{\ell I})}_{T+nI+\ell I}\textup{ or }y\in (E_2)^{1,2+\ell I+j,(2^{1+\ell I+j})}_{2^{j+1}(T+nI+\ell I+1)-1}\]
\[x\in \quadgrad{2^{\ell I}}\E{(1)}{2}{}{0,\ell I}{T+nI+\ell I}\textup{ or }y\in \quadgrad{2^{1+\ell I+j}}\E{(1)}{2}{}{1,2+\ell I+j}{2^{j+1}(T+nI+\ell I+1)-1}\]
can support a differential, $I$ a non-empty $\delta$-admissible sequence.

To see this, all one needs to have learned about the entire $E_2$-page is that it is a subquotient (in which $\imath^2=0$) of the polynomial algebra on symbols
\[\Sqh^A\Sqv^B\deltav_C\imath\in \quadgrad{2^{\ell A+\ell B+\ell C}}\E{(1)}{2}{}{\ell B+nA,2^{\ell A}(nB-\ell B+\ell C)}{2^{\ell A+\ell B}(T+nC+\ell C+1)-1},\]
in which $B$ is $\Sq$-admissible, B does not contain $1$ or $2$, if $C$ is empty then so is $B$, and if $B$ is empty then so is $A$. These conditions imply that $nB-2\ell B\geq2^{\ell B}-1$. \todo{\tiny check} %Moreover, in this subquotient, $\imath^2=0$ (among other relations explained above).

If for $r\geq2$ there is  a differential $d_r$ supported by $y$, %a class in $(E_2)^{1,2+\ell I+j,(2^{1+\ell I+j})}_{2^{j+1}(d+nI+\ell I+1)-1}$, 
then $d_{r}y $ must be a sum of products of $N\geq1$ such classes. The generic such monomial may be written as:
\[\textstyle\prod_{k=1}^{N}\Sqh^{A_k}\Sqv^{B_k}\deltav_{C_k}\imath\in \quadgrad{\sum2^{\ell A_k+\ell B_k+\ell C_k}}\E{(1)}{2}{}{\sum(\ell B_k+nA_k)+N-1,\sum2^{\ell A_k}(nB_k-\ell B_k+\ell C_k)}{-1+\sum2^{\ell A_k+\ell B_k}(T+nC_k+\ell C_k+1)}\]
in which $\ell C_k=0$ for at most one $k$. We derive the following constraints:
\begin{gather}
\textstyle \sum(\ell B_k+nA_k)\geq4-N,\label{1eqnofindices}\\
\log_2(N)+\textstyle \frac{1}{N}\textstyle \sum_k \left[\ell A_k+\ell B_k+\ell C_k\right]\geq1+\ell I+j,\label{2eqnofindices}\\
4+\ell I+j=\textstyle\sum_{k}\left[\ell B_k+nA_{k}\right]+N-1+\textstyle\sum_{k}\left[2^{\ell A_k}(nB_k-\ell B_k+\ell C_k)\right]\label{3eqnofindices},\\
\log_2(N)\geq \textstyle\sum_{k}\left((2^{\ell A_k}-\frac{1}{N})\ell C_k+\left[\left(2^{\ell A_k}(nB_k-\ell B_k)-\frac{1}{N}\ell B_k\right)-\frac{1}{N}(\ell A_k)\right]\right).\label{4eqnofindices}
\end{gather}
The inequality (\ref{1eqnofindices}) is just the requirement that $r\geq2$, while (\ref{2eqnofindices}) results from the observation that $d_r$ preserves the quadratic grading and the convexity of the exponential function. Equation (\ref{3eqnofindices}) holds since the total degree of the differential is one, and (\ref{4eqnofindices}) is derived by rearranging the sum of (\ref{1eqnofindices}), (\ref{2eqnofindices}) and (\ref{3eqnofindices}).
%\[r=N-2+\sum(\ell B_k+nA_k)\geq2.\]
%Now $d_r$ preserves the quadratic grading, and convexity of the exponential function implies that
%\[1+\ell I+j\leq \log_2(N)+\frac{1}{N}\textstyle \sum_k \left[\ell A_k+\ell B_k+\ell C_k\right].\]
%%with equality if and only if all of the $\ell A_k+\ell B_k+\ell C_k$ coincide. This obviously occurs when $N=1$, and also when $N=2$ as we are considering an integer equation $2^\alpha=2^\beta+2^\gamma$.
%As the total degree of the differential is one, we must also have
%\[4+\ell I+j=-1+\textstyle\sum_{k}\left[\ell B_k+nA_{k}+1\right]+\textstyle\sum_{k}\left[2^{\ell A_k}(nB_k-\ell B_k+\ell C_k)\right]\]
%and combining this with the above inequalities yields, after a little manipulation:
%\[\log_2(N)\geq \textstyle\sum_{k}\left((2^{\ell A_k}-\frac{1}{N})\ell C_k+\left[\left(2^{\ell A_k}(nB_k-\ell B_k)-\frac{1}{N}\ell B_k\right)-\frac{1}{N}(\ell A_k)\right]\right).\]
(\ref{4eqnofindices}) is a very strong inequality, since the expression $2^{\ell A_k}(nB_k-\ell B_k)-\frac{1}{N}\ell B_k$ is at least $2^{\ell B_k}-\frac{1}{N}$, and $nB_k-\ell B_k\geq2$ if $\ell B_k\neq0$. Thus, in  (\ref{4eqnofindices}), each expression in square brackets is always non-negative, is at least $2-\frac{1}{N}$ when $\ell B_k\neq0$, and exceeds $2-\frac{1}{N}$ if $\ell B_k\geq2$ or $\ell A_k\neq0$. 

When $N=1$ or $N=3$, $\log_2(N)<2-\frac{1}{N}$, so that (\ref{4eqnofindices}) implies that $\ell B_k=0$ for all $k$, violating (\ref{1eqnofindices}).
When $N\leq2$, $\log_2(N)\leq 2-\frac{1}{N}$, so that (\ref{4eqnofindices}) implies that $\ell B_k\neq0$ for at most one $k$, with $\ell B_k=1$, violating (\ref{1eqnofindices}).
When $N\geq4$, all but at most one of the summands $(2^{\ell A_k}-\frac{1}{N})\ell C_k$ in (\ref{4eqnofindices}) is at least $\frac{3}{4}$, and as $\frac{3}{4}(N-1)\geq\log_2(N)$ when $N\geq4$, (\ref{4eqnofindices}) is violated. Thus $y\in E_2$ is a permanent cycle.

Performing the same calculations for $d_rx$, we find that the inequality (\ref{4eqnofindices}) is unchanged, while (\ref{1eqnofindices}) is replaced by
\begin{gather}
\textstyle \sum(\ell B_k+nA_k)\geq3-N.\label{5eqnofindices}\end{gather}
The argument is unchanged when $N=1$ or $N\geq4$, while if $2\leq N\leq3$ we may still draw the same conclusions from (\ref{4eqnofindices}). When $N=2$, we may assume that $\ell B_1=1$ and $\ell B_2=0$, and although (\ref{5eqnofindices}) is not violated, (\ref{4eqnofindices}) is violated as $\ell C_1\neq0$. When $N=3$, we must have $\ell C_k=0$ for each $k$, and the following equations must be satisfied
\[\ell I-1=\ell C_1+\ell C_2+\ell C_3,\ \  2^{\ell I}=2^{\ell C_1}+2^{\ell C_2}+2^{\ell C_3}.\]
As in the proof of Theorem \ref{W2 to W1 collapse}, these equations imply that $\ell C_1,\ell C_2,\ell C_3$ equal $\ell I-1,\ell I-2,\ell I-2$, in some order. The first equation then implies that $\ell I=2$, implying that $\ell C_k=0$ for more than one $k$, which we have prohibited. Thus $x\in E_2$ is a permanent cycle.
\end{proof}
This theorem has the following corollary, stated in this form due to potential hidden extensions:
\begin{cor}
\label{the corollary on the bousfield kan e2}
Suppose that $X=\mathbb{S}_{T}^{\algs}$ for $T\geq1$. Then the \BKSS\ $E_2$-page $\E{}{2}{\calx}{}{}\cong H^*_{\calw(0)}(H^*_\algs X)$ is isomorphic, as a vector space in $\vect{1}{+}$, to the $\calMh(1)$-subquotient of $F^{\calMh(1)}F^{\calMv(1)}\{\imath\}$ generated by 
the fundamental class $\imath$ and the elements
\[
\{\deltav_{I}\imath,
\Sqh^2\deltav_{\TOP_T^1\!I}\imath,\Sqh^3\deltav_{\TOP_T^1\!I}\imath,
\Sqh^2\deltav_{\TOP_T^2\!I}\imath,\Sqh^3\deltav_{\TOP_T^2\!I}\imath,
\ldots\}\textup{ for $I\in\aDTirr$,}
\]
and subject to  relations generated under $\calMh(1)$-operations by
\[\{
\Sqh^2\deltav_{I}\imath,
\Sqh^2\deltav_{\TOP_T^1\!I}\imath,
\Sqh^2\deltav_{\TOP_T^2\!I}\imath,
\Sqh^2\deltav_{\TOP_T^3\!I}\imath,
\ldots\}\textup{ for $I\in\aDTirr$.}
\]
\end{cor}
\begin{proof}
This follows from the collapsing of the first \CFSS, our knowledge of the generators $\imath$ and (\ref{first generator})-(\ref{last generator}) of $\E{(1)}{2}{}{}{}$, and  a few observations in the low-dimensional cases. 

When $\ell I=0$: in $F^{\calMh(1)}F^{\calMv(1)}\{\imath\}$, by unstableness of the horizontal Steenrod operations, $\Sqh^2\imath=0$, $\Sqh^3\imath=0$ and $\imath^2=\Sqh^1\imath=0$, so that $\imath$ contributes no more to this subquotient than it did as an exterior generator of $\E{(1)}{2}{}{}{}$.

When  $\ell I=1$: in $F^{\calMh(1)}F^{\calMv(1)}\{\imath\}$, the generators (\ref{last generator}) satisfy
$(\deltav_{\TOP_T I}\imath)^{2}=\Sqh^3\deltav_{\TOP_T I}\imath$, and taking the quotient by $\Sqh^2\deltav_{\TOP_T I}\imath$ ensures that these generators produce no more material in $F^{\calMh(1)}F^{\calMv(1)}\{\imath\}$ than the polynomial algebras arising in the $S=1$ case of Corollary \ref{the actual statement of what H1V1 is}.
\end{proof}

%\TODOCOMMENT{\tiny Delete}
%\begin{shaded}\tiny
%\begin{cor}[\textbf{(!!! Old version !!!)}]
%Suppose that $X=\mathbb{S}_{T}^{\algs}$ for $T\geq1$. Then the Bousfield-Kan $E_2$-page $\E{}{2}{\calx}{}{}\cong H^*_{\calw(0)}(H^*_\algs X)$ is isomorphic, as a vector space in $\vect{1}{+}$, to the non-unital commutative algebra coproduct of
%\[\Lambda(\CommOperad)[\imath]\sqcup \textstyle\bigsqcup_{2\leq i\leq T} \Bigl(S(\CommOperad) [(\deltav_{\TOP_T(i)}\imath)^{2}]\sqcup \Lambda(\CommOperad)[\deltav_i\imath]\Bigr)\]
%with the subquotient of $F^{\calMhv(1)}\{\imath\}$ \textbf{outdated usage} generated by the elements:
%\begin{alignat}{2}
%%\label{ell 2plus}
%\{\deltav_{I}\imath,
%\Sqh^2\deltav_{\TOP_T^1\!I}\imath,\Sqh^3\deltav_{\TOP_T^1\!I}\imath,
%\Sqh^2\deltav_{\TOP_T^2\!I}\imath,\Sqh^3\deltav_{\TOP_T^2\!I}\imath,
%\ldots\}\textup{ for $I\in\aDTirr$, $\ell I>1$;}
%\\
%%\label{ell 1}
%\{
%\Sqh^2\deltav_{\TOP_T^2\!I}\imath,\Sqh^3\deltav_{\TOP_T^2\!I}\imath,
%\Sqh^2\deltav_{\TOP_T^3\!I}\imath,\Sqh^3\deltav_{\TOP_T^3\!I}\imath,
%\ldots\}\textup{ for $I\in\aDTirr$, $\ell I=1$;}
%\end{alignat}
%and subject to  relations generated under $\calMh(1)$-operations by
%\begin{alignat}{2}
%%\label{ell 2plus}
%\{
%\Sqh^2\deltav_{I}\imath,
%\Sqh^2\deltav_{\TOP_T^1\!I}\imath,
%\Sqh^2\deltav_{\TOP_T^2\!I}\imath,
%\Sqh^2\deltav_{\TOP_T^3\!I}\imath,
%\ldots\}\textup{ for $I\in\aDTirr$, $\ell I>1$;}
%\\
%%\label{ell 1}
%\{
%%\Sqh^2\deltav_{I}\imath,
%%\Sqh^2\deltav_{\TOP_T^1\!I}\imath,
%\Sqh^2\deltav_{\TOP_T^2\!I}\imath,
%\Sqh^2\deltav_{\TOP_T^3\!I}\imath,
%\ldots\}\textup{ for $I\in\aDTirr$, $\ell I=1$.}
%\end{alignat}
%\end{cor}
%\end{shaded}
%


\SubsectionOrSection{An alternative Bousfield-Kan $E_1$-page}
\label{An alternative Bousfield-Kan E1}
We will now suggest a somewhat artificial $E_1$-page for the \BKSS\ for a  sphere  $X=\mathbb{S}_{T}^{\algs}$ for $T\geq1$, but one that will be motivated by the conjectures and calculations of \S\ref{The Bousfield-Kan spectral sequence for a sphere}. 
Define:
\begin{alignat*}{2}
% Left hand side
\aS{s}&:=\left\{J\ \middle|\ \textup{$J$ a $\Sq$-admissible sequence with }\minDimDelta(I)\leq s+1,\ 1\notin J\right\};\\
\aSirr{s}&:=\left\{J\ \middle|\ \textup{$J$ a $\Sq$-admissible sequence with }e(I)\leq s,\ 1\notin J\right\};\\
\aDTnoplus&:=\left\{I\ \middle|\ \textup{$I$ a $\delta$-admissible sequence with }\minDimP(I)\leq T\right\}.
\end{alignat*}
The difference between $\aDTnoplus$ and $\aDT$ is just that we have removed the requirement that $I$ be non-empty. The following lemma expains the sense in which $\aSirr{s}$ is the subset of irreducible sequences in $\aS{s}$.
\begin{lem}
\label{lemma on the func STOPs}
There is an injective function $\STOP_t:\aS{s}\to \aS{s}$ given by
\[J=(j_{\ell},\ldots,j_1)\overset{\smash{\STOP_s}}{\longmapsto }(s+nJ+1,j_\ell,\ldots,j_1).\]
Moreover, $\aSirr{s}=\aS{s}\setminus\im(\STOP_s)$, and
\[\aS{s}=\textstyle\bigsqcup_{J\in \aSirr{s}}\{J,\STOP_s J,\STOP_s^2 J,\ldots\}.\]
\end{lem}
\noindent The proof is similar to that of Lemma \ref{lemma on the func TOPt}, but the outcome is a little different. Indeed, Lemma \ref{lemma on the func STOPs} shows that if a $\calMh(1)$-expression $\Sqh^{J}x$ contains a top Steenrod operation, then all of the Steenrod operations following it are also top operations.

Define
\begin{equation}\label{defining equation of Eprime}
\Eprime{blank}{1}{\calx}{}{}:=
\Ftwo\left\{\textstyle\prod_{k=1}^N\Sqh^{J_k}\deltav_{I_k}\imath\ \middle|\ 
\genfrac{}{}{0pt}{}{I_k\in \aDTnoplus,\ J_k\in \aS{\ell I_k}}{(J_k,I_k)\neq (J_{k'},I_{k'})\textup{ unless }k'=k}
\right\},
\end{equation}
and define a differential on $\Eprime{blank}{1}{\calx}{}{}$ by:
\newcounter{keepeqno}
\setcounter{keepeqno}{\value{equation}}%
  \begin{list}{(\theequation)}{\usecounter{equation}}%
  \setcounter{equation}{\value{keepeqno}}
\item \label{ITM1} setting $d_1\imath=0$;
\item \label{ITM2} requiring that $d_1$ distributes across the monomials in (\ref{defining equation of Eprime}) according to the Liebniz rule;
\item \label{ITM2.5} %requiring that $\Sqh^{j}x=0$ for $x\in \Eprime{blank}{1}{\calx}{s}{t}$ and $j>s+1$;
requiring, for $x\in \Eprime{blank}{1}{\calx}{s}{t}$, that $\Sqh^{s+2}x=xd_1x$ and $\Sqh^jx=0$ for $j>s+2$;
\item \label{ITM3} requiring, for $x\in \Eprime{blank}{1}{\calx}{s}{t}$, that $d_1\Sqh^jx = \Sqh^jd_1x$;
\item \label{ITM4}
requiring, for $x\in \Eprime{blank}{1}{\calx}{s}{t}$, that $d_1\deltav_ix=
\begin{cases}
\deltav_id_1x,&\textup{if $2\leq i < t$};\\
\deltav_id_1x + \Sqh^2 x,&\textup{if $2\leq i = t$};
\end{cases}$
\item \label{ITM5} enforcing the equation $\deltav_i\Sqh^j=0$;
\item \label{ITM6} enforcing the $\Sq$-Adem relations and the identity $\Sqh^1=0$;
%\item \label{ITM7} using the equation $x^2=\Sqh^{s+1}x$ (for $x\in \Eprime{blank}{1}{\calx}{s}{t}$) whenever a summand in the image of $d_1$ violates the requirement ``$(J_k,I_k)\neq (J_{k'},I_{k'})$ unless $k'=k$''.
\item \label{ITM7} 
whenever a summand in the image of $d_1$ violates the requirement that the factors $\Sqh^{J_k}\deltav_{I_k}\imath$ be unique, applying the unstableness condition
\[(\Sqh^{J_k}\deltav_{I_k}\imath)^{2}=\Sqh^{\smash{\STOP_{\ell I_k}J_k}}\deltav_{I_k}\imath.\]
%s, for $x\in \Eprime{blank}{1}{\calx}{s}{t}$, $x^2=\Sqh^{s+1}x$  and $\Sqh^{j}x=0$ for $j>s+1$. 
\end{list}
Note that (\ref{ITM2.5}), (\ref{ITM6}) and (\ref{ITM7}) imply that $\imath^2=0$ and $\Sqh^{2}\imath=0$.
%%%% requiring that $d_1$ distributes across these products according to the Liebniz rule, requiring that for $x\in \Eprime{blank}{1}{\calx}{s}{t}$:
%%%%\begin{alignat}{2}
%%%%d_1\Sqh^jx
%%%%&=
%%%%\Sqh^jd_1x;
%%%%%&\qquad&\text{(for $x\in \Eprime{blank}{1}{\calx}{s}{t}$)}
%%%%\\
%%%%d_1\deltav_ix
%%%%&=
%%%%\begin{cases}
%%%%\deltav_id_1x,&\textup{if $2\leq i < t$};\\
%%%%\deltav_id_1x + \Sqh^2 x,&\textup{if $2\leq i = t$};
%%%%%\\,&\textup{if }
%%%%\end{cases}
%%%%%&\qquad&\text{(for $x\in \Eprime{blank}{1}{\calx}{s}{t}$)}
%%%%%\\
%%%%%d_1((\Sqh^J\deltav_I\imath) (\Sqh^{J'}\deltav_{I'}\imath) )&=(d_1\Sqh^J\deltav_I\imath)(\Sqh^{J'}\deltav_{I'}\imath)+(\Sqh^J\deltav_I\imath)(d_1\Sqh^{J'}\deltav_{I'}\imath)
%%%%\end{alignat}
%%%%and also 
%%%%%\underline{\scr}\newcounter{keepeqno}
%%%%\setcounter{keepeqno}{\value{equation}}%
%%%%  \begin{list}{(\theequation)\ \ \ }{\usecounter{equation}}%
%%%%  \setcounter{equation}{\value{keepeqno}}
%%%%\item enforcing the equation $\deltav_i\Sqh^j=0$ 
%%%%\item enforcing the $\Sq$-Adem relations
%%%%\item using the equation $x^2=\Sqh^{s+1}x$ (for $x\in \Eprime{blank}{1}{\calx}{s}{t}$) whenever a summand in the image of $d_1$ violates the requirement ``$(J_k,I_k)\neq (J_{k'},I_{k'})$ unless $k'=k$''
%%%%\end{list}
%%THIS IS WHAT THE FREE Mhv1 is: \[S(\CommOperad) \left[\Sqh^J\deltav_Ib\ \middle|\ \genfrac{}{}{0pt}{}{b\in B_t\textup{, $I$ non-empty, $\delta$-admissible with }\minDimP(I)\leq t,}{\textup{$J$ $\Sq$-admissible with }\excess(J)\leq \ell I,\textup{ and }1\notin J}\right]\sqcup \Lambda(\CommOperad)\left[b\ \middle|\ b\in B\right].\]
The key point is that we do not want the differential to be determined by manipulations such as:
\[d_1(\Sqh^5\deltav_{(22,10,5,2)}\imath)\,\textup{``$=$''}\,d_1((\deltav_{(22,10,5,2)}\imath)^{2})\,\textup{``$=$''}\,
2(\deltav_{(22,10,5,2)}\imath)(d_1\deltav_{(22,10,5,2)}\imath)
%+
%(d_1\deltav_{(22,10,5,2)}\imath)(\deltav_{(22,10,5,2)}\imath)
=0,\]
which is why the phrasing of (\ref{ITM2}) and (\ref{ITM7}) is so restrictive.
Indeed, when we define in \S\ref{Operations on the Bousfield-Kan spectral sequence} operations on the Bousfield-Kan spectral sequence,  the top Steenrod operation will \emph{not} equal the product square at $E_1$, but only at $E_2$, and we are mimicking  this behaviour in our definition of $\Eprime{blank}{1}{\calx}{}{}$.

Let us calculate the proposed differential applied to a generator
$\Sqh^J\deltav_I\imath$ of $\Eprime{blank}{1}{\calx}{}{}$ with $I\neq\emptyset$. Suppose that $I=(i_{\ell I},\ldots,i_1)$, with $\deltav_{i_a}$ acting as a top operation at precisely the indices $a={a_n},\ldots,{a_1}$. Then we calculate, 
\begin{alignat*}{2}
d_1\Sqh^J\deltav_I\imath
&=
\Sqh^Jd_1\deltav_I\imath%
\\
&=
%\Sqh^J \deltav_Id_1\imath+
\textstyle\sum_{m=1}^{n}\Sqh^J\deltav_{(i_{\ell I},\ldots,i_{a_{m}+1})} \Sqh^{2} \deltav_{(i_{a_{m}-1},\ldots,i_1)}\imath%
\\
% Left hand side
% Relation
&=
% Right hand side
\begin{cases}
\smash{\Sqh^J\Sqh^2\deltav_{(i_{{\ell I}-1},\ldots,i_1)}\imath},&\textup{$\deltav_{i_{\ell I}}$ a top operation, $2,3\notin J$, $\ell I\geq2$};\\
0,&\textup{otherwise}.
\end{cases}
\end{alignat*}
The first equation holds by (\ref{ITM3}), and the second holds by (\ref{ITM1}) and (\ref{ITM4}).
To explain the third equation, %note first that $\Sqh^J \deltav_Id_1\imath$ vanishes as $\imath$ is a permanent cycle. Next, 
all of the $n$ summands vanish by (\ref{ITM5}), 
except perhaps for the $m=n$ summand, which need not vanish when $a_n=i_{\ell I}$. Even this summand may still vanish, as (\ref{ITM6}) implies that $\Sqh^J\Sqh^2$ vanishes unless it is already $\Sq$-admissible. 

Although the definition of this complex seemed complicated, the differential ends up being quite simple. Indeed, one deduces that, writing $J=(j_{\ell J},\ldots,j_1)$:
\[
\textup{if }J\in\aS{s}\textup{ is non-empty and } 2,3\notin J,\textup{ then }(j_{\ell J},\ldots,j_1,2)\in\aSirr{s-1}.
\]
From this, we conclude that if $J$ is non-empty:
\begin{equation}
\label{cycle conditions 1}
\textup{$\Sqh^J\deltav_I\imath$ is a cycle if and only if $I \in \aDTirr$ or $J$ contains 2 or 3},
\end{equation}
and if $J$ is empty but $I$ is non-empty:
\begin{equation}
\label{cycle conditions 2}
\textup{$\deltav_I\imath$ is a cycle if and only if $I \in \aDTirr$}.\end{equation}
(Recall that $\aDTirr$ contains all of the length one sequences $(i)$ for $2\leq i \leq T$). We can combine all of this information into the following observation, valid for any $I,J$:
\begin{equation}
\label{cycle conditions 3}
\textup{$\Sqh^J\deltav_I\imath$ is a cycle if and only if $I=\emptyset$ or $I \in \aDTirr$ or $J$ contains 2 or 3}.
\end{equation}
The determination of the  homology of $\Eprime{blank}{1}{\calx}{}{}$ will follow from a generalization of this calculation made in \S\ref{The resulting differentials}, in particular Proposition \ref{the description of the interim pages of the sseq}. While the calculations in \S\ref{The resulting differentials} are contingent on Conjectures \ref{conjec1} and \ref{conjec2}, the statements are independent of these conjectures insofar as they apply to $\Eprime{blank}{1}{\calx}{}{}$. As a result, we can state the following:
\begin{cor}[(of Proposition \ref{the description of the interim pages of the sseq})]
The homology of $\Eprime{blank}{1}{\calx}{}{}$ is isomorphic, as a vector space, to $\E{}{2}{\calx}{}{}$ as calculated in Corollary \ref{the corollary on the bousfield kan e2}.
\end{cor}
\begin{proof}
The isomorphism of vector spaces $H_*\Eprime{blank}{1}{\calx}{}{}\to \E{}{2}{\calx}{}{}$ sends the class of one of the cycles $\Sqh^J\deltav_I\imath$ of (\ref{cycle conditions 3}) to the element $\Sqh^J\deltav_I\imath$ of the subquotient of $F^{\calMh(1)}F^{\calMv(1)}\{\imath\}$ identified  in Corollary \ref{the corollary on the bousfield kan e2}. Proposition \ref{the description of the interim pages of the sseq} provides a basis of $H_*\Eprime{blank}{1}{\calx}{}{}$ which can be compared directly with that of the subquotient.
\end{proof}

\end{Calculations of HWn}




%\begin{The cohomology of a trivial unstable Lie algebra over P}
%
%\SectionOrChapter{The cohomology of a trivial unstable Lie algebra over $P$}
%There may be a wrinkle here answering the well definedness vs vanishing question --- maybe I have a mistake, and some operation is only well defined on $meep^1$.
%\TODOCOMMENT{State interest in this calculation}{write $X=\bigoplus_\alpha \Ftwo [r_\alpha]$, a finite direct sum}
%\TODOCOMMENT{Construct the relevant elements $\Sq^J\delta_I\imath_\alpha$ and state a theorem on $H^*X$}{$\imath_\alpha\in H^0_{r_\alpha}X$ is the functional which projects $QX\cong X$ onto $\Ftwo [r_\alpha]$.
%\item Clearly state the contraints relevant for $K$ and $I$.
%\item Also describe which products thereof are not known to be zero, and state a theorem.}
%\TODOCOMMENT{Proposition: any of the $\Sq^J\delta_I\imath_\alpha$ is detected by $\Sqv^{J_n}e\cdots e\Sqv^{J_1}e\delta_I\imath_\alpha$}{again, clearly state the constraints on the $J_i$.
%\item Push out to $ee\Sqv^{J_n}e\cdots e\Sqv^{J_1}e\delta_I\imath_\alpha\in meep^2(n+2)_*^{00**\cdots *}$}
%\TODOCOMMENT{Calculate the groups $meep^2(n)_*^{*0**\cdots *}$}{Observe importance of these groups, since everything's detected in one}
%\TODOCOMMENT{Identify products of $ee\Sqv^{J_{n-2}}e\cdots e\Sq^{J_1}e\delta_I\imath_\alpha\in meep^2(n)_*^{00**\cdots *}$ in $meep^2(n)_*^{*0**\cdots *}$}{I still find this hard to write down}
%\TODOCOMMENT{Prove the theorem on $H^*X$}{start with anything in $meep^2(n)$
%\item it is detected by something in $meep^2(N)_*^{*0**\cdots *}$
%\item anything there is a product of $ee\Sqv^{J_{N-2}}e\cdots e\Sq^{J_1}e\delta_I\imath_\alpha$
%\item this converges down to $\Sq^K_h \Sqv^{J_n}e\cdots e\Sqv^{J_1}e\delta_I\imath_\alpha$, which must equal the original element
%\item this is a permanent cycle, proving that there are never any differentials in any of the cfsseqs
%\item thus $meep^2(0)^*_*$ is a direct sum of certain of the $meep^2(n)^{*0**\cdots *}_*$ via collapsing sequences
%\item stocktake reveals that these terms are everything we hoped}
%\end{The cohomology of a trivial unstable Lie algebra over P}


\begin{The Bousfield-Kan spectral sequence for a sphere}
\SectionOrChapter{The Bousfield-Kan spectral sequence for $\mathbb{S}^{\algs}_T$}
\label{The Bousfield-Kan spectral sequence for a sphere}
For any $T\geq1$, let $X=\mathbb{S}^{\algs}_T$, so that we may write $\E{}{r}{\calx}{}{}$ for the Bousfield-Kan spectral sequence of the sphere $\mathbb{S}^{\algs}_T$.  In this chapter we will give conjectures which will allows us to construct a complete system of differentials in $\E{}{r}{\calx}{}{}$, that would explain the convergence of $\E{}{2}{\calx}{}{}$ (whose underlying vector space was calculated in Corollary \ref{the corollary on the bousfield kan e2}) to
\[\pi_*(\mathbb{S}^{\algs}_T)\cong \Lambda(\CommOperad)[\delta_I\imath\ |\ I\in \aDTe].\]
Here $\imath\in \pi_T(\mathbb{S}^{\algs}_T)$ is the fundamental class (c.f.\ Proposition \ref{homotopy of comm alg sphere}), and we write
\[\aDTe:=\{I\ |\ \textup{$I$ is $\delta$-admissible, $e(I)\leq T$}\}.\]
%%We will also use the notation
%\begin{alignat*}{2}
%\aDTm&:=\{I\ |\ \textup{$I$ is $\delta$-admissible, $\minDimP(I)\leq T$}\},\\
%\aDTnoplus&:=\left\{I\ \middle|\ \textup{$I$ a $\delta$-admissible sequence with }\minDimP(I)\leq t\right\}.
%%\aDTme&:=\{I\ |\ \textup{$I$ is $\delta$-admissible, $\minDimP(I)\leq T<e(I)$}\}.
%\end{alignat*}

\SubsectionOrSection{Some conjectures on the $E_1$-level structure}
\label{Some conjectures on the E1-level structure}
In order to construct all of the differentials needed, we will \emph{assume} from this point on:
\begin{conjecture}
\label{conjec1}
It is possible to modify the definitions of the spectral sequence operations $\mu$, $\Sqh^j$ and $\deltav_i$ defined in \S\ref{Definition of the spectral sequence operations} in order that the $\Sq$-Adem relations and the relations $\deltav_i\Sqh^j=0$ hold on $E_1$ (without compromising the existing properties of these operations summarized in Proposition \ref{adams operations are right for comm} and corollaries \ref{prop on basic product composed with lift}-\ref{Prop on einfty ops composed with lift}).
\end{conjecture}
\noindent That is, we will replace the operations defined in \S\ref{Definition of the spectral sequence operations} with their conjectural counterpart (without no change of notation).

Recall the alternative Bousfield-Kan $E_1$-page defined in \S\ref{An alternative Bousfield-Kan E1}, written $\Eprime{blank}{1}{\calx}{s}{t}$. There was already a map of vector spaces $\Eprime{blank}{1}{\calx}{}{}\to \E{}{1}{\calx}{}{}$  in $\vect{1}{+}$ defined by
\[\Eprime{blank}{1}{\calx}{}{}\ni \textstyle\prod_{k=1}^N\Sqh^{J_k}\deltav_{I_k}\imath\mapsto \textstyle\prod_{k=1}^N\Sqh^{J_k}\deltav_{I_k}\imath\in \E{}{1}{\calx}{}{},\]
and using the conjectural definitions of the operations on  $\E{}{1}{\calx}{}{}$, it is a map of chain complexes. Indeed, we may calculate the differential in $\E{}{1}{\calx}{}{}$ exactly as we calculated in \S\ref{An alternative Bousfield-Kan E1}. Thus, there is an induced map $\Eprime{blank}{2}{\calx}{}{}\to \E{}{2}{\calx}{}{}$ (where we write $\Eprime{blank}{2}{\calx}{}{}$ for the homology of the chain complex $\Eprime{blank}{}{\calx}{}{}$). From now on, we will \emph{also} assume:
\begin{conjecture}
\label{conjec2}
The induced map $\Eprime{blank}{2}{\calx}{}{}\to \E{}{2}{\calx}{}{}$ is an isomorphism (of vector spaces).
\end{conjecture}
\noindent This conjecture is not so unreasonable, since by Corollary \ref{the corollary on the bousfield kan e2} there is an isomorphism of vector spaces $\Eprime{blank}{2}{\calx}{}{}\to \E{}{2}{\calx}{}{}$ given by mapping an element of $\Eprime{blank}{2}{\calx}{}{}$ to the element of $\E{}{2}{\calx}{}{}$ of the same name, under the calculation of $\E{}{2}{\calx}{}{}$ given by Corollary \ref{the corollary on the bousfield kan e2}. In any case, we assume no more than the stated conjectures.

\SubsectionOrSection{The resulting differentials}
\label{The resulting differentials}
We will now analyse the differentials $d_r$ applied to the various terms $\Sqh^J\deltav_I\imath$. Define functions
\[\ell,n,e:\aDTnoplus\to \{0,1,2,\ldots\}\]
which evaluate on a sequence $I=(i_l,\ldots,i_1)$ as follows:
\[\ell(I):=l;\ \ n(I):=i_1+\cdots +i_l;\ \ e(I):=i_l-i_{l-1}-\cdots -i_1=2i_l-n(I).\]
Define a function
\[a:\aDTnoplus \to \mathbb{Z}\]
by $a(I):=\ell(I)-1-(e(I)-T)$.  Now write $\calg$ for the set of $(J,I)$ involved in the definition of $\Eprime{blank}{1}{\calx}{}{}$:
\[\calg:=\{(J,I)\ |\ I\in \aDTnoplus,\ J\in \aS{\ell (I)}\}.\]
We may decompose $\calg$ into three subsets:%, (with notation $J=(j_{\ell(J)},\ldots,j_1)$ implied):
\begin{alignat*}{2}
\calg^{e}
&:=
\{(\emptyset,I)\ |\ I\in \aDTe\},%
\\
\calg'
&:=
\{(J,I)\in\calg\setminus \calg^e\ |\ J=(j_{\ell(J)},\ldots,j_1),\ \ell(J)=0\textup{ \makebox[\widthof{and}][c]{or}\ }a(I)+3-j_1< 0\},%
\\
% Left hand side
\calg''
% Relation
&:=
% Right hand side
\{(J,I)\in\calg\setminus \calg^e\ |\ J=(j_{\ell(J)},\ldots,j_1),\ \ell(J)>0\textup{ and\ }a(I)+3-j_1\geq 0\}.%
% Comment
\end{alignat*}
Every class $\deltav_I\imath$ for $(\emptyset,I)\in\calg^e$ is a permanent cycle. On the other hand, we will prove:
\begin{prop}\label{thereisamap...}
Assuming Conjectures \ref{conjec1} and \ref{conjec2}, there is a bijective map $g:\calg'\to \calg''$ such that if $g(J,I)=(J',I')$ then there is a differential $d_r:\Sqh^J\deltav_I\imath\mapsto \Sqh^{J'}\deltav_{I'}\imath$.
\end{prop}


The class $\delta_I\imath$  is a permanent cycle if $I\in\aDTe$. On the other hand, 
using Conjecture \ref{conjec1} we may mimic the calculation of $d_1\Sqh^J\deltav_I\imath$ made in \S\ref{An alternative Bousfield-Kan E1}. We find that
if $I\in \aDTnoplus \setminus\aDTe$, $\delta_I\imath$  survives to $E_{a(I)+1}$, at which point
\begin{equation}\label{theBasicDiff}
d_{a(I)+1}:\delta_I\imath\mapsto\Sqh^{a(I)+2}\delta_{I^{-}}\imath,
\end{equation}
where we write $I^-$ for the sequence $(i_{\ell(I)-1},\ldots,i_1)$ obtained by removing the outermost entry of $I$.
%Conjecture \ref{conjec1} is used  to obtain this formula \emph{on the nose} on $E_1$, where we would otherwise only obtain it up to homotopy on $E_1$.

For an element $J\in \aS{s}$ with $J=(j_{\ell(J)},\ldots,j_1)$, and any $n\geq 2-j_1$ we will write $\Phi_nJ$ for the sequence
\[\Phi_nJ:=(j_{\ell(J)}+2^{\ell(J)-1}n,\ldots,j_2+2n,j_1+n)\in \aS{s+n}.\]
Then there is a differential, obtained by applying $\Sqh^J$ to the $d_{a(I)+1}$-differential (\ref{theBasicDiff}):
\[d_{2^{\ell (J)}a(I)+1}:\Sqh^J\delta_I\imath \mapsto  \Sqh^{\Phi_{\smash{a(I)}}J}\Sqh^{a(I)+2}\delta_{I^{-}}\imath= \Sqh^{J^{+}}\delta_{I^{-}}\imath ,\]
where $J^+:= (j_{\ell(J)}+2^{\ell(J)-1}a(I),\ldots,j_2+2a(I),j_1+a(I),a(I)+2)$.
We define the map $g$ to send this $(J,I)$ to $(J^{+},I^-)$ whenever $(J,I)\in\calg'$.
\begin{proof}[Proof of Proposition \ref{thereisamap...}]
Firstly, we should check that $g$ is well defined. Suppose first that $J=\emptyset$. Then we must have $e(I)>T $, so that 
\[a(I)+2=\ell(I)-1-(e(I)-T)+2\leq \ell(I^-)+1,\]
a condition required for $J^+$ to have any chance of lying in $\aS{\ell (I^-)}$. After this initial check, it is easy to check the condition required of $\minDimP(J^{+})$. Thus, $(J,I)\in \calg\setminus \calg^e$. We must also check that $a(I^-)+3-(a(I)+2)\geq0$, i.e.\ that $a(I^-)-a(I)\geq -1$, which reduces to the tautological condition $e(I)\geq e(I^{-})$. Thus $g$ is well defined.

The injectivity of $g$ is obvious, but we must check its surjectivity. Suppose for this purpose that $(J,I)\in\calg''$, so that $a(I)+3-j_1\geq 0$. We will begin by producing a differential
\[d_{j_1-1}:\deltav_{I^+}\imath\mapsto \Sqh^{j_1}\deltav_I\imath\]
with $I^{+}$ a $\delta$-admissible sequence $(i_{\ell(I)+1},i_{\ell(I)},\ldots,i_1)$. For this, we need $e(I^{+})\geq e(I)$ (to ensure admissibility of $I^+$) and $a(I^{+})=j_1-2$, but we are otherwise unconstrained, as $j_1\geq2$, and the demand $a(I^{+})\geq0$ will ensure that the additional $\delta_{i_{\ell(I)+1}}$ is defined. Focusing on the constraint $a(I^{+})=j_1-2$:
\[\ell(I^{+})-1-(e(I^{+})-T)=j_1-2 \iff e(I^{+})-e(I)=a(I)+3-j_1,\]
but we have assumed that $a(I)+3-j_1$ is non-negative, so we have no difficulty satisfying this constraint.

Now we use the sequence $\Phi_{-a(I^{+})}J^{-}$ where $J^-:=(j_{\ell(J)},\ldots,j_2)$, producing the required differential
\[d_{2^{\ell(J)-1}a(I^{+})+1}:\Sqh^{\Phi_{\smash{-a(I^{+})}}J^{-}}\deltav_{I^{+}}\imath\to \Sqh^J\delta_I\imath,\]
as long as either $J^{-}$ is empty or  $a(I^{+})+3-(j_2-a(I^{+}))< 0$. If $J^{-}$ is non-empty, then the second condition reduces to the condition that the concatenation $\Phi_{a(I^{+})}\Phi_{-a(I^{+})}J^{-}\star (a(I^{+})+2)$ be $\Sq$-admissible, but  this concatenation is $J$ itself.
\end{proof}
\begin{prop}
\label{thats all the difls}
Assuming Conjectures \ref{conjec1} and \ref{conjec2}, the differentials given in Proposition \ref{thereisamap...}, along with those arising from them by taking products and applying the Liebniz formula, are a complete set of differentials for the \BKSS\ for this sphere.
\end{prop}
\begin{proof}
Although the $E_r$-page of the spectral sequence is not an exterior algebra for any finite $r$, we are working in a spectral sequence of commutative $\Ftwo$-algebras. As such, the differential is not sensitive to the difference between the polynomial algebra 
$S(\CommOperad)[x]$ and the exterior algebra $\Lambda(\CommOperad)[x,x_2,x_4,x_8\ldots]$ where $x_{2^i}$ is placed in the dimension of $x^{2^i}$. In this setting, the upshot is that  the $E_r$-page is isomorphic as a chain complex to an infinite coproduct of exterior algebras, starting with
\[\Eprime{blank}{1}{}{}{}\cong\bigsqcup_{(J,I)\in\calg} \Lambda(\CommOperad)[\Sqh^{J_k}\deltav_{I_k}\imath].\]
We rely on the properties of the Steenrod operations to allow us to deal with terms of the form $x\times x$. Whatever the explaination, the differentials given in Proposition \ref{thereisamap...} are enough to eliminate all summands except for 
\[\E{}{\infty}{}{}{}\cong\bigsqcup_{(\emptyset,I)\in\calg^e} \Lambda(\CommOperad)[\deltav_{I_k}\imath],\]
which really is isomorphic as an algebra to the target $\pi_*(\mathbb{S}^{\algs}_T)$.
\end{proof}
Filtering the sets $\calg'$ and $\calg''$ by the length of the differentials associated with their elements, so that
\[\calg'_r:=\{(J,I)\in\calg'\ |\ \textup{if $g(J,I)=(J',I')$ then $n(J')+\ell(I')\geq n(J)+\ell(I)+r$} \},\]
and $\calg''_r:=\im(g|_{\calg'_r})$, the  proofs of propositions \ref{thereisamap...} and \ref{thats all the difls} also prove:
\begin{prop}
\label{the description of the interim pages of the sseq}
Assuming Conjectures \ref{conjec1} and \ref{conjec2}, there is an isomorphism of chain complexes, for $r\geq2$:
\[\E{}{r}{}{}{}\cong 
\bigsqcup_{(J,I)\in\calg} \Lambda(\CommOperad)[\Sqh^{J_k}\deltav_{I_k}\imath]\sqcup
\bigsqcup_{(J,I)\in\calg'_r} \Lambda(\CommOperad)[\Sqh^{J_k}\deltav_{I_k}\imath]\sqcup
\bigsqcup_{(J,I)\in\calg''_r} \Lambda(\CommOperad)[\Sqh^{J_k}\deltav_{I_k}\imath]
.\]
\end{prop}

Moreover, the complete calculation of the \BKSS\ for a finite connected model in $s\algs$ now follows simply by taking the coproduct of non-unital differential graded algebras at each page, with the appropriate grading shifts, for example:
\[\E{}{r}{(\mathbb{S}_{T_1}^\algs\sqcup \mathbb{S}_{T_2}^\algs)}{}{}\cong \E{}{r}{\mathbb{S}_{T_1}^\algs}{}{}\sqcup \E{}{r}{\mathbb{S}_{T_2}^\algs}{}{}.\]


%
%\begin{shaded}\tiny
%Unimportant:
%\begin{lem}
%If $a(I)+3-j_1\geq 0$ then $d_{2^{\ell (J)}a(I)+1}(\Sqh^J\deltav_I\imath)=0$.
%\end{lem}
%\begin{proof}
%$a(I)+3-j_1\geq 0$ if and only if  $J^{+}$ is not $\Sq$-admissible, in which case:
%\[d_{2^{\ell (J)}a(I)+1}(\Sqh^J\deltav_I\imath) = \sum_{\produces{J^{+}}{(k_{\smash{\ell(J)}},\ldots,k_1,k_0+2)}{\Sq}} d_{2^{\ell (J)}k_0+1}  \Sqh^{\Phi_{-k_0}(k_{\smash{\ell(J)}},\ldots,k_1)} \deltav_{I'}\imath=0, \]
%where the sum is taken over all sequences not containing $1$ produced by $J^+$, and in each summand we choose the correct $I'$ so that $d_{k_0+1} \deltav_{I'}\imath =\Sqh^{k_0+2}\deltav_{I^{-}}\imath$. The point is that this is possible in each summand, since necessarily, $k_0<a(I)$, and moreover, every summand is then the image of an earlier differential.
%\end{proof}
%\end{shaded}
%However, in this case, we can produce $\Sqh^{J}\deltav_I\imath$ as the image of a differential. In order to do this, we will need to produce a differential
%\[d_{j_1-1}:\deltav_{I^+}\imath\mapsto \Sqh^{j_1}\deltav_I\imath\]
%with $I^{+}$ a $\delta$-admissible sequence $(i_{\ell(I)+1},i_{\ell(I)},\ldots,i_1)$. For this, we need $i_{\ell(I)+1}\geq 2i_{\ell(I)}$ (i.e.\ $e(I^{+})\geq e(I)$) and $a(I^{+})=j_1-2$, but we are otherwise unconstrained, as $j_1\geq2$, and the demand $a(I^{+})\geq0$ will ensure that the additional $\delta_{i_{\ell(I)+1}}$ is defined. Focusing on the constraint $a(I^{+})=j_1-2$:
%\[\ell(I^{+})-1-(e(I^{+})-T)=j_1-2 \iff e(I^{+})-e(I)=a(I)+3-j_1,\]
%but we have assumed that $a(I)+3-j_1$ is non-negative.
%
%Now we use the sequence $\Phi_{-a(I^{+})}J^{-}$ where $J^-:=(j_{\ell(J)},\ldots,j_2)$, producing a differential
%\[d_{2^{\ell(J)-1}a(I^{+})+1}:\Sqh^{\Phi_{\smash{-a(I^{+})}}J^{-}}\deltav_{I^{+}}\imath\to \Sqh^J\delta_I\imath.\]
%
%\begin{alignat*}{2}
%% Left hand side
%\ell(I^{+})►
%% Relation
%&=
%% Right hand side
%¿►%
%% Comment
%&\qquad&\text{(ᾮ)}©
%\end{alignat*}
%
%
%
%Then writing $J^-:=(j_{\ell(J)},\ldots,j_2)$
%\[\Sq^{\Phi_{?}J^-}\deltav_{I^+}\imath\]
%
%
%
%On the other hand, suppose that  $J^{+}$ \emph{is} $\Sq$-admissible. Then we have produced a non-zero differential.
%
%Note that for any $2\leq j <a(I)+2$, the class $\Sqh^j\delta_{I^{-}}\imath$ is zero on 
%
%



\end{The Bousfield-Kan spectral sequence for a sphere}


\begin{May sseq and vanishing line}
\SectionOrChapter{A May-Koszul spectral sequence calculating $H^*_{\calw(0)}$}
\label{May sseq and vanishing line}
%{Explain the quadratic filtration of the bar construction, derive a spectral sequence, explain what it implies about the whole thing}

\SubsectionOrSection{The quadratic filtration and resulting spectral sequence}
\label{The quadratic filtration section}
Suppose that $X\in s\calw(n)$ for  $n\geq0$, and write $\textup{QBX}\in s\vect{+}{n}$ for the simplicial bar construction   calculating $H_*^{\calw(n)}X$:
\[(Q^{\calw(n)}B^{\calw(n)}X)_s\cong (F^{\calw(n)})^{s}X_{s}.\] We may view the vector space $\calU^{\calw(n)}X$ as being quadratically graded, concentrated in quadratic grading 1, and as explained in \S\ref{Quadratic grading}, the monad $F^{\calW(n)}$ may be promoted to a monad on $\quadgrad{}\vect{+}{n}$, so that $\textup{QBX}$ is quadratically graded in each simplicial degree individually.

We derive from these gradings the \emph{quadratic filtration}, the following increasing filtration of $N_*\textup{QBX}\in \complexes \vect{+}{n}$:
\[F_mN_*\textup{QBX}=\textstyle\bigoplus_{k\leq m} \quadgrad{k}N_*\textup{QBX}. \]
This definition is the direct analogue of Priddy's definition \cite{PriddyKoszul.pdf}. It appears difficult to use his techniques to calculate, say, $H^*_{\calw(0)}H^*_\algs\mathbb{S}^{\algs}_T$ directly, as the bar construction in $\calw(0)$ grows so much faster than the bar construction in a category of modules, and the resulting spectral sequence is not degenerate. Nonetheless, the quadratic filtration is in finite in each internal degree:
\begin{lem}
Suppose that $n\geq0$, $X\in s\calw(n)$, and $k\geq0$. Then for any $s_k,\ldots,s_1\geq0$ and $t\geq1$,
\[(F_{2^{t-1}}N_*\textup{QBX})_{{s_k,\ldots,s_1}}^t=(N_*\textup{QBX})_{{s_k,\ldots,s_1}}^t.\]
\end{lem}
\begin{proof}
This follows from the observation that every possible unary (resp.\ quadratic) operation increases $t$ by at least one and doubles quadratic gradings (resp. adds quadratic gradings). It is obvious in dimension $t=1$, as there can have been no non-trivial operations applied in this dimension (the grading $t$ is always non-negative). The full statement follows by induction on $t$.
%
%We prove it by induction on the cohomological dimension $t$. In dimension $t=1$ it is evident that no non-trivial operations can have been applied, and so the quadratic grading can only be one. 
%
%Assume the induction hypothesis in dimensions lower than $t$, and suppose that $x\in \textup{QBX}^t$ is homogeneous. We must show that $x\in \quadgrad{2^{t-1}}\textup{QBX}$. If $x\in \quadgrad{1}\textup{QBX}$ then we are done, and otherwise, $x$ has been formed at some stage in the bar construction using some quadratic or unary operation applied to a class or classes with cohomological dimension at most $t-1$, and by induction, quadratic grading at most $2^{t-2}$. As this operation either adds or doubles the quadratic grading(s), $x$ lies in quadratic grading at most $2\cdot 2^{t-2}$.
\end{proof}
Moreover, there is an isomorphism
\[\E{0}{}{N_*\textup{QBX}}{}{}\cong N_*Q^{\calw(n)}B^{\calw(n)}K^{\calw(n)}U^{\calw(n)}X\]
of chain complexes, so that:
\begin{prop}
\label{theMaySseq}
The cohomotopy spectral sequence of the quadratic filtration is a strongly convergent spectral sequence, the \emph{May-Koszul} spectral sequence:
\[\E{\textup{MK}}{1}{N_*\textup{QBX}}{m,s_n,\ldots,s_1}{t}\cong \quadgrad{m}(H^*_{\calw(n)}K^{\calw(n)}U^{\calw(n)}X)^{s_n,\ldots,s_1}_{t}\implies (H^*_{\calw(n)}X)^{s_n,\ldots,s_1}_{t}.\]
If $\pi_*X$ is of finite type, the $E_1$-page may be rewritten as:
\[\E{\textup{MK}}{1}{N_*\textup{QBX}}{m,s_n,\ldots,s_1}{t}\cong \quadgrad{m}(F^{\HA{\calw(n)}}\dual(\pi_*X))^{s_n,\ldots,s_1}_{t},\]
which reduces when $n\geq1$ to:
\[\E{\textup{MK}}{1}{N_*\textup{QBX}}{m,s_n,\ldots,s_1}{t}\cong \quadgrad{m}(F^{\calMhv(n+1)}\dual(\pi_*X))^{s_n,\ldots,s_1}_{t}.\]
\end{prop}
\noindent Notes that all of the spectral sequence operations defined in \S\ref{Cohomology Operations for W and U} respect the quadratic filtration --- the unary operations double quadratic filtrations while the pairing operations sum then. We leave it to the interested reader to derive the resulting theory of operations in the May-Koszul spectral sequence from this fact, for any $n\geq0$.

\SubsectionOrSection{A vanishing line on the Bousfield-Kan $E_2$-page}
\label{A vanishing line on the Bousfield-Kan}
It is possible to obtain by the following method a vanishing line of slope $4/5$ whenever $\pi_1X$ is of finite type. In the interest of brevity however, we prove only the following:
\begin{thm}
\label{vanishing line prop}
If $X\in s\algs$ is connected (with $\pi_*X$ not necessarily of finite type) then the \BKSS\ admits a vanishing line on $E_2$ of slope $1$ and intercept $0$:
\[\E{}{2}{\calx}{s}{t}=0\textup{ \ whenever \ }s\geq 1\cdot( t-s).\]
\end{thm}
\begin{proof}
We will prove that the right derived functors
\[((\mathbb{R}^s\Prim^{\HC{\algs}})W)_t\]
have such a vanishing line for any $W\in \HC{\algs}$ with $W_0=0$. Any such $W$ is the union of its finite-dimensional subobjects, as all of the structure maps in $\calw(0)$ increase the degree $t$, so it is enough to prove this Proposition for finite-dimensional $W$. Then, by passing to duals, it is enough to produce a vanishing line in the isomorphic vector space
\[H^*_{\calw(0)}\dual W.\]
This group is calculated by the May-Koszul spectral sequence whose $E_1$-page is given by
\[\E{\textup{MK}}{1}{}{m,s}{t}\cong \quadgrad{m}(H^*_{\calw(0)}K^{\calw(0)}U^{\calw(0)}\dual W)^{s}_{t}.\]
Now $K^{\calw(0)}U^{\calw(0)}\dual W$ decomposes as a product (for various $T_i\geq1$):
\[K_{0}^{\calw(0),T_1}\times\cdots \times K_{0}^{\calw(0),T_N},\]
so if we can prove that
\[(H^*)^s_t:=(H^{*}_{\calw(0)}K_0^{\calw(0),T})^s_t=0 \textup{ \ whenever \ }s\geq t-s,\]
the same will be true for $H^*_{\calw(0)}\dual W$ by Theorem \ref{Koszul-dual Hilton-Milnor theorem}. %, it is enough to prove a vanishing line for a single Eilenberg-Mac Lane object $K^{\algs}_{T}$. Note that the intercepts do not shift when applying Theorem \ref{Koszul-dual Hilton-Milnor theorem}, even though the coproducts involved shift filtration by one.
% for $\dual W$, it is enough to prove this result when $\calX$ is a sphere.
However, we have already calculated these groups in Corollary \ref{the corollary on the bousfield kan e2}, and found that $(H^*)^s_t$ is spanned by the image of $\imath\in (H^*)^0_T$ under various $\calMv(1)$- and $\calMh(1)$-operations. All of these operations preserve the half-plane specified by $s<t-s$.
\end{proof}


\end{May sseq and vanishing line}

\appendix
\begin{appendices}





\SectionOrChapter{Cohomology operations for Lie algebras}
\label{appendix on Lie coh ops}
In this appendix, we will prove that Priddy's definitions of cohomology operations for simplicial (restricted) Lie algebras coincides with our own. There are three settings which we are interested in: the categories $s\liealgs$, $s\restliealgs$ and $s\calL(n)$ for $n\geq0$. We will work in the third setting in this appendix, as the proofs in the other two cases are strictly simpler.


\SubsectionOrSection{The partially restricted universal enveloping algebra}
\label{The partially restricted universal enveloping algebra}
For the following discussion, we will need one last category of graded vector spaces, $\vect{-}{n}$, an object of which is simply the direct sum of an object $V$ of $\vect{+}{n}$ and a vector space $V_{0,\ldots,0}^{-1}$:
\[V=V^{-1}_{0,\ldots,0}\oplus\bigoplus_{t\geq{1}}\bigoplus_{s_n,\ldots,s_1\geq0}V^t_{s_n,\ldots,s_1}\in\vect{-}{n}.\]
Denote by $\calA(n)$ the following category of graded augmented associative algebras. An object of $\calA(n)$ is a graded vector space
%\[A=A^{-1}_{0,\ldots,0}\oplus\bigoplus_{t\geq{1}}\bigoplus_{s_n,\ldots,s_1\geq0}A^t_{s_n,\ldots,s_1},\]
$A\in \vect{-}{n}$ such that $A^{-1}_{0,\ldots,0}=\Ftwo \langle 1\rangle$ is one-dimensional, spanned by the unit of an associative unital pairing
\[A^{t}_{s_n,\ldots,s_1}\otimes A^{q}_{p_n,\ldots,p_1}\to A^{t+q+1}_{s_n+p_n,\ldots,s_1+p_1}.\]
That is, $A_{0,\ldots,0}^{-1}$ is not part of the data of $A$, but only a graded piece added to hold the unit. Such an algebra is certainly augmented, and the augmentation ideal may be viewed as a forgetful functor $I:\calA(n)\to\calL(n)$, which sends $A$ to the partially restricted Lie algebra
\[\bigoplus_{t\geq{1}}\bigoplus_{s_n,\ldots,s_1\geq0}A^t_{s_n,\ldots,s_1},\]
with bracket $[x,y]:=xy-yx$, and restriction operation $\restn{x}:=x^2$ whenever $x\in A^t_{s_n,\ldots,s_1}$ and not all of $s_n,\ldots,s_1$ zero.

The composite forgetful functor $\calA(n)\overset{I}{\to}\calL(n)\to\vect{+}{n}$ has a left adjoint, none other than the \emph{free associative algebra functor} $F^{\calA(n)}$ (also known as the \emph{tensor algebra functor}). The multiplicative unit $1$ is placed in $A^{-1}_{0,\ldots,0}$, as is appropriate given the grading shift. Moreover, the functor $I$ has a left adjoint, $\UEA$, \emph{the partially restricted universal enveloping algebra} functor, with $\UEA L$ obtained as the quotient of $F^{\calA(n)}L$ by the two-sided ideal generated by any $[x,y]-xy-yx$ and by $\restn{x}-x^2$ with $x$ of restrictable degree. Indeed, there is a composite of adjunctions
\[\xymatrix@R=.3cm@C=1cm{
\vect{+}{n}  \ar@<.6ex>[r]^{F^{\calL(n)}}&
\calL(n)  \ar@<.4ex>[l]^{\textup{forget}} \ar@<.6ex>[r]^{\UEA}&
\calA(n),  \ar@<.4ex>[l]^{I} 
}
\]
showing that $\UEA\circ F^{\calL(n)}\cong F^{\calA(n)}$. As in the non-restricted and fully restricted case, $\UEA L$ is naturally a Hopf algebra, having diagonal defined by the requirement $\Delta x=1\otimes x+x\otimes 1$ for $x\in L\subseteq \UEA L$, and:
\begin{lem}[(PBW theorem)]\label{Partially restricted PBW Theorem}
If $L\in\calL(n)$, then there is a natural increasing filtration of $\UEA L$, the Lie filtration (by powers of $\langle 1\rangle\oplus \im(L\to \UEA(L))$), and the associated graded algebra is naturally isomorphic to $\Ftwo [L_{\textup{\textbf{0}}}]\otimes E[L_{\neq\textup{\textbf{0}}}]$, where $L=L_{\textbf{\textup{0}}}\oplus L_{\neq\textbf{\textup{0}}}$ is the decomposition of $L$ into the sum of its subspaces of in non-restrictable and restrictable degrees respectively.
\end{lem}
Here, $\Ftwo [\DASH]$ and $E[\DASH]$ denote the (shifted, unital) polynomial and exterior algebra functors respectively, which differ from $S(\CommOperad)$ and $\Lambda(\CommOperad)$ only by the addition of the unit in $(\Ftwo [\DASH])^{-1}_{0,\ldots,0}$ and $(E[\DASH])^{-1}_{0,\ldots,0}$. The unit $1\otimes1 $ of this tensor product is in $(\Ftwo [\DASH]\otimes E[\DASH])^{-1}_{0,\ldots,0}$, as the product has a $+1$-shift in the cohomological dimension.
\begin{lem}
The prolonged functor $\UEA:s\calL(n)\to s\calA(n)$ preserves weak equivalences.
\end{lem}
\begin{proof}
Suppose that $L\to L'$ is a weak equivalence in $s\calL(n)$. The Lie filtration makes $C_*(\UEA L)\to C_*(\UEA L')$ a map of filtered commutative differential graded algebras, so there is an induced map of the resulting spectral sequences. By Lemma \ref{Partially restricted PBW Theorem}, the $E^0$-page of the spectral sequence for $\UEA L$ is the differential graded algebra $C_*(\Ftwo [L_{\textbf{0}}]\otimes E[L_{\neq\textbf{0}}])$. By Dold's Theorem (\ref{Dold's theorem}), the $E^1$-page is a functor (determined by the results of \S\ref{Homotopy operations for simplicial commutative algebras}) of $\pi_*(L_{\textbf{0}})$ and $\pi_*(L_{\neq\textbf{0}})$. As the induced maps $\pi_*(L_{\textbf{0}})\to\pi_*(L'_{\textbf{0}})$ and $\pi_*(L_{\neq\textbf{0}})\to\pi_*(L'_{\neq\textbf{0}})$ are isomorphisms, the map of spectral sequences is an isomorphism from $E^1$.
\end{proof}


\SubsectionOrSection{The proof of Proposition \ref{all the lie steenrod ops are the same}}
In this
\todo{\tiny read through} section we will demonstrate Proposition \ref{the point of the appendix}, which is stated for \emph{partially restricted} Lie algebras $L\in s\call(n)$, but can be reinterpreted for objects of $s\liealgs$ or $s\restliealgs$ as necessary. From this result,  propositions \ref{all the lie steenrod ops are the same} and \ref{Wn Halg omnibus} follow.


Let $ L\in s\calL(n)$ be almost free on a fixed choice of subspaces $V_p\subseteq  L_p$.
We will use a bisimplicial model for $\bar{W}\UEA L$:
\[\textbf{B}_{pq}:=\bar{B}_q \UEA L_p=(\UEA L_p)^{\otimes q}\in ss\vect{-}{n},\]
which in each simplicial level $p$ is the standard simplicial bar construction  for calculation of $\Tor^{\UEA L_p}(\Ftwo ,\Ftwo )$ (c.f.\ \citeBOX[\S1]{PriddyKoszul.pdf}). There are natural equivalences
\[C_*\diag{\textbf{B}}\simeq \textup{Tot}(C_*C_*\textbf{B})=\textup{Bar}(C_*\UEA L)\simeq C_* \bar{W}\UEA L,\]
so that $\pi^*\dual\diag{\mathbf{B}}\cong H^*_{\bar{W}}L$.
Here, we have written $\textup{Bar}$ for the bar construction of \citeBOX[\S7]{grpsHPin.pdf}, and the final equivalence is the homomorphism of \cite[Theorem 20.1]{grpsHPin.pdf}. What is a little less well known is that there is a natural weak equivalence of simplicial coalgebras underlying this equivalence of chain complexes, given in \cite[Theorem 1.1]{CarrRemed.pdf}. A simple construction of such a map 
$\diag{\textbf{B}}\to \bar{W}\UEA L$ is, in simplicial level $n$:
\[d_0\otimes d_0^{\circ2}\otimes\cdots \otimes d_0^{\circ n}:(\UEA X_n)^{\otimes n}\to \UEA X_{n-1}\otimes\cdots \otimes \UEA X_0,\]
where we use the conventions of \citeBOX[\S5]{MillerSullivanConjecture.pdf} to define $\bar{W}$.



As such, the operations defined by Priddy on $H^*_{\bar{W}}$  correspond, under this equivalence, to those that we define on $\pi^*\dual\diag{\mathbf{B}}$ by the formulae
\begin{gather*}
\Sq^k:\bigl(\pi^n\dual \diag{\mathbf{B}}\overset{\ExtCohOp^k}{\to}\pi^{n+k}\dual S^2\diag{\mathbf{B}}\overset{\DeltatubfD^*}{\to}\pi^{n+k}\dual \diag{\mathbf{B}}\bigr);\\
\mu:\bigl(S_2(\pi^*\dual\diag{\mathbf{B}})\overset{\ExtCohProd}{\to} \pi^*S_2\dual\diag{\mathbf{B}}\to \pi^*\dual S^2\diag{\mathbf{B}}\overset{\DeltatubfD^*}{\to}\pi^*\dual \diag{\mathbf{B}}\bigr).
\end{gather*}
where $\DeltatubfD$ is the bisimplicial cocommutative coalgebra diagonal:
\[\DeltatubfD:\left(\bar{B} (\UEA L)\overset{\bar{B}(\Delta)}{\to}\bar{B} (\UEA L\otimes \UEA L)\cong\bar{B} (\UEA L)\otimes \bar{B} (\UEA L)\right).\]
Thus, we may forget the functor $\bar{W}$, and restrict our attention to the object $\textbf{B}$ with this coalgebra map.
We are also going to use the simplicial chain complex $\textbf{Q}\in s\,\complexes\vect{-}{n}$:
\[\textbf{Q}_{\bullet *}:=\begin{cases}
Q^{\calL(n)} L_\bullet,&\textup{if }*=1;\\
\Ftwo \{1\},&\textup{if }*=0;
\\0,&\textup{otherwise.}
\end{cases}
\]
with zero differentials in each simplicial level. Of course, we mean that $1\in (\textbf{Q}_{0,0})^{-1}_{0,\ldots,0}$. There is a map of simplicial chain complexes $r:N\uver_*\textbf{B}_\bullet\to \textbf{Q}_{\bullet*}$, defined in level $p$ by the identification $N\uver_0\textbf{B}_p=\Ftwo \{1\}=\textbf{Q}_{p0}$ and the composite:
\[N\uver_1\textbf{B}_{p}=I\UEA L_p\epi I\UEA L_p/(I\UEA L_p)^2\cong Q^{\calL(n)} L_p.\]
\begin{prop}
\label{the point of the appendix} 
The composite
\[N_*\diag{\textbf{\textup{B}}}\simeq\textup{Tot}(N\uhor_*N\uver_*\textbf{\textup{B}})\overset{r}{\to} \textup{Tot}(N\uhor_*\textbf{\textup{Q}}_{\bullet*})=\Ftwo \oplus \Sigma N_*Q^{\calL(n)} L\]
is a weak equivalence of chain complexes under which the operations on $\pi^*\dual\diag{\textup{\textbf{B}}}$ defined using  $\DeltatubfD$  correspond to the operations $\Sq^k:=\psi_{\calL(n)}\circ\ExtCohOp^{k-1}$ and $\mu:=\psi_{\calL(n)}\circ\ExtCohProd$ on $\pi^*(\dual(Q^{\calL(n)} L))=:H^*_{\calL(n)} L$.
\end{prop}
We will prove this proposition using the external spectral sequence operations of \S\ref{singer ext ops sect} in the spectral sequence of $\textup{\textbf{B}}$. By $E_2$, the only interesting non-zero entries of this spectral sequence lie on the horizontal line $q=1$, so that Singer's operations will prove very uniteresting without modification. Our method will be to perform such a modification by using the  chain homotopy $h$ (defined shortly) to shift the horizontal operations one higher in filtration. The shifted homotopy operations will preserve the line $q=1$, and will abut to operations on $E_\infty$ that satisfy the same relations as those on $\diag{\textbf{B}}$. As the abutment filtration is trivial, they must satisfy the same relations at $E_2$. Finally, we will note that what we have produced at $E_2$ is the definition of the Steenrod operations from \S\ref{section: Cohomology operations for simplicial (restricted) Lie algebras}.


As $ L$ is levelwise free, the evident map $F^{\calA(n)}V_p\to \UEA L_p$ is an isomorphism for each $p$, and we define a vertical homotopy $h:N\uver_*\textbf{B}_{p}\to N\uver_{*+1}\textbf{B}_{p}$ by the following formulae (in which the $v_{i_{j}}$ are taken to be in $V_p\subseteq  L_p\subseteq\UEA L_p$):
\begin{alignat*}{2}
h_q:N\uver_q\bar{B}\UEA L_p&\overset{}{\longrightarrow} N\uver_{q+1}\bar{B}\UEA L_p\\
[\smash{\underbrace{v_{i_1}|\cdots |v_{i_{k-1}}}_{\textup{length 1 bars}}}|v_{i_{k}}v_{i_{k+1}}\cdots |\cdots ]
&\overset{}{\longmapsto}
[v_{i_1}|\cdots |v_{i_{k}}|v_{i_{k+1}}\cdots |\cdots ]%
\\
[v_{i_1}|\cdots |v_{i_{q}}]
&\overset{}{\longmapsto}0.
\end{alignat*}
This homotopy is of the same type as that used in \S\ref{Cohomology Operations for W and U}, \S\ref{Koszul Cx section} and \cite[Proof of Theorem 5.3]{PriddyKoszul.pdf}, and commutes with all of the horizontal simplicial structure except $d\uhor_0$, so that $d\uhor h_q+h_qd\uhor =d\uhor_0h_q+h_qd\uhor_0$.
\begin{lem}
\label{barConstNullHtpyLemma}
Under the map $(\textup{Id}+h_{q-1}d\uver+d\uver h_{q}): N\uver_q\bar{B}\UEA L_p\to N\uver_q\bar{B}\UEA L_p$,
\begin{alignat*}{2}
[v_{i_1}\cdots |\cdots |\cdots v_{i_r}]
&\longmapsto0\text{ unless $r=q=1$, in which case}%
\\
[v_{i_1}]
&\longmapsto [v_{i_{1}}].
\end{alignat*}
\end{lem}

%Singer \cite{SingerSteen1.pdf} gives a method of defining horizontal Steenrod operations on the spectral sequence of $\textbf{B}$. His method is to define a \emph{bisimplicial Eilenberg-Zilber map} $K_k:C\left(X\otimes Y\right)\to CX\otimes CY$ for each $k$, a degree $k$ map satisfying $dK_k+K_kd=K_{k-1}+TK_{k-1}T$. His formulae are given in terms of a \emph{special} simplicial Eilenberg-Zilber map. On the dual total complex, the resulting cohomology operations satisfy the Adem operations.

%Our method of proving [] will be to modify Singer's definition, using the chain homotopy $h$ to shift the horizontal operations one higher in filtration. They will still satisfy the same relations at the abutment, and as the abutment filtration is trivial, they must satisfy the same relations at $E_2$. Finally, we will note that what we have produced at $E_2$ is the definition of the Steenrod operations from \S\ref{section: Cohomology operations for simplicial (restricted) Lie algebras}.
\begin{lem}
\label{firstCompositeLemma}
The composite
\[N\uhor_{p}N\uver_2\textup{\textbf{B}}\overset{\DeltatubfD}{\to} N\uhor_{p}N\uver_2(\textup{\textbf{B}}\otimes\textup{\textbf{B}}) \overset{(D\dver^0)^\star}{\to} N\uhor_{p}(N\uver_1\textup{\textbf{B}}\otimes N\uver_1\textup{\textbf{B}})\overset{r\otimes r}{\to}N\uhor_p(\textup{\textbf{Q}}_{\bullet 1}\otimes \textup{\textbf{Q}}_{\bullet 1})\]
vanishes except on terms $[x|y]$ with $x$ and $y$ generators of $ L_p$, which have image $x\otimes y$.
\end{lem}
\begin{proof}
A generic element of the domain is a sum of terms $[x_1\cdots x_I|y_1\cdots y_J]$, with $x_1,\ldots,x_I$ and $y_1,\ldots,y_J$ in $V_p\subseteq L_p$. This element maps under $\DeltatubfD$ to the following sum, taken over all sequences of exponents $a_1,\ldots,a_I,b_1,\ldots,b_J\in\{0,1\}$:
%\[\sum_{\smash{a_1,\ldots,a_I,b_1,\ldots,b_J\in\{0,1\}}}
%\textstyle\left[\prod_{i=1}^Ix_i^{a_i}\middle|\prod_{\smash{j=1}}^Jy_j^{b_j}\right]\otimes
%\left[\prod_{i=1}^Ix_i^{1-a_i}\middle|\prod_{\smash{j=1}}^Jy_j^{1-b_j}\right]
%\in N\uhor_{p}N\uver_2(\textup{\textbf{B}}\otimes\textup{\textbf{B}}),
%\]
\[\sum%_{\smash{a_1,\ldots,a_I,b_1,\ldots,b_J\in\{0,1\}}}
\textstyle\Bigl[x_1^{a_1}\cdots x_I^{a_I}\Bigm|y_1^{b_1}\cdots y_J^{b_J}\Bigr]\otimes
\Bigl[x_1^{1-a_1}\cdots x_I^{1-a_I}\Bigm|y_1^{1-b_1}\cdots y_J^{1-b_J}\Bigr]
%\Bigl[\prod_{i=1}^Ix_i^{1-a_i}\middle|\prod_{\smash{j=1}}^Jy_j^{1-b_j}\\Bigr]
\in N\uhor_{p}N\uver_2(\textup{\textbf{B}}\otimes\textup{\textbf{B}}),
\]
and $(D\dver^0)^\star$ annihilates all terms except for those in which all $a_i$ are $1$ and all $b_j$ are $0$, leaving
\[\bigl[x_1\cdots x_I\bigr]\otimes
\bigl[y_1\cdots y_I\bigr]\in N\uhor_{p}(N\uver_1\textup{\textbf{B}}\otimes N\uver_1\textup{\textbf{B}}).\]
Finally, $r\otimes r$ annihilates this term unless $I=J=1$.
\end{proof}
\begin{lem}
\label{secondCompositeLemma}
The composite
\[N\uhor_{p+1} N\uver_1\textup{\textbf{B}}\overset{\DeltatubfD}{\to} N\uhor_{p+1}N\uver_1(\textup{\textbf{B}}\otimes\textup{\textbf{B}}) \overset{(D\dver^1)^\star}{\to} N\uhor_{p+1}(N\uver_1\textup{\textbf{B}}\otimes N\uver_1\textup{\textbf{B}})\overset{r\otimes r}{\to}N\uhor_{p+1}(\textup{\textbf{Q}}_{\bullet 1}\otimes \textup{\textbf{Q}}_{\bullet 1})\]
vanishes except on terms $[xy]$ with $x$ and $y$ generators of $ L_{p+1}$, which have image $x\otimes y+y\otimes x$.
\end{lem}
\begin{proof}
A generic element of the domain is a sum of terms $[x_1\cdots x_I]$, with $x_1,\ldots,x_I$ in $V_{p+1}\subseteq L_{p+1}$. This element maps under $\DeltatubfD$ to
%\[\sum_{\smash{a_1,\ldots,a_I\in\{0,1\}}}
%\Bigl[\prod_{i=1}^Ix_i^{a_i}\Bigr]\otimes
%\Bigl[\prod_{i=1}^Ix_i^{1-a_i}\Bigr].\]
\[\sum
\Bigl[x_1^{a_1}\cdots x_I^{a_I}\Bigr]\otimes
\Bigl[x_1^{1-a_1}\cdots x_I^{1-a_I}\Bigr].\]
As $\{D^k\}$ was chosen to be a special $k$-cup product, $(D\dver^1)^\star$ acts as the identity in this case.
Finally, $r\otimes r$ annihilates this term unless $I=2$ and $a_1\neq a_2$.
\end{proof}
\begin{lem}
\label{commuting rectangle lemma for lie operations}
There is a commuting diagram:
\[\xymatrix@R=4mm@C=18mm{
N\uhor_{n+k}N\uver_1\textup{\textbf{B}} \ar[r]^-{d\uhor_0h_{n+k}+h_{n+k-1}d\uhor_0}
\ar[d]^-{r}&%r1c1
N\uhor_{n+k-1}N\uver_2\textup{\textbf{B}} \ar[r]^-{(D\dver^0)^\star\circ\DeltatubfD}&%r1c2
N\uhor_{n+k-1}(N\uver_1\textup{\textbf{B}}\otimes N\uver_1\textup{\textbf{B}})\ar[d]^-{r\otimes r}
\\%r1c3
N\uhor_{n+k}Q^{\calL(n)} L  \ar[r]^-{Q^{\calL(n)}(\xi_{\calL(n)})}&%r1c1
N\uhor_{n+k-1}Q^{\calL(n)}( L\smashcoprod  L) \ar[r]^-{j_{\calL(n)}}&%r1c2
N\uhor_{n+k-1}(Q^{\calL(n)} L \otimes Q^{\calL(n)} L )
}\]
\end{lem}
\begin{proof}
Write
$\textup{LHS}=(r\otimes r)\circ (D\dver^0)^\star\circ\DeltatubfD\circ(d\uhor_0h+hd\uhor_0)$ and $\textup{RHS}= \psi_{\calL(n)}\circ r$.
Consider first an element $e=[v_1v_2\cdots v_b]$ of $N\uhor_{n+k}N\uver_1\textbf{B}$ with $b\geq2$. By definition, $r$ vanishes on such an element, so that $\textup{RHS}(e)=0$. Lemma \ref{firstCompositeLemma} states that the map $(r\otimes r)\circ (D\dver^0)^\star\circ\DeltatubfD$ vanishes except on expressions of the form $[u|w]$ for $u,w\in V_{n+k-1}$. However, the expressions of this form appearing in $d\uhor_0h(e)$ coincide with such expressions in $hd\uhor_0(e)$, so that there is a cancellation, and $\textup{LHS}(e)=0$ as hoped. 

Next, consider an element $[v]$ of $N\uhor_{n+k}N\uver_1\textbf{B}$. As $h[v]=0$, and in light of Lemma \ref{firstCompositeLemma}, $\textup{LHS}([v])$ equals the quadratic part of $d\uhor_0v$, after writing $d\uhor_0v$ as an expression in elements of $V_{n+k-1}$. This is exactly the description given in Lemma \ref{psi is basically the quadratic part} of $\textup{RHS}([v])=\psi_{\calL(n)}(v)$.
%\[\textup{RHS}([v])=(j_{\calL(n)}\circ Q^{\calL(n)}(\xi_{\calL(n)}))(v)=\psi_{\calL(n)}(v).\qedhere\]
\end{proof}
\begin{proof}[Proof of Proposition \ref{the point of the appendix}]
%Having put in place these preparations, let us start working on the theorem. 
Fix a cocycle $\alpha\in \dual(N_{n}Q^{\calL(n)} L)$. %We may assume without loss of generality (or are we \textbf{forced} to assume) that  $\alpha$ evaluates to zero... wait... what am I talking about?
Then $\alpha$ may be viewed an a permanent cocycle in $\EZ{}{\infty}{\dual(\textbf{Q}_{\bullet*})}{n,1}{}$ in the spectral sequence obtained by dualizing $\textbf{Q}_{\bullet*}$. 

Singer \citeBOX[(2.14)]{MR2245560} defines  an operator $S^k$ on the total cochain complex of a bisimplicial coalgebra which induces the cohomology operation $\Sq^k_\textup{ext}$.
We will apply the chain-level operator $S^k$  to the class $r^*\alpha\in \EZ{}{\infty}{\textbf{B}}{n,1}{}$. As $\alpha$ is a permanent cycle, $d(r^*\alpha)=0$, and Singer's expression simplifies to:
%\[S^k(r^*\alpha):=\psi^*K^*_{n+1-k}\oldphi(r^*\alpha\otimes r^*\alpha)=T_1+T_2,\textup{ where}\]
\begin{alignat*}{2}
S^k(r^*\alpha):={}&\DeltatubfD^*K^*_{n+1-k}\oldphi(r^*\alpha\otimes r^*\alpha)=T_1+T_2,\textup{ where:}
\\
T_1
:={}&
\DeltatubfD^*D\dver^0D\dhor^{n+1-k}
\oldphi(r^*\alpha\otimes r^*\alpha)\in \dual(N\uhor_{n+k-1}N\uver_2\textbf{B})%
\\
T_2
:={}&
\DeltatubfD^*D\dver^1(TD\dhor^{n-k}T)
\oldphi(r^*\alpha\otimes r^*\alpha)\in \dual(N\uhor_{n+k}N\uver_1\textbf{B}).
\end{alignat*}
%
% $S^k(r^*\alpha)=\psi^*K^*_{n+1-k}\oldphi(r^*\alpha\otimes r^*\alpha)$, which has two terms:
%\[\underbrace{\psi^*(D\uver_0)^*(D^\textup{h}_{n+1-k})^*\oldphi(r^*\alpha\otimes r^*\alpha)}_{\textup{$T_1$, filtration $n+k-1$}}+
%\underbrace{\psi^*(D\uver_1)^*(TD^\textup{h}_{n-k}T)^*
%\oldphi(r^*\alpha\otimes r^*\alpha)}_{\textup{$T_2$, filtration $n+k$}}\]
Our method will be to compress each of these terms into filtration one higher, using the cochain homotopy $h^*:\dual(N\uhor_* N\uver_*\textbf{B})\to \dual(N\uhor_*N\uver_{*-1}\textbf{B})$.
Using Lemma \ref{barConstNullHtpyLemma}:
\[(\textup{Id}+d\uver h^*+ h^*d\uver )T_1=0\textup{ and }(\textup{Id}+d\uver h^*+ h^*d\uver)T_2=0.\]
The first equation holds as $(\textup{Id}+hd\uver+d\uver h)$ is zero on $N\uver_2\textbf{B}$. For the second equation, on $N\uver_1\textbf{B}$,  $(\textup{Id}+hd\uver+d\uver h)$ is the projection onto terms of the form $[v]$, yet Lemma \ref{secondCompositeLemma} shows that the composite
\[((r\otimes r)\circ(T(D\dhor^{n-k})^\star T)\circ (D\dver^1)^\star\circ\DeltatubfD): N\uhor_{n+k}N\uver_1\textbf{B}\to N\uver_{n+k}(\textbf{Q}_{\bullet1}\otimes \textbf{Q}_{\bullet1})\]
vanishes except on terms of the form $[vw]$ (recall that $r$ commutes with the horizontal simplicial structure).

As $d\uhor h^*+ h^*d\uhor$ increases filtration, we have compressed $S^k(r^*\alpha)$ to the filtration $n+k$ expression $(d\uhor h^*+ h^*d\uhor)T_1$, modulo even higher filtration.
%up to even higher filtration ($n+k+1$), we have compressed $S^k(r^*\alpha)$ to the expression $(\deltav h^*+ h^*\deltav)\textup{Term}_1$.
The commuting diagram of Lemma \ref{commuting rectangle lemma for lie operations} is the left square in a larger commuting diagram:
\[\xymatrix@R=4mm@C=15mm{
N\uhor_{n+k}N\uver_1\textbf{B} \ar[d]^-{r}
 \ar[r]^-{(D\dver^0)^\star\circ\DeltatubfD\circ (d\uhor h+hd\uhor)}&%r1c2
N\uhor_{n+k-1}(N\uver_1\textbf{B}\otimes N\uver_1\textbf{B})\ar[d]^-{r\otimes r}\ar[r]^-{ (D\dhor^{n+1-k})^\star}
&
N\uhor_nN\uver_1\textbf{B}\otimes N\uhor_nN\uver_1\textbf{B}\ar[d]^-{r\otimes r}
\\%r1c3
N\uhor_{n+k}Q^{\calL(n)} L  \ar[r]^-{\psi_{\calL(n)}}&%r1c2
N\uhor_{n+k-1}(Q^{\calL(n)} L \otimes Q^{\calL(n)} L )
\ar[r]^-{ (D\dhor^{n+1-k})^\star}
&
N\uhor_nQ^{\calL(n)} L \otimes N\uhor_nQ^{\calL(n)} L 
}\]
Now $\dual(N\uhor_nQ^{\calL(n)} L \otimes N\uhor_nQ^{\calL(n)} L )$ contains the cocycle $\oldphi(\alpha\otimes\alpha)$, and pulling $\oldphi(\alpha\otimes\alpha)$ back to $\dual(N\uhor_{n+k}N\uver_1\textbf{B})$ along the lower composite yields $r^*\psi_{\calL(n)}^*\ExtCohOp^{k-1}(\alpha)$. Pulling back along the upper composite yields the $E_2$ representative of the shifted version of Singer's operations. 
Both spectral sequences collapse at $E_2$ and induce trivial filtrations on their shared target, so that understanding the shifted operations at $E_2$ is equivalent to understanding the operations on $\pi^*\dual\diag{\textbf{B}}$, which we do: they equal Priddy's operations on $\pi^*\bar{W}\UEA L$ \cite[\S5]{PriddySimplicialLie.pdf}.
As $r^*$ is an $E_2$-equivalence, this proves the result. A simple modification proves the result for pairings.
\end{proof}
\SubsectionOrSection{The Chevalley-Eilenberg-May complex}
\label{The Chevalley-Eilenberg-May complex}
Suppose that $M\in\calL(n)$ is a partially restricted Lie algebra of finite type (not simplicial). One can define a differential coalgebra, the \emph{Chevalley-Eilenberg-May complex}, to be the subcoalgebra $\UEAX(M):= E[M_{\textbf{0}}]\otimes \Gamma[M_{\neq\textbf{0}}]$ of the divided power Hopf algebra $\Gamma[M]$ with its usual coalgebra structure (c.f.\ \citeBOX[p.~141]{MayRestLie.pdf}), graded as follows. 
The Hopf algebra $\Gamma[M]$ is to be $\vect{+}{n+1}$-graded, with product and divided square operations
\begin{gather*}
\Gamma[M]_{p,s_n,\ldots,s_1}^{t}\otimes \Gamma[M]_{p',s'_n,\ldots,s'_1}^{t'}\to \Gamma[M]_{p+p'+1,s_n+s'_n,\ldots,s_1+s'_1}^{t+t'+1},
\\
\gamma_2:\Gamma[M]_{p,s_n,\ldots,s_1}^{t}\to \Gamma[M]_{2p+1,2s_n,\ldots,2s_1}^{2t+1},
\end{gather*}
generated by the subspace
\[\Gamma[M]_{0,s_n,\ldots,s_1}^{t}=M_{s_n,\ldots,s_1}^{t},\]
and we define $\UEAX(M)$ to be the coaugmentation coideal of the subcoalgebra spanned by those expressions
\[\gamma_{r_1}(y_1)\cdots \gamma_{r_m}(y_m) \textup{\ \ (with $y_1,\ldots,y_m\in M$ homogeneous)}\]
for which  $r_i\leq1$ when $y_i\in M_{\textbf{0}}$ (i.e.\ $y_i\in M_{0,\ldots,0}^t$). The coalgebra structure map and differential are the restriction to $E[M_{\textbf{0}}]\otimes \Gamma[M_{\neq\textbf{0}}]$ of those given in \citeBOX[p.~141]{MayRestLie.pdf} (after tensoring the formula  \citeBOX[(6.19)]{MayRestLie.pdf} down to a formula on $\overline{X}(M)$, which kills the first term $\Sigma_{i=1}^nf_iy_i$).


This differs by a shift from the standard definitions, given in \cite{MR0024908} in the unrestricted setting, and  given in \cite{MayRestLie.pdf} in the  restricted setting. It also differs from those definitions in that we have taken the coaugmentation coideal.  Correspondingly, $\UEAX(M)$  is a \emph{shift} of the homology (in the sense of \cite{PriddyKoszul.pdf}) of the associated graded algebra appearing in the partially restricted PBW Theorem, Lemma \ref{Partially restricted PBW Theorem}. 
%%In \citeBOX[p.~141]{MayRestLie.pdf}, the typical element of the coalgebra $\Gamma(L)$ is written as 
%%\[f=\gamma_{r_1}(y_1)\cdots \gamma_{r_m}(y_m),\]
%%for homogeneous $y_i\in L$. We define $\UEAX(L)$ to be the subcoalgebra given by the restriction that $r_i\leq1$ when $y_m\in L_{\textbf{0}}$. 

Now let $ L=B^{\calL(n)}M\in s\calL(n)$. Using the equation $\diag{\textbf{B}}= \bar{B}_\bullet(\UEA M)$ of simplicial coalgebras and  May's injection \cite[Theorem 18 and  (7.8)]{MayRestLie.pdf} of $\UEAX(M)$ into the bar construction, there are maps:
\[\Sigma\UEAX(M)\to N_*\bar{B}_\bullet(U'M)\simeq\textup{Tot}(N\uhor_*N\uver_*\textbf{B})\overset{r}{\to} \textup{Tot}(N\uhor_*\textbf{Q}_{\bullet*})=\Ftwo \{1\}\oplus \Sigma N_*Q^{\calL(n)} L.\]
Now the first map, after the  suspension $\Sigma$ shift homological degree, is a degree preserving map of differential coalgebras, both of which only have a shift in the cohomological degree $t$. In light of this discussion, Proposition \ref{the point of the appendix} implies Proposition \ref{Omnibus on CEM cx} (in which we move back into the notation $S(\CommOperad)$ and $\Lambda(\CommOperad)$ for the non-unital commutative and exterior algebra monads).
\begin{prop}
\label{Omnibus on CEM cx}
$H^*_{\calL(n)}M$ may be calculated, as a \emph{non-unital commutative algebra}, as the cohomology algebra of the differential graded algebra $\dual(\UEAX(M))$, where $\dual(\UEAX(M))$ is  the non-unital commutative algebra $\dual(\UEAX(M))=\Lambda(\CommOperad)[\dual M_{\textup{\textbf{0}}}] \sqcup S(\CommOperad)[\dual M_{\neq\textup{\textbf{0}}}]$. This algebra is $\vect{n+1}{+}$-graded, generated by its subspace
\[(\dual(\UEAX(M)))^{0,s_n,\ldots,s_1}_t=(\dual M)^{s_n,\ldots,s_1}_t,\]
and has grading shifted product
\[\dual(\UEAX(M))^{p,s_n,\ldots,s_1}_{t}\otimes \dual(\UEAX(M))^{p',s'_n,\ldots,s'_1}_{t'}\to \dual(\UEAX(M))^{p+p'+1,s_n+s'_n,\ldots,s_1+s'_1}_{t+t'+1}.\]
\end{prop}
\noindent Recall that the coproduct $A\sqcup B$ of non-unital commutative algebras is the direct sum $A\oplus (A\otimes B)\oplus B$.

We are particularly  interested in the case that $M$ is \emph{trivial as a Lie algebra}, but may still have non-zero restriction. In this case, the restriction is in fact a \emph{linear} map, and we may write $\dualrestn{\DASH}:\dual M\to \dual M$ for its dual (a map which we consider to be everywhere defined, but necessarily equal to zero on $M_{\textbf{0}}$). Examination of  \citeBOX[(6.19)]{MayRestLie.pdf} shows:
\begin{prop}
\label{CEM for trivial lie bracket}
If $M$ has bracket zero, then the differential on the cohomological differential graded algebra $\dual(\UEAX(M))$ is defined on generators $\alpha\in \dual M\subseteq \dual(\UEAX(M))$ by the formula
\[\alpha\longmapsto (\!\sqrt[{[2]}]{\alpha}\,)^2.\]
\end{prop}
%
%\SectionOrChapter{Unstable Lie coalgebras over the dual $P$-algebra (move me)}
%\TODOCOMMENT{Give a formal definition of these coalgebras}{goes down to level of simplicial commutative algebras, and brings in the small object argument comonad structure}
%\TODOCOMMENT{Give the construction (via equalizer diagram, etc) of the comonad}{}
%\TODOCOMMENT{Write the maps $j$ and $\gamma_i$ in terms of maps out of $C_{ij}$ and $C_i^j$}{}
%\TODOCOMMENT{Construct operations on homology of such coalgebras}{they agree with those in double dual construction when  of finite type
%\item if not loc fin, by studying gradings, can see that every coalgebra is the union of its finite type (even just `finite') subcoalgebras, and that the cobar construction respects such unions, thus get the same results in infinite case as we did from double dualization.}


\end{appendices}

%%%%%%%%%%%%%%%%%%% 
%%%%%%%%%%%%%%%%%%%End thecontent for pasteover
%%%%%%%%%%%%%%%%%%% 

\begin{todolist}
\SectionOrChapter{To do/consider:}
\begin{enumerate}\squishlist
\setlength{\parindent}{.25in}
%\item which side do you index bar constructions from?
\item TheToDoListStartsHere
\item Uniformize notation for the total complex of a double complex, and for the cohomotopy of the dual of a simplicial vector space.
%\item delete all pagebreaks, hfil, \textbf{}, suchlike
%\item do not confuse ``quadratic grading 2'' with ``quadratic operations''! The quadratic grading 2 part in a $\Pi$-algebra corresponds to the quadratic operations. So far I have not really made that clear --- actually, this thought is the motivation for the name ``quadratic grading''! Should it be the `arity grading'?
\item Need discourse on that grading on $\calw(n)$.
%\item Point out that $(\quadgrad{2}F^{\calU(n+1)})(V)\oplus (\quadgrad{2}F^{\calL(n+1)})(V)=(\quadgrad{2}F^{\calw(n+1)})(V)$, and that the two monoids both include as subs of $F^{\calw(n+1)}$, and one is a quotient monoid. Broadly just need to fully explain all this grading stuff in all of the free constructions.
%\item Remove all bullets from $B^{\DASH}.$
%\item ``unnormalized'' and ``non(-)normalized''... also, introduce the notation $C_*$
%\item maybe raise the subscripts in $F^\calc$ and in $U^\calc$ and in $K^\calc$ and in $Q_\calc$ (which is done)
%\item say what  of finite type means and point out that things with a shifted degree are all the colimit of their  of finite type subobjects
%\item ``$\delta$-algebra'' is not a good name, $\delta$-admissible too!
\item General principle: homotopy operations ($\delta$, {$\lambda$}) are partially defined, cohomology operations are sometimes zero. The two notions interchange upon taking cohomology.
\item ``goerss looked at unstable ops in the reverse adams SSeq, which is how we understand w. I do something for the forwards ass, and the story comes out rather differently.''
%\item ``Goerss' text explains why one might expect these dualities. (After giving the four koszul algebras in the first chapter). Put 2 algs in the example of commutative algs, and then 2 under lie algs.''
\item Install a $\textup{v}$ in each Koszul $\delta$ operation.
%\item search all ``ref''s and check they have the right word, like ``lemma'' before them.
%\item search $F_?$, $B_?$, $Q_?$, $U_?$, $\xi_?$, $j_?$, $\psi_?$...?
%\item Identify and remove certain common phrases, like `Now $x=y$, but $y\neq z$...'
%\item Introduce uniform notation for connective chain complexes.
%\item once and for all, say that Sq-admiss means $\geq1$ and $\delta,P$ means $\geq2$ and $\Lambda$ means $\geq0$
\item Search lemma, theorem, proposition, etc, and fix it all.
\item mention %SuccessiveSpectralSequences_22draft.pdf and po hu
\item subset to be replaced by subseteq
%\item say ``we will use the phrase \emph{degree $s_k=0$}, or \emph{degree $t=0$}''
%\item search ``non'', and have uniform convention on dashes \textbf{we use a dash}
\item include references to the Chevalley-Eilenberg-May complex section of the appendix
%\item in koszul resolutions, often write $x$ instead of $\Sqvstar{\emptyset}x$. Maybe want to write $\Sqv^*(\emptyset,x)$ instead, in general.
%\item should probably say what the structure  on the tensor product of $E_2$-pages is.
%\item most $\delta$ should be $\deltav$
%\item  \textbf{remove any $[x]$'s}, replace with $\overline{x}$.
\item flesh out the looking glass by comparing unstable homotopy operations and cohomology operations using Koszul duality
%\item point out: knowing $\PiAlg$ structure is knowing htpy of spheres
\item clear warnings using ctrl-backslash
%\item all \textup{v} should be \textup{V}, for size consistency
%\item at some point say what a tensor product of chain complexes is
\item reference \url{http://arxiv.org/pdf/1502.06944v1.pdf}
%\item Remove all apostrophes, as in ``it is''
\item fix everything like $QK$
\item $\quadratic_\calc$ is down.
%\item $\star$ for $*$-preimage - introduce that idea . $\xi$, $\theta_i$, $P^i$, $j$, etc.
%\item search for LieOperad and replace as needed with liealgs and restliealgs
%\item include generating cofibrations $\Delta[1]\times S^{i+j}\to \Delta[1]\times C_{ij}$?
%\item if not  of finite type, by studying gradings, can see that every coalgebra is the union of its finite type (even just `finite') subcoalgebras, and that the cobar construction respects such unions, thus get the same results in infinite case as we did from double dualization.
\item search `Now'
%\item search all $\imath$ appearances
\item check all citations in the text.... replace with citeBOX when desired.
%\item capitalize proposition, theorem, lemma, 
\item section or chapter?
%\item search appendix, and ``and \S''
\item search cancel --- shouldn't appear
\item delete some ``and''
\item `we note' is annoying
\item Liebniz?
\item remove `vfpb' and \hfil in many cases
\item cannot can not?
\end{enumerate}
\end{todolist}
\begin{bibliog}
\printbibliography
\end{bibliog}


\begin{random junk}

\SubsectionOrSection{Throatclearing}

Given a bisimplicial vector space $V_{s,t}$, we obtain a double chain complex $N_sN_tV$. There is a first quadrant homotopy spectral sequence $nope{r}{}{V}{}{s,t}$, with $nope{2}{}{V}{}{s,t}=\pi\uhor_s\pi\uver_tV$, converging to $\pi_{s+t}\diag{ V}$.
There is a first quadrant cohomotopy spectral sequence $\Edown{r}{V}{s,t}{}$, with $\Edown{2}{V}{s,t}{}=\dual(\pi\uhor_s\pi\uver_tV)$, converging to $\dual(\pi_{s+t}\diag{ V})$.

Given a cosimplicial simplicial vector space  $V_{s,t}$, we obtain a mixed chain complex $N^sN_tV$.\\
There is a second quadrant homotopy spectral sequence $nope{r}{}{V}{s}{t}$, with $nope{2}{}{V}{s}{t}=\pi\dhor^s\pi\uver_tV$, which will sometimes converge to $\pi_{t-s}\Tot V$.


\noindent first quadrant homotopy --- Stover's van Kampen\\
first quadrant cohomology --- Serre sseq, our GSSeqs\\
second quadrant homotopy --- Adams or Bousfield-Kan spectral sequences\\
second quadrant cohomology --- Dwyer's work Eilenberg-Moore



%\TODOCOMMENT[inline]{
%Note that the forgetful functor $\calw(0)\to\calU(0)$ does not preserve almost free simplicial objects! However, it does preserve levelwise free simplicial objects, which create the right homology groups (see PRLieAlgs.tex). Thus there is a disconnect in the number of available operations on the Bousfield-Kan $E_2$ and on the first Koszul resolution.}


%\TODOCOMMENT[inline]{examples - explain Lie stuff's Koszul duality in terms of Goerss' work}




\end{random junk}


\end{document}


%Goerss:MR1089001
%Singer:MR2245560
%Priddy:PriddySimplicialLie.pdf or PriddyKoszul.pdf
%Curtis:CurtisSimplicialHtpy.pdf, 6Author.pdf
%6Author.pdf
%MillerSullivanConjecture.pdf
%Blanc_Stover-Groth_SS.pdf
%\cite{FresseSimplicialAlgs.pdf}
%DwyerHtpyOpsSimpComAlg.pdf
%DwyerHigherDividedSquares.pdf
%StoverVanKampen.pdf



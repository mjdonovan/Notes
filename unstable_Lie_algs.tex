% !TEX root = z_output/unstable_Lie_algs.tex
\documentclass[11pt]{amsart}
\newcommand{\IncludeAll}{}% <-----
\newcommand{\ShrinkPages}{}%
\newcommand{\PostponeContents}{}%
\ifx\ShrinkPages\undefined\else\usepackage[paperheight=160mm,paperwidth=160mm]{geometry}\fi%220/160
\usepackage{amsmath,amsthm,amssymb} \usepackage{eucal} \usepackage{mathrsfs,nicefrac} \usepackage{amssymb} \usepackage[all]{xy} \usepackage{cancel} \usepackage{color} \usepackage{version} \usepackage{enumerate} \usepackage{mathtools} \newcommand{\xclude}[1]{\excludeversion{#1}} \newcommand{\nclude}[1]{\includeversion{#1}}%mathtools is just used for `mathclap'
\xclude{todolist}
\xclude{Intro}
\ifx\IncludeAll\undefined\else\renewcommand{\xclude}[1]{\nclude{#1}}\fi
\xclude{Contents Page}
\xclude{Conventions and notation}
\xclude{CPiAlgs and CHalgs}
\nclude{BK spec seq}\nclude{connectivity}
\xclude{Constructing (co)homotopy operations}
\xclude{homotopy operations for PRLs}
\xclude{Cohomology Operations for W and U}
\xclude{second quadrant homotopy}%
\xclude{Operations on the Bousfield-Kan spectral sequence}%
\xclude{Koszul complexes}
\xclude{Composite functor spectral sequences}
\xclude{Operations in composite functor spectral sequences}
\xclude{Calculations of HWn for n nonzero}
\xclude{Calculations of HW0}
\xclude{appendices}
\xclude{bibliog}

\usepackage[bookmarks=false,pdftex,pdfborder={0 0 0 [1 1]}]{hyperref}

\headheight=8pt
\topmargin=0pt
\oddsidemargin=18pt
\evensidemargin=18pt
\textheight=610pt
\ifx\ShrinkPages\undefined\else %if you want to shrink the page while editing, do it here...
  \topmargin=-87pt
  \oddsidemargin=-60pt
  \evensidemargin=-60pt
  \textheight=439pt %610/439
\fi
\textwidth=432pt
\footskip=25pt


%>>>>>>>>>>>>>>>>>>>>>>>>>>>>>>
%<<<  Theorem Environments  <<<
%>>>>>>>>>>>>>>>>>>>>>>>>>>>>>>
\theoremstyle{plain}
\newtheorem{thm}{Theorem}[section]
\newtheorem*{thm*}{Theorem}
\newtheorem{lem}[thm]{Lemma}
\newtheorem*{lem*}{Lemma}
\newtheorem{prop}[thm]{Proposition}
\newtheorem*{prop*}{Proposition}
\newtheorem{cor}[thm]{Corollary}
\newtheorem*{cor*}{Corollary}
\newtheorem{defprop}[thm]{Definition-Proposition}
\newtheorem{conjecture}{Conjecture}
\newtheorem*{conjecture*}{Conjecture}
\newtheorem*{claim}{Claim}

\theoremstyle{definition}
\newtheorem{defn}{Definition}[section]
\newtheorem*{defn*}{Definition}


%>>>>>>>>>>>>>>>>>>>>>>>>>>>>>>
%<<<       Operators        <<<
%>>>>>>>>>>>>>>>>>>>>>>>>>>>>>>
\DeclareMathOperator{\ad}{\textbf{ad}}
\DeclareMathOperator{\coker}{coker}
\renewcommand{\ker}{\textup{ker}\,}
\DeclareMathOperator{\End}{End}
\DeclareMathOperator{\Aut}{Aut}
\DeclareMathOperator{\Hom}{Hom}
\DeclareMathOperator{\Maps}{Maps}
\DeclareMathOperator{\Mor}{Mor}
\DeclareMathOperator{\Gal}{Gal}
\DeclareMathOperator{\Ext}{Ext}
\DeclareMathOperator{\Tor}{Tor}
\DeclareMathOperator{\Cotor}{Cotor}
\DeclareMathOperator{\Prim}{Prim}
\DeclareMathOperator{\Tot}{Tot}
\DeclareMathOperator{\Map}{Map}
\DeclareMathOperator{\Der}{Der}
\DeclareMathOperator{\Rad}{Rad}
\DeclareMathOperator{\rank}{rank}
\DeclareMathOperator{\ArfInvariant}{Arf}
\DeclareMathOperator{\KervaireInvariant}{Ker}
\DeclareMathOperator{\im}{im}
\DeclareMathOperator{\coim}{coim}
\DeclareMathOperator{\trace}{tr}
\DeclareMathOperator{\supp}{supp}
\DeclareMathOperator{\ann}{ann}
\DeclareMathOperator{\spec}{Spec}
\DeclareMathOperator{\SPEC}{\textbf{Spec}}
\DeclareMathOperator{\proj}{Proj}
\DeclareMathOperator{\PROJ}{\textbf{Proj}}
\DeclareMathOperator{\fiber}{fib}
\DeclareMathOperator{\cofiber}{cof}
\DeclareMathOperator{\cone}{cone}
\DeclareMathOperator{\skel}{sk}
\DeclareMathOperator{\coskel}{cosk}
\DeclareMathOperator{\conn}{conn}
\DeclareMathOperator*{\colim}{colim}
\DeclareMathOperator*{\limit}{lim}
\DeclareMathOperator*{\hocolim}{hocolim}
\DeclareMathOperator*{\holimit}{holim}
\DeclareMathOperator*{\holim}{holim}
\DeclareMathOperator*{\hofib}{hofib}
\DeclareMathOperator*{\hocof}{hocof}
\DeclareMathOperator*{\hotfib}{thofib}
\DeclareMathOperator*{\equaliser}{eq}
\DeclareMathOperator*{\coequaliser}{coeq}
\DeclareMathOperator{\ch}{ch}
\DeclareMathOperator{\Thom}{Th}
\DeclareMathOperator{\GrthGrp}{GrthGp}
\DeclareMathOperator{\Sym}{Sym}
\DeclareMathOperator{\Prob}{\mathbb{P}}
\DeclareMathOperator{\Exp}{\mathbb{E}}
\DeclareMathOperator{\GeomMean}{\mathbb{G}}
\DeclareMathOperator{\Var}{Var}
\DeclareMathOperator{\Cov}{Cov}
\DeclareMathOperator{\Sp}{Sp}
\DeclareMathOperator{\Seq}{Seq}
\DeclareMathOperator{\Cyl}{Cyl}
\DeclareMathOperator{\Ev}{Ev}
\DeclareMathOperator{\sh}{sh}
\DeclareMathOperator{\intHom}{\underline{Hom}}
\DeclareMathOperator{\Frac}{frac}


%>>>>>>>>>>>>>>>>>>>>>>>>>>>>>>
%<<<  Mathematical Symbols  <<<
%>>>>>>>>>>>>>>>>>>>>>>>>>>>>>>
\newcommand{\DASH}{\textup{--}}

%>>>>>>>>>>>>>>>>>>>>>>>>>>>>>>
%<<<     Greek Letters      <<<
%>>>>>>>>>>>>>>>>>>>>>>>>>>>>>>
\let\oldphi\phi
\let\phi\varphi
\renewcommand{\to}{\longrightarrow}
\newcommand{\from}{\longleftarrow}
\newcommand{\eps}{\varepsilon}

%%>>>>>>>>>>>>>>>>>>>>>>>>>>>>>>
%%<<<     Environments       <<<
%%>>>>>>>>>>>>>>>>>>>>>>>>>>>>>>
\newcommand{\squishlist}{
  \setlength{\itemsep}{.5pt}
  \setlength{\parskip}{0pt}
  \setlength{\parsep}{0pt}}
%>>>>>>>>>>>>>>>>>>>>>>>>>>>>>>
%<<<     Script Letters     <<<
%>>>>>>>>>>>>>>>>>>>>>>>>>>>>>>
\newcommand{\scrQ}{\mathscr{Q}}
\newcommand{\scrW}{\mathscr{W}}
\newcommand{\scrE}{\mathscr{E}}
\newcommand{\scrR}{\mathscr{R}}
\newcommand{\scrT}{\mathscr{T}}
\newcommand{\scrY}{\mathscr{Y}}
\newcommand{\scrU}{\mathscr{U}}
\newcommand{\scrI}{\mathscr{I}}
\newcommand{\scrO}{\mathscr{O}}
\newcommand{\scrP}{\mathscr{P}}
\newcommand{\scrA}{\mathscr{A}}
\newcommand{\scrS}{\mathscr{S}}
\newcommand{\scrD}{\mathscr{D}}
\newcommand{\scrF}{\mathscr{F}}
\newcommand{\scrG}{\mathscr{G}}
\newcommand{\scrH}{\mathscr{H}}
\newcommand{\scrJ}{\mathscr{J}}
\newcommand{\scrK}{\mathscr{K}}
\newcommand{\scrL}{\mathscr{L}}
\newcommand{\scrZ}{\mathscr{Z}}
\newcommand{\scrX}{\mathscr{X}}
\newcommand{\scrC}{\mathscr{C}}
\newcommand{\scrV}{\mathscr{V}}
\newcommand{\scrB}{\mathscr{B}}
\newcommand{\scrN}{\mathscr{N}}
\newcommand{\scrM}{\mathscr{M}}


%\newcommand{\frakq}{\mathfrak{q}}
%\newcommand{\frakw}{\mathfrak{w}}
%\newcommand{\frake}{\mathfrak{e}}
%\newcommand{\frakr}{\mathfrak{r}}
\newcommand{\frakt}{\mathfrak{t}}
%\newcommand{\fraky}{\mathfrak{y}}
%\newcommand{\fraku}{\mathfrak{u}}
%\newcommand{\fraki}{\mathfrak{i}}
%\newcommand{\frako}{\mathfrak{o}}
%\newcommand{\frakp}{\mathfrak{p}}
%\newcommand{\fraka}{\mathfrak{a}}
\newcommand{\fraks}{\mathfrak{s}}
\newcommand{\frakd}{\mathfrak{d}}
%\newcommand{\frakf}{\mathfrak{f}}
%\newcommand{\frakg}{\mathfrak{g}}
%\newcommand{\frakh}{\mathfrak{h}}
%\newcommand{\frakj}{\mathfrak{j}}
%\newcommand{\frakk}{\mathfrak{k}}
%\newcommand{\frakl}{\mathfrak{l}}
%\newcommand{\frakz}{\mathfrak{z}}
%\newcommand{\frakx}{\mathfrak{x}}
%\newcommand{\frakc}{\mathfrak{c}}
%\newcommand{\frakv}{\mathfrak{v}}
\newcommand{\frakb}{\mathfrak{b}}
%\newcommand{\frakn}{\mathfrak{n}}
%\newcommand{\frakm}{\mathfrak{m}}

%>>>>>>>>>>>>>>>>>>>>>>>>>>>>>>
%<<<  Caligraphic Letters   <<<
%>>>>>>>>>>>>>>>>>>>>>>>>>>>>>>
\newcommand{\calQ}{\mathcal{Q}}
\newcommand{\calW}{\mathcal{W}}
\newcommand{\calE}{\mathcal{E}}
\newcommand{\calR}{\mathcal{R}}
\newcommand{\calT}{\mathcal{T}}
\newcommand{\calY}{\mathcal{Y}}
\newcommand{\calU}{\mathcal{U}}
\newcommand{\calI}{\mathcal{I}}
\newcommand{\calO}{\mathcal{O}}
\newcommand{\calP}{\mathcal{P}}
\newcommand{\calA}{\mathcal{A}}
\newcommand{\calS}{\mathcal{S}}
\newcommand{\calD}{\mathcal{D}}
\newcommand{\calF}{\mathcal{F}}
\newcommand{\calG}{\mathcal{G}}
\newcommand{\calH}{\mathcal{H}}
\newcommand{\calJ}{\mathcal{J}}
\newcommand{\calK}{\mathcal{K}}
\newcommand{\calL}{\mathcal{L}}
\newcommand{\calZ}{\mathcal{Z}}
\newcommand{\calX}{\mathcal{X}}
\newcommand{\calC}{\mathcal{C}}
\newcommand{\calV}{\mathcal{V}}
\newcommand{\calB}{\mathcal{B}}
\newcommand{\calN}{\mathcal{N}}
\newcommand{\calM}{\mathcal{M}}

\newcommand{\calq}{\mathcal{Q}}
\newcommand{\calw}{\mathcal{W}}
\newcommand{\cale}{\mathcal{E}}
\newcommand{\calr}{\mathcal{R}}
\newcommand{\calt}{\mathcal{T}}
\newcommand{\caly}{\mathcal{Y}}
\newcommand{\calu}{\mathcal{U}}
\newcommand{\cali}{\mathcal{I}}
\newcommand{\calo}{\mathcal{O}}
\newcommand{\calp}{\mathcal{P}}
\newcommand{\cala}{\mathcal{A}}
\newcommand{\cals}{\mathcal{S}}
\newcommand{\cald}{\mathcal{D}}
\newcommand{\calf}{\mathcal{F}}
\newcommand{\calg}{\mathcal{G}}
\newcommand{\calh}{\mathcal{H}}
\newcommand{\calj}{\mathcal{J}}
\newcommand{\calk}{\mathcal{K}}
\newcommand{\call}{\mathcal{L}}
\newcommand{\calz}{\mathcal{Z}}
\newcommand{\calx}{\mathcal{X}}
\newcommand{\calc}{\mathcal{C}}
\newcommand{\calv}{\mathcal{V}}
\newcommand{\calb}{\mathcal{B}}
\newcommand{\caln}{\mathcal{N}}
\newcommand{\calm}{\mathcal{M}}
\newcommand{\calmv}{\mathcal{M}_\textup{v}}
\newcommand{\calmh}{\mathcal{M}_\textup{h}}
\newcommand{\calMv}{\mathcal{M}_\textup{v}}
\newcommand{\calMh}{\mathcal{M}_\textup{h}}


\usepackage{framed}
\definecolor{shadecolor}{rgb}{.925,0.925,0.925}
\usepackage[style=numeric,%citestyle=numeric,
url=false,doi=false,isbn=false,eprint=false]{biblatex}%
\hypersetup{colorlinks=false,pdfborder={0 0 0}}

\newdir{ >}{{}*!/-7pt/@{>}}
\newdir{ >>}{{}*!/-14pt/@{>}}

\makeatletter
\renewcommand{\@seccntformat}[1]{\csname the#1\endcsname.\quad}
\makeatother

\setcounter{tocdepth}{2}% <--
\theoremstyle{plain}
\newtheorem{theorem}{Theorem}
\newtheorem*{completenesstheorem}{Completeness Theorem}
\newtheorem{twistinglemma}[thm]{Twisting Lemma}
\renewcommand*{\thetheorem}{\Alph{theorem}}
\newtheorem{conjectureAlpha}{Conjecture}
\renewcommand*{\theconjectureAlpha}{\Alph{conjectureAlpha}}

\newcommand{\PMonad}{{\calP^\textup{u}}}
\newcommand{\LambdaMonad}{\Lambda^\textup{u}}
\newcommand{\Palg}{{\calP}}
\newcommand{\deltaalg}{\Delta} %change me
\newcommand{\LieOperad}{{\scrL}}
\newcommand{\CommOperad}{{\scrC}}
\newcommand{\restn}[1]{#1^{[2]}}
\newcommand{\restnwithsubscript}[2]{#1^{[2]}_{#2}}
\newcommand{\restnRepeated}[2]{#1^{[2^{#2}]}}
\newcommand{\dualrestn}[1]{\sqrt[{[2]}]{#1}}
\renewcommand{\dualrestn}[1]{\sqrt[{\!\!\![2]}]{#1}}
\newcommand{\vect}[2]{\calV^{#1}_{#2}}
\newcommand{\BSW}{{\scrG}}
\newcommand{\BSWres}{B^\BSW}%didn't always remember to use
\newcommand{\PiAlg}{\textup{-$\Pi$-Alg}}
\newcommand{\HAlg}{\textup{-$H^*$-Alg}}
\newcommand{\HCoalg}{\textup{-$H_*$-Coalg}}
\newcommand{\quadratic}{\textup{qu}}
\newcommand{\quadgrad}[1]{\textup{q}_{#1}}
\newcommand{\crossterms}{\textup{cr}}
\newcommand{\ExtCohOp}{\textup{Sq}_\textup{ext}}
\newcommand{\vExtCohOp}{\textup{Sq}_\textup{v,ext}}
\newcommand{\hExtCohOp}{\textup{Sq}_\textup{h,ext}}
\newcommand{\ExtCohProd}{\mu_\textup{ext}}
\newcommand{\epi}{{\,\makebox[0cm][l]{\ensuremath\to}\to}}
\newcommand{\epifrom}{{\,\makebox[0cm][l]{\ensuremath\from}\from}}
\newcommand{\mono}{{\to}}
\newcommand{\minDimP}{\overline{m}}
\newcommand{\minDimDelta}{m}
\newcommand{\minDimSq}{\underline{m}}
\newcommand{\excess}{e}
\newcommand{\produces}[3]{#3:#1\sim #2}
\renewcommand{\produces}[3]{#1\rightarrow_{#3} #2}%{J}{I}{P}
\renewcommand{\produces}[3]{#1\overset{\smash{#3}}{\rightarrow} #2}%{J}{I}{P}
\newcommand{\twist}{\omega}
\newcommand{\DeltaUp}{\Delta}% <-- i changed the indices to upper, marking where I did it.
\newcommand{\Nabla}{\nabla}
\newcommand{\Shuffles}[2]{\textup{Sh}_{#1#2}}
\newcommand{\HalfShuffles}[2]{\textup{Sh}_{#1#2}^{\smash{\!\div2}}}
\newcommand{\Nop}{N^{\smash{-}}}
\newcommand{\NOFULLPAGE}{\relax}
\newcommand{\UEA}{U'}%{U^{[2]}}%if this changes, so does some text
\newcommand{\UEAX}{\overline{X}'}%{U^{[2]}}%if this changes, so does some text
\newcommand{\Sq}{\mathrm{Sq}}
\newcommand{\Sqh}{\mathrm{Sq}_\textup{h}}
\newcommand{\Sqv}{\mathrm{Sq}_\textup{v}}
\newcommand{\deltav}{\delta^\textup{v}}
\newcommand{\LieSteen}{\calA}
\newcommand{\aDT}{\textup{adm}_+(\Delta,T)}
\newcommand{\aDTirr}{\textup{adm}_+^\textup{irr}(\Delta,T)}
\newcommand{\F}{\mathbb{F}}
\newcommand{\Id}{\textup{id}}
\newcommand{\complexes}{\textup{ch}_+}
\newcommand{\algs}{{\scrC\!\textit{om}}}
\newcommand{\liealgs}{{\scrL\!\textit{ie}}}
\newcommand{\restliealgs}{{\textit{r}\scrL\!\textit{ie}}}
\newcommand{\algcat}{{\calC}}%supposed to be temporary - only used in adams convergence
\newcommand{\Ftwo}{\F_2}

\newcommand{\E}[5]{[E^{#1}_{#2}#3]^{#4}_{#5}}
\newcommand{\EZ}[5]{[Z^{#1}_{#2}#3]^{#4}_{#5}}
\newcommand{\EB}[5]{[B^{#1}_{#2}#3]^{#4}_{#5}}
\newcommand{\filt}{\textup{filt}}

\newcommand{\uver}{^\textup{v}}
\newcommand{\uhor}{^\textup{h}}
\newcommand{\dver}{_\textup{v}}
\newcommand{\dhor}{_\textup{h}}
\newcommand{\diag}{\Delta}
\newcommand{\dual}{\mathbf{D}}

\newcommand{\smashprod}{\barwedge}%{\blacktriangle}
\newcommand{\Lsmashprod}{\barwedge^\textup{L}}%{\blacktriangle}
\newcommand{\smashcoprod}{\veebar}%{\blacktriangledown}


%\bibliography{papers}
\bibliography{../../Dropbox/logbook/_LOGBOOK/papers}




\title[Unstable Lie algebras and their cohomology]{Unstable Lie algebras and their cohomology}
%\author[M.\ Donovan]{Michael Donovan}

%\address{Department of Mathematics \\ Massachusetts Institute of Technology}
%\email{mdono@math.mit.edu}


\newcommand{\dupdown}[2]{R_{\smash{#1}}}
\newcommand{\caldup}[1]{\calR_{\smash{#1}}}
\newcommand{\caldupdown}[2]{\calR^{\smash{#1}}_{\smash{#2}}}
\newcommand{\plainD}{R}
\newcommand{\algCat}{\calc}
\newcommand{\trip}[3]{{#1}_{\smash{#2}}^{\smash{#3}}}
\newcommand{\barConstructionMightAbbreviate}{b}

\begin{document}
\newcommand{\todo}[2]{\begin{shaded}\begin{itemize}
\setlength{\parindent}{.25in}
\item[{\Large$\smash\diamondsuit$}] #1
\ifblank{#2}{}{\tiny\begin{itemize}
\setlength{\parindent}{.25in}
\item #2
\end{itemize}}
\end{itemize}\end{shaded}
}
\newcommand{\tododone}[2]{}
\newcommand{\todoeasy}[2]{\begin{shaded}\begin{itemize}
\setlength{\parindent}{.25in}
\item[{\Large$\smash\spadesuit$}] #1
\ifblank{#2}{}{\tiny\begin{itemize}
\setlength{\parindent}{.25in}
\item #2
\end{itemize}}
\end{itemize}\end{shaded}
}

\begin{Contents Page}
%\begin{abstract}\end{abstract}
%\maketitle
\ifx\PostponeContents\undefined{\tiny\tableofcontents}\else\relax\fi
\end{Contents Page}

%\noindent So you start with surjections of bounded below chain complexes:
%\[T\to T_1\to T_0.\]
%Take $K_i=\ker(T\to T_i)$. Obviously $K_1\subseteq K_0$, so...
%\[K_1\to K_0 \to T\to T_1\to T_2.\]
%A crappy diagram chase reveals that 
%\[L:=\coker(K_1\to K_0)\cong \ker (T_1\to T_0)=:S.\]
%(Maybe that's a triangulated category axiom.) I.e.:
%\[\xymatrix@R=4mm{
%&L\\
%K_1\ar@{ >->}[r]& K_0 \ar@{ >->}[r]\ar@{->>}[u]& T\ar@{->>}[r]& T_1\ar@{->>}[r]& T_0\\
%&&&S\ar@{ >->}[u]
%}\]
%So we have a tonne of triangles:
%\begin{enumerate}
%\item $K_1\to K_0\to L$
%\item $S\to T_1\to T_0$
%\item $K_1\to T\to T_1$
%\item $K_0\to T\to T_0$
%\end{enumerate}
%I want to know if the following is true:
%the connecting homs of (3) and (4) and the identification $S\cong L$ amount to a morphism of homology long exact sequences, from LES(2) to LES(1).
%
%Is it? Is it the octahedral axiom? Obviously I have no idea what I'm talking about.
%
%
%



\begin{Intro}
This research will form part of the author's PhD thesis. The author would like to thank his advisor, Haynes Miller, for his support and guidance, and John Harper, for sharing helpful insights into the use of cubical diagrams in connectivity analyses such as that of Section \ref{sec:connectivityAnalysis}.
\end{Intro}

\begin{Conventions and notation}
\section{\textbf{Background and conventions}}

\subsection{Universal algebras}\label{Universal algebras}
In this paper we will be dealing with various categories $\calc$ of universal graded algebras over $\Ftwo $. Following \cite[\S2.1]{Blanc_Stover-Groth_SS.pdf}, these are categories whose objects are $G$-graded $\Ftwo $-vector spaces $X=\{X_g\}_{g\in G}$, for some set $G$ of gradings, equipped with a set of operators of the form $X_{g_1}\times \cdots \times X_{g_n}\to X_k$ (with $n\geq1$)  satisfying a set identities, and whose morphisms are graded vector space maps preserving this structure. We assume that the vector space addition  maps $X_g\times X_g\to X_{g}$ are included in the set of operators, in order that the morphisms in $\calc$ are linear maps. Note that we have explicitly excluded nullary operations. 

It need not  be true that all of the defining operators must be linear, but in each of our examples, $\calc$ will be monadic over the category of $G$-graded $\Ftwo $-vector spaces. To explain this, there is a forgetful functor $U^\calc:\calc\to\vect{}{}$, where $\vect{}{}$ is the category of $G$-graded $\Ftwo $-vector spaces, and $U^{\calc}$ admits a left adjoint  $F^\calc:\vect{}{}\to\calc$. %. The functor $U^{\calc}$ will always be monadic, in the sense that 
When the natural comparison functor from $\calc$ to the category of algebras over the monad $U^{\calc}F^{\calc}$ on $\vect{}{}$ is an equivalence, we say that $\calc$ is monadic over $\vect{}{}$.

%graded vector spaces equipped with 
%
%universal graded algebras \cite[\S2.1]{Blanc_Stover-Groth_SS.pdf}: categories whose objects are $G$-graded $\Ftwo $-vector spaces $X=\{X_g\}_{g\in G}$, for some set $G$ of gradings, equipped with an action of set of operators of the form $X_{g_1}\times \cdots X_{g_n}\to X_k$  which satisfy a set identities.
% Such a category possesses certain important structure, that we introduce now. The examples in this paper will have rather more structure than this, and. Firstly, each of our examples will be monadic over the category of $G$-graded $\Ftwo $-vector spaces.
%
%All of the eWe will assume that 
% monadic over $\vect{}{}$, a category either of graded or ungraded $\Ftwo $-vector spaces. That is, 
%
%We will be more specific about the category $\vect{}{}$ shortly.
%As such, it will be helpful to set notation globally. %[Until \S\ref{The example of simplicial commutative F2-algebras} [\textbf{current?}], our comments will apply to any of the algebraic categories $\calL(n)$, $\calU(n)$ or $\calw(n)$ to be defined below. Let $\calc$ be any of these categories, monadic over $\vect{+}{r}$.]
%Until \S????, we will denote by $\calc$ such a category, although we will need some further assumptions on $\calc$ to proceed. %We will make some further assumptions about the category $\calc$ that will hold in all the examples that we use in this paper.
%
%A category $\calc$ of universal graded algebra is a category whose objects areIn our examples, an object $X$ of $\calc$ may be defined as an object $X$ of $\vect{}{}$ equipped with certain binary and unary operations on $X$ (and without any nullary operations). 
%As such, there will be a forgetful functor $U^\calc:\calc\to\vect{}{}$, which will admit a left adjoint, always denoted $F^\calc:\vect{}{}\to\calc$ for notational consistency. The functor $U^{\calc}$ will always be monadic, in the sense that the natural comparison functor from $\calc$ to the category of algebras over the monad $U^{\calc}F^\calc $ on $\vect{}{}$ is an equivalence.

In our examples, the monad $U^{\calc}F^{\calc}$ will admit an augmentation (of monads) $\epsilon:U^{\calc}F^{\calc}\to\Id$, reflecting homogeneity in the relations defining $\calc$. This augmentation has the monad unit $\eta:\Id\to U^{\calc}F^{\calc}$ as a section, and may be thought of as \emph{projection onto generators}.


We will generally omit the functor $U^{\calc}$ from our notation, writing $F^{\calc}$ as shorthand for either the monad $U^{\calc}F^{\calc}$ on $\vect{}{}$ or the comonad $F^{\calc}U^{\calc}$ on $\calc$. We will refer to elements of a free construction $F^\calc U$ using notation such as $f(v_i)$, thought of as a composite $f$ of $\calc$-structure maps applied to generators $v_i\in V\subseteq F^\calc(V)$. We will say that $f(v_i)$ is a \emph{$\calc$-expression}. In this language, the linear maps % for $V\in \vect{}{}$, we have linear maps
\[F^\calc F^\calc V\overset{\mu}{\to} F^\calc V,\ \ V\overset{\eta}{\to}F^\calc V\ \ \textup{and}\ \ F^\calc V\overset{\epsilon}{\to}V, \]
constituting the augmented monad $F^\calc $ on $\vect{}{}$ may be described as follows: $\mu$ collapses a $\calc$-expression  in $\calc$-expressions into a single $\calc$-expression; $\eta$ sends a vector $v$ to the $\calc$-expression $v$; and $\epsilon$ projects a $\calc$-expression onto those summands to which no (non-trivial) operations have been applied.
For $X\in \calc$, the comonad structure maps in $\calc$,
\[F^\calc F^\calc X\overset{\Delta}{\from} F^\calc X\ \ \textup{and}\ \ X\overset{\rho}{\from}F^\calc X,\]
are as follows: on an expression $f(x_i)$, $\Delta$ returns the same expression $f(x_i)$ in which the $x_i\in X$ are viewed as elements of $F^\calc X$, and $\rho$ is the evaluation map equivalent to the $\calc$-structure on $X$.

\subsection{The functor $Q^\calc$ of indecomposables}
Using the augmentation $\epsilon:F^\calc\to\Id$ of monads on $\vect{}{}$, any $V\in\vect{}{}$ becomes an $F^\calc $-algebra, i.e.\ an object of $\calc$. We denote this functor $K^\calc:\vect{}{}\to \calc$; it sends $V\in\vect{}{}$ to \emph{the trivial object on $V$}, which is $V$ equipped with coaction map the projection $\epsilon:F^{\calc}V\to V$. In each of our examples,  $K^{\calc}$ has a left adjoint, $Q^{\calc}:\calc\to\vect{}{}$, which sends $X\in\calc$ to \emph{the quotient of $X$ by the image of its non-trivial operations}. %More precisely, $K^\calc$ sends any object $V\in\vect{+}{r}$ to the $U^{\calc}F^\calc $-algebra whose structure map is $\epsilon_V:U^{\calc}F^\calc V\to V$. 
%The functor $Q^{\calc}$ sends $X\in \calc$ with structure map $\rho:F^\calc U^\calc X\to X$ to the coequalizer in $\vect{+}{r}$ of the maps $U^\calc\rho$ and $\epsilon_{U^\calc X}$. \emph{or, less correctly}
The functor $Q^{\calc}$ sends $X\in \calc$ to the coequalizer in $\vect{}{}$ of $\rho,\epsilon:F^\calc X\to X$.

Note that $F^\calc$ is a section of $Q^{\calc}$, since $Q^{\calc}\circ F^{\calc}$ is adjoint to $U^\calc K^\calc$, which is obviously the identity.

\subsection{Quillen's model structure on $s\calc$ and the bar construction}\label{ssec: quillen model and bar construction}
For any of the algebraic categories $\calc$ herein we use Quillen's simplicial model category structure on the category $s\calc$ \cite{QuillenHomAlg.pdf}, \cite{MillerSullivanConjecture.pdf}, or \cite{Blanc_Stover-Groth_SS.pdf}. In this structure, the weak equivalences (fibrations) are the maps which are weak equivalences (fibrations) of simplicial abelian groups, so that every object is fibrant. 

 A simplicial object $X$ is \emph{almost free} if there are subspaces $V_n\subseteq X_n$ for each $n\geq0$ such that the composite $F^\calc  V_n\to F^\calc  X_n\overset{\rho}{\to} X_n$ is an isomorphism for all $n$, and such that the subspaces $V_n$ are preserved by all of the degeneracies and face maps of $X$ except for $d_0$. An almost free object is cofibrant, and every cofibrant object is a retract of an almost free object \cite[\S3]{MillerSullivanConjecture.pdf}.

A \emph{cofibrant replacement functor} for $s\calc$  is an endofunctor $f$ of $s\calc$ whose image consists only of cofibrant objects equipped with  a natural acyclic fibration $\epsilon:f\Rightarrow \Id $. One classical way to define such a functor is as follows.

%For a simplicial resolution functor $\calc\to s\calc$, we will often use 
Consider the standard simplicial bar construction $\calc\to s\calc$ arising from the $F^\calc\dashv U^\calc$ adjunction \cite{BlumRiehlResolutions.pdf}. More explicitly, given $X\in\calc$, we define $B^\calc X\in s\calc$ by iterated application of the comonad $F^\calc$ to $X\in \calc$:
\[B_s^\calc X=(F^\calc)^{s+1}X.\]
The face maps are given by the formula $d_i=(F^\calc)^i\rho$, and the degeneracies by $s_i=(F^\calc)^i\Delta$. %, where $\rho$ and $\Delta$ are respectively the counit and diagonal of the comonad $F^\calc U^\calc$. 
This object is almost free, with $B_s^\calc X$ generated by its subspace $V_s=(F^\calc )^sX$, moreover, it is standard \cite[\S4]{BlumRiehlResolutions.pdf} that the augmentation $B^\calc X\to X$ is an acyclic fibration.

The construction generalizes to yield a cofibrant replacement functor $B^{\calc}$ on $s\calc$: one applies  $B^{\calc}:\calc\to s\calc$ levelwise to obtain a bisimplicial object, and then takes the diagonal. A standard spectral sequence argument shows that this is a cofibrant replacement functor.

\subsection{Categories of graded $\Ftwo $-vector spaces and linear dualization}
In order that we can be more specific about the types of gradings we have in mind, in this section we introduce  notation for the key categories of graded vector spaces. We will still somethimes write $\vect{}{}$ for a generic category of graded vector spaces, or for the category of ungraded vector spaces, as necessary.
%We write $\vect{}{}$ for the category of ungraded vector spaces.
%That said, when we are introducing general notions, we will use the symbol $\vect{}{}$ to refer either to this category or to one of the categories of graded vector spaces which we now introduce.   %We will write $\vect{}{}$ for a generic category of $\Ftwo $-vector spaces, either graded or ungraded. 

Write $\vect{q}{r}$ for the category of vector spaces with $r$ non-negative homological gradings and $q$ non-negative cohomological gradings, so that an object $V$ of $\vect{q}{r}$ decomposes as
\[V=\bigoplus_{s_r,\ldots,s_1,t_q,\ldots,t_1\geq0}V^{t_q,\ldots,t_1}_{s_r,\ldots,s_1}.\]
The category $\vect{q}{r}$ is equipped with a tensor product:
\[(U\otimes V)^{t_q,\ldots,t_1}_{s_r,\ldots,s_1}=\bigoplus_{s'_i+s''_i=s_i,\,t_j'+t_j''=t_j}U^{t'_q,\ldots,t'_1}_{s'_r,\ldots,s'_1}\otimes V^{t''_q,\ldots,t''_1}_{s''_r,\ldots,s''_1}.\]

Often in this paper we will discuss maps between graded vector spaces which do not preserve degrees. %For a near-example from classical topology, take the whitehead product \[\pi_{t_1}X\times\pi_{t_2}X\to \pi_{t_1+t_2-1}.\] Writing $\pi_{\geq2}X$ for the graded abelian group $\bigoplus_{i\geq 2}\pi_iX$, this product produces a grading-preserving map $(\pi_{\geq2}X)^{\otimes2}\to\Sigma\pi_{\geq2}X$, for $\Sigma$ the evident suspension operator on homologically graded abelian groups.
Although we could encode such maps as grading-preserving maps between appropriate suspensions, we will systematically avoid being so systematic. For example, we will often write $V\otimes V\to V$ for a map which in fact adds one to certain gradings of $V$, and will avoid confusion by explicitly stating the effect of such a map on degrees.

We will often need to consider the linear dual of a vector space $V$, and the standard symbol, $V^*$, will cause ambiguity. Indeed, we will already be using superscripts intensively, so we opt for a modifier written prefix, defining the dualization functor
$\dual:(\vect{q}{r})^\textup{op}\to\vect{r}{q}$ by:
\[(\dual V)_{t_q,\ldots,t_1}^{s_r,\ldots,s_1}:=\hom(V^{t_q,\ldots,t_1}_{s_r,\ldots,s_1},\Ftwo ).\]
We will shortly define cohomology functors
$H_{\calc}^*X:=\dual H^{\calc}_*X$, and we will use the position of the asterisk to demonstrate which of homology and cohomology we mean: one of topology's longest traditions. This is not precisely an exception to our convention, but was worth mentioning.

Often, the vector spaces we are interested in will support an \emph{extra} grading, the \emph{quadratic grading}, so called because certain operations derived from an underlying squaring operation tend to double this extra grading. We do not think of the quadratic grading as either homological or cohomological, so we write it prefix:
\[V=\textstyle\bigoplus_{k\geq1}q_kV.\]


A common pattern for us will be to consider vector spaces with $r$ non-negative homological gradings and a single \emph{strictly positive} cohomological grading:
\[V=\bigoplus_{s_r,\ldots,s_1\geq0,\,t\geq 1}V^{t}_{s_r,\ldots,s_1},\]
and we write $\vect{+}{r}$ for the category of such objects. Similarly, there is a category $\vect{r}{+}$, and  dualization is a functor $\dual:(\vect{+}{r})^\textup{op}\to\vect{r}{+}$.


\subsection{The Dold-Kan correspondence}\label{The Dold-Kan correspondence}
In this paper we will use the following four chain complexes in $\complexes \vect{+}{r}$ associated with a simplicial graded vector-space $V\in s\vect{+}{r}$:
\begin{alignat*}{2}
C_nV&:=V_n&& \textup{ with differential }d=\textstyle\sum_{i=0}^{n}d_i;\\
N_nV&:=\textstyle\bigcap_{0< i\leq n}\ker(d_i:V_n\to V_{n-1})&& \textup{ with differential }d=d_0;\\
\Nop_nV&:=\textstyle\bigcap_{0\leq i< n}\ker(d_i:V_n\to V_{n-1})&& \textup{ with differential }d=d_n;\\
N_n^\div V&:=V_n/(\textup{degenerate $n$-simplices})&& \textup{ with differential }d=\textstyle\sum_{i=0}^{n}d_i.
\end{alignat*}
There are evident inclusions of $N_*V$ and $\Nop_*V$ into $C_*V$, and a projection of $N_*V$ onto $N_*^\div V$, and all of these maps are weak equivalences. Moreover, the composite $N_*V\to N_*^\div V$ is an isomorphism (as is the composite from $\Nop_*V$). It will be helpful to have an explicit formula for the composite
\[C_*V\to N_n^\div V \overset{\cong}{\to}N_nV. \]
\begin{lem}\label{the map nml}
The \emph{normalization} map
\[\textup{nml}=(1+s_0d_1)(1+s_1d_2)\cdots (1+s_{n-1}d_n):V_n\to V_n\]
is an idempotent chain complex endomorphism with image $N_*V$ and kernel the degenerate $n$-simplices of $V$, so that there is a commuting diagram
\[\xymatrix@R=4mm{
N_nV\ar@{ >->}[r]\ar@{=}[d]&%r1c1
C_nV\ar[dl]_-{\textup{nml}}
\ar@{->>}[dr]
&%r1c2
\\%r1c3
N_nV\ar@{ >->}[r]
\ar@/_1em/[rr]|-{\cong}
&%r2c1
C_nV\ar@{->>}[r]&%r2c2
C^\div_nV%r2c3
}\]
\end{lem}
\begin{proof}
It is obvious that $\textup{nml}$ restricts to the identity on $N_nV$, and that $\textup{nml}-\Id$ has image consisting of degenerate simplices. By the simplicial identities, for $1\leq i\leq n$:
\[d_i(1+s_{i-1}d_i)=d_i+d_is_{i-1}d_i=d_i+\Id d_i=0.\]
As for $1\leq j<i$, we also have
\[d_i(1+s_{j-1}d_j)%=d_i+d_is_{j-1}d_j
=d_i+s_{j-1}d_{i-1}d_j=(1+s_{j-1}d_j)d_i,\]
this proves that $d_i\circ\textup{nml}=0$ for $1\leq i\leq n$, or that  $\textup{nml}$ has image inside $N_n V$. Thus $\textup{nml}$ is an idempotent with image $N_nV$. As $N_nV\to  C^\div_nV$ is as isomorphism, the rest is easy.
\end{proof}


Each $N_nV$ retains the internal gradings of $V$, and the functor $N_*$ appears in the \emph{Dold-Kan correspondence} \cite[\S III.2]{goerss-jardine.pdf}, an adjoint equivalence of categories:
\[N:s\vect{+}{r}\rightleftarrows \complexes \vect{+}{r}:\Gamma,\]
under which the homotopy groups of $V\in s\vect{+}{r}$ (as a simplicial set)  are isomorphic to the homology groups of $N_*V$.
%
%As such, we may define the homotopy groups of $V$, written $\pi_*V\in \vect{+}{r+1}$, to be $H_*N_*V$, where $N_*$ is one of the inverse equivalences appearing in the Dold-Kan correspondence [somewhere]: \[N:s\vect{+}{r}\rightleftarrows \complexes \vect{+}{r}:\Gamma.\]
%There are three equivalent definitions of $N_*V$. We will almost always use the description of $N_nV$ as the mutual kernel of $d_1,\ldots,d_n$ in $V_n$, with differential $d_0$ (rather than the equivalent definition as a quotient of the unnormalized complex $C_nV=V_n$), and write $ZN_nV$ for the mutual kernel of $d_0,\ldots,d_n$. 
A cycle in $N_nV$ is an element $x\in V_n$ such that $d_ix=0$ for $0\leq i\leq n$. We write $ZN_nV$ for this group of cycles, and sometimes write $\overline{x}$ for the equivalence class of $x$ in $\pi_*V$.

It will be helpful to end the notational distinction between the chain complex dimension $n$ and the other homological dimensions $s_r,\ldots,s_1$. That is, we will view $\pi_*V$ as a single object of $\vect{+}{r+1}$, defined by
\[(\pi_*V)^t_{s_{r+1},\ldots,s_1}:=(\pi_{s_{r+1}}V)^t_{s_{r},\ldots,s_1}.\]
Now for any collection of indices $s_{r+1},\ldots,s_1\geq0$ and $t\geq1$, define:
%Let $\mathbb{K}_{n,s_r,\ldots,s_1}^t$ and $C\mathbb{K}_{n,s_r,\ldots,s_1}^t$ be the $\in \complexes \vect{+}{r}$ be the chain complexes, with $z$, $h$ and $dh$ all having internal gradings $t,s_r,\ldots,s_1$:
%%%%%%\begin{gather*}
%%%%%%\phantom{C\mathbb{K}_{n,s_r,\ldots,s_1}^t=\Gamma\Bigl(}\xymatrix@R=1.5mm@1@!C{
%%%%%%&{\smash{(n-1)}}&{\smash{(n)}}&{\smash{(n+1)}}&{\smash{(n+2)}}\\
%%%%%%{\makebox[0cm][r]{$\,\mathbb{K}_{n,s_r,\ldots,s_1}^t=\Gamma\Bigl($}\cdots\,} &%r1c1
%%%%%%{\,0\,}\ar[l]
%%%%%%&%r1c2
%%%%%%\Ftwo \langle z\rangle\ar[l]
%%%%%%&%r1c3
%%%%%%{\,0\,}\ar[l]
%%%%%%&{\,0\,}\ar[l]
%%%%%%&{\,\cdots\Bigr),} \ar[l]
%%%%%%\\
%%%%%%%&{\smash{(n-1)}}&{\smash{(n)}} &{\smash{(n+1)}}&{\smash{(n+2)}}\\
%%%%%%{\makebox[0cm][r]{$\,C\mathbb{K}_{n,s_r,\ldots,s_1}^t=\Gamma\Bigl($}\cdots\,} &%r1c1
%%%%%%{\,0\,}\ar[l]
%%%%%%&%r1c2
%%%%%%\Ftwo \langle dh\rangle\ar[l]
%%%%%%&%r1c3
%%%%%%\Ftwo \langle h\rangle\ar[l]
%%%%%%&{\,0\,}\ar[l]
%%%%%%&{\,\cdots\Bigr).} \ar[l]
%%%%%%}
%%%%%%\end{gather*}
\begin{gather*}
\phantom{C\mathbb{K}_{n,s_r,\ldots,s_1}^t=\Gamma\Bigl(}\xymatrix@R=0mm@1@!C{
{\makebox[0cm][r]{$\,\mathbb{K}_{n,s_r,\ldots,s_1}^t=\Gamma\Bigl($}\cdots\,} &%r1c1
{\,0\,}\ar[l]
&%r1c2
\Ftwo \langle z\rangle\ar[l]
&%r1c3
{\,0\,}\ar[l]
&{\,0\,}\ar[l]
&{\,\cdots\Bigr),} \ar[l]
\\
&\raisebox{.4mm}{\tiny\makebox[0cm][r]{degrees:\ \ \ \ \ }$\smash{s_{r+1}\!-\!1}$}
&\raisebox{.4mm}{\tiny$\smash{s_{r+1}  }$}
&\raisebox{.4mm}{\tiny$\smash{s_{r+1}\!+\!1}$}
&\raisebox{.4mm}{\tiny$\smash{s_{r+1}\!+\!2}$}\\
%&{\smash{(n)}}&{\smash{(n+1)}}&{\smash{(n+2)}}\\
%&{\smash{(n-1)}}&{\smash{(n)}} &{\smash{(n+1)}}&{\smash{(n+2)}}\\
{\makebox[0cm][r]{$\,C\mathbb{K}_{n,s_r,\ldots,s_1}^t=\Gamma\Bigl($}\cdots\,} &%r1c1
{\,0\,}\ar[l]
&%r1c2
\Ftwo \langle dh\rangle\ar[l]
&%r1c3
\Ftwo \langle h\rangle\ar[l]
&{\,0\,}\ar[l]
&{\,\cdots\Bigr).} \ar[l]
}
\end{gather*}
Here $z$ and $h$ denote are both to lie in internal cohomological grading $t$ and homological gradings $s_r,\ldots,s_1$.
There is an evident inclusion $\imath:\mathbb{K}_{s_{r+1},s_r,\ldots,s_1}^t\to C\mathbb{K}_{s_{r+1},s_r,\ldots,s_1}^t$. For any $V\in s\vect{+}{r}$, we can identify the subspaces of cycles and boundaries with hom-sets:
\begin{alignat*}{2}
\hom_{s\vect{+}{r}}(\mathbb{K}^t_{s_{r+1},\ldots,s_1},V)&\cong (ZN_{s_{r+1}}V)_{s_r,\ldots,s_1}^t\makebox[0cm][l]{ and}\\
\hom_{s\vect{+}{r}}(C\mathbb{K}^t_{s_{r+1},\ldots,s_1},V)&\cong (N_{s_{r+1}+1}V)_{s_r,\ldots,s_1}^t.
\end{alignat*}
%\[\hom_{s\vect{+}{r}}(\mathbb{K}^t_{s_r,\ldots,s_1},V)\cong (ZN_{s_{r+1}}V)_{s_r,\ldots,s_1}^t\textup{ and }\hom_{s\vect{+}{r}}(C\mathbb{K}^t_{s_{r+1},\ldots,s_1},V)\cong (N_{s_{r+1}+1}V)_{s_r,\ldots,s_1}^t,\]
Under these isomorphisms the chain complex differential $N_{s_{r+1}+1}V\to ZN_{s_{r+1}}V$ corresponds to $\imath^*$. In fact, $\mathbb{K}^t_{s_{r+1},\ldots,s_1}$ represents $\pi_*(\DASH)t_{s_{r+1},\ldots,s_1}$ in the homotopy category of $s\vect{+}{r}$: in a category of simplicial vector spaces, there is no distinction between spheres and Eilenberg-Mac Lane spaces.

A dual theory exists for cosimplicial vector spaces $U$. We will mention the cochain complexes
\begin{alignat*}{2}
C^nU&:=U^n&& \textup{ with differential }d=\textstyle\sum_{i=0}^{n+1}d^i;\\
N^nU&:=U^n/\textstyle\sum_{0< i\leq n}\im(d^i:U_{n-1}\to U_{n})&& \textup{ with differential }d=d^0;\\
%\Nop_nX&:=\textstyle\bigcap_{0\leq i< n}\ker(d_i:X_n\to X_{n-1})&& \textup{ with differential }d=d_n;\\
N^n_\subset U&:=\textstyle\bigcap_{0\leq i\leq n-1}\ker(s^i:U_n\to U_{n-1})&& \textup{ with differential }d=\textstyle\sum_{i=0}^{n+1}d^i.
\end{alignat*}
There are chain complex maps whose leftward composite is an isomorphism,:
\[N^nU  \from C^nU\from N^n_\subset U,\]
and an explicit normalization map 
\[\textup{nml}=(1+d^ns^{n-1})\cdots (1+d^2s^1)(1+d^1s^0):C^nU\to N^n_\subset U\]
with properties dual to the simplicial version. The cohomology of any of these homotopy equivalent cochain complexes defines the \emph{cohomotopy} $\pi^*U$ of $U$.

\subsection{Dold's theorem}
According to Dold \cite{DoldHomologySPs.pdf}, see also \cite[Lemma 3.1]{ChingUnpublished}: 
\begin{thm}[Dold's theorem]\label{Dold's theorem}
Suppose that $F:s\vect{+}{r}\to s\vect{+}{r}$ is a functor preserving weak equivalences, for example, the prolongation of an endofunctor of $\vect{+}{r}$. Then there is a functor $\calF:\vect{+}{r+1}\to\vect{+}{r+1}$ such that the following diagram commutes:
\[\xymatrix@R=4mm{
s\vect{+}{r}\ar[r]^-{F}
\ar[d]^-{\pi_*}
&%r1c1
s\vect{+}{r}\ar[d]^-{\pi_*}
\\%r1c2
\vect{+}{r+1}\ar[r]^-{\calF}
&
\vect{+}{r+1}
}\]
Moreover, if $F\cong F_2\circ F_1$, then $\calF\cong \calF_2\circ \calF_1$.
\end{thm}
\noindent The idea here is that the functor $\pi_*$ induces an equivalence between the homotopy category of $s\vect{+}{r}$ and $\vect{+}{r+1}$. In fact, the inverse equivalence can be lifted to a functor into $s\vect{+}{r}$, namely 
\[V\mapsto \Gamma V,\ \ \vect{+}{r+1}\to s\vect{+}{r},\]
where we view $V$ as a trivial chain complex. Then $\calF$ can be constructed as $\calF V:=\pi_*(F\Gamma V)$.
%given that the homotopy category of simplicial vector spaces is graded vector spaces, the derived functors of a weak equivalence preserving functor $F:s\calV\to s\calV$ are well defined, and there is a commuting diagram

\subsection{Comparison of cohomotopy and dual homotopy}
Suppose that $V\in s\vect{}{}$. Then $C^*\dual V=\dual C_*V$, and there is a natural isomorphism
\[H^*C^*\dual V=H^*\dual C_*V\to\dual H_* C_*V,\ \ \overline{\alpha}\mapsto\textup{``$\overline{v}\mapsto\alpha(v)$''}.\]
%where the second map sends the class of a functional $\alpha:C_nV\to\Ftwo $ to the functional on $H_*C_*V$ given by pairing representatives with $\alpha$.

\subsection{The $\calc$-homology and $\calc$-cohomology functors $H^{\calc}_*$ and $H_{\calc}^*$}



In this thesis we will always define the $\calc$-\emph{homology} of $X\in s\calc$ by the formula:
\[H_*^{\calc}X:=\pi_*(Q^\calc B^\calc X)=H_*N_*(Q^\calc B^\calc X).\]
These homology functors are well defined, as the $Q^\calc\dashv K^\calc$ adjunction is a Quillen adjunction (that $K^\calc$ preserves fibrations and acyclic fibrations is immediate), and indeed we are free to use any cofibrant replacement in place of $B^\calc X$.

It is not always entirely appropriate to call these functors homology. Indeed, Andr\'e-Quillen homology, as defined by Quillen, is the left derived functors of the abelianization functor of \cite[\S II.5]{QuillenHomAlg.pdf}, and it is not true in all of our examples that $Q^\calc$ models the abelianization functor. Goerss \cite[\S4]{MR1089001} explains that this does occur when $\calc$ is the category of non-unital commutative algebras, but it does not occur when $\calc$ is the category of restricted Lie algebras.

When $\calc$ is monadic over $\vect{+}{r}$, we may view the groups $H_*^{\calc}X$ together as an object of $\vect{+}{r+1}$. That is, each homology group $H_s^\calc X$ retains the gradings of $X$, and a new homological grading is added (to the left of the existing homological gradings). We will sometimes avoid substituting into the asterisk, writing expressions such as $(H_*^\calc X)_{s_{r+1},\ldots,s_1}^t$ in place of $(H_{s_{r+1}}^\calc X)_{s_r,\ldots,s_1}^t$.

We define the \emph{$\calc$-cohomology} $H^*_\calc X$ of $X$ to be $\dual(H_*^\calc X)$, or equivalently the cohomotopy groups $\pi^*\dual(Q^\calc X)$ of the dual cosimplicial object.  As we dualize to obtain cohomology, the cohomological gradings and homological gradings are swapped, and $H^*_{\calc}X$ may be viewed as an object of $\vect{r+1}{+}$. 


\begin{lem}\label{lemma on homology class repd by normalized generator}
Suppose that $X\in s\calc$ is almost free with generating subspaces $V_n\subseteq X_n$. Then any homology class in $H_n^{\calc}X\cong \pi_nQ^{\calc}X$ can be represented by the image in $Q^{\calc}X_n$ of an element of $V_n\cap N_nX$.
\end{lem}
\begin{proof}
Consider the maps $\textup{nml}:C_nX\to N_nX$ and $\textup{nml}:C_nQ^\calc X\to N_nQ^\calc X$ of lemma \ref{the map nml}.
Choose a representative $w\in ZN_nQ^\calc X$. As the composite $V_n\subseteq X_n\epi Q^\calc X_n$ is an isomorphism, one can find some $v\in V_n$ such that the image of $v$ in $Q^\calc X_n$ is $w$. Although $v$ need not lie in $N_nX$, $\textup{nml}(v)$ must lie in $N_nX$, and maps to $\textup{nml}(w)\in ZN_nQ^\calc X$ under the quotient map.  However, as $w\in ZN_nQ^\calc X$ and $\textup{nml}$ acts as the identity on this subspace, $w=\textup{nml}(w)$, so $v$ is the element we seek.
\end{proof}
The previous result showed that we may find representatives for any homology class in the subobject $V_n\cap N_nX$, while for other applications it will be preferable simply to pass to the quotient: %We will sometimes wish to work directly with the generating subspaces when working with (co)-homology, and will then apply the following easy observation:
\begin{lem}\label{identify almost free indecs with gens}
Suppose that $X$ is almost free with generating subspaces $V_n\subseteq X_n$. Then the simplicial object $(Q^{\calc}X)_{n}$ may be identified with the collection of vector spaces $\{V_n\}$, using  the composite
\[V_{n}\overset{d_0}{\to}X_n\cong F^{\calc}V_{n-1}\overset{\epsilon}{\to}V_{n-1}\]
as the zeroth face map of $\{V_n\}$, and using the other structure maps of $X$, which preserve the $V_{n}$, as the other structure maps of $\{V_n\}$.
\end{lem}



\subsection{The action of $\Sigma_2$ on $V^{\otimes 2}$}
For any vector space $V\in \vect{}{}$, the tensor power $V^{\otimes2}:=V\otimes V$ has an action of $\Sigma_2$ given by the map $T$ interchanging the two factors. We will write $S_2V$ for the coinvariants and $S^2V$ for the invariants of this action, and $\Lambda^2V$ for the image in $S^2V$ of the \emph{trace map} $\trace=(1+T):S_2V\to S^2V$. Thus, we have written $\Lambda^2V$ as a \emph{subobject} of $S^2V$.
On the other hand, one may view $\Lambda^2V$ as $S_2V/\ker(\trace)$,  the \emph{quotient} of $S_2V$ by the subspace generated by elements of the form $v\otimes v$. %It is often convenient to define maps out of $\Lambda^2V$ by viewing it as a subobject of $S^2V$, and maps into $\Lambda^2V$ by viewing it as a quotient of $S_2V$ (itself a quotient of $V^{\otimes2}$).

For any $V\in \vect{}{}$ there is a natural map
\[S_2\dual V\to \dual S^2V,\ \ \alpha\otimes \beta\mapsto\textup{``$v\otimes w\mapsto \alpha(v)\beta(w)$''}.\]
It is an isomorphism when $V$ is finite-dimensional.

Suppose that $V$ and $W$ are $\Ftwo $-vector spaces, and $p:S_2V\to W$ is a linear map. A \emph{quadratic refinement of $p$} is a function $\sigma:V\to W$ satisfying, for $v_1,v_2\in V$ and $\alpha\in\Ftwo $:
\[\sigma(v_1+v_2)=\sigma(v_1)+\sigma(v_2)+p(v_1\otimes v_2)\textup{ and }\sigma(\alpha v_1)=\alpha^2\sigma(v_1).\]
In fact, the second condition is redundant (over $\Ftwo $), and these conditions are equivalent to the following condition. For any set $B$, define $\Lambda^2B$ to be the set of subsets of $B$ of cardinality exactly two. 
The equivalent condition is that, for every collection of vectors $v_b\in V$ and of coefficients $\alpha_b\in \Ftwo $ indexed by a set $B$, in which all but finitely many of the $\alpha_b$ are zero, the equation
\[\sigma\Bigl(\sum_{b\in B}\alpha_bv_b\Bigr)=\sum_{b\in B}\alpha_b^2\sigma(v_b)+\sum_{\{b,c\}\in \Lambda^2B}\alpha_b\alpha_cp(v_b\otimes v_c)\]
holds.
%\[v\otimes v
%=
%\sum_{b\in B}\alpha_b^2v_b+\sum_{\{b,c\}\in \Lambda^2B}\alpha_b\alpha_c\trace(v_b\otimes v_c)\]
%
%
%
%
%\begin{alignat*}{2}
%v\otimes v
%&=
%\sum_{b\in B}\alpha_b^2v_b+\sum_{\{b,c\}\in \Lambda^2B}\alpha_b\alpha_c\trace(v_b\otimes v_c)%
%\\
%f(v\otimes v)&:=
%\sum_{b\in B}\alpha_b^2\sigma(v_b)+\sum_{\{b,c\}\in \Lambda^2B}\alpha_b\alpha_cp(v_b\otimes v_c)%
%\\
%% Left hand side
%% Relation
%&=
%% Right hand side
%\textstyle\sum_{i<j}\alpha_i\alpha_jp(v_i\otimes v_j)+\textstyle\sum_{i}\alpha_i^2\sigma(v_i)%
%►% Comment
%&\qquad&\text{(ᾮ)}©
%\end{alignat*}
%
If $f:S^2V\to W$ is a linear map, the function
$v\mapsto f(v\otimes v)$
is a quadratic refinement of $\trace\circ f$, and indeed:
\begin{prop}\label{propOnExtendingToInvariants}
For any linear map $p:S_2V\to W$, extensions of $p$ to a linear map $f:S^2V\to W$ are in natural bijection  with quadratic refinements of $p$.
\end{prop}
\begin{proof}
%It is enough to prove this result for finite-dimensional vector spaces $V$, as any vector space $V$ is the filtered colimit of its finite-dimensional subspaces, and the functors $S_2$ and $S^2$ commute with filtered colimits.
Suppose that $V$ has basis $\{v_b\ |\ b\in B\}$. Then $S^2V$ has basis the set 
\[\{\trace(v_b\otimes v_c)\ |\ \{b,c\}\in\Lambda^2B\}\cup\{v_b\otimes v_b\ |\ b\in B\}.\]
This is easy to check for $V$ finite dimensional, and extends to the infinite dimensional case as $S^2$ preserves filtered colimits, and we may calculate $V$ as the colimit
\[\colim_{B'\subset B}\Ftwo \langle B'\rangle=V.\]
In particular, an extension $f$ of $p$ is determined by the quadratic refinement $v\mapsto f(v\otimes v)$. Thus, as long as we can produce an extension $f$ with $\sigma(v)=f(v\otimes v)$ for any quadratic refinement $\sigma$ of $p$, we will have the natural construction we need.

What remains to prove is that the linear map $f$ defined on this basis by
\[\trace(v_b\otimes v_c)\mapsto p(v_b\otimes v_c),\quad v_b\otimes v_b\mapsto \sigma(v_b)\]
does in fact have the property that $f(v\otimes v)=\sigma(v)$ for \textup{all} $v\in V$. Indeed, if we write $v$ in terms of the chosen basis as $v=\sum_{b\in B}\alpha_bv_b$, then
\[v\otimes v
=
\sum_{b\in B}\alpha_b^2v_b+\sum_{\{b,c\}\in \Lambda^2B}\alpha_b\alpha_c\trace(v_b\otimes v_c),\]
and we can apply our definition of the linear map $f$ to this expansion directly, obtaining
\[f(v\otimes v):=
\sum_{b\in B}\alpha_b^2\sigma(v_b)+\sum_{\{b,c\}\in \Lambda^2B}\alpha_b\alpha_cp(v_b\otimes v_c)=\sigma(v).\qedhere\]
\end{proof}
\begin{cor}
There is a natural linear map $\sqrt{\DASH}:S^2V\to V$, the \emph{square root map}, uniquely determined by the requirements:
\[\sqrt{v_1\otimes v_2+v_2\otimes v_1}=0,\ \ \ \sqrt{v\otimes v}=v \textup{ \ for all $v_1,v_2,v\in V$}.\]
\end{cor}
\begin{proof}
This map is the unique extension of $0:S_2V\to V$ corresponding to the quadratic refinement $\Id:V\to V$ of $0$.
\end{proof}
\noindent The evocative square root symbol is doubly approriate, as if $V$ is dual to a finite-dimensional vector space $U\in\vect{}{}$, the linear dual of the square root map,
\[\dual V\to \dual S^2V\overset{\cong }{\from}S_2\dual V\]
equals the \emph{squaring map} $U\to S_2U$, defined by $u\mapsto u\otimes u$.


\subsection{Lie algebras}
As we work in characteristic 2, there is more than one available notion of a Lie algebra. An \emph{$S(\LieOperad)$-algebra} is a vector space $L$ equipped with a bracket $L\otimes L\to L$ satisfying the Jacobi identity and the (anti)-symmetry condition $[x,y]=[y,x]$. A \emph{Lie algebra} (or $\Lambda(\LieOperad)$-algebra) is a vector space $L$ equipped with a bracket $L\otimes L\to L$ satisfying the Jacobi identity and the alternating condition $[x,x]=0$. Finally, a \emph{restricted Lie algebra} \cite{CurtisSimplicialHtpy.pdf,6Author.pdf} (or $\Gamma(\LieOperad)$-algebra) is a Lie algebra equipped with a \emph{squaring} or \emph{restriction} function $\restn{(\DASH)}:L\to L$, satisfying the axioms
\[\restn{(x_1+x_2)}=\restn{x_1}+\restn{x_2}+[x_1,x_2]\textup{ \ and \ }[\restn{x_1},x_2]=[x_1,[x_1,x_2]].\]
The alternating condition implies the (anti)-symmetry condition, and these three types of Lie algebras form a hierarchy: a restricted Lie algebra is in particular a Lie algebra, and a Lie algebra is in particular an $S(\LieOperad)$-algebra.

We will  write $\liealgs$ for the category of ungraded Lie algebras, and $\restliealgs$ for the category of ungraded restricted Lie algebras.

Fresse \cite{FresseSimplicialAlgs.pdf} explains how to construct the monads $S(\LieOperad)$, $\Lambda(\LieOperad)$ and $\Gamma(\LieOperad)$ on $\vect{}{}$ which give rise to these structures, starting with the Lie operad $\LieOperad$. For $V\in\vect{}{}$, it is standard that the functor
\[S(\LieOperad):V\mapsto \bigoplus_{n\geq1}(\LieOperad(n)\otimes V^{\otimes n})_{\Sigma_n}\]
inherits the structure of a monad from the composition maps of $\LieOperad$. Fresse observes that the functor
\[\Gamma(\LieOperad):V\mapsto \bigoplus_{n\geq1}(\LieOperad(n)\otimes V^{\otimes n})^{\Sigma_n}\]
may also be equipped with a monad structure, such that the trace map $S(\LieOperad)\to \Gamma(\LieOperad)$
%\[\trace:S(\LieOperad)(V)\to \Gamma(\LieOperad)(V)\]
is a map of monads, and an intermediate monad may be defined by
\[\Lambda(\LieOperad):V\mapsto\im\bigl(\trace:S(\LieOperad)(V)\to \Gamma(\LieOperad)(V)\bigr).\]
These monads give rise to the three indicated forms of Lie algebras in characteristic 2. Now each of these functors supports a \emph{quadratic grading}:
\[\quadgrad{k}(\Gamma(\LieOperad)V):=(\LieOperad(k)\otimes V^{\otimes k})^{\Sigma_k},\]
and similarly for the other two monads.

 Now $\LieOperad(2)\cong\Ftwo $, so that
\[\quadgrad{2}(S(\LieOperad)V)\cong S_2V,\quad \quadgrad{2}(\Lambda(\LieOperad)V)\cong \Lambda^2V,\quad\textup{and}\quad\quadgrad{2}(\Gamma(\LieOperad)V)\cong S^2V.\]
%Each of these three monads is generated by a single binary operation, contained in quadratic grading 2. 
Moreover, one can view an $S(\LieOperad)$-algebra as a map $S_2L\to L$, a $\Lambda(\LieOperad)$-algebra as a map $\Lambda^2L\to L$, and a $\Gamma(\LieOperad)$-algebra as a map $S^2L\to L$, where in each case we demand the necessary compatibilities that these maps extend to the full monad. %Here, we used the analogous definitions $S_2V:=(V^{\otimes2})_{\Sigma_2}$, $S^2V:=(V^{\otimes2})_{\Sigma_2}$ and $\Lambda^2V:=\im(\trace:S_2V\to S^2V)$.
By pulling back along the natural maps
\[S_2V\to \Lambda^2 V\to S^2V\]
one can demote a restricted Lie algebra to a Lie algebra, or a Lie algebra to an $S(\LieOperad)$-algebra.

A \emph{restictable ideal} in a Lie algebra $L$ is a Lie ideal $I$ of $L$, equipped with a restriction function $\restn{(\DASH)}:I\to I$, satisfying the following axioms, for $x_1,x_2\in I$ and $x_3\in L$:
\[\restn{(x_1+x_2)}=\restn{x_1}+\restn{x_2}+[x_1,x_2]\textup{ \ and \ }[\restn{x_1},x_3]=[x_1,[x_1,x_3]].\]
In fact, let $\mathsf{PRL}$ denote the category of \emph{partially restricted Lie algebras}, whose objects are pairs of vector spaces $(L_+,L_0)$, equipped with a Lie algebra structure on $L_+\oplus L_0$ in which $L_+$ is a restrictable ideal, and whose maps are Lie algebra maps preserving the decomposition and commuting with the partial restrictions. This category is monadic over $\vect{}{}\times\vect{}{}$, the category of pairs of vector spaces, and the value of monad $F^{\mathsf{PRL}}$  on $(V_+,V_0)$ is just an appropriately chosen subalgebra of $\Gamma(\LieOperad)(V_+\oplus V_0)$. We will refer to homogeneous elements of $L_+$ as \emph{restrictable}, and homogeneous elements of $L_0$ as \emph{non-restrictable}.

In \S\ref{The categories Ln} we will define various caregories of graded partially restricted Lie algebras, where restrictability is based on the non-vanishing of certain gradings.


\subsection{Non-unital commutative algebras}
In this paper we will always work with \emph{non-unital} commutative algebras, unless we specify otherwise (\textbf{do we ever?}). As for Lie algebras, there are three different notions of non-unital commutative algebra available in characteristic 2. A \emph{commutative algebra} (or $S(\CommOperad)$-algebra) is a vector space $A$ equipped with an associative commutative pairing $A\otimes A\to A$.  We will work with these often in this paper, and will write $\algs$ for the category of such algebras. In fact, we will so often discuss simplicial non-unital commutative algebras that we will refer to them simply as \emph{simplicial algebras}.

An \emph{exterior algebra} (or $\Lambda(\CommOperad)$-algebra) is a commutative algebra $A$ with the property that $x^2=0$ for all $x\in A$. A \emph{divided power algebra} (or $\Gamma(\CommOperad)$-algebra) is a commutative algebra $A$ equipped with \emph{divided power} operations, as described in \cite[1.2.2]{FresseSimplicialAlgs.pdf} or \cite[\S2]{MR1089001}. In characteristic 2, these operations are all determined by a single operation, the \emph{divided square} $\gamma_2:A\to A$, which satisfies
\[\gamma_2(xy)=x^2\gamma_2(y),\ \gamma_2(\lambda x)=\lambda^2\gamma_2(x)\textup{ and }\gamma_2(x+y)=\gamma_2(x)+\gamma_2(y)+xy.\]
Note that the second condition is in fact extraneous over $\Ftwo $, and that the last condition implies that a divided power algebra is exterior. Thus, $\gamma_2(xy)=x^2\gamma_2(y)=0$.

There is a notion of a \emph{divided power ideal} of a commutative algebra: an ideal $I$ of a commutative algebra $A$ equipped with a compatible divided power structure on $I$. In this case, $I$ is necessarily exterior, although for $x\in A$ and $y\in L$, $\gamma_2(xy)=x^2\gamma_2(y)$ need not be zero.

Again, Fresse \cite{FresseSimplicialAlgs.pdf} explains how to construct the monads $F^{\algs}:=S(\CommOperad)$, $\Lambda(\CommOperad)$ and $\Gamma(\CommOperad)$ on $\vect{}{}$ which give rise to these structures, using the commutative operad $\CommOperad$ instead of $\LieOperad$. Again, there is a quadratic grading definable on these three monads, and each monad is generated in degre 2, so that a commutative algebra may be thought of as a  map $S_2L\to L$, an exterior algebra as a map $\Lambda^2L\to L$, and a divided power algebra as a map $S^2L\to L$.



















\subsection{First quadrant cohomotopy spectral sequences}
Suppose that $V_{p,q}$ is a bisimplicial vector space, ungraded for now. We will follow the standard conventions, those of \cite{MR2245560}, in defining the cohomotopy spectral sequence of $V$. For  more detail, see \cite[\S1.15{}]{MR2245560}.   

There is a double chain complex $C_{p,q}V:=C\uhor_p C\uver_q V=V_{p,q}$, where we have decorated the $C$ functors in order to distinguish them from the functor $C_{**}$ being introduced, and to distinguish the coordinates --- we will always refer to $p$ as the \emph{horizontal} coordinate and $q$ as the \emph{vertical} coordinate. The total complex $TV$, along with a canonical increasing filtration, is defined by 
\[(TV)_n:=\bigoplus_{i=0}^{n}C_{i,n-i}V,\qquad F_p(TV)_n:=\bigoplus_{i=0}^{p}C_{i,n-i}V.\]
The dual total complex $\dual TV$ admits a decreasing filtration defined by
\[F^p\dual TV:=\ker\bigl(\dual TV\epi \dual F_{p-1} TV\bigr).\]
Correspondingly, $H^*(\dual TV)\cong \pi^*(\dual\diag V)$ is equipped with a decreasing filtration (which is evidently finite, exhaustive and Hausdorff), and one defines
\[\E{}{0}{\pi^*(\dual\diag V)}{p,q}{}:=F^{p}\pi^{p+q}(\dual\diag V)/F^{p+1}\pi^{p+q}(\dual\diag V).\]
Then, there is a spectral sequence with
\[\E{}{2}{V}{p,q}{}=\pi\dhor^{p}\pi\dver^{q}(\dual V),\]
and differential $d_r:\E{}{r}{V}{p,q}{}\to \E{}{r}{V}{p+r,q-r+1}{}$ so that $\E{}{r+1}{V}{}{}$ is the cohomology of the cochain complex $(\E{}{r}{V}{}{};d_r)$, and for each fixed $p$ and $q$,
\[\E{}{r}{V}{p,q}{}\textup{ stabilizes to }\E{}{\infty}{V}{p,q}{}\cong \E{}{0}{\pi^*(\dual\diag V)}{p,q}{}\textup{ as $r\rightarrow\infty$}.\]

Typically, $V$ will admit an \emph{augmentation} to a simplicial object $V_{-1}\in s\vect{}{}$, \emph{inducing a weak equivalence} $\diag V\overset{\sim}{\to}V_{-1}$. All of our  augmentations are horizontal maps to a vertical object, i.e.\ an augmentation is a simplicial (in $q$) map:
\[d\uhor_0:V_{0,q}\to V_{-1,q}\textup{ coequalizing }d\uhor_0,d\uhor_1:V_{1,q}\to V_{0,q}.\]
In this case, we view the spectral sequence as a tool for the calculation of the cohomotopy $\pi^*(\dual V_{-1})$, via isomorphisms
\[\E{}{\infty}{V}{p,q}{}\cong \E{}{0}{\pi^*(\dual V_{-1})}{p,q}{}.\]


If $V$ is instead a bisimplicial \emph{graded} vector space $V\in ss\vect{c}{h}$, then we may regard $\E{}{r}{V}{}{}$ as an element of $\vect{h+2}{c}$. That is:
\[\E{}{r}{V}{p,q,s_h,\ldots,s_1}{t_c,\ldots,t_1}:=\E{}{r}{(V^{t_c,\ldots,t_1}_{s_h,\ldots,s_1})}{p,q}{}.\]
In our application of these conventions we will actually have $V\in \vect{+}{h}$, and will write $p=s_{h+2}$ and $q=s_{h+1}$. We will even sometimes have a quadratic grading on $V$, which will transfer to a further grading on the spectral sequence, so our spectral sequences will appear in the format
\[\quadgrad{k}\E{}{r}{V}{s_{h+2},\ldots,s_1}{t}:=\E{}{r}{(\quadgrad{k}V^{t}_{s_h,\ldots,s_1})}{s_{h+2},s_{h+1}}{}.\]

\subsection{Second quadrant homotopy spectral sequences}\label{Second quadrant homotopy spectral sequences}
Suppose that $V^{s}_{t}$ is a cosimplicial simplicial (ungraded) vector space. Then there is a cochain-chain complex 
\[CV^{s}_{t}:=C\dhor^{s}C\uver_{t}V=V^{s}_{t},\]
with $s$ the horizontal and $t$ the vertical coordinate. The \emph{total complex} $TV$, is a chain complex with a canonical decreasing filtration, defined by 
\[(TV)_n=\prod_{t-s=n}CV^{s}_{t},\qquad(F^mTV)_n=\prod_{\substack{t-s=n\\s\geq m}}CV^{s}_{t}.\]
Correspondingly, $H_*( TV)$ is equipped with a decreasing filtration, which is exhaustive, but need not be either hausdorff or finite, and one defines
\[\E{0}{}{H_*( TV)}{s}{t}:=F^{s}H_{t-s}( TV)/F^{s+1}   H_{t-s}( TV).\]
Then, there is a spectral sequence with
\[\E{1}{}{V}{s}{t}=C\dhor^{s}\pi\uver_{t}V,\qquad\E{2}{}{V}{s}{t}=\pi\dhor^{s}\pi\uver_{t}V,\]
and differential $d^r:\E{r}{}{V}{s}{t}\to \E{r}{}{V}{s+r}{t+r-1}$ so that $\E{r+1}{}{V}{}{}$ is the homology of the chain complex $(\E{r}{}{V}{}{};d^r)$. From now on, we suppose that the spectral sequence admits a \emph{vanishing line of slope $\alpha$} on $\E{2}{}{}{}{}$, i.e.\ there exists a constant $c$ such that:
\[\E{2}{}{V}{s}{t}=0\textup{ for }s>c+\alpha(t-s).\]
Then the filtration on $H_*( TV)$ \emph{is} in fact hausdorff and finite, and we can say that for each fixed $s$ and $t$:
\[\E{r}{}{V}{s}{t}\textup{ stabilizes to }\E{\infty}{}{V}{s}{t}\cong \E{0}{}{H_*( TV)}{s}{t}\textup{ as $r\rightarrow\infty$}.\]



Now if we write
\[(\calT_mV)_n:=(TV/F^{m+1} TV)_n\cong\prod_{\substack{t-s=n\\s\leq m}}CV^{s}_{t}, \]
then the $\calT_mV$ form a tower of surjections of chain complexes with inverse limit $\calT_\infty V=TV$.  A result of Bousfield relates this to the totalization tower of $V$:
\begin{lem}[{\cite[Lemma 2.2]{BousfieldHSSCS.pdf}}]
There are natural chain maps $\phi_m:N_*\Tot_mV\to \calT_mV$ for $m\leq\infty$ which induce an isomorphism of towers $\pi_*\Tot_mV\to H_*\calT_mV$.
\end{lem}
\noindent Given a vanishing line, there is no $\lim^1$ in either tower, so that $\pi_*\Tot V\cong H_*\calT V$, and the spectral sequence truly calculates $\pi_{t-s} \Tot V$.

Typically, $V$ will admit a \emph{coaugmentation} from a simplicial object $V_{-1}\in s\vect{}{}$.  We necessarily mean a horizontal map from a vertical object, i.e.\ a coaugmentation is a simplicial (in $t$) map:
\[d\dhor^0:V_{-1,t}\to V_{0,t}\textup{ equalizing }d\dhor^0,d\dhor^1:V_{0,t}\to V_{1,t}.\]
Suppose that this map \emph{induces a weak equivalence} $V_{-1}\overset{\sim}{\to}\Tot V$ (and still that $\E{2}{}{V}{s}{t}$ admits a vanishing line).
In this case, we view the spectral sequence as a tool for the calculation of the homotopy $\pi_*(V_{-1})$, via  isomorphisms
\[\E{\infty}{}{V}{s}{t}\cong \E{0}{}{\pi_*(V_{-1})}{s}{t}.\]

We can again add internal gradings, the richest case being that in which  $V\in cs\vect{c}{h}$ also has a quadratic grading:
\[\quadgrad{k}\E{r}{}{V}{s,s_c,\ldots,s_1}{t,t_h,\ldots,t_1}=\E{r}{}{(\quadgrad{k}V^{s_c,\ldots,s_1}_{t_h,\ldots,t_1})}{s}{t}.\]
%In fact, \textbf{something about comparing the two sseqs........ flip across fibrations.}
%
%
%\begin{shaded}\tiny
%\hfil
%\end{shaded}
%
%
%
%
%
%\begin{cor}
%Suppose that $X^\bullet$ is a cosimplicial object in $s\algs$. Then there is a map of spectral sequences, which is an isomorphism from $E^1$ (\textbf{que?}), from the homotopy spectral sequence of the $\Tot$ tower $\{I\Tot_mX^\bullet\}$ to the spectral sequence of the bicomplex underlying $IX^\bullet$. Moreover, there is an isomorphism $\pi_*I\Tot X^\bullet\to H_*TIX^\bullet$ (to the homology of the total complex of the bicomplex), which is compatible with the map of spectral sequences.
%\end{cor}
%\begin{cor}
%Suppose that $X$ is a connected element of $s\algs$. Then the Bousfield-Kan spectral sequence for $X$ is the spectral sequence of the bicomplex underlying the cosimplicial simplicial vector space $I((c(KQ)^\bullet X)^{\textup{rf}})$. This spectral sequence admits a vanishing line from $E^2$, and so converges strongly to its target, $\pi_{t-s}(I\hat X)$.
%\end{cor}











\subsection{Throatclearing}


we will always write $\E{r}{}{V}{}{}$ for a homological or homotopical spectral sequence and $\E{}{r}{V}{}{}$ for a cohomological one.

Given a bisimplicial vector space $V_{s,t}$, we obtain a double chain complex $N_sN_tV$. There is a first quadrant homotopy spectral sequence $\E{r}{}{V}{}{s,t}$, with $\E{2}{}{V}{}{s,t}=\pi\uhor_s\pi\uver_tV$, converging to $\pi_{s+t}(\diag V)$.
There is a first quadrant cohomotopy spectral sequence $\E{}{r}{V}{s,t}{}$, with $\E{}{2}{V}{s,t}{}=\dual(\pi\uhor_s\pi\uver_tV)$, converging to $\dual(\pi_{s+t}(\diag V))$.

Given a cosimplicial simplicial vector space  $V_{s,t}$, we obtain a mixed chain complex $N^sN_tV$.\\
There is a second quadrant homotopy spectral sequence $\E{r}{}{V}{s}{t}$, with $\E{2}{}{V}{s}{t}=\pi\dhor^s\pi\uver_tV$, which will sometimes converge to $\pi_{t-s}\Tot V$.

\hfil

\noindent first quadrant homotopy --- Stover's van Kampen\\
first quadrant cohomology --- Serre sseq, our GSSeqs\\
second quadrant homotopy --- Adams or Bousfield-Kan spectral sequences\\
second quadrant cohomology --- Dwyer's work Eilenberg-Moore









\end{Conventions and notation} %return me to my home

\subsection{Towers, exact couples and coaugmented cosimplicial objects}\label{Towers, exact couples and coaugmented cosimplicial objects}
Suppose that
\[0=\calT_{-1}\epifrom \calT_0\epifrom \calT_1\epifrom \calT_2\epifrom \cdots \epifrom \calT_\infty\]
is a tower of surjections of chain complexes, with $\calT_\infty$ the inverse limit. Define:
\[\E{0}{}{}{s}{}:=\Sigma^s\ker\bigl(\calT_s\to\calT_{s-1}\bigr);\qquad F^m:=\ker(\calT_{\infty}\to\calT_{m-1});\]
\[\E{1}{}{}{s}{}:=H_*\E{0}{}{}{s}{}=H_{*-s}\ker\bigl(\calT_s\to\calT_{s-1}\bigr).\]

From this data we may derive the following diagram, in which any pair of composable maps consisting of a monomorphism then an epimorphism is a short exact sequence of chain complexes:
\[\cdots\ \vcenter{\xymatrix@R=4mm{\textstyle
0
&
\Sigma^{-0}\E{0}{}{}{0}{}
&
\Sigma^{-1}\E{0}{}{}{1}{}
&
\Sigma^{-2}\E{0}{}{}{2}{}
&
\Sigma^{-3}\E{0}{}{}{3}{}
\\
\calT_\infty\ar@{ >->}[d]\ar@{->>}[u]
&
F^0\ar@{ >->}[d]\ar@{=}[l]\ar@{->>}[u]
&
F^1\ar@{ >->}[d]\ar@{ >->}[l]\ar@{->>}[u]
&
F^2\ar@{ >->}[d]\ar@{ >->}[l]\ar@{->>}[u]
&
F^3\ar@{ >->}[d]\ar@{ >->}[l]\ar@{->>}[u]
\\
\calT_\infty\ar@{=}[r]\ar@{->>}[d]&
\calT_\infty\ar@{=}[r]\ar@{->>}[d]&
\calT_\infty\ar@{=}[r]\ar@{->>}[d]&
\calT_\infty\ar@{=}[r]\ar@{->>}[d]&
\calT_\infty\ar@{->>}[d]
\\
0
&\calT_{-1}\ar@{=}[l]
&\calT_0\ar@{->>}[l]
&\calT_1\ar@{->>}[l]
&\calT_2\ar@{->>}[l]
\\
0\ar@{ >->}[u]
&
0\ar@{ >->}[u]
&
\Sigma^{-0}\E{0}{}{}{0}{}\ar@{ >->}[u]
&
\Sigma^{-1}\E{0}{}{}{1}{}\ar@{ >->}[u]
&
\Sigma^{-2}\E{0}{}{}{2}{}\ar@{ >->}[u]
}}\ \cdots \]
Taking homology, each short exact sequence of chain complexes creates a long exact sequence, and we obtain two \emph{exact couples} (c.f.\ \cite{limits_and_sseq.pdf} or \cite[\S2.2]{mccleary.pdf}), which we juxtapose, using wavy maps to indicate boundary homomorphisms:
\[\cdots\ \vcenter{\xymatrix@R=4mm{\textstyle
0
&
\Sigma^{-0}H\E{1}{}{}{0}{}\ar@{~>}[dr]
&
\Sigma^{-1}H\E{1}{}{}{1}{}\ar@{~>}[dr]
&
\Sigma^{-2}H\E{1}{}{}{2}{}\ar@{~>}[dr]
&
\Sigma^{-3}H\E{1}{}{}{3}{}
\\
H\calT_\infty\ar[d]\ar[u]
&
HF^0\ar[d]\ar@{=}[l]\ar[u]
&
HF^1\ar[d]\ar[l]\ar[u]
&
HF^2\ar[d]\ar[l]\ar[u]
&
HF^3\ar[d]\ar[l]\ar[u]
\\
H\calT_\infty\ar@{=}[r]\ar[d]&
H\calT_\infty\ar@{=}[r]\ar[d]&
H\calT_\infty\ar@{=}[r]\ar[d]&
H\calT_\infty\ar@{=}[r]\ar[d]&
H\calT_\infty\ar[d]
\\
0
&H\calT_{-1}\ar@{=}[l]\ar@{~>}[dr]\ar@{~>}@/^2em/[uu]
&H\calT_0\ar[l]\ar@{~>}[dr]\ar@{~>}@/^2em/[uu]
&H\calT_1\ar[l]\ar@{~>}[dr]\ar@{~>}@/^2em/[uu]
&H\calT_2\ar[l]\ar@{~>}@/^2em/[uu]
\\
0\ar[u]
&
0\ar[u]
&
\Sigma^{-0}H\E{1}{}{}{0}{}\ar[u]
&
\Sigma^{-1}H\E{1}{}{}{1}{}\ar[u]
&
\Sigma^{-2}H\E{1}{}{}{2}{}\ar[u]
}}\ \cdots \]
The vertical boundary homomorphisms $H\calT_m\to \Sigma HF^{m+1}$ in fact form a morphism of exact couples (c.f.\ \cite{limits_and_sseq.pdf}), as follows from Verdier's octahedron axiom (in the triangulated homotopy category of  chain complexes) or a diagram chase. Moreover, the two resulting spectral sequences have the same $E^1$ page, so that they are identical (see also \cite[\S6]{limits_and_sseq.pdf}). %We have made an effort to note these opposite yet equal spectral sequences, as they form a model for more homotopical
This common spectral sequence is simply the spectral sequence of the decreasing filtration $F^m$ on the complex $\calT_{\infty}$ (c.f.\ \cite[\S2.2]{mccleary.pdf}, \cite{BousfieldHSSCS.pdf}).
The intended target $H\calT_\infty$ has decreasing filtration
\[F^m(H\calT_\infty):=\im(HF^m\to H\calT_\infty)=\ker(H\calT_\infty\to H\calT_{m-1}),\]
and one writes $\E{0}{}{H\calT_\infty}{s}{t}=F^mH_{t-s}\calT_\infty/F^{m+1}H_{t-s}\calT_\infty$. [pro-triviality, bousfield 2.3]

One context in which we may make these constructions, is when given any sequence of maps 
\[0=\mathbb{T}_{-1}\from \mathbb{T}_{0} \from  \mathbb{T}_{1}\from \cdots\]
in $s\vect{}{}$. Such a tower may be converted into a homotopy equivalent tower of surjections
\[0=\mathbb{T}'_{-1}\epifrom \mathbb{T}'_{0} \epifrom  \mathbb{T}'_{1}\epifrom \cdots\] and we may use the above constructions with $\calT_m:=C_*\mathbb{T}'$. Homotopy equivalent towers will produce isomorphic spectral sequences from $E^1$. From this perspective, a straightforward way to give a map of spectral sequences that \emph{shifts filtration} is simply to give a map of towers with the corresponding shift.


Suppose now that $V$ is an object of $(s\vect{}{})^{\Delta_+}$, the category of coaugmented cosimplicial objects in the category of simplicial vector spaces. We think of the cosimplicial direction as \emph{horizontal} and the simplicial direction as \emph{vertical}, so that the \emph{coaugmentation} of $V$ is a (horizontal) map from a (vertical) simplicial object $V^{-1}\in s\vect{}{}$,i.e.\ a simplicial (in $t$) map:
\[d\dhor^0:V_{t}^{-1}\to V^{0}_{t}\textup{ equalizing }d\dhor^0,d\dhor^1:V^{0}_{t}\to V^{1}_{t}.\]

There is a cochain-chain complex 
\[CV^{s}_{t}:=C\dhor^{s}C\uver_{t}V=V^{s}_{t},\]
with $s$ the horizontal and $t$ the vertical coordinate. The \emph{total complex} $TV$ is a chain complex with a canonical decreasing filtration, defined by 
\[\textstyle(TV)_n=\prod_{t-s=n}CV^{s}_{t},\qquad d=d\dhor+d\uver,\qquad(F^mTV)_n=\prod_{\substack{t-s=n\\s\geq m}}CV^{s}_{t}.\]
On the other hand, we may write
\[(\calT_mV)_n:=(TV/F^{m+1} TV)_n\cong\textstyle\prod_{\substack{t-s=n\\s\leq m}}CV^{s}_{t}, \]
giving a tower $\calT_mV$ of surjections of chain complexes with inverse limit $\calT_\infty V=TV$. We are in the setting just introduced, and the resulting spectral sequence satisfies\begin{alignat*}{2}
\E{0}{}{V}{s}{t}
&:=
C\dhor^sC\uver_tV,
&\qquad&d^0=d^\textup{v};\\
\E{1}{}{V}{s}{t}
&:=
C\dhor^s\pi\uver_tV,
&\qquad&d^1=d_\textup{h};\\
\E{2}{}{V}{s}{t}
&:=
\pi\dhor^s\pi\uver_tV.
\end{alignat*}
If one defines
\[\E{0}{}{H_*( TV)}{s}{t}:=F^{s}H_{t-s}( TV)/F^{s+1}   H_{t-s}( TV).\]
Then, there is a spectral sequence with
\[\E{1}{}{V}{s}{t}=C\dhor^{s}\pi\uver_{t}V,\qquad\E{2}{}{V}{s}{t}=\pi\dhor^{s}\pi\uver_{t}V,\]
and differential $d^r:\E{r}{}{V}{s}{t}\to \E{r}{}{V}{s+r}{t+r-1}$ so that $\E{r+1}{}{V}{}{}$ is the homology of the chain complex $(\E{r}{}{V}{}{};d^r)$. 
We can again add internal gradings, the richest case being that in which  $V\in cs\vect{c}{h}$ also has a quadratic grading:
\[\quadgrad{k}\E{r}{}{V}{s,s_c,\ldots,s_1}{t,t_h,\ldots,t_1}=\E{r}{}{(\quadgrad{k}V^{s_c,\ldots,s_1}_{t_h,\ldots,t_1})}{s}{t}.\]

The spectral sequence will sometimes admit a \emph{vanishing line of slope $\alpha$} on $\E{2}{}{}{}{}$, i.e.\ there exists a constant $c$ such that:
\[\E{2}{}{V}{s}{t}=0\textup{ for }s>c+\alpha(t-s).\]
In this case, the filtration on $H_*( TV)$ \emph{is} in fact hausdorff and finite, and  for each fixed $s$ and $t$:
\[\E{r}{}{V}{s}{t}\textup{ stabilizes to }\E{\infty}{}{V}{s}{t}\cong \E{0}{}{H_*( TV)}{s}{t}\textup{ as $r\rightarrow\infty$}.\]
%\begin{shaded}\tiny
%\noindent Given a vanishing line, there is no $\lim^1$ in either tower, so that $\pi_*\Tot V\cong H_*\calT V$, and the spectral sequence truly calculates $\pi_{t-s} \Tot V$.
%
%Typically, $V$ will admit a \emph{coaugmentation} from a simplicial object $V_{-1}\in s\vect{}{}$.  We necessarily mean a horizontal map from a vertical object, i.e.\ a coaugmentation is a simplicial (in $t$) map:
%\[d\dhor^0:V_{-1,t}\to V_{0,t}\textup{ equalizing }d\dhor^0,d\dhor^1:V_{0,t}\to V_{1,t}.\]
%Suppose that this map \emph{induces a weak equivalence} $V_{-1}\overset{\sim}{\to}\Tot V$ (and still that $\E{2}{}{V}{s}{t}$ admits a vanishing line).
%\end{shaded}
When the coaugmentation \emph{induces a weak equivalence} $V_{-1}\overset{\sim}{\to}\Tot V$, we view the spectral sequence as a tool for the calculation of the homotopy $\pi_*(V_{-1})$, via  isomorphisms
\[\E{\infty}{}{V}{s}{t}\cong \E{0}{}{\pi_*(V_{-1})}{s}{t}.\]




Bousfield explains how this picture relates to the totalization tower \cite[VII.5]{goerss-jardine.pdf} of $V$:
\begin{lem}[{\cite[Lemma 2.2]{BousfieldHSSCS.pdf}}]
There are natural chain maps $N_*\Tot_mV\to \calT_mV$ for $m\leq\infty$ which induce an isomorphism of towers $\pi_*\Tot_mV\to H_*\calT_mV$. In particular $H_*(TV)\cong \pi_*\Tot V$.
\end{lem}
Not only then do we have a tower under $\calT_\infty V\sim\Tot V$, but $\Tot V$ accepts the coaugmentation map from $C_*V^{-1}$. 
Of course, the coaugmentation map need not be surjective, but if we factor it  as a composite
\[\xymatrix@R=4mm{
V^{-1}\ar@{ >->}[r]^-{\sim}
&%r1c1
r(V^{-1})\ar@{->>}[r]&%r1c2
\Tot V%r1c3
}\]
we may form the following diagram by demanding that the vertical compsites be strict fiber sequences
\[\cdots\ \vcenter{\xymatrix@R=4mm{\textstyle
r(V^{-1})\ar@{ >->}[d]
&
F^0\ar@{ >->}[d]\ar@{=}[l]
&
F^1\ar@{ >->}[d]\ar@{ >->}[l]
&
F^2\ar@{ >->}[d]\ar@{ >->}[l]
&
F^3\ar@{ >->}[d]\ar@{ >->}[l]
\\
r(V^{-1})\ar@{=}[r]\ar@{->>}[d]&
r(V^{-1})\ar@{=}[r]\ar@{->>}[d]&
r(V^{-1})\ar@{=}[r]\ar@{->>}[d]&
r(V^{-1})\ar@{=}[r]\ar@{->>}[d]&
r(V^{-1})\ar@{->>}[d]
\\
0
&\Tot_{-1}\ar@{=}[l]
&\Tot_0\ar@{->>}[l]
&\Tot_1\ar@{->>}[l]
&\Tot_2\ar@{->>}[l]
}}\ \cdots \]
and applying the functor $C_*$.
%\[0=\calT_{-1}V\epifrom \calT_0V\epifrom \calT_1V\epifrom \calT_2V\epifrom \cdots \epifrom \calT_\infty V\from C_*V^{-1}.\]
%Of course, the coaugmentation map need not be surjective, but this will be of little consequence.

We will in general hope that $V^{-1}\to \Tot V$ will be a weak equivalence, and to investigate whether or not this is so, it will be helpful to be able to identify the fibres $F^m$ up to homotopy.
For this we recall a useful relationship between cosimplicial objects and cubical diagrams, explained by Sinha in \cite[Theorem 6.5]{SinhaSpacesOfKnots.pdf}, and expanded on by Munson-Voli\'c \cite{CubicalHomotopyTheory.pdf}. We will only present that part of the theory that we need, and refer the reader to \cite{GoodwillieCalcII}, \cite{LuisGoodwillie.pdf} or \cite{CubicalHomotopyTheory.pdf} for the  theory of cubical diagrams and their homotopy total fibers. For $n\geq0$ let $[n]=\{0,\ldots,n\}$, and define $\calP[n]=\left\{S\subseteq [n]\right\}$ to be the poset category whose morphisms are the inclusions $S\subseteq S'$. Then an \emph{$(n+1)$-cube} in $s\vect{}{}$ is a functor $\calP[n]\to s\vect{}{}$.


Sinha describes a diagram of inclusions of categories
\[\xymatrix@!0@R=10mm@C=13mm{
\calP[-1]\ar[rr]^-{\tau}\ar_-{h_{-1}}[drrr]
&&%r1c1
\calP[0]\ar[rr]^-{\tau}\ar^(.65){\!h_{0}}[dr]
&&%r1c2
\calP[1]\ar[rr]^-{\tau}\ar_(.65){h_{1}\!}[dl]
&&%r1c3
\calP[2]\ar[rr]^-{\tau}\ar^-{h_{2}}[dlll]
&&%r1c4
\cdots \\
&&&\Delta_+\!\!
}\]
The augmented cosimplicial simplicial vector space $V:\Delta_+\to s\vect{}{}$ may be pulled back along $h_m$ to form an $(m+1)$-cubical diagram $h_m^*V$. After noting that $V$ is Reedy fibrant (c.f.\ \cite[{X.4.9}]{YellowMonster}), Sinha explains that there are natural weak equivalences 
\[F^{m+1}\sim\hofib(V^{-1}\to\textup{Tot}_mV) \overset{\smash{\sim}}{\to} \textup{hototfib}(h_m^*V)\]
under which the inclusion $F^m\from F^{m+1}$
%\[\hofib(V^{-1}\to\textup{Tot}_nV)\to \hofib(V^{-1}\to\textup{Tot}_{n-1}V)\]
is identified with the map
\[\textup{hototfib}(h_m^*V)\to \textup{hototfib}(\tau^*h_m^*V)=\textup{hototfib}(h_{m-1}^*V).\]
As $h_{-1}^*V$ is the 0-cube with value $V^{-1}$, the tower of homotopy total fibers is identified up to homotopy with the tower of the $F^m$.

%Suppose that $V^{s}_{t}$ is a cosimplicial simplicial (ungraded) vector space. Then there is a cochain-chain complex 
%\[CV^{s}_{t}:=C\dhor^{s}C\uver_{t}V=V^{s}_{t},\]
%with $s$ the horizontal and $t$ the vertical coordinate. The \emph{total complex} $TV$ is a chain complex with a canonical decreasing filtration, defined by 
%\[\textstyle(TV)_n=\prod_{t-s=n}CV^{s}_{t},\qquad d=d\dhor+d\uver,\qquad(F^mTV)_n=\prod_{\substack{t-s=n\\s\geq m}}CV^{s}_{t}.\]
%On the other hand, we may write
%\[\textstyle(\calT_mV)_n:=(TV/F^{m+1} TV)_n\cong\prod_{\substack{t-s=n\\s\leq m}}CV^{s}_{t}, \]
%giving a tower $\calT_mV$ of surjections of chain complexes with inverse limit $\calT_\infty V=TV$.
%
%We can summarise this situation with a diagram, abbreviating $F^mTV$ to $F^m$ and $\calT_mV$ to $\calT_m$:
%\[\cdots\ \vcenter{\xymatrix@R=4mm{\textstyle
%0
%&
%\frac{F^0}{F^1}
%&
%\frac{F^1}{F^2}
%&
%\frac{F^2}{F^3}
%&
%\frac{F^3}{F^4}
%\\
%TV\ar@{ >->}[d]\ar@{->>}[u]
%&
%F^0\ar@{ >->}[d]\ar@{=}[l]\ar@{->>}[u]
%&
%F^1\ar@{ >->}[d]\ar@{ >->}[l]\ar@{->>}[u]
%&
%F^2\ar@{ >->}[d]\ar@{ >->}[l]\ar@{->>}[u]
%&
%F^3\ar@{ >->}[d]\ar@{ >->}[l]\ar@{->>}[u]
%\\
%TV\ar@{=}[r]\ar@{->>}[d]&
%TV\ar@{=}[r]\ar@{->>}[d]&
%TV\ar@{=}[r]\ar@{->>}[d]&
%TV\ar@{=}[r]\ar@{->>}[d]&
%TV\ar@{->>}[d]
%\\
%0
%&\calT_{-1}\ar@{=}[l]
%&\calT_0\ar@{->>}[l]
%&\calT_1\ar@{->>}[l]
%&\calT_2\ar@{->>}[l]
%\\
%0\ar@{ >->}[u]
%&
%0\ar@{ >->}[u]
%&
%\frac{F^0}{F^1}\ar@{ >->}[u]
%&
%\frac{F^1}{F^2}\ar@{ >->}[u]
%&
%\frac{F^2}{F^3}\ar@{ >->}[u]
%}}\ \cdots \]
%In this diagram, any pair of composable maps consisting of a monomorphism then an epimorphism is a short exact sequence of chain complexes. Taking homology, each such pair creates a long exact sequence, and we obtain two \emph{spread out exact couples}, which we juxtapose, using wavy maps to indicate boundary homomorphisms:
%\[\cdots\ \vcenter{\xymatrix@R=4mm{\textstyle
%0
%&
%H\frac{F^0}{F^1}\ar@{~>}[dr]
%&
%H\frac{F^1}{F^2}\ar@{~>}[dr]
%&
%H\frac{F^2}{F^3}\ar@{~>}[dr]
%&
%H\frac{F^3}{F^4}
%\\
%HTV\ar[d]\ar[u]
%&
%HF^0\ar[d]\ar@{=}[l]\ar[u]
%&
%HF^1\ar[d]\ar[l]\ar[u]
%&
%HF^2\ar[d]\ar[l]\ar[u]
%&
%HF^3\ar[d]\ar[l]\ar[u]
%\\
%HTV\ar@{=}[r]\ar[d]&
%HTV\ar@{=}[r]\ar[d]&
%HTV\ar@{=}[r]\ar[d]&
%HTV\ar@{=}[r]\ar[d]&
%HTV\ar[d]
%\\
%0
%&H\calT_{-1}\ar@{=}[l]\ar@{~>}[dr]\ar@{~>}@/^2em/[uu]
%&H\calT_0\ar[l]\ar@{~>}[dr]\ar@{~>}@/^2em/[uu]
%&H\calT_1\ar[l]\ar@{~>}[dr]\ar@{~>}@/^2em/[uu]
%&H\calT_2\ar[l]\ar@{~>}@/^2em/[uu]
%\\
%0\ar[u]
%&
%0\ar[u]
%&
%H\frac{F^0}{F^1}\ar[u]
%&
%H\frac{F^1}{F^2}\ar[u]
%&
%H\frac{F^2}{F^3}\ar[u]
%}}\ \cdots \]





\begin{CPiAlgs and CHalgs}

\section{\textbf{Homotopy operations and cohomology operations}}
Let $\calc$ be a category of universal graded algebras, monadic over $\vect{+}{r}$. Our goal is to understand construct operations on the homotopy and cohomology of an object of $s\calc$. In \S\ref{homotopy and pialgs} and \S\ref{cohomology and Halgs}, we set out dual frameworks in which these operations can be organised, and in \ref{quadratic part section} and \ref{subseq:The smash coproduct}, we will describe some useful chain level operations that we will use to construct cohomology operations in \S\ref{sec:Constructing cohomology operations}.


\subsection{Homotopy groups and $\calc$-$\Pi$-algebras}\label{homotopy and pialgs}
%The forgetful functor $U^\calc:\calc\to \vect{+}{r}$ used to define the weak equivalences and fibrations in $s\calc$, allows us to consider any object $X\in s\calc$ as an object of $s\vect{+}{r}$. In particular, for any $X\in s\calc$ we may define homotopy groups
Using the forgetful functor $U^\calc:\calc\to \vect{+}{r}$, for any $X\in s\calc$ we may define the homotopy groups $\pi_*X$ of $X$,
%\[(\pi_*X)_{s_{r+1},\ldots,s_1}^{t}:=\pi_{s_{r+1}}(X^t_{s_r,\ldots,s_1}),\]
which we view together as an object  of $\vect{+}{r+1}$. By the definition of the model structure, $\pi_*$ is a homotopical functor.
For any $X\in s\calc$, by adjunction:
%\[\hom_{s\calc}(F^\calc\mathbb{K}^t_{n,s_r,\ldots,s_1},X)\cong (ZN_nX)_{s_r,\ldots,s_1}^t\textup{ and }\hom_{s\calc}(F^\calc C\mathbb{K}^t_{n,s_r,\ldots,s_1},X)\cong (N_{n+1}X)_{s_r,\ldots,s_1}^t,\]
%\begin{gather*}
%\hom_{s\calc}(F^\calc\mathbb{K}^t_{s_{r+1},\ldots,s_1},X)\cong (ZN_{s_{r+1}}X)_{s_r,\ldots,s_1}^t\makebox[0cm][l]{ and}\\
%\hom_{s\calc}(F^\calc C\mathbb{K}^t_{s_{r+1},\ldots,s_1},X)\cong (N_{s_{r+1}+1}X)_{s_r,\ldots,s_1}^t,
%\end{gather*}
\begin{alignat*}{2}
\hom_{s\calc}(F^\calc\mathbb{K}^t_{s_{r+1},\ldots,s_1},X)&\cong (ZN_{s_{r+1}}X)_{s_r,\ldots,s_1}^t\makebox[0cm][l]{ and}\\
\hom_{s\calc}(F^\calc C\mathbb{K}^t_{s_{r+1},\ldots,s_1},X)&\cong (N_{s_{r+1}+1}X)_{s_r,\ldots,s_1}^t,
\end{alignat*}
and indeed $i^*$ plays the same role as above, representing the differential of $N_*X$. Moreover, in the homotopy category corresponding to the above model category structure, $F^\calc\mathbb{K}^t_{n,s_r,\ldots,s_1}$ represents $\pi_n(\DASH)^t_{s_r,\ldots,s_1}$, c.f.\ \cite[\S1]{MR1089001} or \cite[\S3.1.1]{Blanc_Stover-Groth_SS.pdf}. For this reason we will refer to the objects $F^\calc\mathbb{K}^t_{n,s_r,\ldots,s_1}$ as the \emph{spheres} in $s\calc$.

%In light of this fact, the homotopy groups of $X$ possess
By virtue of the algebraic structure posessed by $X$, the homotopy groups $\pi_*X$ possess certain natural algebraic structure, that of a $\calc$-$\Pi$-algebra. Indeed, as any given homotopy group is a representable functor on the homotopy category, natural $N$-ary operations on homotopy groups
\[(\pi_*X)_{s_{r+1}^{1},\ldots,s_{1}^{1}}^{t^1}\times\cdots \times(\pi_*X)_{s_{r+1}^{N},\ldots,s_{1}^{N}}^{t^N}\to (\pi_*X)_{s_{r+1},\ldots,s_{1}}^{t}\]
are in bijective correspondence with elements of the group
\[ \pi_*\Bigl(F^\calc\mathbb{K}_{s_{r+1}^{1},\ldots,s_{1}^{1}}^{t^1}\sqcup\cdots\sqcup F^\calc\mathbb{K}_{s_{r+1}^{N},\ldots,s_{1}^{N}}^{t^N}\Bigr)\makebox[0cm][l]{}_{s_{r+1},\ldots,s_{1}}^{t}.\]
Blanc and Stover \cite{Blanc_Stover-Groth_SS.pdf} define a new category of graded universal algebras, the category of $\calc$-$\Pi$-algebras, monadic over $\vect{+}{r+1}$, whose objects are graded vector spaces $V\in\vect{+}{r+1}$ with a structure map 
\[V_{s_{r+1}^{1},\ldots,s_{1}^{1}}^{t^1}\times\cdots \times V_{s_{r+1}^{N},\ldots,s_{1}^{N}}^{t^N}\to V_{s_{r+1},\ldots,s_{1}}^{t}\]
for every such homotopy class, satisfying certain natural compatibilities.

It is a standard formalism to encode these compatibilities as follows. A \emph{model}  \cite{Blanc_Stover-Groth_SS.pdf} in $s\calc$ is 
an almost free object of $s\calc$ which is weakly equivalent to a coproduct of spheres (for example, $F^{\calc}\Gamma V$ for any $V\in\vect{+}{r+1}$ viewed as a chain complex with zero differential). A \emph{finite model} is a model in which this coproduct is finite. Let $\Pi$  be the $\vect{}{}$-enriched category with objects the finite models in $s\calc$, and morphisms
\[\hom_{\Pi}(M,M'):=\hom_{\textup{ho}(s\calc)}(M,M').\]
Then the category of $\calc$-$\Pi$-algebras  may be defined as the category of $\vect{}{}$-enriched functors $\Pi^\textup{op}\to \vect{}{}$ that send finite coproducts into products (where by $\vect{}{}$ we mean the category of ungraded $\Ftwo $-vector spaces). The category of $\calc$-$\Pi$-algebras is monadic over $\vect{+}{r+1}$, with  forgetful functor $U^{\calc\PiAlg}A$ defined on a functor $A\in \calc\PiAlg$ by:%$A$ under the forget Such a functor $A$ yields an object of $U^{\calc\PiAlg}A$ of $\vect{+}{r+1}$ via
\[(U^{\calc\PiAlg}A)^t_{s_{r+1},\ldots,s_1}:=A(F^\calc\mathbb{K}_{s_{r+1},\ldots,s_{1}}^{t}).\]
Any operators advertised above on $U^{\calc\PiAlg}A$ is then induced by the corresponding homotopy class of a coproduct of spheres, viewed as a map in $\Pi$. %The natural compatibilities referred to above are the conditions required that a collection of structure maps extends to a functor on $\Pi^\textup{op}$.

One obtains the free $\calc$-$\Pi$-algebra on a graded vector space $V\in \vect{+}{r+1}$ using Dold's theorem (\ref{Dold's theorem}). That is, one views $V$ as a chain complex in $\complexes\vect{+}{r}$ with zero differential, and applies the Dold-Kan correspondence and $\calc$-free functor, obtaining an object $F^\calc\Gamma V\in s\calc$. Then:
\[F^{\calc\PiAlg}V=\pi_*(F^\calc\Gamma V).\]
Moreover, as $F^\calc $ is an augmented monad, so is $F^{\calc\PiAlg}$, via the map
\[F^{\calc\PiAlg}V=\pi_*(F^\calc\Gamma V)\overset{\pi_*\epsilon}{\to}\pi_*(\Gamma V)=H_*V=V,\]
and in particular, there is an adjunction $Q^{\calc\PiAlg}\dashv K^{\calc\PiAlg}$.
%For any simplicial object $A\in s\calc$, there is a natural map $\gamma:Q^{\calc\PiAlg}\pi A\to \pi Q^{\calc}A$.
\begin{lem}
For any $A\in s\calc$, the map $\pi_*A\to \pi_*Q^\calc A$ descends to a map 
\[\gamma:Q^{\calc\PiAlg}\pi_* A\to \pi_* Q^{\calc}A.\]
If $A$ is a model in $\calc$ then $\gamma$ is an isomorphism.
\end{lem}
\begin{proof}{[\textbf{what is the status of this?}]} $A$ is homotopic to a wedge of spheres, and $\pi A$ is free on generators in correspondence with the wedge summands.
\end{proof}
\subsection{Cohomology groups and $\calc$-$H^*$-algebras}\label{cohomology and Halgs}
It will in general be preferable for us to consider algebraic structure on cohomology, rather than coalgebraic structure on homology: algebra is in general a more familiar subject than coalgebra, and cohomology has the advantage that it consists of representable functors. That is, in the homotopy category of $s\calc$, the object $K^\calc\mathbb{K}^t_{n,s_r,\ldots,s_1}$ represents the contravariant functor $H^n_{\calc}(\DASH)_t^{s_r,\ldots,s_1}:s\calc\to\vect{}{}$,  c.f.\ \cite[Proposition 4.3]{MR1089001}. Using cohomology groups has the disadvantages associated with double-dualisation.


%The forgetful functor $U^\calc:\calc\to \vect{+}{r}$ used to define the weak equivalences and fibrations in $s\calc$, allows us to consider any object $X\in s\calc$ as an object of $s\vect{+}{r}$. In particular, for any $X\in s\calc$ we may define homotopy groups
%We have defined the cohomology of $X\in s\calc$ to be the dual left-derived functors of the indecomposables functor $Q^\calc:\calc\to \vect{+}{r}$,
%\[(H^{s_{r+1}}_\calc X)^{s_r,\ldots,s_1}_t:=((H_{s_{r+1}}^\calc X)_{s_r,\ldots,s_1}^t)^*:=(\pi_{s_{r+1}}(Q^{\calc}B^\calc X)_{s_r,\ldots,s_1}^t)^*,\]
%which we view together as an object of $\vect{r+1}{+}$.


%In light of this fact, the homotopy groups of $X$ possess
By virtue of the algebraic structure posessed by $X$, the cohomology groups $H_\calc^*X$ possess certain natural algebraic structure, that of a $\calc$-$H^*$-algebra.  As for $\calc$-$\Pi$-algebras, natural $N$-ary operations on cohomology groups
\[(H^*_\calc X)^{s_{r+1}^{1},\ldots,s_{1}^{1}}_{t^1}\times\cdots \times(H^*_\calc X)^{s_{r+1}^{N},\ldots,s_{1}^{N}}_{t^N}\to (H^*_\calc X)^{s_{r+1},\ldots,s_{1}}_{t}\]
are in bijective correspondence with elements of the group
\[ H^*_\calc\Bigl(K^\calc\mathbb{K}_{s_{r+1}^{1},\ldots,s_{1}^{1}}^{t^1}\times\cdots\times K^\calc\mathbb{K}_{s_{r+1}^{N},\ldots,s_{1}^{N}}^{t^N}\Bigr)\makebox[0cm][l]{}^{s_{r+1},\ldots,s_{1}}_{t}.\]
The category of $\calc$-$H^*$-algebras, monadic over $\vect{r+1}{+}$, has objects graded vector spaces $V\in\vect{r+1}{+}$ with a structure map 
\[V^{s_{r+1}^{1},\ldots,s_{1}^{1}}_{t^1}\times\cdots \times V^{s_{r+1}^{N},\ldots,s_{1}^{N}}_{t^N}\to V^{s_{r+1},\ldots,s_{1}}_{t}\]
for every such cohomology class, satisfying certain natural compatibilities.

The formalism required to express these compatibilities is as follows. A \emph{generalized Eilenberg-Mac Lane object}, or \emph{GEM}, in $s\calc$ is an almost free object of $s\calc$ which is weakly equivalent to a product of objects of the form $K^\calc\mathbb{K}_{s_{r+1},\ldots,s_{1}}^{t}$. A \emph{finite GEM} is a GEM in which this product is finite. Let $K$  be the $\vect{}{}$-enriched category with objects the finite GEMs in $s\calc$, and morphisms
\[\hom_{\Pi}(M,M'):=\hom_{\textup{ho}(s\calc)}(M,M').\]
Then the category of $\calc$-$H^*$-algebras  may be defined as the category of $\vect{}{}$-enriched functors $\mathbb{K}\to \vect{}{}$ that preserve finite products. The category of $\calc$-$H^*$-algebras is monadic over $\vect{r+1}{+}$, with forgetful functor defined  on a functor $h:K\to \vect{}{}$  by:%$A$ under the forget Such a functor $A$ yields an object of $U^{\calc\HAlg}A$ of $\vect{r+1}{+}$ via
\[(U^{\calc\HAlg}h)_t^{s_{r+1},\ldots,s_1}:=h(K^\calc\mathbb{K}_{s_{r+1},\ldots,s_{1}}^{t}),\]
and any of the structure maps advertised above is induced by the corresponding cohomology class. %The natural compatibilities referred to above are the conditions required that a collection of structure maps extends to a functor on $\Pi^\textup{op}$.

One obtains the free $\calc$-$H^*$-algebra on a graded vector space $V\in \vect{r+1}{+}$ \emph{of finite type} as follows. One views $\dual V$ as a chain complex in $\complexes\vect{+}{r}$ with zero differential, and applies the Dold-Kan correspondence and $K^\calc$, obtaining an object $K^\calc\Gamma \dual V\in s\calc$. Then:
\[F^{\calc\HAlg}V=H^*_{\calc}K^\calc\Gamma\dual V.\]
Moreover, $F^{\calc\HAlg}$ is an augmented monad: one applies $H^*_{\calc}$ to the natural collapse map $F^{\calc}\Gamma V\to K^{\calc}\Gamma V$, to obtain
\[K^{\calc\HAlg}V\cong H^*_{\calc} F^{\calc}\Gamma V\from H^*_{\calc} K^{\calc}\Gamma V=:F^{\calc\HAlg}V.\]

\[F^{\calc\HAlg}V=\pi_*F^\calc\Gamma V\overset{\pi_*\epsilon}{\to}\pi_*\Gamma V=H_*V=V,\]
and in particular, there is an adjunction $Q^{\calc\HAlg}\dashv K^{\calc\HAlg}$.
%For any simplicial object $A\in s\calc$, there is a natural map $\gamma:Q^{\calc\HAlg}\pi A\to \pi Q^{\calc}A$.

\subsection{The smash coproduct}\label{subseq:The smash coproduct}
For $X_1$ and $X_2$ objects of any algebraic category, for example $\calC$, $\calC\PiAlg$ or $\calC\HAlg$ (to be defined shortly), we define the \emph{smash coproduct} $X_1\smashcoprod X_2$ to be the kernel of the natural map $X_1\sqcup X_2\to X_1\times X_2$. When $X_1=X_2=X$, $X\smashcoprod X$ has a natural action of $\Sigma_2$, and we write $X\smashcoprod^{\Sigma_2} X$ for the subobject of invariant elements under this action.

When $X_1$ and $X_2$ are  objects of $s\calc$, taking this strict fiber is in fact homotopically correct, since the map $X_1\sqcup X_2\to X_1\times X_2$ is always a fibration, and indeed:
\begin{prop}\label{smash coprod}
For $X_1$ and $X_2$ in $s\calc$, the natural $\calC$-$\Pi$-algebra map
\[\pi_*(X_1\times X_2)\to \pi_* X_1\times \pi_* X_2\]
 is an isomorphism. If $X_1$ and $X_2$ are models in $s\calc$, the natural $\calC$-$\Pi$-algebra map 
\[\pi_* X_1\sqcup \pi_* X_2\to\pi_*(X_1\sqcup X_2)\]
takes part in an isomorphism of short exact sequences:
\[\xymatrix@R=4mm{
0\ar[r]&%r1c1
\pi_* X_1\smashcoprod \pi_* X_2
\ar[r]\ar[d]_-{\cong}
%\ar[d]^-{i}_-{\cong}
&%r1c2
\pi_* X_1\sqcup \pi_* X_2
\ar[r]
\ar[d]_-{\cong}\ar[d]_-{\cong}
&%r1c3
\pi_* X_1\times \pi_* X_2
\ar[r]
\ar[d]_-{\cong}
&%r1c4
0\\%r1c5
0\ar[r]&%r1c1
\pi_* (X_1\smashcoprod  X_2)
\ar[r]
&%r1c2
\pi_* (X_1\sqcup X_2)
\ar[r]
&%r2c3
\pi_* (X_1\times X_2)
\ar[r]&0
}\]
\end{prop}
\begin{proof}
The first claim is easy: the forgetful functor is a right adjoint, and $\pi_*$ preserves products (of vector spaces). Consider the commuting diagram
\[\xymatrix@R=4mm{
0\ar[r]&%r1c1
\pi_* X_1\smashcoprod \pi_* X_2
\ar[r]
%\ar[d]^-{i}_-{\cong}
&%r1c2
\pi_* X_1\sqcup \pi_* X_2
\ar[r]
\ar[d]^-{i}
&%r1c3
\pi_* X_1\times \pi_* X_2
\ar[r]
\ar[d]_-{\cong}
&%r1c4
0\\%r1c5
&%r1c1
\pi_* (X_1\smashcoprod  X_2)
\ar[r]
&%r1c2
\pi_* (X_1\sqcup X_2)
\ar[r]
&%r2c3
\pi_* (X_1\times X_2)
}\]
in which the top row is a short exact sequence, and the bottom row is just a three term excerpt of the homotopy long exact sequence of the fiber sequence defining $X_1\smashcoprod X_2$. If $i$ were an isomorphism, the bottom row would also be short exact, and a simple diagram chase would show that $i$ restricts to the isomorphism we desire.


If $X_1$ and $X_2$ are models, the displayed map $i$ is an isomorphism, since both source and target represent the free $\calc$-$\Pi$-algebra on generators corresponding to the sphere summands of $X_1$ and $X_2$ taken together. 
\end{proof}

\subsection{Cofibrant replacement via the small object argument}\label{Cofibrant replacement via the small object argument}
The homotopy of an object $X$ of $s\calc$ was defined simply by application of the forgetful functor $U^{\calc}:\calc\to\vect{}{}$, a definition which is tautologically homotopically correct. On the other hand, in order to define the homology $H_*^{\calc}X$, as the left Quillen functor $Q^{\calc}$ does not preserve all weak equivalences, we must perform a cofibrant replacement before applying $Q^{\calc}$. While the comonadic bar construction $B^{\calc}$ described in \S\ref{ssec: quillen model and bar construction} suffices to define the groups $H_*^{\calc}X$, it lacks the structure that we will need at various points in this work.

Radulescu-Banu's innovation \cite{Radulescu-Banu.pdf} first explained that the cofibrant replacement functor $c:s\calc\to s\calc$ constructed by Quillen's \emph{small object argument}  \cite{QuillenHomAlg.pdf}, which by design already posesses a natural acyclic fibration $\epsilon:c\to\Id$, in fact admits the full structure of a comonad, with diagonal $\beta:c\to cc$. As explained by Blumberg and Riehl \cite[Remark 4.12]{BlumRiehlResolutions.pdf}:
\begin{prop}\label{QcK is a comonad}
The endofunctor $Q^\calc cK^\calc$ of $s\vect{}{}$ admits the structure of a comonad, via the maps
\[Q^\calc cK^\calc\overset{\smash{Q^\calc (\beta)}}{\to}Q^\calc ccK^\calc \overset{\smash{Q^\calc c(\eta)}}{\to} Q^\calc cK^\calc Q^\calc cK^\calc\text{ \ and \ }Q^\calc cK^\calc\overset{\smash{Q^\calc (\epsilon)}}{\to}Q^\calc K^\calc\cong\Id,\]
where $\eta$ denotes the unit of the $Q^\calc\dashv K^\calc$ adjunction.
\end{prop}

Now the functor $c$ of the small object argument depends on the choice of sets of generating cofibrations and acyclic cofibrations. It will be important to our applications that included in this set are certain important cofibrations, namely the inclusion of $0$ into any sphere $F^\calc\mathbb{K}^t_{s_{r+1},\ldots,s_1}$, and any of the cofibrations  defined in \S\ref{three cell complex} and \S\ref{two-cell complex for the deltas} \textbf{$\lambda$?}.

It will be helpful to have included these maps, because of the following facts about the small object argument functor $cX$. It is constructed as the colimit of a (transfinite) sequence of cofibrations:
\[\xymatrix@!0@R=10mm@C=13mm{
\makebox[0cm][r]{$0={}$}c_0X\ar[rr] \ar@{ >->}[drrr]
&&%r1c1
c_1X\ar@{ >->}[rr]\ar[dr]
&&%r1c2
c_2X\ar@{ >->}[rr]\ar[dl]
&&%r1c3
c_3X\ar@{ >->}[rr]\ar[dlll]
&&%r1c4
\cdots \\
&&&X\!
}\]
and given an element $f:A\to B$ of the chosen set of generating cofibrations and a commuting square
\[\xymatrix@R=4mm{
A\ar[r]\ar@{ >->}[d]^-{f}&%r1c1
c_nX\ar[d]\\%r1c2
B\ar[r]&%r2c1
X%r2c2
}\]
there is a canonical choice of map $B\to c_{n+1}X$ making
\[\xymatrix@R=4mm{
A\ar[r]\ar@{ >->}[dd]^-{f}&%r1c1
c_nX\ar[d]\\%r1c2
&
c_{n+1}X\ar[d]\\%r1c2
B\ar[r]\ar[ur]&%r2c1
X%r2c2
}\]
commute. Indeed, the map $c_nX\to c_{n+1}X$ is constructed by attaching a copy of $B$ along the image of $A$ in $c_nX$, for each such commuting square.

We will use this canonical lift later, so establish a little notation. There is a function
\[\hom_{s\algs}(F^{\calc}\mathbb{K}_t,X)\to \hom_{s\algs}(F^{\calc}\mathbb{K}_t,c_1X)\]
denoted $\alpha\mapsto \widetilde{\alpha}$, natural in $X\in s\calc$, and providing a section of
\[\hom_{s\algs}(F^{\calc}\mathbb{K}_t,cX)\overset{\epsilon_*}{\to} \hom_{s\algs}(F^{\calc}\mathbb{K}_t,X)\]
 where we define $\widetilde{\alpha}$ to be the canonical lift corresponding to the square
\[\xymatrix@R=4mm{
0\ar[r]\ar@{ >->}[d]&%r1c1
c_0X\makebox[0cm][l]{${}:=0$}\ar[d]\\%r1c2
F^{\calc}\mathbb{K}_t\ar[r]^-{\alpha}&%r2c1
X%r2c2
}\]

Finally, we note that Radulescu-Banu's construction has a convenient consequence (albeit not crucial) for the construction of homotopy cofibres in $s\calc$. Quillen's small object argument actually provides a functorial factorization
\[\xymatrix@R=4mm{
X\ar@{ >->}[r]
&%r1c1
c^\textup{fac}(g)\ar@{->>}[r]^-{\sim}
& Y.
}\]
 of any map $g:X\to Y$, and in this notation, one might say that we have been writing $cX$ as shorthand for $c^\textup{fac}(0\to X)$. There is a commuting square
\[\xymatrix@R=4mm{
0\ar@{ >->}[r]\ar[d]
&%r1c1
cY\ar@{=}[d]
\\%r1c2
cX\ar[r]^{cg}&%r2c1
cY%r2c2
}\]
which by functoriality induces a commuting diagram (ignoring the dashed map):
\[\xymatrix@R=4mm{
0\ar@{ >->}[r]\ar[d]
&
ccY\ar@{->>}[r]^-{\epsilon_{cY}}
\ar[d]_-{m}
&%r1c1
cY\ar@{=}[d]
\\%r1c2
cX\ar@{ >->}[r]&%r2c1
c^\textup{fac}(cg)\ar@{->>}[r]_-{\sim}
&
cY\ar@{-->}[ul]|-{\beta}
%r2c2
}\]
Radulescu-Banu's diagonal $\beta$ is the dotted map in this diagram, and as it is a comonad diagonal $\epsilon_{cY}\circ \beta=\Id_{cY}$, so that $m\circ \beta$ is a section of the acyclic fibration $c^\textup{fac}(cg)\overset{\sim}{\epi} cY$.

Now we may define the homotopy cofiber as the pushout
\[\xymatrix@R=4mm{
cX\ar@{ >->}[r]\ar[d]&%r1c1
c^\textup{fac}(cg)\ar[d]\\%r1c2
0\ar[r]&%r2c1
\hocof(g)%r2c2
}\]
and by virtue of the construction just given:
\begin{prop}\label{hocof no ziggy-zaggy}
There is a natural construction in $s\calc$ of the homotopy cofiber $\hocof(g)$ of a map $g:X\to Y$, implemented by  natural maps
\[cX\overset{cg}{\to} cY\to \hocof(g).\]
\end{prop}
\noindent This is in contrast to the standard situation, where there is at best a natural zig-zag, even from $cY$ to  $\hocof(g)$.
\subsection{Homology groups and $\calc$-$H_*$-coalgebras}\label{homology and Hcoalgs}
There is a commuting diagram
\[\xymatrix@R=4mm@C=20mm{
s\vect{+}{r}\ar[r]^-{Q^\calc cK^\calc }
\ar[d]^-{\pi_*}
&%r1c1
s\vect{+}{r}\ar[d]^-{\pi_*}
\\%r1c2
\vect{+}{r+1}\ar[r]^-{C_{\calc\HCoalg}}
&
\vect{+}{r+1}
}\]
in which we are using Dold's theorem (\ref{Dold's theorem}) to \emph{define} $C_{\calc\HCoalg}$, the \emph{cofree $\calc$-$H_*$-coalgebra comonad}.
By proposition \ref{QcK is a comonad} and the naturality of Dold's theorem, this is a comonad on $\vect{+}{r+1}$. A $\calc$-$H_*$-coalgebra is simply a coalgebra over this monad, i.e.\ any $h\in\vect{+}{r+1}$ equipped with a coaction map $h\to C_{\calc\HCoalg}h$ satisfying the standard compatibilities. The homology $H_*^\calc X$ of $X\in s\calc$ is a $\calc$-$H_*$-coalgebra using the map
\[\pi_*(Q^{\calc}cX)\overset{\pi_*(Q^{\calc}(\beta))}{\to}\pi_*(Q^{\calc}ccX)\overset{\pi_*(Q^{\calc}c(\eta))}{\to}\pi_*(Q^{\calc}cK^{\calc}Q^{\calc}cX)=C_{\calc\HCoalg}(\pi_*(Q^{\calc}cX)).\]
If $X\sim K^{\calc}V$ for some $V\in s\vect{+}{r}$, then $H_*^{\calc}X\cong C_{\calc\HCoalg}(\pi_*(V))$, and the coaction map of $H_*^{\calc}X$ is none other than the diagonal map of the comonad.

It follows from the definitions that the dual of a $\calc$-$H_*$-algebra is a $\calc$-$H^*$-algebra, but the reverse does not always hold.
\begin{prop}\label{something about dualisation}
If $V\in \vect{}{}$ and $X,X'\in \calc\HCoalg$, then \textbf{is this true? restrict? extend?}:
\begin{gather*}
\dual C_{\calc\HCoalg}V\cong F^{\calc\HAlg}\dual V;\\
Q^{\calc\HAlg}\dual X\cong \dual \textup{Pr}^{\calc\HCoalg} X;\\
\dual(X\times X')\cong \dual X\sqcup \dual X';\makebox[0cm][l]{\ \ and}\\
\dual(X\smashprod X')\cong \dual X\smashcoprod \dual X'.\textbf{not defined yet}
\end{gather*}
\end{prop}
\subsection{The Hurewicz map, primitives and homology completion}\label{The Hurewicz map, primitives and homology completion}
For any $X\in s\calc$, there is a map $\pi_*X\to H_*^{\calc}X$, the \emph{Hurewicz map}, defined as the composite
\[\pi_*X\cong \pi_*(cX)\to \pi_*(Q^{\calc}cX).\]
Indeed, the Hurewicz map provides a coaugmentation of the comonad $C_{\calc\HCoalg}$, the natural transformation $a:\Id\to C_{\calc\HCoalg}$ of endofunctors of $\vect{+}{r+1}$ defined by
\[V\cong \pi_*(cK^{\calc}\Gamma  V)\to \pi_*(Q^{\calc}cK^{\calc}\Gamma  V)=F^{\calc\PiAlg}V.\]
One reading of this observation is:
\begin{lem}\label{hurewicz is a section}
If $X\in s\calC$ is in the image of $K^\calc$, then $Q^\calc X=U^{\calc}X$, and the Hurewicz map of $X$ is a section of the composite
\[H_*^{\calc}X:=\pi_*Q^{\calc}cX\overset{(Q\epsilon)_*}{\to}\pi_*Q^{\calc}X=\pi_*X.\]
\end{lem}
Given that the comonad $C_{\calc\HCoalg}$ has a coaugmentation, we may define the \emph{primitives} of a $\calc$-$H_*$-coalgebra $H$ as the equaliser (of graded vector spaces):
\[\xymatrix@R=4mm@1{
\textup{Pr}(H)\ar[r]
&%r1c1
H\ar@<.5ex>[r]^-{a}
\ar@<-.5ex>[r]_-{\textup{coact}}
&%r1c2
CH%r1c3
}.\]
%$\pi_*(V)\cong \pi_*(cKV)\to\pi_*(QcKV)$, in the sense that $\epsilon\circ a=1$ and $a\circ a=\Delta\circ a$. In particular, the monomorphism $a$ presents $W$ as a subspace of $CW$, and if $H$ is any $C$-coalgebra, we may define the \emph{primitives} in $H$ 
%``Include a bit about colimits, as in \verb|adamsops|''(?????)
%\begin{prop}
%For $W$ a graded vector space, the map $a:W\to CW$ is an isomorphism onto the primitives $\textup{Pr}(CW)$ of the cofree $C$-coalgebra $CW$.
%\end{prop}
We will briefly defer the proof of:
\begin{prop}\label{hurewicz}
The Hurewicz map $\pi_*X\to H_*^{\calc}X$ factors through $\textup{Pr}(H_*^{\calc}X)$, and if $X$ is GEM, the resulting map $\pi_*X\to\textup{Pr}(H_*^\calc X)$ is an isomorphism. In particular, for any $V\in\vect{+}{r+1}$, $\textup{Pr}(C_{\calc\HCoalg}V)\cong V$.
\end{prop}
%\begin{proof}
%Represent an element $\alpha\in\pi_nA$ by a map $\alpha:S^n_\calc\to A$. Then, $h(\alpha)$ is the image of the fundamental homology class $\imath\in H_n(S^n_\calc)$ under the map $\alpha_*$. As $\imath$ is primitive, so is $h(\alpha)$.
%\end{proof}

Radulescu-Banu \cite{Radulescu-Banu.pdf} has constructed a cosimplicial resolution $\calX^\bullet$ of an object $X\in s\calc$ by GEMs, and defined the \emph{homology completion of $X$} to be the totalization $X\hat{\ }:=\textup{Tot}(\calX^\bullet)$. This construction is the analogue of Bousfield and Kan's $R$-completion functor on simplicial sets \cite{BousKanSSeq.pdf}, a construction that has proven extremely useful in classical homotopy theory.

There is an additional difficulty, however,  in constructing the cosimplicial resolution $\calX^{\bullet}$, which is not present in the classical context: since not all simplicial algebras are cofibrant, the naive cosimplicial resolution (with coaugmentation dashed)
\[\makebox[0cm][r]{\,$\qquad $}\vcenter{
\def\labelstyle{\scriptstyle}
\xymatrix@C=1.5cm@1{
X\,
\ar@{-->}[r]
&
\,K^{\calc}Q^{\calc}X\,
\ar[r];[]
&
\,(K^{\calc}Q^{\calc})^2X\,
\ar@<-.65ex>[l];[]
\ar@<+.65ex>[l];[]
\ar@<+.65ex>[r];[]
\ar@<-.65ex>[r];[]
&
\,(K^{\calc}Q^{\calc})^3X\,\makebox[0cm][l]{\,$\cdots. $}
\ar[l];[]
\ar@<-1.3ex>[l];[]
\ar@<+1.3ex>[l];[]
}}\]
fails to be homotopically correct, and  as $Q^{\calc}K^{\calc}=\Id$, it is rather uniteresting. Radulescu-Banu's innovation was to explain that the cofibrant replacement functor $c:s\calc\to s\calc$ constructed by Quillen's small object argument \cite{QuillenHomAlg.pdf} admits a comonad diagonal $\beta:c\to cc$  (already used in \S\ref{homology and Hcoalgs}) and can thus be mixed into the cosimplicial resolution, making it homotopically correct. Specifically, the diagonal is needed in order  to define the coface maps in a coaugmented cosimplicial object:
\[\makebox[0cm][r]{\,$\calX^\bullet:\qquad $}\vcenter{
\def\labelstyle{\scriptstyle}
\xymatrix@C=1.5cm@1{
cX\,
\ar@{-->}[r]
&
\,cK^{\calc}Q^{\calc}cX\,
\ar[r];[]
&
\,c(K^{\calc}Q^{\calc}c)^2X\,
\ar@<-.65ex>[l];[]
\ar@<+.65ex>[l];[]
\ar@<+.65ex>[r];[]
\ar@<-.65ex>[r];[]
&
\,c(K^{\calc}Q^{\calc}c)^3X\,\makebox[0cm][l]{\,$\cdots. $}
\ar[l];[]
\ar@<-1.3ex>[l];[]
\ar@<+1.3ex>[l];[]
}}\]
By an application of Dold's theorem (\ref{Dold's theorem}), if $X\to Y$ is a weak equivalence, so is $\calx^s\to\calY^s$ for each $s$.

The construction is explained and generalized by Blumberg and Riehl \cite[\S4]{BlumRiehlResolutions.pdf}: instead of simply using the unit and counit of the adjunction respectively, one uses the composites discussed in \S\ref{homology and Hcoalgs}:
\[c\overset{\beta}{\to}cc\overset{c\eta c}{\to}cKQc\textup{\quad and\quad }QcK\overset{Q\epsilon K}{\to}QK\to \textup{id}.\]
One says that $X$ is \emph{homology complete} when the map $cX\to X\hat{\ }:=\textup{Tot}(\calX^\bullet)$ is an equivalence.


Comments in \cite[\S4]{BlumRiehlResolutions.pdf}  show that the coaugmented cosimplicial $\calc$-$H_*$-coalgebra $H_*^{\calc}\calX^\bullet$  is weakly equivalent to its coaugmentation $H_*^{\calc}X$ \emph{as a vector space}. We will return to this construction in \S\ref{?????????}.

\begin{proof}[Proof of proposition \ref{hurewicz}]
Note that the maps $d^0,d^1:\calX^{0}\to\calX^{1}$ induce respectively the coaugmentation and coaction maps for $\pi_*\calx^{0}=H_*^{\calc}X$ on homotopy, while $d^0:\calX^{-1}\to\calX^{0}$ induces the Hurewicz map. The very existence of this diagram then shows that the the Hurewicz map factors through the primitives. The following observation of \cite[\S4]{BlumRiehlResolutions.pdf} completes the proof: that this cosimplicial object has extra codegeneracies  when $X=K^\calc V$.
\end{proof}

\subsection{The smash product}\label{subseq:The smash product}
For $X_1$ and $X_2$ objects of any coalgebraic category, in particular $\calC\HCoalg$, we define the \emph{smash product} $X_1\smashprod X_2$ to be the cokernel of the natural map $X_1\sqcup X_2\to X_1\times X_2$. 

{\tiny Finally, we will sometimes need to move from $\calc$-$H_*$-coalgebras to $\calc$-$H^*$-algebras by dualization. \textbf{was going to put a proposition here...}}

The theory changes a little in form after passing from homotopy to homology, and we must introduce the \emph{left derived smash product} in $\calc$. For $A_1$ and $A_2$ in $s\calc$, the natural map $A_1\sqcup A_2\to A_1\times A_2$ is a surjection, and so in general very far from a cofibration. We define the \emph{left derived smash product} $A_1\Lsmashprod A_2$ to be the cofiber of this map.
In light of proposition \ref{hocof no ziggy-zaggy}, there are natural maps
\[c(A_1\sqcup A_2)\to c(A_1\times A_2)\to A_1\Lsmashprod A_2.\]
As in \cite[Proposition 4.6]{MR1089001}, this \emph{cofiber sequence} induces  a homology long exact sequence.
%
%
%Using any a functorial cofibration/acyclic fibration factorization on $s\calc$, form a natural commuting diagram
%\[\xymatrix@R=4mm{
%c(A_1\sqcup A_2)\ar@{ >->}[r]\ar@{->>}[d]^-{\epsilon}_-{\sim}
%&%r1c1
%M\ar@{ ->>}[r]^-{\sim} &%r1c2
%c(A_1\sqcup A_2)\ar@{->>}[d]^-{\epsilon}_-{\sim}\\%r1c3
%A_1\sqcup A_2\ar[rr]&%r2c1
%&%r2c2
%A_1\times A_2%r2c3
%}\]
%Then $A_1\Lsmashprod A_2$ is defined as the pushout in $s\calc$:
%\[\xymatrix@R=4mm{
%c(A_1\sqcup A_2)\ar@{ >->}[r]\ar[d]&%r1c1
%M\ar[d]\\%r1c2
%0\ar[r]&%r2c1
%A_1\Lsmashprod A_2%r2c2
%}\]
%%There is a zigzag $c(A_1\sqcup A_2)\overset{\sim}{\from}M\to A_1\Lsmashprod A_2$. 
%If we use Quillen's small object argument to construct the functorial factorization through $M$  (c.f.\ \cite{QuillenHomAlg.pdf}) it happens that Radulescu-Banu's construction can be extended, not only to produce a diagonal $\psi:c\to cc$, but to produce a natural map $c(A_1\sqcup A_2)$ to a commuting diagram{}

The following result and its proof are dual to proposition \ref{smash coprod} and its proof.
\begin{prop}\label{smash prod}
For $X_1$ and $X_2$ in $s\calc$, the natural $\calC$-$H_*$-coalgebra map
\[H^\calc_*(X_1\sqcup X_2)\from H^\calc_* X_1\sqcup H^\calc_* X_2\]
 is an isomorphism. If $X_1$ and $X_2$ are GEMs in $s\calc$, the natural $\calC$-$H_*$-coalgebra map 
\[H^\calc_* X_1\times H^\calc_* X_2\from H^\calc_*(X_1\times X_2)\]
takes part in an isomorphism of short exact sequences:
\[\xymatrix@R=4mm{
0\ar[r];[]&%r1c1
H^\calc_* X_1\smashprod H^\calc_* X_2
\ar[r];[]\ar[d];[]^-{\cong}
%\ar[d];[]_-{i}_-{\cong}
&%r1c2
H^\calc_* X_1\times H^\calc_* X_2
\ar[r];[]
\ar[d];[]^-{\cong}\ar[d];[]^-{\cong}
&%r1c3
H^\calc_* X_1\sqcup H^\calc_* X_2
\ar[r];[]
\ar[d];[]^-{\cong}
&%r1c4
0\\%r1c5
0\ar[r];[]&%r1c1
H^\calc_* (X_1\Lsmashprod X_2)
\ar[r];[]
&%r1c2
H^\calc_* (X_1\times X_2)
\ar[r];[]
&%r2c3
H^\calc_* (X_1\sqcup X_2)
\ar[r];[]&0
}\]
\end{prop}





\subsection{The quadratic part of a $\calc$-expression}\label{quadratic part section}
In this paper, we will often use a method of constructing cohomology operations used by Goerss in \cite[\S5]{MR1089001}, and here we will set up a framework that can be applied to each case. Suppose that $\calc$ is an algebraic category, monadic over $\vect{}{}$, category of graded vector spaces. 



For $V\in\vect{}{}$, the diagonal map $\Delta:V\to V\oplus V$ of $V$ induces a \emph{diagonal map} $F^\calc V\to F^\calc (V\oplus V)\cong (F^\calc (V))^{\sqcup 2}$, and writing $i_1$ and $i_2$ for the two summand inclusions $F^\calc (V)\to (F^\calc (V))^{\sqcup 2}$, consider the map
\[(F^\calc(\Delta)+i_1+i_2):F^\calc V\to (F^\calc V)^{\sqcup2}.\]
This map  factors through $(F^\calc V)^{\smashcoprod 2}$, and is symmetric. We name this factoring the \emph{cross terms}:
\[\crossterms:F^\calc V\to (F^\calc V)\smashcoprod^{\Sigma_2} (F^\calc V),\]
as it measures the non-linearity in an expression in $F^\calc V$. For example, when $\calc=\algs$, then for $v,w\in V$, using subscripts to denote membership of the first or second copy of $V$:
\[\crossterms(vw)=(v_1+v_2)(w_1+w_2)+v_1w_1+v_2w_2=v_1w_2+w_1v_2.\]

In each case of interest to us, we will give a map, natural and symmetric in $X_1,X_2\in\calc$,
\[j_\calc:Q^\calc(X_1\smashcoprod X_2)\to Q^\calc(X_1)\otimes Q^\calc(X_2),\ \ \textup{\textbf{coalgebra?}}\]
and we will write $\quadratic_\calc$ for \emph{quadratic part} map, the composite
\[\quadratic_\calc:\left(F^\calc V\overset{\crossterms}{\to}(F^\calc V)^{\smashcoprod 2}\to Q^\calc((F^\calc V)\smashcoprod^{\Sigma_2} (F^\calc V))\overset{j_\calc}{\to}S^2(Q^{\calc}F^\calc V)=S^2V\right).\]
When $\calc=\algs$, the map $j_{\algs}$ is defined by $x_1x_2\mapsto x_1\otimes x_2$ when $x_i\in X_i$, and $\quadratic_{\algs}$ is the composite $F^{\algs} V\epi S_2V\overset{\trace}{\to}S^2V$, for example,  $\quadratic_{\algs}(u+vw+xy^2)$ evaluates as:
\[j_{\algs}(v_1w_2+w_1v_2+x_1y_2^2+y_1^2x_2)=v\otimes w+w\otimes v+x\otimes y^2+y^2\otimes x,\]
which equals $v\otimes w+w\otimes v$ %when evaluated in $S^2(Q^{\algs}F^{\algs}V)=S^2V$, 
as hoped. \textbf{do Lie somewhere.}

In each category of interest to us, the following equation of maps $F^\calc F^\calc V\to S^2 V$ will always be satisfied:
\[\quadratic_\calc\circ\mu_V=\quadratic_\calc\circ \epsilon_{F^\calc V} +\quadratic_\calc\circ {F^\calc \epsilon_V},\]
where $\mu$ here stands for the multiplication map of the monad $U^\calc F^\calc $, which is to say that if  $f(g_i)$ is a $\calc$-expression in various $\calc$-expressions $g_i(v_{ij})$, we have 
\[\quadratic(fg_i)(v_{ij})=\quadratic(f\epsilon(g_i))(v_{ij})+\epsilon(f)(\quadratic(g_i)(v_{ij})).\]
This is another expression of homogeneity in the relations defining $\calc$. For an example when $\calc=\algs$, we specify an expression $f(g_1,g_2,g_3):=g_1g_2+g_3\in F^\calc F^\calc V$ in expressions $g_i:=v_{i1}v_{i2}+v_{i3}\in F^\calc V$ for each $i=1,2,3$. Then
\begin{alignat*}{2}
\textup{\small$\quadratic(fg_i)(v_{ij})={}$}
&
\textup{\small$\quadratic((v_{11}v_{12}+v_{13})(v_{21}v_{22}+v_{23})+(v_{31}v_{32}+v_{33}))=\trace(v_{13}\otimes v_{23}+v_{31}\otimes v_{32}),$}
\\
\textup{\small$\quadratic(f\epsilon(g_i))(v_{ij})={}$}
&
\textup{\small$\quadratic((v_{13})(v_{23})+(v_{33}))=\trace(v_{13}\otimes v_{23}),\textup{ and}$}
\\
\textup{\small$\epsilon(f)(\quadratic(g_i)(v_{ij}))={}$}
&
\textup{\small$\quadratic(v_{31}v_{32}+v_{33})=\trace(v_{31}\otimes v_{32}).$}
\end{alignat*}










\end{CPiAlgs and CHalgs}

\begin{BK spec seq}
\section{\textbf{The Bousfield-Kan spectral sequence}}\label{Bousfield-Kan spectral sequence}



In this chapter, we will write $\algcat$ for any category of universal graded $\Ftwo $-algebras satisfying the standing assumptions of \S\ref{Universal algebras}. In \S\ref{sec:connectivityAnalysis} we will specialize to the case when $\algcat$ is either the category $\algs$  of ungraded non-unital commutative algebras the category $\liealgs$ of ungraded Lie algebras, and prove the following conjecture of Radulescu-Banu:
\begin{thm}\label{completenesstheorem}
If either $\algcat=\algs$ or $\algcat=\liealgs$ and $X\in s\algcat$ is connected, then $X$ is naturally equivalent to its homology completion $X\hat{\ }$.
\end{thm}
We are particularly interested in this result because $\pi_*X\hat{\ }$ is the intended target of the \emph{Bousfield-Kan spectral sequence of $X\in s\algcat$}, the second quadrant homotopy spectral sequence (c.f.\ \S\ref{Towers, exact couples and coaugmented cosimplicial objects}) of the resolution $\calX\in cs\algcat$ of $X$. We will also prove that, when $X\in s\algs$ is a connected object  with $H^*_\algs$ of finite type, the spectral sequence supports a vanishing line at $E^2$, in \textbf{\S????}.



\subsection{Identification of $E^1$ and $E^2$}\label{Idnt E1 E2}
In light of \S\ref{Cofibrant replacement via the small object argument} and \S\ref{homology and Hcoalgs}, applying the functor $H_*^{\algcat}$ to $\calX$, one obtains precisely the monadic cobar resolution of $H_*^{\algcat}X$ obtained by repeated application of the monad on $\algcat\HCoalg$ of the adjunction 
\[U^{\algcat\HCoalg}:\algcat\HCoalg\rightleftarrows \vect{}{1}:C^{\algcat\HCoalg}.\]
In more detail, we have a map of coaugmented cosimplicial objects
\[\vcenter{
\def\labelstyle{\scriptstyle}
\xymatrix@C=1.5cm@1{
cX\,
\ar@{-->}[r]\ar@{->>}[d]
&
\,cKQcX\,
\ar[r];[]\ar@{->>}[d]
&
\,c(KQc)^2X\,
\ar@<-.65ex>[l];[]
\ar@<+.65ex>[l];[]
\ar@<+.65ex>[r];[]
\ar@<-.65ex>[r];[]\ar@{->>}[d]
&
\,\cdots
\ar[l];[]
\ar@<-1.3ex>[l];[]
\ar@<+1.3ex>[l];[]\\
QcX\,
\ar@{-->}[r]
&
\,QcKQcX\,
\ar[r];[]
&
\,Qc(KQc)^2X\,
\ar@<-.65ex>[l];[]
\ar@<+.65ex>[l];[]
\ar@<+.65ex>[r];[]
\ar@<-.65ex>[r];[]
&
\,\cdots
\ar[l];[]
\ar@<-1.3ex>[l];[]
\ar@<+1.3ex>[l];[]
}}\]
Write `$C$' for the monad $C^{\algcat\HCoalg}U^{\algcat\HCoalg}$ on $\algcat\HCoalg$. Applying  $\pi_*$, we obtain a cosimplicial Hurewicz map:
\[\vcenter{
\def\labelstyle{\scriptstyle}
\xymatrix@R=1.2cm@C=3.3cm@!0{
{\qquad\qquad\pi_*\calX^\bullet:\!\!\!\!\!\!\!\!\!\!\!\!\!\!}&
\pi_*X\,
\ar@{-->}[r]\ar@{->}[d]
&
\,\pi_*\calX^0\,
\ar[r];[]\ar[d]^-{\cong}
&
\,\pi_*\calX^1\,
\ar@<-.65ex>[l];[]
\ar@<+.65ex>[l];[]
\ar@<+.65ex>[r];[]
\ar@<-.65ex>[r];[]\ar[d]^-{\cong}
&
\,{\cdots\qquad\qquad}
\ar[l];[]
\ar@<-1.3ex>[l];[]
\ar@<+1.3ex>[l];[]
\\
{\qquad\qquad\textup{Pr}(H_*^\algcat X^\bullet):\!\!\!\!\!\!\!\!\!\!\!\!\!\!}&
\textup{Pr}(H_*^\algcat X)\,
\ar@{-->}[r]
\ar@{ >->}[d]
&
\,\textup{Pr}(CH_*^\algcat X)\,
\ar[r];[]
\ar@{ >->}[d]
&
\,\textup{Pr}(C^2H_*^\algcat X)\,
\ar@<-.65ex>[l];[]
\ar@<+.65ex>[l];[]
\ar@<+.65ex>[r];[]
\ar@<-.65ex>[r];[]
\ar@{ >->}[d]
&
\,{\cdots\qquad\qquad}
\ar[l];[]
\ar@<-1.3ex>[l];[]
\ar@<+1.3ex>[l];[]
\\
{\qquad\qquad H_*^\algcat X^\bullet:\!\!\!\!\!\!\!\!\!\!\!\!\!\!}&
H_*^\algcat X\,
\ar@{-->}[r]
&
\,CH_*^\algcat X\,
\ar[r];[]
&
\,C^2H_*^\algcat X\,
\ar@<-.65ex>[l];[]
\ar@<+.65ex>[l];[]
\ar@<+.65ex>[r];[]
\ar@<-.65ex>[r];[]
&
\,{\cdots\qquad\qquad}
\ar[l];[]
\ar@<-1.3ex>[l];[]
\ar@<+1.3ex>[l];[]
}}
\]
The marked maps are isomorphisms since each $\calX^s$ for $s\geq0$ is a GEM, thanks to proposition \ref{hurewicz}.
In particular, we see that:
\begin{alignat*}{2}
\E{1}{}{\calX}{s}{t}&\cong (\textup{Pr}(C^sH_*^{\algcat}X))_{t};\\
\E{2}{}{\calX}{s}{t}&\cong((\mathbb{R}^s\textup{Pr})H_*^{\algcat}X)_t.
\end{alignat*}
Corollary \ref{finite type pres by FW0}, corollary \textbf{Lie version} and proposition \ref{something about dualisation} show that
\begin{thm}\label{identify E2 with derived Q}
If $\algcat$ is either $\algs$ or $\liealgs$, and $X$ is connected with $H^*_\calc X$  of finite type, then $H^*_\calc\calX^s$ is of finite type for each $s$, and:
\begin{alignat*}{2}
\E{1}{}{\calX}{s}{t}&\cong (C^*\dual Q^\algcat B^\algcat H^*_{\algcat}X)^{s}_{t};\\
\E{2}{}{\calX}{s}{t}&\cong (H^*_{\algcat\HAlg}H^*_{\algcat}X)^{s}_{t}.
\end{alignat*}
\end{thm}





\subsection{The Adams tower}\label{sec:derWRTab}\label{sec:relnWithRB}
Bousfield and Kan defined the Bousfield-Kan spectral sequence, or \emph{unstable Adams spectral sequence}, of a simplicial set in two different ways. Their earlier approach \cite{BK_pairings.pdf} was to define the \emph{derivation of a functor with respect to a ring}. This approach constructs the \emph{Adams tower} over the simplicial set in question, and lends itself well to connectivity analyses. Their latter approach, \cite{BousKanSSeq.pdf}, to give a cosimplicial resolution of a simplicial set by simplicial $R$-modules, lends itself more to the analysis of the $E^2$ page, and is directly analogous to Radulescu-Banu's construction described in \S\ref{The Hurewicz map, primitives and homology completion}.

Since the release of \cite{BK_pairings.pdf} and \cite{BousKanSSeq.pdf}, the relationship between the two approaches has been clarified by the introduction of cubical homotopy theory \cite{GoodwillieCalcII}. In this section we will define the Adams tower of a simplicial algebra using a construction analogous to Bousfield and Kan in \cite{BK_pairings.pdf}, and then apply the theory of cubical diagrams to relate it to Radulescu-Banu's construction. 


For any functor $F:s\algcat\to s\algcat$, we define the $r^\textup{th}$ derivation $\dupdown{r}{c}F$ of $F$ with respect to  homology. The definition is recursive, and again involves repeated application of the functor $c$: 
\begin{alignat*}{2}
(\dupdown{0}{c}F)(X)
&:=
F(cX)%
\\
(\dupdown{s}{c}F)(X)
&:=
\hofib\bigl((\dupdown{s-1}{c}F)(cX)\xrightarrow{(\dupdown{s-1}{c}F)(\eta_{cX})} (\dupdown{s-1}{c}F)(KQcX)\bigr)
\end{alignat*}
where $\eta$ is the unit of the adjunction $Q\dashv K$, i.e.\ the natural surjection onto indecomposables, and $\hofib$ is any fixed  functorial construction of the homotopy fiber. These functors fit into a tower via the following composite natural transformation:
\[\delta:\left((\dupdown{s}{c}F)(X)\to (\dupdown{s-1}{c}F)(cX)\overset{(\dupdown{s-1}{c}F)(\epsilon)}{\to} (\dupdown{s-1}{c}F)(X)\right).\]
We have thus constructed a tower
\[\xymatrix@R=4mm{
\cdots 
\ar[r]
&%r1c1
(\dupdown{2}{c}F)X
\ar[r]
&%r1c3
%r1c4
(\dupdown{1}{c}F)X
\ar[r]&%r1c5
(\dupdown{0}{c}F)X=FcX,
}\]%reverse me
which is natural in the object $X$ and the functor $F$.
The functors $\dupdown{r}{c}F$ are homotopical as long as $F$ preserves weak equivalences between cofibrant objects. Employing the shorthand
\[\dupdown{s}{c}X:=(\dupdown{s}{c}\Id )X,\]
we define \emph{the Adams tower of $X$} to be the tower
\[\xymatrix@R=4mm{
\cdots 
\ar[r]
&%r1c1
\dupdown{2}{c}X
\ar[r]
&%r1c3
%r1c4
\dupdown{1}{c}X
\ar[r]&%r1c5
\dupdown{0}{c}X=cX.
}\]

For example, $(\dupdown{2}{c}F)(X)$ is constructed by the following diagram in which every composable pair of parallel arrows is \emph{defined} to be a homotopy fiber sequence.
\[\def\labelstyle{\scriptstyle}
\xymatrix@!0@R=30pt@C=43pt{
(\dupdown{2}{c}F)(X)\ar[d]\\
(\dupdown{1}{c}F)(cX) \ar[rr]\ar[d]         &           &FcccX \ar[rr]\ar[d]         &           &   FcKQccX            \ar[d]  &                  \\
(\dupdown{1}{c}F)(KQcX) \ar[rr] &                     &  FccKQcX \ar[rr] &             & FcKQcKQcX
}\]
In general, $(\dupdown{n+1}{c}F)(X)$ is the homotopy total fiber of an $(n+1)$-cubical diagram:
\[(\dupdown{n+1}{c}F)(X):=\textup{hototfib} \bigl(({\plainD}_{n+1}^{\smash{\square}}F)X\bigr).\]
See \cite{GoodwillieCalcII}, \cite{LuisGoodwillie.pdf} or \cite{CubicalHomotopyTheory.pdf} for the general theory of cubical diagrams. Before defining the cubical diagram $({\plainD}_{n+1}^{\smash{\square}}F)X$, we set notation: for $n\geq0$ let $[n]=\{0,\ldots,n\}$, and define $\calP[n]=\left\{S\subseteq [n]\right\}$ to be the poset category whose morphisms are the inclusions $S\subseteq S'$. Then an $(n+1)$-cube in $s\algcat$ is a functor $\calP[n]\to s\algcat$, and the $(n+1)$-cubical diagram $({\plainD}_{n+1}^{\smash{\square}}F)X:\calP[n]\to s\algcat$ is the functor:
\[S\mapsto Fc(KQ)^{\chi_{n}}c(KQ)^{\chi_{n-1}}c\cdots c(KQ)^{\chi_0}cX\quad \textup{where}\quad \chi_i:=\begin{cases}
1,&\textup{if }i\in S;\\
0,&\textup{if }i\notin S,
%\\,&\textup{if }
\end{cases}
\]
where for $S\subseteq S'$, the map $(({\plainD}_{n+1}^{\smash{\square}}F)X)(S)\to (({\plainD}_{n+1}^{\smash{\square}}F)X)(S')$ is given by applying the counit $\eta:1\to KQ$ in those locations indexed by $S'\setminus S$.

%\subsection{Relationship between the Adams tower and Radulescu-Banu's resolution}\label{sec:relnWithRB}

Now Radulescu-Banu defines the homology completion of $X$ to be the totalization
\[X\hat{\ }:=\textup{Tot}(\calX^\bullet)=\holim (\textup{Tot}_n(\calX^\bullet)),\]
and the Bousfield-Kan spectral sequence to be the spectral sequence of the tower
\[\cdots \to\textup{Tot}_n(\calX^\bullet)\to \textup{Tot}_{n-1}(\calX^\bullet)\to\cdots \]
under $cX$. Our goal in this section is to prove
\begin{prop}\label{towerIdentification}
There is a natural zig-zag of weak equivalences of towers between $\left\{\plainD_{n+1}X\right\}_n$ and $\left\{\hofib(cX\to\textup{Tot}_{n}(\calX^\bullet))\right\}_n$. That is,  the $\textup{Tot}$ tower induces the Adams tower by taking homotopy fibers. In particular, the spectral sequence of the $\textup{Tot}$ tower coincides with the spectral sequence of the Adams tower.
\end{prop}
\begin{shaded}\tiny
Before giving the proof, we recall a useful relationship between cosimplicial objects and cubical diagrams, explained by Sinha in \cite[Theorem 6.5]{SinhaSpacesOfKnots.pdf}, and expanded on by Munson-Voli\'c \cite{CubicalHomotopyTheory.pdf}.
We will only present that part of the theory that we need. There is a diagram of inclusions of categories
\[\xymatrix@!0@R=10mm@C=13mm{
\calP[-1]\ar[rr]^-{\tau}\ar_-{h_{-1}}[drrr]
&&%r1c1
\calP[0]\ar[rr]^-{\tau}\ar^(.65){\!h_{0}}[dr]
&&%r1c2
\calP[1]\ar[rr]^-{\tau}\ar_(.65){h_{1}\!}[dl]
&&%r1c3
\calP[2]\ar[rr]^-{\tau}\ar^-{h_{2}}[dlll]
&&%r1c4
\cdots \\
&&&\Delta_+\!\!
}\]
As the coaugmented cosimplicial object $\calX^\bullet$ is Reedy fibrant (c.f.\ \cite[{X.4.9}]{YellowMonster}), there are natural weak equivalences $\hofib(\calX^{-1}\to\textup{Tot}_n\calX^\bullet) \overset{\smash{\sim}}{\to} \textup{hototfib}(h_n^*(\calX^\bullet))$ under which the tower map 
\[\hofib(\calX^{-1}\to\textup{Tot}_n\calX^\bullet)\to \hofib(\calX^{-1}\to\textup{Tot}_{n-1}\calX^\bullet)\]
is identified with the map
\[\textup{hototfib}(h_n^*(\calX^\bullet))\to \textup{hototfib}(\tau^*h_n^*(\calX^\bullet))=\textup{hototfib}(h_{n-1}^*(\calX^\bullet)).\]
As $\calX^{-1}$ equals $cX$, the tower $\hofib(\calX^{-1}\to\textup{Tot}_n\calX^{\bullet})$ is one of the towers in proposition \ref{towerIdentification}.
\end{shaded}
\begin{proof}[Proof of Proposition \ref{towerIdentification}] 
It will suffice to construct a weak equivalence $h_n^*\calX^\bullet\to (\plainD_n^{\smash{\square}}\Id  )(X)$ of $(n+1)$-cubes. The $(n+1)$-cubical diagram $h_n^*\calX^\bullet$ is defined by
\[(h_n^*\calX^\bullet)(S):= c(KQc)^{\chi_{n}}(KQc)^{\chi_{n-1}}\cdots (KQc)^{\chi_0}X\quad \textup{where}\quad \chi_i:=\begin{cases}
1,&\textup{if }i\in S,\\
0,&\textup{if }i\notin S,
\end{cases}
\]
where we describe the map $(h_n^*\calX^\bullet)(S)\to (h_n^*\calX^\bullet)(S\sqcup\{i\})$, for $i\notin S$, as follows. Let $j$ be the smallest element of $S\sqcup\{n+1\}$ exceeding $i$, so that
\[(h_n^*\calX^\bullet)(S):= \begin{cases}
c(KQc)^{\chi_{n}}\cdots (KQc)^{\chi_{j+1}}(KQ\underline{c})(KQc)^{\chi_{i-1}}\cdots (KQc)^{\chi_0}X,&\textup{if }j\leq n;\\
\underline{c}(KQc)^{\chi_{i-1}}\cdots (KQc)^{\chi_0}X,&\textup{if }j=n+1.
%\\,&\textup{if }
\end{cases}
\]
In the expression for either case, we have distinguished one of the applications of $c$ with an underline, and the map to $(h_n^*\calX^\bullet)(S\sqcup\{i\})$ is induced by the composite $\underline{c}\to cc\to cKQc$ of the diagonal of the comonad $c$ with the unit of the monad $KQ$. 

We now define maps $(h_n^*\calX^\bullet)(S)\to (({\plainD}_{n+1}^{\smash{\square}}\Id )X)(S)$ for $S=\{j_0<j_1<\cdots<j_r\}\subset\{0,\ldots,n\}$. 
The only difference between the domain and codomain is that in $(({\plainD}_{n+1}^{\smash{\square}}\Id )X)(S)$, all $n+2$ applications of $c$ are present, whereas in $(h_n^*\calX^\bullet)(S)$, only $r+2$ appear. The map is then
\[\beta^{n-j_r}KQ\beta^{j_r-j_{r-1}-1}KQ\beta^{j_{r-1}-j_{r-2}-1}KQ\cdots KQ\beta^{j_{1}-j_0-1}KQ\beta^{j_0}X\]
which is to say that we apply the iterated diagonal the appropriate number of times in each $c$ appearing in the domain. As $\beta$ is coassociative, this definition is unambiguous, and the resulting maps assemble to a weak equivalence of $(n+1)$-cubes. 
\end{proof}

\subsection{An alternate definition of the Adams tower}
We'll now give an alternate definition of the Adams tower, given in \cite{BK_pairings_products.pdf}, which is more suited for the definition of spectral sequence operations.

For $Z\in (\vect{}{})^{\Delta_+}$, the category of coaugmented cosimplicial vector spaces, Bousfield and Kan write $VZ$ for a ``path-like construction'' \cite[\S3.1]{BK_pairings_products.pdf} on $Z$ obtained by shifting $Z$ down and forgetting the 0th coface and codegeneracy. That is, $(VZ)^s:=(VZ)^{s+1}$, and:
\begin{align*}
\bigl((VZ)^s\overset{d^i}{\to} (VZ)^{s+1}\bigr)&:=\bigl(Z^{s+1}\overset{d^{i+1}}{\to} Z^{s+2}\bigr)\\
\bigl((VZ)^s\overset{s^i}{\to} (VZ)^{s-1}\bigr)&:=\bigl(Z^{s+1}\overset{d^{i+1}}{\to} Z^{s}\bigr)
\end{align*}
In fact, the unused coface $d^0$ induces a map $v:Z\to VZ$.

For $Y\in s\vect{}{}$, the standard simplicial path fibration (c.f.\ \cite[p.82]{BousKanSSeq.pdf}) produces $\Lambda Y\in s\vect{}{}$ by shifting down and restricting to a kernel:
\[\Lambda Y_s=\ker\bigl(d_{s+1}\cdots d_1:Y_{s+1}\to Y_0\bigr).\]
We forget the $0^\textup{th}$ face and degeneracy as before. This time the unused face map $d_0$ induces a fibration $\lambda:\Lambda Y\to Y$, and $\Lambda Y$ is  contractible.


Each of these constructions can be prolonged to an endofunctor of $(s\algs)^{\Delta_+}$, endofunctors which are  necessary for a key construction of Bousfield and Kan \cite{BK_pairings_products.pdf,BK_pairings.pdf}. Define an endofunctor $D^1$ of the category $(s\algs)^{\Delta_+}$ of augmented cosimplicial objects in $s\algs$, using the pullback (for $W\in (s\algs)^{\Delta_+}$):
\[\xymatrix@R=4mm{
D^1 W \ar[r]
\ar[d]^-{\delta}
&%r1c1
\Lambda VW \ar[d]^-{\lambda}
\\%r1c2
W \ar[r]^-{v}
&%r2c1
VW %r2c2
}\]
Then one can form a tower in $(s\algs)^{\Delta_+}$, (writing $D^n=D^1\circ\cdots D^1$):
\[\xymatrix@R=4mm{
\cdots 
\ar[r]
&%r1c1
D^2W
\ar[r]
&%r1c3
%r1c4
D^1W
\ar[r]&%r1c5
D^0W
\ar@{=}[r]
&
W.
}\]
Restricting to augmentations, there is a tower of fiber sequences in $s\algs$:
\[\xymatrix@R=4mm{
\cdots 
\ar[r]
&%r1c1
(D^2W)^{-1}
\ar[r]\ar[d]^-{d^0}
&%r1c3
%r1c4
(D^1W)^{-1}
\ar[d]^-{d^0}\ar[r]&%r1c5
(D^0W)^{-1}\ar[d]^-{d^0}
\ar@{=}[r]
&
W^{-1}
\\
&
(D^2W)^{0}
&
(D^1W)^{0}
&
(D^0W)^{0}
}\]
whose homotopy long exact sequences fit together to form an exact couple [reference], below which we may insert some dotted arrows to be defined:
\[\xymatrix@R=4mm{
\cdots 
\ar[r]
&%r1c1
\pi_{t-2}((D^2W)^{-1})
\ar[r]\ar[d]^-{d^0}
&%r1c3
%r1c4
\pi_{t-1}((D^1W)^{-1})
\ar[d]^-{d^0}\ar[r]&%r1c5
\pi_{t}((D^0W)^{-1})\ar[d]^-{d^0}
\ar@{=}[r]
&
\pi_{t}(W^{-1})
\\
&
\pi_{t-2}((D^2W)^{0})\ar@{-->}[ul]^-{\partial}
&
\pi_{t-1}((D^1W)^{0})\ar@{-->}[ul]^-{\partial}
&
\pi_{t-0}((D^0W)^{0})\ar@{-->}[ul]^-{\partial}
\\\cdots&
N_\subset^2\pi_tW^\bullet \ar[u]_-{\partial_\textup{it}}^-{\cong} \ar@{..>}[l]_-{d^1} \ar@{ >->}[d]_-{\sim}&
N_\subset^1\pi_tW^\bullet \ar[u]_-{\partial_\textup{it}}^-{\cong} \ar@{..>}[l]_-{d^1} \ar@{ >->}[d]_-{\sim}&
N_\subset^0\pi_tW^\bullet \ar[u]_-{\partial_\textup{it}}^-{\cong} \ar@{..>}[l]_-{d^1} \ar@{ >->}[d]_-{\sim}&
\\
\cdots&
C^2\pi_tW  \ar@{..>}[l]_-{d^1}\ar@{-->}[dl]_-{\partial}&
C^1\pi_tW  \ar@{..>}[l]_-{d^1}\ar@{-->}[dl]_-{\partial}&
C^0\pi_tW  \ar@{..>}[l]_-{d^1}\ar@{-->}[dl]_-{\partial}&
%\pi_t W^{-1} \ar[l]_-{d}
\\
\cdots\ar[r]&
H_t(F^2TW)\ar[r]\ar[u]&
H_t(F^1TW)\ar[r]\ar[u]&
H_t(F^0TW)\ar[r]\ar[u]&
}\]


\[\xymatrix@R=4mm{
\cdots&
C^2\pi_tW  \ar@{..>}[l]_-{d^1}&
C^1\pi_tW  \ar@{..>}[l]_-{d^1}&
C^0\pi_tW  \ar@{..>}[l]_-{d^1}
%\pi_t W^{-1} \ar[l]_-{d}
\\
&
H_{t-2}(\calK_2)\ar[d]\ar@{=}[u]
&
H_{t-1}(\calK_1)\ar[d]\ar@{=}[u]
&
H_{t}(\calK_0)\ar[d]\ar@{=}[u]
\\
\cdots\ar[r]&
H_{t-2}(\calT_2W)\ar[r]\ar@{-->}[ul]_-{\partial}
&
H_{t-1}(\calT_1W)\ar[r]\ar@{-->}[ul]_-{\partial}
&
H_t(\calT_0W)\ar[r]\ar@{-->}[ul]_-{\partial}
&0
}\]

The overwhelming virtue of this construction is the following observation of Bousfield and Kan \cite[Proposition 5.2]{BK_pairings_products.pdf}:
\begin{prop}\label{BK D1 is awesome}
The connecting map
\[\pi_t(\calX^{s})=\pi_t(V\calX^{s-1})\overset{\partial }{\to}\pi_{t-1}(D^1\calX)^s\]
in the homotopy long exact sequence of the fiber sequence $(D^1\calX)^{s-1}\to \calX^{s-1}\to V\calX^{s-1}$ induces an isomorphism of cochain complexes
\[N^s_\subset \pi_t \calX\subseteq N^{s-1}_\subset \pi_t V\calX \to  N^{s-1}_\subset \pi_{t-1} (D^1\calX).\]
\end{prop}
Consequently, the connecting map induces an equivalence
\[C^s_\subset \pi_t (D^1\calX)\to  C^s_\subset \pi_{t-1} (V\calX),\]
and 


Bousfield and Kan explain that the composite
\[N^\fraks\pi_\frakt(D^1X)\overset{\partial_\textup{conn}}{\from}N^\fraks\pi_{\frakt+1}(VX)\supset N^{\fraks+1}\pi_{\frakt+1}(X)\]
is an isomorphism, where we note that the condition to lie in $N^{s+1}(O)\subset O^{s+1}$ is stricter than the condition to lie in $N^s(VO)\subset O^{s+1}$, for $O$ a cosimplicial module. Note that this composite is a cochain map, the reason being that the extra ``$d^0$'' in the coboundary out of $N^{\fraks+1}\pi_{\frakt+1}(X)$ that does not appear in the coboundary out of $N^{\fraks}\pi_{\frakt+1}(VX)$ becomes null after application of $\partial_\textup{conn}$, which is to say that the composite
\[\pi_{\frakt+1}(X^{\fraks+1})\overset{d^0}{\to}\pi_{\frakt+1}(X^{\fraks+2})\overset{\partial_\textup{conn}}{\to}\pi_\frakt((D^1X)^{\fraks+1})\]
is zero, which is clear, as we are just looking at adjacent maps in the homotopy long exact sequence.

\begin{connectivity}
\subsection{Connectivity estimates in the Adams tower}
\label{sec:connectivityAnalysis}
We will now perform the necessary connectivity estimates in the Adams tower in order to prove theorem \ref{completenesstheorem}, namely:  
%
%We wish to prove the following connectivity result for the Adams tower:
\begin{prop}\label{convergenceProp}
Suppose that $\algcat$ is one of the categories $\algs$ or $\liealgs$, that $X\in s\algcat$ is connected, and that  $t\geq1$ and $q\geq2$. Then there is some $f(q,t)\geq t$ such that the map $\pi_q(\dupdown{f(q,t)}{c}X)\to\pi_q(\dupdown{t}{c}X)$ is zero.
\end{prop}
\noindent Note that proposition \ref{towerIdentification} and \ref{convergenceProp} together imply theorem \ref{completenesstheorem}:
\begin{proof}[Proof of theorem \ref{completenesstheorem}]
The fiber sequences $\dupdown{n+1}{c}X\to cX\to \textup{Tot}_n\calX^\bullet$ fit together into a tower of fiber sequences. Taking homotopy limits, one obtains a fiber sequence
\[\holim (\dupdown{n}{c}X)\to cX\to X\hat{\ }.\]
We need to show that $\holim (\dupdown{n}{c}X)$ has zero homotopy groups.
%We may replace the tower $\dupdown{n}{c}X$ with a weakly equivalent tower of fibrations in $s\algcat$ whose set-theoretic inverse limit is the homotopy limit in question. 
Applying \cite[Proposition 6.14]{goerss-jardine.pdf}, there is a short exact sequence
\[0\to \textup{lim}^1\pi_{q+1}(\dupdown{n}{c}X)\to \pi_q(\holim (\dupdown{n}{c}X))\to \textup{lim}\,\pi_{q}(\dupdown{n}{c}X)\to 0.\]
Proposition \ref{convergenceProp} implies that for each $q$, the tower $\{\pi_{q}(\dupdown{n}{c}X)\}_n$ has zero inverse limit and satisfies the Mittag-Leffler condition (c.f.\ \cite[p.264]{YellowMonster}), so that the $\textup{lim}^1$ groups appearing also vanish.
\end{proof}
The application of the small object argument functor $c$ adds to the difficulty of proving the connectivity estimates of proposition \ref{convergenceProp}. We circumvent the difficulty of working with $c$ by shifting to the standard bar construction $B^{\algCat}$ on $s\algCat$, for which we use the abbreviation $\barConstructionMightAbbreviate $.

Define recursively a somewhat less homotopical version $\caldup{s}F$ of the derivations $\dupdown{s}{}F$:
\begin{alignat*}{2}
(\caldup{0}F)(X)
&:=
F(X)%
\\
(\caldup{s}F)(X)
&:=
\ker((\caldup{s-1}F)(\barConstructionMightAbbreviate X)\xrightarrow{(\caldup{s-1}F)(\eta_{\barConstructionMightAbbreviate X})} (\caldup{s-1}F)(KQ\barConstructionMightAbbreviate X))
\end{alignat*}
There are three differences between this definition and that of $\plainD_sF$: here, there is one fewer cofibrant replacement applied, we use $\barConstructionMightAbbreviate $ instead of  $c$, and we take \emph{strict} fibers, not homotopy fibers.
While these functors are not generally homotopical, we define \emph{the modified Adams tower of $X$} to be the tower
\[\xymatrix@R=4mm{
\cdots 
\ar[r]^{\delta}
&%r1c1
\caldup{2}X
\ar[r]^{\delta}
&%r1c3
%r1c4
\caldup{1}X
\ar[r]^{\delta}&%r1c5
\caldup{0}X\makebox[0cm][l]{${}=X$,}
}\]
where $\caldup{s}X$ is again shorthand for $(\caldup{s}\Id )X$, and the tower maps $\delta$ are defined as before.
\begin{prop}\label{prop:modifiedAdamsTower}
There is a natural zig-zag of weak equivalences of towers between the Adams tower of $X$ and the modified Adams tower of $X$. In particular, the modified Adams tower is homotopical.
\end{prop}
\begin{proof}
Let $\mathsf{CR(s\algcat)}$ be the category of cofibrant replacement functors in $s\algcat$. That is, an object of $\mathsf{CR(s\algcat)}$ is a pair, $(f,\epsilon)$, such that $f:s\algcat\to s\algcat$ is a functor whose image consists only of cofibrant objects, and $\epsilon:f\Rightarrow \Id $ is a natural acyclic fibration. Morphisms in $\mathsf{CR(s\algcat)}$ are natural transformations which commute with the augmentations. For any $(f,\epsilon)\in\mathsf{CR(s\algcat)}$ we obtain an alternative definition of the derivations of a functor $F:s\algcat\to s\algcat$:
\begin{alignat*}{2}
(\plainD_{\smash{0}}^{\smash{f}}F)(X)
:=
F(fX),\quad %
(\plainD_{\smash{s}}^{\smash{f}}F)(X)
:=
\hofib((\plainD_{\smash{s-1}}^{\smash{f}}F)(fX)\to (\plainD_{\smash{s-1}}^{\smash{f}}F)(KQfX)).
\end{alignat*}
These functors are natural in $f$, so that a morphism in $\mathsf{CR(s\algcat)}$ induces a weak equivalence of towers. Our proposed zig-zag of towers is: 
\[\plainD_{\smash{s}}= \plainD_{\smash{s}}^{\smash{c}}\Id \overset{}{\from}\plainD_{\smash{s}}^{\smash{\barConstructionMightAbbreviate \circ c}}\Id \overset{}{\to}\plainD_{\smash{s}}^{\smash{\barConstructionMightAbbreviate }}\Id \overset{\gamma_s}{\from}\caldup{s}\barConstructionMightAbbreviate \overset{\caldup{s}\epsilon}{\to}\caldup{s}\Id =\caldup{s}\]
The maps with domain $\plainD_{\smash{s}}^{\smash{\barConstructionMightAbbreviate \circ c}}\Id $ are induced by the maps $\epsilon c:\barConstructionMightAbbreviate \circ c\to c$ and $\barConstructionMightAbbreviate \epsilon:\barConstructionMightAbbreviate \circ c\to \barConstructionMightAbbreviate $ and are evidently natural weak equivalences of towers. The map $\gamma_0:(\caldup{0}\barConstructionMightAbbreviate )X\to (\plainD_{\smash{0}}^{\smash{\barConstructionMightAbbreviate }}\Id )X$ is the identity of $\barConstructionMightAbbreviate X$, and the map $\caldup{0}\epsilon:(\caldup{0}\barConstructionMightAbbreviate )X\to (\plainD_{\smash{0}}^{\smash{\barConstructionMightAbbreviate }}\Id )X$ is $\epsilon:\barConstructionMightAbbreviate X\to X$. Thereafter, $\gamma_s$ and $\caldup{s}\epsilon$ are defined recursively:
\[\qquad \quad \xymatrix@R=4mm{
\makebox[0cm][r]{$(\caldup{s+1}\Id )X:=\,$}\makebox[5cm][l]{$\,\ker((\caldup{s}\Id )(\barConstructionMightAbbreviate X)\to (\caldup{s}\Id )(KQ\barConstructionMightAbbreviate X))$}\\
\makebox[0cm][r]{$(\caldup{s+1}\barConstructionMightAbbreviate )X:=\,$}\makebox[5cm][l]{$\,\ker((\caldup{s}\barConstructionMightAbbreviate )(\barConstructionMightAbbreviate X)\to (\caldup{s}\barConstructionMightAbbreviate )(KQ\barConstructionMightAbbreviate X))$}\ar[d]^-{\textup{incl.}}_-{}%m_2}
\ar[u]_-{\textup{induced by }(\caldup{s}\epsilon,\caldup{s}\epsilon)}^-{}%m_1}
\\%r1c1
\makebox[5cm][l]{$\,\hofib((\caldup{s}\barConstructionMightAbbreviate )(\barConstructionMightAbbreviate X)\to(\caldup{s}\barConstructionMightAbbreviate )(KQ\barConstructionMightAbbreviate X))$}\ar[d]^-{\textup{induced by }(\gamma_{s},\gamma_{s})}_-{}%m_3}
\\%r2c1
\makebox[0cm][r]{$(\plainD_{\smash{s+1}}^{\smash{\barConstructionMightAbbreviate }})X:=\,$}\makebox[5cm][l]{$\,\hofib((\plainD_{\smash{s}}^{\smash{\barConstructionMightAbbreviate }}\Id )(\barConstructionMightAbbreviate X)\to (\plainD_{\smash{s}}^{\smash{\barConstructionMightAbbreviate }}\Id )(KQ\barConstructionMightAbbreviate X))$}%r3c1
}\]
\noindent Lemma \ref{towerWithPowers} shows that the kernels taken are actually kernels of surjective maps, and by induction on $s$, the maps $\gamma_s$ and $\caldup{s}\epsilon$ are weak equivalences.
\end{proof}
The connectivity result will rely on the observation that any element in the $s^\textup{th}$ level of the modified tower maps down to an $(s+1)$-fold expression in $X$. In order to formalise this, when $\algcat=\algs$, we let $P^s:s\algcat\to s\algcat$ be the ``$s^\textup{th}$ power'' functor, the prolongation of the endofunctor $Y\mapsto Y^s$ of $\algcat$, where $Y^s=\textup{im}(\textup{mult}:Y^{\otimes s}\to Y)$. When $\algcat=\algs$, we define $P^s:=\Gamma^s$, the $s^\textup{th}$ term in the lower central series filtration (c.f.\ \cite{6Author.pdf}). Then we have:
\begin{lem}\label{towerWithPowers}
Suppose that  either $\algcat=\algs$ or $\algcat=\liealgs$. The functors $\caldup{r}$, $\caldup{r}\barConstructionMightAbbreviate $ and $\caldup{r}P^{s}$ preserve surjective maps and there is a commuting diagram of functors:
\[\xymatrix@R=4mm{
\cdots 
\ar[r]
&%r1c1
\caldup{r}
\ar[r]
\ar[d]
&%r1c2
\cdots \ar[r]
&%r1c4
\caldup{2}
\ar[r]
\ar[d]
&%r1c4
\caldup{1}
\ar[r]
\ar[d]&%r1c5
\caldup{0}
\ar@{=}[d]
\\%r1c6
\cdots
\ar@{ >->}[r]
&%r2c1
P^{r+1}
\ar@{ >->}[r]
&%r2c3
\cdots 
\ar@{ >->}[r]&%r2c4
P^3
\ar@{ >->}[r]
&P^2
\ar@{ >->}[r]
&%r2c5
\Id %r2c6
}\]
\end{lem}
\begin{proof}
As $\barConstructionMightAbbreviate$ and $P^s$ preserve surjections, we need only check the claims about $\caldup{r}X$ for $X\in s\algcat$, which is constructed as the subobject
\[\caldup{r}X:= \bigcap_{i=1}^{r}\ker\!\left(\barConstructionMightAbbreviate^{r-i}\eta \barConstructionMightAbbreviate^{i}:\barConstructionMightAbbreviate^{r}X\to \barConstructionMightAbbreviate^{r-i}KQ\barConstructionMightAbbreviate^{i}X\right)\]
of $\barConstructionMightAbbreviate^rX$. In dimension $n$, this is the following subset of $(\barConstructionMightAbbreviate^rX)_n:=(F^{n+1})^rX_n$:
\[(\caldup{r}X)_n:=\bigcap_{i=1}^{r}\ker\!\left(F^{(r-i)(n+1)}\eta F^{i(n+1)}:(F^{n+1})^rX_n\to (F^{n+1})^{r-i}KQ(F^{n+1})^iX_n\right).\]
%Where we have written $F$ for $F^{\algcat}$.

Whichever of $\algs$ or $\liealgs$ we are working with, it is possible to construct monomial bases for $FV$ once a basis of $V$ has been chosen. For given $n$ and $r$, first choose a basis of $X_n$; build from it a monomial basis of $FX_n$; build from this a monomial basis of $F^2X_n$; etc. Continue until we have a monomial basis of $F^{r(n+1)}X_n=(b^rX)_n$. The effect of the map $F^{(r-i)(n+1)}\eta F^{i(n+1)}$ on monomials is either to annihilate them or leave them unchanged, depending on whether any non-trivial constructions were employed at the $((n+1)i)^\textup{th}$ stage.
Thus, the subset $(\caldup{r}X)_n$ has basis those iterated monomials in which some non-trivial costruction was used in the $((n+1)i)^\textup{th}$ for $1\leq i\leq r$. The image of such a monomial in $X_n$ lies in $P^r$.


To see that $\caldup{r}$ preserves surjections: if $f:X\to Y$ is a surjection, choose a basis $B\sqcup B'$ of $X_n$ for  which $f$ maps the $B$ bijectively onto a basis of $Y_n$ and $B'$ to zero. We may continue this pattern at each stage of the construction of iterated monomial bases of $F^{r(n+1)}X_n$ and $F^{r(n+1)}Y_n$. That is, we may choose a  basis $C\sqcup C'$ of $F^{r(n+1)}X_n$  such that the monomials in $C$ only involve the elements of $B$ and map under $f$ bijectively onto a basis of $F^{r(n+1)}Y_n$, and such that each monomial in $C'$ involves some element of $B'$, and so vanishes under $f$. This pattern is further preserved in passing to the monomial bases discussed earlier for $(\caldup{r}X)_n$ and $(\caldup{r}Y)_n$, proving the claim that $\caldup{r}$ preserves surjections. 
%
%Thus, choosing a basis of. One simply uses the standard monomial basis
%The $r$ conditions on elements of $(\caldup{r}X)_n$ ensure that their image in $(\caldup{0}X)_n=X_n$ under the iterated tower map is a sum of $(r+1)$-fold products. This completes the construction of the tower of functors.
%
%[\textbf{Reformulate for generalizability:} is not it just that except for simplicial structure you do not see more than a vector space?] In order to prove the surjectivity statements, we must describe the iterated free construction $F^{r(n+1)}X_n$. A basis of $FX_n$ may be given by the \emph{monomials} in a basis of $X_n$, and $F^{r(n+1)}X_n$ has basis given by taking monomials iteratively, $r(n+1)$ times. The subset $(\caldup{r}X)_n$ has basis those iterated monomials in which the monomials formed in the $((n+1)i)^\textup{th}$ iteration have degree at least two for each $1\leq i\leq r$. This simple description of a basis of $(\caldup{r}X)_n$ shows that $\caldup{r}$ preserves surjections. Similar analysis applies to $\caldup{r}\barConstructionMightAbbreviate $ and $\caldup{r}P^s$.
\end{proof}
We are now able to state and prove the key connectivity result:
\begin{lem}\label{connectivityOfDerivedPowers}
Suppose that $X\in s\algcat$ is connected, $t\geq1$ and $s\geq2$. If $\algcat=\algs$, then $(\caldup{t}P^{s})(X)$ is $(s-t)$-connected. If $\algcat=\liealgs$, then $(\caldup{t}P^{s})(X)$ is  $(\log_2(s)+1-t)$-connected.
\end{lem}
%\begin{lem}\label{connectivityOfDerivedPowers}
%For any connected $X\in s\algs$, and any $t\geq1$ and $s\geq2$, $(\caldup{t}P^{s})(X)$ is $(s-t)$-connected.
%For any connected $X\in s\Lambda(\LieOperad)$, and any $t\geq1$ and $s\geq2$, $(\caldup{t}P^{s})(X)$ is $(\log_2(s)+1-t)$-connected.
%\end{lem}
\begin{proof}
We will prove this by induction on $t$. The induction step is simple: %Now let $t\geq2$ and suppose by induction that $(\caldup{t-1}P^{s})(B)$ is $(s-(t-1))$-connected for any connected $B$ and any $s\geq2$. Then 
by lemma \ref{towerWithPowers}, there is a short exact sequence:
\[\xymatrix{
0\ar[r]&
(\caldup{t}P^{s})(X)\ar[r]&
(\caldup{t-1}P^{s})(\barConstructionMightAbbreviate X)\ar[r]&
(\caldup{t-1}P^{s})(Q\barConstructionMightAbbreviate X)\ar[r]&
0.
}\]
Now both $\barConstructionMightAbbreviate X$ and $Q\barConstructionMightAbbreviate X$ are connected, as they have $\pi_0(\barConstructionMightAbbreviate X)=\pi_*X$ is zero by assumption, and $\pi_0(Q\barConstructionMightAbbreviate X)=Q\pi_*X$. By induction we can bound the connectivity of $(\caldup{t-1}P^{s})(\barConstructionMightAbbreviate X)$ and $(\caldup{t-1}P^{s})(Q\barConstructionMightAbbreviate X)$, and the associated long exact sequence shows that $(\caldup{t}P^{s})(X)$ has a connectivity bound at most one degree lower.


For the base case, $t=1$, as $P^s(QX)=0$ for $s\geq2$:
\[(\caldup{1}P^{s})X:=\ker(P^{s}(\barConstructionMightAbbreviate X)\to P^{s}(Q\barConstructionMightAbbreviate X))=P^{s}(\barConstructionMightAbbreviate X).\]
When $\algcat=\liealgs$, a modification \cite[4.3]{6Author.pdf} of a theorem of Curtis \cite[\S5]{Curtis_LCS.pdf} states that $P^{s}(\barConstructionMightAbbreviate X)$ is $\log_2(s)$-connected.
When $\algcat=\algs$, we must demonstrate then that $P^s(\barConstructionMightAbbreviate X)$ is $(s-1)$-connected. For this
 we use a truncation of Quillen's fundamental spectral sequence, as presented in \cite[Thm 6.2]{MR1089001}: the filtration
\[P^s(\barConstructionMightAbbreviate X)\supset P^{s+1}(\barConstructionMightAbbreviate X)\supset P^{s+2}(\barConstructionMightAbbreviate X)\supset\cdots \]
of $P^s(\barConstructionMightAbbreviate X)$ yields a convergent spectral sequence  $E^0_{p,q}\implies \pi_q(P^s(\barConstructionMightAbbreviate a))$, with:
\[E^0_{p,q}=N_q\bigl(P^{p}(\barConstructionMightAbbreviate X)/P^{p+1}(\barConstructionMightAbbreviate X)\bigr)\textup{ if $p\geq s$, and }E^0_{p,q}=0\textup{ if $p<s$.}\]
Examination of the structure of $\barConstructionMightAbbreviate X\barConstructionMightAbbreviate X$ reveals that $P^{p}(\barConstructionMightAbbreviate X)/P^{p+1}(\barConstructionMightAbbreviate X)\cong (Q\barConstructionMightAbbreviate X)^{\otimes p}_{\Sigma_p}$, and moreover, $Q\barConstructionMightAbbreviate X$ is a connected simplicial vector space, as $\pi_0(Q\barConstructionMightAbbreviate X)=Q(\pi_0\barConstructionMightAbbreviate X)=Q(0)=0$. The $t=1$ result now follows from \cite[Satz 12.1]{DoldPuppeSuspension.pdf}: if $V$ is a connected simplicial vector space then $V^{\otimes p}_{\Sigma_p}$ is $(p-1)$-connected. 
\end{proof}
Before we can give the proof of proposition \ref{convergenceProp}, we need the following \emph{twisting lemma}, analogous to that of \cite{BK_pairings.pdf}. Before stating it, we note that $(\caldup{s}\caldup{t})X$ and $\caldup{s+t}X$ are equal by construction.
\begin{lem}
\label{DsDt=Dt+s}
The maps $\caldup{i}\delta:\caldup{n}X\to \caldup{n-1}X$ are homotopic for $0\leq i< n$.
\end{lem}
\begin{proof}
We may reindex the twisting lemma as follows: the maps 
\[\caldup{s}\delta,\caldup{s-1}\delta:\caldup{s+t}X\to \caldup{s+t-1}X\]
are homotopic whenever $s,t\geq1$. Now $\caldup{s+t}X$ is constructed as the subalgebra
\[\caldup{s+t}X:= \bigcap_{i=1}^{s+t}\ker\!\left(\barConstructionMightAbbreviate^{s+t-i}\eta \barConstructionMightAbbreviate^{i}:\barConstructionMightAbbreviate^{s+t}X\to \barConstructionMightAbbreviate^{s+t-i}KQ\barConstructionMightAbbreviate^{i}X\right)\]
of the iterated bar construction $\barConstructionMightAbbreviate^{s+t}X$, and for $0\leq i<s+t$, $\caldup{i}\delta$ is the restriction of the map $\barConstructionMightAbbreviate^i\epsilon \barConstructionMightAbbreviate^{s+t-i-1}:\barConstructionMightAbbreviate^{s+t}X\to \barConstructionMightAbbreviate^{s+t-1}X$.
Proposition \ref{IteratedBarConstructionHomotopy} gives an explicit simplicial homotopy between the maps $\barConstructionMightAbbreviate^s\epsilon \barConstructionMightAbbreviate^{t-1}$ and $\barConstructionMightAbbreviate^{s-1}\epsilon \barConstructionMightAbbreviate^{t}$. Moreover, the naturality of the construction of proposition \ref{IteratedBarConstructionHomotopy} implies that this homotopy does indeed restrict to a homotopy of maps $\caldup{s+t}X\to \caldup{s+t-1}X$.
\end{proof}


Now that we have the twisting lemma, Proposition \ref{convergenceProp} follows:
\begin{proof}[Proof of Proposition \ref{convergenceProp}]
By proposition \ref{prop:modifiedAdamsTower}, it is enough to prove that for any $q\geq0$ and $t\geq1$, $\pi_q(\caldup{f(q,t)}X)\to\pi_q(\caldup{t}X)$ is zero for some $f(q,t)\geq t$.
Apply $\caldup{t}\DASH$ to the diagram of functors constructed in \ref{towerWithPowers} and apply the result to $X$ to obtain a commuting diagram of functors
\[\xymatrix@R=4mm{
\caldup{f(q,t)}X
\ar[r]^-{\caldup{t}\delta}
\ar[d]
&
\cdots \ar[r]^-{\caldup{t}\delta}
&%r1c4
\caldup{t+1}X
\ar[r]^-{\caldup{t}\delta}
\ar[d]&%r1c5
\caldup{t}X
\ar@{=}[d]
\\%r1c6
\caldup{t}P^{f(q,t)-t+1}X
\ar[r]
&%r2c3
\cdots 
\ar[r]&%r2c4
\caldup{t}P^2X
\ar[r]
&%r2c5
\caldup{t}P^1X%r2c6
}\]
By the twisting lemma, \ref{DsDt=Dt+s}, the composite along the top row is homotopic to the map of interest, and factors through $\caldup{t}P^{f(q,t)-t+1}X$. If we choose $f(q,t)=2t+q-1$ when $\algcat=\algs$ and $f(q,t)=2^{t+q-1}+t-1$ when $\algcat=\liealgs$, then lemma \ref{connectivityOfDerivedPowers} shows that $\caldup{t}P^{f(q,t)-t+1}X$  is $q$-connected.
\end{proof}

\subsection{Iterated simplicial bar constructions}\label{sec:ItSimpBar}


We will state and prove a useful result on iterated simplicial bar constructions, used in the proof of the twisting lemma. The result here applies in general in the category $\algcat$ of algebras over a monad. Establishing notation, for any simplicial object $X$ in $\algCat$, we will write 
\[\trip{d}{i,q}{X}:X_q\to X_{q-1}\textup{\ and\ }\trip{s}{i,q}{X}:X_q\to X_{q+1}\]
for the $i^\textup{th}$ face and degeneracy maps out of $X_q$. Suppose that $G$ and $G'$ are endofunctors of $\algCat$, that $\Phi:G\to G'$ is a natural transformation, and that $C,C'\in\algCat$ are objects. Write $[\Phi]:\hom_{\algCat}(C,C')\to\hom_{\algCat}(GC,G'C')$ for the operator sending $m:C\to C'$ to the diagonal composite in the commuting square
\[\xymatrix@R=4mm{
GC
\ar[r]^-{\Phi_C}
\ar[d]_-{Gm}
\ar[dr]|-{[\Phi]m}
&%r1c1
G'C
\ar[d]^-{G'm}
\\%r1c2
GC'
\ar[r]_-{\Phi_{C'}}
&%r2c1
G'C'
%r2c2
}\]
There is an (augmented) simplicial endofunctor, $\frakb\in s({\algCat}^{\algCat})$, derived from the unit and counit of the adjunction:
\[\vcenter{
\def\labelstyle{\scriptstyle}
\xymatrix@C=2cm@1{
{\ \Id\,}
&
\,(F^{\algCat})^1\,
\ar@{-->}[l]|(.65){\frakd_{0,0}}
\ar[r]|(.65){\fraks_{0,0}}
&
\,(F^{\algCat})^2\,
\ar@<-1ex>[l]|(.65){\frakd_{0,1}}
\ar@<+1ex>[l]|(.65){\frakd_{1,1}}
\ar@<+1ex>[r]|(.65){\fraks_{0,1}}
\ar@<-1ex>[r]|(.65){\fraks_{1,1}}
&
\,(F^{\algCat})^3\,\makebox[0cm][l]{\,$\cdots $}
\ar[l]|(.65){\frakd_{1,2}}
\ar@<-2ex>[l]|(.65){\frakd_{0,2}}
\ar@<+2ex>[l]|(.65){\frakd_{2,2}}
%\ar[r]|(.65){\fraks_{1,2}}
%\ar@<+2ex>[r]|(.65){\fraks_{0,2}}
%\ar@<-2ex>[r]|(.65){\fraks_{2,2}}
%&
%\,(F^{\algCat})^4\,\makebox[0cm][l]{\,$\cdots $}
%\ar@<-3ex>[l]|(.65){\frakd_{0,3}}
%\ar@<-1ex>[l]|(.65){\frakd_{1,3}}
%\ar@<+1ex>[l]|(.65){\frakd_{2,3}}
%\ar@<+3ex>[l]|(.65){\frakd_{3,3}}
}}\]
The simplicial bar construction $\barConstructionMightAbbreviate =B^{\algCat}$ on $s\algCat$ is the diagonal of the bisimplicial object obtained by levelwise application of $\frakb$. That is, for $X\in s\algCat$, $\barConstructionMightAbbreviate X$ is the simplicial object with $(\barConstructionMightAbbreviate X)_q:=(F^{\algCat})^{q+1}X_q$, and with
\[\trip{d}{i,q}{\barConstructionMightAbbreviate X}:=[\frakd_{i,q}]\trip{d}{i,q}{X}.\]
The augmentation $\epsilon:\barConstructionMightAbbreviate \to \Id $ is defined on level $q$ by 
\[\epsilon_q=\frakd_{0,0}\frakd_{0,1}\cdots \frakd_{0,q}:(F^{\algCat})^{q+1}\to \Id .\]
We can now construct the simplicial homotopy needed for the twisting lemma, \ref{DsDt=Dt+s}.
\begin{prop}\label{IteratedBarConstructionHomotopy}
The natural transformations $\epsilon_\barConstructionMightAbbreviate $ and $\barConstructionMightAbbreviate \epsilon$ from $\barConstructionMightAbbreviate ^2:s\algcat\to s\algcat$ to $\barConstructionMightAbbreviate :s\algcat\to s\algcat$ are naturally simplicially homotopic.
\end{prop}

\begin{proof}
Write $K=\barConstructionMightAbbreviate ^{2}X$ and $L=\barConstructionMightAbbreviate X$ for the source and target of these maps respectively. Noting the formulae
\[[\frakd_{iq}]^2= [\frakd_{q+i,2q}\circ\frakd_{i,2q+1}]\ \textup{and}\  [\fraks_{iq}]^2= [\fraks_{q+i+2,2q+2}\circ\fraks_{i,2q+1}],\]
we can describe the simplicial structure maps in $K$ and $L$ as follows:
\begin{alignat*}{2}
\trip{d}{iq}{L}&=[\frakd_{iq}]\trip{d}{iq}{X}\\
\trip{s}{iq}{L}&=[\fraks_{iq}]\trip{s}{iq}{X}\\
\trip{d}{iq}{K}&=[\frakd_{q+i,2q}\circ\frakd_{i,2q+1}]\trip{d}{iq}{X}\\
\trip{s}{iq}{K}&=[\fraks_{q+i+2,2q+2}\circ\fraks_{i,2q+1}]\trip{s}{iq}{X}
\end{alignat*}
We can now state an explicit simplicial homotopy between the two maps of interest. Using precisely the notation of \cite[\S5]{MaySimpObj.pdf}, we define $\trip{h}{jq}{}:K_q\to L_{q+1}$, for $0\leq j\leq q$, by the formula
\[\trip{h}{jq}{}:=[\frakd_{j+1,q+2}\circ\cdots \circ\frakd_{j+1,2q+1}]\trip{s}{jq}{X}.\]
We first check that these maps satisfy the defining identities for the notion of simplicial homotopy, numbered (1)-(5) as in \cite[\S5]{MaySimpObj.pdf}. Each identity can be checked in two parts (a)-(b):
{\renewcommand{\circ}{\relax}
\begin{enumerate}\squishlist
\setlength{\parindent}{.25in}
\item We must check that $\trip{d}{i,q+1}{L}\circ \trip{h}{j,q}{}=\trip{h}{j-1,q-1}{}\circ \trip{d}{i,q}{K}$ whenever $0\leq i<j\leq q$, i.e.:
\begin{enumerate}\squishlist
\setlength{\parindent}{.25in}
\item[({\makebox[.51em][c]{a}})] $\trip{d}{i,q+1}{X}\circ \trip{s}{j,q}{X}=\trip{s}{j-1,q-1}{X}\circ \trip{d}{i,q}{X}$,\textup{ and}%$\trip{\frakd}{i,q+1}{}\circ \trip{\fraks}{j,q}{}=\trip{\fraks}{j-1,q-1}{}\circ \trip{\frakd}{i,q}{}$
\item[({\makebox[.51em][c]{b}})]
$\frakd_{i,q+1}\circ
\frakd_{j+1,q+2}\circ\cdots \circ\frakd_{j+1,2q+1}=
\frakd_{j,q+1}\circ\cdots \circ\frakd_{j,2q-1}\circ
\frakd_{q+i,2q}\circ\frakd_{i,2q+1}$.
\end{enumerate}
\item We must check that $\trip{d}{j+1,q+1}{L}\circ \trip{h}{j,q}{}=\trip{d}{j+1,q+1}{L}\circ \trip{h}{j+1,q}{}$ whenever $0\leq j\leq q-1$, i.e.:
\begin{enumerate}\squishlist
\setlength{\parindent}{.25in}
\item[({\makebox[.51em][c]{a}})] $\trip{d}{j+1,q+1}{X}\circ \trip{s}{j,q}{X}=\trip{d}{j+1,q+1}{X}\circ \trip{s}{j+1,q}{X}$,\textup{ and}%$\trip{\frakd}{j+1,q+1}{}\circ \trip{\fraks}{j,q}{}=\trip{\frakd}{j+1,q+1}{}\circ \trip{\fraks}{j+1,q}{}$
\item[({\makebox[.51em][c]{b}})]
$\frakd_{j+1,q+1}\circ
\frakd_{j+1,q+2}\circ\cdots \circ\frakd_{j+1,2q+1}=
\frakd_{j+1,q+1}\circ
\frakd_{j+2,q+2}\circ\cdots \circ\frakd_{j+2,2q+1}$.
\end{enumerate}
\item We must check that $\trip{d}{i,q+1}{L}\circ \trip{h}{j,q}{}=\trip{h}{j,q-1}{}\circ \trip{d}{i-1,q}{K}$ whenever $0\leq j<i-1\leq q$, i.e.:
\begin{enumerate}\squishlist
\setlength{\parindent}{.25in}
\item[({\makebox[.51em][c]{a}})] $\trip{d}{i,q+1}{X}\circ \trip{s}{j,q}{X}=\trip{s}{j,q-1}{X}\circ \trip{d}{i-1,q}{X}$,\textup{ and}%$\trip{\frakd}{i,q+1}{}\circ \trip{\fraks}{j,q}{}=\trip{\fraks}{j,q-1}{}\circ \trip{\frakd}{i-1,q}{}$
\item[({\makebox[.51em][c]{b}})] 
$\frakd_{i,q+1}\circ
\frakd_{j+1,q+2}\circ\cdots \circ\frakd_{j+1,2q+1}=
\frakd_{j+1,q+1}\circ\cdots \circ\frakd_{j+1,2q-1}\circ
\frakd_{q+i-1,2q}\circ\frakd_{i-1,2q+1}$.
\end{enumerate}
\item We must check that $\trip{s}{i,q+1}{L}\circ \trip{h}{j,q}{}=\trip{h}{j+1,q+1}{}\circ \trip{s}{i,q}{K}$ whenever $0\leq i\leq j\leq q$, i.e.:
\begin{enumerate}\squishlist
\setlength{\parindent}{.25in}
\item[({\makebox[.51em][c]{a}})] $\trip{s}{i,q+1}{X}\circ \trip{s}{j,q}{X}=\trip{s}{j+1,q+1}{X}\circ \trip{s}{i,q}{X}$,\textup{ and}%$\trip{\fraks}{i,q+1}{}\circ \trip{\fraks}{j,q}{}=\trip{\fraks}{j+1,q+1}{}\circ \trip{\fraks}{i,q}{}$
\item[({\makebox[.51em][c]{b}})] 
$\fraks_{i,q+1}\circ
\frakd_{j+1,q+2}\circ\cdots \circ\frakd_{j+1,2q+1}=
\frakd_{j+2,q+3}\circ\cdots \circ\frakd_{j+2,2q+3}\circ
\fraks_{q+i+2,2q+2}\circ\fraks_{i,2q+1}$.
\end{enumerate}
\item We must check that $\trip{s}{i,q+1}{L}\circ \trip{h}{j,q}{}=\trip{h}{j,q+1}{}\circ \trip{s}{i-1,q}{K}$ whenever $0\leq j<i\leq q+1$, i.e.:
\begin{enumerate}\squishlist
\setlength{\parindent}{.25in}
\item[({\makebox[.51em][c]{a}})] $\trip{s}{i,q+1}{X}\circ \trip{s}{j,q}{X}=\trip{s}{j,q+1}{X}\circ \trip{s}{i-1,q}{X}$,\textup{ and}%$\trip{\fraks}{i,q++1}{}\circ \trip{\fraks}{j,q}{}=\trip{\fraks}{j,q+1}{}\circ \trip{\fraks}{i-1,q}{}$
\item[({\makebox[.51em][c]{b}})] 
$\fraks_{i,q+1}\circ
\frakd_{j+1,q+2}\circ\cdots \circ\frakd_{j+1,2q+1}=
\frakd_{j+1,q+3}\circ\cdots \circ\frakd_{j+1,2q+3}\circ
\fraks_{q+i+1,2q+2}\circ\fraks_{i-1,2q+1}$.
\end{enumerate}
\end{enumerate}
\noindent Each of these equations follows from the simplicial identities, proving that the $h_{jq}$ form a homotopy. 
Finally, we check that this homotopy is indeed a homotopy between the two maps of interest:
\begin{alignat*}{2}
\trip{d}{0,q+1}{L}\trip{h}{0,q}{}
&=
[\frakd_{0,q+1}\frakd_{1,q+2}\circ\cdots \circ\frakd_{1,2q+1}](\trip{d}{0,q+1}{X}\trip{s}{0q}{X})%
\\
&=
[\frakd_{0,q+1}\frakd_{0,q+2}\circ\cdots \circ\frakd_{0,2q+1}]\trip{\textup{id}}{X_q}{}%
\end{alignat*}
is the action of  $\epsilon_{(\barConstructionMightAbbreviate X)}$ in level $q$, and similarly,
\begin{alignat*}{2}
\trip{d}{q+1,q+1}{L}\trip{h}{q,q}{}
&=
[\frakd_{q+1,q+1}\frakd_{q+1,q+2}\circ\cdots \circ\frakd_{q+1,2q+1}](\trip{d}{q+1,q+1}{X}\trip{s}{qq}{X})
\\
&=
[\frakd_{q+1,q+1}\frakd_{q+1,q+2}\circ\cdots \circ\frakd_{q+1,2q+1}]\trip{\textup{id}}{X_q}{}%
\end{alignat*}
is the action of $\barConstructionMightAbbreviate \epsilon_{X}$ in level $q$.
}%end "do not print \circ"
\end{proof}
\end{connectivity}
\end{BK spec seq}

\begin{Constructing (co)homotopy operations}



\section{\textbf{Constructing homotopy operations}}\label{sec:Constructing homotopy operations}

\subsection{Higher simplicial Eilenberg-Mac Lane maps}[\textbf{$\Nabla$ because} it is the right symbol, should be an upper index since it decreases homological degree.]


In what follows, we will often have a natural map $F$ whose domain and codomain both support a switch map $T$, obtained by interchanging tensor factors. Furthermore, we will so often have use for the expression $TFT$, that we introduce the shorthand $\twist F:=TFT$. %Although this notation has the potential for ambiguity, we will be careful to avoid confusion.
Although this notation is potentially ambiguous, whenever we write $\sigma FG$, for functions $F$ and $G$, we mean $(\sigma F)G$, not $\sigma(FG)$.

Let $\{\DeltaUp^k\}$ be a higher simplicial Eilenberg-Mac Lane map \cite[\S3]{DwyerHtpyOpsSimpComAlg.pdf}, i.e.\ a collection of maps
\[\DeltaUp^k:(CU\otimes CV)_{i+k}\to N(U\otimes V)_i\textup{\ \ defined for $0\leq k\leq i$}\]
natural in simplicial vector spaces $U$ and $V$, such that for $k\geq0$, the identity
\[(1+\twist) \DeltaUp^k=\oldphi^k+\begin{cases}
\DeltaUp^{k-1}\partial+\partial\DeltaUp^{k-1},&\textup{if }k\geq1,\\
\DeltaUp,&\textup{if }k=0,
%\\,&\textup{if }
\end{cases}
\]
holds on classes of dimension at least $2k$, and:
\begin{itemize}
\setlength{\parindent}{.25in}
\item $\DeltaUp:CU\otimes CV\to N(U\times V)$ is the Eilenberg-Zilber map, a chain homotopy equivalence inducing the identity in dimension zero; and
\item $\oldphi^k$ is the map $(CU\otimes CV)_{i+k}\to N(U\otimes V)_i$ which vanishes except on $U_k\otimes V_k$, where its value is just the projection $U_k\otimes V_k\to N(U\times V)_k$.
\end{itemize}
Note that as $\oldphi^0$ commutes with symmetry isomorphisms, so does $\DeltaUp$.





\subsection{External unary homotopy operations}
In this section we recall the definition of certain homotopy operations with domain $\pi_*V$ for any $V\in s\vect{}{}$, implicit in \cite[\S4]{DwyerHtpyOpsSimpComAlg.pdf} (or cartan or Bousfield - explicit?) and explicit in 
\cite[\S3]{MR1089001}, using the functions
\[a \mapsto \DeltaUp^{n-i}(a\otimes a),\ \ N_nV \to N_{n+i}(S_2V).\]
By postcomposing with the maps $S_2V\to \Lambda^{2}V\to S^2V$, we obtain functions from $N_nV$ to $N_{n+i}(\Lambda^2V)$ and $N_{n+i}(S^2V)$.
\begin{prop}[{\cite[Lemma 4.1]{DwyerHtpyOpsSimpComAlg.pdf},\ \cite[\S3]{MR1089001}}] \label{extUnaryHomotOps}
These functions descend to  well defined homotopy operations:
\begin{alignat*}{2}
\delta_i^\textup{ext}:\pi_nV&\to \pi_{n+i}(S_2V),&\quad&(2\leq i\leq n),\\
\lambda_i^\textup{ext}:\pi_nV&\to \pi_{n+i}(\Lambda^2V),&\quad&(1\leq i\leq n),\\
\sigma_i^\textup{ext}:\pi_nV&\to \pi_{n+i}(S^2V),&\quad&(1\leq i\leq n).
\end{alignat*}
The function $N_nV \to N_{n}(S^2V)$ given by $\overline{a}\mapsto \overline{a\otimes a}$ yields a well defined homotopy operation $\sigma_0^\textup{ext}:\pi_nV\to \pi_{n}(S^2V)$. 
For all $n\geq0$, the map $\sigma_n^\textup{ext}:\pi_nV\to\pi_{2n}(S^2V)$ satisfies
\[\sigma_n^\textup{ext}(\overline{x}+\overline{y})=\sigma_n^\textup{ext}(\overline{x})+\sigma_n^\textup{ext}(\overline{y})+\overline{(1+T)\Nabla(x\otimes y)}\quad\text{for $x,y\in ZN_nV$}.\]
%For $2\leq i\leq n$, the function $N_nV \to N_{n+i}S_2V$ yields a well-defined homotopy operation $\sigma_i^\textup{ext}:\pi_nV\to \pi_{n+i}(S_2V)$. 
%For $1\leq i\leq n$, the function $N_nV \to N_{n+i}\Lambda^2V$ yields a well-defined homotopy operation $\sigma_i^\textup{ext}:\pi_nV\to \pi_{n+i}(\Lambda^2V)$. 
%%produces a well defined homotopy operation $\sigma_i^\textup{ext}:\pi_n(V)\to \pi_{n+i}(\Lambda^2V)$. 
%The function \[a\mapsto \oldphi^n(a\otimes a),\ \ N_nV \to N_{n}S^2V\] yields a well-defined homotopy operation $\sigma_0^\textup{ext}:\pi_nV\to \pi_{n}(S^2V)$.
\end{prop}
\begin{proof}
Although all of the operations  are defined in the cited references, we will be a little more explicit about the definition of $\sigma^\textup{ext}_0$, and the final equation of the proposition.

As described in \cite[\S3]{MR1089001}, we might choose to define $\sigma^\textup{ext}_0$ using a universal example, for which the cycle 
\[z\otimes z\in ZN_n(S^2\mathbb{K}_n)\cong \Ftwo \]
is the only possible representative, demonstrating that the formula $\overline{a}\mapsto \overline{a\otimes a}$ yields the correct (well defined) operation. To check that $\sigma_0^\textup{ext}$ satisfies the stated equation, we need only check that it holds on $z_1+z_2\in ZN_0(\mathbb{K}_0\oplus \mathbb{K}_0)\cong \Ftwo \oplus \Ftwo $. But
\[\sigma_0^\textup{ext}(z_1+z_2)-\sigma_0^\textup{ext}(z_1)-\sigma_0^\textup{ext}(z_2)=z_1\otimes z_2+z_2\otimes z_1=(1+T)\Nabla(z_1\otimes z_2),\]
as $\Nabla$ is the identity in dimension zero.

To explain the equation when $n\geq1$, as $\sigma_n(\overline{x}):=\overline{(1+T)\Nabla^0(x\otimes x)}$, we obtain %into the expression  $\sigma_n^\textup{ext}(x+y)-\sigma_n^\textup{ext}(x)-\sigma_n^\textup{ext}(y)$ to obtain
\[\sigma_n^\textup{ext}(\overline{x}+\overline{y})-\sigma_n^\textup{ext}(\overline{x})-\sigma_n^\textup{ext}(\overline{y})
=
\overline{(1+T)\Nabla^0(1+T)(x\otimes y)}\]
and using the symmetry  $T\bigl((1+T)(x\otimes y)\bigr)=(1+T)(x\otimes y)$, and the fact that $\oldphi^0$ vanishes on $(1+T)(x\otimes y)$:
%\begin{alignat*}{2}
%(1+T)\Nabla^0(1+T)(x\otimes y)
%&=
%(1+\omega\Nabla^0)(1+T)(x\otimes y)%
%\\
%&=
%(\Nabla+\oldphi^0)(1+T)(x\otimes y)%
%\\
%&=
%\Nabla(1+T)(x\otimes y)%
%\end{alignat*}
\[
(1+T)\Nabla^0(1+T)(x\otimes y)
=
(1+\omega\Nabla^0)(1+T)(x\otimes y)
=
\Nabla(1+T)(x\otimes y).\qedhere
\]
%where we have used the symmetry
%\[T\bigl((1+T)(x\otimes y)\bigr)=(T+T^2)(x\otimes y)=(1+T)(x\otimes y)\]
%in order to exchange $T\Nabla^0$ and $\omega\Nabla^0$ at the second equality.
\end{proof}

\subsection{External binary homotopy operations}
We will now give an account of various natural external homotopy operations, most of which are binary operations, induced by the  Eilenberg-Mac Lane shuffle map $\nabla:N_*(V)\otimes N_*(V)\to N_*(V\otimes V)$, which is also known as the Eilenberg-Zilber map.
%The homotopy operations we will present will be in the form of a natural transformation of functors $s\vect{+}{n}\to \vect{+}{n+1}$, which will be induced by the Eilenberg-Mac Lane shuffle map $\nabla:N_*(V)\otimes N_*(V)\to N_*(V\otimes V)$. \textbf{Need to clarify what is meant by quad grading.}
\begin{prop}[Various authors]\label{the top external homotopy operations}
There is a natural commuting diagram:
\[\xymatrix@R=4mm{
S_2(\pi_*V)\ar[r]^-{\smash{\widetilde{\nabla}}}
\ar[d]^-{\textup{proj}}
&%r1c1
\pi_*(S_2V)\ar[d]^-{\pi_*(\textup{proj})}
\\%r1c2
\Lambda^2(\pi_*V)\ar[r]^-{\widetilde{\nabla}}
\ar[d]^-{\textup{incl}}
&%r1c1
\pi_*(\Lambda^2V)\ar[d]^-{\pi_*(\textup{incl})}
\\%r1c2
S^2(\pi_*V)\ar[r]^-{\widetilde{\nabla}}&%r2c1
\pi_*(S^2V)%r2c2
}\]
For cycles $x,y\in ZN_*(V)$ and $z\in ZN_n(V)$, the upper horizontal is determined by:
\[\overline{x}\otimes\overline{y}\mapsto \overline{x\otimes y},\]
and the lower horizontal is determined by:
\[\overline{x}\otimes\overline{y}+\overline{y}\otimes \overline{x}\mapsto\overline{\nabla(x\otimes y+y\otimes x)}\textup{, and }\overline{z}\otimes\overline{z}\mapsto\sigma^\textup{ext}_n(\overline{z}).\]
\end{prop}
\begin{proof}
During this proof, write $\widetilde{\nabla}_U$, $\widetilde{\nabla}_M$ and $\widetilde{\nabla}_L$ for the upper, middle and lower horizontal maps. We must demonstrate: that $\widetilde{\nabla}_U$  is well defined; that
\[\ker(\pi_*(\textup{proj})\circ\widetilde{\nabla}_U)\supseteq\ker(\textup{proj}),\]
so that there is a unique map $\widetilde{\nabla}_M$ for which the upper square commutes; and that one may extend the composite $\pi_*(\trace)\circ\widetilde{\nabla}_U$ along the trace map $S_2(\pi_*A)\to S^2(\pi_*A)$ using the operations
$\sigma_n^\textup{ext}$.
A simple diagram chase would then reveal that the bottom square must also commute.
%in order that $\widetilde{\nabla}_U$ extends to  a unique map $\widetilde{\nabla}_M$ for which the diagram commutes.

As $\nabla$ is a chain  map, it produces a well defined map $(\pi_*V)^{\otimes2}\to \pi_*(V^{\otimes2})$, and the fact that $\nabla=\twist\nabla$ implies that this map descends to a well defined map $\nabla_U$.

The kernel of the projection $S_2(\pi_*V)\to \Lambda^2(\pi_*V)$ is spanned by classes of the form $\overline{x}\otimes \overline{x}$, and the image under $\pi_*(\textup{proj})\circ\widetilde{\nabla}_U$ of such a class may be represented by $x\otimes x\in \Lambda^2V$, which equals zero, proving the inclusion of kernels.

Finally, to extend the composite $\pi_*(\trace)\circ\widetilde{\nabla}_U$, we simply need the operations $\sigma_n$ to satisfy the equations of proposition \ref{propOnExtendingToInvariants}, which are part of proposition \ref{extUnaryHomotOps}.
%
%To produce $\widetilde{\nabla}_L$ is to extend the composite $\pi_*(\trace)\circ\widetilde{\nabla}_U$, along the trace map $S_2(\pi_*A)\to S^2(\pi_*A)$.
%
%Since $\nabla$ is a symmetric chain map, the formula given for $\widetilde{\nabla}_L$ does indeed return homotopy classes in $\pi_*(S_2V)$.
%
% If these maps are well defined, the rest is clear. %, however, it is not clear that either of the maps $\widetilde{\nabla}$ is well defined. 
%
%%In fact, we only need to check that  $\widetilde{\nabla}_L$ is well defined: as $\pi_*(\textup{incl})$ is a monomorphism (by \cite[Prop 5.6]{BousOpnsDerFun.pdf}), it will follow that the upper map is well defined. 
%
%By general observations on natural transformations with domain the endofunctor $S^2$ of $\vect{}{}$, in order to extend
%\[\pi_*(\trace)\circ\widetilde{\nabla}_U:S_2(\pi_*V)\to \pi_*(S^2V)\]
%to a natural map $\widetilde{\nabla}_L$, it suffices to specify a natural function $\restn{(\DASH)}:\pi_n(V)\to \pi_{2n}(S^2V)$ satisfying the equation
%\[\restn{(x+y)}=\restn{x}+\restn{y}+\pi_*(\trace)(\widetilde{\nabla}_U(x\otimes y)).\] 
% Then $\widetilde{\nabla}_L$ is the unique extension of $\pi_*(\trace)\circ\widetilde{\nabla}_U$  such that $\widetilde{\nabla}_L(\alpha\otimes \alpha)=\restn{\alpha}$ for $\alpha\in\pi_*\left(V\right)$.
%%\[p=\left(S_2\pi_*(V)\overset{\trace}{\to} S^2\pi_*(V)\overset{\widetilde{\nabla}'}{\to} \pi_*(S^2V)\right)\textup{ and }\restn{(\DASH)}=\left(\pi_*(V)\overset{v\mapsto v\otimes v}{\to} S^2\pi_*(V)\overset{\widetilde{\nabla}'}{\to} \pi_*(S^2V)\right).\]
%We define $\widetilde{\nabla}_L$ in this way, with $\restn{(\DASH)}$ the operation $\sigma_n:\pi_n(V)\to \pi_{2n}(S^2V)$, which satisfies this equation by proposition \ref{extUnaryHomotOps}.
%%This operation does not depend on the choice of representative $x$. Indeed,
%%\[\nabla((x+dh)\otimes (x+dh))=\nabla(x\otimes x)+d\trace(x\otimes h)+d(h\otimes dh)\]
%%
%%
%%
%%When $n\geq0$, this map is defined on cycles by $z\mapsto \nabla(z\otimes z)$, and the required equation is simply the expansion
%%\[\Nabla((x+y)\otimes(x+y))=\Nabla(x\otimes x)+\Nabla(y\otimes y)+(1+T)\Nabla(x\otimes y).\]
\end{proof}

\subsection{Homotopy operations for simplicial commutative algebras}\label{Homotopy operations for simplicial commutative algebras}
Suppose that $A\in s \algs$ is a simplicial non-unital commutative algebra, with multiplication map $\mu:S_2A\to A$. Then by composition with the map $\pi_*(\mu):\pi_*(S_2A)\to \pi_*A$, one obtains unary operations:
\[\delta_i:=\pi_*(\mu)\circ\delta_i^\textup{ext}:\pi_nV\to \pi_{n+i}V,\textup{ defined when }2\leq i\leq n,\]
and a pairing\[\mu:=\pi_*(\mu)\circ\widetilde{\nabla}:S_2(\pi_*V)\to \pi_{*}V.\]
\begin{prop}[Dwyer {\cite{DwyerHtpyOpsSimpComAlg.pdf}}]\label{omnibus on htpy of simp algs}These operations have the following properties:
\begin{enumerate}
\item the pairing $\mu$ gives $\pi_*A$ the structure of a non-unital commutative algebra;
\item the ideal $\bigoplus_{n\geq1}\pi_nA$ is an exterior algebra;
\item the ideal $\bigoplus_{n\geq2}\pi_nA$ is a divided power algebra, with divided square given by the \emph{top} $\delta$-operation, i.e.\ $x\mapsto \delta_nx\text{ for }x\in\pi_nA$;
\item the \emph{non-top operations}, $\delta_i:\pi_nA\to \pi_{n+i}A$ for $2\leq i<n$, are linear;
\item there holds the following \emph{Cartan formula}: for $x\in\pi_nA$, $y\in \pi_mA$ and $2\leq i\leq n$
\[\delta_i(xy)=\begin{cases}
y^2\delta_i(x),&\text{if }m=0;\\
0,&\text{otherwise};
\end{cases}
\]
\item \label{deltaademsunstable} the \emph{$\delta$-Adem relations} hold: if $\delta_i\delta_jx$ is defined, and $i<2j$, then
\[\delta_i\delta_jx:=\sum_{s=\lceil(i+1)/2\rceil}^{\lfloor(i+j)/3\rfloor}{j+s-i-1\choose j-s}\delta_{i+j-s}\delta_sx.\]
\end{enumerate}
\end{prop}
A few comments are in order. Firstly, the proposition distinguishes between the \emph{top} and \emph{non-top} $\delta$ operations, as they have rather different behaviour --- this will be a recurring pattern. Secondly, it is not immediately obvious that the $\delta$-Adem relations make sense, in that it is not obvious that every term in the right hand side is defined. This does indeed happen, by lemma \ref{lemOnAdemChangeInMDeltaPlain} (to follow). 

Now we may define a non-unital associative algebra, $\deltaalg$, as the algebra generated by $\delta_i$ for $i\geq2$, subject to relations
\[\delta_i\delta_j:=\sum_{s=\lceil(i+1)/2\rceil}^{\lfloor(i+j)/3\rfloor}{j+s-i-1\choose j-s}\delta_{i+j-s}\delta_s\textup{ when $i<2j$}.\]
We will say that a sequence $I=(i_\ell,\ldots,i_1)$ of integers $i_j\geq2$ is \emph{$\delta$-admissible} if $i_{j+1}\geq 2i_j$ for $1\leq j <\ell$. For any sequence $I=(i_\ell,\ldots,i_1)$, write $\delta_I$ for the composite $\delta_{i_\ell}\cdots \delta_{i_1}$. Then this relation evidently allows us to write any $\delta_I$ in $\deltaalg$ as a sum of composites $\delta_J$ in which $J$ is $\delta$-admissible. In fact, it follows from \cite[Proposition 2.7]{MR1089001} that the algebra $\deltaalg$ has an \emph{admissible basis}, consisting of those $\delta_I=\delta_{i_\ell}\cdots \delta_{i_{1}}$ with $I$ a $\delta$-admissible sequence. 

It then makes sense to make the following definition. Suppose that $I$ is any non-empty sequence of integers at least $2$, and $J$ is a $\delta$-admissible sequence. Then we will say that \emph{$I$ produces $J$ in $\deltaalg$}, denoted $\produces{I}{J}{\deltaalg}$ if, when $\delta_I$ is written in the $\delta$-admissible basis of $\deltaalg$, $\delta_J$ appears with non-zero coefficient. In this case, $J$ must be $\delta$-admissible, and $I$ must be $\delta$-inadmissible unless $J=I$.

Proposition \ref{omnibus on htpy of simp algs} does not state that $\pi_*A$ is a left module over $\deltaalg$, due to the fact that the $\delta$-operations are not always defined (or even linear). We define
\[\minDimDelta(I):=\max\{(i_1),\,(i_2-i_1),\,(i_3-i_2-i_1),\,\ldots,\,(i_{\ell}-\cdots-i_1)\},
\]
following the convention that $\max(\emptyset)=-\infty$, for any sequence $I$ of integers $i_j\geq2$ (or more generally, for any sequence of non-negative integers).  The intent of this definition is that the composite $\delta_I$, by which we mean
\[\pi_{n}A\overset{\delta_{i_1}}{\to}\pi_{n+i_1}A\overset{\delta_{i_2}}{\to}\cdots \overset{\delta_{i_\ell}}{\to}\pi_{n+i_1+\cdots +i_\ell}A\]
is defined if and only if $n\geq \minDimDelta(I)$. 
Note that when $I$ is a non-empty \emph{$\delta$-admissible} sequence,
\[\minDimDelta(I)=i_{\ell}-i_{\ell-1}-\cdots -i_1=:e(I),\]
the \emph{Serre excess of $I$}. Moreover, if $I$ is $\delta$-admissible, then for any expression $\delta_{i_{\ell}}\cdots \delta_{i_{1}}x$ there is some $k$ with $0\leq k\leq \ell$ such that each of the $k$ the operations $\delta_{i_{\ell}}\cdots \delta_{i_{\ell-k+1}}$ are acting as top operations, and each of the remaining $\ell-k$ are acting as non-top operations.

The following lemma assures us that the $\delta$-Adem relations make sense even in \emph{(\ref{deltaademsunstable})}.
%\begin{lem}\label{lemOnAdemChangeInMDeltaPlain}
%Suppose that $i,j\geq2$ and $i<2j$, and $(i+1)/2\leq s\leq (i+j)/3$. Then $i+j-s\geq 2s$, $i+j-s\geq2$, $s\geq2$, so that the $\delta$-Adem relation writes $\delta_i\delta_j$ as a sum of $\delta$-admissible composites. Moreover,  $\minDimDelta(i,j)\geq \minDimDelta(i+j-s,s)$.
%\end{lem}
%\begin{proof}
%The only tricky inequality is $\minDimDelta(i,j)\geq \minDimDelta(i+j-s,s)$. Now the right hand side must equal $i+j-2s$, and the left hand side is at least $j$. The result follows, since $2s\geq i+1$.
%\end{proof}
\begin{lem}\label{lemOnAdemChangeInMDeltaPlain}
If $\produces{I}{J}{\deltaalg}$, then $\minDimDelta(I)\geq\minDimDelta(J)$.
\end{lem}
\begin{proof}
It is enough to show this result when $I$ and $J$ are disjoint and have length two, in light of the evident algorithm for expressing $\delta_I$ in terms of admissible composites. In the length two case it can be checked directly from the format of the $\delta$-Adem relation, and the inequality is in fact strict (unless $I$ is itself $\delta$-admissible).
\end{proof}
%\noindent Thus, even though $\pi_*$ is not a $\deltaalg$-module, any composite $\delta_I:\pi_nA\to \pi_mA$ that is defined may be written as a sum
Finally, one should note that these operations generate all of the operations in the category $\algs\PiAlg$, and that all of the relations between the operations in $\algs\PiAlg$ are implied by those presented here. Goerss \cite[\S2]{MR1089001} presents this information as follows. First, he observes that there is a simple \emph{Hilton-Milnor theorem} available:
\begin{prop}
Suppose that $A_1$ and $A_2$ are models in $s\algs$. Then $\pi_*(A_1\sqcup A_2)$, which is the coproduct of $\pi_*A_1$ and $\pi_*A_2$ in $\algs\PiAlg$, may be calculated as the non-unital commutative algebra coproduct of $\pi_*A_1$ and $\pi_*A_2$.
\end{prop}
%This may be read as a theorem about coproducts of free $\algs$-$\Pi$-algebras, since $\pi_*(A_1\sqcup A_2)$ is the coproduct in $\algs\PiAlg$ of $\pi_*(A_1)$ and $\pi_*(A_2)$, the theorem being that coproducts of free $\algs$-$\Pi$-algebras may be taken in the category $\algs$.
\noindent Thus, after giving the calculation on a single sphere, the homotopy of finite models is determined, and thus the structure of $\algs\PiAlg$ is well understood:
\begin{prop}[{\cite[Proposition 2.7]{MR1089001}}]
\noindent For $n\geq0$, let $\imath_n$ be the fundamental class in $\pi_n(F^{\algs}\mathbb{K}_n)$. There  are isomorphisms of non-unital commutative algebras:
\begin{alignat*}{2}
\pi_*(F^{\algs}\mathbb{K}_0)
&\cong
S(\CommOperad)[\imath_0]=F^{\algs}[\imath_0];%
\\
\pi_*(F^{\algs}\mathbb{K}_n)
&\cong
\makebox[\widthof{$\Lambda$}][c]{$\Lambda$}(\CommOperad)[\delta_I(\imath_n)\ |\ \textup{$J$ is $\Sq$-admissible, $e(I)\leq n$}]%
&\ \ &\makebox[\widthof{for $n\geq1$}][c]{for $n\geq1$;}\\
% Left hand side
\pi_*(F^{\algs}\mathbb{K}_n)
% Relation
&\cong
% Right hand side
\makebox[\widthof{$\Lambda$}][c]{$\Gamma$}(\CommOperad)[\delta_I(\imath_n)\ |\ \textup{$J$ is $\Sq$-admissible, $e(I)< n$}]%
% Comment
&\ \ &\makebox[\widthof{for $n\geq1$}][c]{for $n\geq2$.}%\text{for $n\geq2$.}
\end{alignat*}
\end{prop}

\subsection{Homotopy operations for simplicial Lie algebras}\label{Homotopy operations for simplicial Lie algebras}
Suppose that $L\in s \Lambda(\LieOperad)$ is a simplicial Lie algebra with bracket $[\,,]:\Lambda^2L\to L$. There are unary operations:
\[\lambda_i:=\pi_*([\,,])\circ\lambda_i^\textup{ext}:\pi_nV\to \pi_{n+i}V,\textup{ defined when }1\leq i\leq n,\]
which we write on the right as $x\mapsto x\lambda_i$, and a bracket
\[[\,,]:=\pi_*([\,,])\circ\widetilde{\nabla}:\Lambda^2(\pi_*V)\to \pi_{*}V.\]

Alternatively, one can suppose that $L\in s \Gamma(\LieOperad)$ is a simplicial restricted Lie algebra with bracket $[\,,]:S^2L\to L$, and construct operations:
\begin{gather*}
\lambda_i:=\pi_*([\,,])\circ\sigma_i^\textup{ext}:\pi_nV\to \pi_{n+i}V,\textup{ defined when }0\leq i\leq n\textup{, and}\\
[\,,]:=\pi_*([\,,])\circ\widetilde{\nabla}:S^2(\pi_*V)\to \pi_{*}V.
\end{gather*}
\begin{prop}[????????]\label{omnibus on htpy of Lie algs}
For $L\in s\Lambda(\LieOperad)$, these operations satisfy:
\begin{enumerate}
\item \label{meep1} the bracket gives $\pi_*L$ the structure of a Lie algebra;
\item \label{meep2} the ideal $\bigoplus_{n\geq1}\pi_nL$ is a restricted Lie algebra, with restriction given by the \emph{top} $\delta$-operation, i.e.\ $\restn{x}=\lambda_nx\text{ for }x\in\pi_nL$;
\item \label{meep3} the \emph{non-top} operations, $\lambda_i:\pi_nL\to \pi_{n+i}L$ for $1\leq i<n$, are linear;
\item \label{meep4} there holds the following \emph{Cartan formula}, for $x\in \pi_*L$, $y\in \pi_nL$ and $1\leq i\leq n$:
\[[x,y\lambda_i]=\begin{cases}
[y,[x,y]],&\text{if }i=n;\\
0,&\text{otherwise};
\end{cases}
\]
\item \label{meep5} the \emph{$\lambda$-Adem relations} hold: if $x\lambda_j\lambda_i$ is defined, and $i>2j$, then
\[x\lambda_j\lambda_i=\sum_{k=0}^{(i-2j)/2-1}{i-2j-2-k\choose k}x\lambda_{i-j-1-k}\lambda_{2j+1+k}.\]
\end{enumerate}
For $L\in s\Gamma(\LieOperad)$, we may omit (\ref{meep1}), modify (\ref{meep2}) to state that the whole of $\pi_*L$ is restricted, and modify (\ref{meep3})-(\ref{meep5}) to include $\lambda_0$.
\end{prop}
Similar comments apply as for commutative algebras, for example, one needs lemma \ref{lemOnAdemChangeInMLambdaPlain} (to follow) to understand why this unstable relation makes sense. 

Now we will view the well known \emph{$\Lambda$-algebra} as the non-unital associative algebra generated by $\lambda_i$ for $i\geq0$, subject to relations \[\lambda_j\lambda_i=\sum_{k=0}^{(i-2j)/2-1}{i-2j-2-k\choose k}\lambda_{i-j-1-k}\lambda_{2j+1+k}\textup{\ for $i>2j$.}\]
We say that a sequence $I=(i_\ell,\ldots,i_1)$ of non-negative integers is \emph{$\lambda$-admissible} if $i_{j+1}\leq 2i_j$ for $1\leq j <\ell$. 
For any sequence $I=(i_\ell,\ldots,i_1)$, if we write $\lambda_I$ for the element $\lambda_{i_1}\cdots \lambda_{i_\ell}$ in $\Lambda$, then the $\Lambda$-algebra has the evident admissible basis, and we may make sense of the symbol $\produces{I}{J}{\Lambda}$. Note that the ordering of the generators in $\lambda_I$ and $\delta_I$ are opposite, for consistency with the fact that we write the $\lambda$-operations on $\pi_*L$ \emph{on the right}. Thus, we may think of $\lambda_I$ as the composite operator  
\[\pi_{n}L\overset{\lambda_{i_1}}{\to}\pi_{n+i_1}L\overset{\lambda_{i_2}}{\to}\cdots \overset{\lambda_{i_\ell}}{\to}\pi_{n+i_1+\cdots +i_\ell}L,\]
again defined only when $\minDimDelta(I)\leq n$, so that $\pi_*L$ is \emph{not} a right module over $\Lambda$.
Note that when $I$ is a non-empty \emph{$\lambda$-admissible} sequence,
$\minDimDelta(I)=i_1$, not the Serre excess, reflecting the observation that when $x\lambda_{i_1}\cdots \lambda_{i_\ell}$ is a $\lambda$-admissible composite, the only operation that  can be a top (i.e.\ restriction) operation is $\lambda_{i_1}$.
The following lemma assures us that the $\lambda$-Adem relations make sense in \emph{(\ref{meep5})}.
%\begin{lem}\label{lemOnAdemChangeInMDeltaPlain}
%Suppose that $i,j\geq2$ and $i<2j$, and $(i+1)/2\leq s\leq (i+j)/3$. Then $i+j-s\geq 2s$, $i+j-s\geq2$, $s\geq2$, so that the $\delta$-Adem relation writes $\delta_i\delta_j$ as a sum of $\delta$-admissible composites. Moreover,  $\minDimDelta(i,j)\geq \minDimDelta(i+j-s,s)$.
%\end{lem}
%\begin{proof}
%The only tricky inequality is $\minDimDelta(i,j)\geq \minDimDelta(i+j-s,s)$. Now the right hand side must equal $i+j-2s$, and the left hand side is at least $j$. The result follows, since $2s\geq i+1$.
%\end{proof}
\begin{lem}\label{lemOnAdemChangeInMLambdaPlain}
If $\produces{I}{J}{\Lambda}$, then $\minDimDelta(I)\geq\minDimDelta(J)$, and $J$ does not contain zero unless $I$ does.
\end{lem}
These operations generate all of the operations in each of the categories $\Lambda(\LieOperad)\PiAlg$ and $\Gamma(\LieOperad)\PiAlg$, and the relations presented here are sufficient. Indeed, although we do not have a Hilton-Milnor theorem available, from \cite[Theorem 8.8 and proof]{CurtisSimplicialHtpy.pdf} (and the original work, \cite{6Author.pdf}) we gather
\begin{prop}
{\textup{PRLieAlgs,Prop3.3}}... Basically, the free functor in $\Gamma(\LieOperad)\PiAlg$ takes the free restricted Lie algebra and attaches non-top $\lambda$ operations on the right.
\end{prop}

\section{\textbf{Constructing cohomology operations}}\label{sec:Constructing cohomology operations}

\subsection{Higher simplicial Alexander-Whitney maps}
Let $\{D_k\}$ be a \emph{special simplicial Alexander-Whitney map},  i.e.\  maps
\[D_k:N(U\otimes V)_{n}\to (CU\otimes CV)_{n+k}\textup{\ \ defined for $n,k\geq0$},\]
natural in simplicial vector spaces $U$ and $V$, such that for $k\geq1$, the identity
\[dD_k+D_kd=(1+\twist) D_{k-1}
\]
holds on classes of any dimension, and:
\begin{itemize}
\setlength{\parindent}{.25in}
\item $D_0$ is the Alexander-Whitney map, a chain homotopy equivalence inducing the identity in dimension zero;
\item the image  of $D_k:C_n(U\otimes V)\to (CU\otimes CV)_{n+k}$ is contained in the subspace
$\textstyle\bigoplus_{i,j\leq n}(C_iU\otimes C_jV)$, and is thus zero unless $k\leq n$; and
\item $D_k$ maps $C_{k}(U\otimes V)$ identically onto $C_{k}U\otimes C_{k}V$.
\end{itemize}
Such maps are described in detail, under the name \emph{special cup-$k$ product}, by Singer in \cite[Definitions 1.91 and 1.94]{MR2245560}, and developed originally in \cite{DoldUber}.
It is \textbf{??? prob not ???} a natural convention to define $D_k=0$ for whenever either $k<0$ or $i<0$, in which case the relation $dD_k+D_kd=D_{k-1}+TD_{k-1}T$ holds for any $k$.

\begin{shaded}
\textbf{Move, maybe.}
Let $\{D^k\}$ be a special dual Alexander-Whitney map \cite[Proposition 5.2]{turner_opns_and_sseqs_I.pdf}, i.e.\  maps
\[D^k:(CR\otimes CS)^{i+k}\to C(R\otimes S)^{i}\textup{\ \ for $i,k\geq0$},\]
natural in cosimplicial vector spaces $R,S$,
with the properties:
\begin{itemize}
\setlength{\parindent}{.25in}
\item $dD^k+D^kd=D^{k-1}+TD^{k-1}T$ for $k\geq1$;
\item $D^0$ is a chain homotopy equivalence inducing the identity in dimension zero;
\item the restriction of $D^k$ to $C^{i}R\otimes C^{j}S$ is zero unless $i\geq k$ and $j\geq k$; and
\item $D^k$ maps $C^{k}R\otimes C^{k}S$ identically onto $C^{k}(R\otimes S)$.
\end{itemize}
It is a natural convention to define $D^k=0$ for whenever either $k<0$ or $i<0$, in which case the relation $dD^k+D^kd=D^{k-1}+TD^{k-1}T$ holds for any $k$.
\end{shaded}

\subsection{External unary cohomotopy operations}
In this section we recall the definition of certain homotopy operations with domain $\dual(\pi_*V)$ (or $\pi^*\dual V$) for any $V\in s\vect{}{}$, \textbf{reference}, using the function
\[D^*_{n-i}\oldphi(\alpha\otimes \alpha)+D^*_{n-i+1}\oldphi(\alpha\otimes d\alpha),\ \ \dual(N_nV) \to \dual(N_{n+i}(S^2V)),\]
where we regard the $D^*_k$ as maps into $\dual(N_*(S^2V))$, and the map $\oldphi$ is that described in \cite[\S1.9]{MR2245560}. Namely, [\textbf{potentially redundant:} as the normalized complex is naturally a retract of the unnormalized complex, we may define] $\oldphi:\dual\left(N_mV\right)\otimes \dual\left(N_mV\right)\to \dual\left(N_m(V\otimes V)\right)$ to be
\[\left(C_mV\overset{\alpha}{\to}\Ftwo \right)\otimes \left(C_mV\overset{\beta}{\to}\Ftwo \right)\mapsto\left(C_m(V\otimes V)\overset{\alpha\otimes\beta}{\to}\Ftwo \otimes \Ftwo \overset{\textup{mult}}{\to}\Ftwo \right).\]
\begin{prop}[same refs, I guess]
These functions descend to  well defined cohomotopy operations:
\[\ExtCohOp^k:\dual(\pi_{n}V)\to \dual(\pi_{n+k}(S^2V)),\text{ \ zero unless $0\leq k\leq n$.}\]
\end{prop}
\begin{proof}
One checks that these cochain operations induce well defined operations $\ExtCohProd$ and $\ExtCohOp^k$ by the same arguments as in \cite[\S1.12]{MR2245560}.
\end{proof}

\begin{shaded}
In this section we recall the definition of certain cohomotopy operations with domain $\pi^*U$ for any $U\in c\vect{}{}$, \textbf{reference}, using the function
\[D^{n-i}(\alpha\otimes \alpha)+D^{n-i+1}(\alpha\otimes d\alpha),\ \ C^nU \to C^{n+i}(S_2U).\]
\begin{prop}[same refs, I guess] 
These functions descend to  well defined cohomotopy operations:
\[\ExtCohOp^k:\pi^{n}U\to \pi^{n+k}(S_2U),\text{ \ zero unless $0\leq k\leq n$.}\]
\end{prop}
\begin{proof}
One checks that these cochain operations induce well defined operations $\ExtCohProd$ and $\ExtCohOp^k$ by the same arguments as in \cite[\S1.12]{MR2245560}.
\end{proof}
Now if $U=\dual V$ for some $V\in s\vect{}{}$, then we may use the natural transformation $S_2\dual\to\dual S^2$ to form the composite
\[\pi^n\dual V\overset{\ExtCohOp^k}{\to}\pi^nS_2\dual V\to \pi^n\dual S^2V,\]
and under the natural isomorphism $\pi^*\dual\cong \dual\pi_*$, this operation is the operation defined before the shading.

\end{shaded}

\subsection{Linearly dual homotopy operations}\label{Linearly dual homotopy operations}
The linear maps $\ExtCohOp^k$ induce dual operators
\[(\ExtCohOp^k)^*:\pi_*(S^2V)\to \pi_{*-k}(V)\]
whenever these groups are locally finite dimensional. Following \cite[\S3]{MR1089001}, one can define the operations $(\ExtCohOp^k)^*$ directly, without the double dualization as described below. %Setting $V=Q^{\calc}B^{\calc}X$ and precomposing with the map $(\psi_\calc)_*:\pi_*(V)\to\pi_{*-1}(S^2V)$ yields homology operations on $H_*^\calc$ whose duals are the cohomology operations of \S\ref{generic coh ops section}.
Goerss \cite[Proposition 3.7]{MR1089001}
%If $\sigma_i$ and $\pi_*(1+T)$ are the maps of \cite[\S3]{MR1089001}, then 
shows that any element of $x\in \pi_m(S^2V)$ can be written as a sum
\[x=\sum_j\pi_*(1+T)(y_j\otimes z_j)+\sum_{\smash{0\leq k\leq \lfloor m/2\rfloor}}\sigma_k(w_k),\]
where $w_k\in \pi_{m-k}(V)$ for $0\leq k\leq\lfloor m/2\rfloor$, and $y_j,z_j\in \pi_{*}(V)$, and that we may \emph{define}:
\[(\ExtCohOp^k)^*(x)=w_k.\]
That is, we are defining a map $(\ExtCohOp^k)^*$ by this requirement, and the method of \cite[\S3]{MR1089001} shows that its linear dual, $((\ExtCohOp^k)^*)^*$ equals $\ExtCohOp^k$. Of course, the chosen \emph{symbol} $(\ExtCohOp^k)^*$ \textbf{(use stars here!)} is then only appropriate in the finite-dimensional case, but this definition of a map $(\ExtCohOp^k)^*$ works in any setting.

One might only need to determine the operations $(\ExtCohOp^k)^*$ for $k<m/2$, so that when $m$ is even we may ignore the dual of the top operation, $(\ExtCohOp^{m/2})^*$. In this case, it is more convenient to rewrite the above equation as follows: 
\[x=\widetilde{\nabla}(v)+\sum_{\smash{0\leq k< m/2}}\sigma_k(w_k),\]
where $v\in(S^2(\pi_*(V)))_{m}$ and $w_k\in \pi_{m-k}(V)$ for $0\leq k<m/2$.


\subsection{External binary cohomotopy operations}
Consider the cochain level pairing
\[\alpha\otimes\beta\mapsto D_0^*\oldphi(\alpha\otimes\beta),\quad\dual(N_*V)\otimes \dual(N_*V)\to \dual(N_*(S^2V)).\]
\begin{prop}\label{the external cohomotopy pairing}
This pairing induces a pairing on cohomotopy,
\[\ExtCohProd:S_2(\dual(\pi_*V))\to \dual(\pi_*(S^2V)),\]
with the property that $\ExtCohProd(\alpha\otimes \alpha)=\ExtCohOp^k\alpha$ for $\alpha\in\dual(\pi_kV)$.
\end{prop}
\begin{shaded}
\begin{prop}\label{the external cohomotopy pairing}
Suppose that $U\in c\vect{}{}$. Then there is a pairing
\[\ExtCohProd:S_2(\pi^*V)\to \pi^*(S_2V),\]
defined by $x\otimes y\mapsto D^0(x\otimes y)$,
with the property that $\ExtCohProd(\alpha\otimes \alpha)=\ExtCohOp^k\alpha$ for $\alpha\in \pi^kV$.
\end{prop}
\end{shaded}


\subsection{Chain level structure $\xi_\calc $ and $\psi_\calc$ for cohomology operations}\label{chain level structure}
\hfil
\begin{shaded}
\begin{enumerate}
%\item frame the $j$ by analogy with the pairing $S_2X\to X$
\item examples - explain Lie stuff's Koszul duality in terms of Goerss' work
\end{enumerate}
\end{shaded}


We will now give generalisations of Goerss' constructions in \cite[\S5]{MR1089001} to yield useful structure on the complexes calculating cohomology.  Suppose that $X\in s\calc$ is almost free, with $V_s\subset X_s$ the freely generating subspace. Then for each $s$, the functor 
\[\Hom_\calc(X_s,\DASH)\cong \Hom_{\vect{}{}}(V_s,U^\calc\DASH)\]
is naturally an $\Ftwo $-vector space. Writing $\phi_s=\F^\calc(\Delta):X_s\to X_s\sqcup X_s$, the addition operation `$\star$' on $\Hom_\calc(X_s,\DASH)$ is given by $f\star g= (f\sqcup g)\circ\phi_s$. Now let
$\overline{\xi}_\calc=((d_0\sqcup d_0)\phi_s)\star(\phi_{s-1}d_0)$, the sum of maps in the $\Ftwo $-vector space $\textup{Hom}_\calc(X_s,X_{s-1}\sqcup X_{s-1})$. It is completely formal to check that $\overline{\xi}_\calc$ maps to zero in the group
\[\textup{Hom}_\calc(X_s,X_{s-1}\times X_{s-1})=\textup{Hom}_\calc(X_s,X_{s-1})\times \textup{Hom}_\calc(X_s,X_{s-1}),\]
and thus $\overline{\xi}_\calc$ factors through a unique map $\xi_\calc:X_s\to X_{s-1}\smashcoprod X_{s-1}$. Furthermore, $\xi_\calc$ enjoys the symmetry $\tau\xi_\calc=\xi_\calc$, and it is again formal to verify the analogue of \cite[Lemma 5.5]{MR1089001}:
\begin{lem}\label{psi is basically the quadratic part}
The map $Q^\calc\xi_\calc$ induces a chain map of degree $-1$ on normalized complexes:
\[N_s(Q^{\calc}X)\to N_{s-1}((Q^{\calc}(X\smashcoprod X))^{\Sigma_2}).\]
The composite
\[\psi_\calc:\left(N_s(Q^{\calc}X)\overset{Q^\calc\xi_\calc}{\to} N_{s-1}((Q^{\calc}(X\smashcoprod X))^{\Sigma_2})\overset{j_\calc}{\to} N_{s-1}(S^2(Q^{\calc}X))\right),\]
is essentially $\quadratic_\calc$, in that if $v\in V_s\cap N_sX$ represents an element of $N_sQ^\calc X$, writing $d_0v=f(w_j)$ for $w_j\in V_{s-1}$, we have $\psi_\calc(v)=\quadratic_\calc(f)(w_j)\in S^2(V_{s-1})$.
\end{lem}
The typical use of this structure is to define cohomology operations using the external cohomotopy operations defined above, i.e.\ natural operations on $H^*_\calc X=\pi^*(\dual(Q^\calc B^\calc X))$ defined by the composites:
\begin{gather*}
H_\calc^{n_1}(X)\otimes H_\calc^{n_2}(X)\overset{\ExtCohProd}{\to} \dual(\pi_{n_1+n_2}(S^2(Q^\calc B^\calc X)))\overset{\psi_\calc^*}{\to} H_\calc^{n_1+n_2+1}(X),\\
H_\calc^{n}(X)\overset{\ExtCohOp^k}{\to} \dual(\pi_{n+k}(S^2(Q^\calc B^\calc X)))\overset{\psi_\calc^*}{\to} H_\calc^{n+k+1}(X).
\end{gather*}
These operations are the duals of natural homology cooperations, defined using the maps of \S\ref{Linearly dual homotopy operations}:
\begin{gather*}
H^\calc_{n_1}(X)\otimes H^\calc_{n_2}(X)\overset{\ExtCohProd}{\from} \pi_{n_1+n_2}(S^2(Q^\calc B^\calc X))\overset{\psi_\calc^*}{\from} H^\calc_{n_1+n_2+1}(X),\\
H^\calc_{n}(X)\overset{\ExtCohOp^k}{\from} \pi_{n+k}(S^2(Q^\calc B^\calc X))\overset{\psi_\calc^*}{\from} H^\calc_{n+k+1}(X).
\end{gather*}

We may produce a somewhat more general result than lemma \ref{psi is basically the quadratic part}:
\begin{prop}\label{general CohOpns given irreducibility}
Suppose that $\theta:F^\calc V\to GV$ is a natural transformation from $F^\calc$ to another endofunctor $G$ of $\vect{}{}$ satisfying the condition:
\[\smash{\theta\circ\mu_V=\theta\circ \epsilon_{F^\calc V} +\theta\circ {F^\calc \epsilon_V}:F^\calc F^\calc V\to G V.}\]
Write $\smash{\widetilde{\theta}}:Q^{\calc}X_s\to G(Q^{\calc}X_{s-1})$ for the following composite, depending on the almost free structure chosen:
\[\smash{Q^{\calc}X_s\overset{\cong}{\to} V_s\overset{d_0}{\to}F^\calc V_{s-1}\overset{\theta}{\to}GV_{s-1}\overset{\cong}{\to}G(Q^{\calc}X_{s-1})}\]
Then $d_0\circ\widetilde{\theta}=\widetilde{\theta}\circ(d_0+d_1)$, and $d_j\circ\widetilde{\theta}=\widetilde{\theta}\circ d_{j+1}$ for $j\geq1$, so that $\widetilde{\theta}$ restricts to degree $-1$ chain maps $N_sQ^\calc X\to N_{s-1}Q^\calc X$ and $C_sQ^\calc X\to C_{s-1}Q^\calc X$.
%For $v\in V_s$, write $d_0v=g(v_i)$ as an expression in certain $v_i\in V_{s-1}$. Then $\widetilde{\theta}(v):=\theta(g)(v_i)$. Here, we view the $g(v_i)$ as lying in $F^\calc V_{s-1}\cong X_{s-1}$, so that $\theta(g(v_i))\in G V_{s-1}\cong G(Q^\calc X_{s-1})$. Then $\widetilde{\theta}$ is a chain map, even descending to the subspace $N_*Q^\calc X$.
\end{prop}
\begin{proof}
In order to see that $d_j\circ\widetilde{\theta}=\widetilde{\theta}\circ d_{j+1}$ for $j\geq1$, we examine the diagram
\[{\xymatrix@R=4mm{
V_s\ar[r]^-{d_0}
\ar@{..>}[d]^-{d_{j+1}}
&%r1c1
F^\calc V_{s-1}\ar[r]^-{\theta}
\ar[d]^-{d_j}
&GV_{s-1}\ar[r]^-{\cong}
\ar@{..>}[d]^-{d_j}
&%r2c2
GQ^{\calc}X_{s-1}\ar[d]^-{GQ^{\calc}(d_j)}
\\%r1c3
V_{s-1}\ar[r]^-{d_0}
&%r2c1
F^\calc V_{s-2}\ar[r]^-{\theta}
&GV_{s-2}\ar[r]^-{\cong}
&%r2c2
GQ^{\calc}X_{s-2}
}}\]
The dotted verticals are available since $X$ is almost free. The left square commutes by the simplicial identities, and the center square commutes by naturality of $\theta$. For the other relationship, we use the following diagram, which commutes except for the leftmost square:
\[\xymatrix@R=4mm{
V_s\ar[r]^-{d_0}
\ar@{..>}[d]_-{d_{1}+\epsilon\circ d_0}
&%r1c1
F^\calc V_{s-1}\ar[r]^-{\theta}
\ar[d]^-{F^\calc (\epsilon\circ d_0)}
&GV_{s-1}\ar[r]^-{\cong}
\ar@{..>}[d]^-{G(\epsilon\circ d_0)}
&%r2c2
GQ^{\calc}X_{s-1}\ar[d]^-{GQ^{\calc}(d_0)}
\\%r1c3
V_{s-1}\ar[r]^-{d_0}
&%r2c1
F^\calc V_{s-2}\ar[r]^-{\theta}
&GV_{s-2}\ar[r]^-{\cong}
&%r2c2
GQ^{\calc}X_{s-2}
}\]
Now the rightmost two squares both commute, so to show that the outer rectangle commutes, it is enough to see that the two composites $V_{s}\to F^\calc V_{s-2}$ are coequalized by $\theta$. Using the simplicial identity $d_0 d_1=d_0d_0$, we are trying to show that $\theta d_0d_0+\theta d_0\epsilon d_0$ and $\theta F^\calc (\epsilon d_0) d_0$ are the same map from $V_s$ to $GV_{s-2}$. Even more, we will show that $\theta d_0+\theta d_0\epsilon$ and $\theta F^\calc (\epsilon d_0)$ are the same map $F^\calc V_{s-1}$ to $GV_{s-2}$.

Starting with an expression $f(v_i)$ with $v_i\in V_{s-1}$, we calculate $\theta d_0 f(v_i)=\theta (fd_0v_i)$, $\theta d_0\epsilon f(v_i)=\theta (\epsilon (f)(d_0v_i))$ and $\theta F^\calc (\epsilon d_0)(f(v_i))=\theta(f)(\epsilon(d_0v_i))$. That these three terms add to zero was the requirement specified for $\theta$.
\end{proof}



\subsection{Cohomology operations for simplicial commutative algebras}\label{The example of simplicial commutative F2-algebras}
Goerss \cite[\S5]{MR1089001} defines cohomology operations, natural in $A\in s \algs$:
\begin{gather*}
P^i=\psi^*_{\algs}\circ\ExtCohOp^i:H^n_{\algs}A\to H_{\algs}^{n+i+1}A;\textup{ and}\\
[\,,]=\psi^*_{\algs}\circ\ExtCohProd :H_{\algs}^nA\otimes H_{\algs}^mA\to H_{\algs}^{n+m+1}A.
\end{gather*}
He also defines a natural operation $\beta:H_{\algs}^0A\to H_{\algs}^1A$. Note that as a result of our use of $\psi^*_{\algs}$, these operations have a grading shift.
\begin{prop}[{\cite[\S5]{MR1089001}}]\label{omnibus on coh of simp algs}These operations have the following properties:
\begin{enumerate}
\item the bracket gives $H^*_{\algs}A$ the structure of an $S(\LieOperad)$-algebra (with grading shift);
\item the operation $\beta$ acts as a restriction defined only in dimension zero, so that for $x,y\in H^0_{\algs}A$ and $z\in H^*_{\algs}A$:
\[\beta(x+y)=\beta(x)+\beta(y)=[x,y],\text{\ \ and \ }[\beta(x),z]=[x,[x,z]];\]
\item the self-square operation on $H^*_{\algs}A$ equals the top $P$-operation:\[P^nx=[x,x]\text{\ \ for $x\in H^n_{\algs}A$};\]
\item \label{P unstable vanishing} if $x\in H^n_{\algs}A$, then $P^ix=0$ unless $2\leq i\leq n$;
\item every $P$-operation is linear;
\item there holds the following \emph{Cartan formula}:  for all $x,y\in   H^n_{\algs}A$ and $i\geq0$,
\[[x,P^iy]=0\]
\item \label{yeah H is a Pmodule}the \emph{$P$-Adem relations} hold: if $i\geq 2j$, then
\[P^iP^jx=\sum_{s=i-j+1}^{i+j-2}{2s-i-1\choose s-j}P^{i+j-s}P^sx.\]
\end{enumerate}
\end{prop}
In this case, \emph{(\ref{yeah H is a Pmodule})} does state that $H^*_{\algs}A$ is a left module over $\Palg$, the \emph{Steenrod algebra for commutative algebras over $\Ftwo $}, the non-unital associative algebra generated by symbols $P^i$ for $i\geq0$, modulo the two sided ideal generated by $P^0$, $P^1$, and the evident \emph{$P$-Adem relations}.


A sequence $I=(i_\ell,\ldots,i_1)$ of integers $i_j\geq2$ is \emph{$P$-admissible} if $i_{j+1}<2i_j$ for $1\leq j <\ell$. For any sequence $I=(i_\ell,\ldots,i_1)$, write $P^I$ for the composite $P^{i_\ell}\cdots P^{i_1}$. It follows from \cite[Theorem I]{MR1089001} that $\Palg$ has an \emph{admissible basis}, consisting of those $P^I=P^{i_\ell}\cdots P^{i_{1}}$ with $I$ a $P$-admissible sequence.  Again, we will say that \emph{$I$ produces $J$ in $\Palg$}, denoted $\produces{I}{J}{\Palg}$ if, when $P_I$ is written in the $P$-admissible basis of $\Palg$, $P_J$ appears with non-zero coefficient. In this case, $J$ must be $P$-admissible, and $I$ must be $P$-inadmissible unless $J=I$.

We define
\[\minDimP(I):=\max\{(i_1),\,(i_2-i_1-1),\,(i_3-i_2-i_1-2),\,\ldots,\,(i_{\ell}-\cdots-i_1-\ell+1)\},\]
for a rather different purpose than in \S\ref{Homotopy operations for simplicial commutative algebras} and \S\ref{Homotopy operations for simplicial Lie algebras}: the composite 
\[P^I:\bigl(H_{\algs}^{n}A\overset{P^{i_1}}{\to}H_{\algs}^{n+i_1+1}A\overset{P^{i_2}}{\to}\cdots \overset{P^{i_\ell}}{\to}H_{\algs}^{n+i_1+\cdots +i_\ell+\ell}A\bigr)\]
is \emph{forced to be zero} by \emph{(\ref{P unstable vanishing})} alone precisely when $n<\minDimDelta(I)$.

As in \S\ref{Homotopy operations for simplicial commutative algebras} and \S\ref{Homotopy operations for simplicial Lie algebras}, if a non-empty sequence $I$  is \emph{$P$-admissible}, we can identify which term is largest in the maximum defining $\minDimP(I)$, namely $i_1$, so that $\minDimP(I)=i_1$. More explicitly, in a (non-vanishing) admissible expression $P^{i_\ell}\cdots P^{i_1}x$, for $x\in H^n_{\algs}A$, the only one of these $P$-operations that can be a top operation is $P^{i_{1}}$.
%Unlike in \S\ref{Homotopy operations for simplicial commutative algebras} and \S\ref{Homotopy operations for simplicial Lie algebras}, having a non-empty sequence $I$  be \emph{$P$-admissible} does not allow us identify which term is largest in the maximum defining $\minDimP(I)$. More explicitly, in an admissible expression $P^{i_\ell}\cdots P^{i_1}x$, for $x\in H^n_{\algs}A$, top $P$-operations may be interspersed with non-top $P$-operations.


%\begin{lem}\label{lemOnAdemChangeInMDeltaPlain}
%Suppose that $i,j\geq2$ and $i<2j$, and $(i+1)/2\leq s\leq (i+j)/3$. Then $i+j-s\geq 2s$, $i+j-s\geq2$, $s\geq2$, so that the $\delta$-Adem relation writes $\delta_i\delta_j$ as a sum of $\delta$-admissible composites. Moreover,  $\minDimDelta(i,j)\geq \minDimDelta(i+j-s,s)$.
%\end{lem}
%\begin{proof}
%The only tricky inequality is $\minDimDelta(i,j)\geq \minDimDelta(i+j-s,s)$. Now the right hand side must equal $i+j-2s$, and the left hand side is at least $j$. The result follows, since $2s\geq i+1$.
%\end{proof}

The following result shows that whenever an expression $P^Jx$ is forced to be zero by \emph{(\ref{P unstable vanishing})}  and we reduce said expression to a sum of $P$-admissible composites, all of the summands are forced to be zero by \emph{(\ref{P unstable vanishing})}.
\begin{lem}\label{lemOnAdemChangeInMP}
If $\produces{I}{J}{\Palg}$, then $\minDimP(J) \geq \minDimP(I)$, with strict inequality when $I$ and $J$ are distinct and of length two.
\end{lem}
The main theorem of Goerss' memoir is that these operations generate all of the operations in the category $\algs\HAlg$, and that all of the relations between the operations in $\algs\HAlg$ are implied by those presented here. In \cite[Chapter V]{MR1089001}, Goerss shows that the listed operations completely capture the cohomology of an object $K^{\algs}\mathbb{K}_n$. He proves a Hilton-Milnor theorem \cite{GoerssHiltonMilnor.pdf}, which he uses in \cite[\S11]{MR1089001} to bootstrap up to a calculation of the cohomology of any GEM in $s\algs$, namely \cite[Theorem I]{MR1089001}. The result states that whenever $V\in \vect{1}{}$,  is a finite-type vector space, not only is 
$F^{\algs}V$
generated by $V$ under the operations of proposition \ref{omnibus on coh of simp algs}, it is as large as is conceivable given the relations presented. We will present in proposition \ref{partialgoerss} a partial version of his result.

It is  interesting to observe that $\Palg$, the Steenrod algebra for commutative algebras is
in fact \emph{Koszul dual}, in the sense of \cite{PriddyKoszul.pdf}, to $\deltaalg$, the algebra which (almost) acts on the homotopy of a simplicial algebra. Indeed, Goerss \emph{calculates} $\Palg$ as the Koszul dual of $\deltaalg$, by means of calculations in a reverse Adams spectral sequence due to Miller \cite{MillerSullivanConjecture.pdf}. We will explore this relationship further when we consider the Bousfield-Kan spectral sequence.

\subsection{The categories $\calw(0)$ and $\calU(0)$}
Suppose that $A\in s\algs$ is \emph{connected}: $\pi_0A=0$. Then $H^0_{\algs}=Q^{\algs}\pi_0A=0$. It is advantageous to work with the cohomology of connected objects, as we do not need to worry about the operation $\beta$. If we say that $V\in \vect{1}{}$ is \emph{connected} when $V^{0}=0$, we may identify the full subcategory of $\vect{1}{}$ on the connected objects with $\vect{+}{}$.

Goerss' result proves that the monad
$F^{\algs\HAlg}$ on $\vect{1}{}$
preserves $\vect{+}{}$ (and indeed it is a general fact that no non-trivial natural cohomology operations decrease dimension).
We will write $\calw(0)$ for the category of connected $\algs$-$H^*$-algebras, so that the monad $F^{\calw(0)}$ on $\vect{+}{}$ is simply the restriction of $F^{\algs\HAlg}$. The way that we will report Goerss' result here is to explain how the monad $F^{\calw(0)}$ may be constructed on objects of $\vect{+}{}$ of finite type.

Let the \emph{category of unstable $\Palg$-modules}, denoted $\calU(0)$, be the category whose objects are $\vect{+}{0}$-graded $\Palg$-modules in which $P^i$ acts with grading $i+1$, and such that $P^i:V^n\to V^{n+i+1}$ is zero unless $i\leq n$. Recall that  we have \emph{already} imposed the $P$-Adem relations and set $P^0$ and $P^1$ to be zero in $\Palg$. 
\begin{prop}
The monad $F^{\calU(0)}$ may be defined as follows:
\begin{alignat*}{2}
F^{\calU(0)}V
:={}&
(\Palg\otimes V)\,/\,\Ftwo \{P^I\otimes v\ |\ V\in V^n,\ \minDimP(I)>n \}%
\\
={}&
(\Palg\otimes V)\,/\,\Ftwo \{P^I\otimes v\ |\ V\in V^n,\ \minDimP(I)>n ,\ I\textup{ is $P$-admissible}\}.%
\end{alignat*}
\end{prop}
\begin{proof}
This follows from Goerss' \cite[Theorem I]{MR1089001} and lemma \ref{lemOnAdemChangeInMP}.
\end{proof}
Now an object of $\calw(0)$ is in particular an object of $\calU(0)$. It is also a (degree shifted) $S(\LieOperad)$-algebra. Thus, there  is a natural map 
\[F^{\calU(0)}F^{S(\LieOperad)}V\to F^{\calw(0)}V.\]
This map is not an isomorphism, but it follows from \cite[Theorem I]{MR1089001} that it is surjective. Moreover, our final reading of Goerss' result is:
\begin{prop}\label{partialgoerss}
For $V\in\vect{+}{}$of finite type,  $F^{\calw(0)}V\in \vect{+}{}$ is the coequalizer:
\[\textup{coeq}\left(\xymatrix@R=4mm{
\Palg\otimes F^{S(\LieOperad)}V
\ar@<+.5ex>[r]^-{\smash{\textup{sb}_1}}
\ar@<-.5ex>[r]_-{\textup{sb}_2}
&%r1c1
\Palg\otimes F^{S(\LieOperad)}V
\ar@{->>}[r]
&
F^{\calu(0)}F^{S(\LieOperad)}V%r1c2V
%r1c3
}\right),\]
where the maps $\textup{sb}_1$ and $\textup{sb}_2$ are defined on  $\Palg\otimes(F^{S(\LieOperad)}V)^{m}$ by
\[\textup{sb}_1(P^I\otimes x)=P^I\otimes [x,x]\textup{ \ and \ }\textup{sb}_2(P^I\otimes x)=P^IP^m\otimes x.\]
Suppose that one chose a homogeneous basis of $V$, constructed from it a monomial  basis of $\Lambda(\LieOperad)V$ (such as any choice of Hall basis), and then lifted these monomials in the evident way to a collection $B$ of elements of $S(\LieOperad)V$. Then a basis of $F^{\calw(0)}V$ would be given by
\[\left\{P^Ib\ \middle|\ b\in B,\ \minDimP(I)\leq|b|,\ \textup{$I$ is $P$-admissible}\right\}.\]
\end{prop}
\begin{cor}\label{finite type pres by FW0}
Suppose that $V\in\vect{+}{}$ is of finite type. Then so is $F^{\calw(0)}V$. \textbf{same for Lie}
\end{cor}
%An object of $\calw(0)$ is simultaneously an $S(\LieOperad)$-algebra and an unstable left $\Palg$-module.
%
%
%\begin{prop}[{\cite[\S5]{MR1089001}}]\label{omnibus on coh of simp algs}These operations have the following properties:
%\begin{enumerate}
%\item the bracket gives $H^*_{\algs}A$ the structure of an $S(\LieOperad)$-algebra (with grading shift);
%\item the operation $\beta$ acts as a restriction defined only in dimension zero, so that for $x,y\in H^0_{\algs}A$ and $z\in H^*_{\algs}A$:
%\[\beta(x+y)=\beta(x)+\beta(y)=[x,y],\text{\ \ and \ }[\beta(x),z]=[x,[x,z]];\]
%\item the self-square operation on $H^*_{\algs}A$ equals the top $P$-operation:\[P^nx=[x,x]\text{\ \ for $x\in H^n_{\algs}A$};\]
%\item \label{P unstable vanishing} if $x\in H^n_{\algs}A$, then $P^ix=0$ unless $2\leq i\leq n$;
%\item every $P$-operation is linear;
%\item there holds the following \emph{Cartan formula}:  for all $x,y\in   H^n_{\algs}A$ and $i\geq0$,
%\[[x,P^iy]=0\]
%\item \label{yeah H is a Pmodule}the \emph{$P$-Adem relations} hold: if $i\geq 2j$, then
%\[P^iP^jx=\sum_{s=i-j+1}^{i+j-2}{2s-i-1\choose s-j}P^{i+j-s}P^sx.\]
%\end{enumerate}
%\end{prop}
%
%
%
%
%
%Now let $\calw(0)$ be the category of connected objects in Goerss' category `$\calw$' \cite[Definition G]{MR1089001}. Explicitly, $\calw(0)$ has objects $\vect{+}{0}$-graded vector spaces $X$, equipped with:
%\begin{enumerate}
%\item a commutative, bilinear product $[\,,]:X^n\otimes X^m\to X^{n+m+1}$ satisfying the Jacobi identity; and
%\item linear maps $P^i:X^n\to X^{n+i+1}$ such that
%\begin{enumerate}
%\item $P^i=0$ unless $2\leq i\leq n$, and $P^nx=[x,x]$;
%\item $[x,P^iy]=0$ for all $x$, $y$ and $i$; and
%\item satisfying the $P$-Adem relation:
%\[P^iP^j=\sum_{s=i-j+1}^{i+j-2}{2s-i-1\choose s-j}P^{i+j-s}P^s\textup{ whenever $i\geq 2j$.}\]
%\end{enumerate}
%\end{enumerate}
%Then Goerss's results \cite[Theorem H]{MR1089001}, restricted to the full subcategory of connected objects in $s\algs$ (i.e.\ objects $A$ with $\pi_0A=0$, or equivalently $H^0_{\algs}A=0$), state that the operations $[\,,]$ and $P^i$ give $H^*A$ the structure of an element of $\calw(0)$, and moreover, that these are the only natural operations on the cohomology of connected objects of $s\algs$ \cite[Theorem I]{MR1089001}. That is, Goerss shows that the above relations are satisfied by the operations defined, and then calculates the cohomology of an abelian object in $s\algs$; neither of these two steps is straightforward.
%
%\textbf{used:}[Let $\calU(0)$ be the category whose objects are $\vect{+}{0}$-graded vector spaces $V$ equipped with linear left $P$-operations $P^i:V^n\to V^{n+i+1}$, which are zero unless  $2\leq i\leq n$, and satisfy the $P$-Adem relation as above.]
%
%Any object $X$ of $\calw(0)$ is in particular an object of $\calU(0)$. It is \emph{not} true, however, that $X$ is a Lie algebra. Indeed, the relations above do not imply that $[x,x]=0$ for any $x\in X$ (see \cite[p.\ 18]{MR1089001}). Rather, $X$ is an algebra over the monad $S(\LieOperad)$ discussed in \S\ref{sec on Lie algs and homotopy ops}. We will describe the construction of the monads $F^{\calw(0)}$ and $F^{\calU(0)}$ in \S\ref{Construction of the monads}.
\begin{shaded}\tiny
Denote by $\Palg$ the algebra arising in Goerss' work, defined as a quotient of a free unital associative algebra:
\[\Palg=F^{\textup{ass}}(P_0,P_1,P_2,\ldots)/\left(P_0=P_1=0,\ \textup{$P$-Adem relation}\right)\]
This associative algebra can be $\vect{+}{0}$-graded, by thinking of the generator $P^i$ as having grading $i+1$ (\textbf{or} even in $\vect{1}{1}$, taking $P^i\in(\Palg)^{i+1}_1$). For any $x\in\Palg$, there is a homomorphism, \emph{right multiplication} by $x$:
\[\textup{m}_x:\Palg\to \Palg,\quad y\mapsto yx.\]
Moreover, $\Palg$ has an \emph{admissible basis}, consisting of those monomials
\[P^I:=P^{i_\ell}P^{i_{\ell-1}}\cdots P^{i_1}\]
for which $I=(i_\ell,\ldots,i_1)$ is a sequence of integers such that each $i_j\geq2$, and  each comparison $i_{j+1}< 2i_j$ holds. Moreover, $\Palg$ has a decreasing filtration $\Palg\supset F^1\Palg\supset F^2\Palg\supset\cdots $, where
\[F^n\Palg=\Ftwo \left\{P^I\,:\,\textup{$I$ non-empty and admissible, }i_j+\cdots +i_1\geq n+j\textup{ for some $j\geq 1$}\right\}.\]
By examining the $P$-Adem relation, one can determine that this is a filtration by left ideals. For a vector space $V\in\vect{+}{0}$, we have
\[F^{\calU(0)}V=\Palg\otimes V/\bigoplus_{i\geq1}(F^n\Palg\otimes V^n)\]
Next, writing $S(\LieOperad)$ for the monad on $\vect{+}{0}$ with cohomological grading shifted, we may construct the monad $F^{\calw(0)}$ (the restriction of the monad $U$ in \cite[p.\ 18]{MR1089001}), by coequalizing two maps $\Palg\otimes S(\LieOperad)V\to F^{\calU(0)} (S(\LieOperad)V)$. The first map is the composite:
\[\xymatrix@R=4mm{
\Palg\otimes S(\LieOperad)V\ar[r]^-{\Id\otimes \textup{frob}}
&%r1c1
\Palg\otimes S(\LieOperad)V\ar@{->>}[r]&%r1c2
F^{\calU(0)}(S(\LieOperad)V).%r1c3
}\]
The second map restricts to each subset $\Palg\otimes (S(\LieOperad)V)^r$ to the composite:
\[\xymatrix@R=4mm{
\Palg\otimes (S(\LieOperad)V)^r\ar[r]^-{\textup{m}_{P^r}\otimes\Id}
&
\Palg\otimes (S(\LieOperad)V)^r\ar@{^{(}->}[r]
&%r1c1
\Palg\otimes S(\LieOperad)V\ar@{->>}[r]&%r1c2
F^{\calU(0)}(S(\LieOperad)V).%r1c3
}\]
One can check, using \cite[p.\ 18]{MR1089001} the following is a coequalizer in $\vect{+}{0}$:
\[\xymatrix@R=4mm{
\Palg\otimes S(\LieOperad)V\ar@<.5ex>[r]\ar@<-.5ex>[r]
&%r1c1
F^{\calU(0)}(S(\LieOperad)V)\ar@{->>}[r]
&%r1c2
F^{\calw(0)}V%r1c3
}\]
The monad $F^{\calw(0)}$ possesses a quadratic grading, as do the $\calw(n)$ for $n\geq1$. One views $V\subset F^{\calw(0)}V$ as lying in quadratic grading 1, and demands that quadratic grading adds under taking brackets and doubles when applying $P$-operations.
\end{shaded}
An object of $\calw(0)$ is in particular an object of $\calU(0)$, and (as all of the $P$ operations are linear), we can define a functor $Q^{\calU(0)}:\calw(0)\to\vect{+}{0}$ which takes the quotient by the image of these operations. Moreover:
\begin{lem}\label{Kill P ops gives lie alg}
For $X\in\calw(0)$, the vector space $Q^{\calU(0)}X\in\vect{+}{}$ inherits a (grading shifted) Lie algebra structure from the bracket of $X$, yielding a factorization:% of indecomposable functors, $Q^{\calw(n)}=Q^{\calL(n)}\circ Q^{\calU(n)}$
\[Q^{\calw(0)}=Q^{\Lambda(\LieOperad)}\circ Q^{\calU(0)}:\left(\calw(0)\to \Lambda(\LieOperad)\to \vect{+}{0}\right).\]
Moreover the composite $Q^{\calU(0)}\circ F^{\calw(0)}$ equals the free construction $F^{\calL(0)}$.
\end{lem}
\begin{proof}
One checks that the bracket is well defined in the quotient, and that taking the quotient by the top $P$ operation imposes the relation $[x,x]=0$, to create a $\Lambda(\LieOperad)$-algebra from the pre-existing $S(\LieOperad)$-algebra structure. The final claim follows from proposition \ref{partialgoerss}.
\end{proof}


\end{Constructing (co)homotopy operations}






\begin{homotopy operations for PRLs}

\section{\textbf{Homotopy operations for partially restricted Lie algebras}}\label{sec on Lie algs and homotopy ops}

\subsection{The categories $\calL(n)$}\label{The categories Ln}
[\textbf{Old version} of spiel on PRLie algs commented out here.%
%Let $\LieOperad$ be the Lie operad in characteristic 2. As explained in [Fresse], the monad $S(\LieOperad)$ on $\vect{}{}$ defined by
%\[S(\LieOperad)(V)=\bigoplus_{n\geq1}(\LieOperad(n)\otimes V^{\otimes n})_{\Sigma_n}\]
%does not return the free Lie algebra on $V$ in the traditional sense. Rather, an $S(\LieOperad)$-algebra is a vector space $L$ equipped with a bracket $L\otimes L\to L$ satisfying the Jacobi identity and the antisymmetry condition $[x,y]=-[y,x]$. This condition does \emph{not} imply the alternating condition $[x,x]=0$ (which is impossible to encode operadically). We will refer to structures of this type as $S(\scrL)$-algebras, to distinguish them from Lie algebras, which we demand satisfy the alternating condition.
%
%Fresse constructs a monad $\Gamma(\LieOperad)$ defined by
%\[\Gamma(\LieOperad)(V)=\bigoplus_{n\geq1}(\LieOperad(n)\otimes V^{\otimes n})^{\Sigma_n},\]
%which represents the free restricted [See Curtis or 6A or something] Lie algebra. He further constructs a norm map $\textup{Tr}:S(\LieOperad)\to \Gamma(\LieOperad)$, and defines a monad $\Lambda(\LieOperad)$ by the formula
%\[\Lambda(\LieOperad)V=\im (\textup{Tr}:S(\LieOperad)V\to \Gamma(\LieOperad)V).\]
%Then $\Lambda(\LieOperad)(V)$ is the free Lie algebra on $V$, subject to the standard axiom, $[x,x]=0$.
%
%We will be interested in \emph{partially restricted Lie algebras}. That is, Lie algebras $L$ equipped with a decomposition $L=L_0\oplus L_1$ such that $L_1$ is a Lie ideal and a map $\restn{(\DASH)}:L_1\to L_1$ satisfying the axioms one would expect for a restriction map. Namely, for any $a,b\in L_1$ and $c\in L$, $\restn{(a+b)}=\restn{a}+\restn{b}+[a,b]$, and $[\restn{a},c]=[a,[a,c]]$. There is a free functor from $\vect{}{}\times \vect{}{}$ to partially restricted Lie algebras, sending $(V_0,V_1)$ to the algebra generated by restrictable elements $V_1$ and non-restrictable elements $V_0$. This free functor may be described as follows. There is a natural $\Sigma_n$-equivariant decomposition: %The expressions for $S(\LieOperad)(V_0\oplus V_1)$ and $\Gamma(\LieOperad)(V_0\oplus V_1)$ both contain the terms:
%\[\LieOperad(n)\otimes (V_0\oplus V_1)^{\otimes n}=\Bigl(\LieOperad(n)\otimes (V_0)^{\otimes n}\Bigr)\oplus\Bigl(\bigoplus(\LieOperad(n)\otimes V_{i_1}\otimes V_{i_2}\otimes\cdots\otimes V_{i_n})\Bigr)\]
%where the direct sum is taken over all sequences $(i_1,\ldots,i_n)\in\{0,1\}^n$ other than $(0,\ldots,0)$.
%%and this decomposition is of $\Sigma_n$-modules.
%The free partially restricted Lie algebra is the subspace of $\Gamma(\LieOperad(V_0\oplus V_1))$ given by
%\[\im \Bigl(S(\LieOperad)(V_0)\overset{\textup{Tr}}{\to}\Gamma(\LieOperad)(V_0\oplus V_1)\Bigr)\oplus\bigoplus_{n\geq1} \Bigl(\bigoplus(\LieOperad(n)\otimes V_{i_1}\otimes V_{i_2}\otimes\cdots\otimes V_{i_n})\Bigr)^{\Sigma_n}\]
]%END commented out old material
For each $n\geq0$, we will be interested in certain categories of Lie algebras monadic over $\vect{+}{n}$, with a grading shift. Broadly, a $\vect{+}{n}$-graded Lie algebra is an element $L\in\vect{+}{n}$ with a structure map $\Lambda^2 L\to L$ which treats gradings as follows
\[L^{t}_{s_n,\ldots,s_1}\otimes L^{t'}_{s'_n,\ldots,s'_1}\to L^{t+t'+1}_{s_n+s'_n,\ldots,s_1+s'_1}.\]
More precisely, we view the Lie operad as an operad in $\vect{+}{n}$, such that
\[\LieOperad(r)=(\LieOperad(r))^{r-1}_{0,\ldots,0},\]
and then a $\vect{+}{n}$-graded Lie algebra is an algebra over the corresponding monad $\Lambda(\LieOperad)$ on $\vect{+}{n}$. In our context, the Lie operad arises as the Koszul dual of the commutative operad, through the constructions in \cite[\S5]{MR1089001}. See \cite[\S5.3.4]{FresseKoszulDuality.pdf} for a discussion of Koszul duality of operads in positive characteristic.

A $\vect{+}{n}$-graded \emph{partially restricted} Lie algebra is to be a graded Lie algebra such that certain graded parts admit a restriction operation. Specifically, there is to be defined a restriction operation
\[\restn{(\DASH)}:L^t_{s_n,\ldots,s_1}\to L^{2t+1}_{2s_n,\ldots,2s_1}\]
whenever not all of $s_n,\ldots,s_{1}$ are zero. We will denote the category of such objects $\calL(n)$. It is monadic over $\vect{+}{n}$, with an adjunction
\[F^{\calL(n)}:\vect{+}{n}\rightleftarrows \calL(n):U^{\calL(n)}.\]
The monad of this adjunction may be constructed as an appropriately chosen submonad of $\Gamma(\LieOperad):\vect{+}{n}\to \vect{+}{n}$ ($\LieOperad$ shifted as above), containing $\Lambda(\LieOperad)$. As such, the free construction $F^{\calL(n)}V$ admits a quadratic grading, which we denote $\quadgrad{k}F^{\calL(n)}V$.  (\textbf{say something} about grading shifts here?)
%To describe the free functor, if $V\in\vect{+}{n}$, write:
%\[V_-=\bigoplus_{t\geq1}V^t_{0,\ldots,0},\textup{ and }V_+=\bigoplus_{t\geq1}\bigoplus_{\exists\,i\textup{\,s.t.\,}s_i\neq0}V^t_{s_n,\ldots,s_1}.\]
%There is a natural $\Sigma_n$-equivariant decomposition: 
%\[\LieOperad(n)\otimes (V_0\oplus V_1)^{\otimes n}=\Bigl(\LieOperad(n)\otimes (V_0)^{\otimes n}\Bigr)\oplus\Bigl(\bigoplus(\LieOperad(n)\otimes V_{i_1}\otimes V_{i_2}\otimes\cdots\otimes V_{i_n})\Bigr)\]
%where the direct sum is taken over all sequences $(i_1,\ldots,i_n)\in\{0,1\}^n$ other than $(0,\ldots,0)$.
%The free partially restricted Lie algebra is the subspace of $\Gamma(\LieOperad(V_0\oplus V_1))$ given by
%\[\im \Bigl(S(\LieOperad)(V_0)\overset{\textup{Tr}}{\to}\Gamma(\LieOperad)(V_0\oplus V_1)\Bigr)\oplus\bigoplus_{n\geq1} \Bigl(\bigoplus(\LieOperad(n)\otimes V_{i_1}\otimes V_{i_2}\otimes\cdots\otimes V_{i_n})\Bigr)^{\Sigma_n}\]

\textbf{Construction of the monads $F^{\calw(0)}$ and $F^{\calU(0)}$}
\tododone{Construct monad $U$ for $\calw$}{Write in terms of monads $\PMonad$ and $S(\scrL)$.\item Decide whether to give the distributive law.
}
\textbf{Moved!!!}

\subsection{Unary $\calL(n)$-homotopy operations and the category $\calU(n+1)$}
Or reorganize... but need to make a point of defining $U(n+1)$

\subsection{The category $\calw(n+1)$ of $\calL(n)$-$\Pi$-algebras}
\textbf{(much of this now moved...)} We will now give an account of the natural operations on the homotopy of a simplicial object of $\calL(n)$.

The homotopy operations we will present will be in the form of a natural transformation of functors $s\vect{+}{n}\to \vect{+}{n+1}$, which will be induced by the Eilenberg-Mac Lane shuffle map $\nabla:N_*(V)\otimes N_*(V)\to N_*(V\otimes V)$. \textbf{Need to clarify what is meant by quad grading.}
\begin{prop}[Various authors]\label{the top homotopy operations for Lie algebras}
There is a natural commuting diagram:
\[\xymatrix@R=4mm{
\quadgrad{2}F_{\calL(n+1)}(\pi_*V)\ar@{-->}[r]^-{\widetilde{\nabla}}
\ar[d]^-{\textup{incl}}
&%r1c1
\pi_*(\quadgrad{2}F_{\calL(n)}V)\ar[d]^-{\pi_*(\textup{incl})}
\\%r1c2
S^2(\pi_*V)\ar[r]^-{\widetilde{\nabla}}&%r2c1
\pi_*(S^2V)%r2c2
}\]
%\[\widetilde{\nabla}:\quadgrad{2}F_{\calL(n+1)}(\pi_*V)\to \pi_*(\quadgrad{2}F_{\calL(n)}V)\]
with horizontal maps defined by the formulae (for cycles $x,y,z\in ZN_*(V)$):
\[\widetilde{\nabla}(\overline{x}\otimes \overline{y}+\overline{y}\otimes \overline{x})=\overline{\nabla(x\otimes y+y\otimes x)}\textup{, and }\widetilde{\nabla}(\overline{x}\otimes\overline{x})=\overline{\nabla(x\otimes x)}.\]
\end{prop}
\begin{proof}
Since $\nabla$ is symmetric, the formulae given for $\widetilde{\nabla}$ do indeed return homotopy classes in either $\pi_*(\quadgrad{2}F_{\calL(n)}V)$ or $\pi_*(S^2V)$, however, it is not clear that either of the maps $\widetilde{\nabla}$ is well defined. We will only need to check that the lower map is well defined: as $\pi_*(\textup{incl})$ is a monomorphism (by \cite[Prop 5.6]{BousOpnsDerFun.pdf}), it will follow that the upper map is well defined. 
%Although it is not clear that $\widetilde{\nabla}$ is well defined, $\widetilde{\nabla}$ will prove to be the unique lift in a diagram: %We will construct it as the unique lift (dotted) of an analogous map $\widetilde{\nabla}'$:
%\[\xymatrix@R=4mm{
%\quadgrad{2}F_{\calL(n+1)}(\pi_*V)\ar@{-->}[r]^-{\widetilde{\nabla}}
%\ar[d]^-{\textup{incl}}
%&%r1c1
%\pi_*(\quadgrad{2}F_{\calL(n)}V)\ar[d]^-{\pi_*(\textup{incl})}
%\\%r1c2
%S^2(\pi_*V)\ar[r]^-{\widetilde{\nabla}'}&%r2c1
%\pi_*(S^2V)%r2c2
%}\]
%in which the map $\pi_*(\textup{incl})$ is a monomorphism (by \cite[Prop 5.6]{BousOpnsDerFun.pdf}). Now $\widetilde{\nabla}'$ will be defined by the same formulae as $\widetilde{\nabla}$. 
\begin{shaded}\tiny
By general observations on natural transformations out of the endofunctor $S^2$ of $\vect{}{}$, to construct a natural map $\widetilde{\nabla}:S^2(\pi_*V)\to\pi_*(S^2V)$ is to give a natural linear map $p:S_2(\pi_*(V))\to \pi_*(S^2V)$ and a natural function $\restn{(\DASH)}:\pi_n(V)\to \pi_{2n}(S^2V)$ satisfying $\restn{(x+y)}=\restn{x}+\restn{y}+p(x\otimes y)$. Then $\widetilde{\nabla}$ is the unique natural map such that $\widetilde{\nabla}\circ\trace=p$ and $\widetilde{\nabla}(a)=p(a\otimes a)$ for $a\in\pi_*\left(V\right)$.
%\[p=\left(S_2\pi_*(V)\overset{\trace}{\to} S^2\pi_*(V)\overset{\widetilde{\nabla}'}{\to} \pi_*(S^2V)\right)\textup{ and }\restn{(\DASH)}=\left(\pi_*(V)\overset{v\mapsto v\otimes v}{\to} S^2\pi_*(V)\overset{\widetilde{\nabla}'}{\to} \pi_*(S^2V)\right).\]
We define $\widetilde{\nabla}$ in this way, with
\[p=\pi_*(\trace)\circ\nabla:\left(S_2(\pi_*(V))\overset{\nabla}{\to}\pi_*(S_2(V))\overset{\pi_*(\trace)}{\to}\pi_*(S^2(V))\right)\]
and with $\restn{(\DASH)}$ the operation $\sigma_n:\pi_n(V)\to \pi_{2n}(S^2V)$ detailed in \cite[\S3]{MR1089001}.
\end{shaded}
%In general, to construct a natural map $g$ from $S^2$ to any other endofunctor $G$ of the category of $\Ftwo $-vector spaces, it is enough to define a natural linear map $p:S_2V\to GV$ and natural function $\restn{(\DASH)}:V\to GV$ satisfying $\restn{(x+y)}=\restn{(x)}+\restn{(y)}+p(x\otimes y)$. Then there is a unique natural map $g:S^2\to G$ such that we have
%\[p=\left(S_2V\overset{\trace}{\to} S^2V\to GV\right)\textup{ and }\restn{(\DASH)}=\left(V\overset{v\mapsto v\otimes v}{\to} S^2V\to GV\right).\]
%The map $\widetilde{\nabla}'$ is exactly obtained by this method, using 
%
%\[p([x]\otimes [y])=[\nabla(x\otimes y+y\otimes x)],\textup{ and }\restn{([x])}=[\nabla(x\otimes x)].\qedhere\]
\end{proof}
It is classical that via this map, the homotopy of a simplicial object $L\in s\calL(n)$ obtains the structure of an object of $\calL(n+1)$:
\begin{prop}[Curtis, 6-A]\label{prop on top operations}
If $L\in s\calL(n)$ has structure map $\rho:F^{\calL(n)}L\to L$, then $\pi_*(L)\in\calL(n+1)$, with structure map extending $\pi_*(\rho)\circ\widetilde{\nabla}:\quadgrad{2}F_{\calL(n+1)}(\pi_*L)\to \pi_*L$.
\end{prop}
\begin{proof}
This theorem follows from the analogous theorem for fully restricted simplicial Lie algebras $L$, once the operations $\widetilde{\nabla}$ have been identified as the operations of \cite[\S8.5]{CurtisSimplicialHtpy.pdf}. In the proposed $\calL(n+1)$-structure, for $x,y
\in N_*L$ the bracket of $\overline{x},\overline{y}\in\pi_*L$ is given by
\[\overline{\rho(\nabla(x\otimes y+y\otimes x))}=\overline{\rho(\trace(\nabla(x\otimes y)))}=\overline{\textup{br}(\nabla(x\otimes y))},\]
and if $n\geq1$, the restriction of an element $\overline{x}\in\pi_nL$ is given by
\[\overline{\rho\nabla(x\otimes x)}=\overline{\rho\trace(\nabla^0(x\otimes x))}=\overline{\textup{br}(\nabla^0(x\otimes x))},\]
and these two formulae coincide with those of \cite{CurtisSimplicialHtpy.pdf}. The point here was that both before and after taking homotopy groups, brackets are defined using the trace. We should also check that in the proposed structure, the proposed restriction $\pi_0L\to \pi_0L$ is given on the chain level by the restriction $L_0\to L_0$, as in \cite{CurtisSimplicialHtpy.pdf}. As $\nabla^0:N_0L\otimes N_0L\to N_0(L\otimes L)$ is the identity, for $\overline{x}\in\pi_*L$:
\[(\pi_*(\rho)\circ\widetilde{\nabla})(\overline{x})=\overline{\rho(\nabla(x\otimes x))}=\overline{\rho(x\otimes x)}=\overline{\restn{x}}.\qedhere\]
\end{proof}
We will now turn to some further operations defined on the homotopy of an object of $s\calL(n)$.
\begin{prop}[\cite{CurtisSimplicialHtpy.pdf,6Author.pdf}]\label{linear operations on homotopy of lie alg}
If $L\in s\calL(n)$ has structure map $\rho:F^{\calL(n)}L\to L$, then there are right operations
\[(\DASH)\lambda_i:(\pi_{s_{n+1}}L)_{s_n,\ldots,s_1}^t\to (\pi_{s_{n+1}+i}L)_{2s_n,\ldots,2s_1}^{2t+1}\]
defined whenever $0\leq i\leq s_{n+1}$ and not all of $i,s_n,\ldots,s_1$ equal zero by the composite
\[(\pi_{s_{n+1}}L)_{s_n,\ldots,s_1}^t\overset{\sigma_i}{\to}(\pi_{s_{n+1}+i}(\quadgrad{2}F_{\calL(n)}L))_{2s_n,\ldots,2s_1}^{2t}\overset{\pi_*(\rho)}{\to}(\pi_{s_{n+1}+i}(L))_{2s_n,\ldots,2s_1}^{2t+1}.\]
The operation $\lambda_i$ is linear when $i<s_{n+1}$, while the top operation $\lambda_{s_{n+1}}$ equals the restriction defined in proposition \ref{prop on top operations}. The operations satisfy the Adem relations for the $\Lambda$-algebra. That is, whenever $i>2j$, and $x$ is a homogeneous element of $\pi_*L$ such that $x\lambda_j\lambda_i$ is defined:
\[x\lambda_j\lambda_i=\sum_{k=0}^{(i-2j)/2-1}{i-2j-2-k\choose k}x\lambda_{i-j-1-k}\lambda_{2j+1+k}.\]
\end{prop}
\begin{proof}
As in the previous proposition, we only need to identify the operations defined with those given by Curtis in the case of a (fully) restricted Lie algebra. This is verified when $i\geq1$ using [Dwyer, 4.4], and when $i=0$ using the fact that $\sigma_0:\pi_*L\to \pi_*(S^2L)$ is the map $\overline{x}\mapsto\overline{x\otimes x}$.
\end{proof}
For $n\geq0$, let $\calU(n+1)$ denote the algebraic category of $\vect{+}{n+1}$-graded vector spaces $V$ equipped with only those right $\lambda$-operations in proposition \ref{linear operations on homotopy of lie alg} that are \emph{linear}. That is, linear operations 
\[(\DASH)\lambda_i:V_{s_{n+1},s_n,\ldots,s_1}^t\to V_{s_{n+1}+i,2s_n,\ldots,2s_1}^{2t+1}\]
defined whenever $0\leq i< s_{n+1}$ and not all of $i,s_{n},\ldots,s_{1}$ are zero, satisfying the Adem relations of proposition \ref{linear operations on homotopy of lie alg}. Summarizing classical results to be found in \cite{CurtisSimplicialHtpy.pdf}:

\begin{prop}\label{compatibilities between U and L in W}
For $L\in s\calL(n)$, the homotopy groups $\pi_*(L)$ are simultaneously an object of $\calL(n+1)$ and of $\calU(n+1)$. The $\calU(n+1)$-operations are annihilated by the bracket. That is, for $i<s_{n+1}$, $x\in\pi_{s_{n+1}}L$ and $y\in \pi_*L$ such that $x\lambda_i$ is defined, we have $[x\lambda_i,y]=0$.
\end{prop}
For $n\geq0$, let $\calw(n+1)$ denote the algebraic category of $\vect{+}{n+1}$-graded vector spaces which are simultaneously an object of $\calU(n+1)$ and $\calL(n+1)$, such that the compatibilities of \ref{compatibilities between U and L in W} are satisfied. This category has a number of useful properties, following from the calculations of \cite{6Author.pdf}, primarily:
\begin{prop}\label{prop on Wnplus1 being the pialgs for Wn}
The category $\calw(n+1)$ is isomorphic to the category $\calL(n)\PiAlg$ of $\calL(n)$-$\Pi$-algebras, so that the operations defined above exhaust the set of natural operations on the homotopy of simplicial objects of $\calL(n)$.
The monad $F^{\calw(n+1)}$ on $\vect{+}{n+1}$ factors as a composite $F_{\calU(n+1)}\circ F_{\calL(n+1)}$, with monad structure arising from a distributive law \cite{BeckDistLaws} of monads on $\vect{+}{n+1}$.
\end{prop}
\begin{proof}
All of these facts are easy to prove after observing that, for $W\in s\vect{}{}$ a wedge of spheres, $\pi_*(F^{\calL(n)}W)$ embeds in $\pi_*(F_{\Gamma(\LieOperad)}W)$, which, along with $\pi_*(F_{\Lambda(\LieOperad)}W)$, is well studied \cite{6Author.pdf}. In order to make this observation, let write $W^\textup{z}$ for $\bigoplus_{t\geq1}W_{0,\ldots,0}^t$, the non-restrictable part of $W$. This is actually a sub-wedge of $W$, the wedge of those summands of $W$ which lie in homological dimension $(0,\ldots,0)$. Now there is a commuting diagram of simplicial vector spaces, containing two short exact sequences:
\[\xymatrix@R=-2mm{
0
\ar[r]
&%r1c1
F^{\calL(n)}W
\ar[r]^{\alpha}
&%r1c2
F_{\Gamma(\LieOperad)}W
\ar[dr]^-(.4){\rho\gamma}\ar@{->>}[dd]^-{\gamma}
&%r1c3
%\Fr{\RestLie{n}}W/\Fr{\PRLie{n}}W
%\ar[r]
%\ar[d]^-{\cong}
%&%r1c4
\\%r1c5
&&&\frac{\displaystyle F_{\Gamma(\LieOperad)}W^\textup{z}}{\displaystyle F_{\Lambda(\LieOperad)}W^\textup{z}}
\ar[r]
&
0
\\
0
\ar[r]
&%r1c1
F_{\Lambda(\LieOperad)}W^\textup{z}
\ar[r]^{\beta}
&%r1c2
F_{\Gamma(\LieOperad)}W^\textup{z}
\ar[ru]^-(.4){\rho}&%r1c3
%\Fr{\RestLie{n}}W^\textup{z}/ F_{\Lambda(\LieOperad)}W^\textup{z}
%\ar[r]
%&%r1c4
%0
}\]
On homotopy groups: $\beta_*$ is injective (its source and target are well understood), so that $\rho_*$ is surjective. $\gamma_*$ is surjective (after all, $\gamma$ is an isomorphism in those internal degrees in which its codomain in non-zero), so that $(\rho\gamma)_*$ is surjective. Thus $\alpha_*$ is injective.
\end{proof}
It will be useful in what follows that $\calw(n+1)$ is monadic over $\vect{+}{n+1}$, not just over a category of graded sets.
We define a quadratic grading on the monad $F^{\calw(n+1)}$ by specifying that elements of $V$ have quadratic grading 1, and demanding that quadratic grading adds under taking brackets and doubles when applying $\lambda$-operations.

Moreover, we may define a functor $Q^{\calU(n+1)}:\calw(n+1)\to\vect{+}{n+1}$, by quotienting by the image of the \emph{non-top} $\lambda$-operations (which, fortunately, are linear), and one can prove the following lemma, the direct analogue of lemma \ref{Kill P ops gives lie alg}:
\begin{lem}\label{Kill lambda ops gives lie alg}
\textbf{Make a point of including $\calw(0)$ here.} For $X\in\calw(n+1)$, $X$ is in particular an object of $\calL(n+1)$, and the vector space $Q^{\calU(n+1)}(X)$ retains this structure, yielding a factorization:% of indecomposable functors, $Q^{\calw(n)}=Q^{\calL(n)}\circ Q^{\calU(n)}$
\[Q^{\calw(n+1)}=Q^{\calL(n+1)}\circ Q^{\calU(n+1)}:\left(\calw(n+1)\to \calL(n+1)\to \vect{+}{n+1}\right).\]
Moreover the composite $Q^{\calU(n+1)}\circ F^{\calw(n+1)}$ equals the free construction $F_{\calL(n+1)}$.
\end{lem}
\begin{proof}
Similar to the proof of lemma \ref{Kill P ops gives lie alg}, using the observation from the proof of proposition \ref{prop on Wnplus1 being the pialgs for Wn} that $\pi_*(F^{\calL(n)}W)\subseteq\pi_*(F_{\Gamma(\LieOperad)}W)$.
\end{proof}



\subsection{Decomposition maps for $\calL(n)$ and $\calw(n)$}
\tododone{Describe the composition maps for these categories}{show the compatibility under $\calL(n)\PiAlg\cong \calw(n+1)$
\item show that the $j$ are \emph{quadratic decomposition maps}, if that language is still around}
Here we will introduce decomposition maps for the categories $\calL(n)$ and $\calw(n)$, and prove a certain compatibility between them. The definitions are simple enough, and the reader can verify that each is well defined. Choose $n\geq0$, then we define decomposition maps $\calc:Q(X\smashcoprod Y)\to QX\otimes QY$:
\begin{alignat*}{2}
%\calw(0):&\quad &
j_{\calw(0)}:P^{i_\ell}\cdots P^{i_1}[z_1,\cdots ,z_a]&\longmapsto\begin{cases}
z_1\otimes z_2,&\textup{if }\ell=0,\,a=2,\,z_1\in X,\,z_2\in Y,
\\0,&\textup{otherwise.}
\end{cases}
\\
%\calw(n+1):&&
j_{\calw(n+1)}:[z_1,\cdots ,z_a]\lambda_{i_1}\cdots \lambda_{i_\ell}&\longmapsto\begin{cases}
z_1\otimes z_2,&\textup{if }\ell=0,\,a=2,\,z_1\in X,\,z_2\in Y,
\\0,&\textup{otherwise.}
\end{cases}
\\
%\calL(n):&&
j_{\calL(n)}:\restnRepeated{[z_1,\cdots ,z_a]}{r}&\longmapsto\begin{cases}
z_1\otimes z_2,&\textup{if }r=0,\,a=2,\,z_1\in X,\,z_2\in Y,
\\0,&\textup{otherwise.}
\end{cases}
\end{alignat*}
[or \textbf{uniformize} all this using the phrase $\calU(n)$-operations.]
\begin{prop}
If $V\in\vect{+}{n}$ then $\quadratic_{\calL(n)}$ equals the composite $F^{\calL(n)} V\epi \quadgrad{2}F_{\calL(n)} V\subseteq S^2V$, and $\quadratic_{\calw(n)}$ equals the composite $F^{\calw(n)} V\epi F^{\calL(n)} V\epi \quadgrad{2}F_{\calL(n)} V\subseteq S^2V$.
\end{prop}
\begin{proof}
Consider the case $\calw(n+1)$ for $n\geq1$. As $\quadratic_\calc$ vanishes except on quadratic grading 2, one only checks terms $[x,y]$, $\restn{x}$, and $x\lambda_i$ (a $\calU(n+1)$-operation, not the restriction):
\begin{alignat*}{2}
\quadratic_{\calw(n+1)}([x,y])&=j_{\calw(n+1)}([{x_1}+{{x_2}},{y_1}+{{y_2}}]+[{x_1},{y_1}]+[{{x_2}},{{y_2}}])\\
&=j_{\calw(n+1)}([{x_1},{{y_2}}]+[{{x_2}},{y_1}])=x\otimes y+y\otimes x,
\end{alignat*}
which is precisely the representation of $[x,y]$ in $\quadgrad{2}F_{\calL(n+1)}V\subseteq V^{\otimes2}$. Similarly, $\quadratic_{\calw(n+1)}(x\lambda_i)$ vanishes (as $\calU(n+1)$-operations are linear), while $\quadratic_{\calw(n+1)}(\restn{x})$ equals $x\otimes x$ as desired. The other cases, including the case of $\calw(0)$, are barely any different.
\end{proof}

\subsection{Cohomology operations for simplicial (restricted) Lie algebras}\label{section: Cohomology operations for simplicial (restricted) Lie algebras}
\todo{Explain the four Koszul algebras in these two sections}{Explain why you get the Koszul dualities
\item refer to appendix for proof of the fact about Lie cohomology}

\textbf{Probably move this to the end of the chapter on lie algs... but edit it here for convenience.}

Old \textbf{spiel} from definition of operations on $H^*_{\calw(n)}$:
By Priddy's work in \cite{PriddySimplicialLie.pdf}, Lie algebra cohomology has Steenrod operations and a product. However, the formulae given by Priddy (involving the $\overline{W}$-construction) are not obviously equivalent to what we have given here. However, it is not hard to check that the definition we have given using $\psi_\calw$ is equivalent to an analogous definition for Lie algebra cohomology (given in [placemarker in appendix]), and theorem [BLAH again] identifies these operations with Priddy's.


\end{homotopy operations for PRLs}


\begin{Cohomology Operations for W and U}
\section{\textbf{Operations on $\calw(n)$- and $\calU(n)$-cohomology}}\label{Cohomology Operations for W and U}

%\subsection{Cohomology for objects of $\calw$}
\tododone{Define $Q^\calw:\calw\to \vect{+}{0}$, and thus the cohomology functor on $\calw$}{Give the bar construction $B^{\calw}$
\item Explain significance as Bousfield-Kan $E^2$}
%We will use the standard simplicial bar construction as our simplicial resolution of $X\in\calw$, writing $B^\calw_sX=U^{s+1}X$. Then the cohomology of $X$, $H^{s}_t(X)\in\vect{1}{+}$, is the cohomology of the cochain complex
%\[C^{s}_{t}=\Hom(N_s(Q^\calw B_{\bullet}X)^{t},\Ftwo ).\]
%We will frequently use the identification $Q^\calw B_{s}X=U^{s}X$.


\subsection{Vertical $\delta$ operations on $H^*_{\calw(0)}$ and $H^*_{\calU(0)}$}
\tododone{Define the operations $\gamma^i$, state and define the $\delta_i$ operations}{still need to clarify the end of the proof of the relations.}
For $V\in \vect{+}{0}$, we will define natural homomorphisms
\[\theta_i:(F_{\calw(0)}V)^{t+i+1}\to V^{t}, \textup{ for $2\leq i<t$}.\]
Indeed, there are natural homomorphisms into the quadratic grading 2 part of $F_{\calw(0)}V$:
\begin{align*}
P^i:V^{t}&\to \quadgrad{2}F_{\calw(0)}V^{t+i+1}, \textup{\quad  for $2\leq i<t$}\\
[\,,]:(S_2V)^{t}&\to \quadgrad{2}F_{\calw(0)}V^{t+1},
\end{align*}
and for given $m\geq1$, the degree $m$, quadratic grading 2 part $\quadgrad{2}F_{\calw(0)}V^m$ decomposes as
%\[UV^{m,(2)}=\left(\bigoplus_{n_1+n_2=m-1}\im ([\,,]_{n_1,n_2})\right) \oplus\left(\bigoplus_{2\leq i< (m-1)/2}\im (P^i_{m-i-1})\right)\]
%
\[\quadgrad{2}F_{\calw(0)}V^{m}=%\left(\bigoplus_{n_1+n_2=m-1}
\im \bigl((S_2V)^{m-1}\overset{[,]}{\to} \quadgrad{2}F_{\calw(0)}V\bigr)%\right) 
\oplus\bigoplus_{\!\!\!\!\!\!2\leq i< (m-1)/2\!\!\!\!\!\!}\im \bigl(V^{m-i-1}\overset{P^i}{\to}\quadgrad{2}F_{\calw(0)}V\bigr).\]
Moreover, each map $P^i:V^t\to \quadgrad{2}F_{\calw(0)}V^{t+i+1}$ appearing in this decomposition is an isomorphism onto its image, so that for $2\leq i <t$ we may construct $\theta_i$ as the composite
\[\theta_i:\left((F_{\calw(0)}V)^{t+i+1}\overset{\textup{proj}}{\makebox[.06ex][l]{$\to$}\to} (\quadgrad{2}F_{\calw(0)}V)^{t+i+1}\overset{\textup{proj}}{\makebox[.06ex][l]{$\to$}\to} \im (P^i)\overset{(P^i)^{-1}}{\to}V^{t}\right).\]
Here we have projected onto the quadratic filtration 2 part, and then further onto the relevant summand in its natural decomposition. Note that although $P^t:V^t\to \quadgrad{2}F_{\calw(0)}V^{2t+1}$ is a non-trivial linear map when $t\geq2$, its image is entangled with the image of the bracket, and we are not able to split it off. Thus we are not able to improve on the bounds $2\leq i< t$.
\begin{prop}\label{operations on goerss homology}
Suppose that $X\in s\calw(0)$ is almost free, so that we may identify $H^*_{\calw(0)}X$ with $\dual(\pi_*Q^{\calw(0)}X)$. Then for $2\leq i <t$, the chain map $\widetilde{\theta_i}$ of proposition \ref{general CohOpns given irreducibility} induces a linear operation
\[\delta_i:(H^*_{\calw(0)}X)^{s}_t\to (H^*_{\calw(0)}X)^{s+1}_{t+i+1}.\] 
These operations are natural in maps preserving the generating subspaces, and satisfy the $\delta$-Adem relation of [Dwyer].
\end{prop}
\noindent For any $X\in s\calw(0)$, the bar construction $Q^{\calw(0)}B^{\calw(0)}X$, whose dual homotopy is $H^*_{\calw(0)}X$, has a natural almost free structure, so that proposition \ref{operations on goerss homology} constructs natural  operations on $H^*_{\calw(0)}X$ for all $X\in s\calw(0)$.

Note that \textbf{these are the tricky, unstable operations. Define $\theta_i^\star$.}
%Now suppose that $X\in\calw(0)$, and consider the bar construction $Q^{\calw(0)}B^{\calw(0)}X$ whose cohomotopy is $H^*_{\calw(0)}X$. Until the end of \S\ref{Cohomology Operations for W and U}, we will write $Q\mathbb{B}$ for this simplicial object, write $F$ for $F_{\calw(0)}$, and identify $Q\mathbb{B}_s$ with $F^sX$.
\begin{proof}[Proof of \ref{operations on goerss homology}]
The conditions of proposition \ref{general CohOpns given irreducibility} are satisfied when $\theta=\theta_i$ and $G$ is the identity functor. It just remains to prove the $\delta$-Adem relations, which we will do using the technique of \cite{PriddyKoszul.pdf}, the point being that the algebra of $\delta$-operations is Koszul dual to $\Palg$. For this, we define a map $\theta_{ij}$, whenever $i<2j$, $2\leq j<t$ and $2\leq i<t+j+1$:
\[\theta_{ij}:\left((F_{\calw(0)}V)^{t+i+j+2}\overset{\textup{proj}}{\makebox[.06ex][l]{$\to$}\to} (F_{\calw(0)}^{(4)}V)^{t+i+j+2}\overset{\textup{proj}}{\makebox[.06ex][l]{$\to$}\to} \im (P^{i,j})\overset{(P^{i,j})^{-1}}{\to}V^{t}\right),\]
where we have split off the image of 
$P^{i,j}=P^iP^j$ %:V^n\to (F_{\calw(0)}^{(4)}V)^{n+i+j+2}\]
as before, which is possible, since neither $P^j$ nor $P^i$ are entangled with the bracket in these ranges. We may identify $Q^{\calw(0)}X_s$ with $V_s$, at the cost of replacing $d_0$ with $\epsilon\circ d_0$, as in lemma \ref{identify almost free indecs with gens}.  Define $\widetilde{\theta_{ij}}$ to be the composite $V_{s+1}\overset{d_0}{\to}FV_s\overset{\theta_{ij}}{\to}V_s$. This will be the nullhomotopy giving the $\delta$-Adem relation. %It'll feel easier to think of it as a nullhomotopy of the un-normalized complex. 
As in the proof of proposition \ref{general CohOpns given irreducibility}, we have $d_k\circ\widetilde{\theta_{ij}}=\widetilde{\theta_{ij}}\circ d_{k+1}$ for $k\geq1$, so its nullhomotopy is the sum
\[\epsilon d_0\widetilde{\theta_{ij}}+\widetilde{\theta_{ij}}(\epsilon d_0+d_1)=(\epsilon d_0\theta_{ij}+\theta_{ij}d_0\epsilon+\theta_{ij}d_0)d_0,\]
using the simplicial identity $d_0d_1=d_0d_0$, and the $\delta$-Adem relation will follow from
\[\theta_{ij}d_0=\Bigl(\epsilon d_0\theta_{ij}+\theta_{ij}d_0\epsilon+\sum\theta_\beta d_0\theta_\alpha\Bigr):FV_{s+1}\to V_s,\]
where the sum is taken over all the combinations of $\alpha,\beta\geq2$ such that $P^iP^j$ appears when $P^\alpha P^\beta$ is written in the $P$-admissible basis. This identity states the following: if $V\in\vect{+}{0}$, and $f(g_k)$ is a nested $\calw(0)$-expression with $g_k\in F_{\calw(0)}V$ and $f(g_k)\in F_{\calw(0)}F_{\calw(0)}V$, then if we write $d_0:F_{\calw(0)}F_{\calw(0)}V\to F_{\calw(0)}V$ for the monad product map, there are only three ways that one may obtain expressions of the form $P^iP^jv$ in $d_0(f(g_k))$: for some $k$, $g_k=P^iP^jv$, and $f $ adds no further operations to this term; $f=P^iP^jg_{k}$ for some $k$ for which $g_{k}=v$ is a unit expression; or for some $k$, $g_k=P^\beta v$, and  $f$ has $P^\alpha(g_k)$ as a summand. In this last case, after applying $d_0$, the composite $P^{\alpha}P^{\beta}v$ may need to be rearranged using the $P$-Adem relation, and we sum over those $\alpha$,$\beta$ producing a summand $P^iP^jv$.

This shows that the nullhomotopy proposed equals the sum of the maps $\widetilde{\theta_\beta}\circ\widetilde{\theta_\alpha}$, and as Goerss \cite{MR1089001} \emph{constructs} the $P$-algebra as the Koszul dual, in the sense of \cite{PriddyKoszul.pdf}, of the algebra of $\delta$-operations, $P^{\alpha }P^{\beta}$ produces a summand $P^iP^j$ if and only if $\delta_i\delta_j$ produces a summand $\delta_{\alpha }\delta_{\beta}$. This proves the result.
\end{proof}
The same constructions work in the category $s\calU(0)$ of simplicial unstable \emph{modules}, the only difference being that when we define $\theta_i$, we need not worry about brackets, and we can define a map
\[\theta_i:F_{\calU(0)}V^{t+i+1}\to V^t\]
whenever $2\leq i\leq t$, so that there is one more operation available on $H^*_{\calU(0)}$ than on $H^*_{\calw(0)}$. It will be useful to encode this structure in a definition. Write $\calMv(1)$ for the algebraic category whose objects are vector spaces $M\in\vect{1}{+}$ with left $\delta$-operations
\[\delta_i:M^{s}_t\to M^{s+1}_{t+i+1},\textup{ defined whenever $2\leq i\leq t$,}\]
satisfying the \textbf{tricky} $\delta$-Adem relations of [Dwyer].
\begin{prop}\label{operations on untable P homology}
Suppose that $X\in s\calU(0)$ is almost free. Then the chain maps $\widetilde{\theta_i}$ of proposition \ref{general CohOpns given irreducibility} give $H^*_{\calU(0)}X$ the structure of an object of $\calMv(1)$, natural in maps preserving the generating subspaces.
\end{prop}
In order to give a basis for a free object in $\calMv(1)$, for a sequence $I=(i_\ell,\ldots,i_1)$ of integers $i_j\geq2$, we define
\[\minDimP(I):=\max\{(i_1),\,(i_2-i_1-1),\,(i_3-i_2-i_1-2),\,\ldots,\,(i_{\ell}-\cdots-i_1-\ell+1)\},
\]
following the convention that $\max(\emptyset)=-\infty$. Moreover, say that the sequence is \emph{$\delta$-admissible} if each $i_j\geq2$ and  $i_{j+1}\geq 2i_j$ for $1\leq j <\ell$.
\begin{lem}\label{basis of element of M(0)}
%For $V\in\vect{1}{+}$ with homogeneous basis $B$, a basis of $F_{\calMv(1)}V$ consists of the classes $\delta_Ib$ where $b\in B\cap V^s_t$ is a basis element and $I$ is a $\delta$-admissible sequence with $t\geq\minDimP(I)$.
For $V\in\vect{1}{+}$ with homogeneous basis $B$, a basis of $F_{\calMv(1)}V$ consists of
\[\left\{\delta_Ib\ \middle|\ b\in B^{s}_t\textup{, $I$ $\delta$-admissible with }\minDimP(I)\leq t\right\}.\]
\textbf{extend to case where $V$ concentrated in Koszul deg 0}\end{lem}
\begin{shaded}
Note that the forgetful functor $\calw(0)\to\calU(0)$ does not preserve almost free simplicial objects! However, it does preserve levelwise free simplicial objects, which create the right homology groups (see PRLieAlgs.tex). Thus there is a disconnect in the number of available operations on the Bousfield-Kan $E^2$ and on the first Koszul resolution.
\end{shaded}

\subsection{Vertical Steenrod operations for $H^*_{\calw(n)}$ and $H^*_{\calU(n)}$ when $n\geq1$}\label{section: vertical Koszul operations n positive}
For $V\in \vect{+}{n}$, we will define natural homomorphisms
\[\theta^i:(F^{\calw(n)}V)^{2t+1}_{s_n+i-1,2s_{n-1},\ldots,2s_1}\to V^{t}_{s_n,\ldots,s_1},\]
which are defined for all $i,s_1,\ldots,s_n\geq0$ and $t\geq1$, but are zero except when $1\leq i \leq s_n$ and not all of $i-1,s_{n-1},\ldots,s_1$ are zero.
These are rather easier to define than in the $n=0$ case above, as the monad $F^{\calw(n)}$ is a simple composite $F_{\calu(n)}F_{\call(n)}$ of monads.
Indeed, there are natural monomorphisms
\[(\DASH)\lambda_{i-1}:V^{t}_{s_n,\ldots,s_1}\to (\quadgrad{2}F_{\calw(n)}V)^{2t+1}_{s_n+i-1,2s_{n-1},\ldots,2s_1}\]
defined only when   $1\leq i\leq n$ and $i-1,s_{n-1},\ldots,s_1$ are not all zero, and an inclusion
\[\textup{incl}:\quadgrad{2}F_{\calL(n)}V\to \quadgrad{2}F_{\calw(n)}V.\]
As in the $n=0$ case, the images of the listed maps are linearly independent and span the quadratic grading 2 part of $F^{\calw(n)}V$. We define $\theta^i$ to be zero unless $1\leq i\leq n$ and $i-1,s_{n-1},\ldots,s_1$ are not all zero, in which case we define it as the composite: %project onto the quadratic grading 2 part, then further onto the summand corresponding to the image of $\lambda_{i-1}$, and finally use the inverse of the map $\lambda_{i-1}$:
\[\theta^i:\Bigl((F^{\calw(n)}V)^{2t+1}_{s_n+i-1,2s_{n-1},\ldots,2s_1}\overset{\textup{proj}\circ\textup{proj}}{\makebox[.06ex][l]{$\to$}\to} \im (\lambda_{i-1})\overset{(\lambda_{i-1})^{-1}}{\to}V^{t}_{s_{n,\ldots,s_1}}\Bigr).\]
One can give exactly the same definitions for the free construction in $\calU(n)$, producing functions $\theta^i:F_{\calU(n)}V\to V$ which are zero under the same conditions as for $\calw(n)$.
Write $\calMv(n+1)$ for the algebraic category whose objects are vector spaces $M\in\vect{n+1}{+}$ with left Steenrod operations
\[\Sqv^i:M^{s_{n+1},\ldots,s_1}_t\to M^{s_{n+1}+1,s_n+i-1,2s_{n-1},\ldots,2s_1}_{2t+1},\]
which are zero except when $1\leq i \leq s_n$ and not all of $i-1,s_{n-1},\ldots,s_1$ are zero, and which
satisfy the $\Sq$-Adem relations of [wherever-Priddy?].

In the present case, $n\geq1$, there is no disparity between the unstableness conditions on $\calw(n)$- and $\calu(n)$-cohomology, so that we can combine the analogues of \ref{operations on goerss homology} and \ref{operations on untable P homology} into:
\begin{prop}\label{vertical steenrod operations prop}
Suppose that $X\in s\calc$ is almost free, where $\calc$ stands for either $\calw(n)$ or $\calU(n)$ with $n\geq1$. Then the chain maps $\smash{\widetilde{\theta^i}}$ of proposition \ref{general CohOpns given irreducibility} give $H^*_{\calc}X$ the structure of an object of $\calMv(n+1)$, natural in maps preserving the generating subspaces.
\end{prop}
\textbf{give some discussion. Point out that the nullhomotopy is unstable.}

In order to give a basis for a free object in $\calMv(n+1)$, for a sequence $I=(i_\ell,\ldots,i_1)$ of integers $i_j\geq1$, we define
\[\minDimSq(I):=\max\{(i_1),\,(i_2-i_1+1),\,(i_3-i_2-i_1+2),\,\ldots,\,(i_{\ell}-\cdots-i_1+(\ell-1))\},
\]
and say that $I$ is \emph{$\Sq$-admissible} if each $i_j\geq1$ and $i_{j+1}\geq 2i_j$ for $1\leq j <\ell$ \textbf{(uniformize this convention)}.
\begin{lem}
For $V\in\vect{n+1}{+}$ with homogeneous basis $B$, a basis of $F_{\calMv(n+1)}V$ consists of
\[\left\{\Sq^J_{\textup{v}}b\ \middle|\ \genfrac{}{}{0pt}{}{b\in B^{s_{n+1},\ldots,s_1}_t\textup{, $J$ $\Sq$-admissible with }\minDimSq(J)\leq s_n,}{\textup{if }s_{n-1}\!=\!\cdots\!=\!s_1\!=\!0\textup{ then $J$ doesn't contain 1}}\right\}.\]


\end{lem}

\subsection{Horizontal Steenrod operations and a product for $H^*_{\calw(n)}$}\label{Horizontal Steenrod operations and a product for HWn}
For any $n\geq 0$, we will construct operations on the homology $H^*_{\calw(n)}$ arising from the Lie structure.

Indeed, suppose that $X\in s\calw(n)$ is almost free. Then $Q^{\calU(n)}X\in s\calL(n)$ is also almost free, on essentially the same generating subspaces. Thus, the cohomotopy of $Q^{\calw(n)}X=Q^{\calL(n)}Q^{\calU(n)}X$ is an instance of simplicial partially restricted Lie algebra cohomology. Cohomology operations of this type are discussed in \S\ref{section: Cohomology operations for simplicial (restricted) Lie algebras} and appendix \ref{appendix on Lie coh ops}. In the present context, we have two equivalent definitions, one using $\psi_{\calL(n)}$ and one using $\psi_{\calw(n)}$. Until the appendix, we will use $\psi_{\calw(n)}$, defining operations
\begin{gather*}
\mu:\left(H_{\calw(n)}^{n_1}(X)\otimes H_{\calw(n)}^{n_2}(X)\overset{\ExtCohProd}{\to} \pi^{n_1+n_2}(\dual(S^2(QX)))\overset{\psi_{\calw(n)}^*}{\to} H_{\calw(n)}^{n_1+n_2+1}(X)\right)\textup{ and}\\
\Sqh^k:\left(H_{\calw(n)}^{n}(X)\overset{\ExtCohOp^{k-1}}{\to} \pi^{n+k-1}(\dual(S^2(QX)))\overset{\psi_{\calw(n)}^*}{\to} H_{\calw(n)}^{n+k}(X)\right).
\end{gather*}
More explicitly:
\begin{prop}
For $X$ an object of $\calw(n)$, $H^*_{\calw(n)}X$ possesses a left action of the Lie Steenrod algebra, consisting of natural homomorphisms
\[\Sqh^j:(H^*_{\calw(n)}X)_t^{s_{n+1},\ldots,s_1}\to (H^*_{\calw(n)}X)_{2t+1}^{s_{n+1}+j,2s_{n},\ldots,2s_1},\]
zero unless $1\leq j\leq s_{n+1}+1$ and not all of $j-1,s_{n},\ldots,s_1$ equal zero (so that if $n=0$, $\Sqv^1$ is necessarily zero). %If $n=0$, $\Sqh^1$ and $\Sqh^2$ are both zero 
[\textbf{$\Sqh^2=0$ when $n=0$?}] Moreover, $H^*_{\calw(n)}X$ supports a non-unital commutative algebra pairing
\[(H^*_{\calw(n)}X)_t^{s_{n+1},\ldots,s_1}\otimes (H^*_{\calw(n)}X)_q^{p_{n+1},\ldots,p_1}\to (H^*_{\calw(n)}X)_{t+q+1}^{s_{n+1}+p_{n+1}+1,s_{n}+p_{n},\ldots,s_1+p_1}.\]
These satisfy the unstableness condition
\[x^2=\Sqh^{s_{n+1}+1}x\text{ for }x\in (H^*_{\calw(n)}X)_t^{s_{n+1},\ldots,s_1}\]
and the Cartan formula
\[\Sqh^k(xy)=\textstyle\sum_{i=0}^k(\Sqh^ix)(\Sqh^{k-i}y).\]
\end{prop}
\begin{proof}
This theorem follows from theorem [BLAH] in the appendix.
\end{proof}
For $n\geq0$, write $\calMh(n+1)$ for the algebraic category whose objects are vector spaces $M\in\vect{n+1}{+}$ with left Steenrod operations and a commutative pairing satisfying the conditions of the proposition. We have simply shown that $H^*_{\calw(n)}$ takes values in $\calMv(n+1)$.

Note that the unstableness condition implies that if $x\in M_t^{0,\ldots,0}$, then $x^2=0$. Indeed
\begin{prop}\label{basis of free horizontal operations algebra}
Suppose that $n\geq1$. For $V\in\vect{n+1}{+}$ with homogeneous basis $B$. Then $F_{\calMh(n+1)}V$ is the quotient of the algebra
\[\Ftwo \left[\Sq^J_{\textup{v}}b\ \middle|\ \genfrac{}{}{0pt}{}{b\in B^{s_{n+1},\ldots,s_1}_t\textup{, $J$ $\Sq$-admissible with }\excess(J)\leq s_{n+1},}{\textup{if }s_{n}=\cdots=s_1=0\textup{ then $J$ does not contain 1}}\right]\]
by the relation $b^2=0$ if $b\in B_t^{0,\ldots,0}$. Here, $e(J):=j_\ell-j_{\ell-1}-\cdots j_1$ is the Serre excess of $J$. \textbf{{check this!}}
\end{prop}
\begin{proof}
By \cite[6.1]{PriddySimplicialLie.pdf}, the true free object is at worst a quotient of what we propose. It is in fact equal to what we propose, because the two-sided ideal in $\LieSteen$ generated by $\Sqh^1$ is spanned by those admissible sequences ending in $\Sqh^1$, so that forcing $\Sq^1_h=0$ in the relevant degrees has no unintended consequences. Another way to express this fact is to say that left multiplication by $\Sq^1_h$ annihilates the augmentation ideal in $\LieSteen$ (the Steenrod algebra for Lie algebras).
\end{proof}

\subsection{Relations between the horizontal and vertical operations}
%With such a broad range of available operations, 
It will be helpful if we are able to reduce %We are interested in calculating the composite $\delta_i\Sqh^jx$, or $\delta_i(xy)$, for $x,y\in H^*_{\calw(n)}X$ \textbf{but only for $n=0$}, so that 
expressions in the the various available operations in a standard format, namely:
\[\prod_i \Sqh^{J_{i}}\deltav_{I_{i}}x_i\textup{ when $n=0$, or }\prod_i \Sqh^{J_{i}}\Sqv^{I_{i}}x_i\textup{ when $n\geq1$}.\]
This is possible, thanks to:
\begin{prop}
Suppose that $x,y\in H^*_{\smash{\calw(0)}}X$. If $\Sqh^jx\in (H^*_{\smash{\calw(0)}}X)^{s}_{t}$, then $\deltav_i\Sqh^{j}x=0$ for  $2\leq i<t$, and if $xy\in (H^*_{\smash{\calw(0)}}X)^{s}_{t}$, then $\deltav_i(xy)=0$ for  $2\leq i<t$.

Suppose that $x,y\in H^*_{\smash{\calw(n)}}X$ for $n\geq1$. Then $\Sqv^{i}\Sqh^{j}x=0$ and $\Sqv^i(xy)=0$.
%The expressions $\deltav_i\Sqh^jx$ and $\deltav_i(xy)$ for $x,y\in H^*_{\calw(0)}X$ are zero whenever defined. 
\end{prop}
\begin{proof}
For the case $n=0$, suppose that $X\in s\calw(0)$ is almost free on generating subspaces $V_s$. It is enough to prove that the composite
\[N_{s+1}(Q^{\calw(0)}X_{s+2})^{t+i+1}
\overset{\smash{\widetilde{\theta_i}}}{\to}
N_s(Q^{\calw(0)}X_{s+1})^{t}
\overset{{\psi_{\calw(n)}}}{\to}
N_{s-1}(S^2(Q^{\calw(0)}X_{s}))^{t-1}
\]
is nullhomotopic, by a similar method to that used in the proof of \ref{operations on goerss homology}. For any $V\in \vect{+}{0}$, there is a natural composite
\[(S_2V)^{t-1}\underset{\smash{\alpha}}{\overset{[\,,\,]}{\to}} (\quadgrad{2}F_{\calw(0)}V)^{t}\underset{\smash{\beta}}{\overset{P^i}{\mono}} (F^{(4)}_{\calw(0)}V)^{t+i+1},\]
whose maps we have labeled $\alpha$ and $\beta$ for convenience.
The map $\beta$ is not a monomorphism on $\im([\,,])$ when $i=t-1$ is even, as in this case, for any $v\in V^{i/2}$,
\[P^i[v,v]=P^{i}P^{i/2}v=\textstyle\sum_{k=i/2+1}^{3i/2-2}{2(k-i/2)-1\choose k-i/2}P^{3i/2-k}P^kv=0,\]
by the Adem relation and the unstableness condition. However, the $\ker(\beta)$ is contained in $\ker(\quadratic_{\calw(0)})$, 
and $\im(\beta\circ\alpha)$ does naturally split off as a direct summand of $(F^{(4)}_{\calw(0)}V)^{t+i+1}$.
 We write $h_i$ for the composite:
\[h_{i}:\Bigl((F_{\calw(0)}V)^{t+i+1}\overset{\textup{proj}}{\epi}
(\im(\beta\circ\alpha))^{t+i+1}\overset{\beta^{-1}}{\to}\frac{\im(\alpha)}{\ker(\beta)\cap\im(\alpha)}\overset{\quadratic_{\calw(0)}}{\to}(S^2V)^{t-1}\Bigr).\]
Identifying $Q^{\calw(0)}X_s$ with $V_s$ as in the proof of \ref{operations on goerss homology}, the nullhomotopy associated with the composite $\widetilde{h_i}:(V_{s+1}\overset{d_0}{\to}FV_s\overset{h_i}{\to}V_s)$ is the sum
\[(\epsilon d_0h_i+h_id_0\epsilon+h_id_0)d_0,\]
and the relation we seek will follow from the identity
\[h_id_0=\Bigl(\epsilon d_0h_i+h_id_0\epsilon+\quadratic_{\calw(0)} d_0\theta_i\Bigr):FV_{s+1}\to V_s,\]
as then $\psi_{\calw(0)}\smash{\widetilde{\theta}_i}=\quadratic_{\calw(0)} d_0\theta_id_0=d_0\widetilde{h_i}+\widetilde{h_i}d_0$. This identity states the following: if $V\in\vect{+}{0}$, and $f(g_k)$ is a nested $\calw(0)$-expression with $g_k\in F_{\calw(0)}V$ and $f(g_k)\in F_{\calw(0)}F_{\calw(0)}V$, then if we write $d_0:F_{\calw(0)}F_{\calw(0)}V\to F_{\calw(0)}V$ for the monad product map, there are only three ways that one may obtain summands of the form $P^i[v_1,v_2]$ in $d_0(f(g_k))\in (F_{\calw(0)}V)^{t+i+1}$: for some $k$, $g_k=P^i[v_1,v_2]$, and $f $ adds no further operations to this term; $f=P^i[g_{k_1},g_{k_2}]$, where $g_{k_1}=v_1$ and $g_{k_2}=v_2$ are unit expressions; or for some $k$, $g_k=[v_1,v_2]$, and  $f$ has $P^i(g_k)$ as a summand.

For the case $n\geq0$, the proof only becomes easier, the main difference being that in the corresponding composite:
\[(\quadgrad{2}F_{\call(n)})^{t-1}\underset{\smash{\alpha}}{{\to}} (\quadgrad{2}F_{\calw(0)}V)^{t}\underset{\smash{\beta}}{\overset{\lambda_{i-1}}{\mono}} (F^{(4)}_{\calw(0)}V)^{t+i-1},\]
both $\alpha$ and $\beta|_{\im(\alpha)}$ are monomorphisms.
\end{proof}

\subsection{Compressing sequences of Steenrod operations}
\begin{thm}\label{thm on compressing seqs of steenrod ops}
Suppose that $n\geq1$ and $V\in \vect{n}{+}$. Then there is a decreasing filtration on $F_{\calMh(n)}V$, the `target filtration', and an isomorphism
\[ f:(F_{\calMh(n+1)}F_{\calMv(n+1)}V)^{s_{n+1},\ldots,s_1}_t\overset{\cong}{\to} E_0^{s_{n+1}}(F_{\calMh(n)}V)^{s_n,\ldots,s_1}_t,\]
defined by requiring that
$f(\Sqv^Iv)=\Sqh^Iv$ for $v\in V$, that $f(w_1w_2)=f(w_1)f(w_1)$ for $w_1,w_2\in F_{\calMh(n+1)}F_{\calMv(n+1)}V$,
and that
\[f(\Sqh^jw)=\Sqh^{j+s_n}f(w)\textup{ for }w\in (F_{\calMh(n+1)}F_{\calMv(n+1)}V)^{s_{n+1},\ldots,s_1}_t.\]
\end{thm}
\begin{proof}
The map $f$ is not a well defined map to $F_{\calMh(n)}V$ since the Adem relations between the $\Sqh$ are not preserved by the proposed map $f$. Write $W(V)$ for the quotient of $\Ftwo [\LieSteen\otimes F_{\calMv(n+1)}V]$ by the unstableness relations and Cartan formula, so that $F_{\calMh(n+1)}F_{\calMv(n+1)}V$ is obtained from $W(V)$ by taking the quotient by the two-sided ideal generated by the (horizontal) Adem relations. Then may define a map $\overline{f}:W(V)\to F_{\calMh(n)}V$ by requiring the same of $\overline{f}$ as of $f$. There is a decreasing filtration on $W(V)$, given by 
\[F^pW(V)=\bigoplus_{s_{n+1}\geq p}\bigoplus_{s_n,\ldots,s_1\geq0}\bigoplus_{t\geq1}W(V)^{s_{n+1},\ldots,s_1}_t\]
 and we define the \emph{target filtration} on the target by $F^p(F_{\calMh(n)}V):=\overline{f}(F^pW(V))$.

The map $\overline{f}$ fails to descend to a well-defined map  $F_{\calMh(n+1)}F_{\calMv(n+1)}V\to F_{\calMh(n)}V$, because it does not annihilate the Adem relations. However, we will show that it does send them into higher filtration, so that $\overline{f}$ induces a well defined map $f$ as advertised: if $w\in W(V)^{s_{n+1},\ldots,s_1}_t$ and $i<2j$, then
%\[f(\Sqh^i\Sqh^jw):=\Sqh^{i+2s_n}\Sqh^{j+s_n}f(w),\]
%and such an assignment does not preserve the Adem relations. Indeed, the Adem relation arising when $i<2j$:
%\[\Sqh^i\Sqh^jw-\sum_{k=0}^{\lfloor i/2\rfloor}{j-k-1\choose i-2k}\Sqh^{i+j-k}\Sqh^{k}w\]
%is sent to
%%\[\Sqh^{i+2s_n}\Sqh^{j+s_n}f(w)-\sum_{k=0}^{\lfloor i/2\rfloor}{j-k-1\choose i-2k}\Sqh^{i+j-k+2s_n}\Sqh^{k+s_n}f(w)\]
\begin{alignat*}{2}
\overline{f}\Bigl(\Sqh^i&\Sqh^jw-\textstyle\sum_{k=0}^{\lfloor i/2\rfloor}{j-k-1\choose i-2k}\Sqh^{i+j-k}\Sqh^{k}w\Bigr)\\
{}:={}&\Sqh^{i+2s_n}\Sqh^{j+s_n}\overline{f}(w)-\textstyle\sum_{k=0}^{\lfloor i/2\rfloor}{j-k-1\choose i-2k}\Sqh^{i+j-k+2s_n}\Sqh^{k+s_n}\overline{f}(w)\\
{}={}&
\Sqh^{i+2s_n}\Sqh^{j+s_n}\overline{f}(w)-\textstyle\sum_{k=s_n}^{\lfloor (i+2s_n)/2\rfloor}{j+s_n-k-1\choose i+2s_n-2k}\Sqh^{(i+2s_n)+(j+s_n)-k}\Sqh^{k}\overline{f}(w)\\
{}={}&\textstyle\sum_{k=0}^{s_n-1}{j+s_n-k-1\choose i+2s_n-2k}\Sqh^{(i+2s_n)+(j+s_n)-k}\Sqh^{k}\overline{f}(w)\\
{}=\makebox[0cm][r]{$:$}{}&\textstyle\sum_{k=0}^{s_n-1}{j+s_n-k-1\choose i+2s_n-2k}\overline{f}(\Sqh^{i+j+2(s_n-k)+1}\Sqv^{k}w),
\end{alignat*}
which is in filtration $s_{n+1}+i+j+2(s_n+1-k)>s_{n+1}+i+j$ (the second equality holds by simply shifting the dummy variable $k$, the third by an Adem relation in the codomain).
%This quantity is non-zero, but is written as the image under $f$ of an element of filtration 

What remains is to show that $f$ is an isomorphism as in the theorem statement, which we approach simply by choosing a set of multiplicative generators for both the domain and codomain. The domain is generated by those expressions $\Sqh^I\Sqv^Jv$, for $v\in V^{s_{n},\ldots,s_1}_t$ running through a basis of $V$, and appropriate $\Sq$-admissible sequences $J$ and $I$. The codomain is generated by expressions $\Sqh^Kv$, for $v\in V^{s_{n},\ldots,s_1}_t$ running through a basis of $V$, and appropriate $\Sq$-admissible sequences $K$. It is a  combinatorial  exercise in the properties of admissible sequences to show that these sets of generators are put in bijection by $f$, and this bijection sends polynomial generators to polynomial generators and exterior generators to exterior generators.
\end{proof}
\end{Cohomology Operations for W and U}



\begin{second quadrant homotopy}

\section{\textbf{Operations on second quadrant homotopy spectral sequences}}\label{second quadrant homotopy}


Mention the work of Turner; of Hackney [1101.3798v3], using universal examples as in Bousfields'  original stuff.


\subsection{Dwyer's work on the dual question}
This is Dwyer's work on second quadrant cohomology spectral sequences.

Explain Dwyer's work for comparison; do some general throatclearing: 
\begin{shaded}\tiny
We produce, for $X$ a cosimplicial simplicial vector space, an external commutative pairing
\[\times^\textup{ext}:(E^rX)^s_{t}\otimes (E^rX)_{t'}^{s'}\to (E^r(S_2X))^{s+s'}_{t+t'}\]
and further operations:
\[\delta_i^\textup{ext}:(E^rX)^s_{t}\to (E^r(S_2X))^{s}_{t+i}\textup{\qquad (defined only when $2\leq i<t-(r-2)$)}\]
\[\Sq^j_\textup{ext}:(E^rX)^s_{t}\to (E^r(S_2X))^{s+j}_{2t}\textup{\qquad (defined up to indeterminacy)}\]
Of course, if $X$ is a mixed simplicial algebra, then postcomposition with the product $X\otimes_{\Sigma_2}X\to X$ produces internal operations on the spectral sequence of $[E^rX]$. The resulting theory lives opposite Bill Dwyer's theory [?] of operations in the spectral sequence of a cosimplicial simplicial coalgebra. The two theories are not equivalent --- although linear dualisation interchanges (finite type) algebras and coalgebras, the symmetry is broken by the choice of filtration direction. The operations $\times^\textup{raw}$, $\delta_i^\textup{raw}$ and $\Sq^j_\textup{raw}$ may be of independent interest, and we will summarise their theory without proofs in theorem XYZ.

In the present context, however, the operations of theorem XYZ are all equal to zero on $E^2$, simply because the Bousfield-Kan spectral sequence is derived from a cosimplicial object which is levelwise an Eilenberg-Mac Lane object. However, we will be able to use the external operations in section XYZ, shifting them one filtration higher in order to obtain non-trivial operations at on the Bousfield-Kan $E^2$.


\textbf{similar para:} Let $X$ be a mixed simplicial commutative non-unital $\Ftwo $-algebra. I'd prefer to construct spectral sequence operations externally, as maps $E^r_{-s,t}(X)\to E^r_{?,?}(X\otimes_{\Sigma_2}X)$, which may sometimes be multivalued. I'll also assume that $X$ admits a coaugmentation $X^{-1}_\bullet\to X^\bullet_\bullet$ from a simplicial algebra $X^{-1}_\bullet$, and that this coaugmentation induces an isomorphism on total homology.
%
%\subsection{Higher cosimplicial Eilenberg-Zilber map}
%Let $\{D^k\}$ be a special cosimplicial Eilenberg-Zilber map \cite[5.2]{turner_opns_and_sseqs_I.pdf}, i.e.\  maps
%\[D^k:(CR\otimes CS)_{-i-k}\to N(R\otimes S)_{-i}\textup{\ \ for $i,k\geq0$},\]
%natural in cosimplicial vector spaces $R,S$,
%with the properties:
%\begin{itemize}
%\setlength{\parindent}{.25in}
%\item $dD^k+D^kd=D^{k-1}+TD^{k-1}T$ for $k\geq1$;
%\item $D^0$ is a chain homotopy equivalence inducing the identity in dimension zero;
%\item the restriction of $D^k$ to $C_{-i}R\otimes C_{-j}S$ is zero unless $i\geq k$ and $j\geq k$; and
%\item $D^k$ maps $C_{-k}R\otimes C_{-k}S$ identically onto $C_{-k}(R\otimes S)$.
%\end{itemize}
%It is a natural convention to define $D^k=0$ for whenever either $k<0$ or $i<0$, in which case the relation $dD^k+D^kd=D^{k-1}+TD^{k-1}T$ holds for any $k$.
%\subsection{Higher simplicial Eilenberg-Mac Lane map}
%Let $\{\Nabla_k\}$ be a higher simplicial Eilenberg-Mac Lane map \cite[\S3]{DwyerHtpyOpsSimpComAlg.pdf}, i.e.\ maps
%\[\Nabla_k:(CU\otimes CV)_{i+k}\to N(U\otimes V)_i\textup{\ \ for $0\leq k\leq i$}\]
%such that the identities
%\[\Nabla_k+T\Nabla_kT=\phi_k+\begin{cases}
%\Nabla_{k-1}\partial+\partial\Nabla_{k-1},&\textup{if }k\geq1;\\
%\Nabla,&\textup{if }k=0.
%%\\,&\textup{if }
%\end{cases}
%\]
%hold on classes of dimension at least $2k$, and:
%\begin{itemize}
%\setlength{\parindent}{.25in}
%\item $\Nabla:CU\otimes CV\to N(U\times V)$ is a chain homotopy equivalence inducing the identity in dimension zero; and
%\item $\phi_k$ is the map $(CU\otimes CV)_{i+k}\to N(U\otimes V)_i$ which vanishes except on $U_k\otimes V_k$, where its value is just the projection $U_k\otimes V_k\to N(U\times V)_k$.
%\end{itemize}
%Note that $\phi$ commutes with symmetry isomorphisms, and thus so must $\Nabla$.

\end{shaded}

\subsection{Maps of mixed simplicial vector spaces}
[\textbf{Introduce} an un-decorated $C$ for the double complex, not the total complex.]
For mixed simplicial vector spaces $X$ and $Y$, write $C(X\otimes_\textup{v}Y)$ for the double complex which in degree $(\fraks,\frakt)$ equals the direct sum $\bigoplus_{t+t'=\frakt}X_t^\fraks\otimes X_{t'}^\fraks$. The following vector space maps are given by prolonging $D^k$, $\Nabla$, $\Nabla_k$ and $\phi_k$, wherever these maps are defined, and by zero elsewhere:
\begin{align*}
{D}^k:(CX\otimes CY)^{\fraks+k}_{\frakt}&\to C(X\otimes_\textup{v}Y)^{\fraks}_{\frakt}&\quad&\textup{(zero unless $0\leq k\leq \fraks$)}\\
{\Nabla}:C(X\otimes_\textup{v} Y)^{\fraks}_{\frakt}&\to C(X\otimes Y)^{\fraks}_{\frakt}\\
{\Nabla}_k:C(X\otimes_\textup{v} Y)^{\fraks}_{\frakt+k}&\to C(X\otimes Y)^{\fraks}_{\frakt}&\quad&\textup{(zero unless $0\leq k\leq \frakt$)}\\
{\phi}_k:C(X\otimes_\textup{v} Y)^{\fraks}_{\frakt+k}&\to C(X\otimes Y)^{\fraks}_{\frakt}&\quad&\textup{(zero unless $k= \frakt\geq0$)}\\
{\rho}:C(X\otimes X)^{\fraks}_{\frakt}&\to C(S_2X)^{\fraks}_{\frakt}&\quad&\textup{(proj onto coinvariants)}
\end{align*}
We have essentially just committed to regarding $\Nabla_k$ as zero where it is not defined. This is certainly not the natural convention, and it results in such unpleasant statements as:
\begin{lem}\label{unpleasant formula}
Suppose that $z\in C(X\otimes_\textup{v} Y)^{\fraks}_{\frakt}$. Then
\[(d\Nabla_k+\Nabla_kd)z=((1+\twist)\Nabla_{k+1}+\phi_{k+1} )z\]
whenever $k\geq0$ and $\frakt$ does not equal either of $2k$ and $2k+1$.
\end{lem}
\noindent We hope that the ends justify the means.

As discussed earlier, we will write $T$ for any symmetry isomorphism, write ``$\twist F$'' as shorthand for the function $TFT$, and  whenever we write $\twist FG$, we will mean $(\twist F)G$.

\subsection{External products}
We will define the external product
\[\times^\textup{ext}:\E{r}{}{X}{s}{t}\otimes \E{r}{}{X}{s'}{t'}\to \E{r}{}{S_2X}{s+s'}{t+t'}\]
using the chain-level formula
\[x\otimes y\mapsto\rho\Nabla D^0(x\otimes y).\]
As both $\Nabla$ and $D^0$ are chain maps preserving filtration, the operations $\times^\textup{ext}$ satisfy a Liebnitz formula.
\begin{prop}
Suppose that $x\in \E{r}{}{X}{s}{t}$ is a permanent cycle such that $t-s>0$. Then $x\times x$ is eventually hit by a differential. That is, $x\times x\in \E{r}{}{S_2X}{2s}{2t}$ is a permanent cycle which represents zero on $\E{\infty}{}{S_2X}{}{}$.
\end{prop}
\begin{proof}
\textbf{Since the spectral sequence converges}, $x$ can be written as $x^0+dy$ for $x^0\in X^{0}_{n}$ such that $dx^0=0$ and $y\in (TX)_{n+1}$. Now $x\times x$ is represented by
\[\rho\Nabla D^0(x^0\otimes x^0)+\rho\Nabla D^0(dy\otimes x^0+x^0\otimes dy+dy\otimes dy).\]
Now the first term here vanishes, as
\[\rho\Nabla D^0(x^0\otimes x^0)=\rho(\phi_0 D^0(x^0\otimes x^0)+\Nabla_0 D^0(x^0\otimes x^0)+\Nabla_0 TD^0(x^0\otimes x^0)),\]
and $\phi_0D^0(x^0\otimes x^0)=0$ as $n\geq1$, and $TD^0(x^0\otimes x^0)=D^0(x^0\otimes x^0)$ since $\{D^k\}$ is special. The remaining term may be expressed as the boundary
\[d(\rho\Nabla(D^0(y\otimes dy)+D^1(dy\otimes x))).\qedhere\]
\end{proof}
\subsection{External Steenrod operations}
In this case, the Steenrod operations are the `surprising' operations, and they are easy to define. We will use the chain-level formula:
\[\textup{SQ}^{i,s}:x\mapsto \rho\Nabla (D^{s-i}(x\otimes x)+D^{s-i+1}(x\otimes dx)).\]
\begin{prop}
Fix $i,s\geq0$ and $r\geq2$. The chain level operation $\textup{SQ}^{i,s}$ defines an \emph{external Steenrod operation} \[\Sq^i_\textup{ext}:\E{r}{}{X}{s}{t}\to \E{r}{}{S_2X}{s+i}{2t},\] with indeterminacy vanishing by $\E{2r-2}{}{S_2X}{s+i}{2t}$. For $x\in \E{r}{}{X}{s}{t}$, $\Sq^i_\textup{ext}(x)$ survives to $\E{2r-1}{}{S_2X}{s+i}{2t}$, and modulo indeterminacy: \[d^{2r-1}(\Sq^i_\textup{ext}x)=\Sq^{i+r-1}_\textup{ext}(d^rx).\]
 The top operation, $\Sq^s_\textup{ext}(x)$, is equal to the product-square $x\times^\textup{ext} x$, and in particular has no indeterminacy. For $i>s$, $\Sq^i_\textup{ext}(x)=0$. $\Sq^1_\textup{ext}(x)=0$ whenever $t\geq2$ [\textbf{re-examine}]. The bottom operation, $\Sq^0_\textup{ext}(x)$, is zero whenever $t>0$, and when $t=0$:
\[\Sq^i_\textup{ext}:\E{r}{}{X}{s}{0}\to \E{r}{}{S_2X}{s}{0}\textup{ is induced by $X\overset{\textup{squaring}}{\to}S_2X$}.\]
 [\textbf{Omit/move}:  If $X$ is a mixed simplicial algebra, postcomposition with the product $S_2X\to X$ yields the same squares at $E^2$ as those arising from the fact that $E^2$ is the cohomotopy of a cosimplicial algebra.]
\end{prop}
\begin{proof}
We readily check that $\textup{SQ}^{i,s}(x)$ has filtration at least $s+i$:
\begin{align*}
\filt(\rho\Nabla (D^{s-i}(x\otimes x))&\geq s+s-(s-i)=s-i,\\\filt(\rho\Nabla (D^{s-i+1}(x\otimes dx))&\geq s+(s+r)-(s-i+1)=s+i+(r-1).
\end{align*}
Thus, we may view $\textup{SQ}^{i,s}(x)$ as an element of $\EZ{0}{}{S_2X}{s+i}{2t}$. 
A straightforward calculation shows that \[d(\textup{SQ}^{i,s}(x))=\rho\Nabla D^{s-i+1}(dx\otimes dx),\] so that, as $x\in \EZ{r}{}{X}{s}{t}$, we calculate:
\[\filt(\textup{SQ}^{i,s}(x))\geq (s+r)+(s+r)-(s-i+1)=(s+i)+(2r-1),\]
so that $\EZ{2r-1}{}{S_2X}{s+i}{2t}$. This demonstrates the survival property, along with the formula commuting the $\Sq^i_\textup{ext}$ with spectral sequence differentials.

To examine the indeterminacy, as a representative of a class in $\E{r}{}{X}{s}{t}$,  $x$ is only determined up to boundaries of $y\in \EZ{r-1}{}{X}{s-r+1}{t-r+2}  $, and we calculate, using analogues of \cite[(1.111),(1.112)]{MR2245560}, that
\[\textup{BC}:=\rho\Nabla [D^{s-i-1}(y\otimes y)+D^{s-i}(y\otimes dy)+D^{s-i+1}(dy\otimes x)]\]
has boundary
\begin{alignat*}{2}
d(\textup{BC})
&=
\rho\Nabla [D^{s-i-1}(dy\otimes y+y\otimes dy)+D^{s-i}(dy\otimes dy)+D^{s-i+1}(dy\otimes dx)]%
\\
&\ \ \ +
\rho\Nabla [0+D^{s-i-1}(y\otimes dy+dy\otimes y)+D^{s-i}(dy\otimes x+x\otimes dy)]%
\\
&=
\rho\Nabla [D^{s-i}(dy\otimes x+x\otimes dy+dy\otimes dy)
+D^{s-i+1}(dy\otimes dx)]\\
&=\textup{SQ}^{i,s}(x)-\textup{SQ}^{i,s}(x+dy).%
\end{alignat*} 
That is, $\textup{BC}$ is a bounding chain for this difference, and
\[\filt(\textup{BC})\geq 2(s-r+1)-(s-i-1)=(s+i)-(2r-3),\]
so that $\Sq^i_\textup{ext}x$ has indeterminacy vanishing by $\E{2r-2}{}{S_2X}{s+i}{2t}$ as claimed. When $i=s$, this result may be improved to $\filt(\textup{BC})\geq 2s-(r-1)$,  as in this case the lowest filtration summand in fact vanishes --- this is one explanation of why the top square has no indeterminacy.

For the statement about the top operation, one calculates that
\[\textup{SQ}^{s,s}(x)-\rho\Nabla D^0(x\otimes x)=\rho D^{1}(x\otimes dx)\in F^{2s+r-1}.\]
When $i\geq s+2$, we have $\textup{SQ}^{i}x=0$, and even with $i=s+1$:
\[\textup{SQ}^{s+1,s}(x)=\rho D^{0}(x\otimes dx)\in F^{2s+r},\]
so that $\Sq^{s+1}_\textup{ext}x$ is zero in $\E{r}{}{S_2X}{2s+1}{2t+1}$ for $r\geq2$.
For the statement about $\Sq^0(x)$:
\[\textup{SQ}^{0,s}(x)-\rho\Nabla D^s(x\otimes x)=\rho D^{s+1}(x\otimes dx)\in F^{s+r-1},\]
so that (using the fact that $\{D^k\}$ is special):
\begin{alignat*}{2}
\textup{SQ}^{0,s}(x)
&\equiv
\rho\Nabla D^s(x^s_t\otimes x^s_t)&&\textup{(mod $F^{s+r-1}$)}%
\\
&\quad =
\rho(\phi_0+(1{+}\twist)\Nabla_0)(x^s_t\otimes_\textup{v} x^s_t)%
&\quad&\text{($x^s_t\otimes_\textup{v} x^s_t\in C(X\otimes_\textup{v}X)_{2t}^{s}$)}\\
&\quad =
\rho\phi_0(x^s_t\otimes_\textup{v} x^s_t)\in C(X\otimes_{\Sigma_2} X)_{t}^{s}.
\end{alignat*}
Finally, $\Sq^1_\textup{ext}(x)\in\E{2}{}{S_2X}{}{}$ vanishes for $x\in \E{2}{}{X}{s}{t}$ with $t\geq2$ since it is the image of a $d^1$, namely $d^1(\delta_t^\textup{ext}(x))$, where $\delta_t^\textup{ext}(x)$ will be defined in proposition \ref{Prop on delta external}.
\end{proof}
It may seem that there is an extra non-zero Steenrod operation coming from the chain level operation $\textup{SQ}^{s+1,s}$, but one notes that on $E^r$ it is just the operation $x\mapsto x\times d^rx$.
%\begin{prop}
%The chain-level operation $\textup{SQ}^{s+1,s}$ defines an operation $\Sq^{s+1}_\textup{over}:E^r_{-s,t}(X)\to E^r_{-2s-r,2t+r-1}(X\otimes_{\Sigma_2}X)$, with indeterminacy vanishing by $E^{???}$. For $x\in E^r_{-s,t}(X)$, $\Sq^{s+1}_\textup{over}(x)$ survives to $E^{???}$, and moreover, $d^{r}(\Sq^{s+1}_\textup{over}x)=\Sq^{s+r}(d^rx)=d^rx\times d_rx$ modulo indeterminacy.
%\end{prop}
%\begin{proof}
%$\textup{Sq}^{s+1,s}x=\rho\Nabla D^0(x\otimes dx)\in F_{2s+r}$ has boundary $\rho\Nabla D^0(dx\otimes dx)\in F_{2s+2r}$, so we are getting a $d_r$.
%\end{proof}

\subsection{Spectral sequence operations $\delta_i$}

For any $k$, positive or otherwise, write $\mathbb{D}_k:(C(X)\otimes C(Y))_i\to C(X\otimes Y)_{i-k}$ for the map:
\[\mathbb{D}_r(z)= \sum_{\alpha-\beta=r}\Nabla_\alpha\twist^\alpha D^\beta(z).
%=\begin{cases}
%\sum_{j\geq0}\Nabla_{j}\twist^{j}D^{j-r},&\textup{if }r\leq0;\\
%\sum_{j\geq0}\Nabla_{j+r}\twist^{j+r}D^{j},&\textup{if }r\geq0.
%%\\,&\textup{if }
%\end{cases}
\]
\begin{lem}\label{DkIsNiceToFiltration}
If $x\in F_s$ and $y\in F_{s'}$, then $\mathbb{D}_k(x\otimes  y)\in F_s\cap F_{s'}$.
\end{lem}
\begin{proof}
We may assume that $x$ and $y$ are each homogeneous, with $x\in X^{s}_{t}$ and $y\in Y^{s'}_{t'}$. As $\{D^k\}$ is special, $D^{\beta}(x\otimes y)=0$ unless $\beta\leq \min\{s,s'\}$, in which case
\[\filt(D^{\beta}(x\otimes y))\geq s+s'-\beta\geq s+s'-\min\{s,s'\}=\max\{s,s'\}.\qedhere\]
\end{proof}
\begin{lem}\label{boundaryVsBBD}
For all $k$, positive or otherwise, the equation:
\[d\mathbb{D}_k+\mathbb{D}_kd= (1+\twist)\mathbb{D}_{k+1}+\Nabla\twist D^{-k-1}\]
holds when applied to an element $z$ of $(CX\otimes CX)_{n}$ with $n\geq 2(k+1)$.
\end{lem}
\begin{proof}
\newcommand{\twolinesum}[2]{\mathop{\sum_{\mathclap{#1}}}_{\mathclap{#2}}}
\newcommand{\onelinesum}[1]{\sum_{\mathclap{#1}}}
We may assume that $ z$ is homogeneous, $z\in (CX\otimes CX)_{t}^{s}$ with $n=t-s\geq2(k+1)$. Choose $\alpha$ and $\beta$ such that $\alpha-\beta=k$. Then $(\twist^\alpha D^\beta(z))\in C(X\otimes_\textup{v} Y)^{s-\beta}_{t}$.

We will need to apply lemma \ref{unpleasant formula} to calculate, for $\alpha-\beta=k$:
\[(d\Nabla_\alpha+\Nabla_\alpha d)(\twist^\alpha D^\beta(z))=((1+\twist)\Nabla_{\alpha+1}+\phi_{\alpha+1} )(\twist^\alpha D^\beta(z)),\]
but lemma \ref{unpleasant formula} does not apply when $t=2\alpha+e$ for $e\in\{0,1\}$. Fortunately, in that case $D^\beta(z)$ is zero, so the equation holds by default: after all, if $t=2\alpha+e$, our assumed inequality on $n$ implies:
\[\beta=\frac{t-e}{2}-k\leq \frac{t-e}{2}-\frac{t-s-2}{2}=\frac{s+2-e}{2}>\frac{s}{2}.\]

After these observations and under the our conventions on the $\Nabla_\alpha$ and $D^\beta$, all but one of the following manipulations is totally formal:%. The only sticking point is that $d\Nabla_{-1}+\Nabla_{-1}d$ equals $0$, as opposed to $(1{+}\twist)\Nabla_0+\phi_0$ --- we rectify this anomaly by adding in the term $((1+\twist)\Nabla_{0}+\phi_{0})\twist D^{-k-1}(z)$:
%
% Now $D^\beta(z)$ is zero unless $\beta\leq 2s$, so that the term can be ignored unless $\alpha=\beta+k\leq 2s+k$. As $t\geq 2(2s+k+1)\geq 2(\alpha+1)$, we can apply the rule.
\begin{alignat*}{2}
(d\mathbb{D}_k+\mathbb{D}_kd)(z)
:={}&
\onelinesum{\alpha-\beta=k}\left(d\Nabla_\alpha\twist^\alpha D^\beta+\Nabla_\alpha\twist^\alpha D^\beta d\right)(z)%
\\
={}&
\onelinesum{\alpha-\beta=k}\left((d\Nabla_\alpha+
\Nabla_\alpha d)\twist^\alpha D^\beta+
\Nabla_\alpha\twist^\alpha (dD^\beta+
D^\beta d)\right)(z)%
\\
={}&
\onelinesum{\alpha-\beta=k,\ \alpha\geq0}((1{+}\twist)\Nabla_{\alpha+1}+\phi_{\alpha+1})\twist^\alpha D^\beta(z)+ \onelinesum{\alpha-\beta=k} \Nabla_\alpha\twist^\alpha(1{+}\twist) D^{\beta-1}(z)%
\\
={}&
\onelinesum{\alpha-\beta=k+1,\ \alpha\geq1}((1{+}\twist)\Nabla_{\alpha}+\phi_{\alpha})\twist^{\alpha-1} D^\beta(z)+ \onelinesum{\alpha-\beta=k+1} \Nabla_\alpha\twist^\alpha(1{+}\twist) D^{\beta}(z)%
%\\
%={}&
%\onelinesum{\alpha-\beta=k+1} \left(((1+\twist)\Nabla_{\alpha}+ \phi_{\alpha})\twist\twist^{\alpha} D^\beta+\Nabla_\alpha(1+\twist)\twist^\alpha D^{\beta}\right)(z) + {}%
%\\
%&\ \ \ \ \ \ \ \ \ \ \ +
%((1+\twist)\Nabla_{0}+\phi_{0})\twist D^{-k-1}(z)
\end{alignat*}
Using the identity $(1+\twist)\Nabla_{0}+\phi_{0}=\Nabla$ for the first equation, and the observation that $(1+\twist)F\twist G+F(1+\twist)G=(1+\twist)(FG)$  for the second (with $F=\Nabla_\alpha$ and $G=\twist^\alpha D^\beta$):
%\[(d\mathbb{D}_k+\mathbb{D}_kd)(z)-\Nabla\twist D^{-k-1}(z)=\onelinesum{\alpha-\beta=k+1} \left(((1+\twist)\Nabla_{\alpha}+ \phi_{\alpha})\twist\twist^{\alpha} D^\beta+\Nabla_\alpha(1+\twist)\twist^{\alpha}D^{\beta}\right)(z)\]
\begin{alignat*}{2}
(d\mathbb{D}_k+\mathbb{D}_kd)(z)-\Nabla\twist D^{-k-1}(z)
&=
\onelinesum{\alpha-\beta=k+1} \left(((1+\twist)\Nabla_{\alpha}+ \phi_{\alpha})\twist\twist^{\alpha} D^\beta+\Nabla_\alpha(1+\twist)\twist^{\alpha}D^{\beta}\right)(z)%
\\
&=
\onelinesum{\alpha-\beta=k+1} \left((1+\twist)(\Nabla_\alpha\twist^\alpha D^\beta)+ \phi_{\alpha}\twist^{\alpha+1} D^\beta\right)(z)%
\\
% Left hand side
% Relation
&=
% Right hand side
(1+\twist)\mathbb{D}_{k+1}(z)+\onelinesum{\alpha} \phi_{\alpha}\twist^{\alpha+1} D^{\alpha-k-1}(z)%
\end{alignat*}
Due to the application of $\phi_{\alpha}$, each sumand $\phi_{\alpha}\twist^{\alpha+1} D^{\alpha-k-1}(z)$ is zero unless $t=2\alpha$, but in that case, $s=2\alpha-n<2(\alpha-k-1)$, and $D^{\alpha-k-1}(z)$ vanishes as $\{D^k\}$ is special.
%
%On the other hand, $(1+\twist)F\twist G+F(1+\twist)G=(1+\twist)(FG)$ holds for any $F$ and $G$, and in particular for $F=\Nabla_\alpha$ and $G=\twist^\alpha D^\beta$, this quantity differs from
%\[(1+\twist)\mathbb{D}_{k+1}(z)=\sum_{\alpha-\beta=k+1}(1+\twist)\Nabla_\alpha\twist^\alpha D^\beta(z)\]
%only by the terms
%\[\sum_{\mathclap{\alpha-\beta=k+1}} \phi_{\alpha}\twist^{\alpha-1}D^{\beta}(z)=\sum_{\mathclap{\alpha-\beta=k+1}} \phi_{\alpha}\twist^{\alpha-1}D^{\alpha-k-1}(z_{2\alpha}^{2\alpha-n})\]
%but the cosimplicial degree $2\alpha-n$ is less than $2(\alpha-k-1)$, and $\{D^k\}$ is special, so that each term $D^{\alpha-k-1}(z_{2\alpha}^{2\alpha-n})$ vanishes.
\end{proof}

















We will be able to define (sometimes multivalued) operations
\[\delta^\textup{ext}_i:\E{r}{}{X}{s}{t}\to \E{r}{}{S_2X}{s}{t+i}\textup{\ \ \ for $2\leq i\leq\max\{n,t-(r-1)\}$}\]
using the chain-level formula (writing $n=t-s=|x|$):
\[x\mapsto \textup{DEL}_{i}(x):=\rho(\mathbb{D}_{n-i}(x\otimes x)+\mathbb{D}_{n-i-1}(dx\otimes x)).\]
Lemma \ref{DkIsNiceToFiltration} shows immediately that $\textup{DEL}_i$ preserves filtration.
To proceed, we will need a formula for the boundary of $\textup{DEL}_i(x)$:
\begin{prop}
\label{dvsDEL}
For $2\leq i\leq t$ and $x\in \EZ{1}{}{X}{s}{t}$:
\[d(\textup{DEL}_i(x))+ \textup{DEL}_i(dx)=\begin{cases}
\textup{SQ}^{t-i+1,s}(x),&\textup{if }n+1\leq i \leq t;\\
\rho\Nabla D^0(x\otimes dx),&\textup{if }i=n;\\
0,&\textup{if }2\leq i< n.
\end{cases}\]
\end{prop}
\begin{proof}
We may apply lemma \ref{boundaryVsBBD} to calculate $d\mathbb{D}_{n-i}(x\otimes x)$ and $d\mathbb{D}_{n-i-1}(dx\otimes x)$, since 
\[|x\otimes x|=2n> 2(n-i+1)\textup{ \ and \ }|dx\otimes x|=2n-1> 2(n-i-1+1).\]
Note that the second inequality would be reversed if $i\leq1$, which is one explanation for the lack of operations $\delta^\textup{ext}_1$ and $\delta^\textup{ext}_0$. We calculate:
\begin{alignat*}{2}
d\bigl(\textup{DEL}_i(x)\bigr)+{}& \textup{DEL}_i(dx)
=
\rho d\bigl(\mathbb{D}_{n-i}(x\otimes x)+\mathbb{D}_{n-i-1}(dx\otimes x)\bigr)+\rho\mathbb{D}_{n-i-1}(dx\otimes dx)%
\\
&=
\rho\Bigl\{d\mathbb{D}_{n-i}(x\otimes x)\Bigr\}+\rho\Bigl\{d\mathbb{D}_{n-i-1}(dx\otimes x))+\mathbb{D}_{n-i-1}(d(dx\otimes x))\Bigr\}%
\\
&=
\rho\Bigl\{\mathbb{D}_{n-i}d(x\otimes x)+(1+\twist)\mathbb{D}_{n-i+1}(x\otimes x)+\Nabla\twist D^{i-n-1}(x\otimes x)\Bigr\}
\\
&\ \ \ \ +\rho\Bigl\{(1+\twist)\mathbb{D}_{n-i}(dx\otimes x)+\Nabla\twist D^{i-n}(dx\otimes x)\Bigr\}
\end{alignat*}
Everything cancels except for $\rho\Nabla(D^{i-n-1}(x\otimes x)+D^{i-n}(x\otimes dx))$ which equals $\textup{SQ}^{t-i+1,s}(x)$. We have studied this expression above, explaining the three cases.
\end{proof}
Suppose now that $x\in \E{r}{}{X}{s}{t}$. In light of the above calculation, when $n<i\leq t$, the purpose of $\delta^\textup{ext}_i(x)$ must be to support a $d^{t-i+1}$-differential to $\Sq_\textup{ext}^{t-i+1}(x)$. Thus, we would not expect to be able to define $\delta^\textup{ext}_i(x)$ when $t-i+1<r$; indeed, the following result will construct $\delta^\textup{ext}_i(x)$ whenever $i\leq t-(r-1)$. Moreover, the Steenrod operation $\Sq^{t-i+1}(x)$ has indeterminacy vanishing by $\E{2(r-1)}{}{S_2X}{s+t-i+1}{2t}$, and one should expect that whenever $t-i+1<2(r-1)$, $\delta^\textup{ext}_i(x)$ will be multivalued, but that the set of values for $\delta^\textup{ext}_i(x)$ will map onto the set of values for $\Sq_\textup{ext}^{t-i+1}(x)$ under $d^{t-i+1}$. 
We are not saying that the indeterminacy of $\delta^\textup{ext}_i(x)$ must vanish by a certain page, but rather that one expects multiple values of $\delta^\textup{ext}_i(x)$ to all fail to be permanent cycles together.
\begin{prop}\label{Prop on delta external}
The chain-level map $\textup{DEL}_i$ produces (potentially multivalued) operations $\delta^\textup{ext}_i:\E{r}{}{X}{s}{t}\to \E{r}{}{S_2X}{s}{t+i}$ whenever $r\geq1$ and $2\leq i\leq \max\{n,t-(r-1)\}$. If either $i\leq n+1$ or $i\leq t+1-2(r-1)$, the operations are single-valued.
\end{prop}
\begin{proof}
Suppose that $x\in \EZ{r}{}{X}{s}{t}$. Then proposition \ref{dvsDEL} shows that $d\textup{DEL}_i(x)\in F^{s+r}C(S_2X)$ as long as $i\leq \max\{n,t-(r-1)\}$. We must now examine whether this map is well defined on $\E{r}{}{X}{s}{t}$, which is to examine the difference $\textup{DEL}_i(x)-\textup{DEL}_i(x+dy)$, for $y\in \EZ{r-1}{}{X}{s-r+1}{t+r+2}$. We should only expect to be successful when $i\leq t+1-2(r-1)$.
Using the symbol `$\cdot$' to abbreviate `$\otimes$', with lemma \ref{boundaryVsBBD} we calculate:
\small
\begin{alignat*}{4}
d\mathbb{D}_{n-i-1}(x\cdot dy)&=
{\mathbb{D}_{n-i-1}(dx\cdot dy)}&&+
(1{+}\twist)\mathbb{D}_{n-i}(x\cdot dy)&&+
\Nabla\twist D^{-(n-i)}(x\cdot dy)\\
d\mathbb{D}_{n-i}(dy\cdot y)&=
\mathbb{D}_{n-i}(dy\cdot dy)&&+
(1{+}\twist)\mathbb{D}_{n-i+1}(dy\cdot y)&&+
\Nabla\twist D^{-(n-i+1)}(dy\cdot y)\\
d\mathbb{D}_{n-i+1}(y\cdot y)&=
\mathbb{D}_{n-i+1}d(y\cdot y)&&+
{(1{+}\twist)\mathbb{D}_{n-i+2}(y\cdot y)}&&+
\Nabla\twist D^{-(n-i+2)}(y\cdot y)
\end{alignat*}
\normalsize
Now write $H:=\rho(\mathbb{D}_{n-i-1}(x\cdot dy)+\mathbb{D}_{n-i}(dy\cdot y)+\mathbb{D}_{n-i+1}(y\cdot y))$. By lemma \ref{DkIsNiceToFiltration}, $H\in F^{s-r+1}$, and the above calculations show that $H$ has boundary
\small
%\[(\textup{DEL}_i(x)-\textup{DEL}_i(x+dy))+\rho\Nabla(D^{i-n-2}(y\cdot y)+D^{i-n-1}(y\cdot dy)+D^{i-n}(dy\cdot x))+\rho\mathbb{D}_{n-i-1}(dx\cdot dy).\]
\[dH=\textup{DEL}_i(x)-\textup{DEL}_i(x+dy)+T_1+T_2+T_3,\]
\normalsize
where $\textup{DEL}_i(x)-\textup{DEL}_i(x+dy)\in F^s$ is the quantity of interest, and
\begin{align*}
T_1:=\rho\Nabla(D^{i-n-2}(y\cdot y))&\begin{cases}
\in F^{s+(t-i)-2(r-2)},&\textup{if }n+2\leq i\leq t-1;\\
=0,&\textup{if }i\leq n+1.
%\\,&\textup{if }
\end{cases}
\\
T_2:=\rho\Nabla(D^{i-n-1}(y\cdot dy))&\begin{cases}
\in F^{s+(t-i)-(r-2)},&\textup{if }n+1\leq i\leq t-1;\\
=0,&\textup{if }i\leq n.
%\\,&\textup{if }
\end{cases}
\\
T_3:=\rho\Nabla(D^{i-n}(dy\cdot x))&\begin{cases}
\in F^{s+(t-i)},&\textup{if }n\leq i\leq t-1;\\
=0,&\textup{if }i\leq n-1.
%\\,&\textup{if }
\end{cases}
\end{align*}
Now as we have $i<t$, $T_3\in F^{s+1}$, and can be ignored. If $i\leq n$, then $T_1=T_2=0$, and we've shown that the operation is well defined. If $i=n+1$, then $T_2$ may be non-zero, but it lies in $F^{s+1+s-r}$, and as we may take $r\leq s$ [\textbf{revisit}], the operation is again well defined.
If, however, $i\geq n+2$, then $T_1+T_2$  can only be assured to lie in filtration above $s$ only when $i\leq t+1-2(r-1)$.
%\textbf{The following is an observation that I thought might improve this result, but which doesn't seem to - the idea was to leave out the third row of stuff. Maybe it could still be good if we leave out some number of the summands from the third row, but it sounds pretty dicey. Should try it in order to extend rande of definition. Also should check whether one should expect the operations to be defined all the way up to $i\leq t-r+1$. ``Now suppose that $x\in F_s$, and $y\in F_{s-r+1}$. Let us look at $\mathbb{D}_{n-i+1}(dy\cdot y)$. As $dy$ has filtration $s$, each term $\Nabla_{\alpha}\twist^{\alpha}D^\beta(dy\cdot y)$ with $\beta>s$ vanishes, and each term with $\beta<s$ has filtration higher than $dy$. Thus, modulo $F_{s+1}$, $\mathbb{D}_{n-i+1}(dy\cdot y)\equiv\Nabla_{t-i+1}\twist^{t-i+1}D^s((dy)^s_{t}\cdot y^{s}_{t+1})$.''}

To summarise: there is some $H$ in filtration at least that of $y$, such that
%\[dH=\textup{DEL}_i(x)-\textup{DEL}_i(x+dy)+T, \textup{ where } dT=\textup{SQ}^{i,s}(x)-\textup{SQ}^{i,s}(x+dy)\]
\begin{alignat*}{2}
dH
&=
\textup{DEL}_i(x)-\textup{DEL}_i(x+dy)+T, \textup{ and }%
\\
dT
&=
\textup{SQ}^{t-i+1,s}(x)-\textup{SQ}^{t-i+1,s}(x+dy)
\end{alignat*}
%%%%%%%%%%%%\usepackage[amsmath,amsthm,thmmarks]{ntheorem}
So that, in particular, [so what?]
\[d\textup{DEL}_i(x+dy)-\textup{SQ}^{t-i+1,s}(x+dy)=d\textup{DEL}_i(x)-\textup{SQ}^{t-i+1,s}(x)\qedhere\]
\end{proof}

\subsection{Commuting $\delta$ and $\Sq$}
\textbf{Unchecked from here...} 
When we do not lift any classes up a filtration, at $E^1$, we have the relation $\delta_i\Sq^jx=0$ (whenever $\delta_i$ is non-top).
\begin{proof}
Suppose that $X\in Z^1_{-s,t}$. We can assume that $x\in X^s_t$ for this purpose, so that $d_\textup{v}x=0$. Then $\Sq^jx$ is represented by the image of $D^{s-j}(x\otimes x)+D^{s-j+1}(x\otimes d_\textup{h}x)$ under the composite
\[N^{s+j}(N_tX\otimes N_tX)\overset{\widetilde{\Nabla}}{\to}N^{s+j}N_{2t}(X\otimes_{\Sigma_2} X)\overset{\mu}{\to}N^{s+j}N_{2t}(X)\overset{\textup{``$\delta_i$''}}{\to}N^{s+j}N_{2t+i}(X)\]
or maybe I'd prefer to write that it is represented by the image of ``an appropriate tensor product of cofaces of $x\in (\pi_tX)^s$'', under the composite
\[N^{s+j}(\pi_tX\otimes \pi_tX)\overset{\Nabla\textup{-product}}{\to}N^{s+j}\pi_{2t}(X)\overset{\delta_i}{\to}N^{s+j}\pi_{2t+i}(X).\]
But non-top $\delta_i$ annihilates sums of products, and we are getting zero on $E^1$.
\end{proof}

\subsection{Summary of results}

As $E^1$ is obtained from $E^0$ by taking homology in the simplicial direction, there are $\delta^\textup{ext}$-operations on $E^1$, i.e.\ $E^1_{-s,t}(X)\to E^1_{-s,t+i}(X\otimes_{\Sigma_2}X)$. They are unstable with respect to internal degree $t$, not with respect to total degree $n$. All of these operations are linear except for the top operation \cite[4.2]{DwyerHtpyOpsSimpComAlg.pdf}, and commute with the differentials in the cosimplicial direction. Thus, we obtain operations $\delta^\textup{alg}_i:E^2_{-s,t}(X)\to E^2_{-s,t+i}(X\otimes_{\Sigma_2}X)$ for $2\leq i<t$.
\begin{prop}
On $E^2$, for $i<t$, the spectral sequence operations $\delta^\textup{ext}_i$ defined by $\textup{DEL}_i$ equal the operations $\delta^\textup{alg}_i$. The same holds on $E^1$ when $2\leq i\leq t$. In particular, the $\delta^\textup{ext}$-operations on the spectral sequence satisfy the $\delta$-Adem relations \textbf{if there is a commutative algebra structure around}.
\end{prop}
\begin{proof}
The operations $\delta^\textup{alg}_i$ are constructed by taking a representative $z\in C_{-s,t}X$ such that $\partial_\textup{sim}z=0$, and forming $\rho\Nabla_{t-i}(z\otimes_\textup{v} z)$, where $z\otimes_\textup{v} z$ denotes the element of $C(X\otimes_\textup{v}X)_{-s,2t}$.

Let us calculate the leading term of $\textup{DEL}_{i}(x)$ whenever $2\leq i\leq t$. Write $x=x^s_t+x^{>s}_{>t}$, where $x^s_t\in X^s_t$, and $x^{>s}_{>t}$ lies in filtration $s+1$. Then the leading term will be $\rho\Nabla_{t-i}\twist^{t-i}D^s (x^s_t\otimes x^s_t)$. Moreover, as $\{D^k\}$ is \emph{special}, this equals $\rho{\Nabla}_{t-i}(x^s_t\otimes_\textup{v} x^s_t)\in C(X\otimes_{\Sigma_2} X)_{-s,t+i}$, where $(x^s_t\otimes_\textup{v} x^s_t)$ denotes the element of $C(X\otimes_\textup{v}X)_{-s,2t}$. Thus `$x^s_t\mapsto \twist^{t-i}D^s(x^s_t\otimes x^s_t)$' implements the assignment `$z\mapsto z\otimes_\textup{v} z$' discussed above, and we are getting a representative for $\delta^\textup{alg}_i(x)$.
\end{proof}




\begin{thm}
By postcomposition with the multiplication $S_2X\to X$, the operations $\Sq^j_\textup{ext}$, $\delta_i^\textup{ext}$ and $\times^\textup{ext}$ induce operations $\Sq^j_\textup{raw}$, $\delta_i^\textup{raw}$, $\times^\textup{raw}$ on $E^r(X)$. The Steenrod operations vanish whenever $i<r$, after all:
\begin{enumerate}\squishlist
\setlength{\parindent}{.25in}
\item When $i<t-s$, $d^r\delta^\textup{raw}_i(x)=\delta^\textup{raw}_i(d^rx)$.
\item When $i=t-s$, $d^r\delta^\textup{raw}_i(x)=\begin{cases}
\delta^\textup{raw}_i(d^rx),&\textup{if }s>0;\\
\delta^\textup{raw}_i(d^rx)+x\times d^rx,&\textup{if }s=0.
%\\,&\textup{if }
\end{cases}$
\item When $i>t-s$, $d^r\delta^\textup{raw}_i(x)=\begin{cases}
\delta^\textup{raw}_i(d^rx),&\textup{if }r<t-i+1;\\
\delta^\textup{raw}_i(d^rx)+\Sq_\textup{raw}^{t-i+1}(x),&\textup{if }r=t-i+1;\\
\Sq_\textup{raw}^{t-i+1}(x),&\textup{if }r>t-i+1.\\
%\\,&\textup{if }
\end{cases}$
\end{enumerate}
Broadly speaking then, the $\delta^\textup{raw}_i$ operations for $i>t-s$ support a differential hitting either a Steenrod operation or another $\delta^\textup{raw}_i$ operation, and all Steenrod operations applied to permanent cycles are hit in this way. If $X$ admits a coaugmentation from a simplicial algebra $X$, then the $\delta^\textup{raw}$-operations with $i\leq n$ at $E^\infty$ agree with the $\delta^\textup{raw}$-operations on the target of the spectral sequence.

The $\delta^\textup{raw}$-operations (resp.\ Steenrod operations)  at $E^2$ agree with those following from the observation that $E^1$ is the normalization of a cosimplicial $\textup{Com}$-$\Pi$-algebra (resp.\ algebra). As such, the $\delta_i^\textup{raw}$ and $\Sq^j_\textup{raw}$ satisfy the expected Adem relations. Finally, at $E^1$, we have the relation $\delta_i^\textup{raw}\Sq^j_\textup{raw}x=0$ (whenever $\delta_i^\textup{raw}$ is non-top). \textbf{Probably more detail needed}, for example, when is $\Sq_\textup{raw}^0=0$ or $\Sq_\textup{raw}^1=0$?

[from above]: For $x\in E^r_{-s,t}(X)$, $\Sq^i_\textup{ext}(x)$ survives to $E^{2r-1}$, and moreover, $d^{2r-1}(\Sq^i_\textup{ext}x)=\Sq^{i+r-1}_\textup{ext}(d^rx)$ modulo indeterminacy. The top operation, $\Sq^s_\textup{ext}(x)$, is equal to the product-square $x\times^\textup{ext} x$, and $\Sq^i_\textup{ext}(x)=0$ for $i>s$. The bottom operation, $\Sq^0_\textup{ext}(x)$, is zero whenever $t>0$, and $\Sq^1_\textup{ext}(x)=0$ for all $t\geq2$. If $X$ is a mixed simplicial algebra, postcomposition with the product $S_2X\to X$ yields the same squares at $E^2$ as those arising from the fact that $E^2$ is the cohomotopy of a cosimplicial algebra.
\end{thm}

\end{second quadrant homotopy}

\begin{Operations on the Bousfield-Kan spectral sequence}
\section{\textbf{Operations in the Bousfield-Kan spectral sequence}}\label{Operations on the Bousfield-Kan spectral sequence}
\textbf{whole chapter $X\mapsto \calX$?}
\textbf{Let $\algs$ be $\algs$ or $\liealgs$}
\subsection{A shift in filtration}
Suppose that $A\in cs\algs$, so that $A$ is equipped with a pairing $S_2A\to A$. We may employ the eternal operations of \S?????? to obtain such operations as
\[\E{r}{}{A}{s}{t}\overset{\delta_i^\textup{ext}}{\to} \E{r}{}{S_2A}{s}{t+i}\to \E{r}{}{A}{s}{t+i}.\]
Whenever $A=\calX$, by which we mean Radulescu-Banu's resolution of some $X\in s\algs$, the results of this chapter will show that these operations must all equal zero. Indeed, the map $\E{r}{}{S_2\calX}{s}{t+i}\to \E{r}{}{\calX}{s}{t+i}$ can be lifted up one filtration, essentially since the resolution is by GEMs. That said, we will need work with  Radulescu-Banu's resolution directly in order to make our constructions.

\begin{shaded}\tiny
For $Z\in c\vect{}{}$, Bousfield and Kan write $VZ$ for a ``path-like construction'' \cite[\S3.1]{BK_pairings_products.pdf} on $Z$ obtained by shifting $Z$ down and forgetting the 0th coface and codegeneracy. That is, $(VZ)^s:=(VZ)^{s+1}$, and:
\begin{align*}
\bigl((VZ)^s\overset{d^i}{\to} (VZ)^{s+1}\bigr)&:=\bigl(Z^{s+1}\overset{d^{i+1}}{\to} Z^{s+2}\bigr)\\
\bigl((VZ)^s\overset{s^i}{\to} (VZ)^{s-1}\bigr)&:=\bigl(Z^{s+1}\overset{d^{i+1}}{\to} Z^{s}\bigr)
\end{align*}
In fact, the unused coface $d^0$ induces a map $v:Z\to VZ$.

For $Y\in s\vect{}{}$, the standard simplicial path fibration (c.f.\ \cite[p.82]{BousKanSSeq.pdf}) produces $\Lambda Y\in s\vect{}{}$ by shifting down and restricting to a kernel:
\[\Lambda Y_s=\ker\bigl(d_{s+1}\cdots d_1:Y_{s+1}\to Y_0\bigr).\]
We forget the $0^\textup{th}$ face and degeneracy as before. This time the unused face map $d_0$ induces a fibration $\lambda:\Lambda Y\to Y$, and $\Lambda Y$ is  contractible.



Each of these constructions can be prolonged to an endofunctor of $cs\vect{}{}$. 
Now it is possible to recall a key construction of Bousfield and Kan \cite{BK_pairings_products.pdf,BK_pairings.pdf}. Define $D^1\calX \in cs\algs$ to be the pullback:
\[\xymatrix@R=4mm{
D^1\calX \ar[r]
\ar[d]^-{\delta}
&%r1c1
\Lambda V\calX \ar[d]^-{\lambda}
\\%r1c2
\calX \ar[r]^-{v}
&%r2c1
V\calX %r2c2
}\]
The overwhelming virtue of this construction is the following observation of Bousfield and Kan \cite[Proposition 5.2]{BK_pairings_products.pdf}:
\begin{prop}\label{BK D1 is awesome}
The connecting map
\[\pi_t(\calX^{s})=\pi_t(V\calX^{s-1})\overset{\partial }{\to}\pi_{t-1}(D^1\calX)^s\]
in the homotopy long exact sequence of the fiber sequence $(D^1\calX)^{s-1}\to \calX^{s-1}\to V\calX^{s-1}$ induces an isomorphism of cochain complexes
\[N^s_\subset \pi_t \calX\subseteq N^{s-1}_\subset \pi_t V\calX \to  N^{s-1}_\subset \pi_{t-1} (D^1\calX).\]
\end{prop}
Consequently, the connecting map induces an equivalence
\[C^s_\subset \pi_t (D^1\calX)\to  C^s_\subset \pi_{t-1} (V\calX),\]
and 
\end{shaded}

Our goal will be to create a factorization of the multiplication map $\mu:\calX \smashcoprod  \calX \to \calX $ through $\delta:D^1\calX \to \calX $. This will be possible (using a zig-zag), as $\calX^s_t=(c(KQc)^{s+1}X)_t$ is a  Radulescu-Banu resolution. In this case, not only does
\[(V\calX)^s_t=(c(KQc)^{s+2}X)_t\in cs\algs\]
have cosimplicial and simplicial structure maps,
but there is a cosimplicial simplicial algebra structure on the object $\overline{V}\calX$ obtained by omitting the leftmost replacement $c$:
\[(\overline{V}\calX)^s_t=((KQc)^{s+2}X)_t\in cs\algs.\]
That is, we do not \emph{need} the outermost cofibrant replacement to define $\overline{V}\calX$, as in passing from $\calX$ to $V\calX$ one discards $d^0$. Of course, there is a $cs\algs$-map $\epsilon:VX\to \overline{V}X$ which is a weak equivalence in each cosimplicial level. Finally, the composite
\[\overline{v}:=\epsilon\circ v:\left(\calX\overset{v}{\to}V\calX\overset{\epsilon}{\epi} \overline{V}\calX\right)\]
is a levelwise fibration of simplicial algebras, as it is defined in cosimplicial degree $s$ by the formula $\overline{v}=\eta:c(KQc)^{s+2}A\to KQc(KQc)^{s+2}A$. The object $\overline{V}\calX$ has two key advantages: $\overline{v}$ is a fibration in each cosimplicial level, and $\overline{V}\calX$ is a trivial object in $s\calC$ (i.e.\ it is in the image of $K^{\algs}$). This second property implies that  $\overline{V}\calX$ is an abelian group object in $s\calc$ in each cosimplicial level. This is clear, as every vector space is a group object, and $K^{\algs}$ is a right adjoint; one might also say that since all the structure maps in $\overline{V}\calX$ are trivial, simple vector space addition commutes with them. We write
\[\textup{add}:\overline{V}\calX\times \overline{V}\calX\to \overline{V}\calX\]
for the group operation. Under the identifications arising from propositions(?) \ref{something about dualisation} and \ref{smash prod}, the map $\textup{add}$ induces the abelian group and cogroup structures on $H_*^\algs\overline{V}\calX$ and $H^*_\algs\overline{V}\calX$ expected
\begin{gather*}
H_*^\algs\overline{V}\calX\times H_*^\algs\overline{V}\calX\to H_*^\algs\overline{V}\calX\\
H^*_\algs\overline{V}\calX\sqcup H^*_\algs\overline{V}\calX\from H^*_\algs\overline{V}\calX
\end{gather*}

The observation that $\overline{v}$ is a fibration leads us to define $\overline{D}^1X$ to be the strict fiber
\[\xymatrix@R=4mm{
D^1\calX \ar[r]
\ar[d]^-{\delta}
&%r1c1
0\ar[d]\\
\calX \ar@{->>}[r]^-{\overline{v}}
&%r2c1
\overline{V}\calX %r2c2
}\]
There is a commuting diagram in $cs\algs$ (in which double-headed arrows denote maps which are fibrations in $s\algs$ in each cosimplicial level):
\[\xymatrix@R=4mm{
0\ar[r]\ar[d]
&%r1c1
0\ar[d]
\ar[r]
&%r1c2
\Lambda \overline{V}X\ar@{->>}[d]^-{\lambda}
&%r1c3
\Lambda VX\ar[l]^-{\Lambda(\epsilon)}
\ar@{->>}[d]^-{\lambda}
\\%r1c4
0\ar[r]
%\ar[dr]_-{\mu}
&%r2c1
\overline{V}X\ar@{=}[r]
&%r2c2
\overline{V}X&%r2c3
VX\ar@{->>}[l]^-{\epsilon}
\\%r2c4
X\smashcoprod_{\Sigma_2} X\ar[r]^-{\mu}\ar[u]
&%r3c1
X\ar@{->>}[u]_-{\overline{v}}
\ar@{=}[r]
&%r3c2
X\ar@{->>}[u]_-{\overline{v}}&%r3c3
X\ar[u]_-{v}\ar@{=}[l]%r3c4
\\
\makebox[0cm][r]{pullbacks:\qquad}X\smashcoprod_{\cancel{\Sigma_2}} X\ar[r]^-{\overline{\mu}}
&
\overline{D}^1X\ar[r]^-{\textup{$E_2$-eq}}
&
\smash{\widetilde{D}^1X}&
\ar[l]_-{\textup{$E_2$-eq}}
D^1X
}\]
The bottom left square commutes, as $\overline{v}\circ\mu=\mu\circ(\overline{v}\smashcoprod\overline{v})$, and 
$\mu:\overline{V}X\smashcoprod \overline{V}X\to \overline{V}X$ is  zero since $\overline{V}X$ is trivial. \textbf{Explain why they are $E_2$-equivalences}.

This diagram has produced a factorisation $\overline{\mu}:X\smashcoprod_{\Sigma_2}  X \to \overline{D}^1X$ of $\mu$, and a zig-zag of $E^2$-equivalences from $\overline{D}^1X$ to $D^1X$.  In each cosimplicial level, each of the objects in the top row is contractible, yielding homotopy long exact sequences. The resulting connecting homomorphisms commute:
\[\mathclap{\xymatrix@R=4mm{
\pi_{\frakt}(\overline{D}^1X)&%r1c3
\pi_{\frakt+1}(\overline{V}X)\ar[l]_-{\partial_\textup{conn}}
\\%r1c4
\pi_{\frakt}(D^1X)\ar@{-}[u]^-{\textup{?????????zig-zag}}_-{\cong}
&%r2c3
\pi_{\frakt+1}(VX)\ar[u]_-{\cong}\makebox[0cm][l]{\ \ \ put me on prev page}
\ar[l]_-{\partial_\textup{conn}}
}}\]
\begin{shaded}\tiny
Bousfield and Kan explain that the composite
\[N^\fraks\pi_\frakt(D^1X)\overset{\partial_\textup{conn}}{\from}N^\fraks\pi_{\frakt+1}(VX)\supset N^{\fraks+1}\pi_{\frakt+1}(X)\]
is an isomorphism, where we note that the condition to lie in $N^{s+1}(O)\subset O^{s+1}$ is stricter than the condition to lie in $N^s(VO)\subset O^{s+1}$, for $O$ a cosimplicial module. Note that this composite is a cochain map, the reason being that the extra ``$d^0$'' in the coboundary out of $N^{\fraks+1}\pi_{\frakt+1}(X)$ that does not appear in the coboundary out of $N^{\fraks}\pi_{\frakt+1}(VX)$ becomes null after application of $\partial_\textup{conn}$, which is to say that the composite
\[\pi_{\frakt+1}(X^{\fraks+1})\overset{d^0}{\to}\pi_{\frakt+1}(X^{\fraks+2})\overset{\partial_\textup{conn}}{\to}\pi_\frakt((D^1X)^{\fraks+1})\]
is zero, which is clear, as we are just looking at adjacent maps in the homotopy long exact sequence.
\end{shaded}

\begin{shaded}\tiny
The next result states that the exterior operations, when lifted up one filtration level, induce the operations derived on unstable Lie coalgebra homology, albeit with a shift in the indexing of the Steenrod operations. Writing $\lambda$ for the lifting map
\[\lambda:\left(E^r_{s,t}(X\smashcoprod_{\Sigma_2}X)\to E^r_{s,t}(D^1X)\cong E^r_{s+1,t+1}(X)\right)\textup{$\qquad (r\geq2)$}\]
\begin{thm}
We have the following identities of operations on the Bousfield-Kan $E^2$ page:
\[\lambda\circ\delta_i^\textup{ext}=\delta_i\]
\[\lambda\circ\Sq^{j-1}_\textup{ext}=\Sq^j\]
\[\lambda\circ\times^\textup{ext}=\times\]
\end{thm}
In light of this result, we have extended the unstable Lie algebra cohomology operations defined in section XYZ to operations on the whole spectral sequence, with certain known differentials between them, and abutting to the operations on homotopy as appropriate. The relations we know between the unstable Lie algebra cohomology operations persist to relations on the higher pages. To summarise:
\begin{thm}
There are operations
\begin{alignat*}{2}
\times:E^r_{-s,t}\otimes E^r_{-s',t'}&\to E^r_{-s-s'-1,t+t'+1}&&\\
\Sq^j:E^r_{-s,t}&\to E^r_{-s-j,2t+1}&&\textup{\quad (zero unless $r+1\leq j\leq s+1$)}\\
\delta_i:E^r_{-s,t}&\to E^r_{-s-1,t+i+1}&&\textup{\quad (defined when $2\leq j\leq\max(t-r+1,t-s)$)}
\end{alignat*}
%\[\times:E^r_{-s,t}\otimes E^r_{-s',t'}\to E^r_{-s-s'-1,t+t'+1}\]
%\[\Sq^j:E^r_{-s,t}\to E^r_{-s-j,2t+1}\textup{\quad (zero unless $r+1\leq j\leq s+1$)}\]
%\[\delta_i:E^r_{-s,t}\to E^r_{-s-1,t+i+1}\textup{\quad (defined when $2\leq j\leq\max(t-r+1,t-s)$)}\]
\begin{enumerate}\squishlist
\setlength{\parindent}{.25in}
\item When $i<t-s$, $d_r\delta_i(x)=\delta_i(d_rx)$.
\item When $i=t-s$, $d_r\delta_i(x)=\begin{cases}
\delta_i(d_rx),&\textup{if }s>0;\\
\delta_i(d_rx)+x\times d_rx,&\textup{if }s=0.
%\\,&\textup{if }
\end{cases}$
\item When $i>t-s$, $d_r\delta_i(x)=\begin{cases}
\delta_i(d_rx),&\textup{if }r<t-i+1;\\
\delta_i(d_rx)+\Sq^{t-i+2}(x),&\textup{if }r=t-i+1;\\
\Sq^{t-i+2}(x),&\textup{if }r>t-i+1.\\
%\\,&\textup{if }
\end{cases}$
\end{enumerate}
The operations $\delta_i:E^\infty_{-s,t}\to E^\infty_{-s-1,t+i+1}$, for $i\leq t-s$, agree with the homotopy operations $\delta_i:\pi_{t-s}\to \pi_{t-s+i}$ on the target of the spectral sequence, and similarly for the product at $E^\infty$.

The $\delta_i$ and $\Sq^j$ satisfy the expected Adem relations, and the relation $\delta_i\Sq^jx=0$ (when?). The top Steenrod operation, $\Sq^{s+1}$, equals the squaring operation under the product. (more: identify every ingredient in \cite[{?2.1}]{DwyerHtpyOpsSimpComAlg.pdf}, and in \cite[{?5.3}]{PriddySimplicialLie.pdf}, including what happens on the axes.)

For $x\in E^r_{-s,t}(X)$, $\Sq^i(x)$ survives to $E^{2r-1}$, and moreover, $d_{2r-1}(\Sq^ix)=\Sq^{i+r-1}(d_rx)$ modulo indeterminacy.  [The bottom operation, $\Sq^1(x)$, is zero whenever $t>0$\textbf{ silly if in connected case, should just say `is zero'}]. In particular, all Steenrod operations commute with $d_1$, in that $d_1\Sq^ix=\Sq^id_1x$.
\end{thm}

\end{shaded}










\subsection{A chain-level construction $\xi_\textup{res}^*$ inducing $\xi_{\calC\HAlg}$}\label{sec xires}
In \S\ref{chain level structure}, we defined a map
\[\xi_{\calc\HAlg}:B^{s+1}_{\calc\HAlg}H^*_{\calc}X\to B^{s}_{\calc\HAlg}H^*_{\calc}X\smashcoprod B^{s}_{\calc\HAlg}H^*_{\calc}X,\]
which we used in \S\ref{Horizontal Steenrod operations and a product for HWn} to define Steenrod operations and a product on $H^{*}_{\calc\HAlg}H^*_{\calc}X$ when $\calc=\algs$.
We will now construct, at the level of Radulescu-Banu's resolution, a map
\[\xi_\textup{res}^\star:\calX^s\Lsmashprod \calX^s\to V\calX^s,\]
which induces, on cohomology, the map
\[B^{s}_\calc H^*_\calc X\smashcoprod B^{s}_\calc H^*_\calc X\cong H^*_{\calc}(\calX^s\Lsmashprod \calX^s)\overset{\xi_{\calc\HAlg}}{\from} H^*_{\calc}(V\calX^s)\cong B^{s+1}_\calc H^*_\calc X\]
under the isomorphisms of theorem \ref{identify E2 with derived Q} and proposition \ref{smash prod}.

 %This map could \emph{not} be a cosimplicial map, as it  $(H_*^{\calc}X^\bullet)^{\times2}\to H_*^{\calc}VX^\bullet$, as it must increase cosimplicial degree by one. However, it \emph{does} induce a chain map on the \cancel{normalised?} complexes of primitives, as we might expect from \ref{psi is basically the quadratic part}.
We will need to abbreviate a little for the sake of compactness. Fix a cosimplicial degree $s$. Write $\calX$ for $\calX^s$, $V$ for $V\calX^s$, $\overline{V}$ for $\overline{V}\calX^s$, dots for categorical products, and superscripts for categorical self-products. There is a diagram
\[\newcommand{\Times}{{\cdot}}
\mathclap{\xymatrix@R=4mm{
\calX\Lsmashprod \calX
\ar@{-->}@/^1em/[rrrrd]^-(.9){\xi_\textup{res}^\star}
\\%r3c5
\ar[u]
\ar@/^1.3em/[rrrr]^-(.1){\overline{\xi}_\textup{res}^\star}
c(\calX^2)\ar[r]_-(.7){c(\epsilon,\textup{id})\circ\beta}
\ar@{..>}[dr]_-(.8){(\epsilon,\textup{id})}
&%r4c1
c(\calX^2\Times c(\calX^2))\ar[r]_-{c(\overline{v}^2\Times c(\textup{add}\circ\epsilon^2))}
\ar@{..>}[d]^-{\epsilon}
%\ar[ddd]
&%r4c2
c(\overline{V}^2\Times \calX)
\ar[r]_-{c(\textup{add}\Times \overline{v})}
\ar@{..>}[d]^-{\epsilon}
%\ar[ddd]
&%r4c3
c(\overline{V}\Times \overline{V})\ar[r]_-{c(\textup{add})}
\ar@{..>}[d]^-{\epsilon}
%\ar[dd]^-{\beta}
&%r4c4
V\ar@{..>}[d]^-{\epsilon}
\\%r4c5
c(\calX\sqcup \calX)\ar[u]
&\calX^2\Times c(\calX^2)
\ar@{..>}[r]^-{\overline{v}^2\Times c(\textup{add}\circ\epsilon^2)}
&
\overline{V}^2\Times \calX
\ar@{..>}[r]^-{\textup{add}\Times \overline{v}}
&
\overline{V}\Times \overline{V}\ar@{..>}[r]^-{\textup{add}}
&
\overline{V}
}}\]
in which we define $\overline{\xi}_\textup{res}^\star$ to be the composite of the horizontal solid arrows. The subdiagram consisting of solid and dotted arrows strictly commutes, and we will define   $\xi_\textup{res}^\star$ to be the unique map up to homotopy such that the full diagram homotopy commutes, after showing that the composite $c(\calX\sqcup\calX)\to V$ is null.
The maps defined here need a little clarification, during which we will resume writing cosimplicial degrees:
\begin{gather*}
c(\epsilon,Id)\circ\beta:\bigl( c(\calX^s)^2\overset{\beta}{\to}cc(\calX^s)^2\overset{c(\Id,\Id)}{\to}c(c(\calX^s)^2\cdot c(\calX^s)^2)\overset{c(\epsilon\cdot \Id)}{\to}c((\calX^s)^2\cdot c(\calX^s)^2) \bigr)\\
c(\textup{add}\circ\epsilon^2):\bigl(c((\calX^s)^2)\overset{c(\epsilon^2)}{\to}c((\overline{V}\calX^{s-1})^2)\overset{c(\textup{add})}{\to} c(\overline{V}\calX^{s-1})=\calX^{s} \bigr).
\end{gather*}
Fortunately, the fact that the diagram (without the dashed arrow) commutes is obvious: the small triangle commutes by counitality of $\beta$, and the three squares commute by naturality of $\epsilon:c\to\Id$.
\begin{prop}\label{res xi induces xi}
The map $\overline{\xi}_\textup{res}^\star$ induces the map $\overline{\xi}_{\calc\HAlg}$ on cohomology, and descends to a map ${\xi}_\textup{res}^\star:\calX\Lsmashprod\calX\to V $ as suggested by the dashed arrow above. This map induces the map ${\xi}_{\calc\HAlg}^\star$ on homology.
\end{prop}
\begin{proof}
Under the isomorphisms of propositions \ref{something about dualisation} and \ref{smash prod}, if we apply $\pi^*(\dual Q^\calc(\DASH))$ to the solid maps in this diagram, we obtain (abbreviating $H^*_\calc$ to $H$):
\[\newcommand{\Times}{{\sqcup}}
\mathclap{\xymatrix@R=4mm{
(H\calX)^{\smashcoprod 2}
\ar@{->}@/_1em/[rrrrd];[]_-(.1){\xi_{\calC\HAlg}}
\\%r3c5
\ar@{->>}[u];[]
\ar@/_1.3em/[rrrr];[]_-(.91){\overline{\xi}_{\calC\HAlg}}
(H\calX)^{\sqcup2}\ar[r];[]^-{(\textup{id},\textup{id})}
&%r4c1
(H\calX)^{\sqcup2}\Times(H\calX)^{\sqcup2}\ar[r];[]^-{d_0\Times d_0\Times \phi_s}
%\ar[ddd];[]
&%r4c2
(HV)^{\sqcup2}\Times H\calX
\ar[r];[]^-{\phi_{s+1}\Times d_0}
%\ar[ddd];[]
&%r4c3
HV\Times HV\ar[r];[]^-{\phi_{s+1}}
%\ar[dd];[]_-{\beta}
&%r4c4
HV\\%r4c5
H\calX\times H\calX\ar@{ >->}[u];[]
}}\]
One observes that the horizontal composite is the very definition of $\overline{\xi}_{\calC\HAlg}$. We know from \S??? that $\overline{\xi}_{\calC\HAlg}$ factors through the smash coproduct, which is how we were able to fill in the dashed arrow on cohomology.

In order to obtain a map $\xi^\star_\textup{res}$, it is enough to check that the composite $c(\calX\sqcup \calX)\to  V$ is null. However, as $V$ is a GEM, a map into $V$ is null if and only if it is zero on cohomology. We have just stated that the map $\overline{\xi}_{\calC\HAlg}$ factors through $(H\calX)^{\smashcoprod 2}$, which is to say that the composite $HV\to H\calX\times H\calX$ is zero.
\end{proof}
This map $\overline{\xi}_\textup{res}^\star$ is very rich, but it will be important to note that postcomposition with $\epsilon$ destroys much of that richness. That is, reading off the dotted portion of the above commuting diagram:
\begin{lem}\label{lemma: epsilon is destructive}
The map $\epsilon\circ \overline{\xi}_\textup{res}^\star$ equals the following sum in $\hom_{s\vect{}{}}  (c(\calX^2), \overline{V})$:
\[\bigl(\overline{v}\circ c(\textup{add})\circ c(\epsilon^2)\bigr)+\bigl(\overline{v}\circ\pi_1\circ\epsilon\bigr)+\bigl( \overline{v}\circ\pi_2\circ\epsilon\bigr),\]
where the $\pi_i$ are the two projections $\calX^{\times2}\to \calX$.
\end{lem}

\subsection{A three-cell complex with non-trivial product}\label{three cell complex}
Fix $t,t'\geq1$. There is a map $F^{\calc}\mathbb{K}_{t+t'}\to F^{\calc}\mathbb{K}_t\sqcup F^{\calc}\mathbb{K}_{t'}$ sending the fundamental class $z_{t+t'}$ to the shuffle product of the two fundamental classes in the codomain:
\[z_{t+t'}\mapsto \mu(\nabla(z_t\otimes z_{t'})),\]
where $\mu$ is the structural pairing in $\calc$. Consider the complex $J_{t,t'}$ formed as the pushout:
\[\xymatrix@R=4mm{
F^{\calc}\mathbb{K}_{t+t'}\ar[r]
\ar@{ >->}[d]
&%r1c1
F^{\calc}\mathbb{K}_t\sqcup F^{\calc}\mathbb{K}_{t'}\ar@{ >->}[d]
\\%r1c2
F^{\calc}C\mathbb{K}_{t+t'}\ar[r]
&%r2c1
J_{t,t'}%r2c2
}\]
The left vertical is evidently almost free (and thus a cofibration), and thus constructed map $F^{\calc}\mathbb{K}_t\sqcup F^{\calc}\mathbb{K}_{t'}\to J_{t,t'}$ is an almost free map ([ref miller]).
The generating subspace $V_{t+t'+1}\subset (J_{t,t'})_{t+t'+1}$ has a $t+t'+1$-dimensional generator $h_{t,t'}$, the image of the cone class $h$ in $(\mathbb{K}_{t+t'})_{t+t'+1}$ (c.f.\ \S\ref{The Dold-Kan correspondence}). Moreover, the object $J_{t,t'}$ is cofibrant, and $h_{t,t'}$ becomes a cycle in $Q^\calc J_{t,t'}$: $d_ih_{t,t'}=0$ for $i\geq1$, and $d_0h_{t,t'}:=z_{t+t'}$, which we  have identified with the decomposable element $\mu(\nabla(z_t\otimes z_{t'}))$ in passing to the pushout.

The homology long exact sequence shows that $H_*J_{t,t'}$ is three-dimensional, containing classes $ z_t$, $ z_{t'}$ and $h_{t,t'}$.
\begin{prop}\label{prop on three cell}
Under $\Delta:H_*^{\calc}J_{t,t'}\to (S^2H_*^{\calc}J_{t,t'})_{*-1}$, $h_{t,t'}\mapsto\imath_t\otimes\imath_{t'}+\imath_{t'}\otimes\imath_t$. \textbf{other ops?}
\end{prop}
\begin{proof}
The representative $g$ has the property that $d_0(g)=\mu(\Nabla(\imath_t\otimes\imath_{t'}))$ and $d_i(g)=0$ for $i>0$. By lemma \ref{psi is basically the quadratic part} and the description of $\quadratic_\calc$ in \S\ref{quadratic part section}:
\[\psi_\calc(g)=
\quadratic_\calc(\mu(\Nabla(\imath_t\otimes\imath_{t'})))=
\trace(\Nabla(\imath_t\otimes\imath_{t'})))\in (S^2Q^\calc J_{t,t'})_{t+t'}.\qedhere\]
\end{proof}
%Now suppose that $L$ is a (not necessarily zero-square) algebra, and that $\alpha,\beta$ are homotopy classes in $\pi_n (L)$ and $\pi_m (L)$ respectively. Then we will be able to use the complex $J_{n,m}$ to construct a homology class $g_{\alpha,\beta}\in H_{n+m+1}(L)$ whose diagonal is $\alpha\otimes\beta+\beta\otimes\alpha$ [probably just when $L$ is zero square, using the map $L\times L\to L$. Else, get a class in $H(L^2)$ with diagonal $\overline{\alpha}\otimes\overline{\overline{\beta}}+ \overline{\overline{\beta}}\otimes\overline{\alpha}$]. Firstly, note that there is a commuting diagram
%\[\xymatrix@R=4mm{
%0\ar[r]
%\ar@{ >->}[d]
%&%r1c1
%0\ar[d]
%\\%r1c2
%S^n\sqcup S^m\ar[r]^-{\alpha\sqcup\beta}
%&%r2c1
%L^{\times2}%r2c2
%}\]
%which furnishes $c_1(L^{\times2})$ with a map from $S^n\sqcup S^m$. Next, there is a diagram %\newdir{ >}{{}*!/-7pt/\dir{>}}
%\[\xymatrix@R=4mm{
%S^{n+m}\ar@{->}[r]_-{\textup{EZ}^\mu(\imath_n,\imath_m)}
%\ar@{ >->}[d]
%&%r1c1
%S^n\sqcup S^m\ar@{ >->}[d]
%\ar[r]
%&c_1(L^{\times2})\ar[d]
%\\%r1c2
%CS^{n+m}\ar[r]\ar@/_1em/[rr]_-{0}
%&%r2c1
%C{n,m}\ar@{..>}[r]
%%r2c2
%&L^{\times2}}\]
%where the outer rectangle commutes since in $L^{\times2}$, $(a,0)\cdot(0,b)=(0,0)$. This gives a map $C_{n,m}\to c_2(L^{\times2})$ whose restriction to the subcomplex $S^n\sqcup S^m$ gives the homotopy classes $\alpha$ and $\beta$. The image $g_{\alpha,\beta}$ of the class $g\in(C_{n,m})_{n+m+1}$ is then an explicit representative for the homology class sought.

\subsection{A chain level construction of $j^*_{\calC\HAlg}$}\label{chain level construction of j}
We can use this cofibration to construct, at the chain level, the image under 
\[j^*_{\calC\HAlg}:\textup{Pr}_t(HY)\otimes \textup{Pr}_{t'}(HZ)\to \textup{Pr}_{t+t'+1}(HY\smashprod HZ)\]
of a tensor product $\alpha\otimes\beta$ of \emph{spherical} homology classes. Abbreviating $H_*^\calc$ to $H$:
\begin{prop}
There is a function
\[\overline{F}:\hom_{s\algs}(F^{\calc}\mathbb{K}_t,Y)\times\hom_{s\algs}(F^{\calc}\mathbb{K}_{t'},Z)\to \hom_{s\algs}(J_{t,t'},c(Y\times Z)),\]natural in $Y,Z\in s\algs$,
such that the function
\[F:\hom_{s\algs}(F^{\calc}\mathbb{K}_t,Y)\times\hom_{s\algs}(F^{\calc}\mathbb{K}_{t'},Z)\to \pi_{t+t'+1}(Q^\calc  c(Y\times Z))=:H^\calc_{t+t'+1}(Y\times Z)\]
defined by $F(\alpha,\beta):=H^\calc_*(\overline{F}(\alpha,\beta))(h_{t,t'})$ makes the following diagram commute:
\[\mathclap{\xymatrix@R=4mm{
%(H(Y\times Z)^{\otimes2})_{t+t'}\ar[d]^-{\textup{id}+\tau}
%&
%\ar@/_2em/@<-3ex>[dd]_-{(\textup{id}+T)\circ \textup{hur}^{\otimes2}}
%\ar[ddr]|(.34){(\textup{id}+T)\circ \textup{hur}^{\otimes2}}
%\ar[l]_-{\textup{hur}^{\otimes2}}
%\hom_{s\algs}(F^{\calc}\mathbb{K}_t,Y)\times\hom_{s\algs}(F^{\calc}\mathbb{K}_{t'},Z)
{\genfrac{}{}{0pt}{}{\hom_{s\algs}(F^{\calc}\mathbb{K}_t,Y)\times}{\ \hom_{s\algs}(F^{\calc}\mathbb{K}_{t'},Z)\phantom{\times}\!}\!\!}
\ar[d]^-{F}
\ar[r]^-{\textup{hur}^{\otimes2}}
&%r1c1
\ar[ddl]_(.34){\Id+T}
\textup{Pr}(HY)_t\otimes\textup{Pr}(HZ)_{t'}\ar[r]^-{j^*_{\calC\HAlg}}
&%r1c2
\textup{Pr}(HY\smashprod HZ)_{t+t'+1}\ar@{ >->}[d]
\\%r1c5
%(H(Y\times Z)^{\otimes2})_{t+t'}
%&
H_{t+t'+1}(Y\times Z)\ar[r]%\ar[l]_-{\Delta}
\ar[d]^-{\Delta}
&%r2c1
(HY\times HZ)_{t+t'+1}\ar@{->>}[r]
\ar[d]^-{\Delta}
&%r2c2
(HY\smashprod  HZ)_{t+t'+1}
\\
%&
(S^2H(Y\times Z))_{t+t'}\ar[r]
&(S^2(HY\times HZ))_{t+t'}
}}\]
\end{prop}
\noindent The map $(\Id+T)$ is the tensor product of the maps
\[\textup{Pr}(H^\calc_*Y)_t\subset H^\calc_tY\to H^\calc_t(Y\times Z)\ \textup{and}\ \textup{Pr}(H^\calc_*Z)_t\subset H^\calc_tZ\to H^\calc_t(Y\times Z)\]
followed by $(\Id+T):(H^\calc_*(Y\times Z))^{\otimes 2}\to S^2H^\calc_*(Y\times Z)$.
%The $\calc$-$H_*$-coalgebra $HY\times HZ$ is calculated as the direct sum $HY\oplus HZ$, and
%\[(HY\oplus HZ)^{\otimes 2}=(HY)^{\otimes 2}\oplus (HY\otimes HZ\oplus HZ\otimes HY)\oplus (HZ)^{\otimes 2}.\]
\begin{proof}
The value of $\overline{F}$ on $(\alpha,\beta)$ is defined as follows. Construct canonical lifts (c.f.\ \S\ref{Cofibrant replacement via the small object argument}):
\[\vcenter{\xymatrix@R=4mm{
%0\ar[r]
&%r1c1
c_1(Y\times Z)\ar[d]
\\%r1c2
F^{\calc}\mathbb{K}_t\ar[ru]^-{\widetilde{(\alpha,0)}}\ar[r]_-{(\alpha,0)}
&%r2c1
Y\times Z%r2c2
}}
\textup{\quad and\quad }
\vcenter{\xymatrix@R=4mm{
%0\ar[r]
&%r1c1
c_1(Y\times Z)\ar[d]
\\%r1c2
F^{\calc}\mathbb{K}_t\ar[ru]^-{\widetilde{(0,\beta)}}\ar[r]_-{(0,\beta)}
&%r2c1
Y\times Z%r2c2
}}\]
There is a commuting diagram (since in $Y\times Z$, products \textbf{(or brackets)} of $(a,0)$ with $(0,b)$ vanish):
\[\xymatrix@R=4mm@C=15mm{
F^{\calc}\mathbb{K}_{\frakt}\ar@{->}[r]_-{\textup{EZ}^\mu(\imath_t,\imath_{t'})}
\ar@{ >->}[d]
&%r1c1
F^{\calc}\mathbb{K}_t\sqcup F^{\calc}\mathbb{K}_{t'}\ar@{ >->}[d]
\ar[r]^{\widetilde{(\alpha,0)}\sqcup\widetilde{(0,\beta)}}
&c_1(Y\times Z)\ar[d]
\\%r1c2
CF^{\calc}\mathbb{K}_{\frakt}\ar[r]\ar@/_1em/[rr]_-{0}
&%r2c1
J_{t,t'}\ar@{-->}[r]
%r2c2
&Y\times Z}\]
Corresponding to the right square is a map $J_{t,t'}\to c_2(Y\times Z)$, and the composite with the cofibration $c_2(Y\times Z)\to c(Y\times Z)$ is $\overline{F}(\alpha,\beta)$. This function $\overline{F}$ is evidently natural in $Y$ and $Z$, and so then is $F$.

The diagram consists of a square, a triangle and a hexagon.
The square commutes as the horizontals are maps in $\calC\HCoalg$, and we can see that the triangle commutes because we understand the $\calC\HCoalg$ structure of $H_*(J_{t,t'})$ (and $H^\calc_*(\overline{F}(\alpha,\beta))$ is a map of $\calc$-$H_*$-coalgebras). As all of the maps in the hexagon are natural, we may check that it commutes on the universal example alone: 
\[(\imath_t,\imath_{t'})\in\hom_{s\algs}(F^{\calc}\mathbb{K}_t,F^{\calc}\mathbb{K}_t)\times\hom_{s\algs}(F^{\calc}\mathbb{K}_{t'},F^{\calc}\mathbb{K}_{t'}).\]
 That is, it is enough to check that the following hexagon, with a one element set at the top left entry, commutes:
\[\mathclap{\xymatrix@R=4mm{
\{(\imath_t,\imath_{t'})\}\ar[d]^-{F}
\ar[r]^-{\textup{hur}^{\otimes2}}
&%r1c1
\textup{Pr}_t(HF^{\calc}\mathbb{K}_t)\otimes\textup{Pr}_{t'}(HF^{\calc}\mathbb{K}_{t'})\ar[r]^-{j^*_{\calC\HAlg}}
&%r1c2
\textup{Pr}_{\frakt+1}(HF^{\calc}\mathbb{K}_t\smashprod HF^{\calc}\mathbb{K}_{t'})\ar@{ >->}[d]^-{\textup{inc}}
\\%r1c5
H_{\frakt+1}(F^{\calc}\mathbb{K}_t\times F^{\calc}\mathbb{K}_{t'})\ar[r]^-{r}
&%r2c1
(HF^{\calc}\mathbb{K}_t\times HF^{\calc}\mathbb{K}_{t'})_{\frakt+1}\ar@{->>}[r]^-{\textup{proj}}
&%r2c2
(HF^{\calc}\mathbb{K}_t\smashprod  HF^{\calc}\mathbb{K}_{t'})_{\frakt+1}
}}\]
In this diagram, $j^*_{\calC\HAlg}$ and $\textup{inc}$ are isomorphisms of 1-dimensional vector spaces, so it is enough to check that $r(F(\imath_t,\imath_{t'}))$ does not lie in the kernel of $\textup{proj}$, i.e.:
\[r(F(\imath_t,\imath_{t'}))\notin(HF^{\calc}\mathbb{K}_t\sqcup HF^{\calc}\mathbb{K}_{t'})_{\frakt+1} = H_{\frakt+1}F^{\calc}\mathbb{K}_t\oplus H_{\frakt+1}F^{\calc}\mathbb{K}_{t'}=0,\]
yet $\Delta(r(F(\imath_t,\imath_{t'})))=\imath_t \otimes\imath_{t'}+\imath_{t'}\otimes\imath_{t}\neq 0$, using the commuting square and triangle already established.
\end{proof}
We record here a useful calculation:
\begin{lem}\label{handy lemma for conn hom}
The composite 
\[F^{\calc}\mathbb{K}_t\sqcup F^{\calc}\mathbb{K}_{t'}\to J_{t,t'}\overset{\overline{F}(\alpha,\beta)}{\to}c(X\times X)\overset{c(\textup{add}\circ(\epsilon^2))}{\to}X\]
equals $\widetilde{\epsilon\alpha}\sqcup \widetilde{\epsilon\beta}$. In particular, $(c(\textup{add}\circ(\epsilon^2))\circ \overline{F}(\alpha,\beta))(\mu\nabla(\imath_t\otimes \imath_{t'}))=\overline{\mu}(\nabla(\widetilde{\epsilon\alpha} \otimes\widetilde{\epsilon\beta}))$.% under the composite $J_{t,t'}\to X$.
\end{lem}
\begin{proof}We may calculate the restrictions to the two summands individually, and by symmetry, we need only consider:
%Using naturality of $\overline{F}$ and the fact that $h=c(\textup{add}\circ(\epsilon\times\epsilon))$:
\newcommand{\Times}{{\cdot}}
\[
\xymatrix@R=4mm{
F^{\calc}\mathbb{K}_t\ar@{ >->}[r]&
J_{t,t'}\ar[r]^-{\overline{F}(\alpha,\beta)}
&%r2c1
c(X^\fraks\Times X^\fraks)\ar[r]^-{c(\epsilon\times\epsilon)}
&%r2c2
c(\overline{V}X^{\fraks-1}\Times \overline{V}X^{\fraks-1})\ar[r]^-{c(\textup{add})}
&%r2c3
c(\overline{V}X^{\fraks-1})\makebox[0cm][l]{\,$=X^\fraks.$}\phantom{=}%r2c4
}\]
The composite $J_{t,t'}\to c(\overline{V}X^{\fraks-1}\Times \overline{V}X^{\fraks-1})$ equals $\overline{F}(\epsilon\alpha,\epsilon\beta)$, due to the naturality of $\overline{F}$. By definition of $\overline{F}$, the composite $F^{\calc}\mathbb{K}_t\to c(\overline{V}X^{\fraks-1}\Times \overline{V}X^{\fraks-1})$ equals $\widetilde{(\epsilon\alpha,0)}$. The naturality of the operation $\alpha\mapsto\widetilde{\alpha} $ of \S\ref{Cofibrant replacement via the small object argument} finishes the proof.
\end{proof}
\subsection{A two-cell complex with non-trivial $P^i$ operation}\label{two-cell complex for the deltas}
In this section, we give a construction of a two-cell complex whose cohomology has a $P^i$ connecting the two cells.
Fix $t,i$ with  $2\leq i \leq t$. There is a map $F^{\calc}\mathbb{K}_{t+i}\to F^{\calc}\mathbb{K}_t$ defined by
\[z_{t+i}\mapsto \mu(\nabla_{t-i}(z_t\otimes z_t)),\]
where $\mu$ is the structural pairing in $\calc$. Consider the complex $\Theta_{t,i}$ formed as the pushout:
\[\xymatrix@R=4mm{
F^{\calc}\mathbb{K}_{t+i}\ar[r]
\ar@{ >->}[d]
&%r1c1
F^{\calc}\mathbb{K}_t\ar@{ >->}[d]
\\%r1c2
F^{\calc}C\mathbb{K}_{t+i}\ar[r]
&%r2c1
\Theta_{t,i}%r2c2
}\]
By the same observations as made in \S\ref{three cell complex}, this map is a cofibration, and $H_*^{\calc}\Theta_{t,i}$ has cohomology spanned by $\imath_t$ and $h_{t,i}$ in dimension $t+i+1$. For dimension reasons, $\imath_t$ is primitive. On the other hand:
\begin{prop}\label{prop on two cell delta}
In $H^*_{\calc}\Theta_{t,i}$, $P^i\imath_t^*=h_{t,i}^*$.
\end{prop}
\begin{proof}
We will calculate the action of $(P^j)^*$ and $\Delta$ on $h_{t,i}$. By the same methods  as in the proof of proposition \ref{prop on three cell}:
\[\psi_\calc(g)=\trace(\Nabla_{t-i}(\imath_t\otimes\imath_t)))\in (S^2Q^\calc \Theta_{t,i})_{t+i},\]
which represents $\sigma_i^\textup{ext}\imath_t$. so that the defining equation
\[(\psi_\calc)_*(h_{t,i})=\textstyle\sum_{j}\pi_*(1+T)(y_j\otimes z_j)+\sum_k\sigma_k((P^k)^*h_{t,i})\]
degenerates to $\sigma_i((P^k)^*h_{t,i})=\sigma_i^\textup{ext}\imath_t$.
\end{proof}

\subsection{A chain level construction of $\theta^*_i$}
\begin{prop}\label{propOnKoszulDelta}
For $2\leq i\leq t$, there is a function
\[\overline{G}:\hom_{s\vect{}{}}(S^t,W)\to \hom_{s\algs}(\Theta_{t,i},cKW),\]
natural in $W\in s\vect{}{}$, and satisfying $\overline{G}(\alpha)(\imath_t)=\widetilde{\alpha}$, such that the function
\[G:\hom_{s\algs}(S^t,W)\to \pi_{t+i+1}(QcKW)=:H_{t+i+1}(KW)\]
defined by $G(\alpha):=H_*(\overline{G}(\alpha))(h)$ equals the following composite whenever $2\leq i<t$:
\[\hom_{s\vect{}{}}(S^t,W)\epi \pi_t W\overset{\textup{hur}}{\to} H_{t}(KW) \overset{\theta_i^*}{\to}H_{t+i+1}(KW).\]
\end{prop}
%\begin{shaded}\tiny
%\[{{\xymatrix@R=4mm{
%\hom_{s\vect{}{}}(S^t,W)\ar[d]^-{G}
%\ar@{->>}[r]
%&%r1c1
%\pi_t W\ar[d]^-{\textup{hur}}
%\\
%H_{t+i+1}(KW)\ar@{=}[r]^-{\theta_i^*}
%&
%H_{t}(KW)
%}}} \cancel{{\xymatrix@R=4mm{
%\hom_{s\vect{}{}}(S^t,W)\ar[d]^-{G}
%\ar[r]^-{\textup{hur}}
%&%r1c1
%\pi_t W\ar[d]^-{\check P^i\otimes(\DASH)\makebox[0cm][l]{\,\textbf{???? when $i=t$}}}
%\\
%H_{t+i+1}(KW)\ar@{=}[r]
%&%r2c1
%(C\pi V)_{t+i+1}
%}}}\]
%\end{shaded}
\begin{proof}
The value of $\overline{G}$ on $\alpha$ is defined as follows. There is a commuting diagram %(since in $Y\times Z$, products of the form $(a,0)(0,b)$ vanish):
\[\xymatrix@R=4mm@C=15mm{
S^{t+i}\ar@{->}[r]_-{\mu\Nabla_{t-i}(\imath_t\otimes\imath_{t})}
\ar@{ >->}[d]
&%r1c1
S^t\ar@{ >->}[d]
\ar[r]^{\widetilde{\alpha}}
&c_1(KW)\ar[d]
\\%r1c2
CS^{t+i}\ar[r]\ar@/_1em/[rr]_-{0}
&%r2c1
\Theta_{t,i}\ar@{-->}[r]
%r2c2
&KW}\]
Corresponding to the right square is a map $\Theta_{t,i}\to c_2(KW)$, and the composite with the cofibration $c_2(KW)\to c(KW)$ is $\overline{G}(\alpha)$. This function $\overline{G}$ is evidently natural in $W$, and so then is $G$.

The statement about $G$ is natural in $W\in s\vect{}{}$, so may be checked on the universal example $\imath_t\hom_{s\vect{}{}}(\mathbb{K}_t,\mathbb{K}_t)$. As $\overline{G}(\imath)$ is a map of $\algs$-$H_*$-coalgebras: $(P^i)^*G(\imath_t)=\textup{hur}(\imath_t)$; $(P^j)^*G(\imath_t)=0$ for all $j\neq i$ with $2\leq j\leq (t+i)/2$; and $\Delta G(\imath_t)$ vanishes unless $i=t$, in which case it equals $\imath\otimes \imath$. By construction, $G(\imath_t)$ lies in quadratic grading $2$ of the cofree construction:
\[G(\imath_t)\in\quadgrad{2}H_{t+i+1}(K\mathbb{K}_t)=\quadgrad{2}C_{\algs\HCoalg}\{\imath_t\}.\]
These conditions suffice to identify $G(\imath_t)$.
%and it is clear that the only element of quadratic grading $2$
%
%Under the homology coalgebra map $\overline{G}(\alpha)_*:H\Theta_{t,i}\to HKW$, we have $h\mapsto G(\alpha)$ and $\imath_t\mapsto\textup{hur}(\alpha)$. \textbf{Hopefully we can figure out enough about cofree homology coalgebras and their right $P$-action to see that $G(\alpha)$ can only equal the Koszul operation on $\textup{hur}(\alpha)$.} In the non-top case, we have zero coproduct and only $(\DASH)P^i$ firing, while in the top case we have coproduct $\textup{hur}(\alpha)\otimes \textup{hur}(\alpha)$ instead.
\end{proof}





\subsection{$E^1$-level identification of the product and Steenrod operations}
Throughout, I'll continue with the above abbreviations, and write $\frakt=t+t'$. We are going to need the following commutative diagram of vector spaces \textbf{the point being that once we've got it, we will precompose it with the relevant cosimplicial A-W or E-Z map, to get either the product or the Steenrod square, each time in either sense.}:

\begin{prop}
There is a commuting diagram:
\[\mathclap{\xymatrix@R=4mm{
\pi_t(X)\otimes \pi_{t'}(X)\ar[dd]^-{\textup{hur}^{\otimes 2}}
\ar[r]^-{\nabla}
&%r1c1
\pi_{\frakt}(X\smashcoprod  X)\ar[r]^-{\overline{\mu}}
&%r1c2
\pi_{\frakt}(\overline{D}^1X)&%r1c3
\pi_{\frakt+1}(\overline{V}X)
\ar[l]_-{\partial_\textup{conn}}
\\%r1c4
&%r2c1
&%r2c2
\pi_{\frakt}(D^1X)\ar@{-}[u]^-{\textup{zig-zag}}_-{\cong}
&%r2c3
\pi_{\frakt+1}(VX)\ar[u]_-{\cong}
\ar[l]_-{\partial_\textup{conn}}
\ar[d]^-{\cong}
\\%r2c4
\textup{Pr}_t(HX)\otimes \textup{Pr}_{t'}(HX)\ar[r]^-{j^*_{\algs\PiAlg}}
&%r3c1
\textup{Pr}_{\frakt+1}(HX\smashprod HX)\ar[rr]^-{\xi^*_{\calC\HAlg}}
&%r3c2
&%r3c3
\textup{Pr}_{\frakt+1}(HVX)%r3c4
}}\]
\end{prop}
\begin{proof}
It will help to modify and augment this diagram a little. Indeed, for each cardinality one subset $\{(\alpha,\beta)\}\subset \hom_{s\algs}(S^t,X)\times\hom_{s\algs}(S^{t'},X)$, there is a diagram:
\[\mathclap{\xymatrix@R=4mm{
\{(\alpha,\beta)\}
%\pi_t(X)\otimes \pi_{t'}(X)
\ar@{~>}[dd]_-{j\circ(\textup{hur}^{\otimes 2})}
\ar[r]^-{\nabla}
\ar@{..>}[ddddr]^-(.2){F}
&%r1c1
\pi_{\frakt}(X\smashcoprod  X)\ar[r]^-{\overline{\mu}_*}
&%r1c2
\pi_{\frakt}(\overline{D}^1X)&%r1c3
\pi_{\frakt+1}(\overline{V}X)\ar[l]_-{\partial_\textup{conn}}
\ar@{~}[dd]^-{\textup{zig-zag}}_-{\cong}
\\%r1c4
&%r2c1
&%r2c2
\pi_{\frakt+1}(Q\overline{V}X)\ar@{..>}[ur]_-{=}&%r2c3
%\pi_{\frakt+1}(X^{\fraks+1})\ar[u]_-{\cong}
%\ar[d]^-{\cong}
\\%r2c4
%\textup{Pr}_t(HX^{\fraks})\otimes \textup{Pr}_{t'}(HX^{\fraks})\ar[r]^-{j}
%r3c1
\textup{Pr}_{\frakt+1}(HX\smashprod HX)
\ar@{~>}[rrr]^-{\xi_{\calC\HCoalg}}
\ar@{ >->}[d]
%\ar@{ >->}[dr]
&
&%r3c2
&%r3c3
\textup{Pr}_{\frakt+1}(HVX)\ar@{ >->}[d]
\\%r3c4
(HX\smashprod HX)_{\frakt+1}\ar@{=}[r]
&\pi_{\frakt+1}Q(\textup{cof})
\ar[r]^-{\pi_*(Q\xi_\textup{res})}
&
\pi_{\frakt+1}QVX\ar@{=}[r]
\ar@{..>}[uu]_-(.75){\pi_*(Q\epsilon)}
&
H_{\frakt+1}VX
\\
(HX\times HX)_{\frakt+1}\ar@{=}[r]\ar@{->>}[u]
&\pi_{\frakt+1}Qc(X\times X)\ar@{->>}[u]
\ar@{..>}[ur]_-(.6){\pi_*(Q\overline{\xi}_\textup{res})}
}}\]
Although all of the arrows in this modified diagram have already been defined, we've decorated some of them for emphasis.  It will be enough to check that each of these modified diagrams commutes, since the collection of such $(\alpha,\beta)$ will exhaust all of the pure tensors in $\pi_t(X)\otimes \pi_{t'}(X)$. What we need to prove is that the large rectangle consisting of wavy and solid arrows commutes.

The composite of the dotted maps equals the composite of the wavy maps, by results above.  That is, proposition ??? states that the two composites $\{(\alpha,\beta)\}\to (HX\smashprod HX)_{\frakt+1}$ are equal. The content of proposition \ref{res xi induces xi} is that the small triangle and square at the bottom of the diagram each commute, and the two composites $\textup{Pr}_{\frakt+1}(HX\smashprod HX)\to H_{\frakt+1}VX$ are equal. Finally, the two composites $\textup{Pr}_{\frakt+1}(HVX)\to \pi_{\frakt+1}(\overline{V}X)$ are equal, by lemma \ref{hurewicz is a section}.

Thus the image of $(\alpha,\beta)$ under either the wavy or the dotted composite equals the image of $h_{t,t'}\in\pi_{\frakt+1} (QJ_{t,t'})$ under the composite
\[QJ_{t,t'}\overset{Q\overline{F}(\alpha,\beta)}{\to}Qc(X\times X)\overset{Q\overline{\xi}_\textup{res}}{\to}QVX\overset{Q\epsilon}{\to}Q\overline{V}X=\overline{V}X,\]
which, by lemma \ref{lemma: epsilon is destructive},  decomposes as the sum of the three maps $\overline{v}\circ c(\textup{add})\circ c(\epsilon^2)\circ\overline{F}(\alpha,\beta)$ and $\overline{v}\circ\pi_i\circ\epsilon\circ\overline{F}(\alpha,\beta)$ for $i=1$ and $2$.
%Indeed, write `$\textup{dot}$' for the composite of the dotted maps, and write $h_{t,t'}\in QJ_{t,t'}$ for the a representative of $h_{t,t'}\in\pi_{\frakt+1}(QJ_{t,t'})$. A representative for $\textup{dot}(\alpha,\beta)$ is given by the image of $h_{t,t'}$ under the composite
%\[QJ_{t,t'}\overset{Q\overline{F}(\alpha,\beta)}{\to}Qc(X^\fraks\times X^\fraks)\overset{Q\overline{\xi}_\textup{res}}{\to}QX^{\fraks+1}\overset{Q\epsilon}{\to}QX^{\fraks+1}_\textup{fl}=X^{\fraks+1}_\textup{fl},\]
%which is the sum of the images of $h_{t,t'}$ under the three composites
%\[\xymatrix@R=4mm{
%&
%&%r1c1
%QcA\ar@/^.55em/[dr]^-{Q\epsilon}
%&%r1c2
%&%r1c3
%&%r1c4
%\\%r1c5
%QJ_{t,t'}
%\ar[r]^-{Q\overline{F}(\alpha,\beta)}
%&Qc(A^2)\ar[rr]^-{Qh}
%\ar@/^.5em/[ur]^-{Qc(\pi_1)}
%\ar@/_.5em/[dr]_-{Qc(\pi_2)}
%&%r2c1
%&%r2c2
%QA\ar[r]^-{Q\eta}
%&%r2c3
%QK_+\makebox[0cm][l]{$\,=K_+$}%\ar@{=}[r]
%\\%r2c5
%&
%&%r3c1
%QcA\ar@/_.55em/[ur]_-{Q\epsilon}&%r3c2
%&%r3c3
%}\]
The composite $\pi_1\circ \epsilon\circ\overline{F}(\alpha,\beta):J_{t,t'}\to X$, by construction of $\overline{F}$, is the (dashed) map out of the pushout in the diagram:
\[\xymatrix@R=4mm@C=15mm{
S^{\frakt}\ar@{->}[r]_-{\mu\nabla(\imath_t\otimes\imath_{t'})}
\ar@{ >->}[d]
&%r1c1
S^t\sqcup S^{t'}\ar@{ >->}[d]
\ar[dr]^{\alpha\sqcup0}
\\%r1c2
CS^{\frakt}\ar[r]\ar@/_1em/[rr]_-{0}
&%r2c1
J_{t,t'}\ar@{-->}[r]
%r2c2
&X}\]
Now $h_{t,t'}$ is in the image of the map $CS^{\frakt}\to J_{t,t'}$, and so maps to zero under the dashed map to $X$. Similarly, the composite $\pi_2\circ \epsilon\circ\overline{F}(\alpha,\beta)$ vanishes on $h_{t,t'}$. Thus, the image of $(\alpha,\beta)$ under the dotted composite is represented by
\[A:=(\overline{v}\circ c(\textup{add}\circ \epsilon^2)\circ\overline{F}(\alpha,\beta))(h_{t,t'}).\]
Consider
%We must show that $\partial_\textup{conn}(\textup{dot}(\alpha,\beta))=\overline{\mu}_*(\nabla(\alpha\otimes\beta))$.
%All of the previous discussion together has shown $\textup{dot}(\alpha,\beta)$ is the image in $Q\overline{V}X=\overline{V}X$ of the cycle $h_{t,t'}\in QJ_{t,t'}$ under the top row of 
the following commuting diagram:
\[\xymatrix@R=4mm@C=12mm{
\makebox[0cm][r]{$h_{t,t'}\in\,$}QJ_{t,t'}\ar[r]^-{Q\overline{F}(\alpha,\beta)}
&%r1c2
Qc(X^2)\ar[r]^-{Qc(\textup{add}\circ\epsilon^2)}
&%r1c3
QX\ar[r]^-{Q\overline{v}}
&%r1c4
Q\overline{V}X\makebox[0cm][l]{$\,\ni A$}\\%r1c5
\makebox[0cm][r]{$h_{t,t'}\in\,$}
J_{t,t'}\ar[r]^-{\overline{F}(\alpha,\beta)}
\ar@{->>}[u]
\ar[d]_-{d_0}
&%r1c2
c(X^2)\ar[r]^-{c(\textup{add}\circ\epsilon^2)}
\ar@{->>}[u]
\ar[d]_-{d_0}
&%r1c3
X\ar@{->>}[r]^-{\overline{v}}
\ar@{->>}[u]
\ar[d]^-{d_0}
&%r1c4
\overline{V}X\makebox[0cm][l]{$\,\ni A$}\ar@{=}[u]
\\%r2c5
\makebox[0cm][r]{$\mu\nabla(\imath_t\otimes\imath_{t'})\in\,$}J_{t,t'}\ar[r]^-{\overline{F}(\alpha,\beta)}
&%r1c2
c(X^2)\ar[r]^-{c(\textup{add}\circ\epsilon^2)}
&%r1c3
X\makebox[0cm][l]{$\,\ni\overline{\mu}(\nabla(\widetilde{\epsilon\alpha} \otimes\widetilde{\epsilon\beta}))$}
&%r1c4
%\\%r3c5
%\makebox[0cm][r]{$\mu\nabla(\imath_t\otimes\imath_{t'})\in\,$}S^t\sqcup S^{t'}\ar[urr]_-{\widetilde{\epsilon\alpha} \sqcup \widetilde{\epsilon\beta}}
%\ar@{ >->}[u]
%&%r4c2
%&%r4c3
%&%r4c4
%%r4c5
}\]
The element $h_{t,t'}\in N_{\frakt+1}(J_{t,t'})$ may been used to populate the whole diagram as shown. To understand the images of $h_{t,t'}$ at either end of the bottom row, note that $d_0h_{t,t'}=\mu\nabla(\imath_t\otimes \imath_{t'})$ by construction, and lemma \ref{handy lemma for conn hom} states that under the maps of bottom row, $\mu\nabla(\imath_t\otimes \imath_{t'})$ maps to $\overline{\mu}(\nabla(\widetilde{\epsilon\alpha} \otimes\widetilde{\epsilon\beta}))$.

%As $\alpha\sim \widetilde{\epsilon\alpha}$ and $\beta\sim \widetilde{\epsilon\beta}$, it is enough to prove that
The data in the bottom right corner of this diagram demonstrates that $\partial_\textup{conn}\overline{A}\in\pi_\frakt(\overline{D}^1X)$ is represented by ${\overline{\mu}(\nabla(\widetilde{\epsilon\alpha} \otimes\widetilde{\epsilon\beta}))}$,
which suffices, as $\alpha\sim \widetilde{\epsilon\alpha}$ and $\beta\sim \widetilde{\epsilon\beta}$.
%%%%%%
%%%%%%
%%%%%%By the expression $(\overline{v}\circ c(\textup{add}\circ \epsilon^2)\circ\overline{F}(\alpha,\beta))(h_{t,t'})$ we have meant the image of $h_{t,t'}$ under the composite
%%%%%%\[J_{t,t'}\to QJ_{t,t'} \to Qc(X^2)\to QX\to Q\overline{V}X=\overline{V}X.\]
%%%%%%The diagram shows that this is the image under $\overline{v}$ of the element $E=c(\textup{add}\circ\epsilon^2)\overline{F}(\alpha,\beta)(h_{t,t'})\in N_{\frakt+1}X$. Thus, the value of the required boundary homomorphism is represented by $d_0E$, which necessarily lies in the subspace $\overline{D}^1X$ of $X$.
%%%%%%
%%%%%%
%%%%%%
%%%%%%Moreover, at the right of this diagram we see exactly the maps needed in order to construct the connecting homomorphism for the short exact sequence defining $\overline{D}^1X^\fraks$.
%%%%%%%\[0\to\overline{D}^1X^\fraks\to X^\fraks\overset{\overline{v}}{\to} \overline{V}X^\fraks\to 0,\]
%%%%%%This confirms that $\partial_\textup{conn}(\textup{dot}(\alpha,\beta))$ is the image under $\widetilde{\epsilon\alpha} \sqcup \widetilde{\epsilon\beta}$ of $\mu\nabla(\imath_t\otimes\imath_{t'})$.
\end{proof}

\pagebreak
\subsection{$E^1$-level identification of the $\delta$-operations}
\begin{prop}
Whenever $2\leq i<t$ there is a commuting diagram
\[\xymatrix@R=4mm{
\pi_tX\ar[r]^-{\delta_i^\textup{ext}}
\ar[d]^-{\textup{hur}}
&%r1c1
\pi_{t+i}(S_2X)\ar[r]^-{\overline{\mu}}
&
\pi_{t+i}(\overline{D}^1X)
\\%r1c2
H_tX\ar[r]^-{\theta^\star_i}&%r2c1
H_{t+i+1}X\ar@{=}[r]
&
\pi_{t+i+1}(\overline{V}X)
\ar[u]^-{\partial_\textup{conn}}
}\]
\end{prop}
%Now to get to the point, we have a commuting diagram
%\[\xymatrix@R=4mm{
%\hom_{s\algs}(S^t,X)\ar[r]^-{\delta_i}
%\ar[d]^-{\epsilon_*}
%&%r1c1
%\pi_{t+i}(X\smashcoprod_{\Sigma_2} X)\ar[r]^-{\mu}
%&
%\pi_{t+i}(\overline{D}^1X)
%\\%r1c2
%\hom_{s\vect{}{}}(S^t,\overline{X})\ar[r]^-{G}&%r2c1
%\pi_{t+i+1}(Qc\overline{X})\ar@{=}[r]
%&
%\pi_{t+i+1}(\overline{V}X)
%\ar[u]^-{\partial_\textup{conn}}
%}\]
\noindent \textbf{the point:} The lower map $\pi_t(X)\to \pi_{t+i+1}(VX)$ is the Koszul $\delta_i$ operation, by \ref{propOnKoszulDelta}, while the top path, using the `inverse' of $\partial_\textup{conn}$ is the lifter spectral sequence $\delta^\textup{v}_i$ operation.
\begin{proof}
Choose a representative $\alpha\in\hom_{s\calc}(S^t, X)$. Then, setting $W=\overline{X}$ in proposition \ref{propOnKoszulDelta}, we obtain a map $\overline{G}(\epsilon\alpha):\Theta_{t,i}\to X$ such that $(\theta_i^\star\circ\textup{hur})(\alpha)$ is represented by
%\[\overline{G}(\epsilon\alpha)(h_{t,i})\in N_{t+i+1}(QX),\]
%or equivalently,
\[(\overline{v}\circ \overline{G}(\epsilon\alpha))(h_{t,i})
\in N_{t+i+1}(\overline{V}).\]
We populate the following commuting diagram using the element $h_{t,i}\in N_{t+i+1}(\Theta_{t,i})$:
%\[\xymatrix@R=10mm@C=30mm@!0{
%\makebox[0cm][r]{$h_{t,i}\in\,$}(\Theta_{t,i})_{t+i+1}\ar[d]_-{d_0}\ar[r]^-{\overline{G}(\epsilon\alpha)}&%r1c1
%X_{t+i+1}\ar[d]_-{d_0}
%\ar[r]^-{\overline{v}}
%&%r1c2
%(\overline{V})_{t+i+1}
%\makebox[0cm][l]{$\,\ni (\eta\circ \overline{G}(\epsilon\alpha))(h_{t,i})$}
%\\%r1c3
%\makebox[0cm][r]{$\mu\Nabla_{t-i}(\imath_t\otimes \imath_t)\in\,$}(\Theta_{t,i})_{t+i}\ar[r]^-{\overline{G}(\epsilon\alpha)}
%&%r2c1
%X_{t+i}\makebox[0cm][l]{$\,\ni\mu\Nabla_{t-i}(\widetilde{\epsilon\alpha}\otimes \widetilde{\epsilon\alpha})$}
%&%r2c2
%}\]
%
\[\xymatrix@R=10mm@C=30mm@!0{
\makebox[0cm][r]{$h_{t,i}\in\,$}N_{t+i+1}\Theta_{t,i}\ar[d]_-{d_0}\ar[r]^-{\overline{G}(\epsilon\alpha)}&%r1c1
N_{t+i+1}X\ar[d]_-{d_0}
\ar[r]^-{\overline{v}}
&%r1c2
N_{t+i+1}(\overline{V})
\makebox[0cm][l]{$\,\ni (\eta\circ \overline{G}(\epsilon\alpha))(h_{t,i})$}
\\%r1c3
\makebox[0cm][r]{$\mu\Nabla_{t-i}(\imath_t\otimes \imath_t)\in\,$}ZN_{t+i}\Theta_{t,i}\ar[r]^-{\overline{G}(\epsilon\alpha)}
&%r2c1
ZN_{t+i}X\makebox[0cm][l]{$\,\ni\mu\Nabla_{t-i}(\widetilde{\epsilon\alpha}\otimes \widetilde{\epsilon\alpha})$}
&%r2c2
}\]
Here, the value of $d_0h_{t,i}$ is known by definition of $\Theta_{t,i}$, 
and the fact that $\overline{G}(\epsilon\alpha)(\imath_t)=\widetilde{\epsilon\alpha}$ (see the prop) allows us to calculate $\overline{G}(\epsilon\alpha)d_0h_{t,i}$. Finally, in order to calculate $\partial_\textup{conn}(\theta_i^\star\circ\textup{hur})(\alpha)$, we find a preimage under $N_{t+i+1}X\overset{\overline{v}}{\to}N_{t+i+1}\overline{V}$ of the representative $(\eta\circ \overline{G}(\epsilon\alpha))(h_{t,i})$, and then apply the differential $d_0$. We may use the preimage $\overline{G}(\epsilon\alpha)$, which maps to $\mu\Nabla_{t-i}(\widetilde{\epsilon\alpha}\otimes \widetilde{\epsilon\alpha})\in N_{t+i}\overline{D}^1X$ under $d_0$. This is homotopic to $\mu\Nabla_{t-i}(\alpha\otimes \alpha)$, which represents $\overline{\mu}\delta_i(\alpha)$.
\end{proof}













\end{Operations on the Bousfield-Kan spectral sequence}



\begin{Koszul complexes}

\section{\textbf{Koszul complexes calculating $\calU(n)$-homology}}
In this section we will discuss the Koszul resolutions that one may use to calculate $H_*^{\calU(n)}X$ for $X$ a (non-simplicial) locally finite object of $\calU(n)$ or $\calw(n)$, using Priddy's technique \cite{PriddyKoszul.pdf}, adapted to an unstable context, as in \cite{CurtisSimplicialHtpy.pdf}.

\subsection{The (co-)Koszul complex}
Write $\Nop_*,N_*^\div,C_*$ for three of the four chain complexes associated with $Q^{\calU(n)}B^{\calU(n)}X$. Each of these complexes admits an increasing filtration, the length filtration, with $F_\ell C_s$ generated by those terms in which there are at most $\ell$ generators appearing in the $s$ free constructions in $C_s= Q^{\calU(n)}B^{\calU(n)}_sX\cong (F^{\calU(n)})^sX$, and the filtrations on $\Nop_*$ and $N^\div_*$ induced by that on $C_*$. Note that $F_{s-1}N^\div_s=0$, so that  $F_{s-1}\Nop_s=0$. Note further that $d(F_sN_s^-)\subseteq F_{s-1}N_{s-1}^-$, which is easily checked on the isomorphic filtered complex $N_*^\div$.

Write $E^r_{\ell,s}$ for the spectral sequence of the filtered complex $\Nop_*X$, so that $E^0_{\ell,p}$ is the associated graded complex. As $F_{s-1} \Nop_s=0$, $E^r_{\ell,s}=0$ for $\ell<s$, and $E^0_{s,s}$ is the subspace $F_s\Nop_s$ of $\Nop_s$. The $d_0$-differential vanishes on $E^0_{s,s}$ since $d(F_sN_s^-)\subseteq F_{s-1}N_{s-1}^-$ (\textbf{?????}). We conclude that $E^1_{s,s}=F_s\Nop_s$, with $d^1$-differential $E^1_{s,s}\to E^1_{s-1,s-1}$ identified with the differential of $\Nop_*$. On the other hand, Priddy shows that $E^1_{\ell,s}=0$  for $\ell>s$. Thus, the groups
\[K_s^{\calU(n)}X:=E^1_{s,s}=d^{-1}(F_{s-1}\Nop_{s-1})=F_s\Nop_s\]
form a subcomplex of $\Nop_*$ whose inclusion is a homotopy equivalence. Rather than determining these groups directly, Priddy's theory works with their duals, $K^*_{\calU(n)}X$, which form a cochain complex with homology $H^*_{\calU(n)}X$. In fact, Priddy's theory shows that the cochain complex $K^*_{\calU(n)}X$, the \emph{co-Koszul complex}, is actually a differential unstable left module over the same operations as its cohomology $H^*_{\calU(n)}X$, and indeed that this unstable module is \emph{free}. More precisely,  $K_0^{\calU(n)}X= X$, and $K^*_{\calU(n)}X$ is free on the  subspace $X^*$.
\begin{prop}\label{the cokoszul complex is free}
Suppose that $n\geq0$, and $X$ is a locally finite object of $\calU(n)$. The chain maps $\widetilde{\theta^i}$ ($\widetilde{\theta_i}$ when $n=0$) on $C_*Q^{\calU(n)}B^{\calU(n)}X$ restrict to the subcomplex $K_*^{\calU(n)}X$, and induce an $\calMv(n+1)$-structure on $K^*_{\calU(n)}X$ which commutes with the differentials. The inclusion $\dual X\cong K^0_{\calU(n)}X\subseteq K^*_{\calU(n)}X$ induces an isomorphism $F_{\calMv(n+1)}(\dual X)\to K^*_{\calU(n)}X$. Moreover, this $\calMv(n+1)$-structure on $K^*_{\calU(n)}$ induces the $\calMv(n+1)$-structure on $H^*_{\calU(n)}X$ of [above]. 
\end{prop}


%\subsection{The co-Koszul complex for $H_*^{\calU(0)}$}
Although it is easier to calculate the co-Koszul complex, we will need to understand the Koszul complex itself, in order to calculate the $\calw(n+1)$-structure of $H_*^{\calU(n)}$. For this, we will introduce a little notation. Firstly, we will return to the convenient bar notation from which the bar construction gets its name, c.f.\ \cite{PriddyKoszul.pdf}. Next, we will write $\produces{J}{I}{\deltaalg}$ \cancel{whenever $J\neq I$} and the element $\delta_J\in\deltaalg$ contains the term $P^I$ when written in the admissible basis. \textbf{Clarify that $\leftarrow$.}

\begin{prop}\label{propDerivedIndTrivialUobject n=0}
Suppose that $X\in\calU(0)$ has homogeneous basis $B$. Then $K_*^{\calU(0)}X$ has basis
\[\left\{\delta(I,b)\ \middle|\ b\in B^t\textup{, $I$ $\delta$-admissible with }\minDimP(I)\leq t\right\},\]
where we define
\[\delta(I,x):=
%\cancel{\left[P^{i_\ell}\middle|\cdots\middle|P^{i_1} \right]x}+
\sum_{\produces{K}{I}{\deltaalg}}\left[P^{k_\ell} \middle|\cdots\middle|P^{k_1} \right]x\textup{ for $x\in X^t$}.\]
%Supposing that $X$ is locally finite, let $B^*\subset X^*$ represent the basis dual to $B$, and denote by $b^*\in B^*$ the element dual to $b\in B$. Then $K^*_{\calU(0)}X$ has basis $\{\delta_I(b^*)\}$, and this basis is dual to the basis just presented.
%
If $X$ is locally finite, there is a basis $\{\delta_I(b^*)\}$ of $K^*_{\calU(0)}X$ constructed using proposition \ref{the cokoszul complex is free}, lemma \ref{basis of element of M(0)} and the basis of $\dual X$ dual to $B$. The bases $\{\delta_I(b^*)\}$ and $\{\delta(I,b)\}$ are dual.

The differential of $K^{\calU(0)}_*X$ is given by the formula:
\[d(\delta(I,x))=\sum_{ \substack{\produces{K}{I}{\deltaalg}\\(k_{\ell},\ldots,k_2){\,\deltaalg\textup{-admis.}}}}\!\!\!\!\!\!\!\!\!\!\!\! \delta((k_{\ell},\ldots,k_2),P^{k_1}x),\]
summing over those $K=(k_{\ell},\ldots,k_1)$ such that $(k_{\ell},\ldots,k_2)$ is $\delta$-admissible, and yet $\produces{K}{I}{\deltaalg}$.
\end{prop}
Note that the class $\delta(I,x)$ is still defined for $x\in X^t$ and $\minDimP(I)>t$, it is simply equal to zero. This definition is well behaved, since $\delta$-Adem relations only decrease $\minDimP$. Indeed, we may further restrict the two sums appearing in this proposition be requiring that $\minDimP(K)\leq t$ in each case, but there is no need. Dually, in the coKoszul complex, the operations $\delta_i$ are \emph{undefined} when out of range!
\begin{proof}\textbf{Need something here showing that this sum is actually finite. Same in next proof, need well definedness.}
Firstly, we may assume that $X$ is locally finite, as any object of $\calU(0)$ is the union of its locally finite subobjects. It is enough to check that $\delta(I,b)$ is in fact a member of $\Nop_*$, not just of $C_*$, as then the collection $\delta(I,b)$ will evidently be the dual basis to the $\delta_I(b^*)$: in the sum defining $\delta(I,b)$, the only $\delta$-admissible sequence $K$ appearing is $K=I$.  % the basis element $\delta_J(b^*_2)\in K^*$ pairs non-trivially with $\delta(I,b_1)\in K_*$ if and only if $I=J$ and $b_1=b_2$. 

Using \cite[Lemma 3.2]{PriddyKoszul.pdf}, to check that $\delta(I,b)\in\Nop_*$, we only need to check that $d(\delta(I,b))\in F_{s-1}C_{s-1}$.
To check this membership condition is to check that $\delta(I,b)$ pairs to zero with $\im(d^*:\dual(F_sC_{s-1})\to\dual(F_sC_s))$. Priddy's proof shows that $\dual(F_sC_s)$ is spanned by functionals $[(P^{k_s})^*|\cdots |(P^{k_1})^*]f$, for $f\in \dual X$.  This functional evaluates to zero on $\delta(I,b)$ unless $f(b)=1$ and $\produces{K}{I}{\deltaalg}$. However, the image of $d^*$, as determined by Priddy, is spanned by the space of `$\delta$-Adem relations' (see \cite[Theorem 2.5 and proof]{PriddyKoszul.pdf}). These evaluate to zero on any $\delta(I,b)$, tautologically: in reducing a $\delta$-relation to a sum of $\delta$-admissible expressions using $\delta$-relations, one obtains zero.

\textbf{Alternative ending: }To check the membership condition is to check that $\delta(I,b)$ pairs to zero with $\im(d^*:\dual(F_sN_{s-1})\to\dual(F_sN_s))$. Priddy's proof shows that $\dual(F_sN_s)$ is spanned by functionals $[(P^{k_s})^*|\cdots |(P^{k_1})^*]b^*$. %Viewing $\delta(I,c)$ as an element of the double dual,
\[\left([(P^{k_s})^*|\cdots |(P^{k_1})^*]b^*\right)\left(\delta(I,c)\right)= b^*(c)\cdot\left(\textup{$\delta_I$ coeff.\ of when $\delta_K\in\deltaalg$}\right).\]
% $\delta(I,c)$ is the functional which sends, and this functional evaluates to zero on $\delta(I,b)$ unless $b=b_2$ and $\produces{K}{I}{\deltaalg}$. 
However, the image of $d^*$, as determined by Priddy, is spanned by the space of `$\delta$-Adem relations' (see \cite[Theorem 2.5 and proof]{PriddyKoszul.pdf}), and these tautologically evaluate to zero on any $\delta(I,b)$.
\end{proof}
Now the same analysis applies in the $n\geq1$ case. Here, we write the bar construction on the right, as we have a right action. This turns back into a left action, as the Lie Steenrod algebra is Koszul dual to the \emph{opposite} of the $\Lambda$-algebra, with an index shift, so that $\Sq^i$ corresponds to $\lambda_{i-1}$ for $i\geq1$ \cite[\S7.1]{PriddyKoszul.pdf}.
\begin{prop}\label{propDerivedIndTrivialUobject n at least 1}
Suppose that $n\geq1$ and $X\in\calU(n)$ has homogeneous basis $B$. Then $K_*^{\calU(n)}X$ has basis
\[\left\{\Sq_{\textup{v}}(J,b)\ \middle|\ \genfrac{}{}{0pt}{}{b\in B_{s_{n},\ldots,s_1}^t\textup{, $J$ $\Sq$-admissible with }\minDimSq(J)\leq s_n,}{\textup{if }s_{n-1}\!=\!\cdots\!=\!s_1\!=\!0\textup{ then $J$ doesn't contain 1}}\right\}.\]
where we define
\[\Sqv(J,b):=
%\cancel{\left[P^{i_\ell}\middle|\cdots\middle|P^{i_1} \right]b}+
\sum_{\produces{K}{J}{\Sq}}b\left[\lambda_{k_1-1} \middle|\cdots\middle|\lambda_{k_\ell-1} \right],\]
but only when $J$ and $b$ satisfy the conditions on $\minDimSq(J)$ and on the appearance of 1 in $J$.
If $X$ is locally finite, this basis is dual to the $\{\Sqv^J\}$ basis of $K^*_{\calU(n)}X$, as in proposition \ref{propDerivedIndTrivialUobject n=0}. The differential of $K^{\calU(n)}_*X$ is given by the formula:
\[d(\Sqv(I,x))=\sum_{ \substack{\produces{K}{I}{\Sq}\\(k_{\ell},\ldots,k_2){\,\Sq\textup{-admis.}}}}\!\!\!\!\!\!\!\!\!\!\!\! \Sqv((k_{\ell},\ldots,k_2),x\lambda_{k_1-1}),\]
summing over $K=(k_{\ell},\ldots,k_1)$ such that $(k_{\ell},\ldots,k_2)$ is $\Sq$-admissible, $K$ doesn't contain 1 if $s_{n-1}=\cdots s_1=0$, and yet $\produces{K}{I}{\Sq}$.
\end{prop}
[\textbf{On well definedness of the term $\Sqv(J,b)$:} {Search for }``Lemma \ref{lemOnAdemChangeInM} demonstrates that if $I$ and $J$ are sequences of'' in PRLieAlgs.
In this case, the condition $\produces{K}{J}{\Sq}$ forces \textbf{[look above, ineq \& non-zero]}, so that every term appearing in the sum is automatically non-zero. On the other hand, the class $\Sqv(J,b)$ is only defined under certain conditions. This adds up to \textbf{further evidence} that the Steenrod operations are all defined, just with some zero.]
\subsection{The $\calw(n+1)$-structure on $H_*^{\calU(n)}X$ for $X\in\calw(n)$}\label{section on structure on homology of koszul cx}
Suppose that $X\in\calw(n)$. We now have two monomorphic quasi-isomorphisms of chain complexes with homology is $H_*^{\calU(n)}X$, and we denote their composite $\jmath$:
\[\jmath:\left(K_*^{\calU(n)}X\subseteq \Nop_*Q^{\calU(n)}B^{\calU(n)}X\subseteq \Nop_*Q^{\calU(n)}B^{\calw(n)}X\right).\]
Now $H^{\calU(n)}_*X$ is an element of $\calw(n+1)$, since it can be calculated as the homotopy of $Q^{\calU(n)}B^{\calw(n)}X\in s\calL(n)$, and this is the structure needed for the composite functor spectral sequences discussed above. We will go some way to calculating this structure in this section --- our method will be to take classes in the Koszul complex, map them into the large complex using $\jmath$, perform the operations in question, and then homotope the outcome back into the Koszul complex.

We will need a little notation for elements of the various bar constructions. We will label the $s+1$ free constructions in $B^{\calw(n)}_{s}X$ with subscripts in angle brackets: 
\[B^{\calw(n)}_{s}X= F^{\calw(n)}_{\langle -1\rangle}F^{\calw(n)}_{\langle 0\rangle}\cdots F^{\calw(n)}_{\langle s-1\rangle}X\]%corrected
so that we can then indicate in which free construction operations are being performed. For example, when $n=0$ and $x,y\in X$, $B_2^{\calw(0)}X$ contains an element
\[[P^i_{\langle 0\rangle}x,P^j_{\langle 1\rangle}y]_{\langle -1\rangle}:=[\eta P^i\eta^2 x,\eta^2P^j\eta y]\]%corrected
where we write $\eta:V\to F^{\calw(n)}$ for the unit of the monad on $\vect{+}{n}$ (omitting as always the forgetful functor). That is: we apply $P^j$, not to $y\in X$, but rather to $\eta y$, the corresponding generator of $F_{\langle 1\rangle}^{\calw(n)}X$; we only apply $P^i$ to $\eta^2 x$, the generator of $F^{\calw(n)}_{\langle 0\rangle}F^{\calw(n)}_{\langle 1\rangle}X$; and finally the bracket is taken in the outermost construction.%corrected

With this notation in hand, the map $\jmath$ is induced by the assignment
\begin{alignat*}{2}
[P^{i_s}|\cdots |P^{i_1}]x&\longmapsto P_{\langle 0\rangle}^{i_s}P_{\langle 1\rangle}^{i_s-1}\cdots P_{\langle s-1\rangle}^{i_1}x &\quad&(\textup{if }n=0),\\
x[\lambda_{i_1}|\cdots |\lambda_{i_s}]&\longmapsto x\lambda_{i_1\langle s-1\rangle}\cdots \lambda_{i_{s-1}\langle1\rangle}\lambda_{i_s\langle 0\rangle} &\quad&(\textup{if }n\geq1).%both corrected
\end{alignat*}
Before making calculations, we recall the formulae for Lie algebra homotopy operations [Curtis]. Let $\Shuffles{p}{q}$ be the set of $(p,q)$-shuffles, that is, pairs $(\alpha,\beta)$ where $\alpha=(\alpha_{p-1},\ldots,\alpha_0)$ and $\beta=(\beta_{q-1},\ldots,\beta_0)$ are disjoint monotonically decreasing sequences that together partition the set $\{0,\ldots,p+q-1\}$. Let $s_{\alpha}$ denote the iterated degeneracy operator $s_{\alpha_{p-1}}\cdots s_{\alpha_0}$.
Finally, let $\HalfShuffles{i}{i}$ denote the subset of $\Shuffles{i}{i}$ consisting of those shuffles $(\alpha,\beta)\in\Shuffles{i}{i}$ such that $\beta_{i-1}=2i-1$. The formulae of \cite[\S8]{CurtisSimplicialHtpy.pdf}, for $z\in ZK_p(X)$ and $w\in ZK_q(X)$ cycles representing classes $\overline{z},\overline{w}\in H_*^{\calU(n)}X$ are as follows:
\begin{alignat*}{3}
[\overline{z},\overline{w}]\textup{ is represented by }&&\sum_{(\alpha,\beta)\in\Shuffles{p}{q}}[s_\beta(\jmath z), s_\alpha(\jmath w)]_{\langle -1\rangle}&\in Q^{\calU(n)}B^{\calw(n)}_{p+q}X,\\
\overline{z}\lambda_i\textup{ is represented by }&&\sum_{(\alpha,\beta)\in\HalfShuffles{i}{i}}[s_\beta(\jmath z), s_\alpha(\jmath z)]_{\langle -1\rangle}&\in Q^{\calU(n)}B^{\calw(n)}_{p+i}X,&\quad&(0<i\leq p),\\
\overline{z}\lambda_0\textup{ is represented by }&&\restnwithsubscript{(z)}{\langle -1\rangle}&\in Q^{\calU(n)}B^{\calw(n)}_{p}X,&&(\textup{when defined}).
\end{alignat*}%all three corrected
It will be important to understand these sums. Suppose that $z\in ZK_p^{\calU(n)}X$ (for $n\geq1$). Then  $z$ may be written as a sum of terms of the form $x\lambda_{i_1\langle p-1\rangle}\cdots \lambda_{i_p\langle 0\rangle}$, and
\begin{lem}\label{what degens do to iterated operations}
If $(\alpha,\beta)\in\Shuffles{p}{q}$, then $s_\beta(x\lambda_{i_1\langle p-1\rangle}\cdots \lambda_{i_p\langle 0\rangle})=x\lambda_{i_1\langle \alpha_{p-1}\rangle}\cdots \lambda_{i_p\langle \alpha_0\rangle}$.
\end{lem}
We will also need the following consequence of the simplicial identities:

\begin{lem}\label{LemmaOnSimplicialRelations}
Choose $i\geq1$ and $\alpha=(\alpha_{p-1},\ldots,\alpha_0)$ with $\alpha_{p-1}>\cdots >\alpha_0\geq0$.
\begin{enumerate}[i)]\squishlist
\setlength{\parindent}{.25in}
\item[i)] If neither $i-1$ nor $i-2$ appear in $\alpha$, then  $d_{i-1}s_\alpha=s_{\alpha'}d_{i'}$ for some $\alpha'$ and $i'$.
\item[ii)] If exactly one of $i-1$ and $i-2$ appears in $\alpha$, then  $d_{i-1}s_\alpha$ does not depend on which of $i-1$ and $i-2$ appeared in $\alpha$.
\end{enumerate}
\end{lem}

\begin{prop}\label{LieBracketsTrivial}
The Lie bracket $H_p^{\calU(n)}X\otimes H_q^{\calU(n)}X\to H_{p+q}^{\calU(n)}X$ vanishes except when $p+q=0$. 
The Lie algebra structure on $H_0^{\calU(n)}X$ is induced by that on $X$: if $z,w\in X$ represent $\overline{z},\overline{w}\in H_0^{\calU(n)}X$, then $[\overline{x},\overline{y}]$ is represented by the cycle $[x,y]\in ZC_0(Q^{\calU(n)}B^{\calw(n)}X)$.
\end{prop}
\noindent This theorem shows that $H_*^{\calU(X)}$ is trivial in positive dimensions \emph{as a Lie algebra}, but the \emph{restriction} need not be trivial (c.f.\ propositions \ref{QkTrivial} and \ref{Q0ZeroByPriddyAlg}).
\begin{proof}
We will give the proof for $n\geq1$, but it works the same way for $n=0$. In fact, when $n=0$ we can ignore all discussion of top and non-top operations.

Use the abbreviation $\mathbb{B}:=Q^{\calU(n)}B^{\calw(n)}X\in s\calL(n)$. Then $\mathbb{B}$ is almost free on the subspaces $V_s=F^{\calw(n)}_{\langle 0\rangle}\cdots F^{\calw(n)}_{\langle s-1\rangle}X$. Choose representatives $z\in ZK_p^{\calU(n)}X$ and $w\in ZK_q^{\calU(n)}X$. Now  for any $(\alpha,\beta)\in\Shuffles{p}{q}$, the elements $s_{0}s_\beta(\jmath z)$ and $s_{0}s_\alpha(\jmath w)$ both lie in $V_{p+q+1}$, and it is only a minor abuse of notation to define:
\[a:=\sum_{(\alpha,\beta)\in\Shuffles{p}{q}}[s_{0}s_\beta(\jmath z), s_{0}s_\alpha(\jmath w)]_{\langle 0\rangle}\in C_{p+q+1}\mathbb{B}.\]
Using the simplicial identity $d_0s_0=\Id$, we have $d_{0}a=\sum [s_\beta(\jmath z), s_\alpha(\jmath w)]_{\langle -1\rangle}$, the representative given for $[\overline{z},\overline{w}]$. Moreover, we will find that $d_ia=0$ for $i>0$, except when $p=q=0$, in which case $d_1a=[x,y]$. Thus, in either case, $a$ is the required homotopy in $C_*\mathbb{B}$.

Using the simplicial identity $d_1s_0=\Id$, we have $d_{1}a=\sum [s_\beta(\jmath z), s_\alpha(\jmath w)]_{\langle 0\rangle}$. Now for every pair $(\alpha,\beta)$ indexing this sum, unless $p=q=0$, one of $\alpha$ or $\beta$, say $\beta$, will contain $0$. Then by lemma \ref{what degens do to iterated operations}, every summand in $s_{\alpha}(\jmath z)$ is in the image of some \emph{non-top} $\lambda_{i\langle 0\rangle}$, and as $[x\lambda_i,y]=0$ whenever $\lambda_i$ is not a top operation, the entire expression vanishes in the construction $F^{\calw(n)}_{\langle 0\rangle}$.

What remains is to show that $d_{i}a=0$ for $2\leq i\leq p+q+1$. As $d_is_0=s_0d_{i-1}$ for $i\geq2$:
\[d_{i}a=\sum [s_{0}d_{i-1}s_\beta(\jmath z), s_{0}d_{i-1}s_\alpha(\jmath w)]_{\langle 0\rangle}.\]
For this, we will define an involution $\rho_i$ of the set $\Shuffles{p}{q}$ indexing the sum, for $2\leq i\leq p+q+1$.
If $\alpha$ and $\beta$ do not each contain exactly one of $i-1$ and $i-2$, then $\rho_i$ fixes $(\alpha,\beta)$. Otherwise, $\rho_i$ interchanges the positions of $i-1$ and $i-2$ in $(\alpha,\beta)$. To avoid confusion, we note that $\rho_{p+q+1}$ is the identity, as neither $\alpha$ nor $\beta$ ever contain $p+q$.

If $(\alpha,\beta)$ is a fixed point of $\rho_i$, then one of $\alpha$ and $\beta$, say $\alpha$, contains neither of $i$ and $i-1$. Then by lemma \ref{LemmaOnSimplicialRelations}(i), $d_{i-1}s_\alpha(\jmath w)=s_{\alpha'}d_{i'}(\jmath w)=0$, as $\jmath w\in Z\Nop_*$. Thus, the summands corresponding to fixed points vanish.
On the other hand, given a shuffle $(\alpha,\beta)$ which is not fixed by $\rho_i$, lemma \ref{LemmaOnSimplicialRelations}(ii) shows that the summand corresponding to $(\alpha,\beta)$ equals the summand corresponding to $\rho_i(\alpha,\beta)$, so these two summands cancel with each other.
\end{proof}
\begin{prop}\label{QkTrivial}
Suppose that $z\in ZK^{\calU(n)}_{s_{n+1}}(X)_{s_n,\ldots,s_1}^t$. If $n\geq1$, $\overline{z}\lambda_k=0$ for $1\leq k\leq {s_{n+1}}$. If $n=0$, writing $s=s_1$ we have $\overline{z}\lambda_k=0$ for $2\leq k\leq {s}$, and $\lambda_1$ may be defined at the level of the Koszul complex by the assignment
\[z=\textstyle\sum_{j}P(i_{s}^{(j)},\ldots,i_1^{(j)})x^{(j)}{\longmapsto}\textstyle\sum_{j}P(t,i_{s}^{(j)},\ldots,i_1^{(j)})x^{(j)}\in ZK^{\calU(n)}_{s+1}(X)^{2t+1},\]
where we have written $(i_{s}^{(j)},\ldots,i_1^{(j)})$ for the various $\delta$-admissible sequences corresponding to the summands of $z$. That is, $\lambda_1$ adjoins a top operation to the left of each sequence $I^{(j)}$.
\end{prop}
\begin{proof}
We will first prepare for the calculation of $\lambda_1$ in case $n=0$.
Write $e$ for the proposed representative $\sum_{j}P(t,i_{s}^{(j)},\ldots,i_1^{(j)})x^{(j)}$ of $\overline{z}\lambda_1$. This $e$ does in fact lie in the Koszul complex: the sequences $(t,i_{s}^{(j)},\ldots,i_1^{(j)})$ are indeed $\delta$-admissible, since
\[t=(i_{s}+1)+|P(i_{s-1},\ldots,i_1)x^{(j)}|\geq2i_{s}+1,\]
since we have demanded that $\minDimP(i_{s},\ldots,i_1)\leq|x^{(j)}|$. Then, as long as the image $\jmath e$ in $\Nop_{s+1}Q^{\calU(n)}B^{\calw(n)}X^{2t+1}$ is homotopic to the standard representative of $\overline{z}\lambda_1$, the calculation for $\lambda_1$ is complete. This chain is given by the formula
\[\!\!\!\!\sum_{j,\,\produces{K}{(t,i_{s}^{(j)},\ldots,i_1^{(j)})}{\deltaalg}}\!\!\!\!\left[P^{k_{s+1}} \middle|\cdots\middle|P^{k_1} \right]x^{(j)}
 =\!\!\!\!\sum_{j,\,\produces{H}{(i_{s}^{(j)},\ldots,i_1^{(j)})}{\deltaalg}}\!\!\!\!\left[P^t\middle|P^{h_{s}} \middle|\cdots\middle|P^{h_1} \right]x^{(j)}=P^{t}_{\langle 0\rangle}s_0(\jmath z),\]
where the equality here comes from the observation that for each $j$:
%\[\produces{(k_{s_1+1},\ldots,k_1)}{(t,i_{s}^{(j)},\ldots,i_1^{(j)})}{\deltaalg}\textup{ iff }k_{s_1+1}=t\textup{ and }\produces{(k_{s_1},\ldots,k_1)}{(i_{s}^{(j)},\ldots,i_1^{(j)})}{\deltaalg}\]
\[\textup{if }\produces{(k_{s+1},\ldots,k_1)}{(t,i_{s}^{(j)},\ldots,i_1^{(j)})}{\deltaalg}\textup{ and }k_{s+1}\neq t\textup{, then }\minDimP(k_{s+1},\ldots,k_1)>|x^{(j)}|.\]
To understand this observation, as $\delta$-Adem relations only decrease $\minDimP$ [above somewhere], we may reduce to the case where $(k_s,\ldots,k_1)$ is already $\delta$-admissible and $\produces{(k_{s+1},k_{s})}{(t,k_{s+1}+k_{s}-t)}{\deltaalg}$. Then
\begin{alignat*}{2}
\minDimP(k_{s+1},\ldots)&\geq k_{s}-(k_{s-1}+1)-(k_{s-2}+1)-\cdots  \\
&> t-(k_{s+1}+k_s-t+1)-(k_{s-1}+1)-(k_{s-2}+1)-\cdots  \\
&=t-i^{(j)}_{s}-\cdots -s=|x^{(j)}|
\end{alignat*}
where: the non-strict inequality is by definition of $\minDimP$; the strict inequality follows by examining the $\delta$-Adem relation for $\delta_{k_{s+1}}\delta_{k_{s}}$; the first equality holds as $\deltaalg$ is graded by the sum of the indices; and the second equality holds as $t$ is the top operation, so that the maximum $\minDimP(t,i^{(j)}_{s},\ldots)$ is attained at its final argument.

With this in hand, we return the general case, $1\leq k\leq p$ and $n\geq0$, our goal being to produce a nullhomotopy, except when $n=0$ and $k=1$, when we need a homotopy to $P^t_{\langle 0\rangle}s_0(\jmath z)$. We proceed as in the previous proof, defining
\[a:=\sum_{(\alpha,\beta)\in\HalfShuffles{k}{k}}[s_{0}s_\beta(\jmath z), s_{0}s_\alpha(\jmath z)]_{\langle 0\rangle}\in C_{p+k+1}\mathbb{B}_{2s_n,\ldots,2s_1}^{2t+1}.\]
Then $d_0a$ is the representative for $\overline{z}\lambda_k$, and $d_1a=0$ as in the previous proof (with no analogue of the special case $p=q=0$). Now consider the same involutions $\rho_i$ as in the previous proof, now action on $\Shuffles{k}{k}$. When $2\leq i< 2k$, these $\rho_i$ preserves $\HalfShuffles{k}{k}$. When $2k<i\leq p+k+1$, $\rho_i$ is the identity, so preserves $\HalfShuffles{k}{k}$ trivially. Thus, $d_ia=0$ for all $2\leq i\leq p+k+1$ with $i\neq2k$, as  the cancellations still all occur within the smaller sum
\[d_{i}a=\sum_{(\alpha,\beta)\in\HalfShuffles{k}{k}} [s_{0}d_{i-1}s_\beta(\jmath z), s_{0}d_{i-1}s_\alpha(\jmath z)]_{\langle 0\rangle}.\]
To address the question of $\rho_{2k}$, we define an alternative involution $\widetilde{\rho}_{2k}$ of $\Shuffles{k}{k}$ as follows.  If $\alpha$ and $\beta$ do not each contain exactly one of $2k-2$ and $2k-1$ each, then $\widetilde{\rho}_{2k}$ fixes $(\alpha,\beta)$. Otherwise, we define $\widetilde{\rho}_{2k}(\alpha,\beta):=\rho_{2k}(\beta,\alpha)$, which is to say that $\widetilde{\rho}_{2k}$ swaps \emph{everything but} $2k-2$ and $2k-1$.

Now the summands in this formula exhibit an extra symmetry, given that $z$ is repeated. This symmetry, along with lemma \ref{LemmaOnSimplicialRelations}(ii), shows that all the summands corresponding to shuffles not fixed by $\widetilde{\rho}_{2k}$ cancel out. When $k>1$, the fixed points of $\widetilde{\rho}_{2k}$ are only those shuffles in which one of $\alpha$ and $\beta$ contains neither $2k-2$ nor $2k-1$, and these summands vanish, by \ref{LemmaOnSimplicialRelations}(i), as in previous arguments. When $k=1$, however, $\widetilde{\rho}_{2k}$ has an \emph{extra} fixed point, the shuffle $((0),(1))$, which fails to differ from its image under $\widetilde{\rho}_{2k}$.

In sum, we have shown that $d_0a=0$ represents $\overline{z}\lambda_i$, and that $d_ia=0$ whenever $1\leq i\leq p+k+1$, except when $k=1$ and $i=2$. In this case, we perform the final calculation:
\[d_2a=[s_{0}d_1s_1(\jmath z), s_{0}d_1s_0(\jmath z)]_{\langle 0\rangle}=[s_{0}(\jmath z), s_{0}(\jmath z)]_{\langle 0\rangle}=\begin{cases}
0,&\textup{if }n\geq1,\\
P^{t}_{\langle 0\rangle}s_0(\jmath z),&\textup{if }n=0.
\end{cases}\]
That is, if $n\geq1$, this self-bracket vanishes (an object of $\calw(n)$ for $n\geq1$ is a Lie algebra), while if $n=0$, the self-bracket is equal to the top $P$-operation, in this case $P^t$. %Thus, for $n\geq1$, we have $\overline{z}\lambda_1=0$ as needed, and for $n=0$, we have shown that $\overline{z}\lambda_1=0$ may be represented by $P^{t}_{\langle 0\rangle}s_0(\jmath z)$, and one simply observes. %The result is then proven for $n\geq1$, as $a$ turns out to be a nullhomotopy for all $k\geq1$. When $n=0$, this only holds for $k\geq2$, and when
\end{proof}
\begin{prop}\label{Q0ZeroByPriddyAlg}
Suppose that $n\geq1$, and $z\in ZK^{\calU(n)}_{s_{n+1}}(X)_{s_n,\ldots,s_1}^t$ where not all of $s_n,\ldots,s_1$ equal zero. If $s_{n+1}=0$ then $\overline{z}\lambda_0$ is represented by $z\lambda_{s_n}\in X_{2s_n,\ldots,2s_1}^{2t+1}$. Suppose instead that $s_{n+1}>0$, and 
\[z=\textstyle\sum_{j}\Sqv(I^{(j)},x^{(j)}),\]
for some collection $I^{(j)}=(i^{(j)}_{p},\ldots,i^{(j)}_{1})$ of  $\Sq$-Admissible sequences, and corresponding homogeneous elements $x^{(j)}$ of $X$. Suppose further that, for all $j$, $x^{(j)}\lambda_{i-1}=0$ whenever $i\geq i^{{(j)}}_1$. Then $\overline{z}\lambda_0=0$.
\end{prop}
\begin{proof}
Write $p:=s_{n+1}$. The same homotopy $a$ as in the previous cases shows that $\widetilde{z}\lambda_0$ is represented by $\restnwithsubscript{(z)}{\langle 0\rangle}=z\lambda_{s_n\langle 0\rangle}$ when $p>0$, and by $z\lambda_{s_n}\in X$ when $p=0$, so that we may restrict to the case $p>0$. Then, $z\lambda_{s_n\langle 0\rangle}$ is the image of
\[E:=\sum_{j,\,\produces{K}{I^{(j)}}{\Sq}}x^{(j)}\left[\lambda_{k_1^{(j)}-1} \middle|\cdots\middle| \lambda_{k_{p-1}^{(j)}-1} \middle|\lambda_{k_{p}^{(j)}-1}\lambda_{s_n} \right]\in ZF_{p+1}N^\div_{p}Q^{\calU(n)}B^{\calU(n)}X\]
Dualizing Priddy's work, namely \cite[proof of Theorem 5.2]{PriddyKoszul.pdf}, gives a sequence of homotopies which move this cycle into $F_pN_p^\div$. Indeed, given an expression
\[e=y\left[\lambda_{g_1-1} \middle|\cdots \middle|\lambda_{g_{r-2}-1}\middle|
\lambda_{g_{r-1}-1}\lambda_{g_r-1}\middle|\lambda_{g_{r+1}-1}\middle|\cdots\middle|\lambda_{g_{p+1}-1}\right]\in F_{p+1}N_p^\div,\]
where $g_{r}-1\leq 2(g_{r-1}-1)$ (so that the expression $\lambda_{g_{r-1}-1}\lambda_{g_r-1}$ is $\Lambda$-admissible), define:
\[\Gamma(e):=\begin{cases}
y\left[\lambda_{g_1-1} \middle|\cdots \middle|
\lambda_{g_{r-1}-1}\middle|\lambda_{g_r-1}\middle|\cdots\middle|\lambda_{g_{p+1}-1}\right],&\textup{if }(g_{p+1},\ldots,g_{1})\textup{ is $\Sq$-admissible};\\
0,&\textup{otherwise}.
\end{cases}\]
If we further define $\Gamma$ to be zero on $F_pN_p^\div$, then $\Gamma:F_{p+1}N^\div_p\to F_{p+1}N^\div_{p+1}$ may be used as a chain homotopy to compress $E\in ZF_{p+1}N^\div_p$ into $ZF_pN^\div_p$:
\[(\Id+d\Gamma)^uE\textup{ stabilizes to an element of } ZF_pN_p^\div\textup{ as $u\to\infty$.}\]
As we repeatedly  apply $(\Id+d\Gamma)$ to this $e$, because $a_1\geq b_1$ whenever $\produces{(b_2,b_1)}{(a_2,a_1)}{\Lambda}$, the very leftmost $\lambda$-operation in any of the expressions that appear is $\lambda_{m-1}$ for some $m\geq g_1$, and every term in $(\Id+d\Gamma)^ue\in ZF_pN_p^\div$ will be of the form $y\lambda_{m-1}[{}\cdots{}]$ for some $m\geq g_1$.

Applying these observations in the very specific circumstances of this proposition, along with the observation that for each $j$ and $K$ indexing the sum $E$ we have $k_1^{(j)}\geq i_1^{(j)}$ (as $a_1\geq b_1$ whenever $\produces{(a_2,a_1)}{(b_2,b_1)}{\Sq}$ \textbf{would prefer to say `by x.y'}), one derives that $(\Id+d\Gamma)^uE=0$, so that $E$ is nullhomotopic. 
%
%
%Priddy's process returns a (large) sum of terms of the form
%\[x^{(j)}\lambda_{\kappa_0-1}\left[\lambda_{\kappa_1-1} \middle|\cdots\middle| \lambda_{\kappa_{p-1}-1} \middle|\lambda_{\kappa_p-1}\right]\textup{ where }\kappa_0\geq i_1^{(j)}.\]
%Under the very specific circumstances of this proposition, each of these terms vanishes.
\end{proof}
\end{Koszul complexes}




\begin{Composite functor spectral sequences}

\section{\textbf{Composite functor spectral sequences}}
The subject of this paper is to identify the derived functors $H^*_{\calw(0)}X:=\dual(\mathbb{L}_*Q^{\calw(0)}X)$, for $X\in\calw(0)$. More generally, we will now present a spectral sequence whose goal is to calculate $H^*_{\calw(n)}X$ for $X\in\calw(n)$. This will be a composite functor spectral sequence analogous to Miller's spectral sequence in \cite{MillerSullivanConjecture.pdf}. The factorization of $Q^{\calw(n)}$ we will use is of course 
\[Q^{\calw(n)}=\left(\calw(n)\overset{Q^{\calU(n)}}{\to}\calL(n)\overset{Q^{\calL(n)}}{\to}\vect{+}{n}\right)\]
There is an added challenge in this context --- indeed, the available factorization of $Q^{\calw(n)}$ is through a non-abelian category. Thus, the standard technology for composite functor spectral sequences does not apply, and we must use Blanc and Stover's methods \cite{Blanc_Stover-Groth_SS.pdf}. They observe that the left derived functors $\mathbb{L}_*Q^{\calU(n)}X$ are calculated as the homotopy groups of a simplicial object in $\calL(n)$, namely $Q^{\calU(n)}B^{\calw(n)}X$. As such, they have the structure of a $\calL(n)\textup{-$\Pi$-algebra}$, i.e.\ they form an object of $\calw(n+1)$.  After verifying that the functor $Q^{\calU(n)}$ satisfies the requisite acyclicity condition (indeed it preserves free objects), one may apply \cite[Theorem 4.4]{Blanc_Stover-Groth_SS.pdf}: there is a spectral sequence, with $E_r\in\vect{+}{n+2}$,
\[(E^2)_{s_{n+2},\ldots,s_1}^t=((H_*^{\calw(n+1)})(\mathbb{L}_*Q^{\calU(n)})X)_{s_{n+2},\ldots,s_1}^t\implies ((H_*^{\calw(n)})X)_{s_{n+2}+s_{n+1},s_n,\ldots,s_1}^t\]
\begin{prop}
For $X\in s\calw(n)$, the groups $\mathbb{L}_*Q^{\calU(n)}X$ are isomorphic to $H_*^{\calU(n)}X$, in which we view $X$ as an element of $s\calU(n)$.
\end{prop}
\begin{proof}
We may take $X$ to be almost free in $s\calw(n)$, and calculate $\mathbb{L}_*Q^{\calU(n)}X$ simply as $\pi_*Q^{\calU(n)}X$. Then $X$, viewed as an object of $s\calU(n)$, is levelwise free, but potentially not almost free. We need to show then that $\pi_*Q^{\calU(n)}X$ does indeed calculate $H_*^{\calU(n)}X$ whenever $X\in s\calU(n)$ is \emph{levelwise} free, which is to say that the map $Q^{\calU(n)}B^{\calU(n)}X\to Q^{\calU(n)}X$ is a weak equivalence in $s\vect{}{}$. For this, $Q^{\calU(n)}B^{\calU(n)}X$ is the diagonal of a bisimplicial vector space, say $Q^{\calU(n)}B_q^{\calU(n)}X_p$, and we use the spectral sequence arising from filtering in the $p$ direction. As $X$ is levelwise free, the $E^1$-page is concentrated in $q=0$, and is isomorphic to the chain complex $N_p(Q^{\calU(n)}X)$.
\end{proof}
We will prefer to work with the dual spectral sequence, (\cancel{which is} equivalent, as we have specified that elements of $\calw(n)$ are locally finite), which has $E_r\in\vect{n+2}{+}$, and we may write as:
\[(E_2)^{s_{n+2},\ldots,s_1}_t=((H^*_{\calw(n+1)})(H_*^{\calU(n)})X)^{s_{n+2},\ldots,s_1}_t\implies ((H^*_{\calw(n)})X)^{s_{n+2}+s_{n+1},s_n,\ldots,s_1}_t\]
These are the homology and cohomology spectral sequences for a bisimplicial abelian group (precisely, an object of $ss\vect{+}{n}$), and we will need to work with this object directly. Before we do, we will give some general recollections and constructions relevant to the spectral sequence.

\subsection{The Blanc-Stover comonad in categories monadic over $\Ftwo $-vector spaces}
Fix an algebraic category $\calc$, monadic over a category of graded $\Ftwo $-vector spaces $\vect{}{}$. % --- although we intend to work over a category of graded vector spaces, but for now we opt for notational simplicity \textbf{do we really need to do that?}). 
As we are working over a category of vector spaces, rather than a category of graded sets, we can refine Blanc and Stover's definition  of a useful comonad on $s\calc$. While Blanc and Stover use the notation `$W$' in \cite{Blanc_Stover-Groth_SS.pdf} and $\scrV$ in \cite{StoverVanKampen.pdf}, we will use the symbol $\BSW$, to avoid notational confusion in the present work. \textbf{{work on this!}}

%\begin{shaded}\tiny
%It will help to fix a little notation. First, for $n\geq0$, let $\mathbb{K}^n\in \complexes \vect{}{}$ and $C\mathbb{K}^n\in \complexes \vect{}{}$ be the chain complexes
%\[(\mathbb{K}^n)_j=\begin{cases}
%\Ftwo \langle z\rangle,&\textup{if }j=n;\\
%0,&\textup{otherwise},
%\end{cases}\qquad 
%(C\mathbb{K}^n)_j=\begin{cases}
%\Ftwo \langle h\rangle,&\textup{if }j=n+1;\\
%\Ftwo \langle dh\rangle,&\textup{if }j=n;\\
%0,&\textup{otherwise},
%\end{cases}
%\]
%%with $d:(C\mathbb{K}^n)_{n+1}\to (C\mathbb{K}^n)_{n}$ the identity of $\Ftwo $.
%There is an evident inclusion $\imath:\mathbb{K}\to C\mathbb{K}$. For any $V\in\complexes\vect{}{}$, we have
%\[\hom_{\complexes\vect{}{}}(\mathbb{K}^n,V)\cong ZN_nV,\textup{ and }\hom_{\complexes\vect{}{}}(C\mathbb{K}^n,V)\cong N_{n+1}V,\]
%and the differential $d:N_{n+1}V\to ZN_nV$ corresponds to $\imath^*$ under these isomorphisms. If $N:s\vect{}{}\rightleftarrows \complexes \vect{}{}
%:\Gamma$ are the inverse equivalences of the Dold-Kan correspondence, define
%\[S^n_{\calc}:=F^\calc \Gamma(\mathbb{K}^n)\textup{ and }CS^n_{\calc}:=F^\calc \Gamma(C\mathbb{K}^n).\]
%Corresponding to the above structure, $S^n_{\calc}$ contains an element $z$ in dimension $n$, $CS^n_{\calc}$ contains an element $h$ in dimension $n+1$, there is a map $\imath:S^n_{\calc}\to CS^n_{\calc}$ sending $z$ to $d_0h$. For $L\in s\calc$, $\imath^*$ represents the differential under the isomorphisms
%\[\hom_{s\calc}(S_\calc^n,L)\cong ZN_nL,\textup{ and }\hom_{s\calc}(CS_\calc^n,L)\cong N_{n+1}L.\]
%\end{shaded}

In this context [\textbf{Change} all $S_\calc^n$ to the new notation], Blanc-Stover's comonad $\BSW$, applied to $L\in s\calc$, is the pushout
\[\xymatrix@R=4mm{
\bigsqcup_{n\geq0}\bigsqcup_{N_{n+1}L}S_\calc^n
\ar[r]\ar[d]^-{\bigsqcup\bigsqcup\imath}
&%r1c1
\bigsqcup_{n\geq0}\bigsqcup_{ZN_{n}L}S_\calc^n\ar[d]\\%r1c2
\bigsqcup_{n\geq0}\bigsqcup_{N_{n+1}L}CS_\calc^n
\ar[r]&%r2c1
\BSW L%r2c2
}\]
where under the top horizontal map, the copy of $S^n_\calc$ corresponding to $x\in N_{n+1}L$ maps identically onto the copy of $S^n_\calc$ corresponding to $dx=\imath^*x$. It will be useful to write $h_x$ for the element of $\BSW L$ corresponding to $h$ in the copy of $CS^n_\calc$ corresponding to $x\in N_{n+1}L$, and similarly, $z_x$ for the element of $\BSW L$ corresponding to $z$ in the copy of $S^n_\calc$ corresponding to $x\in ZN_{n}L$. The comonad structure maps $\epsilon:\BSW L\to L$ and $\Delta:\BSW L\to \BSW^2L$ are then determined by the equations
\[\epsilon(h_x)=x,\ \epsilon(z_y)=y,\ \Delta(h_x)=h_{h_x},\textup{ and }\Delta(z_y)=z_{z_y}\textup{ for $x\in N_{n+1}L$ and $y\in ZN_nL$.}\]

Moreover, $\BSW L$ is homotopy equivalent to a coproduct of spheres. Indeed, for each $n$, let $B_nL=\im (d:N_{n+1}L\to ZN_nL)$, and choose a section $h_0$ of the surjection $d:N_{n+1}L\to B_nL$. Then $\BSW L$ contains the contractible subobject $C_0:=\bigsqcup_n\bigsqcup_{h\in \im (h_0)} CS^n_{h_0(f)}$, and $\BSW L/C_0$ is homotopic to a wedge of spheres:
%\[\BSW L/C_0\cong \left(\bigsqcup_{h\in N_{n+1}L\setminus\im (h_0)}CS^n_h/\partial(CS^n_h)\right) \sqcup\left(\bigsqcup_{f\in ZN_nL\setminus B_nL}S^n_f\right)\]
\[\BSW L/C_0\cong \left(\bigsqcup_{N_{n+1}L\setminus\im (h_0)}F^\calc (CS^n_{\vect{}{}}/\partial(CS^n_{\vect{}{}}))\right) \sqcup\left(\bigsqcup_{f\in ZN_nL\setminus B_nL}S^n_\calc\right)\]
We would like to find a \emph{subspace} of $\pi_*(\BSW L)$ which freely generates it as a $\calc\textup{-$\Pi$-algebra}$. Even better, we have the following rendition of [Stover? Blanc-Stover?]. We give the proof since we will need to be explicit about some parts of it in what follows.
\begin{prop}
For $L\in s\calc$ (\textbf{s, right?}), $\pi_*(B^{\BSW}L)$ is an almost free (monadic over $\vect{}{}$) simplicial $\calc\textup{-$\Pi$-algebra}$ weakly equivalent to $\pi_*(L)$.
\end{prop}
\begin{proof}
That the augmentation to $\pi_*(L)$ is a weak equivalence follows from Stover's result \cite[2.7]{StoverVanKampen.pdf}. The only change from Blanc-Stover is that $\pi_*(B^{\BSW}L)$ is almost free over the category $\vect{}{}$, rather than the category of pointed sets.
For a set $S$, write $\Ftwo \langle S\rangle$ for the vector space generated by the symbols $\underline{s}$ for $s\in S$. There is a natural map $\Ftwo \langle d\rangle :\Ftwo \langle N_{n+1}L\rangle \to \Ftwo \langle ZN_nL\rangle $, and a natural monomorphism $\alpha:\ker(\Ftwo \langle d\rangle)\to \pi_*(\BSW L)$, defined by
\[\alpha(\underline{x_1}-\underline{x_0})=[h_{x_1}-h_{x_0}],\textup{ for $x_1,x_2\in N_{n+1}L$ with $dx_1=dx_2$}.\]
Moreover, there is a natural map $\beta:\Ftwo [ZN_nL]\to\pi_*(\BSW L)$ (which is not monomorphic) defined by
\[\beta(\underline{x})=[z_{x}],\textup{ for $x\in ZN_{n}L$}.\]
From the above expression for $\BSW L/C_0$, one sees that $\im (\alpha)$ and $\im (\beta)$ are linearly independent subspaces of $\pi_*(\BSW L)$, and $\pi_*(\BSW L)$ is free on $\im (\alpha)\oplus\im (\beta)$. From this description it is clear that the generating subspaces are preserved by maps in the image of $\BSW $. Thus, every face and degeneracy map except for $s_0$ and $d_0$ preserves the generators.

In order that $s_0$ preserves generators, we note that the diagonal of $\BSW$ sends the subspaces $\im (\alpha_L)$ and $\im (\beta_L)$ into the subspaces $\im (\alpha_{\BSW L})$ and $\im (\beta_{\BSW L})$. That $\im (\beta_L)$ maps into $\im (\beta_{\BSW L})$ is immediate. For $\im (\alpha_L)$, the image of $h_{x_1}-h_{x_0}$ under the diagonal is $h_{h_{x_1}}-h_{h_{x_0}}$, which is in $\im(\alpha_{\BSW L})$, since $dh_{x_1}=z_{dx_1}=z_{dx_0}=dh_{x_0}$.
\end{proof}



%There is a natural map $\Ftwo [d]:\Ftwo [N_{n+1}X]\to \Ftwo [ZN_nX]$, and a natural monomorphism $\alpha:\ker(\Ftwo [d])\to\pi_*(\BSW X)$. Indeed, writing $\underline{z}$ for the generator of $\Ftwo [N_{n+1}X]$ corresponding to $z\in N_{{n+1}X}$, the group $\ker(\Ftwo [d])$ is generated by differences $\underline{z}-\underline{z}'$ for $z,z'\in N_{n+1}X$ having $dz=dz'$, and we define $\alpha(\underline{z}-\underline{z}')$ to be the homotopy class of the difference of the cones corresponding to $z$ and $z'$, whose boundaries have been identified in the colimit. Moreover, there is a natural map $\beta:\Ftwo [ZN_nX]\to\pi_*(\BSW X)$ (which is not monomorphic), which sends a generator $z\in ZN_nX$ to the homotopy class in $\BSW X$ of the corresponding summand in the top right corner of the colimit diagram. From the above expression for $\BSW X/C_0$, one sees that $\im (\alpha)$ and $\im (\beta)$ are linearly independent subspaces of $\pi_*(\BSW X)$, and $\pi_*(\BSW X)$ is free on $\im (\alpha)\oplus\im (\beta)$. From this description it is clear that the generating subspaces are preserved by maps in the image of $\BSW $.
%
%Moreover, the diagonal of the comonad also preserves the subspaces $\im (\alpha)$ and $\im (\beta)$. That $\im (\alpha)$ is preserved is evident from the definitions. For $\im (\beta)$, note that $\im (\beta)\subset\pi_*(\BSW X)$ is spanned by terms $D_h-D_{h'}$ for $h,h':CS\to X$ satisfying $h\imath=h'\imath$. The diagonal applied to $D_h-D_{h'}$ may be written $D_{D_h}-D_{D_{h'}}$, and one notes that $D_h\imath=S_{h\imath}=S_{h'\imath}=D_{h'}\imath$.
%
%\textbf{Move this para: }Blanc and Stover explain that $\BSW $ actually has the structure of a comonad on $s\calL(n)$. As such, for any simplicial Lie algebra $L\in s\calL(n)$, there is a bisimplicial object $B^\BSW L$, the bar construction using the comonad $\BSW $, given by $B_{s_2}^\BSW L_{s_1}=(\BSW^{s_2+1}L)_{s_1}$. Flesh out the point here...
%
%
%Now the utility of the comonad structure lies in resolving an object $L\in s\calL(n)$ by the comonadic bar construction, $B^\BSW(L)$. This discussion shows that, except for $d_0$, all the maps in $B^\BSW L$ preserve the subspaces of generators of $\pi_*(B^\BSW L)$, which is to say that this object of $s\calw(n+1)$ is almost free. Moreover, according to [Stover 2.6], the augmentation map $\pi_*(B^\BSW L)\to \pi_*(L)$ is a weak equivalence in $s\calw(n+1)$. Thus, this Blanc-Stover $\BSW $-construction provides an almost-free replacement of $\pi_*(L)$ in $s\calw(n+1)$.





\subsection{A chain-level diagonal on the $\BSW $ construction}
\label{Subsection: Chain level diagonal}
We have seen, for $L\in s\calc$, that $\pi_*(\BSW L)$ is naturally a free object in $\calc\PiAlg$. As such, there is a diagonal
\[\phi_{\calc\PiAlg}:\pi_*(\BSW L)\to \pi_*(\BSW L)\sqcup \pi_*(\BSW L).\]
In this section, we will describe how $\phi_{\calc\PiAlg}$ is the map on homotopy induced by a morphism $\phi_\BSW :\BSW L\to \BSW L\sqcup \BSW L$ in $s\calc$ (so that $i\circ\phi_{\calc\PiAlg}=\pi_*(\phi_{\BSW})$, where $i$ is the natural isomorphism $\pi_*(\BSW L)\sqcup\pi_*(\BSW L)\to \pi_*(\BSW L\sqcup \BSW L)$).

In order to construct a map $\phi_\BSW $, we construct a commuting diagram
\[\xymatrix@R=4mm{
S^n_{\calc}\ar[r]^-{\phi_1}
\ar[d]^-{\imath}
&%r1c1
S^n_{\calc}\sqcup S^n_{\calc}\ar[d]^-{\imath\sqcup\imath}
\\%r1c2
CS^n_{\calc}\ar[r]^-{\phi_2}
&%r2c1
CS^n_{\calc}\sqcup CS^n_{\calc}%r2c2
}\qquad \textup{\raisebox{-6mm}{by applying $F^\calc$ to }}\qquad \xymatrix@R=4mm{
S^n_{\vect{}{}}\ar[r]^-{\Delta}
\ar[d]^-{\imath}
&%r1c1
S^n_{\vect{}{}}\sqcup S^n_{\vect{}{}}\ar[d]^-{\imath\sqcup\imath}
\\%r1c2
CS^n_{\vect{}{}}\ar[r]^-{\Delta}
&%r2c1
CS^n_{\vect{}{}}\sqcup CS^n_{\vect{}{}}%r2c2
}\]
The maps $\phi_1$ and $\phi_2$ can then be applied respectively to all of the sphere and cone classes appearing in $\BSW L$.
%In fact, this diagram can even be formed in $s\vect{+}{n}$, before application of $F^\calc :\vect{+}{n}\to\calc$, and $\phi_1$ and $\phi_2$ are just the levelwise application of the diagonal map on a simplicial vector space. 
To understand the effect of $\phi_{\BSW }$ on homotopy, it is enough to identify where the generators of $\pi(\BSW L)$ are sent in $\pi(\BSW L)\sqcup\pi(\BSW L)$, which is easy.
\begin{lem}
$\BSW L$ is naturally a (strict) commutative cogroup object, having comultiplication map $\phi_{\BSW }$, counit map $0:\BSW L\to 0$, and inverse map $\Id:\BSW L\to \BSW L$. In particular, $\hom(\BSW L,\DASH)$ takes values in $\ensuremath{\Ftwo }$-vector spaces.
\end{lem}
\begin{proof}
There are a few axioms to check, for example, left counitality: that the composite 
\[\xymatrix@R=4mm@1{
\BSW L\ar[r]^-{\phi_{\BSW }}
&%r1c1
\BSW L\sqcup \BSW L\ar[r]^-{\Id\sqcup0}
&%r1c2
\BSW L%r1c3
}\] is the identity. This follows since $(\Id\sqcup0)\phi_1$ is the identity of $S^n$ and $(\Id\sqcup0)\phi_2$ is the identity of $CS^n$. The other axioms follow similarly.
\end{proof}

Writing$\star$ for the group operation on $\hom_{s\calc}(\BSW L,L')$, we have the following:
\begin{lem}
For maps $f,g:\BSW L\to L'$ we have 
\[Q^{\calc}(f\star g)=(Q^{\calc}f+Q^{\calc}g):Q^{\calc}\BSW L\to Q^{\calc}L'.\]
\end{lem}
\begin{proof}
It is enough to check that $Q^{\calc}(\phi_\BSW ):Q^{\calc}\BSW L\to Q^{\calc}(\BSW L\sqcup \BSW L)$ equals the diagonal map $Q^{\calc}\BSW L\to Q^{\calc}\BSW L\oplus Q^{\calc}\BSW L$. For this, $Q^{\calc}$ converts all the colimits involved in the construction of $\BSW L$ to colimits in $s\vect{+}{n}$, and $Q^{\calc}\phi_1$ and $Q^{\calc}\phi_2$ are both precisely the diagonal map.
\end{proof}

Now let $\overline{\xi}_{\BSW }$ denote the following composite:
%\[W^2X\overset{\phi_{\BSW }}{\to}(W^2X)^{\sqcup2}\overset{\phi_{\BSW }\sqcup \epsilon}{\to}(W^2X)^{\sqcup2}\sqcup(WX) \overset{\epsilon^{\sqcup2}\sqcup\phi_{ch}}{\to}((WX)^{\sqcup2})^{\sqcup2}\overset{\textup{fold}}{\to}(WX)^{\sqcup2}\]
\[\makebox[0mm][r]{$\overline{\xi}_{\BSW }:\ $}\BSW^2L\overset{\phi_{\BSW }}{\to}(\BSW^2L)^{\sqcup2}\overset{a\sqcup b}{\to}(\BSW L)^{\sqcup2}\]
where $a,b:\BSW^2L\to(\BSW L)^{\sqcup2}$ are the composites
\[\xymatrix@R=2mm{
\makebox[0mm][r]{$a:\ $}\BSW^2L\ar[r]^-{\phi_{\BSW }}
&%r1c1
(\BSW^2L)^{\sqcup2}\ar[r]^-{\epsilon^{\sqcup2}}
&%r1c2
(\BSW L)^{\sqcup2}\\
\makebox[0mm][r]{$b:\ $}\BSW^2L\ar[r]^-{\epsilon}
&%r1c1
(\BSW L)\ar[r]^-{\phi_{\BSW }}
&%r1c2
(\BSW L)^{\sqcup2}
}\]
\begin{lem}
The composite 
$\BSW^2L\overset{\overline{\xi}_{\BSW }}{\to}(\BSW L)^{\sqcup2}\to(\BSW L)^{\times2}$ is zero.
\end{lem}
\begin{proof}
This follows from the observation that both composites $(\Id\sqcup0)\overline{\xi}_{\BSW }$ and $(0\sqcup\Id)\overline{\xi}_{\BSW }$ equal $\epsilon:\BSW^2L\to \BSW L$.
\end{proof}
In particular, $\overline{\xi}_{\BSW }$ factors through the smash coproduct, defining a natural map
\[\xi_{\BSW }:\BSW^2L\to (\BSW L)^{\smashcoprod 2}.\]

\begin{lem}\label{LemmaOn xi}
We have $(i\circ \xi_{\calc\PiAlg})=\pi_*(\xi_{\BSW}):\pi_*(\BSW^{p+k+1}L)\to\pi_*((\BSW^{p+k}L)^{\wedge 2})$, where $i$ is the natural isomorphism $(\pi_*(\BSW^{p+k}))^{\wedge 2}\to \pi_*((\BSW^{p+k})^{\wedge 2})$ of proposition \ref{smash coprod}.
\end{lem}
\begin{proof}
In view of the short exact sequences of proposition \ref{smash coprod}, this is equivalent to $(i\circ \overline{\xi}_{\calc\PiAlg})=\pi_*(\overline{\xi}_{\BSW}):\pi_*(\BSW^{p+k+1}L)\to\pi_*((\BSW^{p+k}L)^{\sqcup 2})$, which holds as $(i\circ\phi_{\calc\PiAlg})=\pi_*(\phi_{\BSW})$.
\end{proof}
\begin{lem}
$(d_i)^{\smashcoprod 2}\xi_\BSW =\xi_\BSW d_{i+1}$ for $i\geq1$, and $(d_0)^{\smashcoprod 2}\xi_\BSW = (\xi_\BSW d_{0})\star(\xi_\BSW d_{1})$, so that the map $Q^{\calc}\xi_\BSW $ induces a degree $(-1,0)$ bicomplex map:
\[N_*N_*(Q^{\calc}B^\BSW_{\bullet}L)_{s_2,s_1}\to
  N_*N_*(Q^{\calc}((B^\BSW_{\bullet}L)^{\smashcoprod 2}))_{s_2-1,s_1}.\]
\end{lem}
\begin{proof}
I used to say `it is just like Goerss'... \textbf{give proof.}
\end{proof}
\textbf{Maybe define $\psi_\BSW$} here, as opposed to defining it in a couple of pages, and give a description of it.

\subsection{Definition of the spectral sequence}
\todo{Give the construction of the spectral sequence}{Do it in general, then in the specific cases
\item Conclude with `we will now look at some constructions required to give operations on the sseq.'}
Fix $X\in\calw(n)$, and define $L=Q^{\calU(n)}B^{\calw(n)}X$. Let $\BSWres L$ be the bisimplicial object obtained from the simplicial bar construction using the comonad $\BSW$. Then the spectral sequence we are interested in is the spectral sequence of the bisimplicial object
\[Q^{\calL(n)}\BSWres L.\]
Now $L$ is an almost free object of $\calL(n)$, with $\pi_*L=H_*^{\calU(n)}X$. Thus $\pi_*(\BSWres L)$ is an almost free resolution of $H_*^{\calU(n)}X$ in $s\calw(n+1)$. Moreover, $\pi_*(Q^{\calL(n)}\BSWres L)=Q^{\calw(n+1)}\pi_*(\BSWres L)$, so that
\[E^2_{**}=H_*^{\calw(n+1)}H_*^{\calU(n)}X.\]
\textbf{Some} comments here on the convergence target.

\subsection{The edge homomorphism}
For $X\in s\calw(n)$, the spectral sequence
\[(E_2)^{s_{n+2},\ldots,s_1}_t=(H^*_{\calw(n+1)}(H_*^{\calU(n)}X))^{s_{n+2},\ldots,s_1}_t\implies (H^*_{\calw(n)}X)^{s_{n+2}+s_{n+1},s_n,\ldots,s_1}_t\]
has edge homomorphism
\[(H^*_{\calw(n)}X)^{s_{n+1},\ldots,s_{1}}_t\epi
E_0^{0}(H^*_{\calw(n)}X)^{s_{n+1},\ldots,s_{1}}_t\cong (E_\infty)^{0,s_{n+1},\ldots,s_1}_t\subset (E_2)^{0,s_{n+1},\ldots,s_1}_t\]
which we may compose with the inclusion
\[(E_2)^{0,s_{n+1},\ldots,s_1}_t=(\dual(Q^{\calw(n+1)}H_*^{\calU(n)}X))^{s_{n+1},\ldots,s_1}_t\subseteq (H^*_{\calU(n)}X)^{s_{n+1},\ldots,s_1}_t.\]
\textbf{to form the ``edge composite''}
One may then ask whether this composite commutes with the operations defined in \S\ref{Cohomology Operations for W and U}.
\begin{prop}\label{edgehomproposition}
Suppose that $n\geq1$, $1\leq i \leq s_n$, and not all of $i-1,s_{n-1},\ldots,s_1$ are zero. This composite commutes with the vertical Steenrod operations of proposition \ref{vertical steenrod operations prop} (\textbf{do not need these conditions}):
\[\xymatrix@R=4mm{
(H^*_{\calw(n)}X)^{s_{n+1},\ldots,s_1}_t\ar[r]^-{\Sqv^i}
\ar[d]^-{\textup{edge comp.}}
&%r1c1
(H^*_{\calw(n)}X)^{s_{n+1}+1,s_n+i-1,2s_{n-1},\ldots,2s_1}_{2t+1}\ar[d]^-{\textup{edge comp.}}
\\%r1c2
(H^*_{\calU(n)}X)^{s_{n+1},\ldots,s_1}_t\ar[r]^-{\Sqv^i}
&%r1c1
(H^*_{\calU(n)}X)^{s_{n+1}+1,s_n+i-1,2s_{n-1},\ldots,2s_1}_{2t+1}
}\]
Setting $n=0$, suppose that $2\leq i <t$. The same composite commutes with the (vertical) $\delta$-operations of propositions \ref{operations on goerss homology} and \ref{operations on untable P homology}:
\[\xymatrix@R=4mm{
(H^*_{\calw(0)}X)^{s}_t\ar[r]^-{\delta_i}
\ar[d]^-{\textup{edge comp.}}
&%r1c1
(H^*_{\calw(0)}X)^{s+1}_{t+i+1}\ar[d]^-{\textup{edge comp.}}
\\%r1c2
(H^*_{\calU(0)}X)^{s}_t\ar[r]^-{\delta_i}
&%r1c1
(H^*_{\calU(0)}X)^{s+1}_{t+i+1}
}\]
\end{prop}
\begin{proof}For this proof, we will suppress the `$(n)$' notation, as the proof is the same for all $n\geq0$. We will also suppress all internal gradings, and write $*$ for the grading $s_{n+1}$. The edge homomorphism composite is dual to
\[H_*^{\calw}X:=\pi_*(Q^{\calw}B^{\calw}X)\overset{d_0^\textup{h}}{\from}\pi_0^\textup{h}\pi_*^\textup{v}(Q^{\calL}B^{\BSW} Q^{\calU}B^{\calw}X)\overset{z_{\DASH}}{\from}\pi_*(Q^{\calU}B^{\calw}X)\cong H_*^{\calU}X.\]
Abbreviating further by setting $D:=Q^{\calU}B^{\calw}X$ and $C:=Q^{\calL}B^{\BSW} Q^{\calU}B^{\calw}X$, the map $z_{\DASH}$ sends the class of $x\in ZN_*D$ to $[z_x]\in \pi_0^\textup{h}\pi^\textup{v}_*C$. This assignment does not produce a well defined map $\pi_*D\to N^\textup{h}_0\pi^\textup{v}_*C$, as if $y\in ZN_* D$ represents the same class as $x$, $[z_y]$ need not equal $[z_x]$ in $N^\textup{h}_0\pi^\textup{v}_*C$: we only know that $[z_{x-y}]=0\in N_0^\textup{h}\pi_*^\textup{v}C$.
%SPELLING OUT THAE NULLHOMOTOPY: However, $x-y=d^\textup{v}_0w$ for some $w\in N_{n+1}D$, and $h_w\in N^\textup{h}_0N^\textup{v}_{n+1}C$ satisfies $d_0^\textup{v}h_w=z_{x-y}$, so that $[z_{x-y}]=0\in N_0^\textup{h}\pi_*^\textup{v}C$. 
Although $[z_{x-y}]$ need not equal $[z_{x}-z_{y}]$ in $N_0^\textup{h}\pi_*^\textup{v}C$, the element [$z_{z_{x}-z_{y}}-z_{z_{x-y}}]\in N_1^\textup{h}\pi_*^\textup{v}C$ provides a homotopy:
%
% there is a (horizontal) homotopy between the two, which we will present as the element $z_{z_{x}-z_{y}}\in C_1^\textup{h}\pi_*^\textup{v}C$ in the unnormalized complex:
\[d_0^\textup{h}\!\left([z_{z_{x}-z_{y}}-z_{z_{x-y}}]\right)=[z_x-z_y]-[z_{x-y}]\textup{, \  and \ }d_1^\textup{h}\!\left([z_{z_{x}-z_{y}}-z_{z_{x-y}}]\right)=[z_{{x}-{y}}-z_{{x}-{y}}]=0.\]
We may model the final isomorphism as follows. Write $U^{\calw,\calU}:\calw\to\calU$ for the forgetful functor. For any $V\in\vect{+}{n}$, there is a natural inclusion $F_{\calU}V\to U^{\calw,\calU} F_{\calw}V$ in the category  $\calU$, adjoint to the inclusion $V\to F_{\calw}V$. This morphism yields an inclusion of bar constructions, a weak equivalence $B^{\calU}U^{\calw,\calU}X\to U^{\calw,\calU}B^{\calw}X$ in $s\calU$. Suppressing the forgetful functors, for $X\in\calw$, we have a weak equivalence $Q^{\calU}B^{\calU}X\to Q^{\calU}B^{\calw}X$ inducing the isomorphism. Our conclusion is then that the entire composite $H_*^{\calU}X\to H_*^{\calw}X$ is the map on homotopy induced by the composite 
\[Q^{\calU}B^{\calU}X\to Q^{\calU}B^{\calw}X \epi Q^{\calw}B^{\calw}X,\]
and the operations we are considering are easily understood in relation to this map.
\end{proof}


\end{Composite functor spectral sequences}



\begin{Operations in composite functor spectral sequences}
\section{\textbf{Operations in composite functor spectral sequences}}\label{Operations in composite functor spectral sequences}
Singer \cite{MR2245560} developed an extremely useful theory of products and Steenrod operations in the first quadrant cohomology spectral sequence arising from a bisimplicial cocommutative coalgebra. In the applications we have in mind, however, the bisimplicial object that we have to hand  will \emph{not} be a coalgebra [\textbf{analogy} with \S?]. This will not stop us from applying Singer's techniques after we make the appropriate modifications. The idea is to externalize Singer's operations, so that for every bisimplicial vector space $V$, there are  external operations of type:
\[\E{}{r}{V}{**}{}\to \E{}{r}{S^2V}{**}{}\textup{ \ and \ }S_2\E{}{r}{V}{**}{}\to \E{}{r}{S^2V}{**}{}.\]
These will be compatible with external operations of type: [\textbf{think about me}]
\[H^*(\dual(TV))\to H^*(\dual(TS^2V))\textup{ and }H^*(\dual(TV))\otimes H^*(\dual(TV))\to H^*(\dual(TS^2V)).\]


When $V$ is  in fact a bisimplicial cocommutative coalgebra, one recovers Singer's theory by composing with the map of spectral sequences induced by the coproduct:
\[\E{}{r}{S^2V}{**}{}\to \E{}{r}{V}{**}{}.\]
%When $V$ is a cocommutative coalgebra, the diagonal map $V\to S^2V$ induces a map $E_r^{**}(S^2V)\to E_r^{**}(V)$, postcomposition with which yields Singer's operations, however we will not have access to such structure in what follows. Thus, we will briefly reprise Singer's results in this \emph{external} language, in order that we have them available.
\subsection{External spectral sequence operations following Singer}
We now summarise some key aspects of Singer's work in \cite{MR2245560}, in particular theorems 2.15, 2.16, 2.17 and 2.22, and Proposition 2.21. \textbf{The key construction} is $S^k$ \textbf{(define me)}--- study its effect on filtration and leading filtration term.

For all $p,q\geq0$ and all $r\geq2$, there are well-defined vector space homomorphisms:
\begin{alignat*}{2}
\ExtCohOp^k:\E{}{r}{V}{p,q}{}&\to \E{}{r}{S^2V}{p,q+k}{}&\qquad&\text{if }0\leq k \leq q,\\
\ExtCohOp^k:\E{}{r}{V}{p,q}{}&\to \E{}{r+k-2}{S^2V}{p+k-q,2q}{}&\qquad&\text{if }q\leq k\leq q+r-2,\\
\ExtCohOp^k:\E{}{r}{V}{p,q}{}&\to \E{}{2r-2}{S^2V}{p+k-q,2q}{}&\qquad&\text{if }q+r-2\leq k.
\end{alignat*}
which commute with the differentials (c.f.\ \cite[Theorem 2.17]{MR2245560}), and an external (not necessarily exterior) commutative product operation which satisfies the Liebnitz rule:
\[\ExtCohProd:\E{}{r}{V}{p_1,q_1}{}\otimes \E{}{r}{V}{p_2,q_2}{}\to \E{}{r}{S^2V}{p_1+p_2,q_1+q_2}{}.\]
Those operation with domain $\E{}{2}{V}{}{}$ have codomain $\E{}{2}{S^2V}{}{}$, and we reindex these operations as follows: %are of the form $\E{}{2}{V}{**}{}\to \E{}{2}{S^2V}{**}{}$.
\begin{alignat*}{3}
\vExtCohOp^k&=\,\ExtCohOp^k&:\,&\E{}{2}{V}{p,q}{}\to \E{}{2}{S^2V}{p,q+k}{}&\qquad&\text{if }0\leq k \leq q,\\
\vExtCohOp^k&=0&&&&\text{if }k > q,\\
\hExtCohOp^k&=\ExtCohOp^{q+k}&:\,&\E{}{2}{V}{p,q}{}\to \E{}{2}{S^2V}{p+k,2q}{}&\qquad&\text{if }0\leq k.
\end{alignat*}
Under the identification $\E{}{2}{V}{**}{}=\pi\dhor^{*}\pi\dver^{*}(\dual V)$, the operations $\vExtCohOp^k$ % :\pi\dhor^{p}\pi\dver^{q}(\dual V)\to \pi\dhor^{p}\pi\dver^{q+k}(\dual S^2V)$ 
are  obtained by applying $\pi\dhor^p$ to the linear maps of \S????:
\[\pi\dver^{q}(\dual V)\overset{\ExtCohOp^k}{\to} \pi\dver^{q+k}(S_2\dual V)\to \pi\dver^{q+k}(\dual S^2V)\]
On the other hand, the operation $\hExtCohOp^k$ equals the composite:
\[\pi\dhor^p\pi\dver^q\dual V
\overset{\ExtCohOp^k}{\to} 
\pi\dhor^{p+k}(S_2\pi\dver^*\dual V)^{2q}
\overset{\pi\dhor^{p+k}(\ExtCohProd)}{\to}
\pi\dhor^{p+k}\pi\dver^{2q}S_2\dual V
\to 
\pi\dhor^{p+k}\pi\dver^{2q}\dual S^2 V,
\]
 and the pairing $\ExtCohProd:\E{}{2}{V}{**}{}\to \E{}{2}{S^2V}{**}{}$ equals:
\[S^2\pi\dhor^*\pi\dver^*\dual V
\overset{\ExtCohProd}{\to} 
\pi\dhor^*(S_2\pi\dver^*\dual V)
\overset{\pi\dhor^{*}(\ExtCohProd)}{\to}
\pi\dhor^{*}\pi\dver^{*}S_2\dual V
\to 
\pi\dhor^{*}\pi\dver^{*}\dual S^2 V.
\]

These operations on $E_2$ determine the operations at each $E_r$, $r>2$. The operations $\ExtCohOp^k$ commute with differentials as appropriate. Finally, the $\ExtCohOp^k$ stabilize to well defined maps on $E_\infty$, and there is a commuting diagram
\[\xymatrix@R=4mm{
\E{}{\infty}{V}{p,q}{}\ar[r]^-{\ExtCohOp^k}\ar[d]^-{\pi}
&%r1c1
\E{}{\infty}{S^2V}{p,q+k}{}
\ar[d]^-{\pi}
\\%r1c2
\E{}{0}{H^*(\dual(TV))}{p,q}{}\ar[r]^-{\ExtCohOp^k}&%r2c1
\E{}{0}{H^*(\dual(TS^2V))}{p,q+k}{}%r2c2
}\]
whenever $0\leq k\leq q$, and a commuting diagram
\[\xymatrix@R=4mm{
\E{}{\infty}{V}{p,q}{}\ar[r]^-{\ExtCohOp^k}\ar[d]^-{\pi}
&%r1c1
\E{}{\infty}{S^2V}{p+k-q,2q}{}
\ar[d]^-{\pi}
\\%r1c2
\E{}{0}{H^*(\dual(TV))}{p,q}{}\ar[r]^-{\ExtCohOp^k}&%r2c1
\E{}{0}{H^*(\dual(TS^2V))}{p+k-q,2q}{}%r2c2
}\]
whenever $q\leq k$ (which summarises also Singer's computation of how the $\ExtCohOp^k$ interact with the filtration on cohomology).
%Moreover, for $r\geq2$ and $p,q,k\geq0$ given, if $\alpha\in E_r^{p,q}$, then $\Sq_\textup{ext}^k\alpha$ and $\Sq_\textup{ext}^kd_r\alpha$ both survive to $E_t$, where
%\[t=\begin{cases}
%r,&\textup{if }0\leq k \leq q-r+1;\\
%2r+k-q-1,&\textup{if }q-r+1\leq k\leq q;
%\\2r-1,&\textup{if }q\leq k,
%\end{cases}
%\]
%and in $E_t$ we have the relation $d_t\Sq^k_textup{ext}\alpha=\Sq^k_textup{ext}d_r\alpha$.

If $V$ admits an augmentation $d\dhor^0:V\to V_{-1}$, then $\ExtCohOp^k$ and $\ExtCohProd$ commute with the induced maps on cohomotopy, so that there are commuting diagrams %$\dual(d\dhor^0)$
\[\xymatrix@R=4mm{
\pi^{m}(\dual(V_{-1}))\ar[r]^-{\ExtCohOp^k}
\ar[d]%^-{\lambda^*}
&%r1c1
\pi^{m+k}(\dual(S^2V_{-1}))\ar[d]%^-{\dual(S^2\lambda)}
&%r1c3
S_2\pi^{*}(\dual(V_{-1}))\ar[r]^-{\ExtCohProd}
\ar[d]%^-{\lambda^*}
&%r1c1
\pi^{*}(\dual(S^2V_{-1}))\ar[d]%^-{\dual(S^2\lambda)}
\\%r1c5
H^{m}(\dual(TV))\ar[r]^-{\ExtCohOp^k}
&%r2c1
H^{m+k}(\dual(TS^2V))
&%r2c3
S_2H^{*}(\dual(TV))\ar[r]^-{\ExtCohProd}
&%r2c1
H^{*}(\dual(TS^2V))
}\]


%Moreover, if the bisimplicial object $V$ admits an augmentation $\lambda:V_{0\bullet}\to U_{\bullet}$, (\textbf{didn't I} do away with $\bullet$?) then %such that the natural map $\lambda_*:CV\to CU$ is a weak equivalence,
%Singer proves the following relationship between the spectral sequence operations and the operations of \S\ref{generic coh ops section}:
%\begin{thm}[Proposition 2.1]
%The $\ExtCohOp^k$ and $\ExtCohProd$ commute with $\lambda^*$, so that there are commuting diagrams
%\[\xymatrix@R=4mm{
%H^{m}(\dual(CU))\ar[r]^-{\ExtCohOp^k}
%\ar[d]^-{\lambda^*}
%&%r1c1
%H^{m+k}(\dual(CS^2U))\ar[d]^-{\dual(S^2\lambda)}
%&%r1c3
%H^{m_1}(\dual(CU))\otimes H^{m_2}(\dual(CU))\ar[r]^-{\ExtCohProd}
%\ar[d]^-{\lambda^*}
%&%r1c1
%H^{m_1+m_2}(\dual(CS^2U))\ar[d]^-{\dual(S^2\lambda)}
%\\%r1c5
%H^{m}(\dual(CV))\ar[r]^-{\ExtCohOp^k}
%&%r2c1
%H^{m+k}(\dual(CS^2V))
%&%r2c3
%H^{m_1}(\dual(CV))\otimes H^{m_2}(\dual(CV))\ar[r]^-{\ExtCohProd}
%&%r2c1
%H^{m_1+m_2}(\dual(CS^2V))
%}\]
%\end{thm}
%
%Singer separates the operations $\ExtCohOp^k:E^{p,q}_2\to E_{2}$ according to the pattern of change in the spectral sequence coordinates of their output. Indeed, he defines
%\begin{alignat*}{2}
%\vExtCohOp^k=\ExtCohOp^k:E^{p,q}_2(V)&\to E^{p,q+k}_2(S^2V)&\qquad&\text{if }0\leq k \leq q,\\
%\hExtCohOp^k=\ExtCohOp^{q+k}:E^{p,q}_2(V)&\to E^{p+k,2q}_2(S^2V)&\qquad&\text{if }0\leq k \leq p,
%\end{alignat*}
%and gives explicit formulae for these $E_2$ operations, and for the product at $E_2$ [Singer 2.23].








\subsection{Application to composite functor spectral sequences}


In order to use Singer's constructions in the present work, we will use the map of double complexes:
\[\psi_{\BSW}=j_{\calL(n)}\circ Q^{\calL(n)}\xi_\BSW:N_{p+1}N_q(Q^{\calL(n)}\BSWres L)_{s_n,\ldots,s_1}^{t+1}\to N_{p}N_q(S^2(Q^{\calL(n)}\BSWres L))_{s_n,\ldots,s_1}^{t}\]
to define a spectral sequence map
\[E_2^{p,q}(S^2(Q^{\calL(n)}\BSWres L))^{s_n,\ldots,s_1}_t\overset{(\psi_\BSW)^*}{\to} E_2^{p+1,q}(Q^{\calL(n)}\BSWres L)^{s_n,\ldots,s_1}_{t+1}\]
We may then define internal spectral sequence operations
\begin{alignat*}{2}
\Sq^k=\psi_{\BSW}^*\circ\ExtCohOp^{k-1}:(E^{p,q}_r)^{s_n,\ldots,s_1}_t&\to (E^{p+1,q+k-1}_r)^{2s_n,\ldots,2s_1}_{2t+1}&\qquad&\text{if }0\leq k-1 \leq q,\\
\Sq^k=\psi_{\BSW}^*\circ\ExtCohOp^{k-1}:(E^{p,q}_r)^{s_n,\ldots,s_1}_t&\to (E^{p+k-q,2q}_{r+k-q-1})^{2s_n,\ldots,2s_1}_{2t+1}&\qquad&\text{if }q\leq k-1\leq q+r-2,\\
\Sq^k=\psi_{\BSW}^*\circ\ExtCohOp^{k-1}:(E^{p,q}_r)^{s_n,\ldots,s_1}_t&\to (E^{p+k-q,2q}_{2r-2})^{2s_n,\ldots,2s_1}_{2t+1}&\qquad&\text{if }q+r-2\leq k-1.
\end{alignat*}
Which, at $E_2$, we may write (dropping internal degrees) as \textbf{(precisely?:)}
\begin{alignat*}{2}
\Sqv^k=\psi_{\BSW}^*\circ\vExtCohOp^{k-1}= \psi_{\BSW}^*\circ\ExtCohOp^{k-1}:E^{p,q}_2&\to E^{p+1,q+k-1}_2&\qquad&\text{if }0\leq k-1 \leq q,\\
\Sqh^k=\psi_{\BSW}^*\circ\hExtCohOp^{k-1}= \psi_{\BSW}^*\circ\ExtCohOp^{q+k-1}:E^{p,q}_2&\to E^{p+k,2q}_2&\qquad&\text{if }0\leq k-1 \leq p,
\end{alignat*}

And similarly, a pairing
\[\mu=\psi_{\BSW}^*\circ\ExtCohProd:(E^{p,q}_r)^{s_n,\ldots,s_1}_t\otimes (E^{p',q'}_r)^{s'_n,\ldots,s'_1}_{t'}\to (E^{p+p'+1,q+q'}_r)^{s_n+s'_n,\ldots,s_1+s'_1}_{t+t'+1}\]
The reader might now guess the key results:
\begin{thm}\label{E2CompFuncLieOperationsID}
At $E_2$, the operations $\Sqh^k$ and $\mu$ are equal to the $H^*_{\calw(n+1)}$-cohomology operations on $E_2=H^*_{\calw(n+1)}H_*^{\calU(n)}X$ of the same name, as defined in [\S?].
\end{thm}
\begin{thm}\label{E2CompFuncKosOperationsID}
At $E_2$, the operations $\Sqv^k$ are equal to the $H^*_{\calw(n+1)}$-cohomology operations on $E_2=H^*_{\calw(n+1)}H_*^{\calU(n)}X$ of the same name, as defined in [\S?].
\end{thm}
\begin{thm}\label{EInftyCompFuncOperationsID}
At $E_\infty$, the operations $\Sq^k$ are compatible with the operations $\Sqh^k$ defined on the target $H^*_{\calw(n)}(X)$, as defined in [\S?].
\end{thm} 
\subsection{Proof of theorems \ref{E2CompFuncLieOperationsID}-\ref{EInftyCompFuncOperationsID}}

\begin{proof}[Proof of \ref{E2CompFuncLieOperationsID}]
We will prove this theorem in various parts. Firstly, we will work on the operations $\Sqh^k$ and $\mu$ at $E_2$.
For this we will rely on a commuting diagram, where we write $L=Q^{\calU(n)}X^c\in s\calL(n)$ and $Y=\BSWres L\in ss\calL(n)$:
\[\xymatrix@C=13mm@R=4mm{
(N_p^\textup{h}\pi_*^{\textup{v}}Q^{\calL(n)}Y)^{\otimes2}&%r1c1
(N_p^\textup{h}Q^{\calw(n+1)}\pi_*^{\textup{v}}Y)^{\otimes2}
\ar[l]_-{(N^\textup{h}_*(\gamma))^{\otimes 2}}
\\%r1c2
N_{p+k-1}^\textup{h}((\pi_*^{\textup{v}}Q^{\calL(n)}Y)^{\otimes2})
\ar[u]^-{D_{p-k+1}^\textup{h}}
&%r2c1
N_{p+k-1}^\textup{h}((Q^{\calw(n+1)}\pi_*^{\textup{v}}Y)^{\otimes2})
\ar[u]^-{D_{p-k+1}}
\ar[l]_-{N^\textup{h}_*(\gamma^{\otimes 2})}
\\%r2c2
N_{p+k-1}^\textup{h}(\pi_*^{\textup{v}}((Q^{\calL(n)}Y)^{\otimes2}))
\ar[u]^-{N^\textup{h}_*\pi_*^\textup{v}(D_0^\textup{v})}
&%r3c1
\\%r3c2
N^\textup{h}_{p+k}(\pi_*^{\textup{v}} Q^{\calL(n)}Y)
\ar[u]^-{\pi_*^\textup{v}(\psi_{\BSW})}
&%r4c1
N^\textup{h}_{p+k}(Q^{\calw(n+1)}\pi_*^{\textup{v}} Y)
\ar[uu]_-{\psi_{\calw(n+1)}}
\ar[l]_-{N^\textup{h}_*(\gamma)}
%r4c2
}\]
In this diagram, all of the horizontal maps are isomorphisms, and the two vertical composites are chain maps. (by Singer's Theorem 2.23),  the left hand vertical composite is that used to define the horizontal operations $\Sq^{k}_h$ on $E_2$ (\textbf{(right?)}). On the other hand, the right vertical is used to define the operations on the $\calw(n+1)$-cohomology groups with which the $E_2$-page can be identified. The same proof will apply for $\mu$, if we replace the maps $D_j$ in the top square with $D_0^\textup{h}$, and modify the subscripts $p$ as needed.

What remains is to prove that the bottom square commutes. It may be expanded into the \emph{outer square} in the following larger commuting diagram, which we will abbreviate by omitting the subscripts on indecomposables functors:
%\[
%\def\objectstyle{\scriptstyle}
%\xymatrix@R=8mm@C=13mm@!0{
%%&%r1c1
%%(\pi (QW^{n+1}L))^{\otimes2}\ar[rrrrrr];[]_-{=}
%%&%r1c2
%%&%r1c3
%%&%r1c4
%%&%r1c5
%%&%r1c6
%%&%r1c7
%%(\pi (QW^{n+1}L))^{\otimes2}\\\\%r1c8
%&%r2c1
%\pi((QW^{p+k}L)^{\otimes2}) \ar[rrr]_-{D_0^{\textup{v}}}
%&%r2c2
%&%r2c3
%&%r2c4
%(\pi (QW^{p+k}L))^{\otimes2}&%r2c5
%&%r2c6
%&%r2c7
%(Q\pi(W^{p+k}L))^{\otimes2} \ar[lll]^-{\gamma^{\otimes2}}\\\\%r2c8
%&%r3c1
%\pi Q((W^{p+k}L)^{\smashcoprod 2})\ar[uu]^-{\pi(j)}
%\ar[dl]
%&%r3c2
%&%r3c3
%&%r3c4
%Q\pi((W^{p+k}L)^{\smashcoprod 2})\ar[lll]_-{\gamma}\ar[dl]
%&%r3c5
%&%r3c6
%&%r3c7
%Q((\pi( W^{p+k}L))^{\smashcoprod 2})\ar[dl]
%\ar[uu]_-{j}
%\ar[lll]_-{Q(i)}
%\\
%%r3c1
%\pi Q((W^{p+k}L)^{\sqcup 2})
%&%r3c2
%&%r3c3
%&%r3c4
%Q\pi((W^{p+k}L)^{\sqcup 2})\ar[lll]_-(.3){\gamma}
%&%r3c5
%&%r3c6
%&%r3c7
%Q((\pi( W^{p+k}L))^{\sqcup 2})
%\ar[lll]_-(.3){Q(i)}
%%r3c8
%\\\\
%&%r3c1
%\pi Q(W^{p+k+1}L)
%\ar@/_.5em/[uuu]_-(.25){\pi Q(\xi_{\textup{ch}})}
%\ar[uul]^-(.65){\pi Q(\overline{\xi}_{\textup{ch}})}
%&%r3c2
%&%r3c3
%&%r3c4
%Q\pi(W^{p+k+1}L)\ar[lll]_-{\gamma}
%\ar@/_.5em/[uuu]_-(.25){Q\pi(\xi_{\textup{ch}})}
%\ar[uul]^-(.65){Q\pi(\overline{\xi}_{\textup{ch}})}
%&%r3c5
%&%r3c6
%&%r3c7
%Q(\pi( W^{p+k+1}L))
%\ar@/_.5em/[uuu]_-(.25){Q(\xi_{\calw(n+1)})}
%\ar[lll]_-{=}
%\ar[uul]^-(.65){Q(\overline{\xi}_{\calw(n+1)})}
%}\]
\[
\def\objectstyle{\scriptstyle}
\xymatrix@R=6mm@C=13mm@!0{
%&%r1c1
%(\pi (QW^{n+1}L))^{\otimes2}\ar[rrrrrr];[]_-{=}
%&%r1c2
%&%r1c3
%&%r1c4
%&%r1c5
%&%r1c6
%&%r1c7
%(\pi (QW^{n+1}L))^{\otimes2}\\\\%r1c8
&%r2c1
\pi((Q\BSW^{p+k}L)^{\otimes2}) \ar[rrr]_-{D_0^{\textup{v}}}
&%r2c2
&%r2c3
&%r2c4
(\pi (Q\BSW^{p+k}L))^{\otimes2}&%r2c5
&%r2c6
&%r2c7
(Q\pi(\BSW^{p+k}L))^{\otimes2} \ar[lll]^-{\gamma^{\otimes2}}\\\\%r2c8
&%r3c1
\pi Q((\BSW^{p+k}L)^{\smashcoprod 2})\ar[uu]^-{\pi(j)}
&%r3c2
&%r3c3
&%r3c4
Q\pi((\BSW^{p+k}L)^{\smashcoprod 2})\ar[lll]_-{\gamma}
&%r3c5
&%r3c6
&%r3c7
Q((\pi( \BSW^{p+k}L))^{\smashcoprod 2})
\ar[uu]_-{j}
\ar[lll]_-{Q(i)}
\\\\
&%r3c1
\pi Q(\BSW^{p+k+1}L)
\ar[uu]_-{\pi Q(\xi_{\BSW})}
&%r3c2
&%r3c3
&%r3c4
Q\pi(\BSW^{p+k+1}L)\ar[lll]_-{\gamma}
\ar[uu]_-{Q\pi(\xi_{\BSW})}
&%r3c5
&%r3c6
&%r3c7
Q(\pi( \BSW^{p+k+1}L))
\ar[uu]_-{Q(\xi_{\calw(n+1)})}
\ar[lll]_-{=}
}\]
The bottom left square commutes simply by naturality of $\gamma$, while the bottom right square is an instance of lemma \ref{LemmaOn xi}. What remains is to check that the hexagon commutes. For notational convenience, write $A=\BSW^{p+k}L$, $\textup{br}:(\pi_* A)^{\otimes 2}\to (\pi_* A)^{\smashcoprod 2}$ for the bracket $a\otimes b\mapsto [a,b]$ on homotopy, and similarly write $\textup{br}:A^{\otimes 2}\to A^{\smashcoprod 2}$ for the chain level bracket. Now any element of $Q((\pi_* A)^{\smashcoprod 2})$ is the class of an expression $\textup{br}(\overline{x}\otimes\overline{y})+E$ where $\overline{x},\overline{y}\in\pi_* A$ are represented by $x,y\in N_*A$, and $E\in(\pi_* A)^{\smashcoprod 2}$ is a sum of terms which are $\lambda$-operations applied to 3-fold brackets of elements of $\pi_* A$ \textbf{[explain this better]}. The map $Q(i)$ is induced by the Eilenberg-Mac Lane map shuffle map $\nabla$ [](\textbf{what exactly?}), and
\[\textup{br}(\overline{x}\otimes\overline{y})\overset{\gamma\circ Q(i)}{\mapsto}\overline{\textup{br}(\nabla(x\otimes y))}\overset{\pi_*(j)}{\mapsto}\overline{(\nabla(x\otimes y))}\overset{D_0}{\mapsto}\overline{x}\otimes \overline{y}.\]
The last mapping follows from the fact that $D_0\circ\nabla=\Id$, as $\{D_k\}$ is special.
Similarly, our understanding of the construction of $\lambda$-operations and iterated brackets shows that $E$ is annihilated by $\pi_*(j)\circ\gamma\circ Q(i)$. As $E$ is also annihilated by $j$, and $j(\textup{br}(\overline{x}\otimes\overline{y})) =\overline{x}\otimes\overline{y}$, the hexagon commutes. \textbf{This is poorly written.}
\end{proof}
\begin{proof}[Proof of \ref{E2CompFuncKosOperationsID}]
We must identify the operations $\Sqv^i=\psi_{\BSW}^*\circ\vExtCohOp^{i-1}$ with the $\calw(n+1)$-cohomology operations $\Sqv^i$ defined in \S\ref{section: vertical Koszul operations n positive} using the map $\gamma_i$. However, the $\gamma_i$ are defined on the bar construction, while $\psi_{\BSW}^*$ is defined on the Blanc-Stover resolution. In order to make the comparison, we will need to choose a weak equivalence of resolutions of $\pi_* L$ in $s\calw(n+1)$
\[\chi:B^{\calw(n+1)}_{\bullet}\pi_*L\to \pi_*(\BSWres L).\]
We will use Miller's techniques [Miller p. 55] to define such a $\chi$ explicitly. For brevity, write $\mathbb{B}$ for the object $B_i^{\calw(n+1)}\pi_*L\in s\calw(n+1)$. Write $V_i$ for the subspace $(F^{\calw(n+1)})^{i}\subset \mathbb{B}$ of generators. For each $m\geq0$, write $F_mV_i$ for the subspace of $V_i$ consisting of degeneracies of elements of $V_j$ for $j\leq m$. Then write $F_m\mathbb{B}$ for the subobject of $\mathbb{B}$ which is almost free on the subspaces $F_mV_i$.

Now for each $m$, $V_m$ splits into a direct sum
$V_m=V'_m\oplus F_{m-1}V_m$,
where $V'_m\subset N_m^\textup{h}\mathbb{B}$. To make this classical insight explicit, consider the endomorphism $c:\mathbb{B}_m\to \mathbb{B}_m$ introduced in the proof of lemma \ref{lemma on homology class repd by normalized generator}, an idempotent with image $N_m^\textup{h}\mathbb{B}$, preserving the subspace $V_m$. As $\im((\Id-c)|_{V_m})\subset F_{m-1}V_m$, we may define $V'_m=c(V_m)$. \textbf{(read this paragraph sometime)}

In order to define $\chi$, we recursively define its restriction to the skeleta $F_m\mathbb{B}$. In order to extend a map $\chi_{m-1}:F_{m-1}\mathbb{B}\to \pi_*(\BSWres L)$ to a map $\chi_m:F_m\mathbb{B}\to \pi_*(\BSWres L)$, we need only to specify the values of $\chi_m$ on $V'_m$, which is to choose a lift in the diagram
\[\xymatrix@R=4mm{
V'_m\ar@{-->}[r]^-{\chi_m}
\ar[d]^-{d_0^\textup{h}}
&%r1c1
N^\textup{h}_m\pi_*(\BSWres L)\ar[d]^-{d_0^\textup{h}}
\\%r1c2
ZN^\textup{h}_{m-1}\mathbb{B}\ar[r]^-{\chi_{m-1}}
&%r2c1
ZN^\textup{h}_{m-1}\pi_*(\BSWres L)%r2c2
}\]
However, we will need to record some chain level information for what follows. Recursively for $m\geq-1$ [\textbf{say} ``view these as augmented simplicial objects, and start with $\chi_{-1}$ the identity''], we will construct functions $\overline{\chi}_m:V'_m\to ZN^\textup{v}_*\BSWres_m L$, with the property that $\im(\overline{\chi}_m)$ is contained in the span of the classes $z_{w}$ for $w\in ZN^\textup{v}_*\BSWres_{m-1}L$. In order to do this, one may choose a basis of $V'_m$, and then for each basis element $v$, choose a $\calw(n+1)$-expression $e(s^\textup{h}_{\alpha_j}w_j)$ for $v$, with $w_j\in V'_{n_j}$ for integers $n_j\leq m-1$ and degeneracy operators $s_{\alpha_j}:V'_{n_j}\to V_{m-1}$. Then we will define
\[\overline{\chi}_m(v)=z_{e^\textup{rep}(s^\textup{h}_{\alpha_j}\overline{\chi}_{n_j}(w_j))},\]
where by $e^\textup{rep}$, we mean the cycle in $ZN_*^\textup{v}(\BSWres_mL)$ obtained from the cycles $s^\textup{h}_{\alpha_j}\overline{\chi}_{n_j}(w_j)\in ZN_*\BSWres_{m-1}L$ using the formulae of [Curtis' explicit formulae], according to the chosen expression $e$ for $v$.
Now the class of $\overline{\chi}_m(v)$ in $\pi_*(\BSWres_mL)$ is in fact in $N^\textup{h}_m\pi_*(\BSWres L)$, since (c.f.\ [Stover lemma 2.7]) for $1\leq i \leq m$:
\[d_i^\textup{h}\overline{\chi}_m(v) =z_{d^\textup{h}_{i-1}e^\textup{rep}(s^\textup{h}_{\alpha_j}\overline{\chi}_{n_j}(w_j))}\]
where by construction $e^\textup{rep}(s^\textup{h}_{\alpha_j}\overline{\chi}_{n_j}(w_j))\in ZN^\textup{v}_*(\BSWres_{m-1} L)$ represents $\chi_{m-1}(d_{0}^\textup{h}(v))\in ZN_{m-1}^\textup{h}\pi_*(\BSWres  L)$. As such, the subscript $d^\textup{h}_{i-1}e^\textup{rep}(s^\textup{h}_{\alpha_j}\overline{\chi}_{n_j}(w_j))\in ZN^\textup{v}_*(\BSWres_{m-2} L)$ represents $0\in\pi_*(\BSWres_{m-2}L)$. Choosing a nullhomotopy $H\in N^\textup{v}_{*+1}(\BSWres_{m-2}L)$, $h_H$ is a null-homotopy of $d_i^\textup{h}\overline{\chi}_m(v)$, showing that the class of  $\overline{\chi}_m(v)$ lies in $N^\textup{h}_m\pi_*(\BSWres L)$.
%\[e^\textup{rep}(s^\textup{h}_{\alpha_j}\overline{\chi}_{n_j}(w_j))=\chi_{m-1}(d_{0}^\textup{h}(v))\in ZN^\textup{v}_*\]
Thus $\overline{\chi}_m$ does induce a map $\chi_m:V'_m\to N_m^\textup{h}\pi_*(\BSWres L)$, completing the construction of $\chi$.






Recall that the operations of \S\ref{section: vertical Koszul operations n positive} are the maps induced on cohomology by the degree -1 endomorphism $\gamma_i$ of the chain complex $N_*(Q^{\calw(n+1)} B^{\calw(n+1)}_{\bullet}\pi_*L$:
\[\gamma_i:N_{s_{n+1}+1}(Q^{\calw(n+1)} B^{\calw(n+1)}_{\bullet}\pi_*L)^{2t+1}_{s_{n+1}+i-1,\ldots}\to N_{s_{n+1}}(Q^{\calw(n+1)} B^{\calw(n+1)}_{\bullet}\pi_*L)^{t}_{s_{n+1},\ldots}.\]
If we write $V=Q^{\calL(n)}\BSWres L$ for the double complex yielding the spectral sequence, our goal is to identify these operations with the spectral sequence operations
\[\psi_{\BSW}^*\circ\vExtCohOp^{i-1}:\left(E^{p,q}_2(V)\overset{\vExtCohOp^{i-1}}{\to} E^{p,q+i-1}_2(S^2V)\overset{\psi_{\BSW}^*}{\to}E^{p+1,q+i-1}_2(V)\right)\]
using the equivalence $Q\chi$ in $s\vect{}{}$ induced by $\chi$:%a weak equivalence $Q^{\calw(n+1)}(\chi)$ in , which we denote:
\[Q\chi:\left(Q^{\calw(n+1)}B^{\calw(n+1)}_{\bullet} \pi_*L\overset{Q^{\calw(n+1)}(\chi)}{\to} Q^{\calw(n+1)}\pi_*(\BSWres )\overset{\cong }{\to}\pi_*^\textup{v}(V)\right).\]
%Now, $\gamma_i$ induces a map on $\calw(n+1)$-homology, whose dual we would like to identify with the composite:

The composite $\psi_{\BSW}^*\circ\vExtCohOp^{i-1}$ is itself the dual of an operator:
\[\left(E_{p,q}^2(V)\overset{(\vExtCohOp^{i-1})^*}{\from} E_{p,q+i-1}^2(S^2V)\overset{\psi_{\BSW}}{\from}E_{p+1,q+i-1}^2(V)\right),\]
as Singer gives the formula on $E^2$ \textbf{and it is easy} for vertical operations, and we talked about the dual operations $(\vExtCohOp^{i-1})^*$ like this in []. Thus, what we are to prove is equivalent to a commuting diagram, for $1\leq i\leq q$:
\[\xymatrix@R=4mm{
N_{p+1}(QB\pi_*L)_{q+i-1}\ar[rr]^-{(\gamma_i)_*}
\ar[d]^-{Q\chi}
&%r1c1
&%r1c2
N_p(QB\pi_*L)_q\ar[d]^-{Q\chi}
\\%r1c3
N_{p+1}(\pi_*(V))_{q+i-1}\ar[r]^-{\psi_{\BSW}}
&%r2c1
N_{p}(\pi_*(S^2V))_{q+i-1}\ar[r]^-{(\vExtCohOp^{i-1})^*}
&%r2c2
N_p(\pi_*(V))_q%r2c3
}\]

Given the description of the operations $(\vExtCohOp^{i-1})^*$, it will suffice to show that the composite
\[(QB_{p+1}\pi_*L)_{q+i-1}\overset{Q\chi}{\to}(\pi_*(Q^{\calL(n)}\BSWres_{p+1}L))_{q+i-1}\overset{\psi_{\BSW}}{\to}(\pi_*(S^2(Q^{\calL(n)}\BSWres_pL)))_{q+i-1}\]
equals the sum of the composites, for $1\leq i \leq q$ (and not all of ... are zero?): (these are the non-top)
\[(QB_{p+1}\pi_* L)_{q+i-1}\overset{(\gamma_i)_*}{\to} (QB_{p}\pi_* L)_q\overset{Q\chi}{\to} (\pi_*Q\BSWres_{p}L)_q\overset{\sigma_{i-1}}{\to} (\pi_*(S^2(Q^{\calL(n)}\BSWres_{p}L)))_{q+i-1},\]
added to the composite:
\[(QB_{p+1}\pi_* L)_{q+i-1}\overset{\psi_{\calw(n+1)}}{\to}(S^2(QB_{p}\pi_* L))_{q+i-1}\overset{S^2(Q\chi)}{\to}
(S^2(\pi_*Q\BSWres_pL))_{q+i-1}\overset{\widetilde{\nabla}}{\to}
(\pi_*(S^2(Q^{\calL(n)}\BSWres_{p}L)))_{q+i-1},\]


By lemma \ref{lemma on homology class repd by normalized generator}, we may represent any homology class of interest by an element $e\in V'_{p+1}$. We name the element in question:
\[e:=f^{p+1}_{p+1}(u_j)\in (V'_{p+1})_{q+i-1}\subset F^{\calw(n+1)}V_p\textup{ for various }u_j\in V_{p}\]
and if $f_{p+1}^\textup{rep}$ is a $\calL(n)$-expression which implements the $\calw(n+1)$-expression $f_{p+1}$ at the chain level:
\begin{alignat*}{2}
%e&:=f^{p+1}_{p+1}\cdots f^1_1x_0\in V'_{p+1}\\
\chi(e)&=z_{f_{p+1}^\textup{rep}\chi(u_j)}\\
\psi_{\BSW}(\chi(e))&=\quadratic_{\calL(n)}(f_{p+1}^\textup{rep})(\chi(u_j))
\end{alignat*}
since, by construction, $\chi(u_j)$ is contained in the span of the $z$ classes.  Taking $\quadratic(f_{p+1}^\textup{rep})$, we extract the part of $f_{p+1}^\textup{rep}$ corresponding to the quadratic grading 2 part of $f_{p+1}$, in $\quadgrad{2}F^{\calw(n+1)}$. That is, we may write $f_{p+1}\in F^{\calw(n+1)}V_p$ as
\[f_{p+1}=\quadratic_{\calw(n+1)}(f)(u_j)+\textstyle\sum_{1\leq i\leq q} \lambda_{i-1}(\gamma_ie) + w\in F^{\calw(n+1)}V_p,\]
where $w\in F^{\calw(n+1)}V_p$ is the quadratic grading$\makebox[0cm][l]{}\neq2$ part of $f_{p+1}$, if we view $\quadratic_{\calw(n+1)}(f)\in S^2V_p$ as an element of $F^{\calw(n+1)}V_p$ via the inclusion $F_{\calL(n+1)}V_p\to F^{\calw(n+1)}V_p$. Then
%\[\quadratic_{\calL(n)}(f_{p+1}^\textup{rep})(\chi(u_k ))=\quadratic_{\calL(n)} (\textup{!}(\widetilde{\nabla}(y)(\chi(u_k))))+\sum_j \quadratic_{\calL(n)}(\lambda^\textup{rep}_{n_j})(\chi(x_j))\in \pi_*(S^2(Q^{\calL(n)}G_pL))\]
\begin{alignat*}{2}
\psi_\BSW(\chi(e))&=\quadratic_{\calL(n)} \left(\widetilde{\nabla}(\quadratic_{\calw(n+1)}(f))(\chi(u_j))+\sum \sigma_{i-1}(\gamma_if)(\chi(u_j))\right)\\
&=\widetilde{\nabla}(\quadratic_{\calw(n+1)}(f))(\chi(u_j))+\sum \sigma_{i-1}(\gamma_if)(\chi(u_j))\\
&=\left(\widetilde{\nabla}\circ S^2(Q\chi)\circ \psi_{\calw(n+1)}+\sum\sigma_{i-1}\circ Q\chi\circ(\gamma_i)_*\right)(e)
\end{alignat*}
Where we have removed the function $\quadratic_{\calL(n)}$ as its argument already has quadratic grading 2.
\end{proof}

\begin{proof}[Proof of \ref{EInftyCompFuncOperationsID}]
Given the general theory of Singer's external operations, we only need to show that the diagram of chain complexes
\[\xymatrix@R=4mm{
T_{m}(Q^{\calL(n)}\BSWres L)
\ar[r]^-{\psi_{\BSW}}
\ar[d]^-{\epsilon}
&%r1c1
T_{m-1}(S^2(Q^{\calL(n)}\BSWres L))\ar[d]^-{\epsilon}
\\%r1c3
N_{m}(Q^{\calL(n)}L)
\ar[r]^-{\psi_{\calw(n)}}
&%r1c1
N_{m-1}(S^2(Q^{\calL(n)} L))
}\]
commutes up to homotopy (recall that $\psi_{\BSW}$ reduces filtration by one). The required chain homotopy $\Phi$ is constructed as follows. Let $\Phi$ be zero except on 
\[N_0^\textup{h}N_*^\textup{v}(Q\BSWres L)=N_*Q^{\calL(n)}\BSW Q^{\calU(n)}B^{\calw(n)}X.\]
The generic element of $N_mQ^{\calL(n)}\BSW Q^{\calU(n)}B^{\calw(n)}X$ may be written as a sum of terms
\[z_{f^0(g^1_{i_1}(h^2_{i_1i_2}))}\textup{\quad where $f^0(g^1_{i_1}(h^2_{i_1i_2}))\in ZN_m Q^{\calU(n)}B^{\calw(n)}X$}\]
and further terms
\[h_{f^0(g^1_{i_1}(h^2_{i_1i_2}))}\textup{\quad where $f^0(g^1_{i_1}(h^2_{i_1i_2}))\in N_{m} Q^{\calU(n)}B^{\calw(n)}X$}.\]
In these descriptions, $f$ is an operator in $F^{\calL(n)}$, while $g$, $h$, etc.\ are operators in $F^{\calw(n)}$.
For brevity we will write $K=k_{f^0(g^1_{i_1}(h^2_{i_1i_2}))}$ for either of $z_{f^0(g^1_{i_1}(h^2_{i_1i_2}))}$ and $h_{f^0(g^1_{i_1}(h^2_{i_1i_2}))}$.
We then define a homotopy 
\[\Phi:T_*(Q^{\calL(n)}\BSWres L)\to N_*(S^2(Q^{\calw(n)}B^{\calw(n)}X))\]
by the formula
\[K\mapsto q(f)(g^1_{i_1}(h^2_{i_1i_2})).\]
This definition makes sense (and yields a non-trivial map) because $f^0$ is an operator in $Q^{\calU(n)}F^{\calw(n)}=F^{\calL(n)}$.
The chain map $d\Phi+\Phi d$ is a sum of three terms:
\begin{enumerate}\squishlist
\setlength{\parindent}{.25in}
\item[(a)] $d\circ\Phi:N_0^\textup{fil}N_*^\textup{int}(Q\BSWres L)\overset{\Phi}{\to} N_*(S^2(QL))\overset{d}{\to} N_{*-1}(S^2(QL))$ \textbf{sub in for X?}
\item[(b)] $\Phi\circ d^\textup{int}:N_0^\textup{fil}N_{*}^\textup{int}(Q\BSWres L)\overset{d^\textup{int}}{\to} N_0^\textup{fil}N_{*-1}^\textup{int}(Q\BSWres L)\overset{\Phi}{\to} N_{*-1}(S^2(QL))$
\item[(c)] $\Phi\circ d^\textup{fil}:N_1^\textup{fil}N_{*-1}^\textup{int}(Q\BSWres L)\overset{d^\textup{fil}}{\to} N_0^\textup{fil}N_{*-1}^\textup{int}(Q\BSWres L)\overset{\Phi}{\to} N_{*-1}(S^2(QL))$
\end{enumerate}
Let us start by identifying (a) and (b) applied to $K$:
\begin{alignat*}{2}
(d\circ\Phi)(K)&=d( q(f)(g^1_{i_1}(h^2_{i_1i_2})))\\
&=q(f)(g^0_{i_1}(h^1_{i_1i_2}))\\
&=q(f)(u(g^0_{i_1})(h^1_{i_1i_2}))\quad\textup{(as we calculate in $S^2(QL)$)}\\
(\Phi\circ d^\textup{int})(K)&=\begin{cases}
\Phi( k_{f^0(g^0_{i_1})(h^1_{i_1i_2}))}),&\textup{if `$k$' stands for `$h$'};\\
\Phi(0),&\textup{if `$k$' stands for `$z$'};
%\\,&\textup{if }
\end{cases}\\
&=q(f(g_{i_1}))(h^1_{i_1i_2})\quad\textup{(in either case).}
\end{alignat*}
By (the definition of quadratic part? calculation for this case? Decide.), the sum of these two terms equals $q(uf(g_{i_1}))(h^1_{i_1i_2})$, which is exactly the formula for $(\psi_{\calw(n)}\circ\epsilon)(K)$.

It remains to show that $\Phi\circ d^\textup{fil}$ coincides with $\epsilon^{\otimes 2}\circ\psi_{\calw(n)}$. These two maps are only non-zero on the graded part $N_1^\textup{fil}N_{*-1}^\textup{int}(Q\BSWres L)\subseteq Q^{\calL(n)}\BSW^2 L$ of the total complex, and an element therein is a linear combination
\[K':=\sum_j k_{e_j\left(s_{\alpha_{ji_0}} k_{f^0_{ji_0}g^1_{ji_0i_1}h^2_{ji_0i_1i_2}}\right)}\]
which satisfies the equation $d_1^\textup{h}(K')=0$, i.e.:
%\[d_1^\textup{fil}(K')=\sum_j k_{e_j\left(s_{\alpha_{ji_0}} {f^0_{ji_0}g^1_{ji_0i_1}h^2_{ji_0i_1i_2}}\right)}=0.\]
%We may rewrite this equation as
\[\sum_j k_{e_j(f_{ji_0})\left(s_{\alpha_{ji_0}} {g^1_{ji_0i_1}h^2_{ji_0i_1i_2}}\right)}=0\textup{ in }N_{*-1}Q^{\calL(n)}\BSW Q^{\calU(n)}B^{\calw(n)}X.\]
Since the symbol $k$ is used in this condition, it is \emph{very} strong, implying that as $j$ varies, the subscripts 
\[e_j(f_{ji_0})\left(s_{\alpha_{ji_0}} {g^1_{ji_0i_1}h^2_{ji_0i_1i_2}}\right)\in Q^{\calU(n)}F^{\calw(n)}\cdots F^{\calw(n)}X\]
each repeat an even number of times. %As the subscripts are elements of $Q^{\Lambda}[F^{\calw(n)}]^{*}M$,
These coincidences also occur if we take quadratic parts of the $e_j(f_{ji_0})$, so that there holds the following equation in $N_{*-1}(S^2(QL))$:
%\[\sum_j {q(e^0_j(f^0_{ji_0}))\left(s_{\alpha_{ji}} g^1_{ji_1}h^2_{ji_1i_2}\right)}=0\in N_{*-1}((QX)^{\otimes2}).\]
\begin{gather*}
\sum q(e_j(f_{ji_0}))\left(s_{\alpha_{ji_0}} {g^1_{ji_0i_1}h^2_{ji_0i_1i_2}}\right)=0,\textup{ or equivalently}\\
\sum q(e_j(u(f_{ji_0})))\left(s_{\alpha_{ji_0}} {g^1_{ji_0i_1}h^2_{ji_0i_1i_2}}\right)
=
\sum q(u(e_j)(f_{ji_0}))\left(s_{\alpha_{ji_0}} {g^1_{ji_0i_1}h^2_{ji_0i_1i_2}}\right).
\end{gather*}
The proof is completed upon noting that the left hand side of this equation equals $(\epsilon^{\otimes 2}\circ \psi_{\BSW})(K')$, while the right hand side equals $(\Phi\circ d^\textup{fil})(K')$. Indeed:
\begin{alignat*}{2}
(\epsilon^{\otimes 2}\circ \psi_{\BSW})(K')
&=
\epsilon^{\otimes 2}\left(\sum q(e_j)\left(s_{\alpha_{ji_0}} k_{f^0_{ji_0}g^1_{ji_0i_1}h^2_{ji_0i_1i_2}}\right)\right)%
\\
&=
\sum q(e_j)\left(s_{\alpha_{ji_0}} f^0_{ji_0}g^1_{ji_0i_1}h^2_{ji_0i_1i_2}\right)%
\\
&=
\sum q(e_j(u(f_{ji_0})))\left(s_{\alpha_{ji_0}} g^1_{ji_0i_1}h^2_{ji_0i_1i_2}\right)=\textup{LHS}
\end{alignat*}
\textbf{(clarify $\psi_{\BSW}$?)} where we may simplify $f_{ji_0}$ to $u(f_{ji_0})$ since we take values in $S^2(QL)$. \textbf{(subscripts on $Q$)} Moreover,
\begin{alignat*}{2}
d^\textup{fil}(K')
&=
\sum e_j\left(s_{\alpha_{ji_0}} k_{f^0_{ji_0}g^1_{ji_0i_1}h^2_{ji_0i_1i_2}}\right)%
\\
&=
\sum u(e_j)\left(s_{\alpha_{ji_0}} k_{f^0_{ji_0}g^1_{ji_0i_1}h^2_{ji_0i_1i_2}}\right)%
&\qquad&\text{(in $N_{*-1}^\textup{int}(Q\BSW L)$)}\\
\Phi(d^\textup{fil}(K'))&=
\sum u(e_j)\left( q(f_{ji_0})(s_{\alpha_{ji_0}}g^1_{ji_0i_1}h^2_{ji_0i_1i_2})\right)%
&\qquad&\text{(relevant $s_{\alpha_{ji_0}}$ are $\Id$)}\\
&=
\sum q(u(e_j)(f_{ji_0}))\left(s_{\alpha_{ji_0}} {g^1_{ji_0i_1}h^2_{ji_0i_1i_2}}\right) =\textup{RHS}.
\end{alignat*}
To explain further the third equality,  the expression $\sum e_j(s_{\cdots }k_{\cdots })$ represents a sum of $\calL(n)$-expressions $e_j$ in various generators $s_{\cdots }k_{\cdots }$ of the almost free object $\BSW L\in s\calL(n)$. As we have passed to the indecomposables $Q^{\calL(n)}\BSW L$, (\textbf{note}\emph{ above (or here) that we can only say this kind of thing as the monad is really homogeneous}), any non-unit expression is sent to zero, and an expression only lies in the normalization $N_{*-1}Q^{\calL}\BSW L$ if its unital part consists only of classes $z_{\cdots}$ and $h_{\cdots }$, and not of their degeneracies. Thus, any degeneracies $s_{\cdots }$ appearing in the unital part $u(e_j)$ are in fact simply the identity, and can be ignored when we apply $\Phi$.
\end{proof}













\end{Operations in composite functor spectral sequences}


\begin{Calculations of HWn for n nonzero}
\section{\textbf{Calculations of ${\calw(n)}$-cohomology for $n\geq 1$}}
In this section, we will calculate the value of $H^*_{\calw(n)}X$ for certain locally finite-dimensional objects $X$ of $\calw(n)$. In each subsection, we will write $X=V^*_{(n)}$, so that $X$ has underlying vector space dual to $V_{{(n)}}\in\vect{n}{+}$, and define:
\[V_{(k+1)}:=H^*_{\calU(k)}V^*_{(k)}\textup{ and }V^*_{(k+1)}:=H_*^{\calU(k)}V^*_{(k)}\textup{ for $k\geq n$},\]
so that $V^*_{(k+1)}$ is an object of $\calw(k+1)$ for each $k$, vector space dual to $V_{(k+1)}\in\vect{k+1}{+}$.
%We will calculate $V^*_{(k+1)}$ as an object of $\calw(k+1)$ at each stage. 
Having all of this data will allow us to draw conclusions about $H^*_{\calw(n)}V^*_{(n)}$, using, for each $k\geq n$, the $(k+1)^{\textup{st}}$ composite functor spectral sequence:
\[(E_{2,(k+1)})^{s_{k+2},\ldots,s_1}_{t}:=(H^*_{\calw(k+1)}V^*_{(k+1)})^{s_{k+2},\ldots,s_1}_{t}\implies (H^*_{\calw(k)}V^*_{(k)})^{s_{k+2}+s_{k+1},s_k,\ldots,s_1}_{t}.\]
The $1^{\textup{st}}$ composite functor spectral sequence, which calculates $H^*_{\calw(0)}$ from $H^*_{\calw(1)}$, will only appear [later].



\subsection{When $X\in\calw(n)$ is one-dimensional}
Let $X=V^*_{(n)}\in\calw(n)$ be a one dimensional object of $\calw(n)$, %dual to a vector space $V_{(n)}=(V_{(n)})^S_T=\langle v\rangle\in\vect{n}{+}$, where $T\geq1$ and $S\geq0$ (\textbf{better:} 
dual to a one-dimensional vector space $V_{(n)}\in\vect{n}{+}$, with non-zero element $v\in(V_{(n)})^S_T$. Write $v^*\in X^T_{S}$ for the non-zero element of $X$. As every $\calw(n)$-operation changes degrees, $X$ is necessarily trivial.
\begin{prop}\label{iterative calc of the Vk all trivial}
For each $k\geq n$
\[V_{(k)}=F_{\calMv(k)}F_{\calMv(k-1)}\cdots F_{\calMv(n+1)}V_{(n)},\]
and $V^*_{(k)}$ is a trivial object of $\calw(k)$.
\end{prop}
\begin{proof}
The proof is by induction, with the case $k=n$ simply our standing assumptions. If the statement holds for $V_{(k)}$, then proposition \ref{propDerivedIndTrivialUobject n at least 1} shows that the Koszul complex calculating $V_{(k+1)}$ has zero differentials, as $V_{(k)}$ has trivial $\calw(k)$-structure, so that $V_{(k+1)}=F_{\calM_{\textup{v}}(k+1)}V_{(k)}$. This has trivial $\calw(k+1)$-structure, by the results of \S\ref{section on structure on homology of koszul cx}.
\end{proof}
Our next step is to calculate, for $k\geq n$, the groups:
\[(E_{2,(k+1)})^{s_{k+2},0,s_k,\ldots,s_1}_{t}:=(H^*_{\calw(k+1)}V^*_{(k+1)})^{s_{k+2},0,s_k,\ldots,s_1}_{t}\cong H^{s_{k+2}}_{\calL(k)}V^*_{(k)}.\]
The isomorphism shown here follows from the observation that in dimension $s_{k+1}=0$, an object of $\calw(k+1)$ is nothing more than an object of $\calL(k)$. More precisely, consider the functor $\DASH_\textbf{0}:\vect{+}{k+1}\to \vect{+}{k}$ given by
\[(Y_\textbf{0})^t_{s_k,\ldots,s_1}:=Y^t_{0,s_k,\ldots,s_1}.\]
Then $\DASH_\textbf{0}$ induces a functor $\DASH_\textbf{0}:\calw(k+1)\to \calL(k)$, such that, for all $Y\in\calw(k+1)$:
\[(F_{\calw(k+1)}(Y))_\textbf{0}\cong F_{\calL(k)}(Y_\textbf{0})\textup{ and }(Q^{\calw(k+1)}Y)_\textbf{0}\cong Q^{\calL(k)}(Y_\textbf{0}),\]
so that $(Q^{\calw(k+1)}B^{\calw(k+1)}Y)_{\textbf{0}}\cong(Q^{\calL(k)}B^{\calL(k)}Y_{\textbf{0}})$ for any $Y\in s\calw(k+1)$, and thus:
\begin{prop}
Suppose that $Y\in s\calw(k+1)$, where $k\geq0$. Then
\[(H^*_{\calw(k+1)}Y)^{s_{k+2},0,s_k,\ldots,s_1}_{t}\cong (H^{s_{k+2}}_{\calL(k)}Y_\textup{\textbf{0}})^{s_k,\ldots,s_1}_t.\]
\end{prop}
Returning to the calculation at hand,
\begin{prop}\label{calculation in internal dimension zero}
For each $k\geq n$, there is an isomorphism of commutative algebras:
\[(E_{2,(k+1)})^{s_{k+2},0,s_k,\ldots,s_1}_{t}\cong(\dual\UEAX(V^*_{(k)}))^{s_{k+2},s_k,\ldots,s_1}_t.\]
%where the argument $V_{(k)}^*=(V_{(k+1)}^*)_\textbf{\textup{0}}$ of the Chevalley-Eilenberg-May complex functor $\UEAX$ is a trivial object of $\calL(k)$. 
When $v\in V_{(n)}$ is restrictable,  $\dual\UEAX(V^*_{(k)})$ is the polynomial algebra $\Ftwo [V_{(k)}]$ (with degree shifts: see appendix \ref{The Chevalley-Eilenberg-May complex}). When $v\in V_{(n)}$ is not restrictable, $V_{(k)}=\F_{2}\{v\}$ is one-dimensional, and $\dual\UEAX(V^*_{(k)})$ is the exterior quotient $E(v)$ of $\Ftwo [v]$, generated by the element $v$. In either case, for each individual value of the grading $t$, the group $(E_{2,(k+1)})^{s_{k+2},0,s_k,\ldots,s_1}_{t}$ is finite-dimensional.
\end{prop}
\begin{proof}
The only further observation necessary to prove this isomorphism is that if $v\in V_{(n)}$ is restrictable, every element of the trivial partially restricted Lie algebra $V_{(k)}$ is in restrictable degree, and that if $v\in V_{(n)}$ is not restrictable, each $V_{(k)}$ is one-dimensional, concentrated in non-restrictable degree. For the finiteness property, one simply notes that the $V_{(k)}$ have such a property, and that there is a degree shift in the algebra structure.
\end{proof}
%\begin{thm}
%There is an isomorphism $F_{\calMh(2)}F_{\calMv(2)}V_{(1)}\to H^*_{\calw(1)}V_{(1)}^* $, under which the filtration on $H^*_{\calw(1)}V_{(1)}^*$ arising from the composite functor spectral sequence coincides with the filtration on $F_{\calMh(2)}F_{\calMv(2)}V_{(1)}$ given in theorem \ref{thm on compressing seqs of steenrod ops}.
%\end{thm}

Consider the diagram:
\[\xymatrix@R=1.5mm{
&
H^*_{\calw({n+1})}H_*^{\calU(n)}V^*_{(n)}\ar@{=>}[dl]^-{g_{n+1}}\ar@{=}[d]&
H^*_{\calw({n+2})}H_*^{\calU({n+1})}V^*_{({n+1})}\ar@{=>}[dl]^-{g_{n+2}}\ar@{=}[d]&
%H^*_{\calw({n+3})}H_*^{\calU({n+2})}V^*_{({n+2})}\ar@{=>}[dl]^-{g_{n+3}}\ar@{=}[d]&
%H^*_{\calw(5)}H_*^{\calU({n+3})}V^*_{({n+3})}\ar@{=>}[dl]^-{g_5}\ar@{=}[d]
\\
 H^*_{\calw(n)}V^*_{(n)}
&H^*_{\calw({n+1})}V^*_{({n+1})}
&H^*_{\calw({n+2})}V^*_{({n+2})}
%&H^*_{\calw({n+3})}V^*_{({n+3})}
%&H^*_{\calw(5)}V^*_{(5)}
\\
&&\makebox[5cm][r]{\,$\smash{\cdots }$}
\\
 F_{\calM_{\textup{h}}({n+1})}F_{\calM_{\textup{v}}({n+1})}V_{(n)}\ar[uu]_-{\rho_n}\ar@{=}[d]
&F_{\calM_{\textup{h}}({n+2})}F_{\calM_{\textup{v}}({n+2})}V_{({n+1})}\ar[uu]_-{\rho_{n+1}}\ar@{=}[d]\ar@{=>}[ld]^-{f_{n+1}}
&F_{\calM_{\textup{h}}({n+3})}F_{\calM_{\textup{v}}({n+3})}V_{({n+2})}\ar[uu]_-{\rho_{n+2}}\ar@{=}[d]\ar@{=>}[ld]^-{f_{n+2}}
%&F_{\calM_{\textup{h}}(5)}F_{\calM_{\textup{h}}(5)}V_{({n+3})}\ar[uu]_-{\rho_{n+3}}\ar@{=}[d]\ar@{=>}[ld]
%&F_{\calM_{\textup{h}}(6)}F_{\calM_{\textup{h}}(6)}V_{(5)}\ar[uu]_-{\rho_5}\ar@{=}[d]\ar@{=>}[ld]
\\
 F_{\calMh({n+1})}V_{({n+1})}
&F_{\calMh({n+2})}V_{({n+2})}
&F_{\calMh({n+3})}V_{({n+3})}
%&F_{\calMh(5)}V_{(5)}
%&F_{\calMh(6)}V_{(6)}
}\]
For each $k\geq n$, the map $\rho_k$ is induced by the inclusion $V_{(k)}\cong H^{0}_{\calw(k)}V^*_{(k)}\subseteq H^*_{\calw(k)}V^*_{(k)}$ (which exists as $V_{(k)}$ is trivial) and the $F_{\calMh(k+2)}$- and $F_{\calMv(k+2)}$-operations defined on $H^*_{\calw(k)}V_{(k)}$. (Note that $\rho_{k}$ is a graded map, since the effect of these operations on dimensions is the same in its domain and codomain.)

The double arrow $g_{k+1}$, representing the convergence of the $(k+1)^{\textup{st}}$ composite functor spectral sequence, is in truth shorthand for the function
\[g_{k+1}:\left(E_{\infty,(k+1)}\to E_0(H^*_{\calw(k)}V^*_{(k)})\right)\]
so that $g_{k+1}$ may only be defined on the permanent cycles within $H^*_{\calw(k+1)}H_*^{\calU(k)}V^*_{(k)}$, and lands in the graded object associated to the \emph{target filtration} on $H^*_{\calw(k)}V^*_{(k)}$, the filtration associated with the spectral sequence $E_{2,(k+1)}\implies H^*_{\calw(k)}V^*_{(k)}$. Similarly, we employ the double arrow $f_{k+1}$ as shorthand for the function of theorem \ref{thm on compressing seqs of steenrod ops}, which is defined on the entirety of $F_{\calM_{\textup{h}}(k+2)}F_{\calM_{\textup{v}}(k+2)}V_{(k+1)}$, but whose true codomain is the graded object $E_0(F_{\calMh(k+1)}V_{(k+1)})$ associated with the target filtration defined in theorem \ref{thm on compressing seqs of steenrod ops}.
\begin{thm}\label{thm on collapsing of most sseqs}
For each $k\geq n$, $\im(\rho_{k+1})$ consists of permanent cycles and $\rho_k$ preserves the target filtrations, so that it is possible to form the composites $g_{k+1}\circ \rho_{k+1}$ and $E_0(\rho_{k})\circ f_{k+1}$. These composites are equal, and moreover, $\rho_k$ is an isomorphism. In particular, for $k\geq n$, the  $(k+1)^{\textup{st}}$ composite functor spectral sequence collapses at $E_2$.
\end{thm}
Before giving the proof, we remark that in some dimensions, $\rho_k$ is already known to be an isomorphism.
\begin{prop}\label{isomorphism rho k in some dims}
For $k\geq n$, $\rho_k$ is an isomorphism in dimension $s_k=0$:
\[\rho_k:(F_{\calM_{\textup{h}}(k+1)}F_{\calM_{\textup{v}}(k+1)}V_{(k)})^{s_{k+1},0,s_{k-1},\ldots,s_1}_{t} \overset{\cong}{\to}(H^*_{\calw(k)}V^*_{(k)})^{s_{k+1},0,s_{k-1},\ldots,s_1}_{t}.\]
\end{prop}
\begin{proof}
In this dimension, $\rho_k$ factors as
\begin{alignat*}{2}
(F_{\calM_{\textup{h}}(k+1)}F_{\calM_{\textup{v}}(k+1)}V_{(k)})^{s_{k+1},0,s_{k-1},\ldots,s_1}_{t}
&=
(F_{\calM_{\textup{h}}(k+1)}F_{\calM_{\textup{v}}(k+1)}V_{(k)}^{\textbf{\textup{0}}})^{s_{k+1},0,s_{k-1},\ldots,s_1}_{t}\\
&=(F_{\calM_{\textup{h}}(k+1)}V_{(k)}^{\textbf{\textup{0}}})^{s_{k+1},0,s_{k-1},\ldots,s_1}_{t}\\
&\cong (\dual\UEAX(\dual(V_{(k)}^{\textbf{\textup{0}}})))^{s_{k+1},0,s_{k-1},\ldots,s_1}_{t}\\
&\cong (H^*_{\calw(k)}V^*_{(k)})^{s_{k+1},0,s_{k-1},\ldots,s_1}_{t}
\end{alignat*}
Here, we are viewing $V_{(k)}^{\textbf{\textup{0}}}$, the subspace of $V_{(k)}$ in degree $s_k=0$, as an object of $\vect{k+1}{+}$ in order to apply $F_{\calMv(k+1)}$. The inclusion $F_{\calM_{\textup{h}}(k+1)}F_{\calM_{\textup{v}}(k+1)}V_{(k)}^{\textbf{\textup{0}}}\subseteq F_{\calM_{\textup{h}}(k+1)}F_{\calM_{\textup{v}}(k+1)}V_{(k)}$ restricts to the identity in degree $s_k=0$, explaining the first equality. The second equality is similar: any non-trivial $\calMv(k+1)$-operation lands outside degree $s_k=0$. The first isomorphism follows from proposition \ref{basis of free horizontal operations algebra} --- as $V_{(k)}^{\textbf{\textup{0}}}$ is concentrated in dimension $s_{k+1}=0$, the proposition ensures that $F_{\calM_{\textup{h}}(k+1)}V_{(k)}^{\textbf{\textup{0}}}$ is a quotient of the polynomial algebra on $V_{(k)}^{\textbf{\textup{0}}}$, and indeed, the same quotient as $\dual\UEAX(\dual(V_{(k)}^{\textbf{\textup{0}}}))$. The second isomorphism is proposition \ref{calculation in internal dimension zero}.
\end{proof}
\begin{proof}[Proof of \ref{thm on collapsing of most sseqs}]
For each $k\geq n$, we will use the diagram
\[\xymatrix@R=4mm{
%&&H_{\calw(k+1)}^{0}H_*^{\calU(k)}V_{(k)}^*\\
%V_{(k)}\ar@{..>}[r]^-{i_0}\ar@{..>}[rd]_-{j_0}&
H^*_{\calw(k)}V^*_{(k)}\ar@{..>}@/^1.22em/[rr]|-{\textup{\,edge composite\,}}
&H^*_{\calw(k+1)}H_*^{\calU(k)}V^*_{(k)}\ar@{=>}[l]^-{g_{k+1}}%\ar@{..>}@<.35ex>[r]^-{}%\ar[u]_-{\textup{proj onto $*=0$}}
&H^*_{\calU(k)}V^*_{(k)}\ar@{..>}[l]^-{i_1}
&V_{(k)}\ar@{..>}[l]|-{\,i_2\,}\ar@{..>}[ld]^-{j_2}\ar@{..>}@/_1.72em/[lll]|-{\,i_0\,}
\\
%V_{(k)}\ar@{..>}[r]\ar@{..>}[u]^-{\cong}&
F_{\calM_{\textup{h}}(k+1)}F_{\calM_{\textup{v}}(k+1)}V_{(k)}\ar[u]^-{\rho_k}
&W(F_{\calMv(k+1)}V_{(k)})\ar[u]\ar@{->>}[l]_-{\overline{f}_{k+1}}\ar[u]_-{\rho_{k+1}}
&F_{\calMv(k+1)}V_{(k)}\ar@{..>}[l]_-{j_1}\ar@{=}[u]%\ar@{..>}@/^1.02em/[ll]|-{\,j_3\,}
%&V_{(k)}\ar@{..>}[l]\ar@{..>}[u]_-{=}\ar@{..>}@/^1.52em/[lll]|-{\textup{inc}}
%\\
% F_{\calMh(k+1)}F_{\calMv(k+1)}V_{(k)}
%&F_{\calMh(k+2)}V_{(k+2)}
}\]
where $W(F_{\calMv(k+1)}V_{(k)})$ is the object introduced in the proof of theorem \ref{thm on compressing seqs of steenrod ops}, of which $F_{\calM_{\textup{h}}(k+2)}F_{\calM_{\textup{v}}(k+2)}F_{\calMv(k+1)}V_{(k)}$ is a quotient. Here, the maps $j_1,j_2$ are the evident inclusions of generators, while the maps $i_0,i_1,i_2$ are the inclusions arising because $V_{(k)}$ is trivial.

 We may define 
\[c:=(\rho_k\circ \overline{f}_{k+1}\circ j_1):F_{\calMv(k+1)}V_{(k)}\to H^*_{\calw(k)}V^*_{(k)},\]
without the need to pass to any associated graded objects. By construction of $\overline{f}_{k+1}$,  $c$ is induced by the inclusion $i_0$ and the $\calMv(k+1)$-structure of $H^*_{\calw(k)}V^*_{(k)}$.

The edge composite is the composite of a surjection, a monomorphism $m_1$, and an isomorphism $m_2$ (with inverse $i_1$):
\[H^*_{\calw(k)}V^*_{(k)}\epi E_{\infty,(k+1)}^{0,*}\overset{m_1}{\to}E_{2,(k+1)}^{0,*}\overset{m_2}{\to}H^*_{\calU(k)}V^*_{(k)}.\]
Moreover, $c$ is a section of the edge composite, since both maps are compatible with $\calMv(k+1)$-structures (proposition \ref{edgehomproposition}), and their composite is the identity on $V_{(k)}\subseteq  F_{\calMv(k+1)}V_{(k)}$.
In particular, the edge composite is a surjection, so that $m_{1}$ is an isomorphism. That is, every class in $\im(i_1)$ is a permanent cycle. Singer's work (c.f.\ \S??????) then shows that $\im(\rho_{k+1})$ consists of permanent cycles, as permanent cycles are preserved by the $F_{\calMh(k+2)}$- and $F_{\calMv(k+2)}$-operations on $E_{2,(k+1)}$.

%The claim that $g_{k+1}\circ \rho_{k+1}=E_0(\rho_{k})\circ f_{k+1}$ is best proven together with the claim that $\rho_k$ preserves filtration.

Any section of $H^*_{\calw(k)}V^*_{(k)}\epi E_{\infty,(k+1)}^{0,*}=E_{2,(k+1)}^{0,*}$ will realize, up to filtration, the restriction of $g_{k+1}$ to $E_{\infty,(k+1)}^{0,*}$, so we choose
\[E_{2,(k+1)}^{0,*}\overset{m_2}{\to} H^*_{\calU(k)}V^*_{(k)}\cong F_{\calMv(k+1)}V_{(k)}\overset{c}{\to}H^*_{\calw(k)}V^*_{(k)}.\]
In particular, $g_{k+1}\circ \rho_{k+1}\circ j_1=g_{k+1}\circ i_1=c\circ m_2\circ i_1=c$, up to filtration. More precisely, $g_{k+1}\circ \rho_{k+1}\circ j_1$ equals the composite
\begin{gather*}
F_{\calMv(k+1)}V_{(k)}\overset{c}{\to} H^*_{\calw(k)}V^*_{(k)}\epi E_0^0(H^*_{\calw(k)}V^*_{(k)}).
\end{gather*}
Now the target filtrations on the domain and codomain of $\rho_k$ are induced by the filtrations on the domain and codomain of $\rho_{k+1}$ by cohomological dimension $s_{k+2}$, and $\rho_{k+1}$ is a graded map. Thus, for any $w\in F^pW(F_{\calMv(k+1)}V_{(k)})$, we must see that $\rho_k(\overline{f}_{k+1}(w))$ coincides with $g_{k+1}(\rho_{k+1}(w))$ modulo $F^{p+1}H^*_{\calw(k)}V^*_{(k)}$, as this will prove both that $\rho_k$ preserves target filtrations and that  $g_{k+1}\circ \rho_{k+1}=E_0(\rho_{k})\circ f_{k+1}$. However, this coincidence follows from the fact that $c=g_{k+1}\circ \rho_{k+1}\circ j_1$, as $W(F_{\calMv(k+1)}V_{(k)})$ is generated by $\im(j_1)$ under $F_{\calMh(k+2)}$- and $F_{\calMv(k+2)}$-operations, and the definition of $\overline{f}_{k+1}$ is modeled on the interaction of $g_{k+1}$ with these operations, as studied by Singer (c.f.\ \S??????).

What remains is to show that the maps $\rho_k$ are isomorphisms. Suppose that 
\[x_{(k)}\in (E_{2,(k)})^{s_{k+1}^{k},\ldots,s_1^{k}}_t=(H^*_{\calw(k)}V^*_{(k)})^{s_{k+1}^{k},\ldots,s_1^{k}}_t.\]
Now $x_{(k)}$ is detected by some permanent cycle $x_{(k+1)}\in E_{2,(k+1)}$, which is detected by some permanent cycle $x_{(k+2)}\in E_{2,(k+2)}$, and so on, giving a sequence of elements
\begin{gather*}
x_{(r)}\in (E_{2,(r)})^{s_{r+1}^{r},\ldots,s_1^{r}}_t=(H^*_{\calw(r)}V^*_{(r)})^{s_{r+1}^{r},\ldots,s_1^{r}}_t\textup{ for $r\geq k$,}\\
\textup{where }s^{r}_{r+1}+s^{r}_{r}=s^{r-1}_{r}\textup{ and }s^{r}_{i}=s^{r-1}_{i}\textup{  for $1\leq i\leq r-1$ and $r>k$}.
\end{gather*}
We will say that $x_{(k)}$ has iterated filtration at least $(s^{k+1}_{k+2},s^{k+2}_{k+3},s^{k+3}_{k+4},\ldots)$ whenever a sequence of such classes $x_{(r)}$ exists, and partially order the set of possible iterated filtrations lexicographically. Then $x_{(r)}$ only determines $x_{(k)}$ modulo elements of $E_{2,(k)}$ of higher iterated filtration.

Simply because these gradings are always non-negative, it is inevitable that $s_r^r=0$ for some $r\geq k$, so that by proposition \ref{isomorphism rho k in some dims}, $x_{(r)}=\rho_ry_{(r)}$ for some $y_{(r)}\in F_{\calM_{\textup{h}}(r+1)}F_{\calM_{\textup{v}}(r+1)}V_{(r)}$. Moreover, one only needs to examine finitely many sequences of gradings, each of the form
\[(s^r_{r+1},0,s^{r}_{r-1},\ldots, s^{r}_{k+1},s^k_{k},\ldots,s^k_1)\textup{ where }s^k_{k+1}=s^r_{r+1}+s^{r}_{r-1}+s^{r}_{r-2}+\cdots +s^{r}_{k+1}.\]
This, along with proposition \ref{calculation in internal dimension zero}, shows that $(H^*_{\calw(k)}V^*_{(k)})^{s_{k+1}^{k},\ldots,s_1^{k}}_t$ is finite dimensional for each given value of $t$.



By the commutativity established above, $x_{(k)}\equiv\rho_{k}f_{k+1}\cdots f_{r-1}f_r(y_{(r)})$, modulo higher iterated filtration. As this equality holds in a group which is finite dimensional for each given $t$, this establishes the surjectivity of $\rho_k$, and that
%
%
%
%{calculation in internal dimension zero},
%\[x_{(r)}\in (E_{2,(r)})^{s_{r+1}^{r},0,s_{r-1}^{r},\ldots,s_1^{r}}_{t}\cong\UEAX(V_{(r-1)}^*)^*,\]
%which is by definition generated by products of elements of 
%\[V_{(r-1)}\cong H^{0}_{\calw(r)}H_0^{\calU(r-1)}V_{(r-1)}\subseteq E_{2,(r)}.\]
%Such products are all contained in the image of $\rho_r$, so that they are all permanent cycles. But even more, we have seen that elements in the image of some $\rho_r$ are permanent cycles that detect elements in the image of $\rho_{r-1}$ whenever $r>1$. That is, \textbf{(filtration confusing)}, there is a sequence of permanent cycles (well defined up to higher filtration each time), each detecting the next. One of these is $x_{(k)}$, if we desire. Thus each $\rho_k$ is surjective, and 
every one of the spectral sequences is degenerate. Thus, we have shown that all of the maps $g_k$ are in fact isomorphisms, or rather that in the following commuting square, the $g_{k+1}$ are isomorphisms:
\[\xymatrix@R=4mm{
E_{0}(H^*_{\calw(k)}V^*_{(k)})
&H^*_{\calw(k+1)}H_*^{\calU(k)}V^*_{(k)}\ar@{->}[l]^-{\cong}_-{{g}_{k+1}}
\\
%V_{(k)}\ar@{..>}[r]\ar@{..>}[u]^-{\cong}&
E_0(F_{\calM_{\textup{h}}(k+1)}V_{(k+1)})\ar@{->>}[u]^-{E_0(\rho_k)}
&F_{\calM_{\textup{h}}(k+2)}F_{\calM_{\textup{v}}(k+2)}F_{\calMv(k+1)}V_{(k)}\ar@{->}[l]^-{\cong}_-{{f}_{k+1}}\ar@{->>}[u]_-{\rho_{k+1}}
}\]
For each $k$, $\rho_k$ is injective if and only if $E_0(\rho_k)$ is injective. This holds by repeated application of the snake lemma, using the fact that $\rho_k$ is surjective, and the observation that for any given value of the grading $t$, the group $(F_{\calM_{\textup{h}}(k+1)}V_{(k+1)})_t$ is finite dimensional, so that the filtrations of both the domain and codomain of $\rho_k$ are eventually zero in each degree $t$. More specifically,
\[\rho_k:(F_{\calM_{\textup{h}}(k+1)}V_{(k+1)})_t^{s_{k+1},\ldots,s_1}\to (H^*_{\calw(k)}V^*_{(k)})_t^{s_{k+1},\ldots,s_1}\]
is injective if and only if
\[E_0(\rho_k):E_0^{s'_{k+2}}(F_{\calM_{\textup{h}}(k+1)}V_{(k+1)})_t^{s'_{k+1},s_k,\ldots,s_1}\to E_0^{s'_{k+2}}(H^*_{\calw(k)}V^*_{(k)})_t^{s'_{k+1},s_k,\ldots,s_1}\]
is injective whenever $s'_{k+2}+s'_{k+1}=s_{k+1}$. As in the argument for surjectivity, in order to check that all the $\rho_k$ are injective, we now only need to check that every map
\[(F_{\calM_{\textup{h}}(r+1)}F_{\calM_{\textup{v}}(r+1)}V_{(r)})^{s_{r+1}^{r},0,s_{r-1}^{r},\ldots,s_1^{r}}_{t} \overset{\rho_r}{\to}(H^*_{\calw(r)}V^*_{(r)})^{s_{r+1}^{r},0,s_{r-1}^{r},\ldots,s_1^{r}}_{t}\]
is injective, which is part of proposition \ref{isomorphism rho k in some dims}.
%%%%%%However, in this degree, $\rho_r$ is an isomorphism, as it factors as
%%%%%%\begin{alignat*}{2}
%%%%%%(F_{\calM_{\textup{h}}(r+1)}F_{\calM_{\textup{v}}(r+1)}V_{(r)})^{s_{r+1}^{r},0,s_{r-1}^{r},\ldots,s_1^{r}}_{t}
%%%%%%&=
%%%%%%(F_{\calM_{\textup{h}}(r+1)}F_{\calM_{\textup{v}}(r+1)}V_{(r-1)})^{s_{r+1}^{r},0,s_{r-1}^{r},\ldots,s_1^{r}}_{t}\\
%%%%%%&=(F_{\calM_{\textup{h}}(r+1)}V_{(r-1)})^{s_{r+1}^{r},0,s_{r-1}^{r},\ldots,s_1^{r}}_{t}\\
%%%%%%&\cong (\UEAX(V_{(r-1)}^*)^*)^{s_{r+1}^{r},0,s_{r-1}^{r},\ldots,s_1^{r}}_{t}.
%%%%%%\end{alignat*}
%%%%%%Here, we have written $V_{(r-1)}$ for the subspace of $V_{(r)}$ in degree $s_r=0$, which we view as an object of $\vect{r+1}{+}$. Then the inclusion $F_{\calM_{\textup{h}}(r+1)}F_{\calM_{\textup{v}}(r+1)}V_{(r-1)}\subseteq F_{\calM_{\textup{h}}(r+1)}F_{\calM_{\textup{v}}(r+1)}V_{(r)}$ restricts to the identity in degree $s_r=0$, explaining the first equality. The second equality is similar: any non-trivial $\calMv(r+1)$-operation lands outside degree $s_r=0$. The final isomorphism follows from proposition \ref{basis of free horizontal operations algebra} --- as $V_{(r-1)}$ is concentrated in dimension $s_{n+1}=0$, the proposition ensures that $F_{\calM_{\textup{h}}(r+1)}V_{(r-1)}$ is a quotient of the polynomial algebra on $V_{(r-1)}$, and indeed, the same quotient as $\UEAX(V_{(r-1)}^*)^*$.
%
%, note that on $V_{(r-1)}$, there are no non-trivial horizontal Steenrod operations except for the top Steenrod operation, the squaring operation. That is, for any $v\in (V_{(r-1)})^{0,0,s_{r-1},\ldots, s_1}_t$, suppose that $\Sqh^{i_\ell}\cdots\Sqh^{i_1}$ is an admissible sequence such that $\Sqh^{i_\ell}\cdots\Sqh^{i_1}v$ is not forced to be zero \textbf{(did we ever use the word \emph{excess}, or give a basis of $\calMh$?)}. Then we must have $i_j=2^j$ for each $j$, and  $\Sqh^{i_\ell}\cdots\Sqh^{i_1}v=v^{2^\ell}$. Note that the axiom $\Sqh^1v=0$ applies iff $s_{r-1}=\cdots =s_1=0$, i.e.\ iff $v$ is not restrictable, and so is an exterior generator in $\UEAX(V_{(r-1)}^*)^*$. That is, $F_{\calM_{\textup{h}}(r+1)}V_{(r-1)}$ and $\UEAX(V_{(r-1)}^*)^*$ are generated by $V_{(r-1)}$ using the same operations (commutative products), and these operations satisfy the same relations (only that classes in $(V_{(r-1)})_t^{0,\ldots,0}$ are exterior), proving the isomorphism. See also proposition \ref{basis of free horizontal operations algebra}.
\end{proof}
\subsection{A Koszul dual Hilton-Milnor theorem}
This is an opportune moment to prove:
\begin{thm}\label{Koszul-dual Hilton-Milnor theorem}
Suppose that $X,Y\in\calw(n)$ are locally finite, with $n\geq0$. Then 
\[H^*_{\calw(n)}(X\oplus Y)\cong H^*_{\calw(n)}(X)\otimes H^*_{\calw(n)}(Y).\]
\end{thm}
\begin{proof}
Basically, $H_{*}^{\calU(k)}$ is additive, and $H^*_{\calL(k)}$ turns $\oplus$ into $\otimes $. We do induction on iterated filtration. Importantly, all of the spectral sequences involved are multiplicative. \textbf{Note that $n=0$ works as well?}
\end{proof}

\subsection{A two-dimensional example in $\calw(2)$}
In this section, we suppose that $T\geq1$, and let $X=V^*_{(2)}\in\calw(2)$ be the two-dimensional object of $\calw(2)$ spanned by non-zero classes 
\[v_{0}^{*}\in (V^*_{(2)})^{T}_{0,1}\textup{ and }v_{1}^{*}\in (V^*_{(2)})^{2T+1}_{0,2}\]
such that $v^*_{1}=v^*_0\lambda_{0}$, and all other operations are trivial. Note that this non-trivial operation may be written $v^*_{1}=\restn{(v^*_0)}$: the $\lambda_0$ is in fact a top operation.
\begin{prop}\label{2d example in w2}
For all $k\geq2$, $V^*_{(k)}$ is two-dimensional, spanned by
\[v_{0}^{*}\in (V^*_{(k)})^{T}_{0,\ldots,0,1}\textup{ and }v_{1}^{*}\in (V^*_{(k)})^{2T+1}_{0,\ldots,0,2},\]
with  $v^*_{1}=\restn{(v^*_0)}$ the only non-trivial operation.
\end{prop}
\begin{proof}
An induction as in the proof of proposition \ref{iterative calc of the Vk all trivial}, using the fact that at each stage, the only non-trivial $\lambda$-operation is a \emph{top} operation, and thus does not yield a differential in $K_*^{\calU(k)}V_{(k)}$. One also uses propositions \ref{LieBracketsTrivial}, \ref{QkTrivial} and \ref{Q0ZeroByPriddyAlg} to calculate the $\calw(k+1)$-structure of $V_{(k+1)}$ at each stage.
\end{proof}
\begin{prop}
For each $k\geq2$,
\[(H^*_{\calw(k+1)}V^*_{(k+1)})^{s_{k+2},0,s_k,\ldots,s_1}_{t}\cong (H^{s_{k+2}}_{\calL(k)}V^*_{(k)})^{s_k,\ldots,s_1}_t\cong (\Ftwo [v_1^{2}]\otimes E(v_0))_t^{s_{k+2},s_{k},\ldots,s_1}.\]
%The only non-zero classes in these groups are %
These groups are zero unless $s_k=\cdots =s_2=0$.
\end{prop}
\begin{proof}
One performs this calculation in the Chevalley-Eilenberg-May complex $\dual\UEAX(V^*_{(k)})$, which is the differential graded algebra $\Ftwo [v_0,v_1]$ with differential defined by
\[d(v_0)=(\dualrestn{v_0})^2=0,\ \ d(v_1)=(\dualrestn{v_1})^2=v_0^2.\qedhere\]
\end{proof}
By a greatly simplified version of the proof of theorem \ref{thm on collapsing of most sseqs}:
\begin{cor}\label{statement of result on 2d w2 example}
For each $k\geq2$, $(E_{2,(k+1)})^{s_{k+2},s_{k+1},\ldots,s_1}_{t}$ is zero unless $s_{k+1}=\cdots s_2=0$, so that the spectral sequence $E_{2,(k+1)}\implies H^*_{\calw(k)}V^*_{(k)}$ collapses, and in particular,
\[H^*_{\calw(2)}V^*_{(2)}\cong \Ftwo [v_1^{2}]\otimes E(v_0).\]
\end{cor}

%\subsection{An infinite-dimensional example in $\calw(1)$}
%In this section, we suppose that $S,T\geq1$, and let $X=V^*_{(1)}\in\calw(1)$ be the infinite dimensional object of $\calw(1)$ spanned by non-zero classes
%\[v_{j}^{*}\in (V^*_{(1)})^{2^{j}(T+1)-1}_{S+j}\textup{ for $j\geq 0$},\] such that $v_{j+1}^*=v^*_j\lambda_{1}$ for $j\geq 0$, and all other operations are trivial. %(We must have $S\geq1$ for this definition to make sense: $v_0\lambda_1$ must be defined.)
%\begin{prop}\label{calc of koszul complex in inf dim example}
%The Koszul complex $K_*^{\calU(1)}V_{(1)}^*$ has basis
%\[\left\{\Sq_{\textup{v}}(J,v_j^*)\ \middle|\ \textup{$j\geq0$, $J$ is $\Sq$-admissible, $\minDimSq(J)\leq S+j$ and $1\notin J$}\right\} \]
%%\ \genfrac{}{}{0pt}{}{\textup{$J$ $\Sq$-admissible with }\minDimSq(J)\leq S+k,}{\textup{$J$ doesn't contain 1}}\right\},\]
%and all differentials zero except for:
%\[\Sq_{\textup{v}}((i_\ell,\ldots,i_2,2),v^*_{j})\longmapsto \Sq_{\textup{v}}((i_\ell,\ldots,i_2),v^*_{j+1}),\]
%so that $V^*_{(2)}:=H_*^{\calU(1)}V_{(1)}^{*}$ is the following subquotient of $K_*^{\calU(1)}V_{(1)}^*$:
%\[\frac{{\F_{2}\left\{\Sq_{\textup{v}}(J,v_j^*)\ \middle|\ \makebox[\widthof{$\textup{$j\geq0$}$}][c]{$\textup{$j\geq0$}$}\textup{, $J$ is $\Sq$-admissible, $\minDimSq(J)\leq S+j$ and}\makebox[\widthof{$\textup{ $1,2,3\notin J$}$}][c]{$\textup{ $1,2\notin J$}$}\right\}}}{{\F_{2}\left\{\Sq_{\textup{v}}(J,v_j^*)\ \middle|\ \makebox[\widthof{$\textup{$j\geq0$}$}][c]{$\textup{$j\geq1$}$}\textup{, $J$ is $\Sq$-admissible, $\minDimSq(J)\leq S+j$ and}\makebox[\widthof{$\textup{ $1,2,3\notin J$}$}][c]{$\textup{ $1,2,3\notin J$}$}\right\}}}.\]
%Equivalently, $V_{(2)}$ is the subquotient of $F_{\calM_v(2)}V_{(1)}$ in which we restrict to the sub-$\calMv(2)$-object generated by the elements%\textbf{{(Can some of these be 0? yes if $S=1$, at least. How does this effect the calculation here?)}}
%\[\{v_0,\Sqv^2v_{1},\Sqv^3v_{1},\Sqv^2v_{2},\Sqv^3v_{2},\Sqv^2v_{3},\Sqv^3v_{3},\ldots\}\]
%and in which we set $\Sqv^2 v_{j}$ to zero for all $j\geq0$.  All of the $\Sqv^3v_{j}$ are non-zero, except when $j=S=1$. If $S\geq2$, then $V_{(2)}^{*}$ is trivial as an object of $\calw(2)$. %, and
%%\[V_{(k)}=F_{\calMv(k)}F_{\calMv(k-1)}\cdots F_{\calMv(3)}V_{(2)}.\]
%If $S=1$, then $V^*_{(2)}$ supports a single non-zero operation:
%\[(V_{(2)}^{*})_{0,1}^{T}\ni v_0^*\overset{\lambda_0}{\longmapsto}v_1^*\in (V_{(2)}^{*})_{0,2}^{2T+1}.\]
%% $\Sqv(\emptyset,v_0^*)\lambda_0=\Sqv(\emptyset,v_1^*)$. \textbf{fill in details on higher $V_{(k)}$}
%\end{prop}
%\begin{proof}
%The basis for the Koszul complex is just a reading of [above], but we must think a little about the differentials. As $\lambda_1$ is the only non-zero operation:
%\[d(\Sqv(J,v^*_j))=\sum_{ \substack{\produces{(k_{\ell},\ldots,k_2,2)}{J}{\Sq}\\(k_{\ell},\ldots,k_2){\,\Sq\textup{-admis.}}}}\!\!\!\!\!\!\!\!\!\!\!\! \Sqv((k_{\ell},\ldots,k_2),v_{j+1}^*).\]
%Consider a sequence $(k_\ell,\ldots,k_2,2)$ corresponding to a summand of this formula. Supposing that $\ell\geq2$ and $\Sqv^{k_2}\Sqv^2$ is not $\Sq$-admissible, it follows that $k_2$ is either 3 or 2, so that $\Sqv^{k_2}\Sqv^2$ is either zero or $\Sqv^{3}\Sqv^1$. As $J$ does not contain 1, and the two-sided ideal in $\LieSteen$ generated by $\Sqh^1$ is spanned by those admissible sequences ending in $\Sqh^1$, it cannot happen that $\produces{(k_{\ell},\ldots,k_2,2)}{J}{\Sq}$. Thus, the only summand appearing is that in which $(k_{\ell},\ldots,k_2,2)=J$, confirming our description of the differential. Taking the homology of this differential provides the formula for $V_{(2)}^*$, and dualizing provides that for $V_{(2)}$.
%
%In order to determine $V_{(2)}^*$ as an object of $\calw(2)$, note first that \ref{LieBracketsTrivial} and \ref{QkTrivial} show that all operations are zero except perhaps for $\lambda_0$. %Now $(\Sqv(J,w^*))\lambda_0$ is evidently zero, 
%Consider the operation $\lambda_0$ applied to a cycle of the form $\Sqv(J,v_{j}^*)\in K_{*}^{\calU(1)}V^*_{(1)}$ with $J\neq\emptyset$ (so that $1,2\notin J$). As $J$ ends in an integer no less than 3, and as $\lambda_1$ is the only non-zero operation in $V_{(1)}^*$, the second part of proposition \ref{Q0ZeroByPriddyAlg} implies that $\Sqv(J,v_{j}^*)\lambda_0=0$.
%
%In the case $J=\emptyset$, proposition \ref{Q0ZeroByPriddyAlg} states that $(\Sqv(\emptyset,v_{j}^*))\lambda_0\in V_{(2)}^*$ is represented by $(\Sqv(\emptyset,v_{j}^*\lambda_{S+j}))$, which is zero unless $j=0$ and $S=1$. Thus the only non-zero operation on $V_{(2)}^*$ is $v_0^*\lambda_0=v_1^*$ in the case $S=1$.
%\end{proof}
%
%\begin{prop}
%The spectral sequence $H^*_{\calw(2)}V_{(2)}^*\implies H^*_{\calw(1)}V_{(1)}^*$ collapses.
%\end{prop}
%\begin{proof}
%Initially, we assume that $S\geq2$. The first point is to observe that the generators $v_0$ and $\Sqv^3v_j$ ($j\geq1$) of $V_{(2)}$ are all permanent cycles in
%$(H^{*}_{\calw(2)}V_{(2)})^{0**}_{*}$. For $v_0\in (E_2)^{00S}_{T}$, this is obvious. It is less obvious for $\Sqv^3v_j$ ($j\geq1$), whose only opportunity to support a differential is
%\[\Sqv^3v_j\in (E_2)^{0,1,2+S+j}_{2^{j+1}(T+1)-1}\overset{d_2}{\to}(E_2)^{2,0,2+S+j}_{2^{j+1}(T+1)-1}.\]
%Fortunately, this target group is zero, due to the constraint that $s_2=0$. To see this, note that this group is spanned by the products of three classes in $(E_{2})^{00*}_*$, namely:
%\[v_{j_1}v_{j_2}v_{j_3}\in (E_{2})^{2,0,3S+j_1+j_2+j_3}_{(2^{j_1}+2^{j_2}+2^{j_3})(T+1)-1},\]
%and if this target group is non-zero, these indices must coincide.
%In order that $2^{j+1}$ equals $2^{j_{1}}+2^{j_{2}}+2^{j_{3}}$ it must happen that $j_1,j_2,j_3$ equal $j,j-1,j-1$ (in some order), but then $2+S+j=3S+j_1+j_2+j_3$ implies that $S+j=2$. This is impossible, as $S\geq2$ and $j\geq1$.
%
%Next, we can derive that $\Sqv^Jv_j$ is a permanent cycle for all $J,j$ such that $J$ is $\Sq$-admissible, with final entry 3 when $j>0$. For this, we will use  proposition \ref{edgehomproposition}, that there is a commuting diagram:%as $V_{(2)}^*$ is a trivial object, there is an isomorphism $V_{(2)}\cong H^{0}_{\calw(2)}V_{(2)}^*$. In light of this isomorphism, proposition \ref{edgehomproposition} shows that the edge homomorphism itself is compatible with the action of the vertical Steenrod operations, in the sense that
%\[\xymatrix@R=4mm{
%(H^*_{\calw(1)}V_{(1)}^*)^{s_{2},s_1}_t\ar[r]^-{\Sqv^i}
%\ar[d]^-{\textup{edge hom}}
%&%r1c1
%(H^*_{\calw(1)}V_{(1)}^*)^{s_{2}+1,s_1+i-1}_{2t+1}\ar[d]^-{\textup{edge hom}}
%\\%r1c2
%(E_{2})^{0,s_{2},s_1}_t\ar@{=}[d]%\ar[r]^-{\Sqv^i}
%&%r1c1
%(E_2)^{0,s_{2}+1,s_1+i-1}_{2t+1}\ar@{=}[d]\\
%(V_{(2)})^{s_{2},s_1}_t\ar[r]^-{\Sqv^i}
%&%r1c1
%(V_{(2)})^{s_{2}+1,s_1+i-1}_{2t+1}
%}\]
%Now as we have shown that the classes $v_0$ and $\Sqv^{3}v_j$  ($j\geq1$) are all permanent cycles, they are in the image of the edge homomorphism. Then this diagram shows that all of $V_{(2)}$ is in the image of the edge homomorphism, so that every element  of $V_{(2)}$ is a permanent cycle. Finally, as $V_{(2)}$ is a trivial object of $\calw(2)$, $E_{2}$ is (freely) generated by $V_{(2)}$ under the $\calmh(3)$- and $\calmv(3)$-operations, by theorem \ref{thm on collapsing of most sseqs}. As we understand how these operations interact with the differential, we have shown that the spectral sequence collapses.
%
%When $S=1$ we must modify this proof a little. Indeed, in this case, $E_{2}=H^*_{\calw(2)}V_{(2)}^*$ is no longer generated by $V_{(2)}$ under the $\calmh(3)$- and $\calmv(3)$-operations. Rather, propositions \ref{2d example in w2} and \ref{Koszul-dual Hilton-Milnor theorem}, theorem \ref{thm on collapsing of most sseqs} and corollary \ref{statement of result on 2d w2 example} show that 
%%\[H^*_{\calw(2)}V_{(2)}^*\cong H^*_{\calw(2)}(\Ftwo \{v_0^*,v_1^*\}\oplus (V'_{(2)})^*)\cong \Ftwo [v_1^{2}]\otimes E(v_0)\otimes \calmh(3)\calmv(3)V'_{(2)} ,\]
%\begin{alignat*}{2}
%H^*_{\calw(2)}V_{(2)}^*&=H^*_{\calw(2)}(\Ftwo \{v_0^*,v_1^*\}\oplus (V'_{(2)})^*)\\
%&\cong \Ftwo [v_1^{2}]\otimes E(v_0)\otimes \calmh(3)\calmv(3)V'_{(2)}
%\end{alignat*}
%where $(V'_{(2)})^*$ is the trivial object of $\calw(2)$ dual to $V'_{(2)}$, the subquotient of $F_{\calM_v(2)}V_{(1)}$ in which we restrict to the sub-$\calMv(2)$-object generated by the elements
%\[\{\Sqv^2v_{2},\Sqv^3v_{2},\Sqv^2v_{3},\Sqv^3v_{3},\ldots\}\]
%an in which we set $\Sqv^2 v_{j}$ to zero for all $j\geq2$. This object is chosen so that $V_{(2)}^*$ decomposes as a direct sum $\Ftwo \{v_0^*,v_1^*\}\oplus (V'_{(2)})^*$ (of objects of $\calw(2)$), in light of the description given in  proposition \ref{calc of koszul complex in inf dim example}.
%
%Now we may proceed as before, except that in the case $S=1$, the class $v_1^{2}$ becomes relevant, as $v_1$ is no longer present in $H^*_{\calw(2)}V_{(2)}^*$, while $\Sqv^{3}v_1$ becomes zero. That $v_1^2\in(H^*_{\calw(2)}V_{(2)}^*)^{1,0,4}_{4T-3}$ cannot support differentials is obvious, while for $j\geq2$, the same degree argument as before shows that $\Sqv^3v_j$ is a permanent cycle. As before, an argument with the edge homomorphism shows that every element of $V'_{(2)}$ is a permanent cycle, and we have found enough permanent cycles to generate $E_2$ under  the $\calmh(3)$- and $\calmv(3)$-operations.
%
% \textbf{Obviously we need better notation for the nested vertical constructions.}
%\end{proof}
%
%
%






\subsection{An infinite-dimensional example in $\calw(1)$}
In this section, we suppose that $S,T\geq1$, and let $X=V^*_{(1)}\in\calw(1)$ be the infinite dimensional object of $\calw(1)$ spanned by non-zero classes
\[v_{j}^{*}\in (V^*_{(1)})^{2^{j}(T+1)-1}_{S+j}\textup{ for $j\geq 0$},\] such that $v_{j+1}^*=v^*_j\lambda_{1}$ for $j\geq 0$, and all other operations are trivial. %(We must have $S\geq1$ for this definition to make sense: $v_0\lambda_1$ must be defined.)
\begin{prop}\label{calc of koszul complex in inf dim example}
The Koszul complex $K_*^{\calU(1)}V^*_{(1)}$ has basis
\[\left\{\Sq_{\textup{v}}(J,v_j^*)\ \middle|\ \textup{$j\geq0$, $J$ is $\Sq$-admissible, $\minDimSq(J)\leq S+j$ and $1\notin J$}\right\} \]
%\ \genfrac{}{}{0pt}{}{\textup{$J$ $\Sq$-admissible with }\minDimSq(J)\leq S+k,}{\textup{$J$ doesn't contain 1}}\right\},\]
and all differentials zero except for:
\[\Sq_{\textup{v}}((i_\ell,\ldots,i_2,2),v^*_{j})\longmapsto \Sq_{\textup{v}}((i_\ell,\ldots,i_2),v^*_{j+1}).\]
\end{prop}
\begin{proof}
The basis for the Koszul complex is just a reading of [above], but we must think a little about the differentials. As $\lambda_1$ is the only non-zero operation:
\[d(\Sqv(J,v^*_j))=\sum_{ \substack{\produces{(k_{\ell},\ldots,k_2,2)}{J}{\Sq}\\(k_{\ell},\ldots,k_2){\,\Sq\textup{-admis.}}}}\!\!\!\!\!\!\!\!\!\!\!\! \Sqv((k_{\ell},\ldots,k_2),v_{j+1}^*).\]
Consider a sequence $(k_\ell,\ldots,k_2,2)$ corresponding to a summand of this formula. Supposing that $\ell\geq2$ and $\Sqv^{k_2}\Sqv^2$ is not $\Sq$-admissible, it follows that $k_2$ is either 3 or 2, so that $\Sqv^{k_2}\Sqv^2$ is either zero or $\Sqv^{3}\Sqv^1$. As $J$ does not contain 1, and the two-sided ideal in $\LieSteen$ generated by $\Sqh^1$ is spanned by those admissible sequences ending in $\Sqh^1$, it cannot happen that $\produces{(k_{\ell},\ldots,k_2,2)}{J}{\Sq}$. Thus, the only summand appearing is that in which $(k_{\ell},\ldots,k_2,2)=J$, confirming our description of the differential.
\end{proof}
\begin{prop}\label{Sg1 calc of V2}
When $S\geq2$, $V^*_{(2)}:=H_*^{\calU(1)}V^*_{(1)}$ is the subquotient
\[\frac{{\F_{2}\left\{\Sq_{\textup{v}}(J,v_j^*)\ \middle|\ \makebox[\widthof{$\textup{$j\geq0$}$}][c]{$\textup{$j\geq0$}$}\textup{, $J$ is $\Sq$-admissible, $\minDimSq(J)\leq S+j$ and}\makebox[\widthof{$\textup{ $1,2,3\notin J$}$}][c]{$\textup{ $1,2\notin J$}$}\right\}}}{{\F_{2}\left\{\Sq_{\textup{v}}(J,v_j^*)\ \middle|\ \makebox[\widthof{$\textup{$j\geq0$}$}][c]{$\textup{$j\geq1$}$}\textup{, $J$ is $\Sq$-admissible, $\minDimSq(J)\leq S+j$ and}\makebox[\widthof{$\textup{ $1,2,3\notin J$}$}][c]{$\textup{ $1,2,3\notin J$}$}\right\}}}\]
 of $K_*^{\calU(1)}V^*_{(1)}$. Equivalently, $V_{(2)}$ is the subquotient of $F_{\calM_v(2)}V_{(1)}$ in which we restrict to the sub-$\calMv(2)$-object generated by the elements%\textbf{{(Can some of these be 0? yes if $S=1$, at least. How does this effect the calculation here?)}}
\[\{v_0,\Sqv^2v_{1},\Sqv^3v_{1},\Sqv^2v_{2},\Sqv^3v_{2},\Sqv^2v_{3},\Sqv^3v_{3},\ldots\}\]
and in which we set $\Sqv^2 v_{j}$ to zero for all $j\geq0$.   As an object of $\calw(2)$, $V^*_{(2)}$ is trivial.
\end{prop}
\begin{prop}\label{Se1 calc of V2}
When $S=1$, $V^*_{(2)}:=H_*^{\calU(1)}V^*_{(1)}$ is the subquotient
\[\frac{{\F_{2}\left\{\Sq_{\textup{v}}(J,v_j^*)\ \middle|\ \makebox[\widthof{$\textup{$j\geq2$}$}][c]{$\textup{$j\geq0$}$}\textup{, $J$ is $\Sq$-admissible, $\minDimSq(J)\leq S+j$ and}\makebox[\widthof{$\textup{ $1,2,3\notin J$}$}][c]{$\textup{ $1,2\notin J$}$}\right\}}}{{\F_{2}\left\{\Sq_{\textup{v}}(J,v_j^*)\ \middle|\ \makebox[\widthof{$\textup{$j\geq2$}$}][c]{$\textup{$j\geq2$}$}\textup{, $J$ is $\Sq$-admissible, $\minDimSq(J)\leq S+j$ and}\makebox[\widthof{$\textup{ $1,2,3\notin J$}$}][c]{$\textup{ $1,2,3\notin J$}$}\right\}}}\]
 of $K_*^{\calU(1)}V^*_{(1)}$. Equivalently, $V_{(2)}$ is the subquotient of $F_{\calM_v(2)}V_{(1)}$ in which we restrict to the sub-$\calMv(2)$-object generated by the elements%\textbf{{(Can some of these be 0? yes if $S=1$, at least. How does this effect the calculation here?)}}
\[\{v_0,v_1,\Sqv^2v_{2},\Sqv^3v_{2},\Sqv^2v_{3},\Sqv^3v_{3},\ldots\}\]
and in which we set $\Sqv^2 v_{j}$ to zero for all $j\geq1$.  As an object of $\calw(2)$, $V^*_{(2)}$ admits a single non-zero operation, $\lambda_0:v_0^*\longmapsto v_1^*$, and so decomposes as the direct sum of
$\Ftwo \{v_0^*,v_1^*\}$ with a trivial object $\dual(V'_{(2)})$, dual to $V'_{(2)}$, the subquotient of $F_{\calM_v(2)}V_{(1)}$ in which we restrict to the sub-$\calMv(2)$-object generated by
$\{\Sqv^2v_{2},\Sqv^3v_{2},\Sqv^2v_{3},\Sqv^3v_{3},\ldots\}$ and set $\Sqv^2 v_{j}$ to zero for all $j\geq2$.
\end{prop}
%
%\begin{prop}
%If $S\geq2$, then $V_{(2)}^{*}$ is trivial as an object of $\calw(2)$. %, and
%%\[V_{(k)}=F_{\calMv(k)}F_{\calMv(k-1)}\cdots F_{\calMv(3)}V_{(2)}.\]
%If $S=1$, then $V^*_{(2)}$ supports a single non-zero operation:
%\[(V_{(2)}^{*})_{0,1}^{T}\ni v_0^*\overset{\lambda_0}{\longmapsto}v_1^*\in (V_{(2)}^{*})_{0,2}^{2T+1}.\]
%% $\Sqv(\emptyset,v_0^*)\lambda_0=\Sqv(\emptyset,v_1^*)$. \textbf{fill in details on higher $V_{(k)}$}
%\end{prop}
\begin{proof}[Proof of \ref{Sg1 calc of V2} and \ref{Se1 calc of V2}]
For any $S\geq1$, taking the homology of this differential provides the formula for $V^*_{(2)}$, and dualizing provides that for $V_{(2)}$.
In order to determine $V^*_{(2)}$ as an object of $\calw(2)$, note first that \ref{LieBracketsTrivial} and \ref{QkTrivial} show that all operations are zero except perhaps for $\lambda_0$. %Now $(\Sqv(J,w^*))\lambda_0$ is evidently zero, 
Consider the operation $\lambda_0$ applied to a cycle of the form $\Sqv(J,v_{j}^*)\in K_{*}^{\calU(1)}V^*_{(1)}$ with $J\neq\emptyset$ (so that $1,2\notin J$). As $J$ ends in an integer no less than 3, and as $\lambda_1$ is the only non-zero operation in $V^*_{(1)}$, the second part of proposition \ref{Q0ZeroByPriddyAlg} implies that $\Sqv(J,v_{j}^*)\lambda_0=0$.

In the case $J=\emptyset$, proposition \ref{Q0ZeroByPriddyAlg} states that $(\Sqv(\emptyset,v_{j}^*))\lambda_0\in V^*_{(2)}$ is represented by $(\Sqv(\emptyset,v_{j}^*\lambda_{S+j}))$, which is zero unless $j=0$ and $S=1$. Thus the only non-zero operation on $V^*_{(2)}$ is $v_0^*\lambda_0=v_1^*$ in the case $S=1$.
\end{proof}
\begin{thm}\label{W2 to W1 collapse}
The spectral sequence $H^*_{\calw(2)}V^*_{(2)}\implies H^*_{\calw(1)}V^*_{(1)}$ collapses, with
\[E_2=H^*_{\calw(2)}V^*_{(2)}\cong\begin{cases}
F_{\calmh(3)}F_{\calmv(3)}V_{(2)},&\textup{if }S\geq2;\\
F_{\calmh(3)}F_{\calmv(3)}V'_{(2)}\otimes \Ftwo [v_1^{2}]\otimes E(v_0),&\textup{if }S=1.
%\\,&\textup{if }
\end{cases}
\]
\end{thm}
\begin{proof}
The calculations of $E_2$ follow from propositions \ref{Koszul-dual Hilton-Milnor theorem}, \ref{2d example in w2}, \ref{Sg1 calc of V2} and \ref{Se1 calc of V2}, theorem \ref{thm on collapsing of most sseqs} and corollary \ref{statement of result on 2d w2 example}. What remains is to prove the collapsing result in each case.

Suppose that $S\geq2$. The first point is to observe that the generators $v_0$ and $\Sqv^3v_j$ ($j\geq1$) of $V_{(2)}$ are all permanent cycles in
$(H^*_{\calw(2)}V_{(2)})^{0**}_{*}$. For $v_0\in (E_2)^{00S}_{T}$, this is obvious. It is less obvious for $\Sqv^3v_j$ ($j\geq1$), whose only opportunity to support a differential is
\[\Sqv^3v_j\in (E_2)^{0,1,2+S+j}_{2^{j+1}(T+1)-1}\overset{d_2}{\to}(E_2)^{2,0,2+S+j}_{2^{j+1}(T+1)-1}.\]
Fortunately, this target group is zero, due to the constraint that $s_2=0$. To see this, note that this group is spanned by the products of three classes in $(E_{2})^{00*}_*$, namely:
\[v_{j_1}v_{j_2}v_{j_3}\in (E_{2})^{2,0,3S+j_1+j_2+j_3}_{(2^{j_1}+2^{j_2}+2^{j_3})(T+1)-1},\]
and if this target group is non-zero, these indices must coincide.
In order that $2^{j+1}$ equals $2^{j_{1}}+2^{j_{2}}+2^{j_{3}}$ it must happen that $j_1,j_2,j_3$ equal $j,j-1,j-1$ (in some order), but then $2+S+j=3S+j_1+j_2+j_3$ implies that $S+j=2$. This is impossible, as $S\geq2$ and $j\geq1$.

Next, we can derive that $\Sqv^Jv_j$ is a permanent cycle for all $J,j$ such that $J$ is $\Sq$-admissible, with final entry 3 when $j>0$. For this, we will use  proposition \ref{edgehomproposition}, that there is a commuting diagram:%as $V_{(2)}^*$ is a trivial object, there is an isomorphism $V_{(2)}\cong H^{0}_{\calw(2)}V_{(2)}^*$. In light of this isomorphism, proposition \ref{edgehomproposition} shows that the edge homomorphism itself is compatible with the action of the vertical Steenrod operations, in the sense that
\[\xymatrix@R=4mm{
(H^*_{\calw(1)}V^*_{(1)})^{s_{2},s_1}_t\ar[r]^-{\Sqv^i}
\ar[d]^-{\textup{edge hom}}
&%r1c1
(H^*_{\calw(1)}V^*_{(1)})^{s_{2}+1,s_1+i-1}_{2t+1}\ar[d]^-{\textup{edge hom}}
\\%r1c2
(E_{2})^{0,s_{2},s_1}_t\ar@{=}[d]%\ar[r]^-{\Sqv^i}
&%r1c1
(E_2)^{0,s_{2}+1,s_1+i-1}_{2t+1}\ar@{=}[d]\\
(V_{(2)})^{s_{2},s_1}_t\ar[r]^-{\Sqv^i}
&%r1c1
(V_{(2)})^{s_{2}+1,s_1+i-1}_{2t+1}
}\]
Now as we have shown that the classes $v_0$ and $\Sqv^{3}v_j$  ($j\geq1$) are all permanent cycles, they are in the image of the edge homomorphism. Then this diagram shows that all of $V_{(2)}$ is in the image of the edge homomorphism, so that every element  of $V_{(2)}$ is a permanent cycle. Finally, as $E_{2}$ is (freely) generated by $V_{(2)}$ under the $\calmh(3)$- and $\calmv(3)$-operations, and we understand how these operations interact with the differential, this shows that the spectral sequence collapses.

%%When $S=1$ we must modify this proof a little, as $E_{2}=H^*_{\calw(2)}V_{(2)}^*$ is no longer generated by $V_{(2)}$ under the $\calmh(3)$- and $\calmv(3)$-operations. Rather, propositions \ref{2d example in w2} and \ref{Koszul-dual Hilton-Milnor theorem}, theorem \ref{thm on collapsing of most sseqs} and corollary \ref{statement of result on 2d w2 example} show that 
%%%\[H^*_{\calw(2)}V_{(2)}^*\cong H^*_{\calw(2)}(\Ftwo \{v_0^*,v_1^*\}\oplus (V'_{(2)})^*)\cong \Ftwo [v_1^{2}]\otimes E(v_0)\otimes F_{\calmh(3)}F_{\calmv(3)}V'_{(2)} ,\]
%%\begin{alignat*}{2}
%%H^*_{\calw(2)}V_{(2)}^*&=H^*_{\calw(2)}(\Ftwo \{v_0^*,v_1^*\}\oplus (V'_{(2)})^*)\\
%%&\cong \Ftwo [v_1^{2}]\otimes E(v_0)\otimes F_{\calmh(3)}F_{\calmv(3)}V'_{(2)}.
%%\end{alignat*}
%where $(V'_{(2)})^*$ is the trivial object of $\calw(2)$ dual to $V'_{(2)}$, the subquotient of $F_{\calM_v(2)}V_{(1)}$ in which we restrict to the sub-$\calMv(2)$-object generated by the elements
%\[\{\Sqv^2v_{2},\Sqv^3v_{2},\Sqv^2v_{3},\Sqv^3v_{3},\ldots\}\]
%an in which we set $\Sqv^2 v_{j}$ to zero for all $j\geq2$. This object is chosen so that $V_{(2)}^*$ decomposes as a direct sum $\Ftwo \{v_0^*,v_1^*\}\oplus (V'_{(2)})^*$ (of objects of $\calw(2)$), in light of the description given in  proposition \ref{calc of koszul complex in inf dim example}.
Instead $E_2$ is generated by the class $v_1^{2}$  (and we no longer need to consider $\Sqv^{3}v_1$, as it is zero when $S=1$). That $v_1^2\in(H^*_{\calw(2)}V^*_{(2)})^{1,0,4}_{4T-3}$ cannot support differentials is obvious, while for $j\geq2$, the same degree argument as before shows that $\Sqv^3v_j$ is a permanent cycle. As before, an argument with the edge homomorphism shows that every element of $V'_{(2)}$ is a permanent cycle, so $E_2$ is again generated by permanent cycles under the $\calmh(3)$- and $\calmv(3)$-operations, completing the proof.

 \textbf{Obviously we need better notation for the nested vertical constructions.}
\end{proof}

\begin{cor}
Some corollary giving full structure of $H^*_{(1)}$.
\end{cor}




\end{Calculations of HWn for n nonzero}

\begin{Calculations of HW0}


\section{\textbf{Calculations of ${\calw(0)}$-cohomology}}
Let $X=V^*_{(0)}\in\calw(0)$ be a one dimensional object of $\calw(0)$, dual to a one-dimensional vector space $V_{(0)}\in\vect{0}{+}$, with non-zero element $\imath\in(V_{(0)})_T$. Write $\imath^*\in X^T$ for the non-zero element of $X$. As every $\calw(0)$-operation changes degrees, $X$ is necessarily trivial.

In order to state our next result, define
\[\aDT:=\left\{I\ \middle|\ \textup{$I$ a non-empty $\delta$-admissible sequence with }\minDimP(I)\leq T\right\},\]
and define a function $t:\aDT\to \aDT$ by
\[I=(i_{\ell},\ldots,i_1)\overset{\smash{t}}{\longmapsto }(T+nI+\ell,i_\ell,\ldots,i_1)\]
This is indeed a well defined injective endomorphism of the set $\aDT$, in that it preserves admissibility and the condition $\minDimP(I)\leq T$. The claim about $\minDimP(I)$ holds essentially by definition. For $\delta$-admissibility:
\begin{alignat*}{2}
i_{\ell}
&\leq
\ell-1+i_{\ell-1}+\cdots +i_1+T&\qquad&\textup{(as $\minDimP(I)\leq T$)}%
\\
% Left hand side
2i_{\ell}
% Relation
&\leq
% Right hand side
\ell-1+i_{\ell}+\cdots +i_1+T\\
&< \ell+i_{\ell}+\cdots +i_1+T%
\end{alignat*}
as required. The strict inequality here is interesting: suppose that $I=(i_\ell,\ldots,i_1)\in \aDT$ and 
\[(i_{j+1},\ldots,i_1)=t(i_{j},\ldots,i_1)\textup{ for some $j$, $1\leq j<i-1$.}\]
Thanks to this strict inequality, it need not be true that $I=t^{i-1-j}(i_{j+1},\ldots,i_1)$. %This is in contrast to the situation for standard unstable algebras over the Steenrod algebra. Suppose that $x\in H^n(X;\Ftwo )$, for $X$ a topological space, and $I$ is a $\Sq$-admissible sequence with excess $e(I)\leq $
Define:
\[\aDTirr:= \aDT\setminus\im(t:\aDT\to\aDT),\]
the set of sequences in $\aDT$ not in the image of $t$, so that we may decompose $\aDT$ as the disjoint union
\[\aDT=\bigsqcup_{\smash{I\in \aDTirr}}\left\{I,tI,t^2I,\ldots\right\}.\]
\begin{prop}
\label{calc of V1 from W0 sphere}
$V^*_{(1)}:=H_*^{\calU(0)}V^*_{(0)}$ has basis $\{\imath^*\}\sqcup\left\{\delta(I,\imath^*)\ \middle|\ I\in\aDT\right\}$ and all $\calw(1)$-operations trivial except for $\lambda_1$, which is defined (only when $\ell(I)\geq1$) by
\[\delta(I,\imath^*)\overset{\smash{\lambda_1}}{\longmapsto }\delta(tI,\imath^*).\]
Thus, as an object of $\calw(1)$, $V^*_{(1)}$ decomposes as a direct sum
\[\Ftwo \{\imath^*\}\oplus\bigoplus_{\smash{I\in \aDTirr}}\Ftwo \left\{\delta(I,\imath^*)\lambda_1^j\ |\ j\geq0\right\}.\]
\end{prop}
\begin{proof}
The basis of the Koszul complex was described in proposition \ref{propDerivedIndTrivialUobject n=0}, and the Koszul differential is zero as $X$ is trivial. The $\lambda$-operations were calculated in proposition \ref{QkTrivial}.
\end{proof}
Now we have put considerable effort into calculating $H^*_{\calw(1)}$ of each summand in this decomposition: theorem \ref{thm on collapsing of most sseqs} proves that
\[H^*_{\calw(1)}(\Ftwo \{\imath^*\})\cong F_{\calmh(2)}F_{\calmv(2)}(\Ftwo \{\imath\})\cong E(\imath),\]
while propositions \ref{Sg1 calc of V2} and \ref{Se1 calc of V2} calculate
\[H^*_{\calw(1)}\left(\Ftwo \left\{\delta(I,\imath^*)\lambda_1^j\ |\ j\geq0\right\}\right)\textup{ for $I\in\aDTirr$.}\]
With a view to calculating the first composite functor spectral sequence,
we catalogue a collection of generators of $E_{2,(1)}$ under the $\calmv(2)$- and $\calmh(2)$-operations. The `fundamental class' $\imath\in (E_2)^{0,0,(1)}_{d}$ is an exterior generator (arising in theorem  \ref{thm on collapsing of most sseqs}). %, as are the classes $\delta_k^\textup{v}\imath\in (E_2)^{0,1,(2)}_{d+k+1}$ for $2\leq k\leq d$ (denoted $v_0$ in theorem \ref{W2 to W1 collapse} \textbf{or the corollary to that: write it)).
Moreover, for all $I\in\aDTirr$, there are further generators, arising in theorem \ref{W2 to W1 collapse}: \textbf{or the corollary to that: write it)}
\begin{alignat*}{2}
\delta_I^{\textup{v}}\imath&\in (E_2)^{0,\ell I,(2^{\ell I})}_{d+nI+\ell I}&\qquad&\\
\Sq_{\textup{v}}^{3}\delta_{t^j\!I}^{\textup{v}}\imath&\in (E_2)^{1,2+\ell I+j,(2^{1+\ell I+j})}_{2^{j+1}(d+nI+\ell I+1)-1}&\qquad&\textup{(when $j\geq1$, but not $j=\ell I=1$),}\\
(\delta_{tI}^{\textup{v}}\imath)^{2}&\in (E_2)^{1,4,(2^{3})}_{2^{2}(d+nI+\ell I+1)-1}&\qquad&\textup{(when $\ell I=1$)},
\end{alignat*}
where they are referred to  as $v_0$, $v_j$ and $v_1^2$ respectively.
Note that this final generator, $(\delta_{tI}^{\textup{v}}\imath)^{2}$, has the same degrees as the \emph{missing} generator $\Sq_{\textup{v}}^{3}\delta_{t^j\!I}^{\textup{v}}\imath$ when $j=\ell I=1$.
\begin{thm}
The first composite functor spectral sequence collapses at $E_2$:
\[E_{2,(1)}=H^*_{\calw(1)}V^*_{(1)}\implies H^*_{\calw(0)}V^*_{(0)}.\]
\end{thm}
\begin{proof}
Evidently the funcdamental class is a permanent cycle, so prove that the spectral sequence collapses, it is enough to show that no classes
%
%
%
%
%\[x\in (E_2)^{0,\ell I,(2^{\ell I})}_{d+nI+\ell I}\textup{ or }y\in (E_2)^{1,2+\ell I+j,(2^{1+\ell I+j})}_{2^{j+1}(d+nI+\ell I+1)-1}\]
%can support a differential, whenever $j\geq1$ and $I$ is a non-empty $\delta$-admissible sequence.
%
%Now the whole $E_2$ page is a subquotient of the polynomial algebra on symbols
%\[\Sqh^A\Sqv^B\delta_C^\textup{v}\imath\in (E_2)^{\ell B+nA,2^{\ell A}(nB-\ell B+\ell C),(2^{\ell A+\ell B+\ell C})}_{2^{\ell A+\ell B}(d+nC+\ell C+1)-1},\]
%in which either $B$ is empty or has rightmost entry a 3.
%Thus, if there is to be a differential $d_r$ (for $r\geq2$) supported by $y$, %a class in $(E_2)^{1,2+\ell I+j,(2^{1+\ell I+j})}_{2^{j+1}(d+nI+\ell I+1)-1}$, 
%then $d_{r}(y) $ must be a product of $N$ such classes:
%\[\textstyle\prod_{k=1}^{N}\Sqh^{A_k}\Sqv^{B_k}\delta_{C_k}^\textup{v}\imath\in (E_2)^{-1+\sum(\ell B_k+nA_k+1),\sum2^{\ell A_k}(nB_k-\ell B_k+\ell C_k),(\sum2^{\ell A_k+\ell B_k+\ell C_k})}_{-1+\sum2^{\ell A_k+\ell B_k}(d+nC_k+\ell C_k+1)},\]
%and $r=-2+\sum(\ell B_k+nA_k+1)$.
%Now $d_r$ preserves the quatratic grading, and convexity of the exponential function implies
%\[1+\ell I+j\leq \textstyle\sum_k \left[\ell A_k+\ell B_k+\ell C_k\right].\]
%As the total degree of the differential is one, we must also have
%\[4+\ell I+j=-1+\textstyle\sum_{k}\left[\ell B_k+nA_{k}+1\right]+\textstyle\sum_{k}\left[2^{\ell A_k}(nB_k-\ell B_k+\ell C_k)\right]\]
%and combining this with the above inequality yields, after a little manipulation
%\[4\geq\textstyle\sum_k\left[(nA_k-\ell A_k)+1+(2^{\ell A_k}-1)\ell C_k+2^{\ell A_k}(nB_k-\ell B_k)\right],\]
%where each quantity $(nA_k-\ell A_k)$, $1$, $(2^{\ell A_k}-1)\ell C_k$ and $2^{\ell A_k}(nB_k-\ell B_k)$ is non-negative.
%One sees readily that if any $\ell A_k$ is non-zero, the corresponding summand would exceed 4, as in this case, $B_k$ must be non-empty, so that $nB_k-\ell B_k\geq2$. Thus we may take all $A_k=0$, and the inequality becomes
%\[4\geq\textstyle\sum_k\left[1+(nB_k-\ell B_k)\right].\]
%Now if $\ell B_k\geq1$ then  $(nB_k-\ell B_k)\geq2$ and if $\ell B_k\geq2$ then  $(nB_k-\ell B_k)\geq7$. Thus there can be at most one term with $\ell B_k\neq0$, and for any such term, $\ell B_k=1$. In the case $\ell B_1=1$, we can have at most one other term, and the target group is either $(E_2)_*^{1,*,*}$ or $(E_2)_*^{2,*,*}$ --- neither of these groups can be the target of a $d_r$ for $r\geq2$. This leaves us investigating the case in which all $A_k$ and $B_k$ are empty, where this inequality becomes $N\leq 4$, so that  $r=N-2$ must equal 2. That is, in this case we are demanding that the gradings in
%\[(E_2)^{3,\sum \ell C_k,(\sum2^{\ell C_k})}_{-1+\sum(d+nC_k+\ell C_k+1)}\textup{ and }(E_2)^{3,1+\ell I+j,(2^{1+\ell I+j})}_{2^{j+1}(d+nI+\ell I+1)-1}\]
%coincide
%
%
%
%
\[x\in (E_2)^{0,\ell I,(2^{\ell I})}_{d+nI+\ell I}\textup{ or }y\in (E_2)^{1,2+\ell I+j,(2^{1+\ell I+j})}_{2^{j+1}(d+nI+\ell I+1)-1}\]
can support a differential, $I$ a non-empty $\delta$-admissible sequence.

To see this, all one needs to know about the entire $E_2$ page is that it is a subquotient (in which $\imath^2=0$) of the polynomial algebra on symbols
\[\Sqh^A\Sqv^B\delta_C^\textup{v}\imath\in (E_2)^{\ell B+nA,2^{\ell A}(nB-\ell B+\ell C),(2^{\ell A+\ell B+\ell C})}_{2^{\ell A+\ell B}(d+nC+\ell C+1)-1},\]
in which $B$ is $\Sq$-admissible, B does not contain $1$ or $2$, if $C$ is empty then so is $B$, and if $B$ is empty then so is $A$. These conditions imply that $nB-2\ell B\geq2^{\ell B}-1$ (\textbf{recheck!}). %Moreover, in this subquotient, $\imath^2=0$ (among other relations explained above).

If there is to be a differential $d_r$ supported by $y\in E_2$, %a class in $(E_2)^{1,2+\ell I+j,(2^{1+\ell I+j})}_{2^{j+1}(d+nI+\ell I+1)-1}$, 
then $d_{r}(y) $ must be a sum of products of $N\geq1$ such classes. The generic such monomial may be written as:
\[\textstyle\prod_{k=1}^{N}\Sqh^{A_k}\Sqv^{B_k}\delta_{C_k}^\textup{v}\imath\in (E_2)^{\sum(\ell B_k+nA_k)+N-1,\sum2^{\ell A_k}(nB_k-\ell B_k+\ell C_k),(\sum2^{\ell A_k+\ell B_k+\ell C_k})}_{\sum2^{\ell A_k+\ell B_k}(d+nC_k+\ell C_k+1)-1}\]
in which $\ell C_k=0$ for at most one $k$. We derive the following:
\begin{gather}
\textstyle \sum(\ell B_k+nA_k)\geq4-N,\label{1eqnofindices}\\
\log_2(N)+\textstyle \frac{1}{N}\textstyle \sum_k \left[\ell A_k+\ell B_k+\ell C_k\right]\geq1+\ell I+j,\label{2eqnofindices}\\
4+\ell I+j=\textstyle\sum_{k}\left[\ell B_k+nA_{k}\right]+N-1+\textstyle\sum_{k}\left[2^{\ell A_k}(nB_k-\ell B_k+\ell C_k)\right]\label{3eqnofindices},\\
\log_2(N)\geq \textstyle\sum_{k}\left((2^{\ell A_k}-\frac{1}{N})\ell C_k+\left[\left(2^{\ell A_k}(nB_k-\ell B_k)-\frac{1}{N}\ell B_k\right)-\frac{1}{N}(\ell A_k)\right]\right).\label{4eqnofindices}
\end{gather}
The inequality (\ref{1eqnofindices}) is just the requirement that $r\geq2$, while (\ref{2eqnofindices}) results from the observation that $d_r$ preserves the quadratic grading and the convexity of the exponential function. Equation (\ref{3eqnofindices}) holds since the total degree of the differential is one, and (\ref{4eqnofindices}) is derived by rearranging the sum of (\ref{1eqnofindices}), (\ref{2eqnofindices}) and (\ref{3eqnofindices}).
%\[r=N-2+\sum(\ell B_k+nA_k)\geq2.\]
%Now $d_r$ preserves the quadratic grading, and convexity of the exponential function implies that
%\[1+\ell I+j\leq \log_2(N)+\frac{1}{N}\textstyle \sum_k \left[\ell A_k+\ell B_k+\ell C_k\right].\]
%%with equality if and only if all of the $\ell A_k+\ell B_k+\ell C_k$ coincide. This obviously occurs when $N=1$, and also when $N=2$ as we are considering an integer equation $2^\alpha=2^\beta+2^\gamma$.
%As the total degree of the differential is one, we must also have
%\[4+\ell I+j=-1+\textstyle\sum_{k}\left[\ell B_k+nA_{k}+1\right]+\textstyle\sum_{k}\left[2^{\ell A_k}(nB_k-\ell B_k+\ell C_k)\right]\]
%and combining this with the above inequalities yields, after a little manipulation:
%\[\log_2(N)\geq \textstyle\sum_{k}\left((2^{\ell A_k}-\frac{1}{N})\ell C_k+\left[\left(2^{\ell A_k}(nB_k-\ell B_k)-\frac{1}{N}\ell B_k\right)-\frac{1}{N}(\ell A_k)\right]\right).\]
This is a very strong inequality, since the expression $2^{\ell A_k}(nB_k-\ell B_k)-\frac{1}{N}\ell B_k$ is at least $2^{\ell B_k}-\frac{1}{N}$, and $nB_k-\ell B_k\geq2$ of $\ell B_k\neq0$. Thus, in  (\ref{4eqnofindices}), each expression in square brackets is always non-negative, is at least $2-\frac{1}{N}$ when $\ell B_k\neq0$, and exceeds $2-\frac{1}{N}$ if $\ell B_k\geq2$ or $\ell A_k\neq0$. 

When $N=1$ or $N=3$, $\log_2(N)<2-\frac{1}{N}$, so that (\ref{4eqnofindices}) implies that $\ell B_k=0$ for all $k$, violating (\ref{1eqnofindices}).
When $N\leq2$, $\log_2(N)\leq 2-\frac{1}{N}$, so that (\ref{4eqnofindices}) implies that $\ell B_k\neq0$ for at most one $k$, with $\ell B_k=1$, violating (\ref{1eqnofindices}).
When $N\geq4$, all but at most one of the summands $(2^{\ell A_k}-\frac{1}{N})\ell C_k$ in (\ref{4eqnofindices}) is at least $\frac{3}{4}$, and as $\frac{3}{4}(N-1)\geq\log_2(N)$ when $N\geq4$, (\ref{4eqnofindices}) is violated. Thus $y\in E_2$ is a permanent cycle.

Performing the same calculations for $d_r(x)$, we find that the inequality (\ref{4eqnofindices}) is unchanged, while (\ref{1eqnofindices}) is replaced by
\begin{gather}
\textstyle \sum(\ell B_k+nA_k)\geq3-N.\label{5eqnofindices}\end{gather}
The argument is unchanged when $N=1$ or $N\geq4$, while if $2\leq N\leq3$ we may still draw the samw conclusions from (\ref{4eqnofindices}). When $N=2$, we may assume that $\ell B_1=1$ and $\ell B_2=0$, and although (\ref{5eqnofindices}) is not violated, (\ref{4eqnofindices}) is violated as $\ell C_1\neq0$. When $N=3$, we must have $\ell C_k=0$ for each $k$, and the following equations must be satisfied
\[\ell I-1=\ell C_1+\ell C_2+\ell C_3,\ \  2^{\ell I}=2^{\ell C_1}+2^{\ell C_2}+2^{\ell C_3}.\]
As in the proof of theorem \ref{W2 to W1 collapse}, these equations imply that $\ell C_1,\ell C_2,\ell C_3$ equal $\ell I-1,\ell I-2,\ell I-2$, in some order. The first equation then implies that $\ell I=2$, implying that $\ell C_k=0$ for more than one $k$, which we have prohibited. Thus $x\in E_2$ is a permanent cycle.
\end{proof}
\begin{cor}
Some corollary giving full structure of $H^*_{(0)}$.
\end{cor}
%
%\[3+\log_2(N)+\frac{1}{N}\textstyle \sum_k \left[\ell A_k+\ell B_k+\ell C_k\right]\geq -1+\textstyle\sum_{k}\left[\ell B_k+nA_{k}+1\right]+\textstyle\sum_{k}\left[2^{\ell A_k}(nB_k-\ell B_k+\ell C_k)\right]\]
%
%\[4+\log_2(N)+\frac{1}{N}\textstyle \sum_k \left[\ell A_k+\ell B_k+\ell C_k\right]\geq \textstyle\sum_{k}\left[\ell B_k+nA_{k}+1\right]+\textstyle\sum_{k}\left[2^{\ell A_k}(nB_k-\ell B_k+\ell C_k)\right]\]
%
%%\[4+\log_2(N)+\frac{1}{N}\textstyle \sum_k \left[\ell A_k+\ell C_k\right]\geq \textstyle\sum_{k}\left[\frac{N-1}{N}\ell B_k+nA_{k}+1+2^{\ell A_k}(nB_k-\ell B_k+\ell C_k)\right]\]
%
%\[4+\log_2(N)-N\geq \textstyle\sum_{k}\left[\frac{N-1}{N}\ell B_k+(nA_{k}-\frac{1}{N}\ell A_k)+2^{\ell A_k}(nB_k-\ell B_k)+(2^{\ell A_k}-\frac{1}{N})\ell C_k\right].\]
%We have written the right hand side as a sum of $4k$ non-negative terms, so this is quite a strong inequality.
%\begin{alignat*}{2}
%3
%&=
%\textstyle\ell B_1+(nA_{1}-\ell A_1)+2^{\ell A_1}(nB_1-\ell B_1)+(2^{\ell A_1}-1)\ell C_1%
%&\qquad&\text{($N=1$)}\\
%3
%&=
%\textstyle\sum_{k}\left[\frac{1}{2}\ell B_k+(nA_{k}-\frac{1}{2}\ell A_k)+2^{\ell A_k}(nB_k-\ell B_k)+(2^{\ell A_k}-\frac{1}{2})\ell C_k\right]%
%&\qquad&\text{($N=2$)}\\
%8/3
%&>
%\textstyle\sum_{k}\left[\frac{2}{3}\ell B_k+(nA_{k}-\frac{1}{3}\ell A_k)+2^{\ell A_k}(nB_k-\ell B_k)+(2^{\ell A_k}-\frac{1}{3})\ell C_k\right]%
%&\qquad&\text{($N=3$)}\\
%2
%&\geq
%\textstyle\sum_{k}\left[\frac{3}{4}\ell B_k+(nA_{k}-\frac{1}{4}\ell A_k)+2^{\ell A_k}(nB_k-\ell B_k)+(2^{\ell A_k}-\frac{1}{4})\ell C_k\right]%
%&\qquad&\text{($N\geq 4$)}
%\end{alignat*}
%Now we just go about checking that these inequalities show that no non-zero differential $d_r(y)$ can exist for $r\geq2$. Indeed, the requirement that $r\geq2$ reduces to the requirement
%\[\sum(\ell B_k+nA_k)\geq4-N\]
%which forces
%\[\log_2(N)\geq \textstyle\sum_{k}\left[-\frac{1}{N}(\ell A_k)+[2^{\ell A_k}(nB_k-\ell B_k)-\frac{1}{N}\ell B_k]+(2^{\ell A_k}-\frac{1}{N})\ell C_k\right].\]
%
%N=1: $\ell B_k=0$, done.
%
%N=2: $\ell C_1=\ell C_2=1$
%
%
%When $N=1$, demanding that $r\geq2$ implies that $\ell B_1$
%
%\[4\geq \textstyle\sum_{k}\left[\frac{N-1}{N}\ell B_k+(nA_{k}-\frac{1}{N}\ell A_k)+\frac{2}{5}+2^{\ell A_k}(nB_k-\ell B_k)+(2^{\ell A_k}-\frac{1}{N})\ell C_k\right]\]
%
%\[3= \textstyle\sum_{k}\left[\frac{1}{2}\ell B_k+(nA_{k}-\frac{1}{2}\ell A_k)+2^{\ell A_k}(nB_k-\ell B_k)+(2^{\ell A_k}-\frac{1}{2})\ell C_k\right] (\textup{N=2})\]
%
%\[\frac{8}{3}>1+\log_2(3)\geq \textstyle\sum_{k}\left[\frac{2}{3}\ell B_k+(nA_{k}-\frac{1}{3}\ell A_k)+2^{\ell A_k}(nB_k-\ell B_k)+(2^{\ell A_k}-\frac{1}{3})\ell C_k\right] (\textup{N=3})\]
%
%
%\[2\geq4-\frac{N}{2}\geq \textstyle\sum_{k}\left[\frac{3}{4}\ell B_k+(nA_{k}-\frac{1}{4}\ell A_k)+2^{\ell A_k}(nB_k-\ell B_k)+(2^{\ell A_k}-\frac{1}{4})\ell C_k\right] (\textup{$N\geq4$})\]
%\textup{}
%
%%\[4\geq\textstyle\sum_k\left[(nA_k-\ell A_k)+1+(2^{\ell A_k}-1)\ell C_k+2^{\ell A_k}(nB_k-\ell B_k)\right],\]
%where each quantity $(nA_k-\ell A_k)$, $1$, $(2^{\ell A_k}-1)\ell C_k$ and $2^{\ell A_k}(nB_k-\ell B_k)$ is non-negative.
%One sees readily that if any $\ell A_k$ is non-zero, the corresponding summand would exceed 4, as in this case, $B_k$ must be non-empty, so that $nB_k-\ell B_k\geq2$. Thus we may take all $A_k=0$, and the inequality becomes
%\[4\geq\textstyle\sum_k\left[1+(nB_k-\ell B_k)\right].\]
%Now if $\ell B_k\geq1$ then  $(nB_k-\ell B_k)\geq2$ and if $\ell B_k\geq2$ then  $(nB_k-\ell B_k)\geq7$. Thus there can be at most one term with $\ell B_k\neq0$, and for any such term, $\ell B_k=1$. In the case $\ell B_1=1$, we can have at most one other term, and the target group is either $(E_2)_*^{1,*,*}$ or $(E_2)_*^{2,*,*}$ --- neither of these groups can be the target of a $d_r$ for $r\geq2$. This leaves us investigating the case in which all $A_k$ and $B_k$ are empty, where this inequality becomes $N\leq 4$, so that  $r=N-2$ must equal 2. That is, in this case we are demanding that the gradings in
%\[(E_2)^{3,\sum \ell C_k,(\sum2^{\ell C_k})}_{-1+\sum(d+nC_k+\ell C_k+1)}\textup{ and }(E_2)^{3,1+\ell I+j,(2^{1+\ell I+j})}_{2^{j+1}(d+nI+\ell I+1)-1}\]
%coincide
\end{Calculations of HW0}

%\begin{The cohomology of a trivial unstable Lie algebra over P}
%
%\section{\textbf{The cohomology of a trivial unstable Lie algebra over $P$}}
%There may be a wrinkle here answering the well definedness vs vanishing question --- maybe I have a mistake, and some operation is only well defined on $E^1$.
%\todo{State interest in this calculation}{write $X=\bigoplus_\alpha \Ftwo [r_\alpha]$, a finite direct sum}
%\todo{Construct the relevant elements $\Sq^J\delta_I\imath_\alpha$ and state a theorem on $H^*X$}{$\imath_\alpha\in H^0_{r_\alpha}X$ is the functional which projects $QX\cong X$ onto $\Ftwo [r_\alpha]$.
%\item Clearly state the contraints relevant for $K$ and $I$.
%\item Also describe which products thereof are not known to be zero, and state a theorem.}
%\todo{Proposition: any of the $\Sq^J\delta_I\imath_\alpha$ is detected by $\Sqv^{J_n}e\cdots e\Sqv^{J_1}e\delta_I\imath_\alpha$}{again, clearly state the constraints on the $J_i$.
%\item Push out to $ee\Sqv^{J_n}e\cdots e\Sqv^{J_1}e\delta_I\imath_\alpha\in E^2(n+2)_*^{00**\cdots *}$}
%\todo{Calculate the groups $E^2(n)_*^{*0**\cdots *}$}{Observe importance of these groups, since everything's detected in one}
%\todo{Identify products of $ee\Sqv^{J_{n-2}}e\cdots e\Sq^{J_1}e\delta_I\imath_\alpha\in E^2(n)_*^{00**\cdots *}$ in $E^2(n)_*^{*0**\cdots *}$}{I still find this hard to write down}
%\todo{Prove the theorem on $H^*X$}{start with anything in $E^2(n)$
%\item it is detected by something in $E^2(N)_*^{*0**\cdots *}$
%\item anything there is a product of $ee\Sqv^{J_{N-2}}e\cdots e\Sq^{J_1}e\delta_I\imath_\alpha$
%\item this converges down to $\Sq^K_h \Sqv^{J_n}e\cdots e\Sqv^{J_1}e\delta_I\imath_\alpha$, which must equal the original element
%\item this is a permanent cycle, proving that there are never any differentials in any of the cfsseqs
%\item thus $E^2(0)^*_*$ is a direct sum of certain of the $E^2(n)^{*0**\cdots *}_*$ via collapsing sequences
%\item stocktake reveals that these terms are everything we hoped}
%\end{The cohomology of a trivial unstable Lie algebra over P}
\appendix
\begin{appendices}


\section{\textbf{A May spectral sequence for $H^*_{\calw(0)}$}}
\todo{Explain the quadratic filtration of the bar construction}{derive a spectral sequence
\item figure out what it implies about the whole thing
}

\section{\textbf{Cohomology operations for Lie algebras}}\label{appendix on Lie coh ops}
\subsection{The partially restricted universal enveloping algebra}
\todo{Recall Priddy's definition of Lie algebra cohomology, and results}{Explain that they do not calculate everything for us already, since the actual simplicial Lie algebras we are looking at are not GEMs --- their homotopy supports non-trivial operations.
\item maybe compare $\overline{W}$ and the bar construction
\item maybe not, but if you do not, point out that the operations you get come from the diagonal of the bar construction below which is a genuine simplicial coalgebra}
In this section we will prove [????????], using an improved version of Priddy's [Proof operations coincide]. We will need one last category of graded vector spaces, $\vect{-}{n}$, an object of which is simply the direct sum of an object $V$ of $\vect{+}{n}$ and a vector space $V_{0,\ldots,0}^{-1}$:
\[V=V^{-1}_{0,\ldots,0}\oplus\bigoplus_{t\geq{1}}\bigoplus_{s_n,\ldots,s_1\geq0}V^t_{s_n,\ldots,s_1}\in\vect{-}{n}.\]
Denote by $\calA(n)$ the following category of graded augmented associative algebras. An object of $\calA(n)$ is a graded vector space
%\[A=A^{-1}_{0,\ldots,0}\oplus\bigoplus_{t\geq{1}}\bigoplus_{s_n,\ldots,s_1\geq0}A^t_{s_n,\ldots,s_1},\]
$A\in \vect{-}{n}$ such that $A^{-1}_{0,\ldots,0}=\Ftwo \langle 1\rangle$ is one-dimensional, spanned by the unit of an associative unital pairing
\[A^{t}_{s_n,\ldots,s_1}\otimes A^{q}_{p_n,\ldots,p_1}\to A^{t+q+1}_{s_n+p_n,\ldots,s_1+p_1}.\]
That is, $A_{0,\ldots,0}^{-1}$ is not part of the data of $A$, but only a graded piece added to hold the unit. Such an algebra is certainly augmented, and the augmentation ideal may be viewed as a forgetful functor $I:\calA(n)\to\calL(n)$, which sends $A$ to the partially restricted Lie algebra
\[\bigoplus_{t\geq{1}}\bigoplus_{s_n,\ldots,s_1\geq0}A^t_{s_n,\ldots,s_1},\]
with bracket $[x,y]:=xy-yx$, and restriction operation $\restn{x}:=x^2$ whenever $x\in A^t_{s_n,\ldots,s_1}$ and not all of $s_n,\ldots,s_1$ zero.

The composite forgetful functor $\calA(n)\overset{I}{\to}\calL(n)\to\vect{+}{n}$ has a left adjoint, none other than the tensor algebra functor $T$ (with unit placed in $A^{-1}_{0,\ldots,0}$ as appropriate). Moreover, the functor $I$ has a left adjoint, $\UEA$, \emph{the partially restricted universal enveloping algebra} functor, with $\UEA L$ obtained as the quotient of $T(L)$ by the two-sided ideal generated by any $[x,y]-xy-yx$ and by $\restn{x}-x^2$ with $x$ of restrictable degree. Indeed, there is a composite of adjunctions
\[\xymatrix@R=.3cm@C=1cm{
\vect{+}{n}  \ar@<.6ex>[r]^{F^{\calL(n)}}&
\calL(n)  \ar@<.4ex>[l]^{\textup{forget}} \ar@<.6ex>[r]^{\UEA}&
\calA(n),  \ar@<.4ex>[l]^{I} 
}
\]
showing that $\UEA\circ F^{\calL(n)}\cong T$. As in the non-restricted and fully restricted case, $\UEA L$ is naturally a Hopf algebra, having diagonal defined by the requirement $\Delta x=1\otimes x+x\otimes 1$ for $x\in L\subset \UEA L$, and:
\begin{lem}[Partially restricted PBW theorem]\label{Partially restricted PBW theorem}
If $L\in\calL(n)$, then there is a natural increasing filtration of $\UEA L$, the Lie filtration (by powers of $\langle 1\rangle\oplus \im(L\to \UEA(L))$), and the associated graded algebra is naturally isomorphic to $\Ftwo [L_{\textup{\textbf{0}}}]\otimes E[L_{\neq\textup{\textbf{0}}}]$, where $L=L_{\textbf{\textup{0}}}\oplus L_{\neq\textbf{\textup{0}}}$ is the decomposition of $L$ into the sum of its subspaces of in non-restrictable and restrictable degrees respectively.
\end{lem}
Here, $\Ftwo [\DASH]$ and $E[\DASH]$ denote the \emph{unital} polynomial algebra and exterior algebra functors, which differ from $S(\CommOperad)$ and $\Lambda(\CommOperad)$ only by the addition of the unit in $(\Ftwo [\DASH])^{-1}_{0,\ldots,0}$ and $(E[\DASH])^{-1}_{0,\ldots,0}$ respectively. The unit $1\otimes1 $ of this tensor product is in $(\Ftwo [\DASH]\otimes E[\DASH])^{-1}_{0,\ldots,0}$, as the product has a $+1$-shift in the cohomological dimension.
\begin{lem}
The prolonged functor $\UEA:s\calL(n)\to s\calA(n)$ preserves weak equivalences.
\end{lem}
\begin{proof}
Suppose that $L\to L'$ is a weak equivalence in $s\calL(n)$. The Lie filtration makes $C_*(\UEA L)\to C_*(\UEA L')$ a map of filtered commutative differential graded algebras, so there is an induced map of the resulting spectral sequences. By \ref{Partially restricted PBW theorem}, the $E^0$ page of the spectral sequence for $\UEA L$ is the differential graded algebra $C_*(\Ftwo [L_{\textbf{0}}]\otimes E[L_{\neq\textbf{0}}])$, where $\Ftwo [L_{\textbf{0}}]$ denotes a polynomial algebra, while $E[L_{\neq\textbf{0}}]$ denotes an exterior algebra. By Dold's theorem (\ref{Dold's theorem}), the $E^1$ page is a functor of $\pi_*(L_{\textbf{0}})$ and $\pi_*(L_{\neq\textbf{0}})$. As the induced maps $\pi_*(L_{\textbf{0}})\to\pi_*(L'_{\textbf{0}})$ and $\pi_*(L_{\neq\textbf{0}})\to\pi_*(L'_{\neq\textbf{0}})$ are isomorphisms, the map of spectral sequences is an isomorphism from $E^1$.
\end{proof}

\subsection{The proof of [????????]}

Let $ L_\bullet\in s\calL(n)$ be almost free on a fixed choice of subspaces $V_p\subseteq  L_p$.
We are interested in using the bisimplicial vector space:
\[\textbf{B}_{pq}:=\overline{B}_q \UEA L_p=(\UEA L_p)^{\otimes q}\in ss\vect{-}{n},\]
which in each simplicial level $p$ is the standard simplicial bar construction [Priddy early chapter] calculating $\Tor_{\UEA L_p}(\Ftwo ,\Ftwo )$.
This is a bisimplicial cocommutative coalgebra, with diagonal:
\[\psi_{\textup{\textbf{B}}}:\left(\overline{B}_\bullet (\UEA L_\bullet)\overset{\overline{B}(\Delta)}{\to}\overline{B}_\bullet (\UEA L_\bullet\otimes \UEA L_\bullet)\cong\overline{B}_\bullet (\UEA L_\bullet)\otimes \overline{B}_\bullet (\UEA L_\bullet)\right).\]
We are also going to use the (somewhat artificial) simplicial chain complex $\textbf{Q}\in s\,\complexes\vect{-}{n}$
\[\textbf{Q}_{\bullet *}:=\begin{cases}
Q^{\calL(n)} L_\bullet,&\textup{if }*=1;\\
\Ftwo \langle 1\rangle,&\textup{if }*=0;
\\0,&\textup{otherwise.}
\end{cases}
\]
with zero differentials in each simplicial level. Now there is a map of simplicial chain complexes $r:N_*^\textup{v}\textbf{B}_\bullet\to \textbf{Q}_{\bullet*}$ \textbf{(what kind of normalization is convenient?)}, defined in level $p$ by the identification $N^\textup{v}_0\textbf{B}_p=\Ftwo \langle 1\rangle=\textbf{Q}_{p0}$ and the composite:
\[N^\textup{v}_1\textbf{B}_{p\bullet}=I\UEA L_p\epi I\UEA L_p/(I\UEA L_p)^2\cong Q^{\calL(n)} L_p.\]
In order to prove [????????], we plan to show
\begin{prop}\label{the point of the appendix} The composite
\[N_*\Delta\textbf{\textup{B}}\simeq\textup{Tot}(N^\textup{h}_*N^\textup{v}_*\textbf{\textup{B}})\overset{r}{\to} \textup{Tot}(N^\textup{h}_*\textbf{\textup{Q}}_{\bullet*})=\Ftwo \oplus \Sigma N_*Q^{\calL(n)} L\]
is a weak equivalence of chain complexes under which the operations $\Sq^k$ and $\mu$ on $\pi^*(\dual\Delta\textup{\textbf{B}})$, which arise from the simplicial coalgebra structure $\psi_{\textup{\textbf{B}}}$ of $\Delta \textup{\textbf{B}}$, correspond to the operations $\Sq^k:=\psi_{\calL(n)}\circ\ExtCohOp^{k-1}$ and $\mu:=\psi_{\calL(n)}\circ\ExtCohProd$ on $\pi^*(\dual(Q^{\calL(n)} L))=:H^*_{\calL(n)} L$.
\end{prop}

As $ L$ is levelwise free, the evident map $T(V_p)\to \UEA L_p$ is an isomorphism for each $p$, and we define a vertical homotopy $h:N_*^\textup{v}\textbf{B}_{p\bullet}\to N_{*+1}^\textup{v}\textbf{B}_{p\bullet}$ by the following formulae (in which the $v_{i_{j}}$ are taken to be in $V_p\subseteq  L_p\subseteq\UEA L_p$):
\begin{alignat*}{2}
h_q:N^\textup{v}_q\overline{B}\UEA L_p&\overset{}{\longrightarrow} N^\textup{v}_{q+1}\overline{B}\UEA L_p\\
[\smash{\underbrace{v_{i_1}|\cdots |v_{i_{k-1}}}_{\textup{length 1 bars}}}|v_{i_{k}}v_{i_{k+1}}\cdots |\cdots ]
&\overset{}{\longmapsto}
[v_{i_1}|\cdots |v_{i_{k}}|v_{i_{k+1}}\cdots |\cdots ]%
\\
[v_{i_1}|\cdots |v_{i_{q}}]
&\overset{}{\longmapsto}0.
\end{alignat*}
This homotopy is of the same type as that used in \S[???????] and [Priddy], and commutes with all of the horizontal simplicial structure except $d^\textup{h}_0$, so that $d^\textup{h}h_q+h_qd^\textup{h} =d^\textup{h}_0h_q+h_qd^\textup{h}_0$.
\begin{lem}\label{barConstNullHtpyLemma}
Under the map $(\textup{Id}+h_{q-1}d^\textup{v}+d^\textup{v}h_{q}): N^\textup{v}_q\overline{B}\UEA L_p\to N^\textup{v}_q\overline{B}\UEA L_p$,
\begin{alignat*}{2}
[v_{i_1}\cdots |\cdots |\cdots v_{i_r}]
&\longmapsto0\text{ unless $r=q=1$, in which case}%
\\
[v_{i_1}]
&\longmapsto [v_{i_{1}}].
\end{alignat*}
\end{lem}
[\textbf{modernize this spiel:}] Now Singer \cite{SingerSteen1.pdf} gives a method of defining horizontal Steenrod operations on the spectral sequence of $\textbf{B}_{\bullet\bullet}$. His method is to define a \emph{bisimplicial Eilenberg-Zilber map} $K_k:C\left(X\otimes Y\right)\to CX\otimes CY$ for each $k$, a degree $k$ map satisfying $dK_k+K_kd=K_{k-1}+TK_{k-1}T$. His formulae are given in terms of a \emph{special} simplicial Eilenberg-Zilber map. On the dual total complex, the resulting cohomology operations satisfy the Adem operations.

Our method of proving [] will be to modify Singer's definition, using the chain homotopy $h$ to shift the horizontal operations one higher in filtration. They will still satisfy the same relations at the abutment, and as the abutment filtration is trivial, they must satisfy the same relations at $E^2$. Finally, we will note that what we have produced at $E^2$ is the definition of the Steenrod operations from \S[??????].
\begin{lem}\label{firstCompositeLemma}
The composite
\[N_{p}^\textup{h}N_2^\textup{v}\textup{\textbf{B}}\overset{\psi_{\textup{\textbf{B}}}}{\to} N_{p}^\textup{h}N_2^\textup{v}(\textup{\textbf{B}}\otimes\textup{\textbf{B}}) \overset{D_0^\textup{v}}{\to} N_{p}^\textup{h}(N_1^\textup{v}\textup{\textbf{B}}\otimes N_1^\textup{v}\textup{\textbf{B}})\overset{r\otimes r}{\to}N^\textup{h}_p(\textup{\textbf{Q}}_{\bullet 1}\otimes \textup{\textbf{Q}}_{\bullet 1})\]
vanishes except on terms $[x|y]$ where $x$ and $y$ are generators of $ L_p$, which have image $x\otimes y$.
\end{lem}
\begin{proof}
A generic element of the domain is a sum of terms $[x_1\cdots x_I|y_1\cdots y_J]$, with $x_1,\ldots,x_I$ and $y_1,\ldots,y_J$ in $V_p\subseteq L_p$. This element maps under $\psi_{\textup{\textbf{B}}}$ to
%\[\sum_{a_1=0}^1\cdots \sum_{a_I=0}^1
%\sum_{b_1=0}^1\cdots \sum_{b_J=0}^1
%\left[\prod_{i=1}^Ix_i^{a_i}\middle|\prod_{\smash{j=1}}^Jy_j^{b_j}\right]\otimes
%\left[\prod_{i=1}^Ix_i^{1-a_i}\middle|\prod_{\smash{j=1}}^Jy_j^{1-b_j}\right]\]
\[\sum_{\smash{a_1,\ldots,a_I,b_1,\ldots,b_J\in\{0,1\}}}
\left[\prod_{i=1}^Ix_i^{a_i}\middle|\prod_{\smash{j=1}}^Jy_j^{b_j}\right]\otimes
\left[\prod_{i=1}^Ix_i^{1-a_i}\middle|\prod_{\smash{j=1}}^Jy_j^{1-b_j}\right]
\in N_{p}^\textup{h}N_2^\textup{v}(\textup{\textbf{B}}\otimes\textup{\textbf{B}}),
\]
and $D_0^\textup{v}$ annihilates all terms except for those in which all $a_i$ are $1$ and all $b_j$ are $0$, leaving
\[\left[\prod_{i=1}^Ix_i\right]\otimes
\left[\prod_{\smash{j=1}}^Jy_j\right]\in N_{p}^\textup{h}(N_1^\textup{v}\textup{\textbf{B}}\otimes N_1^\textup{v}\textup{\textbf{B}}).\]
Finally, $r\otimes r$ annihilates this term unless $I=J=1$.
\end{proof}
\begin{lem}\label{secondCompositeLemma}
The composite
\[N_{p+1}^\textup{h} N_1^\textup{v}\textup{\textbf{B}}\overset{\psi_{\textup{\textbf{B}}}}{\to} N_{p+1}^\textup{h}N_1^\textup{v}(\textup{\textbf{B}}\otimes\textup{\textbf{B}}) \overset{D_1^\textup{v}}{\to} N_{p+1}^\textup{h}(N_1^\textup{v}\textup{\textbf{B}}\otimes N_1^\textup{v}\textup{\textbf{B}})\overset{r\otimes r}{\to}N^\textup{h}_{p+1}(\textup{\textbf{Q}}_{\bullet 1}\otimes \textup{\textbf{Q}}_{\bullet 1})\]
vanishes except on terms $[xy]$ where $x$ and $y$ are generators of $ L_{p+1}$, which have image $x\otimes y+y\otimes x$.
\end{lem}
\begin{proof}
A generic element of the domain is a sum of terms $[x_1\cdots x_I]$, with $x_1,\ldots,x_I$ in $V_{p+1}\subseteq L_{p+1}$. This element maps under $\psi_{\textup{\textbf{B}}}$ to
\[\sum_{\smash{a_1,\ldots,a_I\in\{0,1\}}}
\left[\prod_{i=1}^Ix_i^{a_i}\right]\otimes
\left[\prod_{i=1}^Ix_i^{1-a_i}\right].\]
As $\{D_k\}$ was chosen to be a special $k$-cup product, $D_1^\textup{v}$ acts as the identity.
Finally, $r\otimes r$ annihilates this term unless $I=2$ and $a_1\neq a_2$.
\end{proof}
\begin{lem}\label{commuting rectangle lemma for lie operations}
There is a commuting diagram:
\[\xymatrix@R=4mm@C=18mm{
N_{n+k}^\textup{h}N_1^\textup{v}\textup{\textbf{B}} \ar[r]^-{d^\textup{h}_0h_{n+k}+h_{n+k-1}d^\textup{h}_0}
\ar[d]^-{r}&%r1c1
N_{n+k-1}^\textup{h}N_2^\textup{v}\textup{\textbf{B}} \ar[r]^-{D^\textup{v}_0\circ\psi_{\textup{\textbf{B}}}}&%r1c2
N_{n+k-1}^\textup{h}(N_1^\textup{v}\textup{\textbf{B}}\otimes N_1^\textup{v}\textup{\textbf{B}})\ar[d]^-{r\otimes r}
\\%r1c3
N_{n+k}^\textup{h}Q^{\calL(n)} L  \ar[r]^-{Q^{\calL(n)}(\xi_{\calL(n)})}&%r1c1
N_{n+k-1}^\textup{h}Q^{\calL(n)}( L\smashcoprod  L) \ar[r]^-{j_{\calL(n)}}&%r1c2
N_{n+k-1}^\textup{h}(Q^{\calL(n)} L \otimes Q^{\calL(n)} L )
}\]
\end{lem}
\begin{proof}
Write
$\textup{LHS}=(r\otimes r)\circ D^\textup{v}_0\circ\psi_{\textup{\textbf{B}}}\circ(d_0^\textup{h}h+hd^\textup{h}_0)$ and $\textup{RHS}= \psi_{\calL(n)}\circ r$.
Consider first an element $e=[v_1v_2\cdots v_b]$ of $N_{n+k}^\textup{h}N_1^\textup{v}\textbf{B}$ with $b\geq2$. By definition, $r$ vanishes on such an element, so that $\textup{RHS}(e)=0$. Lemma \ref{firstCompositeLemma} states that the map $(r\otimes r)\circ D^\textup{v}_0\circ\psi_{\textup{\textbf{B}}}$ vanishes except on expressions of the form $[u|w]$ for $u,w\in V_{n+k-1}$. However, the expressions of this form appearing in $d^\textup{h}_0h(e)$ coincide with such expressions in $hd^\textup{h}_0(e)$, so that there is a cancellation, and $\textup{LHS}(e)=0$ as hoped. [\textbf{maybe} explain further.]

Next, consider an element $[v]$ of $N_{n+k}^\textup{h}N_1^\textup{v}\textbf{B}$. As $h[v]=0$, and in light of lemma \ref{firstCompositeLemma}, $\textup{LHS}([v])$ equals the quadratic part of $d_0^\textup{h}v$, after writing $d_0^\textup{h}v$ as an expression in elements of $V_{n+k-1}$. This is exactly the description given in lemma \ref{psi is basically the quadratic part} of $\textup{RHS}([v])=\psi_{\calL(n)}(v)$.
%\[\textup{RHS}([v])=(j_{\calL(n)}\circ Q^{\calL(n)}(\xi_{\calL(n)}))(v)=\psi_{\calL(n)}(v).\qedhere\]
\end{proof}
\begin{proof}[Proof of \ref{the point of the appendix}]
%Having put in place these preparations, let us start working on the theorem. 
Fix a cocycle $\alpha\in \dual(N_{n}Q^{\calL(n)} L_\bullet)$. %We may assume without loss of generality (or are we \textbf{forced} to assume) that  $\alpha$ evaluates to zero... wait... what am I talking about?
Then $\alpha$ may be viewed an a permanent cocycle at $E_2^{n,1}$ in the spectral sequence of $\dual(\textbf{Q}_{\bullet*})$. We will apply the chain-level Steenrod operation $S^k$ of [Singer ???], (which defines the Steenrod operation $\Sq^k$ on the cohomology of the total complex), to the class $r^*\alpha$. As $\alpha$ is a permanent cycle, $\delta(r^*\alpha)=0$, and
%\[S^k(r^*\alpha):=\psi^*K^*_{n+1-k}\oldphi(r^*\alpha\otimes r^*\alpha)=T_1+T_2,\textup{ where}\]
\begin{alignat*}{2}
S^k(r^*\alpha):={}&\psi_{\textup{\textbf{B}}}^*K^*_{n+1-k}\oldphi(r^*\alpha\otimes r^*\alpha)=T_1+T_2,\textup{ where}
\\
T_1
={}&
\psi_{\textup{\textbf{B}}}^*(D^\textup{v}_0)^*(D^\textup{h}_{n+1-k})^*
\oldphi(r^*\alpha\otimes r^*\alpha)\in \dual(N^\textup{h}_{n+k-1}N^\textup{v}_2\textbf{B})%
\\
T_2
={}&
\psi_{\textup{\textbf{B}}}^*(D^\textup{v}_1)^*(TD^\textup{h}_{n-k}T)^*
\oldphi(r^*\alpha\otimes r^*\alpha)\in \dual(N^\textup{h}_{n+k}N^\textup{v}_1\textbf{B})
\end{alignat*}
%
% $S^k(r^*\alpha)=\psi^*K^*_{n+1-k}\oldphi(r^*\alpha\otimes r^*\alpha)$, which has two terms:
%\[\underbrace{\psi^*(D^\textup{v}_0)^*(D^\textup{h}_{n+1-k})^*\oldphi(r^*\alpha\otimes r^*\alpha)}_{\textup{$T_1$, filtration $n+k-1$}}+
%\underbrace{\psi^*(D^\textup{v}_1)^*(TD^\textup{h}_{n-k}T)^*
%\oldphi(r^*\alpha\otimes r^*\alpha)}_{\textup{$T_2$, filtration $n+k$}}\]
Our method will be to compress each of these terms into filtration one higher, using the cochain homotopy $h^*:\dual(N^\textup{h}_* N^\textup{v}_*\textbf{B})\to \dual(N^\textup{h}_*N^\textup{v}_{*-1}\textbf{B})$
Using lemma \ref{barConstNullHtpyLemma}:
\[(\textup{Id}+\delta^\textup{v}h^*+ h^*\delta^\textup{v})T_1=0\textup{ and }(\textup{Id}+\delta^\textup{v}h^*+ h^*\delta^\textup{v})T_2=0.\]
The first equation holds as $(\textup{Id}+hd^\textup{v}+d^\textup{v}h)$ is zero on $N^\textup{v}_2\textbf{B}$. For the second equation, on $N^\textup{v}_1\textbf{B}$,  $(\textup{Id}+hd^\textup{v}+d^\textup{v}h)$ is the projection onto terms of the form $[v]$, yet lemma \ref{secondCompositeLemma} shows that the composite
\[((r\otimes r)\circ(TD^\textup{h}_{n-k}T)\circ D^\textup{v}_1\circ\psi_{\textup{\textbf{B}}}): N^\textup{h}_{n+k}N^\textup{v}_1\textbf{B}\to N^\textup{v}_{n+k}(\textbf{Q}_{\bullet1}\otimes \textbf{Q}_{\bullet1})\]
vanishes except on terms of the form $[vw]$ (recall that $r$ commutes with the horizontal simplicial structure).

As $\delta^\textup{h}h^*+ h^*\delta^\textup{h}$ increases filtration, we have compressed $S^k(r^*\alpha)$ to the filtration $n+k$ expression $(\delta^\textup{h}h^*+ h^*\delta^\textup{h})T_1$, modulo even higher filtration.
%up to even higher filtration ($n+k+1$), we have compressed $S^k(r^*\alpha)$ to the expression $(\delta^\textup{v}h^*+ h^*\delta^\textup{v})\textup{Term}_1$.
Now the commuting diagram of lemma \ref{commuting rectangle lemma for lie operations} is the left square in a larger commuting diagram:
\[\xymatrix@R=4mm@C=15mm{
N_{n+k}^\textup{h}N_1^\textup{v}\textbf{B} \ar[d]^-{r}
 \ar[r]^-{D^\textup{v}_0\circ\psi_{\textup{\textbf{B}}}\circ (d^\textup{h}h+hd^\textup{h})}&%r1c2
N_{n+k-1}^\textup{h}(N_1^\textup{v}\textbf{B}\otimes N_1^\textup{v}\textbf{B})\ar[d]^-{r\otimes r}\ar[r]^-{ D^\textup{h}_{n+1-k}}
&
N^\textup{h}_nN^\textup{v}_1\textbf{B}\otimes N^\textup{h}_nN^\textup{v}_1\textbf{B}\ar[d]^-{r\otimes r}
\\%r1c3
N_{n+k}^\textup{h}Q^{\calL(n)} L  \ar[r]^-{\psi_{\calL(n)}}&%r1c2
N_{n+k-1}^\textup{h}(Q^{\calL(n)} L \otimes Q^{\calL(n)} L )
\ar[r]^-{ D^\textup{h}_{n+1-k}}
&
N^\textup{h}_nQ^{\calL(n)} L \otimes N^\textup{h}_nQ^{\calL(n)} L 
}\]
Now $\dual(N^\textup{h}_nQ^{\calL(n)} L \otimes N^\textup{h}_nQ^{\calL(n)} L )$ contains the cocycle $\oldphi(\alpha\otimes\alpha)$, and pulling $\oldphi(\alpha\otimes\alpha)$ back to $\dual(N_{n+k}^\textup{h}N_1^\textup{v}\textbf{B})$ along the lower composite yields $r^*\psi_{\calL(n)}^*\ExtCohOp^{k-1}(\alpha)$. Pulling back along the upper composite yields the $E^2$ representative of the shifted version of Singer's operations. As $r^*$ is an $E^2$-equivalence, and the filtrations are trivial, this proves the result. A simple modification proves the result for pairings.
\end{proof}

\subsection{The Chevalley-Eilenberg-May complex}\label{The Chevalley-Eilenberg-May complex}
Suppose that $L\in\calL(n)$ is locally finite dimensional. We define a differential coalgebra, the \emph{Chevalley-Eilenberg-May complex}, to be the subcoalgebra $\UEAX(L):= E(L_{\textbf{0}})\otimes \Gamma(L_{\neq\textbf{0}})$ of the divided power coalgebra $\Gamma(L)$ with its usual coalgebra structure. This is the homology (in the sense of \cite{PriddyKoszul.pdf}) of the associated graded algebra appearing in the partially restricted PBW theorem, lemma \ref{Partially restricted PBW theorem}. On \cite[p.\ 141]{MayRestLie.pdf}, the typical element of the coalgebra $\Gamma(L)$ is written as 
\[f=\gamma_{r_1}(y_1)\cdots \gamma_{r_m}(y_m),\]
for homogeneous $y_i\in L$. We define $\UEAX(L)$ to be the subcoalgebra given by the restriction that $r_i\leq1$ when $y_m\in L_{\textbf{0}}$. The coalgebra structure map and differential are the restriction to $E(L_{\textbf{0}})\otimes \Gamma(L_{\neq\textbf{0}})$ of those given on \cite[p.\ 141]{MayRestLie.pdf} (after tensoring the formula  \cite[(6.19)]{MayRestLie.pdf} down to a formula on $\overline{X}(L)$, which kills the first term $\Sigma_{i=1}^nf_iy_i$).

Now let $ L=B^{\calL(n)}L\in s\calL(n)$. Using the weak equivalence $\Delta\textbf{B}\to \overline{B}_\bullet(\UEA L)$ of simplicial coalgebras, and  May's injection \cite[Theorem 18 and  (7.8)]{MayRestLie.pdf} of $\UEAX(L)$ into the bar construction, there are maps (up to a zigzag):
\[\UEAX(L)\to N_*\overline{B}_\bullet(U'L)\simeq\textup{Tot}(N^\textup{h}_*N^\textup{v}_*\textbf{B})\overset{r}{\to} \textup{Tot}(N^\textup{h}_*\textbf{Q}_{\bullet*})=\Ftwo \langle 1\rangle\oplus \Sigma N_*Q^{\calL(n)} L.\]
Now the first map is a map of differential coalgebras, so that proposition \ref{the point of the appendix} implies that $\Ftwo \langle 1\rangle\oplus \Sigma H^*_{\calL(n)}L$ may be calculated, as an \emph{algebra}, by the differential graded algebra $\dual(\UEAX(L))$, naturally defined as the quotient $\dual(\UEAX(L))=E(\dual L_{\textbf{0}}) \otimes\Ftwo [\dual L_{\neq\textbf{0}}]$ of $\Ftwo [\dual L]$.

We are interested in the case that $L$ is \emph{trivial as a Lie algebra}, but may still have non-zero restriction. In this case, the restriction is in fact a \emph{linear} map, and we may write $\dualrestn{\DASH}:\dual L\to \dual L$ for its dual (a map which we consider to be everywhere defined, but necessarily equal to zero on $L_{\textbf{0}}$). Examination of  \cite[(6.19)]{MayRestLie.pdf} shows that:
\begin{prop}
If $L$ has bracket zero, then the differential on $\dual(\UEAX(L))$ is defined on $\alpha\in \dual L\subseteq \dual(\UEAX(L))$ by the formula
\[\alpha\longmapsto (\!\sqrt[{[2]}]{\alpha}\,)^2.\]

\end{prop}

\section{\textbf{Unstable Lie coalgebras over the dual $P$-algebra (move me)}}
\todo{Give a formal definition of these coalgebras}{goes down to level of simplicial commutative algebras, and brings in the small object argument comonad structure}
\todo{Give the construction (via equalizer diagram, etc) of the comonad}{}
\todo{Write the maps $j$ and $\gamma_i$ in terms of maps out of $C_{ij}$ and $C_i^j$}{include generating cofibrations such as $S^{i+j}\to C_{ij}$ and $\Delta[1]\times S^{i+j}\to \Delta[1]\times C_{ij}$}
\todo{Construct operations on homology of such coalgebras}{they agree with those in double dual construction when locally finite
\item if not locally finite, by studying gradings, can see that every coalgebra is the union of its finite type (even just `finite') subcoalgebras, and that the cobar construction respects such unions, thus get the same results in infinite case as we did from double dualization.}


\end{appendices}

\begin{todolist}
\section{\textbf{To do/consider:}}
\begin{enumerate}\squishlist
\setlength{\parindent}{.25in}
%\item which side do you index bar constructions from?
%\item Monadicity over vector spaces is important and should be emphasised, especially for the diagonal on the Blanc-Stover resolution.
\item Maybe I should keep finiteness to myself, given that the small object argument might be around, or even the W comonad.
%\item change all $\calw$'s to $\calK$'s? NAH
\item Change the notations $i$ and $\gamma$ for $\Pi$-algebra stuff.
\item Uniformize notation for the total complex of a double complex, and for the cohomotopy of the dual of a simplicial vector space.
\item In bit about quadratic decomposition, say ``I'll just prove well definedness in specific cases.'' ???
%\item Need a bit of theory about splittings of monads, homogeneity
\item unital part of an expression.
\item delete all pagebreaks
\item At present the notation $\gamma$ is repeated all over the place.
\item $\gamma_i$ and $\gamma^i$ notations must go.
\item Need a cogent section on stuff like: $S^2_\textup{pr?}\pi_*M\to \pi_*M$. What is a restricted Lie algebra, what is are the ``top operations'' $S^2\pi_*(L)\to \pi_*(S^2L)\to \pi_*(L)$ - are they defined by $\nabla$? (yes). Use $\widetilde{\nabla}$ for the top ops.
\item Uniform notation for the class represented by $x$... $[x]$? $\overline{x}$?
\item get an arrow for epimorphisms and one for monomorphisms, and use them
%\item vector space, not vectorspace
\item do not confuse ``quadratic grading 2'' with ``quadratic operations''! The quadratic grading 2 part in a $\Pi$-algebra corresponds to the quadratic operations. So far I haven't really made that clear --- actually, this thought is the motivation for the name ``quadratic grading''! Should it be the `arity grading'?
\item Need discourse on that grading on $\calw(n)$.
\item Point out that $(\quadgrad{2}F_{\calU(n+1)})(V)\oplus (\quadgrad{2}F_{\calL(n+1)})(V)=(\quadgrad{2}F_{\calw(n+1)})(V)$, and that the two monoids both include as subs of $F_{\calw(n+1)}$, and one is a quotient monoid. Broadly just need to fully explain all this grading stuff in all of the free constructions.
\item Consider making every symbol $(E_2)^{s_5,\ldots,s_1}_t$ to $(E_2^{s_5,s_4})^{s_3,\ldots,s_1}_t$.
\item Remove all bullets from $B^{\DASH}.$
\item ``unnormalized'' and ``non(-)normalized''... also, introduce the notation $C_*$
\item $\longmapsto$ instead of $\mapsto$
%\item maybe raise the subscripts in $F^\calc$ and in $U^\calc$ and in $K^\calc$ and in $Q_\calc$ (which is done)
\item say what locally finite means and point out that things with a shifted degree are all the colimit of their locally finite subobjects
\item ``$\delta$-algebra'' is not a good name, $\delta$-admissible too!
\item General principle: homotopy operations ($\delta$, $\lambda$) are partially defined, cohomology operations are sometimes zero. The two notions interchange upon taking cohomology.
\item co-Koszul vs coKoszul.
\item ``goerss looked at unstable ops in the reverse adams SSeq, which is how we understand w. I do something for the forwards ass, and the story comes out rather differently.''
\item 
``Goerss' text explains why one might expect these dualities. (After giving the four koszul algebras in the first chapter). Put 2 algs in the example of commutative algs, and then 2 under lie algs.''
\item Install a $\textup{v}$ in each Koszul $\delta$ operation.
\item Am I writing $n>0$ or $n\geq1$? Choose one... $n\geq1$
\item $\Sq$-admissible? Steenrod-Admissible? \textbf{it means ``not containing 0''}
\item search all ``ref''s and check they have the right word, like ``lemma'' before them.
\item search $F_?$, $B_?$, $Q_?$, $U_?$, $\xi_?$, $j_?$, $\psi_?$...?
\item get a better macro for $\epi$
\item Identify and remove certain common phrases, like `Now $x=y$, but $y\neq z$...'
\item Introduce uniform notation for connective chain complexes.
\item once and for all, say that Sq-admiss means $\geq1$ and $\delta,P$ means $\geq2$ and $\Lambda$ means $\geq0$
\item forecast the pr UEA in the section where we discuss lie algebras first.
\item Search lemma, theorem, proposition, etc, and fix it all.
\item mention %SuccessiveSpectralSequences_22draft.pdf and po hu
\item say `edge composite', or similar, not edge hom thm
\item search mapsto, replace with longmapsto, or just redefine
\item subset to be replaced by subseteq
\item say ``sometimes view element of $\vect{r}{+}$ as an element of $\vect{r+1}{+}$ concentrated in degree $s_{r+1}=0$'' and say ``we will use the phrase \emph{degree $s_k=0$}, or \emph{degree $t=0$}''
\item define top and non-top
\item search ``non'', and have uniform convention on dashes \textbf{we use a dash}
\item include references to the Chevalley-Eilenberg-May complex section of the appendix
\item explain the frequent use of the word `trivial'
\item mention that $\Ftwo \{\}$ just means `spanned by', while $\Ftwo []$ means poly alg. WAIT... new notation!
\item in koszul resolutions, often write $x$ instead of $\Sqv(\emptyset,x)$. Maybe want to write $\Sqv^*(\emptyset,x)$ instead, in general.
\item should probably say what the structure  on the tensor product of $E_2$-pages is.
\item most $\delta$ should be $\deltav$
\item $\smash{\widetilde{f}}$ is normally best smashed
\item  \textbf{remove any $[x]$'s}, replace with $\overline{x}$.
\item flesh out the looking glass by comparing unstable homotopy operations and cohomology operations using Koszul duality
\item point out: knowing $\PiAlg$ structure is knowing htpy of spheres
\item probably want to make the big four algebras unital, so that they are easy to tensor with... 
\item reassess every use of the phrase Hilton-Milnor
\item Use cohomotopy of the dual!!!! it is classier.
\item neither (co) nor (co-)!!!
\item clear warnings using ctrl-backslash
\item clarify external and exterior
\item all \textup{v} should be \textup{V}, for size consistency
\item at some point say what a tensor product of chain complexes is
\item two methods of stating indeterminacy are at odds just a little
\item edd $\textup{ext}$ in codwyer chapter
%\item Adams $\rightarrow$ Bousfield-Kan
\item reference \url{http://arxiv.org/pdf/1502.06944v1.pdf}
\item change some $\Lambda(\LieOperad)$ to $\liealgs$
\item Delta subscripts and D superscripts are bad.
\item Remove all apostrophes, as in ``it is''
\item use `homology completion'
\item fix everything like $QK$
\item decorate ``$\textup{Pr}$''
\item finite-type vs finite type
\item $\quadratic_\calc$ is down.
\item cite all the asesome bk papers
\item need uniform convention on the j's
\item $\star$ for $*$-preimage - introduce that idea 
\end{enumerate}
\end{todolist}
\begin{bibliog}
\printbibliography
\end{bibliog}

\begin{Contents Page}
%\begin{abstract}\end{abstract}
%\maketitle
\ifx\PostponeContents\undefined\else\vfil\pagebreak\tiny\tableofcontents\relax\fi
\end{Contents Page}

\end{document}


%Goerss:MR1089001
%Singer:MR2245560
%Priddy:PriddySimplicialLie.pdf or PriddyKoszul.pdf
%Curtis:CurtisSimplicialHtpy.pdf
%6Author.pdf
%MillerSullivanConjecture.pdf
%Blanc_Stover-Groth_SS.pdf
%\cite{FresseSimplicialAlgs.pdf}
%DwyerHtpyOpsSimpComAlg.pdf

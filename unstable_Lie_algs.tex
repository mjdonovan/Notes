% !TEX root = z_output/unstable_Lie_algs.tex
\documentclass[11pt]{amsart}
\usepackage{amsmath,amsthm,amssymb}
\usepackage{eucal}
\usepackage{mathrsfs,nicefrac}
\usepackage{amssymb}
\usepackage[all]{xy}
\usepackage{cancel}
\usepackage{color}
\usepackage{enumerate}
\usepackage{mathtools}%used for mathclap
\usepackage{dsfont}%creates \mathds{1}

\usepackage[bookmarks=false,pdftex,pdfborder={0 0 0 [1 1]}]{hyperref}

\headheight=8pt
\topmargin=0pt
\textheight=610pt
\textwidth=432pt
\oddsidemargin=18pt
\evensidemargin=18pt
\footskip=25pt


%>>>>>>>>>>>>>>>>>>>>>>>>>>>>>>
%<<<  Theorem Environments  <<<
%>>>>>>>>>>>>>>>>>>>>>>>>>>>>>>
\theoremstyle{plain}
\newtheorem{thm}{Theorem}[section]
\newtheorem*{thm*}{Theorem}
\newtheorem{lem}[thm]{Lemma}
\newtheorem*{lem*}{Lemma}
\newtheorem{prop}[thm]{Proposition}
\newtheorem*{prop*}{Proposition}
\newtheorem{cor}[thm]{Corollary}
\newtheorem*{cor*}{Corollary}
\newtheorem{defprop}[thm]{Definition-Proposition}
\newtheorem*{punchline}{Punchline}
\newtheorem{conjecture}{Conjecture}
\newtheorem*{conjecture*}{Conjecture}
\newtheorem*{claim}{Claim}

\theoremstyle{definition}
\newtheorem{defn}{Definition}[section]
\newtheorem*{defn*}{Definition}


%>>>>>>>>>>>>>>>>>>>>>>>>>>>>>>
%<<<       Operators        <<<
%>>>>>>>>>>>>>>>>>>>>>>>>>>>>>>
\DeclareMathOperator{\ad}{\textbf{ad}}
\DeclareMathOperator{\coker}{coker}
\renewcommand{\ker}{\textup{ker}\,}
\DeclareMathOperator{\End}{End}
\DeclareMathOperator{\Aut}{Aut}
\DeclareMathOperator{\Hom}{Hom}
\DeclareMathOperator{\Maps}{Maps}
\DeclareMathOperator{\Mor}{Mor}
\DeclareMathOperator{\Gal}{Gal}
\DeclareMathOperator{\Ext}{Ext}
\DeclareMathOperator{\Tor}{Tor}
\DeclareMathOperator{\Cotor}{Cotor}
\DeclareMathOperator{\Prim}{Prim}
\DeclareMathOperator{\Tot}{Tot}
\DeclareMathOperator{\Map}{Map}
\DeclareMathOperator{\Der}{Der}
\DeclareMathOperator{\Rad}{Rad}
\DeclareMathOperator{\rank}{rank}
\DeclareMathOperator{\ArfInvariant}{Arf}
\DeclareMathOperator{\KervaireInvariant}{Ker}
\DeclareMathOperator{\im}{im}
\DeclareMathOperator{\coim}{coim}
\DeclareMathOperator{\trace}{tr}
\DeclareMathOperator{\supp}{supp}
\DeclareMathOperator{\ann}{ann}
\DeclareMathOperator{\spec}{Spec}
\DeclareMathOperator{\SPEC}{\textbf{Spec}}
\DeclareMathOperator{\proj}{Proj}
\DeclareMathOperator{\PROJ}{\textbf{Proj}}
\DeclareMathOperator{\fiber}{fib}
\DeclareMathOperator{\cofiber}{cof}
\DeclareMathOperator{\cone}{cone}
\DeclareMathOperator{\skel}{sk}
\DeclareMathOperator{\coskel}{cosk}
\DeclareMathOperator{\conn}{conn}
\DeclareMathOperator*{\colim}{colim}
\DeclareMathOperator*{\limit}{lim}
\DeclareMathOperator*{\hocolim}{hocolim}
\DeclareMathOperator*{\holimit}{holim}
\DeclareMathOperator*{\holim}{holim}
\DeclareMathOperator*{\hofib}{hofib}
\DeclareMathOperator*{\hotfib}{thofib}
\DeclareMathOperator*{\equaliser}{eq}
\DeclareMathOperator*{\coequaliser}{coeq}
\DeclareMathOperator{\ch}{ch}
\DeclareMathOperator{\Thom}{Th}
\DeclareMathOperator{\GrthGrp}{GrthGp}
\DeclareMathOperator{\Sym}{Sym}
\DeclareMathOperator{\Prob}{\mathbb{P}}
\DeclareMathOperator{\Exp}{\mathbb{E}}
\DeclareMathOperator{\GeomMean}{\mathbb{G}}
\DeclareMathOperator{\Var}{Var}
\DeclareMathOperator{\Cov}{Cov}
\DeclareMathOperator{\Sp}{Sp}
\DeclareMathOperator{\Seq}{Seq}
\DeclareMathOperator{\Cyl}{Cyl}
\DeclareMathOperator{\Ev}{Ev}
\DeclareMathOperator{\sh}{sh}
\DeclareMathOperator{\intHom}{\underline{Hom}}
\DeclareMathOperator{\Frac}{frac}


%>>>>>>>>>>>>>>>>>>>>>>>>>>>>>>
%<<<  Mathematical Symbols  <<<
%>>>>>>>>>>>>>>>>>>>>>>>>>>>>>>
\newcommand{\DASH}{\textup{--}}

%>>>>>>>>>>>>>>>>>>>>>>>>>>>>>>
%<<<     Greek Letters      <<<
%>>>>>>>>>>>>>>>>>>>>>>>>>>>>>>
\let\oldphi\phi
\let\phi\varphi
\renewcommand{\to}{\longrightarrow}
\newcommand{\from}{\longleftarrow}
\newcommand{\eps}{\varepsilon}

%%>>>>>>>>>>>>>>>>>>>>>>>>>>>>>>
%%<<<     Environments       <<<
%%>>>>>>>>>>>>>>>>>>>>>>>>>>>>>>
\newcommand{\squishlist}{
  \setlength{\itemsep}{.5pt}
  \setlength{\parskip}{0pt}
  \setlength{\parsep}{0pt}}
%>>>>>>>>>>>>>>>>>>>>>>>>>>>>>>
%<<<     Script Letters     <<<
%>>>>>>>>>>>>>>>>>>>>>>>>>>>>>>
\newcommand{\scrQ}{\mathscr{Q}}
\newcommand{\scrW}{\mathscr{W}}
\newcommand{\scrE}{\mathscr{E}}
\newcommand{\scrR}{\mathscr{R}}
\newcommand{\scrT}{\mathscr{T}}
\newcommand{\scrY}{\mathscr{Y}}
\newcommand{\scrU}{\mathscr{U}}
\newcommand{\scrI}{\mathscr{I}}
\newcommand{\scrO}{\mathscr{O}}
\newcommand{\scrP}{\mathscr{P}}
\newcommand{\scrA}{\mathscr{A}}
\newcommand{\scrS}{\mathscr{S}}
\newcommand{\scrD}{\mathscr{D}}
\newcommand{\scrF}{\mathscr{F}}
\newcommand{\scrG}{\mathscr{G}}
\newcommand{\scrH}{\mathscr{H}}
\newcommand{\scrJ}{\mathscr{J}}
\newcommand{\scrK}{\mathscr{K}}
\newcommand{\scrL}{\mathscr{L}}
\newcommand{\scrZ}{\mathscr{Z}}
\newcommand{\scrX}{\mathscr{X}}
\newcommand{\scrC}{\mathscr{C}}
\newcommand{\scrV}{\mathscr{V}}
\newcommand{\scrB}{\mathscr{B}}
\newcommand{\scrN}{\mathscr{N}}
\newcommand{\scrM}{\mathscr{M}}


\newcommand{\frakq}{\mathfrak{q}}
\newcommand{\frakw}{\mathfrak{w}}
\newcommand{\frake}{\mathfrak{e}}
\newcommand{\frakr}{\mathfrak{r}}
\newcommand{\frakt}{\mathfrak{t}}
\newcommand{\fraky}{\mathfrak{y}}
\newcommand{\fraku}{\mathfrak{u}}
\newcommand{\fraki}{\mathfrak{i}}
\newcommand{\frako}{\mathfrak{o}}
\newcommand{\frakp}{\mathfrak{p}}
\newcommand{\fraka}{\mathfrak{a}}
\newcommand{\fraks}{\mathfrak{s}}
\newcommand{\frakd}{\mathfrak{d}}
\newcommand{\frakf}{\mathfrak{f}}
\newcommand{\frakg}{\mathfrak{g}}
\newcommand{\frakh}{\mathfrak{h}}
\newcommand{\frakj}{\mathfrak{j}}
\newcommand{\frakk}{\mathfrak{k}}
\newcommand{\frakl}{\mathfrak{l}}
\newcommand{\frakz}{\mathfrak{z}}
\newcommand{\frakx}{\mathfrak{x}}
\newcommand{\frakc}{\mathfrak{c}}
\newcommand{\frakv}{\mathfrak{v}}
\newcommand{\frakb}{\mathfrak{b}}
\newcommand{\frakn}{\mathfrak{n}}
\newcommand{\frakm}{\mathfrak{m}}

%>>>>>>>>>>>>>>>>>>>>>>>>>>>>>>
%<<<  Caligraphic Letters   <<<
%>>>>>>>>>>>>>>>>>>>>>>>>>>>>>>
\newcommand{\calQ}{\mathcal{Q}}
\newcommand{\calW}{\mathcal{W}}
\newcommand{\calE}{\mathcal{E}}
\newcommand{\calR}{\mathcal{R}}
\newcommand{\calT}{\mathcal{T}}
\newcommand{\calY}{\mathcal{Y}}
\newcommand{\calU}{\mathcal{U}}
\newcommand{\calI}{\mathcal{I}}
\newcommand{\calO}{\mathcal{O}}
\newcommand{\calP}{\mathcal{P}}
\newcommand{\calA}{\mathcal{A}}
\newcommand{\calS}{\mathcal{S}}
\newcommand{\calD}{\mathcal{D}}
\newcommand{\calF}{\mathcal{F}}
\newcommand{\calG}{\mathcal{G}}
\newcommand{\calH}{\mathcal{H}}
\newcommand{\calJ}{\mathcal{J}}
\newcommand{\calK}{\mathcal{K}}
\newcommand{\calL}{\mathcal{L}}
\newcommand{\calZ}{\mathcal{Z}}
\newcommand{\calX}{\mathcal{X}}
\newcommand{\calC}{\mathcal{C}}
\newcommand{\calV}{\mathcal{V}}
\newcommand{\calB}{\mathcal{B}}
\newcommand{\calN}{\mathcal{N}}
\newcommand{\calM}{\mathcal{M}}

\newcommand{\calq}{\mathcal{Q}}
\newcommand{\calw}{\mathcal{W}}
\newcommand{\cale}{\mathcal{E}}
\newcommand{\calr}{\mathcal{R}}
\newcommand{\calt}{\mathcal{T}}
\newcommand{\caly}{\mathcal{Y}}
\newcommand{\calu}{\mathcal{U}}
\newcommand{\cali}{\mathcal{I}}
\newcommand{\calo}{\mathcal{O}}
\newcommand{\calp}{\mathcal{P}}
\newcommand{\cala}{\mathcal{A}}
\newcommand{\cals}{\mathcal{S}}
\newcommand{\cald}{\mathcal{D}}
\newcommand{\calf}{\mathcal{F}}
\newcommand{\calg}{\mathcal{G}}
\newcommand{\calh}{\mathcal{H}}
\newcommand{\calj}{\mathcal{J}}
\newcommand{\calk}{\mathcal{K}}
\newcommand{\call}{\mathcal{L}}
\newcommand{\calz}{\mathcal{Z}}
\newcommand{\calx}{\mathcal{X}}
\newcommand{\calc}{\mathcal{C}}
\newcommand{\calv}{\mathcal{V}}
\newcommand{\calb}{\mathcal{B}}
\newcommand{\caln}{\mathcal{N}}
\newcommand{\calm}{\mathcal{M}}

\newcommand{\twist}{\sigma}

\usepackage{framed}
\definecolor{shadecolor}{rgb}{.925,0.925,0.925}
\usepackage[style=numeric,%citestyle=numeric,
url=false,doi=false,isbn=false,eprint=false]{biblatex}%
\hypersetup{colorlinks=false,pdfborder={0 0 0}}



\makeatletter
\renewcommand{\@seccntformat}[1]{\csname the#1\endcsname.\quad}
\makeatother


\theoremstyle{plain}
\newtheorem{theorem}{Theorem}
\newtheorem*{completenesstheorem}{Completeness Theorem}
\newtheorem{twistinglemma}[thm]{Twisting Lemma}
\renewcommand*{\thetheorem}{\Alph{theorem}}
\newtheorem{conjectureAlpha}{Conjecture}
\renewcommand*{\theconjectureAlpha}{\Alph{conjectureAlpha}}

\newcommand{\pgr}{\mathds{1}}
\renewcommand{\pgr}{\textup{+}}
\newcommand{\PMonad}{{\calP^\textup{u}}}
\newcommand{\LambdaMonad}{\Lambda^\textup{u}}
\newcommand{\Palg}{{\calP}}
\newcommand{\LieOperad}{{\scrL}}
\newcommand{\restn}[1]{{#1}^{[2]}}
\newcommand{\vect}[2]{\calV^{#1}_{#2}}
\newcommand{\BSW}{{\scrG}}%\scrH\scrI\scrJ\scrK\scrL\scrM\scrN\scrO\scrP\scrQ\scrR\scrS\scrT\scrY\scrZ}}

\newcommand{\NOFULLPAGE}{\relax}

\newcommand{\Sq}{\mathrm{Sq}}
\newcommand{\Comm}{\calC}
\newcommand{\F}{\mathbb{F}}

%\bibliography{papers}
\bibliography{../../Dropbox/logbook/_LOGBOOK/papers}




\title[Unstable Lie algebras and their cohomology]{Unstable Lie algebras and their cohomology}
%\author[M.\ Donovan]{Michael Donovan}

%\address{Department of Mathematics \\ Massachusetts Institute of Technology}
%\email{mdono@math.mit.edu}



\newcommand{\dupdown}[2]{D_{\smash{#1}}}
\newcommand{\caldup}[1]{\calD_{\smash{#1}}}
\newcommand{\caldupdown}[2]{\calD^{\smash{#1}}_{\smash{#2}}}

\begin{document}
\newcommand{\todo}[2]{\begin{shaded}\begin{itemize}
\setlength{\parindent}{.25in}
\item[{\Large$\smash\diamondsuit$}] #1
\ifblank{#2}{}{\tiny\begin{itemize}
\setlength{\parindent}{.25in}
\item #2
\end{itemize}}
\end{itemize}\end{shaded}
}
\newcommand{\todoeasy}[2]{\begin{shaded}\begin{itemize}
\setlength{\parindent}{.25in}
\item[{\Large$\smash\spadesuit$}] #1
\ifblank{#2}{}{\tiny\begin{itemize}
\setlength{\parindent}{.25in}
\item #2
\end{itemize}}
\end{itemize}\end{shaded}
}
%\begin{abstract}\end{abstract}
%\maketitle
%\tableofcontents

\section{Conventions and notation}
\subsection{Categories of universal graded algebras}
In this paper we will be dealing with various categories of universal graded algebras [BS?,Quillen], and as such, it serves to set notation globally. The categories we will deal with are all monadic over a category of graded $\F_2$-vector spaces (of finite type ??),
%Indeed, for any two integers $m,n\geq0$, let $\vect{m}{n}$ be the category of finite type graded $\F_2$-vector spaces with $m$ cohomological gradings and $n$ homological gradings. That is, an object $V$ of $\vect{m}{n}$ may be written
%\[V=\bigoplus_{p_m,\ldots,p_1,q_n,\ldots,q_1\geq0}V_{q_n,\ldots,q_1}^{p_m,\ldots,p_1},\]
%in which each $V_{q_n,\ldots,q_1}^{p_m,\ldots,p_1}$ is a finite dimensional $\F_2$-vector space. In fact, we will restrict to 
namely, the category of vector spaces with $r$ non-negative homological gradings and a single strictly positive cohomological grading. We will denote the category of such $\F_2$-vector spaces by the symbol $\vect{+}{r}$, so that an object $V$ of $\vect{+}{r}$ decomposes as
\[V=\bigoplus_{s_r,\ldots,s_1\geq0,\,t\geq 1}V^{t}_{s_r,\ldots,s_1}.\]
Similarly, we write $\vect{r}{+}$ for the category of $\F_2$-vector spaces with one positive homological grading and $r$ non-negative cohomological gradings. Lastly, we write $\vect{}{}$ for the category of (ungraded) $\F_2$-vector spaces, and  $\vect{+}{}$ for $\vect{+}{0}$.

For each algebraic category $\calC$ of interest in this paper, an object $X$ of $\calC$ may be defined as a graded vector space $X\in\vect{+}{r}$ along with certain binary and unary operations on $X$ (and without any nullary operations). As such, there will be a forgetful functor $U_\calC:\calC\to\vect{+}{r}$, which will admit a left adjoint, always denoted $F_\calC:\vect{+}{r}\to\calC$ for notational consistency. The functor $U_{\calC}$ will always be monadic, in the sense that the natural comparison functor from $\calC$ to the category of algebras over the monad $U_{\calC}F_{\calC}$ on $\vect{+}{r}$ is an equivalence.

It will also happen that for each $\calC$, there is a functor $K_\calC:\vect{+}{r}\to \calC$ which sends $V\in\vect{+}{r}$ to the `trivial object on $V$', and $K_{\calC}$ has a left adjoint, $Q_{\calC}:\calC\to\vect{+}{r}$, which sends $X\in\calC$ to `the quotient of $X$ by the image of its non-trivial operations'. In terms of free constructions, the monad $U_{\calC}F_{\calC}$ will admit an augmentation $\epsilon:U_{\calC}F_{\calC}\to\textup{id}$, and the functor $K_\calC$ sends any object $V\in\vect{+}{r}$ to the $U_{\calC}F_{\calC}$-algebra whose structure map is $\epsilon_V:U_{\calC}F_{\calC}V\to V$. Moreover, the functor $Q_{\calC}$ sends $X\in \calC$ with structure map $\rho:F_\calC U_\calC X\to X$ to the coequaliser in $\vect{+}{r}$ of the maps $U_\calC\rho$ and $\epsilon_{U_\calC X}$.

\subsection{Quillen's model structure on $s\calC$ and $\calC$-homology}
For any of the algebraic categories $\calC$ herein we use Quillen's simplicial model category structure on the category $s\calC$. In this structure, the weak equivalences (fibrations) are the maps which are weak equivalences (fibrations) of simplicial abelian groups (so that every object is fibrant), and every cofibrant object is a retract of an \emph{almost free} object. A simplicial object $X_\bullet$ is almost free if there are subspaces $V_n\subseteq X_n$ for each $n\geq0$ such that the structure map $F_{\calC}U_\calC V_n\to X_n$ is an isomorphism for all $n$, and such that these subspaces are preserved by all of the degeneracies and face maps of $X_\bullet$ except for $d_0$.

For a simplicial resolution functor $\calC\to s\calC$, we'll often use the standard simplicial bar construction arising from the $F_\calC\dashv U_\calC$ adjunction [cite?]. More explicitly, given $X\in\calC$, we define $B_\bullet^\calC X\in s\calC$ by iteration of the comonad $F_\calC U_\calC$, although we will tend to supress the $U_\calC$ in our notation:
\[B_s^\calC X=(F_\calC)^{s+1}X.\]
The face maps are given by the formula $d_i=(F_\calC)^i\epsilon$, and the degeneracies by $s_i=(F_\calC)^i\beta$, where $\epsilon$ and $\beta$ are respectively the counit and diagonal of the comonad $F_\calC U_\calC$. This object is almost free, with $B_s^\calC X$ generated by its subspace $(F_{\calC})^sX$. It is standard that the augmentation $B_\bullet^\calC X\to X$ is an acyclic fibration.

For $X\in\calC$, we define the $\calC$-\emph{homology} of $X$ by the formula:
\[H_*^{\calC}X=\pi_*(Q_\calC B_\bullet^\calC X)=H_*N_*(Q_\calC B_\bullet^\calC X).\]
This is well defined, as the $Q_\calC\dashv K_\calC$ adjunction is a Quillen adjunction (indeed $K_\calC$ evidently preserves fibrations and acyclic fibrations), and we are free to use any cofibrant replacement in place of $B_\bullet^\calC X$. We define the $\calC$-cohomology $H^*_\calC X$ to be the linear dual of $H_*^\calC X$. Note that if $\calC$ is monadic over $\vect{+}{r}$, then $H_*^{\calC}X\in \vect{+}{r+1}$, and $H^*_{\calC}X\in \vect{r+1}{+}$. That is, each homology group $H_s^\calC X$ retains the gradings of $X$, and a new homological grading is added (to the left of the existing homological gradings). If we dualise to obtain cohomology, the cohomological gradings and homological gradings are swapped. We will uniformly avoid substituing into the location of an upper or lower $*$, for the sake of clarity. For example, suppose that $\calC$ is monadic over $\vect{+}{2}$, then we will write $(H_*^\calC X)_{s_3,s_2,s_1}^t$, instead of $(H_{s_3}^\calC X)_{s_2,s_1}^t$


\subsection{The Blanc-Stover comonad}
Fix an algebraic category $\calC$, monadic over $\vect{}{}$ (more generally, monadic over some category of graded vector spaces, but for now we opt for notational simplicity). As we are working over a category of vector spaces, we can refine Blanc and Stover's definition \cite{Blanc_Stover-Groth_SS.pdf} of a useful comonad `$W$' on $s\calC$. (Did I end up changing the character for this?)

It will help to fix a little notation. First, for $n\geq0$, let $\mathbb{S}^n\in \textup{ch}_+ \vect{}{}$ and $C\mathbb{S}^n\in \textup{ch}_+ \vect{}{}$ be the chain complexes
\[(\mathbb{S}^n)_j=\begin{cases}
\F_2,&\textup{if }j=n;\\
0,&\textup{otherwise},
\end{cases}\qquad 
(C\mathbb{S}^n)_j=\begin{cases}
\F_2,&\textup{if }j=n\textup{ or }n+1;\\
0,&\textup{otherwise},
\end{cases}
\]
with $d:(C\mathbb{S}^n)_{n+1}\to (C\mathbb{S}^n)_{n}$ the identity of $\F_2$. There is an evident inclusion $\imath:\mathbb{S}\to C\mathbb{S}$. For any $V\in\textup{ch}_+\vect{}{}$, we have
\[\hom_{\textup{ch}_+\vect{}{}}(\mathbb{S}^n,V)\cong ZN_nV,\textup{ and }\hom_{\textup{ch}_+\vect{}{}}(C\mathbb{S}^n,V)\cong N_{n+1}V,\]
and the differential $d:N_{n+1}V\to ZN_nV$ corresponds to $\imath^*$ under these isomorphisms. If $N:s\vect{}{}\rightleftarrows \textup{ch}_+ \vect{}{}
:\Gamma$ are the inverse equivalences of the Dold-Kan correspondence, define
\[S^n_{\calC}:=F_{\calC}\Gamma(\mathbb{S}^n)\textup{ and }CS^n_{\calC}:=F_{\calC}\Gamma(C\mathbb{S}^n).\]
There is still a map $\imath:S^n_{\calC}\to CS^n_{\calC}$, and $\imath^*$ still represents the differential under the isomorphisms
\[\hom_{s\calC}(S_\calC^n,X)\cong ZN_nX,\textup{ and }\hom_{s\calC}(CS_\calC^n,X)\cong N_{n+1}X.\]

In this context, Blanc-Stover's comonad on $s\calC$ may be written as the pushout
\[\xymatrix@R=4mm{
\bigsqcup_{n\geq0}\bigsqcup_{h\in N_{n+1}X}S_\calC^n
\ar[r]\ar[d]^-{\bigsqcup\bigsqcup\imath}
&%r1c1
\bigsqcup_{n\geq0}\bigsqcup_{f\in ZN_{n}X}S_\calC^n\ar[d]\\%r1c2
\bigsqcup_{n\geq0}\bigsqcup_{h\in N_{n+1}X}CS_\calC^n
\ar[r]&%r2c1
\BSW X%r2c2
}\]
where under the top horizontal map, the copy of $S^n_\calC$ corresponding to $h$ maps identically onto the copy of $S^n_\calC$ corresponding to $dh=\imath^*h$. 

Say `this is a monad' at this point. Moreover, $\BSW X$ is homotopy equivalent to a coproduct of spheres. Indeed, for each $n$, let $B_nX=\textup{Im}(d:N_{n+1}X\to ZN_nX)$, and choose a section $h_0$ of the surjection $d:N_{n+1}X\to B_nX$. Then $\BSW X$ contains the contractible subobject $C_0:=\bigsqcup_n\bigsqcup_{h\in \textup{Im}(h_0)} CS^n_{h_0(f)}$, and $\BSW X/C_0$ is a wedge of spheres:
\[\BSW X/C_0\cong \left(\bigsqcup_{h\in N_{n+1}X\setminus\textup{Im}(h_0)}CS^n_h/\partial(CS^n_h)\right) \sqcup\left(\bigsqcup_{f\in ZN_nX\setminus B_nX}S^n_f\right)\]
We would like to find a subspace of $\pi_*(\BSW X)$ which freely generates it as an element of $\calW(n+1)$. Even better, we have:
\begin{prop}
For $X\in\calC$, $\pi_*(B_\bullet^{\BSW}X)$ is an almost free $\calC\textup{-$\Pi$-algebra}$ weakly equivalent to $\pi_*(X)$.
\end{prop}
For this purpose, observe that there's a natural map $\F_2[d]:\F_2[N_{n+1}X]\to \F_2[ZN_nX]$, and a natural monomorphism $\alpha:\ker(\F_2[d])\to\pi_*(\BSW X)$. The group $\ker(\F_2[d])$ is generated by differences $z-z'$ for $z,z'\in N_{n+1}X$ having $dz=dz'$, and we define $\alpha(z-z')$ to be the homotopy class of the difference of the cones corresponding to $z$ and $z'$, whose boundaries have been identified in the colimit. Moreover, there's a natural map $\beta:\F_2[ZN_nX]\to\pi_*(\BSW X)$ (which is not monomorphic), which sends a generator $z\in ZN_nX$ to the homotopy class in $\BSW X$ of the corresponding summand in the top right corner of the colimit diagram. From the above expression for $\BSW X/C_0$, one sees that $\textup{Im}(\alpha)$ and $\textup{Im}(\beta)$ are linearly independent subspaces of $\pi_*(\BSW X)$, and $\pi_*(\BSW X)$ is free on $\textup{Im}(\alpha)\oplus\textup{Im}(\beta)$. From this description it is clear that the generating subspaces are preserved by maps in the image of $\BSW $.

Moreover, the diagonal of the comonad also preserves the subspaces $\textup{Im}(\alpha)$ and $\textup{Im}(\beta)$. That $\textup{Im}(\alpha)$ is preserved is evident from the definitions. For $\textup{Im}(\beta)$, note that $\textup{Im}(\beta)\subset\pi_*(\BSW X)$ is spanned by terms $D_h-D_{h'}$ for $h,h':CS\to X$ satisfying $h\imath=h'\imath$. The diagonal applied to $D_h-D_{h'}$ may be written $D_{D_h}-D_{D_{h'}}$, and one notes that $D_h\imath=S_{h\imath}=S_{h'\imath}=D_{h'}\imath$.





\textbf{Move this para: }Blanc and Stover explain that $\BSW $ actually has the structure of a comonad on $s\calL(n)$. As such, for any simplicial Lie algebra $L_\bullet\in s\calL(n)$, there is a bisimplicial object $B_\bullet^\BSW L_\bullet$, the bar construction using the comonad $\BSW $, given by $B_{s_2}^\BSW L_{s_1}=(\BSW^{s_2+1}L_\bullet)_{s_1}$. Flesh out the point here...


Now the utility of the comonad structure lies in resolving an object $L_\bullet\in s\calL(n)$ by the comonadic bar construction, $B^\BSW_\bullet(L_\bullet)$. This discussion shows that, except for $d_0$, all the maps in $B_\bullet^\BSW L_\bullet$ preserve the subspaces of generators of $\pi_*(B_\bullet^\BSW L_\bullet)$, which is to say that this object of $s\calW(n+1)$ is almost free. Moreover, according to [Stover 2.6], the augmentation map $\pi_*(B_\bullet^\BSW L_\bullet)\to \pi_*(L)$ is a weak equivalence in $s\calW(n+1)$. Thus, this Blanc-Stover $\BSW $-construction provides an almost-free replacement of $\pi_*(L)$ in $s\calW(n+1)$.



\subsection{Lie algebras in characteristic 2}
Let $\LieOperad$ be the Lie operad in characteristic 2. As explained in [Fresse], the monad $S(\LieOperad)$ on $\vect{}{}$ defined by
\[S(\LieOperad)(V)=\bigoplus_{n\geq1}(\LieOperad(n)\otimes V^{\otimes n})_{\Sigma_n}\]
does not return the free Lie algebra on $V$ in the traditional sense. Rather, an $S(\LieOperad)$-algebra is a vector space $L$ equipped with a bracket $L\otimes L\to L$ satisfying the Jacobi identity and the antisymmetry condition $[x,y]=-[y,x]$. This condition does \emph{not} imply the alternating condition $[x,x]=0$ (which is impossible to encode operadically). We will refer to structures of this type as $S(\scrL)$-algebras, to distinguish them from Lie algebras, which we demand satisfy the alternating condition.

Fresse constructs a monad $\Gamma(\LieOperad)$ defined by
\[\Gamma(\LieOperad)(V)=\bigoplus_{n\geq1}(\LieOperad(n)\otimes V^{\otimes n})^{\Sigma_n},\]
which represents the free restricted [See Curtis or 6A or something] Lie algebra. He further constructs a norm map $\textup{Tr}:S(\LieOperad)\to \Gamma(\LieOperad)$, and defines a monad $\Lambda(\LieOperad)$ by the formula
\[\Lambda(\LieOperad)V=\textup{Im}(\textup{Tr}:S(\LieOperad)V\to \Gamma(\LieOperad)V).\]
Then $\Lambda(\LieOperad)(V)$ is the free Lie algebra on $V$, subject to the standard axiom, $[x,x]=0$.

We'll be interested in \emph{partially restricted Lie algebras}. That is, Lie algebras $L$ equipped with a decomposition $L=L_0\oplus L_1$ such that $L_1$ is a Lie ideal and a map $\restn{(\DASH)}:L_1\to L_1$ satisfying the axioms one would expect for a restriction map. Namely, for any $a,b\in L_1$ and $c\in L$, $\restn{(a+b)}=\restn{a}+\restn{b}+[a,b]$, and $[\restn{a},c]=[a,[a,c]]$. There is a free functor from $\vect{}{}\times \vect{}{}$ to partially restricted Lie algebras, sending $(V_0,V_1)$ to the algebra generated by restrictable elements $V_1$ and non-restrictable elements $V_0$. This free functor may be described as follows. There is a natural $\Sigma_n$-equivariant decomposition: %The expressions for $S(\LieOperad)(V_0\oplus V_1)$ and $\Gamma(\LieOperad)(V_0\oplus V_1)$ both contain the terms:
\[\LieOperad(n)\otimes (V_0\oplus V_1)^{\otimes n}=\Bigl(\LieOperad(n)\otimes (V_0)^{\otimes n}\Bigr)\oplus\Bigl(\bigoplus(\LieOperad(n)\otimes V_{i_1}\otimes V_{i_2}\otimes\cdots\otimes V_{i_n})\Bigr)\]
where the direct sum is taken over all sequences $(i_1,\ldots,i_n)\in\{0,1\}^n$ other than $(0,\ldots,0)$.
%and this decomposition is of $\Sigma_n$-modules.
The free partially restricted Lie algebra is the subspace of $\Gamma(\LieOperad(V_0\oplus V_1))$ given by
\[\textup{Im}\Bigl(S(\LieOperad)(V_0)\overset{\textup{Tr}}{\to}\Gamma(\LieOperad)(V_0\oplus V_1)\Bigr)\oplus\bigoplus_{n\geq1}\Bigl(\bigoplus(\LieOperad(n)\otimes V_{i_1}\otimes V_{i_2}\otimes\cdots\otimes V_{i_n})\Bigr)\]



\section{Unstable Lie algebras over Goerss' $P$-algebra}

\subsection{Goerss' algebraic category $\calW$}
\todoeasy{Recall definition of Goerss' category $\calW$}{Define them to be locally finite and connected.}
We're interested in blah blah blah. Restrict to a subcategory, those which are connected, locally of finite type...


\subsection{Construction of the monad $U$}
\todoeasy{Construct monad $U$ for $\calW$}{Write in terms of monads $\PMonad$ and $S(\scrL)$.\item Decide whether to give the distributive law.
\item give the $\vect{m}{n}$ notation here: finite type with various gradings.
\item Write $\vect{m}{n}$ for the category of loc fin vecspaces with $m$ upper, $n$ lower indices
\item Use blackboard bold symbol $\pgr$ to denote only strictly positive gradings
}
Write $\Palg$ for Goerss' algebra, generated by $P^2,P^3,\ldots,$. This can be viewed as a graded associative algebra in $\vect{1}{0}$, by thinking of the generator $P^i$ as having grading $i+1$ (or even in $\vect{1}{1}$, taking $P^i\in(\Palg)^{i+1}_1$). For an object $V\in\vect{+}{0}$, the tensor product $\Palg\otimes V$ is an object of $\vect{+}{0}$ in the evident way. Write $\PMonad$ for the monad for unstable modules (in which everything is defined, but you set various $P$ to zero).

Next, the Lie operad $\LieOperad$ can be $\vect{1}{0}$-graded by viewing the elements of $\scrL(n)$ as having grading $n-1$. Write $S(\LieOperad)$ for the free $S(\scrL)$-algebra monad (\textbf{already mentioned this above} following FresseSimplicialAlgs.pdf and his posets paper, 1.2.16), and $\Lambda(\LieOperad)$ for the good Lie algebra monad. Both of these monads are now monads on $\scrV(1,0)$.

In order to construct the monad $U$ of [Goerss], we'll coequalize two maps:
\[\xymatrix@R=4mm{
\Palg\otimes S(\LieOperad)V\ar[r]^-{\textup{id}\otimes \textup{frob}}
&%r1c1
\Palg\otimes S(\LieOperad)V\ar@{->>}[r]&%r1c2
\PMonad(S(\LieOperad)V)%r1c3
}\]
and the map $\Palg\otimes S(\LieOperad)V\to\PMonad (S(\LieOperad)V)$ whose restriction to each subset $\Palg\otimes (S(\LieOperad)V)^r$ is the composite
\[\xymatrix@R=4mm{
\Palg\otimes (S(\LieOperad)V)^r\ar[r]^-{\textup{mult}_{P^r}\otimes\textup{id}}
&
\Palg\otimes (S(\LieOperad)V)^r\ar@{^{(}->}[r]
&%r1c1
\Palg\otimes S(\LieOperad)V\ar@{->>}[r]&%r1c2
\PMonad(S(\LieOperad)V)%r1c3
}\]
One can check, using Goerss' description of the monad $U$, that it can be described by the coequaliser in $\vect{+}{0}$:
\[\xymatrix@R=4mm{
\Palg\otimes S(\LieOperad)V\ar@<.5ex>[r]\ar@<-.5ex>[r]
&%r1c1
\PMonad(S(\LieOperad)V)\ar@{->>}[r]
&%r1c2
UV%r1c3
}\]
The functor $U$ becomes a monad using the commutation relations given by Goerss, so that all of $U,\PMonad,S(\LieOperad),\Lambda(\LieOperad)$ are monads on graded vector spaces $\vect{+}{0}$. Note further that there is a natural isomorphism of functors $Q_P\circ U=\Lambda(\LieOperad)$. Finally, we note that the monad $U$ posesses a weight grading, analogous to that on the $\delta$-operations used also by [Goerss], the quadratic grading.



%\subsection{Construction of Operations on Andre-Quillen homology}
\todo{Explain why $H^*_{\calC}$ is in $\calW$ (optional)}{Give $j$ map to construct everything, along with the relevant AW/EZ.
\item The the AW/EZ in a way chosen to be consistent with what will be done in the Adams operations paper. Look in adamsops.tex}
%We'll briefly recall why 



\todoeasy{State Goerss' theorem in these terms}{talk about cohomology operations for $\calC$ and Yoneda's lemma}
\subsection{Cohomology for objects of $\calW$}
\todoeasy{Define $Q^\calW:\calW\to \vect{+}{0}$, and thus the cohomology functor on $\calW$}{Give the bar construction $B_\bullet^{\calW}$
\item Explain significance as Adams $E^2$
\item Explain the model structure with almost free maps}
We will use the standard simplicial bar construction as our simplicial resolution of $X\in\calW$, writing $B^\calW_sX=U^{s+1}X$. Then the cohomology of $X$, $H^{s}_t(X)\in\vect{1}{+}$, is the cohomology of the cochain complex
\[C^{s}_{t}=\Hom(N_s(Q_\calW B_{\bullet}X)^{t},\F_2).\]
We'll frequently use the identification $Q_\calW B_{s}X=U^{s}X$.

\section{Cohomology operations for unstable Lie algebras over $P$}
\subsection{$\delta$ operations on $H^*_{\calW}$}
\todoeasy{Define the operations $\gamma_i$, state and define the $\delta_i$ operations}{still need to clarify the end of the proof of the relations.}
For $V^{*}\in \vect{+}{0}$, we will define natural homomorphisms
\[\gamma^i:(UV)^{n+i+1}\to V^{n}, \textup{ for $2\leq i<n$}.\]
Indeed, there are natural maps
\begin{align*}
P^i:V^{n}&\to UV^{n+i+1,(2)}, \textup{\quad  for $2\leq i<n$}\\
[\,,]:V^{n_1}\otimes V^{n_2}&\to UV^{n_1+n_2+1,(2)},
\end{align*}
and for given $m\geq1$, the quadratic grading 2 part of $UV^m$ decomposes as
%\[UV^{m,(2)}=\left(\bigoplus_{n_1+n_2=m-1}\textup{Im}([\,,]_{n_1,n_2})\right) \oplus\left(\bigoplus_{2\leq i< (m-1)/2}\textup{Im}(P^i_{m-i-1})\right)\]
%
\[UV^{m,(2)}=\left(\bigoplus_{n_1+n_2=m-1}\textup{Im}(V^{n_1}\otimes V^{n_2}\overset{[,]}{\to} UV)\right) \oplus\left(\bigoplus_{2\leq i< (m-1)/2}\textup{Im}(V_{m-i-1}\overset{P^i}{\to}UV)\right).\]
Moreover, the map $P^i:V^n\to UV^{n+i+1,(2)}$ is an isomorphism onto its image, so that for $2\leq i <n$ we may construct $\gamma^i$ as the composite
\[\gamma^i:\left((UV)^{n+i+1}\overset{\textup{proj}}{\makebox[.06ex][l]{$\to$}\to} (UV)^{n+i+1,(2)}\overset{\textup{proj}}{\makebox[.06ex][l]{$\to$}\to} \textup{Im}(P^i)\overset{(P^i)^{-1}}{\to}V^{n}\right).\]
Here we have projected onto the quadratic filtration 2 part, and then further onto the relevant summand in its natural decomposition. Note that although $P^n:V^n\to UV^{2n+1}$ is a nontrivial map, its image is entangled with the image of the bracket, and we are not able to split it off. Thus we are not able to improve on the bounds $2\leq i< n$.
\begin{prop}
There are natural operations $\delta_i:H^{s}_tW\to H^{s+1}_{t+i+1}W$ defined whenever $2\leq i <t$. If $\alpha\in C^{s}_t$ is a cocycle, $\delta_i\alpha$ is the composite
\[N_{s+1}(Q_U B_{\bullet}W)^{t+i+1}\overset{\gamma^i}{\to}N_s(Q_U B_{\bullet}W)^{t}\overset{\alpha}{\to}\F_2,\]
after identifying $Q_U B_{s+1}W$ with $U^{s}W$ and $Q_U B_{s}W$ with $U^{s+1}W$. These operations satisfy the Adem relation (see Dwyer) for $\delta$ operations.
\end{prop}
\begin{proof}
One needs to see that $\gamma^i$ is a chain homomorphism respecting the normalisation functor:\[\xymatrix@R=4mm{
N_{s+1}(Q_U B_{\bullet}W)^{t+i+1}\ar[r]^-{\gamma^i}
\ar[d]^-{\partial}&%r1c1
N_s(Q_U B_{\bullet}W)^{t}
\ar[d]^-{\partial}
\\%r1c2
N_{s}(Q_U B_{\bullet}W)^{t+i+1}\ar[r]^-{\gamma^i}&%r2c1
N_{s-1}(Q_U B_{\bullet}W)^{t}%r2c2
}\]
For this, it is enough to produce commuting squares:
\[\xymatrix@R=4mm{
(U^sW)^{t+i+1}\ar[r]^-{\gamma^i}
\ar[d]^-{d_0+d_1}&%r1c1
(U^{s-1}W)^{t}
\ar[d]^-{d_0}
\\%r1c2
(U^{s-1}W)^{t+i+1}\ar[r]^-{\gamma^i}&%r2c1
(U^{s-2}W)^{t}%r2c2
}\]
and for $j\geq1$:
\[\xymatrix@R=4mm{
(U^sW)^{t+i+1}\ar[r]^-{\gamma^i}
\ar[d]^-{d_{j+1}}&%r1c1
(U^{s-1}W)^{t}
\ar[d]^-{d_j}
\\%r1c2
(U^{s-1}W)^{t+i+1}\ar[r]^-{\gamma^i}&%r2c1
(U^{s-2}W)^{t}%r2c2
}\]
The later squares are easy, and and for the first one, starting with $f_{(s)}g_{(s-1)}h_{(s-2)}\in U^sW$, we find $\gamma^i d_0=u(f)\gamma^i (g)(h_{(s-2)})$,
$\gamma^i d_1=\gamma^i (fg)(h_{(s-2)})$, and
$d_0\gamma^i =\gamma^i (f)u(g)(h_{(s-2)})$. These three quanitities sum to zero, as the single operation $P^i$ is indecomposable. This proves the well-definedness of these operations.

What remains is to check that the Adem relations are satisfied. The necessary argument is a modification of that given in Priddy's work [Koszul], which accommodates for the presence of the extra operations. As the algebra of $\delta$-operations is Koszul dual (in Priddy's sense) to $\Palg$, the Adem relations follow. [$\diamondsuit$ Should I include the compression to the quadratic grading 2 parts?]% For this, show that we can compress to the terms in which there is quadratic grading 2 in each construction. Then the Adem relation is the image under the differential of something obvious.
\end{proof}

\subsection{Steenrod operations and a commutative product for $H^*_{\calW}$}
\todoeasy{Define and state $\Sq$ operations and products}{Need to define $j:Q(X\wedge Y)\to QX\otimes QY$}
We'll now give a construction of operations on the homology $H^*_{\calW}$ which arise from the Lie structure on elements of $\calW$. We'll follow the framework set out by Goerss in the derivation [] of the $\calW$ structure operations on Andre-Quillen homology. The first construction is that of the smash product of objects $X_1,X_2\in\calW$, which is defined to be the kernel of the natural map from the coproduct $X_1\sqcup X_2$ to the product $X_1\times X_2$, so that there is a short exact sequence:
\[\xymatrix@R=4mm{
0\ar[r]
&%r1c1
X_1\wedge X_2\ar[r]
&%r1c2
X_1\sqcup X_2\ar[r]
&%r1c3
X_1\times X_2\ar[r]&0.
}\]
We leave the interested reader to verify:
\begin{lem}
There is a unique map $j_\calW:Q(X_1\wedge X_2)^t\to (QX_1\otimes QX_2)^{t-1}$ such that
\[P^{i_1}\cdots P^{i_r}[z_1,\cdots ,z_a]\overset{j_\calW}{\mapsto}\begin{cases}
z_1\otimes z_2,&\textup{if }r=0,\,a=2,\,z_1\in X_1\textup{ and } z_2\in X_2,
\\0,&\textup{otherwise, }
\end{cases}
\]
where each of $z_1,\ldots,z_a$ is an element of either $X_1$ or $X_2$, and $[z_1,\ldots,z_s]$ represents some choice of bracketing. It is natural in $X_1$ and $X_2$.
\end{lem}
Next, if $X_\bullet\in s\calW$ is an almost free simplicial object, there is a cogroup object structure $\phi_s:X_s\to X_s\sqcup X_s$ on each level $X_s$ induced by the diagonal map $V_s\to V_s\oplus V_s$ of the generating subspace $V_s$. Write $\overline{\xi}_\calW$ for the difference of the maps $(d_0\sqcup d_0)\phi_s$ and $\phi_{s-1}d_0$ in the group $\textup{Hom}_\calW(X_s,X_{s-1}\sqcup X_{s-1})$. Then $\overline{\xi}_\calW$ factors through a unique map $\xi_\calW:X_s\to X_{s-1}\wedge X_{s-1}$, and one can verify:
\begin{lem}
The map $Q_\calW\xi_\calW$ induces a chain map of (lower) degree $-1$ on normalised complexes:
\[N_s(Q_{\calW}X_\bullet)^t\to N_{s-1}(Q_{\calW}(X_\bullet\wedge X_\bullet))^t.\]
\end{lem}
We now have access to the composite
\[\psi_\calW:\left(N_s(Q_{\calW}X_\bullet)^t\overset{Q_\calW\xi_\calW}{\to} N_{s-1}(Q_{\calW}(X_\bullet\wedge X_\bullet))^t\overset{j_\calW}{\to} N_{s-1}(Q_{\calW}X_\bullet\otimes Q_\calW X_\bullet)^{t-1}\right),\]
which is in fact symmetric, giving a map
\[\psi_{\calW}:N_s(Q_{\calW}X_\bullet)^t\to N_{s-1}(S^2(Q_{\calW}X_\bullet))^{t-1}.\]

From this map we can construct Steenrod operations and a commutative product on the cohology groups $H^*_\calW X$. Explicitly, there are natural homomorphisms
\[\Sq^j:H_t^{s}X\to H_{2t+1}^{s+j}X,\]
defined as follows. Suppose that $\alpha\in C_t^{s}$ is a cocycle. Then we define $\Sq^j\alpha$ as the composite
\[N_{s+j}(Q_UB_{\bullet}X)^{2t+1}\overset{\psi_\calW}{\to}N_{s+j-1}((Q_UB_{\bullet}X)^{\otimes2})^{2t}\overset{\mathbb{D}_{s-j+1}}{\to}
((N_*Q_UB_{\bullet}X)^{\otimes2})^{2t}_{2s}\overset{\alpha\otimes\alpha}{\to}\F_2.
\]
Another way to say this, in the style of Goerss (and others), is to define a function
\[\Theta^j:C_{t}^{s}\to C_{2t+1}^{s+j+1},\ \ \Theta^j(\alpha)=\psi^*_\calW\mathbb{D}_{s-j+1}^*(\alpha\otimes\alpha)+ \psi^*_\calW\mathbb{D}_{s-j+2}^*(\alpha\otimes d\alpha).\]
Note the shift of one in the indices in this formula, as compared with similar formulae [Goerss or Dwyer].
Then, (like Goerss shows), $d\Theta^j\alpha=\Theta^jd\alpha$, and we can define $\Sq^j$ to be the map on cohomotopy defined by $\Theta^j$. Similarly, we can define a cochain complex homomorphism
\[\Psi:C_t^{s}\otimes C_{t'}^{s'}\to C_{t+t'+1}^{s+s'+1}\]
by the formula $\psi^*_\calW\mathbb{D}_0^*$, and use this to define a pairing of cohomology.
\begin{thm}
These operations satisfy the Axioms *write them out*.
\end{thm}
This theorem follows from theorem [BLAH] in the appendix, after the following observation. 

An object of $\calW$ is in particular a $\PMonad$-module, and (as all of the $P$ operations are linear), we can define a functor $Q_{\PMonad}:\calW\to\vect{+}{0}$ by quotienting by the image of these operations. Writing $\calL$ for the category of (good) Lie algebras in $\vect{+}{0}$, one may check:
\begin{lem}
For $X\in\calW$, the vector space $Q_{\PMonad}(X)$ inherits the structure of an object of $\calL$ from the bracket of $X$, yielding a factorization:% of indecomposable functors, $Q_{\calW(n)}=Q_{\calL(n)}\circ Q_{\LambdaMonad(n)}$
\[Q_{\calW}=Q_{\calL}\circ Q_{\PMonad}:\left(\calW\to \calL\to \vect{+}{0}\right).\]
\end{lem}
\begin{proof}
One checks that the bracket is well defined in the quotient, and that taking the quotient by the top $P$ operation imposes the relation $[x,x]=0$ in the resulting Lie algebra, which is thus a good Lie algebra.
\end{proof}
%The functor $Q_\calW:\calW\to \vect{+}{0}$ may be written as a composite of two functors:
%\[\calW\overset{Q_\Palg}{\to}\Lambda(\LieOperad)\textup{-mod}\overset{Q_{\LieOperad}}{\to}\vect{+}{0},\]
%where the functor $Q_\Palg$ takes the quotient by the image of every $P$-operation. It follows from the axioms in $\calW$ that this quotient bears the structure of a (good) Lie algebra. 
Moreover, the simplicial Lie algebra $Q_\PMonad B_\bullet X$ is an almost free simplicial Lie algebra, so that the dual homotopy of $Q_\calW B_\bullet X=Q_{\LieOperad} Q_\PMonad B_\bullet X$ is an instance of Lie algebra cohomology. \textbf{Is this exactly true? It doesn't matter much if not... just need to give extensive enough calculation in the appendix}

By [Priddy's prim coh ops], Lie algebra cohomology has Steenrod operations and a product. However, the formulae given by Priddy (involving the $\overline{W}$-construction) are not obviously equivalent to what we have given here. However, it is not hard to check that the definition we have given using $\psi_\calW$ is equivalent to an analogous definition for Lie algebra cohomology (given in [placemarker in appendix]), and theorem [BLAH again] identifies these operations with Priddy's.

{\tiny
We can identify the bar construction $Q_UB_sX$ with
$Q_{\Lambda(\LieOperad)}\Lambda(\LieOperad)U^sX$,
which is $Q_{\Lambda(\LieOperad)}$ applied to the almost free simplicial Lie algebra $\Lambda(\LieOperad)U^sX$. As such, its homotopy has operations induced by the map
\[\xi_\LieOperad:\left(Q_{\Lambda(\LieOperad)}\Lambda(\LieOperad)U^{s+1}X
\overset{Q(d_0\sqcup d_0\psi+\psi d_0)}{\to}
Q_{\Lambda(\LieOperad)}((\Lambda(\LieOperad)U^sX)^{\wedge2})\overset{j}{\to}
(Q_{\Lambda(\LieOperad)}\Lambda(\LieOperad)U^sX)^{\otimes2}\right)\]
where by $d_0$ we mean the map $\Lambda(\LieOperad)(P\circ S(\LieOperad)/\sim)\to \Lambda(\LieOperad)$ arising from killing the $P$ part and identifying all the Lie stuff together. This only sees the Lie stuff, not the part that has any $P$ in it. Thus it could be written as $f(g)\mapsto q^\LieOperad(f)(g)$.

The operations $\Sq^j:H_t^{s}X\to H_{2t+1}^{s+j}X$ are defined as follows. Suppose that $\alpha\in C_t^{s}$ is a cocycle. Then we define $\Sq^j\alpha$ as the composite
\[N_{s+j}(Q_UB_{\bullet}X)^{2t+1}\overset{\xi_\calL}{\to}N_{s+j-1}((Q_UB_{\bullet}X)^{\otimes2})^{2t}\overset{\mathbb{D}_{s-j+1}}{\to}
((N_*Q_UB_{\bullet}X)^{\otimes2})^{2t}_{2s}\overset{\alpha\otimes\alpha}{\to}\F_2.
\]
Another way to say this, in the style of Goerss (and others), is to define a function
\[\Theta^j:C_{t}^{s}\to C_{2t+1}^{s+j+1},\ \ \Theta^j(\alpha)=\phi^*_\calL\mathbb{D}_{s-j+1}^*(\alpha\otimes\alpha)+ \phi^*_\calL\mathbb{D}_{s-j+2}^*(\alpha\otimes d\alpha).\]
Then, (like Goerss shows), $d\Theta^j\alpha=\Theta^jd\alpha$, and we can define $\Sq^j$ to be the map on cohomotopy defined by $\Theta^j$. Similarly, we can define a cochain complex homomorphism
\[\Psi:C_t^{s}\otimes C_{t'}^{s'}\to C_{t+t'+1}^{s+s'+1}\]
by the formula $\psi^*_\calL\mathbb{D}_0^*$, and use this to define a pairing of cohomology.
\begin{prop}
There are natural homomorphisms
\[\Sq^j:H_t^{s}X\to H_{2t+1}^{s+j}X,\]
zero unless $3\leq j\leq s+1$, and satisfying the Adem relations for the Steenrod algebra. There is a natural nonunital commutative algebra pairing
\[H_t^{s}X\otimes H_{t'}^{s'}X\to H_{t+t'+1}^{s+s'+1}X.\]
These satisfy an unstableness condition:
\[x^2=\Sq^{s+1}x\text{ for }x\in H^{s}_tX\]
and the standard Cartan formula:
\[\Sq^k(xy)=\sum_{i=0}^k(\Sq^ix)(\Sq^{k-i}y).\]
\end{prop}
\begin{proof}
\textbf{To do:} Insert the proof of the Steenrod relations from prliealgs.tex, citing Priddy or really Singer.
\end{proof}
}
\todo{Give theorem relating all the given operations}{prove the $\delta$ Adem relations using Priddy compression
\item show that the Sq-ops and products are instances of those on the cohomology of a Lie algebra (in particular that $j_\calW$ restricts to $j_\calL$), then quote the appendix
\item The commutation relation for $\delta$ vs $\Sq$/product is written on the wall
\item need to figure out what happens in low dimensions}
Do a Steenrod or product operation, and then a $\delta_i$, and you're using the composite of this map with
\[Q_\calW U^{s+3}X=U^{s+2}X\to U^{s+1}X=Q_\calW U^{s+2}X\]
which all boils down to
\[P^i_{(s+2)}f_{(s+1)}g\mapsto q^{\calL}(f)(g)\]
Anyway, I have a nullhomotopy of this map, which might be called $((P^i)^{-1}q^{\calL})$, and is stuck on the wall in the upper right corner.


\section{Unstable Lie algebras over the $\Lambda$-algebra}
\todoeasy{Define the categories $\calW(n)$, $\calL(n)$ ($n=0$ is Goerss' case)}{$\calW(n)$ and $\calL(n)$ are monadic over $\vect{+}{n}$, but we won't carefully describe the monads.}
In preparation for our calculations of cohomology in $\calW$, we introduce here a family of algebraic categories of a similar nature. That is, we will introduce, for each $n\geq 0$, algebraic categories $\calW(n)$ and $\calL(n)$, such that $\calW(0)=\calW$, and for each $n\geq0$:
\begin{itemize}
\setlength{\parindent}{.25in}
\item objects of $\calW(n)$ and $\calL(n)$ are $\vect{+}{n}$-graded;
\item the indecomposables functor $Q_{\calW(n)}:\calW(n)\to \vect{+}{n}$ factors through the indecomposables functor $Q_{\calL(n)}:\calL(n)\to \vect{+}{n}$; and
\item the category $\calW(n+1)$ is equivalent to the category $\calL(n)\textup{-$\Pi$-Alg}$ of algebras for the homotopy operations of an object of $s\calL(n)$.
\end{itemize}
The definitions of the categories $\calW(n)$ for $n\geq1$ are all of a similar flavor, and a little different to the definition above of $\calW(0)=\calW$. 

For $n\geq1$, $\calL(n)$ is a category of (good) Lie algebras, with a restriction operation partially defined. An object $Y$ of $\calL(n)$ is a Lie algebra with bracket
\[[\DASH,\DASH]:Y^t_{s_n,\ldots,s_1}\otimes Y^q_{p_n,\ldots,p_1}\to Y^{t+q+1}_{s_n+p_n,\ldots,s_1+p_1}\]
(satisfying the Jacobi identity and $[x,x]=0$ for all $x$), and restriction operations
\[\restn{(\DASH)}:Y_{s_n,\ldots,s_1}^t\to Y_{2s_n,\ldots,2s_1}^{2t+1}\]
defined whenever not all of $s_n,\ldots,s_{1}$ are zero.
For $x,y\in Y^t_{s_n,\ldots,s_1}$ and $z\in Y$ homogeneous, the restriction operations satisfy \[\restn{(x+y)}=\restn{(x)}+\restn{(y)}+[x,y]\textup{ and }[\restn{(x)},z]=[x,[x,z]].\]
We have defined the bracket and restrictions on homogeneous elements only, but there is no problem extending these operations to non-homogeneous elements using the bilinearity of the bracket and the above formula for the restriction of a sum.


Building on this definition, an object $X$ of the category $\calW(n)$ is to be an object of $\calL(n)$ admitting certain unstable right $\Lambda$ operations. Namely, whenever $0\leq i\leq s_n$ and not all of $i,s_{n-1},\ldots,s_{1}$ are zero, there is defined an operation
\[(\DASH)\lambda_i:X^t_{s_n,\ldots,s_1}\to X^{2t+1}_{s_n+i,2s_{n-1},\ldots,2s_1}.\]
%The operation $(\DASH)\lambda_i$ is required to be linear whenever $i< s_n$, and 
When $i=s_n$ (and $\lambda_i$ is defined) it is required to equal the restriction:
\[(\DASH)\lambda_{s_n}=\restn{(\DASH)}:X^t_{s_n,\ldots,s_1}\to X^{2t+1}_{2s_n,2s_{n-1},\ldots,2s_1}.\]
In particular, we have already specified that this top $\lambda$-operation is a quadratic refinement \textbf{(acceptable language?)} of the bracket, and how the two operations interact [label that equation]. Contrastingly, the non-top operations are required to be linear, and to be killed by the bracket, so that for any $x,y\in X^t_{s_n,\ldots,s_1}$ and $i<s_n$ (with not all of $i,s_{n-1},\ldots,s_{1}$ being zero) and homogeneous $z\in L$:
\[(x+y)\lambda_i=(x)\lambda_i+(y)\lambda_i\textup{ and }[(x)\lambda_i,z]=0.\]
Finally, these right operations satisfy the $\lambda$-Adem relations. That is, whenever $i>2j$, and $x$ is a homogeneous element of $X$ with $((x)\lambda_j)\lambda_i$ defined:
\[((x)\lambda^j)\lambda^i=\sum_{k=0}^{(i-2j)/2-1}{i-2j-2-k\choose k}((x)\lambda^{i-j-1-k})\lambda^{2j+1+k}.\]

Next, we'll define another category, $\LambdaMonad(n)$, whose objects are $\vect{+}{n}$-graded vector spaces with certain unstable $\lambda$ operations. An object $X$ of $\LambdaMonad(n)$ is simply an object of $\vect{+}{n}$ with operations
\[(\DASH)\lambda_i:X^t_{s_n,\ldots,s_1}\to X^{2t+1}_{s_n+i,2s_{n-1},\ldots,2s_1}\]
defined whenever $0\leq i< s_n$ and not all of $i,s_{n-1},\ldots,s_{1}$ are zero. Note the strict inequality $i<s_n$, so that we define all the $\lambda$ operations except for the top $\lambda$ operation. These operations are all assumed to be linear and to satisfy the same $\lambda$-Adem relation.

The objects of $\calW(n)$ may be described as those graded vector spaces with both $\calL(n)$ and $\LambdaMonad(n)$ structure satisfying certain compatibilities. In fact, it is not difficult to show that the free $\calW(n)$ object monad factors as a composite
\[F_{\calW(n)}=\LambdaMonad(n)\circ F_{\calL(n)}:\vect{+}{n}\to \vect{+}{n},\]
and that the monad structure on $F_{\calW(n)}$ arises from a distributive law [{\tiny Beck Dist Laws in seminar on triples and cat homology thy}]. It is important to note that this is monadic down only as far as $\vect{+}{n}$. In particular, we may define a functor $Q_{\LambdaMonad(n)}:\calW(n)\to\vect{+}{n}$, by quotienting by the image of the \emph{non-top} $\lambda$-operations (which, fortunately, are linear). It follows from the axioms in $\calW(n)$ that:
\begin{lem}
For $X\in\calW(n)$, the vector space $Q_{\LambdaMonad(n)}(X)$ inherits the structure of an object of $\calL(n)$ from the bracket and restriction of $X$, yielding a factorization:% of indecomposable functors, $Q_{\calW(n)}=Q_{\calL(n)}\circ Q_{\LambdaMonad(n)}$
\[Q_{\calW(n)}=Q_{\calL(n)}\circ Q_{\LambdaMonad(n)}:\left(\calW(n)\to \calL(n)\to \vect{+}{n}\right).\]
Moreover the composite $Q_{\LambdaMonad(n)}\circ F_{\calW(n)}$ equals the the free construction $F_{\calL(n)}$.
\end{lem}






































\todoeasy{Derived functors of $Q^{\calW(n)}:\calW(n)\to \vect{+}{n}$ are $H^*_{\calW(n)}:\calW(n)\to \vect{n+1}{+}$}{and we're interested in these for $n=0$.}
\todoeasy{Define functors $Q^P:\calW(0)\to\calL(0)$ and $Q^{\Lambda(n)}:\calW(n)\to \calL(n)$ for $n\geq1$}{explain that when $n>0$ you divide by nontop operations only, and the opposite for $n=0$
\item point is that $Q^{\calW(n)}$ factors as these followed by $Q^{\calL(n)}:\calL(n)\to \vect{+}{n}$}
\todoeasy{Note that $\calL(n)$-$\Pi$-\textup{alg} is the category $\calW(n+1)$}{comment that this is why $\calW(n)$ will be interesting for $n>0$}
%\todo{Define cohomology $H_{\calW(n)}^*:\calW(n)\to \vect{n+1}{+}$}{}
\todoeasy{Define homology $H^{\Lambda(n)}_*:\calW(n)\to \vect{+}{n+1}$ and enrich codomain to $\calW(n+1)$}{also do $n=0$ case: define homology $H^P_*:\calW(0)\to \vect{+}{1}$ and enrich codomain to $\calW(1)$}

\section{Cohomology operations for unstable Lie algebras over $\Lambda$}
\todoeasy{Do all the same work as in the $n=0$ case, to get horizontal and vertical $\Sq$ and products}{must explain in detail what the available operations are
\item relations between them less important}
\subsection{Vertical steenrod operations for $H^*_{\calW(n)}$, $n\geq1$}
Fix $n\geq1$. For $V\in \vect{+}{n}$, we will define natural homomorphisms
\[\gamma_i:(F_{\calW(n)}V)^{2t+1}_{s_n+i-1,2s_{n-1},\ldots,2s_1}\to V^{t}_{s_n,\ldots,s_1},\]
whenever $1\leq i \leq s_n$ and not all of $i-1,s_{n-1},\ldots,s_1$ are zero.
{\tiny Meh: These are rather easier to define than in the $n=0$ case above, as the monad $F_{\calW(n)}$ is a simple composite of monads.}
Indeed, there are natural monomorphisms
\[(\DASH)\lambda_{i-1}:V^{t}_{s_n,\ldots,s_1}\to (F_{\calW(n)}V)^{2t+1,(2)}_{s_n+i-1,2s_{n-1},\ldots,2s_1}\]
defined whenever $1\leq i\leq n$ and $i-1,s_{n-1},\ldots,s_1$ are not all zero, and an inclusion
\[\textup{incl}:(F_{\calL(n)}V)^{*,(2)}\to F_{\calW(n)}V^{*,(2)}.\]
As in the $n=0$ case, the images of the listed maps are linearly independent and span the quadratic grading 2 part of $F_{\calW(n)}V^*$. We define $\gamma_i$ as before --- project onto the quadratic grading 2 part, then further onto the summand corresponding to the image of $\lambda_{i-1}$, and finally use the inverse of the map $\lambda_{i-1}$:
\[\gamma_i:\left((F_{\calW(n)}V)^{2t+1}_{s_n+i-1,2s_{n-1},\ldots,2s_1}\overset{\textup{proj}\circ\textup{proj}}{\makebox[.06ex][l]{$\to$}\to} \textup{Im}(\lambda_{i-1})\overset{(\lambda_{i-1})^{-1}}{\to}V^{t}_{s_{n,\ldots,s_1}}\right).\]
\begin{prop}
There are natural operations
\[\Sq_\textup{v}^i:(H^*_{\calW(n)}X)^{s_{n+1},\ldots,s_1}_t\to (H^*_{\calW(n)}X)^{s_{n+1}+1,s_n+i-1,2s_{n-1},\ldots,2s_1}_{2t+1}\]
defined whenever $1\leq i \leq s_n$ and not all of $i-1,s_{n-1},\ldots,s_1$ are zero. If $\alpha\in C^{s_{n+1},\ldots,s_1}_t$ is a cocycle, $\Sq_\textup{v}^i\alpha$ is the composite
\[N_{s_{n+1}+1}(Q_U B_{\bullet}X)^{2t+1}_{s_{n}+i-1,2s_{n-1},\ldots,2s_1}\overset{\gamma_i}{\to}N_{s_{n+1}}(Q_U B_{\bullet}X)^{t}_{s_{n},s_{n-1},\ldots,s_1}\overset{\alpha}{\to}\F_2,\]
after identifying $Q_U B_{s+1}X$ with $U^{s}X$ and $Q_U B_{s}X$ with $U^{s+1}X$. These operations satisfy the Adem relation for Steenrod operations.
\end{prop}
\begin{proof}
The proof here is obtained by modification of the corresponding proof in the $n=0$ case. Note that we use the Koszul duality between the Steenrod and $\Lambda$ algebras (Priddy Koszul Resolutions). One might make a note of the fact that all of the necessary relation terms are actually available --- nothing is ruled out due to dimension issues.
\end{proof}

\subsection{Horizontal steenrod operations and a commutative product for $H^*_{\calW}$}
Say `it follows as in the $n=0$ case'

\section{Composite functor spectral sequences}
\todo{Define the spectral sequence $H^*_{\calW(n)}\Longleftarrow H^*_{\calW(n+1)}H_*^{\Lambda(n)}$}{reference Blanc-Stover, but then give the construction of bisimplicial object (using `$W$' comonad)}
The subject of this paper is to identify the derived functors $H^*_{\calW(0)}X:=(\mathbb{L}_*Q_{\calW(0)}X)^*$, for $X\in\calW(0)$. More generally, we'll now present a spectral sequence that may be used to calculate $H^*_{\calW(n)}X$ for $X\in\calW(n)$. This will a composite functor spectral sequence analogous to Miller's spectral sequence in []. The factorization of $Q_{\calW(n)}$ we will use is of course 
\[Q_{\calW(n)}=\left(\calW(n)\overset{Q_{\calU(n)}}{\to}\calL(n)\overset{Q_{\calL(n)}}{\to}\vect{+}{n}\right)\]
There is an added challenge in this context --- indeed, the available factorization of $Q_{\calW(n)}$ is through a non-abelian category. Thus, the standard technology for composite functor spectral sequences does not apply, and we must use Blanc and Stover's methods []. They observe that the left derived functors $\mathbb{L}_*Q_{\calU(n)}X$ are calculated as the homotopy groups of a simplicial object in $\calL(n)$, namely $Q_{\calU(n)}B_\bullet^{\calW(n)}X$. As such, they have the structure of a $\calL(n)\textup{-$\Pi$-algebra}$, i.e.\ they form an object of $\calW(n+1)$.  After verifying that the functor $Q_{\calU(n)}$ satisfies the requisite acyclicity condition (indeed it preserves free objects), one may apply [relevant BS theorem]: there is a spectral sequence, with $E_r\in\vect{+}{n+2}$,
\[(E_2)_{s_{n+2},\ldots,s_1}^t=((H_*^{\calW(n+1)})(\mathbb{L}_*Q_{\calU(n)})X)_{s_{n+2},\ldots,s_1}^t\implies ((H_*^{\calW(n)})X)_{s_{n+2}+s_{n+1},s_n,\ldots,s_1}^t\]
We'll prefer to work with the dual spectral sequence, (which is equivalent, as we have specified that elements of $\calW(n)$ are locally finite), which has $E^r\in\vect{n+2}{+}$:
\[(E^2)^{s_{n+2},\ldots,s_1}_t=((H^*_{\calW(n+1)})(\mathbb{L}_*Q_{\calU(n)})X)^{s_{n+2},\ldots,s_1}_t\implies ((H^*_{\calW(n)})X)^{s_{n+2}+s_{n+1},s_n,\ldots,s_1}_t\]
These are the homology and cohomology spectral sequences for a bisimplicial abelian group (precisely, an object of $ss\vect{+}{n}$), and we will need to work with this object directly.
\subsection{The Blanc-Stover comonad}
Blanc and Stover \cite{Blanc_Stover-Groth_SS.pdf} define a comonad $\BSW$ (they use the character $W$, which is already overused in this article) on $s\calL(n)$ by defining $\BSW Z$ to be the pushout
\[\xymatrix@R=4mm{
\bigsqcup_{h,n\geq0}S^n_h
\ar[r]\ar[d]^-{\bigsqcup\imath}
&%r1c1
\bigsqcup_{f,n\geq0}S^n_f\ar[d]\\%r1c2
\bigsqcup_{h,n\geq0}CS^n_h
\ar[r]&%r2c1
\BSW X%r2c2
}\]
where the leftmost two colimits are taken over all maps $h:CS^n\to X$ and the top right colimit is take over all maps $f:S^n\to X$.
%where $H^n_X:=\hom_{\calL(n)}(CS^n,X)$ and $S^n_X:=\hom_{\calL(n)}(S^n,X)$
Here, the left vertical is the coproduct of many copies of the standard inclusion $\imath:S^n\to CS^n$, and the top horizontal map sends a summand $S^n_h$ onto the sphere $S^n_{h\imath}$.

It will be useful to reinterpret this construction via the Dold-Kan correspondence. Indeed, $\hom_{s\calL(n)}(S^n,X)=\hom_{s\vect{+}{n}}(S^n,UX)=\hom_{\textup{ch}\vect{+}{n}}(S^n,UX)=$

Blanc and Stover explain that $\BSW $ actually has the structure of a comonad on $s\calL(n)$. As such, for any simplicial Lie algebra $L_\bullet\in s\calL(n)$, there is a bisimplicial object $B_\bullet^\BSW L_\bullet$, the bar construction using the comonad $\BSW $, given by $B_{s_2}^\BSW L_{s_1}=(\BSW^{s_2+1}L_\bullet)_{s_1}$. Flesh out the point here...

Now $\BSW X$ is homotopy equivalent to a coproduct of spheres. Indeed, for each map $f:S^n\to X$ that extends to a map $CS^n\to X$, \emph{choose} an extension $h_0(f):CS^n\to X$ (so that $h_0(f)\imath=f$). Then $\BSW X$ contains the contractible subobject $C_0:=\bigsqcup_f CS^n_{h_0(f)}$. Moreover, $\BSW X/C_0$ is a wedge of spheres:
\[\BSW X/C_0\cong \left(\bigsqcup_{h\textup{ not chosen}}CS^n_h/\partial(CS^n_h)\right) \sqcup\left(\bigsqcup_{f\textup{ not null}}S^n_f\right)\]
Representatives for a \emph{set} of generators of $\pi_*(\BSW X)$ are 
\[\{CS^n_h-CS^n_{h_0(h\imath)}\ |\ \textup{$h$ was not chosen}\}\cup\{S^n_f\ |\ \textup{$f$ not null}\}.\]
However, it is preferable to find a subspace of $\pi_*(\BSW X)$ which freely generates it as an element of $\calW(n+1)$.
%A thought here is that there are maps of vector spaces we can use. Let $VH_X$ be the free vector space on the homologies to $X$, and let $VS_X$ be the free vector space on the spheres.
For this purpose, observe that there's a natural map $i^*:\F_2[H_X]\to \F_2[S_X]$, and a natural map $\alpha:\ker(i^*)\to\pi_*(\BSW X)$. Moreover, there's a natural map $\beta:\F_2[S_X]\to\pi_*(\BSW X)$. Then $\textup{Im}(\alpha)$ and $\textup{Im}(\beta)$ are linearly independent subspaces of $\pi_*(\BSW X)$, and $\pi_*(\BSW X)$ is free on $\textup{Im}(\alpha)\oplus\textup{Im}(\beta)$. From this description it is clear that the generating subspaces are preserved by maps in the image of $\BSW $. Moreover, the diagonal of the comonad also preserves the subspaces $\textup{Im}(\alpha)$ and $\textup{Im}(\beta)$. That $\textup{Im}(\alpha)$ is preserved is evident from the definitions. For $\textup{Im}(\beta)$, note that $\textup{Im}(\beta)\subset\pi_*(\BSW X)$ is spanned by terms $D_h-D_{h'}$ for $h,h':CS\to X$ satisfying $h\imath=h'\imath$. The diagonal applied to $D_h-D_{h'}$ may be written $D_{D_h}-D_{D_{h'}}$, and one notes that $D_h\imath=S_{h\imath}=S_{h'\imath}=D_{h'}\imath$.

Now the utility of the comonad structure lies in resolving an object $L_\bullet\in s\calL(n)$ by the comonadic bar construction, $B^\BSW_\bullet(L_\bullet)$. This discussion shows that, except for $d_0$, all the maps in $B_\bullet^\BSW L_\bullet$ preserve the subspaces of generators of $\pi_*(B_\bullet^\BSW L_\bullet)$, which is to say that this object of $s\calW(n+1)$ is almost free. Moreover, according to [Stover 2.6], the augmentation map $\pi_*(B_\bullet^\BSW L_\bullet)\to \pi_*(L)$ is a weak equivalence in $s\calW(n+1)$. Thus, this Blanc-Stover $\BSW $-construction provides an almost-free replacement of $\pi_*(L)$ in $s\calW(n+1)$.
%
%
%
% Now we may choose the contractible subobjects of each $W^{n+1}X$ in such a way that the degeneracies $W^{n+1}X\to W^{n+2}X$ are homotopic to wedge inclusions of spheres, c.f.\ \cite[2.5]{StoverVanKampen.pdf}. In particular, the degeneracies of $\pi(W^{n+1}X)\in s\calW(n+1)$ preserve the chosen subspaces of generators. Moreover, all the face maps except $d_0$ are in the image of the functor $W$, and thus preserve subspaces of generators.
%
%
%
% which on homotopy yields an almost free simplicial resolution of $\pi_*(L)\in \calW(n+1)$:
%\[\vcenter{
%\def\labelstyle{\scriptstyle}
%\xymatrix@C=1.5cm@1{
%\pi_*(L)
%&
%\pi_*(WL)
%\ar[r]|(.65){s_0}
%\ar@{-->}[l]|(.65){d_0}
%&
%\pi_*(W^2L)
%\ar@<-1ex>[l]|(.65){d_0}
%\ar@<+1ex>[l]|(.65){d_1}
%\ar@<+1ex>[r]|(.65){s_0}
%\ar@<-1ex>[r]|(.65){s_1}
%&
%\pi_*(W^3L)
%\ar[l]|(.65){d_1}
%\ar@<-2ex>[l]|(.65){d_0}
%\ar@<+2ex>[l]|(.65){d_2}
%\ar[r]|(.65){s_1}
%\ar@<+2ex>[r]|(.65){s_0}
%\ar@<-2ex>[r]|(.65){s_2}
%&
%\pi_*(W^4L)\makebox[0cm][l]{\,$\cdots $}
%\ar@<-3ex>[l]|(.65){d_0}
%\ar@<-1ex>[l]|(.65){d_1}
%\ar@<+1ex>[l]|(.65){d_2}
%\ar@<+3ex>[l]|(.65){d_3}
%}}\]
%
%
%
% So, if by almost free we mean that there are vector spaces of generators which are preserved by everything except $d_0$, we have shown that $\pi(W^{\bullet+1}X)$ is an almost free object in $s\calW(n+1)$.
%Now there is a bisimplicial object $W^{n+1}X$ using the comonad structure, and we may choose the contractible subobjects of each $W^{n+1}X$ in such a way that the degeneracies $W^{n+1}X\to W^{n+2}X$ are homotopic to wedge inclusions of spheres, c.f.\ \cite[2.5]{StoverVanKampen.pdf}. In particular, the degeneracies of $\pi(W^{n+1}X)\in s\calW(1)$ preserve generators.


\subsection{The Blanc-Stover Spectral sequence}
We'll now give the derivation of the spectral sequence for $X\in\calW(n)$. Indeed, writing $B^{\calW(n)}_{\bullet}X\in s\calW(n)$ for the simplicial bar construction on $X$, we use the double complex associated with the bisimplicial object
\[(Q_{\calL(n)}B^\BSW_{\bullet}(Q_{\calU(n)}X^c))\in ss\vect{+}{n},\]
where $X^c$ is shorthand for the cofibrant replacement of $X$ in $s\calW(n)$ given by the bar construction: $B_\bullet^{\calW(n)}X$.









\subsection{A chain-level diagonal on the $\BSW $ construction}
\todo{give the construction of the chain level diagonal}{}
We have seen, for $L\in s\calL(n)$, that $\pi_*(\BSW L)$ is a free object in $\calW(n+1)$. As such, there is a diagonal
\[\phi_{\calW(n+1)}:\pi_*(\BSW L)\to \pi_*(\BSW L)\sqcup \pi_*(\BSW L).\]
In this section, we'll describe how this map is the map on homotopy induced by a morphism $\phi_\textup{ch}:\BSW L\to \BSW L\sqcup \BSW L$ in $s\calL(n)$ (note that homotopy preserves coproducts of coproducts of spheres). 

In order to construct a map $\phi_\textup{ch}$, it is enough to construct a commuting diagram
\[\xymatrix@R=4mm{
S^n\ar[r]^-{\phi_1}
\ar[d]^-{\imath}
&%r1c1
S^n\sqcup S^n\ar[d]^-{\imath\sqcup\imath}
\\%r1c2
CS^n\ar[r]^-{\phi_2}
&%r2c1
CS^n\sqcup CS^n%r2c2
}\]
The maps $\phi_1$ and $\phi_2$ can then be applied respectively to all of the sphere and cone classes appearing in $\BSW L$.
In fact, this diagram can even be formed in $s\vect{+}{n}$, before application of $F_{\calL(n)}:\vect{+}{n}\to\calL(n)$, and $\phi_1$ and $\phi_2$ are just the levelwise application of the diagonal map on a simplicial vector space. To understand the effect of $\phi_{\textup{ch}}$ on homotopy, it is enough to identify where the generators of $\pi(\BSW L)$ are sent in $\pi(\BSW L)\sqcup\pi(\BSW L)$, which is easy.
\begin{lem}
$\BSW L$ is naturally a (strict) commutative cogroup object, having comultiplication map $\phi_{\textup{ch}}$, counit map $0:\BSW L\to 0$, and inverse map $\textup{id}:\BSW L\to \BSW L$. In particular, $\hom(\BSW L,\DASH)$ takes values in $\ensuremath{\F_2}$-vector spaces.
\end{lem}
\begin{proof}
There are a few axioms to check, for example, left counitality: that the composite 
\[\xymatrix@R=4mm@1{
\BSW L\ar[r]^-{\phi_{\textup{ch}}}
&%r1c1
\BSW L\sqcup \BSW L\ar[r]^-{\textup{id}\sqcup0}
&%r1c2
\BSW L%r1c3
}\] is the identity. This follows since $(\textup{id}\sqcup0)\phi_1$ is the identity of $S^n$ and $(\textup{id}\sqcup0)\phi_2$ is the identity of $CS^n$. The other axioms follow similarly.
\end{proof}

Using the notation $*$ for the group operation on $\hom_{s\calL(n)}(\BSW L,L')$, we have the following:
\begin{lem}
For maps $f,g:\BSW L\to L'$ we have 
\[Q_{\calL(n)}(f*g)=(Q_{\calL(n)}f+Q_{\calL(n)}g):Q_{\calL(n)}\BSW L\to Q_{\calL(n)}L'.\]
\end{lem}
\begin{proof}
It's enough to check that $Q_{\calL(n)}(\phi_\textup{ch}):Q_{\calL(n)}\BSW L\to Q_{\calL(n)}(\BSW L\sqcup \BSW L)$ equals the diagonal map $Q_{\calL(n)}\BSW L\to Q_{\calL(n)}\BSW L\oplus Q_{\calL(n)}\BSW L$. For this, $Q_{\calL(n)}$ converts all the colimits involved in the construction of $\BSW L$ to colimits in $s\vect{+}{n}$, and $\phi_1$ and $\phi_2$ are both precisely the diagonal map.
\end{proof}

Now let $\overline{\xi}_{\textup{ch}}$ denote the following composite:
%\[W^2X\overset{\phi_{\textup{ch}}}{\to}(W^2X)^{\sqcup2}\overset{\phi_{\textup{ch}}\sqcup \epsilon}{\to}(W^2X)^{\sqcup2}\sqcup(WX) \overset{\epsilon^{\sqcup2}\sqcup\phi_{ch}}{\to}((WX)^{\sqcup2})^{\sqcup2}\overset{\textup{fold}}{\to}(WX)^{\sqcup2}\]
\[\makebox[0mm][r]{$\overline{\xi}_{\textup{ch}}:\ $}\BSW^2L\overset{\phi_{\textup{ch}}}{\to}(\BSW^2L)^{\sqcup2}\overset{a\sqcup b}{\to}(\BSW L)^{\sqcup2}\]
where $a,b:\BSW^2L\to(\BSW L)^{\sqcup2}$ are the composites
\[\xymatrix@R=2mm{
\makebox[0mm][r]{$a:\ $}\BSW^2L\ar[r]^-{\phi_{\textup{ch}}}
&%r1c1
(\BSW^2L)^{\sqcup2}\ar[r]^-{\epsilon^{\sqcup2}}
&%r1c2
(\BSW L)^{\sqcup2}\\
\makebox[0mm][r]{$b:\ $}\BSW^2L\ar[r]^-{\epsilon}
&%r1c1
(\BSW L)\ar[r]^-{\phi_{\textup{ch}}}
&%r1c2
(\BSW L)^{\sqcup2}
}\]
\begin{lem}
The composite 
$\BSW^2L\overset{\overline{\xi}_{\textup{ch}}}{\to}(\BSW L)^{\sqcup2}\to(\BSW L)^{\times2}$ is zero.
\end{lem}
\begin{proof}
This follows from the easy observation that both composites $(\textup{id}\sqcup0)\overline{\xi}_{\textup{ch}}$ and $(0\sqcup\textup{id})\overline{\xi}_{\textup{ch}}$ equal $\epsilon:\BSW^2L\to \BSW L$.
\end{proof}
In particular, $\overline{\xi}_{\textup{ch}}$ factors through the smash product, defining a natural map $\xi_{\textup{ch}}:\BSW^2L\to (\BSW L)^{\wedge 2}$.

\textbf{For the time being, write $L$ for the object $Q_{\calU(n)}X^c$ in $s\calL(n)$.}
\begin{lem}
$(d_i)^{\wedge 2}\xi_\textup{ch}=\xi_\textup{ch}d_{i+1}$ for $i\geq1$, and $(d_0)^{\wedge 2}\xi_\textup{ch}= (\xi_\textup{ch}d_{0})*(\xi_\textup{ch}d_{1})$.
The map $Q_{\calL(n)}\xi_\textup{ch}$ induces a degree $(-1,0)$ bicomplex map:
\[N_*N_*(Q_{\calL(n)}B^\BSW_{\bullet}L)^t_{s_{n+2},\ldots,s_1}\to
  N_*N_*(Q_{\calL(n)}((B^\BSW_{\bullet}L)^{\wedge 2}))^t_{s_{n+2}-1,s_{n+1},\ldots,s_1}.\]
\end{lem}
\begin{proof}
I used to say `it's just like Goerss'...
\end{proof}
As in the earlier discourse, composition with $j_{\calL(n)}$ yields degree $(-1,0)$ bicomplex map:
\[\psi_\textup{ch}:N_*N_*(Q_{\calL(n)}B^\BSW_{\bullet}L)^t_{s_{n+2},\ldots,s_1}\to
N_*N_*(S^2(Q_{\calL(n)}B^\BSW_{\bullet}L))^{t-1}_{s_{n+2}-1,s_{n+1},\ldots,s_1}.
\]




\todo{use Singer's techniques to give spectral sequence operations}{prove compatibility at $E^2$ for horizontal $\Sq$ and product
\item prove compatibility at $E^2$ for vertical $\Sq$ with Koszul operation
\item prove compatibility at $E^\infty$ for all $\Sq$ and product}
\todo{Discuss the edge homomorphism $e:\left(E^2(n)\to\!\!\!\!\!\!\!\!\!\to E^\infty(n+1)\subseteq E^2_{0}(n+1)\right)$}{figure out what the relevant theorem here should be}

\section{The cohomology of a trivial unstable Lie algebra over $P$}
\todo{State interest in this calculation}{write $X=\bigoplus_\alpha \F_2[r_\alpha]$, a finite direct sum}
\todo{Construct the relevant elements $\Sq^J\delta_I\imath_\alpha$ and state a theorem on $H^*X$}{$\imath_\alpha\in H^0_{r_\alpha}X$ is the functional which projects $QX\cong X$ onto $\F_2[r_\alpha]$.
\item Clearly state the contraints relevant for $K$ and $I$.
\item Also describe which products thereof are not known to be zero, and state a theorem.}
\todo{Proposition: any of the $\Sq^J\delta_I\imath_\alpha$ is detected by $\Sq^{J_n}_\textup{v}e\cdots e\Sq^{J_1}_\textup{v}e\delta_I\imath_\alpha$}{again, clearly state the constraints on the $J_i$.
\item Push out to $ee\Sq^{J_n}_\textup{v}e\cdots e\Sq^{J_1}_\textup{v}e\delta_I\imath_\alpha\in E^2(n+2)_*^{00**\cdots *}$}
\todo{Calculate the groups $E^2(n)_*^{*0**\cdots *}$}{Observe importance of these groups, since everything's detected in one}
\todo{Identify products of $ee\Sq^{J_{n-2}}_\textup{v}e\cdots e\Sq^{J_1}_\textup{v}e\delta_I\imath_\alpha\in E^2(n)_*^{00**\cdots *}$ in $E^2(n)_*^{*0**\cdots *}$}{I still find this hard to write down}
\todo{Prove the theorem on $H^*X$}{start with anything in $E^2(n)$
\item it's detected by something in $E^2(N)_*^{*0**\cdots *}$
\item anything there is a product of $ee\Sq^{J_{N-2}}_\textup{v}e\cdots e\Sq^{J_1}_\textup{v}e\delta_I\imath_\alpha$
\item this converges down to $\Sq^K_h \Sq^{J_n}_\textup{v}e\cdots e\Sq^{J_1}_\textup{v}e\delta_I\imath_\alpha$, which must equal the original element
\item this is a permanent cycle, proving that there are never any differentials in any of the cfsseqs
\item thus $E^2(0)^*_*$ is a direct sum of certain of the $E^2(n)^{*0**\cdots *}_*$ via collapsing sequences
\item stocktake reveals that these terms are everything we hoped}

\appendix
\section{Unstable Lie coalgebras over the dual $P$-algebra (optional)}
\todo{Give a formal definition of these coalgebras}{goes down to level of simplicial commutative algebras, and brings in the SOA comonad structure}
\todo{Give the construction (via equaliser diagram, etc) of the comonad}{}
\todo{Write the maps $j$ and $\gamma_i$ in terms of maps out of $C_{ij}$ and $C_i^j$}{include generating cofibrations such as $S^{i+j}\to C_{ij}$ and $\Delta[1]\times S^{i+j}\to \Delta[1]\times C_{ij}$}
\todo{Construct operations on homology of such coalgebras}{they agree with those in double dual construction when locally finite
\item if not locally finite, by studying gradings, can see that every coalgebra is the union of its finite type (even just `finite') subcoalgebras, and that the cobar construction respects such unions, thus get the same results in infinite case as we did from double dualisation.}


\section{Operations on Lie algebra cohomology}
\todo{Recall Priddy's definition of Lie algebra cohomology, and results}{Explain that they don't calculate everything for us already, since the actual simplicial Lie algebras we're looking at are not GEMs --- their homotopy supports nontrivial operations.}
\todo{Modify Priddy's spectral sequence argument to identify his operations with ours}{uses Singer's work}


\section{A May spectral sequence for $H^*_{\calW}$}
\todo{Explain the quadratic filtration of the bar construction}{derive a spectral sequence}
\todo{figure out what it implies about the whole thing.}{}


\section{To do/consider:}
\begin{enumerate}\squishlist
\setlength{\parindent}{.25in}
\item introduce the category $\calU(n)$, uniformize notation for free constructions. Scrap Goerss' $U$ notation, instead write $F_{\calW}$.
\item which side do you index bar constructions from?
\item Do I insert a section of conventions, including: %\cancel{cugas} and \cancel{model structures}, free functors using standard notation 
$F_?$ thought of as monads without further notation, %indecomposable, categories of vector spaces, good and bad Lie algebras, 
higher Eilenberg-Zilber maps, %never substituting out stars for gradings and
having all gradings at the very right, tensor products of vector spaces,\ldots?
\item Comment that we care about derived indecomposables, not AQ homology.
\item choose a notation for the quadratic grading? like $\textup{qu}_2V^{*}_{***}$?
\item Monadicity over vector spaces is important and should be emphasised, especially for the diagonal on the Blanc-Stover resolution.
\item Maybe I should keep finiteness to myself, given that the SOA might be around, or even the W comonad.
\item change all $\calW$'s to $\calK$'s?
\end{enumerate}




\end{document}



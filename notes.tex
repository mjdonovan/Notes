\documentclass[11pt]{article}
%\usepackage{cancel}
\usepackage{amsmath,amsthm,amssymb}
\usepackage{mathrsfs,nicefrac}
\usepackage{amssymb}
\usepackage{epsfig}
\usepackage[all]{xy}
\usepackage{sseq}


\newcommand{\II}{\mathcal{I}}

\DeclareSymbolFont{AMSb}{U}{msb}{m}{n}
\DeclareMathSymbol{\N}{\mathbin}{AMSb}{"4E}
\DeclareMathSymbol{\Z}{\mathbin}{AMSb}{"5A}
\DeclareMathSymbol{\R}{\mathbin}{AMSb}{"52}
\DeclareMathSymbol{\Q}{\mathbin}{AMSb}{"51}
\DeclareMathSymbol{\PP}{\mathbin}{AMSb}{"50}
\DeclareMathSymbol{\I}{\mathbin}{AMSb}{"49}
\DeclareMathSymbol{\C}{\mathbin}{AMSb}{"43}
\DeclareMathSymbol{\A}{\mathbin}{AMSb}{"41}
\DeclareMathSymbol{\F}{\mathbin}{AMSb}{"46}

\newcommand{\ad}{\textup{\textbf{ad}}}
\newcommand{\sll}{\mathfrak{sl}}
\newcommand{\gl}{\mathfrak{gl}}
\newcommand{\GL}{\mbox{GL}}
\newcommand{\SL}{\mbox{SL}}
\newcommand{\tr}{\mbox{tr\ }}
\newcommand{\Mat}{\mbox{Mat}}
\newcommand{\Lie}{\mbox{\textbf{Lie }}}
\newcommand{\Der}{\textup{Der}}
\newcommand{\End}{\mbox{End\ }}
\newcommand{\im}{\mbox{im}}
\newcommand{\pim}{\mbox{pim}}
\newcommand{\supp}{\mbox{supp\,}}
\newcommand{\Ker}{\mbox{ker\ }}
\renewcommand{\ker}{\textup{ker}\,}
\newcommand{\coker}{\textup{coker}\,}

\newcommand{\PPPP}{\mathcal{P}}

\newcommand{\aaa}{\mathfrak{a}}
\newcommand{\mmm}{\mathfrak{m}}
\newcommand{\qqq}{\mathfrak{q}}
\newcommand{\ppp}{\mathfrak{p}}
\newcommand{\g}{\mathfrak{g}}
\newcommand{\h}{\mathfrak{h}}
\newcommand{\m}{\mathfrak{m}}
\newcommand{\He}{\mathfrak{H}}
\newcommand{\shfF}{\mathscr{F}}
\newcommand{\shfG}{\mathscr{G}}
\newcommand{\shfH}{\mathscr{H}}
\newcommand{\shfL}{\mathscr{L}}
\newcommand{\sHOMFG}{\mathscr{H}\textit{\!\!om}(\shfF,\shfG)}
\newcommand{\sHOM}{\mathscr{H}\textit{\!\!om}}
\newcommand{\Hom}{\textup{Hom}}
\newcommand{\Ext}{\textup{Ext}}
\newcommand{\Map}{\textup{Map}}
\newcommand{\sk}{\vspace*{1em}}
\renewcommand{\phi}{\varphi}
%\newcommand{\ann}{\textup{ann}}
\DeclareMathOperator{\ann}{ann}
\newcommand{\Squ}{\textup{Sq}}


\theoremstyle{plain}
\newtheorem{thm}{Theorem}[section]
\newtheorem*{thm*}{Theorem}
\newtheorem{lem}[thm]{Lemma}
\newtheorem*{lem*}{Lemma}
\newtheorem{prop}[thm]{Proposition}
%\newtheorem*{prop}{Proposition}
\newtheorem{cor}[thm]{Corollary}
\newtheorem{defprop}[thm]{Definition-Proposition}


\theoremstyle{definition}
\newtheorem{defn}{Definition}[section]
\newtheorem{exmp}{Example}[section]
\newtheorem{asspt}{Assumption}[section]
\newtheorem{notation}{Notation}[section]
\newtheorem{exercise}{Exercise}[section]


% 1-inch margins, from fullpage.sty by H.Partl, Version 2, Dec. 15, 1988.
\topmargin 0pt
\advance \topmargin by -\headheight
\advance \topmargin by -\headsep
\textheight 9.1in
\oddsidemargin 0pt
\evensidemargin \oddsidemargin
\marginparwidth 0.5in
\textwidth 6.5in

\parindent 0in
\parskip 1.5ex
%\renewcommand{\baselinestretch}{1.25}

\newcommand{\RAD}[1]{\textup{rad}(#1)}
\newcommand{\SPC}[1]{\textup{sp}(#1)}
\newcommand{\RED}[1]{{#1}_\textup{red}}
\newcommand{\REDpsh}[1]{{#1}_\textup{red}^-}
\newcommand{\OO}{\mathcal{O}}
\newcommand{\spec}{\textup{spec}\,}
\newcommand{\spce}{\textup{sp}}
\newcommand{\proj}{\textup{proj}\,}
\newcommand{\HOMs}{\textup{Hom}_\mathfrak{Sch}}
\newcommand{\HOMr}{\textup{Hom}_\mathfrak{Rings}}

\renewcommand{\to}{\longrightarrow}
\renewcommand{\mapsto}{\longmapsto}
\newcommand{\eps}{\varepsilon}

\newcommand{\catC}{\mathcal{C}}
\newcommand{\catD}{\mathcal{D}}
\newcommand{\catSet}{\textit{Sets}}
\newcommand{\op}{\textup{op}}
\newcommand{\dash}{\textup{---}}
\newcommand{\id}{\textup{id}}
\newcommand{\mapsfrom}{\,\reflectbox{$\mapsto$}\ }
\newcommand{\cone}{\textup{cone}}
\newcommand{\COMMENT}[1]{}

\newcommand{\calL}{\mathcal{L}}

\newcommand{\myheading}[1]{\section*{#1}}

\begin{document}
\myheading{Principal $G$-bundles on $\Sigma A$ --- clutching v.\ classifying}
Let $A$ be a pointed space, and $G$ a topological group. Suppose that $\beta:A\to \Omega BG$ and $\gamma:\Sigma A\to BG$ are adjoint maps. We can produce a principal $G$-bundle on $\Sigma A$ by either using a clutching construction with $\beta$, or pulling back $EG$ along $\gamma$. This note shows that the two bundles obtained are isomorphic.

Firstly, use the HLP to construct the map $L$ below (where $\star$ is the constant map to the basepoint of $EG$, and $\widetilde\epsilon$ is evaluation $(t,\sigma)\mapsto \sigma(t)$):
\[\xymatrix{
\{0\}\times\Omega BG\ar[r]^{\ \ \ \ \star}\ar[d]&EG\ar[d]^{\pi}\\
I\times\Omega BG\ar[r]^{\ \ \ \ \widetilde\epsilon}\ar@{-->}[ru]^L&BG
}\]
Then the equivalence $\phi:\Omega BG\to G$ is given by $\sigma\mapsto L(1,\sigma)$, where $G$ is identified with the fiber of $\pi$. Now define:
%\[E_cG:=G\times C(\Omega BG)\sqcup G / (g,1,\sigma)\sim gL(1,\sigma).\]
%There is a map $E_cG\to \Sigma\Omega BG$ defined by sending $(g,t,\sigma)\mapsto(t,\sigma)$. Moreover, there is a map $E_cG\to EG$ defined by $(g,t,\sigma)\mapsto gL(t,\sigma)$. Thus we have a morphism of principal $G$-bundles lying over the counit $\epsilon_{BG}$:
%\[\xymatrix{
%E_cG\ar[r]\ar[d]&EG\ar[d]\\
%\Sigma\Omega BG\ar[r]^{\ \ \epsilon_{BG}}&BG
%}\]%\text{\ \ which implies that }\]
%Thus $E_cG\simeq\epsilon_{BG}^*(EG)$. Now, let $E_c\beta$ be the space:
%\[E_c\beta:=G\times C(A)\sqcup G / (g,1,a)\sim gL(1,\beta(a)).\]
%Then $E_c\beta$ maps to $\Sigma A$ via $(g,t,a)\mapsto(t,a)$. Moreover, there is a map $E_c\beta\to E_cG$ defined by $(g,t,a)\mapsto (g,t,\beta(a))$.
%
%
%Now define:
%\[E_cG:=G\times C(\Omega BG)\sqcup G / (g,1,\sigma)\sim gL(1,\sigma).\]
%There is a map $E_cG\to \Sigma\Omega BG$ defined by sending $(g,t,\sigma)\mapsto(t,\sigma)$. Moreover, there is a map $E_cG\to EG$ defined by $(g,t,\sigma)\mapsto gL(t,\sigma)$. Thus we have a morphism of principal $G$-bundles lying over the counit $\epsilon_{BG}$:
%\[\xymatrix{
%E_cG\ar[r]\ar[d]&EG\ar[d]\\
%\Sigma\Omega BG\ar[r]^{\ \ \epsilon_{BG}}&BG
%}\]%\text{\ \ which implies that }\]
%Thus $E_cG\simeq\epsilon_{BG}^*(EG)$. Now, 
Let $E_c\beta$ be the space:
\[E_c\beta:=G\times C(A)\sqcup G / (g,1,a)\sim gL(1,\beta(a)).\]
Then $E_c\beta$ maps to $\Sigma A$ via $(g,t,a)\mapsto(t,a)$. Moreover, there is a map $E_c\beta\to EG$ defined by $(g,t,a)\mapsto gL(t,\beta(a))$, which makes the following diagram commute (as is checked): 
\[\xymatrix{
E_c\beta\ar[rr]\ar[d]&&EG\ar[d]&
(g,t,a)\ar@{|->}[rr]\ar@{|->}[d]&&gL(t,\beta(a))\ar@{|->}[d]\\
\Sigma A\ar[r]^{\Sigma\beta}&\Sigma\Omega BG\ar[r]^{\epsilon_{BG}}&BG&
(t,a)\ar@{|->}[r]&(t,\beta(a))\ar@{|->}[r]&\beta(a)(t)
}\]
As $\gamma=\epsilon_{BG}\circ\Sigma\beta$, this shows that $E_cB$ (obtained from $\beta$ by clutching) is isomorphic to $\gamma^*EG$.
\end{document}


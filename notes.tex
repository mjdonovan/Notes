% !TEX root = z_output/_notes.tex
%%%%%%%%%%%%%%%%%%%%%%%%%%%%%%%%%%%%%%%%%%%%%%%%%%%%%%%%%%%%%%%%%%%%%%%%%%%%%%%%
%%%%%%%%%%%%%%%%%%%%%%%%%%% 80 characters %%%%%%%%%%%%%%%%%%%%%%%%%%%%%%%%%%%%%%
%%%%%%%%%%%%%%%%%%%%%%%%%%%%%%%%%%%%%%%%%%%%%%%%%%%%%%%%%%%%%%%%%%%%%%%%%%%%%%%%
\documentclass[11pt]{article}
\newcommand{\dontloadhyperref}{}
\usepackage{fullpage}
\usepackage{amsmath,amsthm,amssymb}
\usepackage{mathrsfs,nicefrac}
\usepackage{amssymb}
\usepackage{epsfig}
\usepackage[all,2cell]{xy}
\usepackage{sseq}
\usepackage{tocloft}
\usepackage{cancel}
\usepackage[strict]{changepage}
\usepackage{color}
\usepackage{tikz}
\usepackage{extpfeil}
\usepackage{version}
\usepackage{framed}
\definecolor{shadecolor}{rgb}{.925,0.925,0.925}

%\usepackage{ifthen}
%Used for disabling hyperref
\ifx\dontloadhyperref\undefined
%\usepackage[pdftex,pdfborder={0 0 0 [1 1]}]{hyperref}
\usepackage[pdftex,pdfborder={0 0 .5 [1 1]}]{hyperref}
\else
\providecommand{\texorpdfstring}[2]{#1}
\fi
%>>>>>>>>>>>>>>>>>>>>>>>>>>>>>>
%<<<        Versions        <<<
%>>>>>>>>>>>>>>>>>>>>>>>>>>>>>>
%Add in the following line to include all the versions.
%\def\excludeversion#1{\includeversion{#1}}

%>>>>>>>>>>>>>>>>>>>>>>>>>>>>>>
%<<<       Better ToC       <<<
%>>>>>>>>>>>>>>>>>>>>>>>>>>>>>>
\setlength{\cftbeforesecskip}{0.5ex}

%>>>>>>>>>>>>>>>>>>>>>>>>>>>>>>
%<<<      Hyperref mod      <<<
%>>>>>>>>>>>>>>>>>>>>>>>>>>>>>>

%needs more testing
\newcounter{dummyforrefstepcounter}
\newcommand{\labelRIGHTHERE}[1]
{\refstepcounter{dummyforrefstepcounter}\label{#1}}


%>>>>>>>>>>>>>>>>>>>>>>>>>>>>>>
%<<<  Theorem Environments  <<<
%>>>>>>>>>>>>>>>>>>>>>>>>>>>>>>
\ifx\dontloaddefinitionsoftheoremenvironments\undefined
\theoremstyle{plain}
\newtheorem{thm}{Theorem}[section]
\newtheorem*{thm*}{Theorem}
\newtheorem{lem}[thm]{Lemma}
\newtheorem*{lem*}{Lemma}
\newtheorem{prop}[thm]{Proposition}
\newtheorem*{prop*}{Proposition}
\newtheorem{cor}[thm]{Corollary}
\newtheorem*{cor*}{Corollary}
\newtheorem{defprop}[thm]{Definition-Proposition}
\newtheorem*{punchline}{Punchline}
\newtheorem*{conjecture}{Conjecture}
\newtheorem*{claim}{Claim}

\theoremstyle{definition}
\newtheorem{defn}{Definition}[section]
\newtheorem*{defn*}{Definition}
\newtheorem{exmp}{Example}[section]
\newtheorem*{exmp*}{Example}
\newtheorem*{exmps*}{Examples}
\newtheorem*{nonexmp*}{Non-example}
\newtheorem{asspt}{Assumption}[section]
\newtheorem{notation}{Notation}[section]
\newtheorem{exercise}{Exercise}[section]
\newtheorem*{fact*}{Fact}
\newtheorem*{rmk*}{Remark}
\newtheorem{fact}{Fact}
\newtheorem*{aside}{Aside}
\newtheorem*{question}{Question}
\newtheorem*{answer}{Answer}

\else\relax\fi

%>>>>>>>>>>>>>>>>>>>>>>>>>>>>>>
%<<<      Fields, etc.      <<<
%>>>>>>>>>>>>>>>>>>>>>>>>>>>>>>
\DeclareSymbolFont{AMSb}{U}{msb}{m}{n}
\DeclareMathSymbol{\N}{\mathbin}{AMSb}{"4E}
\DeclareMathSymbol{\Octonions}{\mathbin}{AMSb}{"4F}
\DeclareMathSymbol{\Z}{\mathbin}{AMSb}{"5A}
\DeclareMathSymbol{\R}{\mathbin}{AMSb}{"52}
\DeclareMathSymbol{\Q}{\mathbin}{AMSb}{"51}
\DeclareMathSymbol{\PP}{\mathbin}{AMSb}{"50}
\DeclareMathSymbol{\I}{\mathbin}{AMSb}{"49}
\DeclareMathSymbol{\C}{\mathbin}{AMSb}{"43}
\DeclareMathSymbol{\A}{\mathbin}{AMSb}{"41}
\DeclareMathSymbol{\F}{\mathbin}{AMSb}{"46}
\DeclareMathSymbol{\G}{\mathbin}{AMSb}{"47}
\DeclareMathSymbol{\Quaternions}{\mathbin}{AMSb}{"48}


%>>>>>>>>>>>>>>>>>>>>>>>>>>>>>>
%<<<       Operators        <<<
%>>>>>>>>>>>>>>>>>>>>>>>>>>>>>>
\DeclareMathOperator{\ad}{\textbf{ad}}
\DeclareMathOperator{\coker}{coker}
\renewcommand{\ker}{\textup{ker}\,}
\DeclareMathOperator{\End}{End}
\DeclareMathOperator{\Aut}{Aut}
\DeclareMathOperator{\Hom}{Hom}
\DeclareMathOperator{\Maps}{Maps}
\DeclareMathOperator{\Mor}{Mor}
\DeclareMathOperator{\Gal}{Gal}
\DeclareMathOperator{\Ext}{Ext}
\DeclareMathOperator{\Tor}{Tor}
\DeclareMathOperator{\Map}{Map}
\DeclareMathOperator{\Der}{Der}
\DeclareMathOperator{\Rad}{Rad}
\DeclareMathOperator{\rank}{rank}
\DeclareMathOperator{\ArfInvariant}{Arf}
\DeclareMathOperator{\KervaireInvariant}{Ker}
\DeclareMathOperator{\im}{im}
\DeclareMathOperator{\coim}{coim}
\DeclareMathOperator{\trace}{tr}
\DeclareMathOperator{\supp}{supp}
\DeclareMathOperator{\ann}{ann}
\DeclareMathOperator{\spec}{Spec}
\DeclareMathOperator{\SPEC}{\textbf{Spec}}
\DeclareMathOperator{\proj}{Proj}
\DeclareMathOperator{\PROJ}{\textbf{Proj}}
\DeclareMathOperator{\fiber}{F}
\DeclareMathOperator{\cofiber}{C}
\DeclareMathOperator{\cone}{cone}
\DeclareMathOperator{\skel}{sk}
\DeclareMathOperator{\coskel}{cosk}
\DeclareMathOperator{\conn}{conn}
\DeclareMathOperator{\colim}{colim}
\DeclareMathOperator{\limit}{lim}
\DeclareMathOperator{\ch}{ch}
\DeclareMathOperator{\Vect}{Vect}
\DeclareMathOperator{\GrthGrp}{GrthGp}
\DeclareMathOperator{\Sym}{Sym}
\DeclareMathOperator{\Prob}{\mathbb{P}}
\DeclareMathOperator{\Exp}{\mathbb{E}}
\DeclareMathOperator{\GeomMean}{\mathbb{G}}
\DeclareMathOperator{\Var}{Var}
\DeclareMathOperator{\Cov}{Cov}
\DeclareMathOperator{\Sp}{Sp}
\DeclareMathOperator{\Seq}{Seq}
\DeclareMathOperator{\Cyl}{Cyl}
\DeclareMathOperator{\Ev}{Ev}
\DeclareMathOperator{\sh}{sh}
\DeclareMathOperator{\intHom}{\underline{Hom}}
\DeclareMathOperator{\Frac}{frac}



%>>>>>>>>>>>>>>>>>>>>>>>>>>>>>>
%<<<   Cohomology Theories  <<<
%>>>>>>>>>>>>>>>>>>>>>>>>>>>>>>
\DeclareMathOperator{\KR}{{K\R}}
\DeclareMathOperator{\KO}{{KO}}
\DeclareMathOperator{\K}{{K}}
\DeclareMathOperator{\OmegaO}{{\Omega_{\Octonions}}}

%>>>>>>>>>>>>>>>>>>>>>>>>>>>>>>
%<<<   Algebraic Geometry   <<<
%>>>>>>>>>>>>>>>>>>>>>>>>>>>>>>
\DeclareMathOperator{\Spec}{Spec}
\DeclareMathOperator{\Proj}{Proj}
\DeclareMathOperator{\Sing}{Sing}
\DeclareMathOperator{\shfHom}{\mathscr{H}\textit{\!\!om}}
\DeclareMathOperator{\WeilDivisors}{{Div}}
\DeclareMathOperator{\CartierDivisors}{{CaDiv}}
\DeclareMathOperator{\PrincipalWeilDivisors}{{PrDiv}}
\DeclareMathOperator{\LocallyPrincipalWeilDivisors}{{LPDiv}}
\DeclareMathOperator{\PrincipalCartierDivisors}{{PrCaDiv}}
\DeclareMathOperator{\DivisorClass}{{Cl}}
\DeclareMathOperator{\CartierClass}{{CaCl}}
\DeclareMathOperator{\Picard}{{Pic}}
\DeclareMathOperator{\Frob}{Frob}


%>>>>>>>>>>>>>>>>>>>>>>>>>>>>>>
%<<<  Mathematical Objects  <<<
%>>>>>>>>>>>>>>>>>>>>>>>>>>>>>>
\newcommand{\sll}{\mathfrak{sl}}
\newcommand{\gl}{\mathfrak{gl}}
\newcommand{\GL}{\mbox{GL}}
\newcommand{\PGL}{\mbox{PGL}}
\newcommand{\SL}{\mbox{SL}}
\newcommand{\Mat}{\mbox{Mat}}
\newcommand{\Gr}{\textup{Gr}}
\newcommand{\Squ}{\textup{Sq}}
\newcommand{\catSet}{\textit{Sets}}
\newcommand{\RP}{{\R\PP}}
\newcommand{\CP}{{\C\PP}}
\newcommand{\Steen}{\mathscr{A}}
\newcommand{\Orth}{\textup{\textbf{O}}}

%>>>>>>>>>>>>>>>>>>>>>>>>>>>>>>
%<<<  Mathematical Symbols  <<<
%>>>>>>>>>>>>>>>>>>>>>>>>>>>>>>
\newcommand{\DASH}{\textup{---}}
\newcommand{\op}{\textup{op}}
\newcommand{\CW}{\textup{CW}}
\newcommand{\ob}{\textup{ob}\,}
\newcommand{\ho}{\textup{ho}}
\newcommand{\st}{\textup{st}}
\newcommand{\id}{\textup{id}}
\newcommand{\Bullet}{\ensuremath{\bullet} }
\newcommand{\sprod}{\wedge}

%>>>>>>>>>>>>>>>>>>>>>>>>>>>>>>
%<<<      Some Arrows       <<<
%>>>>>>>>>>>>>>>>>>>>>>>>>>>>>>
\newcommand{\nt}{\Longrightarrow}
\let\shortmapsto\mapsto
\let\mapsto\longmapsto
\newcommand{\mapsfrom}{\,\reflectbox{$\mapsto$}\ }
\newcommand{\bigrightsquig}{\scalebox{2}{\ensuremath{\rightsquigarrow}}}
\newcommand{\bigleftsquig}{\reflectbox{\scalebox{2}{\ensuremath{\rightsquigarrow}}}}

%\newcommand{\cofibration}{\xhookrightarrow{\phantom{\ \,{\sim\!}\ \ }}}
%\newcommand{\fibration}{\xtwoheadrightarrow{\phantom{\sim\!}}}
%\newcommand{\acycliccofibration}{\xhookrightarrow{\ \,{\sim\!}\ \ }}
%\newcommand{\acyclicfibration}{\xtwoheadrightarrow{\sim\!}}
%\newcommand{\leftcofibration}{\xhookleftarrow{\phantom{\ \,{\sim\!}\ \ }}}
%\newcommand{\leftfibration}{\xtwoheadleftarrow{\phantom{\sim\!}}}
%\newcommand{\leftacycliccofibration}{\xhookleftarrow{\ \ {\sim\!}\,\ }}
%\newcommand{\leftacyclicfibration}{\xtwoheadleftarrow{\sim\!}}
%\newcommand{\weakequiv}{\xrightarrow{\ \,\sim\,\ }}
%\newcommand{\leftweakequiv}{\xleftarrow{\ \,\sim\,\ }}

\newcommand{\cofibration}
{\xhookrightarrow{\phantom{\ \,{\raisebox{-.3ex}[0ex][0ex]{\scriptsize$\sim$}\!}\ \ }}}
\newcommand{\fibration}
{\xtwoheadrightarrow{\phantom{\raisebox{-.3ex}[0ex][0ex]{\scriptsize$\sim$}\!}}}
\newcommand{\acycliccofibration}
{\xhookrightarrow{\ \,{\raisebox{-.55ex}[0ex][0ex]{\scriptsize$\sim$}\!}\ \ }}
\newcommand{\acyclicfibration}
{\xtwoheadrightarrow{\raisebox{-.6ex}[0ex][0ex]{\scriptsize$\sim$}\!}}
\newcommand{\leftcofibration}
{\xhookleftarrow{\phantom{\ \,{\raisebox{-.3ex}[0ex][0ex]{\scriptsize$\sim$}\!}\ \ }}}
\newcommand{\leftfibration}
{\xtwoheadleftarrow{\phantom{\raisebox{-.3ex}[0ex][0ex]{\scriptsize$\sim$}\!}}}
\newcommand{\leftacycliccofibration}
{\xhookleftarrow{\ \ {\raisebox{-.55ex}[0ex][0ex]{\scriptsize$\sim$}\!}\,\ }}
\newcommand{\leftacyclicfibration}
{\xtwoheadleftarrow{\raisebox{-.6ex}[0ex][0ex]{\scriptsize$\sim$}\!}}
\newcommand{\weakequiv}
{\xrightarrow{\ \,\raisebox{-.3ex}[0ex][0ex]{\scriptsize$\sim$}\,\ }}
\newcommand{\leftweakequiv}
{\xleftarrow{\ \,\raisebox{-.3ex}[0ex][0ex]{\scriptsize$\sim$}\,\ }}

%>>>>>>>>>>>>>>>>>>>>>>>>>>>>>>
%<<<    xymatrix Arrows     <<<
%>>>>>>>>>>>>>>>>>>>>>>>>>>>>>>
\newdir{ >}{{}*!/-5pt/@{>}}
\newcommand{\xycof}{\ar@{ >->}}
\newcommand{\xycofib}{\ar@{^{(}->}}
\newcommand{\xycofibdown}{\ar@{_{(}->}}
\newcommand{\xyfib}{\ar@{->>}}
\newcommand{\xymapsto}{\ar@{|->}}

%>>>>>>>>>>>>>>>>>>>>>>>>>>>>>>
%<<<     Greek Letters      <<<
%>>>>>>>>>>>>>>>>>>>>>>>>>>>>>>
%\newcommand{\oldphi}{\phi}
%\renewcommand{\phi}{\varphi}
\let\oldphi\phi
\let\phi\varphi
\renewcommand{\to}{\longrightarrow}
\newcommand{\from}{\longleftarrow}
\newcommand{\eps}{\varepsilon}

%>>>>>>>>>>>>>>>>>>>>>>>>>>>>>>
%<<<  1st-4th & parentheses <<<
%>>>>>>>>>>>>>>>>>>>>>>>>>>>>>>
\newcommand{\first}{^\text{st}}
\newcommand{\second}{^\text{nd}}
\newcommand{\third}{^\text{rd}}
\newcommand{\fourth}{^\text{th}}
\newcommand{\ZEROTH}{$0^\text{th}$ }
\newcommand{\FIRST}{$1^\text{st}$ }
\newcommand{\SECOND}{$2^\text{nd}$ }
\newcommand{\THIRD}{$3^\text{rd}$ }
\newcommand{\FOURTH}{$4^\text{th}$ }
\newcommand{\iTH}{$i^\text{th}$ }
\newcommand{\jTH}{$j^\text{th}$ }
\newcommand{\nTH}{$n^\text{th}$ }

%>>>>>>>>>>>>>>>>>>>>>>>>>>>>>>
%<<<    upright commands    <<<
%>>>>>>>>>>>>>>>>>>>>>>>>>>>>>>
\newcommand{\upcol}{\textup{:}}
\newcommand{\upsemi}{\textup{;}}
\providecommand{\lparen}{\textup{(}}
\providecommand{\rparen}{\textup{)}}
\renewcommand{\lparen}{\textup{(}}
\renewcommand{\rparen}{\textup{)}}
\newcommand{\Iff}{\emph{iff} }

%>>>>>>>>>>>>>>>>>>>>>>>>>>>>>>
%<<<     Environments       <<<
%>>>>>>>>>>>>>>>>>>>>>>>>>>>>>>
\newcommand{\squishlist}
{ %\setlength{\topsep}{100pt} doesn't seem to do anything.
  \setlength{\itemsep}{.5pt}
  \setlength{\parskip}{0pt}
  \setlength{\parsep}{0pt}}
\newenvironment{itemise}{
\begin{list}{\textup{$\rightsquigarrow$}}
   {  \setlength{\topsep}{1mm}
      \setlength{\itemsep}{1pt}
      \setlength{\parskip}{0pt}
      \setlength{\parsep}{0pt}
   }
}{\end{list}\vspace{-.1cm}}
\newcommand{\INDENT}{\textbf{}\phantom{space}}
\renewcommand{\INDENT}{\rule{.7cm}{0cm}}

\newcommand{\itm}[1][$\rightsquigarrow$]{\item[{\makebox[.5cm][c]{\textup{#1}}}]}


%\newcommand{\rednote}[1]{{\color{red}#1}\makebox[0cm][l]{\scalebox{.1}{rednote}}}
%\newcommand{\bluenote}[1]{{\color{blue}#1}\makebox[0cm][l]{\scalebox{.1}{rednote}}}

\newcommand{\rednote}[1]
{{\color{red}#1}\makebox[0cm][l]{\scalebox{.1}{\rotatebox{90}{?????}}}}
\newcommand{\bluenote}[1]
{{\color{blue}#1}\makebox[0cm][l]{\scalebox{.1}{\rotatebox{90}{?????}}}}


\newcommand{\funcdef}[4]{\begin{align*}
#1&\to #2\\
#3&\mapsto#4
\end{align*}}

%\newcommand{\comment}[1]{}

%>>>>>>>>>>>>>>>>>>>>>>>>>>>>>>
%<<<       Categories       <<<
%>>>>>>>>>>>>>>>>>>>>>>>>>>>>>>
\newcommand{\Ens}{{\mathscr{E}ns}}
\DeclareMathOperator{\Sheaves}{{\mathsf{Shf}}}
\DeclareMathOperator{\Presheaves}{{\mathsf{PreShf}}}
\DeclareMathOperator{\Psh}{{\mathsf{Psh}}}
\DeclareMathOperator{\Shf}{{\mathsf{Shf}}}
\DeclareMathOperator{\Varieties}{{\mathsf{Var}}}
\DeclareMathOperator{\Schemes}{{\mathsf{Sch}}}
\DeclareMathOperator{\Rings}{{\mathsf{Rings}}}
\DeclareMathOperator{\AbGp}{{\mathsf{AbGp}}}
\DeclareMathOperator{\Modules}{{\mathsf{\!-Mod}}}
\DeclareMathOperator{\fgModules}{{\mathsf{\!-Mod}^{\textup{fg}}}}
\DeclareMathOperator{\QuasiCoherent}{{\mathsf{QCoh}}}
\DeclareMathOperator{\Coherent}{{\mathsf{Coh}}}
\DeclareMathOperator{\GSW}{{\mathcal{SW}^G}}
\DeclareMathOperator{\Burnside}{{\mathsf{Burn}}}
\DeclareMathOperator{\GSet}{{G\mathsf{Set}}}
\DeclareMathOperator{\FinGSet}{{G\mathsf{Set}^\textup{fin}}}
\DeclareMathOperator{\HSet}{{H\mathsf{Set}}}
\DeclareMathOperator{\Cat}{{\mathsf{Cat}}}
\DeclareMathOperator{\Fun}{{\mathsf{Fun}}}
\DeclareMathOperator{\Orb}{{\mathsf{Orb}}}
\DeclareMathOperator{\Set}{{\mathsf{Set}}}
\DeclareMathOperator{\sSet}{{\mathsf{sSet}}}
\DeclareMathOperator{\Top}{{\mathsf{Top}}}
\DeclareMathOperator{\GSpectra}{{G-\mathsf{Spectra}}}
\DeclareMathOperator{\Lan}{Lan}
\DeclareMathOperator{\Ran}{Ran}

%>>>>>>>>>>>>>>>>>>>>>>>>>>>>>>
%<<<     Script Letters     <<<
%>>>>>>>>>>>>>>>>>>>>>>>>>>>>>>
\newcommand{\scrQ}{\mathscr{Q}}
\newcommand{\scrW}{\mathscr{W}}
\newcommand{\scrE}{\mathscr{E}}
\newcommand{\scrR}{\mathscr{R}}
\newcommand{\scrT}{\mathscr{T}}
\newcommand{\scrY}{\mathscr{Y}}
\newcommand{\scrU}{\mathscr{U}}
\newcommand{\scrI}{\mathscr{I}}
\newcommand{\scrO}{\mathscr{O}}
\newcommand{\scrP}{\mathscr{P}}
\newcommand{\scrA}{\mathscr{A}}
\newcommand{\scrS}{\mathscr{S}}
\newcommand{\scrD}{\mathscr{D}}
\newcommand{\scrF}{\mathscr{F}}
\newcommand{\scrG}{\mathscr{G}}
\newcommand{\scrH}{\mathscr{H}}
\newcommand{\scrJ}{\mathscr{J}}
\newcommand{\scrK}{\mathscr{K}}
\newcommand{\scrL}{\mathscr{L}}
\newcommand{\scrZ}{\mathscr{Z}}
\newcommand{\scrX}{\mathscr{X}}
\newcommand{\scrC}{\mathscr{C}}
\newcommand{\scrV}{\mathscr{V}}
\newcommand{\scrB}{\mathscr{B}}
\newcommand{\scrN}{\mathscr{N}}
\newcommand{\scrM}{\mathscr{M}}

%>>>>>>>>>>>>>>>>>>>>>>>>>>>>>>
%<<<     Fractur Letters    <<<
%>>>>>>>>>>>>>>>>>>>>>>>>>>>>>>
\newcommand{\frakQ}{\mathfrak{Q}}
\newcommand{\frakW}{\mathfrak{W}}
\newcommand{\frakE}{\mathfrak{E}}
\newcommand{\frakR}{\mathfrak{R}}
\newcommand{\frakT}{\mathfrak{T}}
\newcommand{\frakY}{\mathfrak{Y}}
\newcommand{\frakU}{\mathfrak{U}}
\newcommand{\frakI}{\mathfrak{I}}
\newcommand{\frakO}{\mathfrak{O}}
\newcommand{\frakP}{\mathfrak{P}}
\newcommand{\frakA}{\mathfrak{A}}
\newcommand{\frakS}{\mathfrak{S}}
\newcommand{\frakD}{\mathfrak{D}}
\newcommand{\frakF}{\mathfrak{F}}
\newcommand{\frakG}{\mathfrak{G}}
\newcommand{\frakH}{\mathfrak{H}}
\newcommand{\frakJ}{\mathfrak{J}}
\newcommand{\frakK}{\mathfrak{K}}
\newcommand{\frakL}{\mathfrak{L}}
\newcommand{\frakZ}{\mathfrak{Z}}
\newcommand{\frakX}{\mathfrak{X}}
\newcommand{\frakC}{\mathfrak{C}}
\newcommand{\frakV}{\mathfrak{V}}
\newcommand{\frakB}{\mathfrak{B}}
\newcommand{\frakN}{\mathfrak{N}}
\newcommand{\frakM}{\mathfrak{M}}

\newcommand{\frakq}{\mathfrak{q}}
\newcommand{\frakw}{\mathfrak{w}}
\newcommand{\frake}{\mathfrak{e}}
\newcommand{\frakr}{\mathfrak{r}}
\newcommand{\frakt}{\mathfrak{t}}
\newcommand{\fraky}{\mathfrak{y}}
\newcommand{\fraku}{\mathfrak{u}}
\newcommand{\fraki}{\mathfrak{i}}
\newcommand{\frako}{\mathfrak{o}}
\newcommand{\frakp}{\mathfrak{p}}
\newcommand{\fraka}{\mathfrak{a}}
\newcommand{\fraks}{\mathfrak{s}}
\newcommand{\frakd}{\mathfrak{d}}
\newcommand{\frakf}{\mathfrak{f}}
\newcommand{\frakg}{\mathfrak{g}}
\newcommand{\frakh}{\mathfrak{h}}
\newcommand{\frakj}{\mathfrak{j}}
\newcommand{\frakk}{\mathfrak{k}}
\newcommand{\frakl}{\mathfrak{l}}
\newcommand{\frakz}{\mathfrak{z}}
\newcommand{\frakx}{\mathfrak{x}}
\newcommand{\frakc}{\mathfrak{c}}
\newcommand{\frakv}{\mathfrak{v}}
\newcommand{\frakb}{\mathfrak{b}}
\newcommand{\frakn}{\mathfrak{n}}
\newcommand{\frakm}{\mathfrak{m}}

%>>>>>>>>>>>>>>>>>>>>>>>>>>>>>>
%<<<  Caligraphic Letters   <<<
%>>>>>>>>>>>>>>>>>>>>>>>>>>>>>>
\newcommand{\calQ}{\mathcal{Q}}
\newcommand{\calW}{\mathcal{W}}
\newcommand{\calE}{\mathcal{E}}
\newcommand{\calR}{\mathcal{R}}
\newcommand{\calT}{\mathcal{T}}
\newcommand{\calY}{\mathcal{Y}}
\newcommand{\calU}{\mathcal{U}}
\newcommand{\calI}{\mathcal{I}}
\newcommand{\calO}{\mathcal{O}}
\newcommand{\calP}{\mathcal{P}}
\newcommand{\calA}{\mathcal{A}}
\newcommand{\calS}{\mathcal{S}}
\newcommand{\calD}{\mathcal{D}}
\newcommand{\calF}{\mathcal{F}}
\newcommand{\calG}{\mathcal{G}}
\newcommand{\calH}{\mathcal{H}}
\newcommand{\calJ}{\mathcal{J}}
\newcommand{\calK}{\mathcal{K}}
\newcommand{\calL}{\mathcal{L}}
\newcommand{\calZ}{\mathcal{Z}}
\newcommand{\calX}{\mathcal{X}}
\newcommand{\calC}{\mathcal{C}}
\newcommand{\calV}{\mathcal{V}}
\newcommand{\calB}{\mathcal{B}}
\newcommand{\calN}{\mathcal{N}}
\newcommand{\calM}{\mathcal{M}}

%>>>>>>>>>>>>>>>>>>>>>>>>>>>>>>
%<<<<<<<<<DEPRECIATED<<<<<<<<<<
%>>>>>>>>>>>>>>>>>>>>>>>>>>>>>>

%%% From Kac's template
% 1-inch margins, from fullpage.sty by H.Partl, Version 2, Dec. 15, 1988.
%\topmargin 0pt
%\advance \topmargin by -\headheight
%\advance \topmargin by -\headsep
%\textheight 9.1in
%\oddsidemargin 0pt
%\evensidemargin \oddsidemargin
%\marginparwidth 0.5in
%\textwidth 6.5in
%
%\parindent 0in
%\parskip 1.5ex
%%\renewcommand{\baselinestretch}{1.25}

%%% From the net
%\newcommand{\pullbackcorner}[1][dr]{\save*!/#1+1.2pc/#1:(1,-1)@^{|-}\restore}
%\newcommand{\pushoutcorner}[1][dr]{\save*!/#1-1.2pc/#1:(-1,1)@^{|-}\restore}










\newcommand{\myheading}[1]
{{\noindent\Large #1}

}
\newcommand{\shrt}[1]{\makebox[0cm]{\ensuremath{#1}}}
\renewcommand{\myheading}[1]{\subsection{#1}}
\setcounter{secnumdepth}{1}
\begin{document}
\tableofcontents

\pagebreak

\myheading{The Spectral Sequence Associated with a Filtered Chain Complex}
\newcommand{\theexp}[4]{\raisebox{-.1cm}[.5cm][0cm]{\mbox{$\begin{array}{c}H_{#1}(\frac{F_{#2}}{F_{#3}}) \\=E^1_{#4} \end{array}$}}}%used in the Filtered chain complex SS
Suppose that $C_*=\varinjlim(F_0C_*\hookrightarrow F_1C_*\hookrightarrow F_2C_*\hookrightarrow\cdots)$ is a filtered chain complex. Then there is a spectral sequence $\{E^r_{s,t}\}$ converging to $H_{s+t}(C_*)$, such that $E^1_{s,t}=H_{s+t}(F_sC_*/F_{s-1}C_*)$. Here the target of convergence is graded via $F_s(H_{s+t}(C_*))=\im(H_{s+t}(F_sC_*)\to H_{s+t}(C_*))$. Thus it is claimed that $E^\infty_{s,t}\simeq F_s(H_{s+t}(C_*))/F_{s-1}(H_{s+t}(C_*))$. Consider the following diagram:
% \[\xymatrix{
% &&&0\ar@{.>}[d]&\\
% &&&H_{s+t-1}(F_{s-r-1})\ar[d]^\alpha \ar[r]&\theexp{s+t-1}{s-r-1}{s-r-2}{s-r-1,t+r}\\
% &&&H_{s+t-1}(F_{s-r})\ar@{.>}[d] \ar[r]^{g'}&\theexp{s+t-1}{s-r}{s-r-1}{s-r,t+r-1}\\
% &&&H_{s+t-1}(F_{s-2}) \ar[d]\ar[r]&\theexp{s+t-1}{s-2}{s-3}{s-2,t+1}\\
% \theexp{s+t+1}{s+1}{s}{s+1,t}\ar[r]&H_{s+t}(F_s)\ar[r]^g\ar[d]&
% \theexp{s+t}{s}{s-1}{s,t}\ar[r]^\partial\ar@/^2pc/[rruu]^(.3){\text{``$d_r$''}}
% &H_{s+t-1}(F_{s-1}) \ar[r]&\theexp{s+t-1}{s-1}{s-2}{s-1,t}\\
% \theexp{s+t+1}{s+2}{s+1}{s+2,t-1}\ar[r]&H_{s+t}(F_{s+1})\ar@{.>}[d]\\
% \theexp{s+t+1}{s+r}{s+r-1}{s+r,t-r+2}\ar[r]^{\partial'}&H_{s+t}(F_{s+r-1})\ar[d]^{\beta}\\
% \theexp{s+t+1}{s+r+1}{s+r}{s+r,t-r+1}\ar[r]&H_{s+t}(F_{s+r})\ar@{.>}[d]\\
% &H_{s+t}(C_*)
% }\]
%
\[\xymatrix@C=.5cm@R=.7cm{
&&&0\ar@{.>}[d]&\\
&&&H_{s+t-1}(F_{s-r-1})\ar[d]^\alpha \ar[r]&\theexp{s+t-1}{s-r-1}{s-r-2}{s-r-1,t+r}\\
&H_{s+t}(F_{s-1})\ar[d]^\gamma&&H_{s+t-1}(F_{s-r})\ar@{.>}[d] \ar[r]^{g'\ \ }&\theexp{s+t-1}{s-r}{s-r-1}{s-r,t+r-1}\\
%&&&H_{s+t-1}(F_{s-2}) \ar[d]\ar[r]&\theexp{s+t-1}{s-2}{s-3}{s-2,t+1}\\
%the bendy arrow was:\ar@/^2pc/[rruu]^(.3){\text{``$d_r$''}}
\theexp{s+t+1}{s+1}{s}{s+1,t}\ar[r]&H_{s+t}(F_s)\ar[r]^{g\ \ }\ar@{.>}[d]&
\theexp{s+t}{s}{s-1}{s,t}\ar[r]_\partial\ar[rru]^(.4){\text{``$d_r$''}}
&H_{s+t-1}(F_{s-1}) \ar[r]&\theexp{s+t-1}{s-1}{s-2}{s-1,t}\\
%\theexp{s+t+1}{s+2}{s+1}{s+2,t-1}\ar[r]&H_{s+t}(F_{s+1})\ar@{.>}[d]\\
\theexp{s+t+1}{s+r}{s+r-1}{s+r,t-r+2}\ar[r]_{\ \ \partial'}\ar[rru]^(.4){\text{``$d_r$''}}&H_{s+t}(F_{s+r-1})\ar[d]^{\beta}\\
\theexp{s+t+1}{s+r+1}{s+r}{s+r,t-r+1}\ar[r]&H_{s+t}(F_{s+r})\ar@{.>}[d]
&&\makebox[0cm]{
{\small {(Transcribed from a lecture  by Mark Behrens, 29/3/11.)}}
}
\\
&H_{s+t}(C_*)
}\]
\begin{alignat*}{2}
\text{Define:\qquad } E^r_{s,t}&:=Z^r_{s,t}/B^r_{s,t}\text{, where:}\\
Z^r_{s,t}&:=\partial^{-1}(\im(H_{s+t-1}(F_{s-r})\to H_{s+t-1}(F_{s-1})))\subseteq E^1_{s,t}\\
B^r_{s,t}&:=g(\ker(H_{s+t}(F_{s})\to H_{s+t}(F_{s+r-1})))\subseteq E^1_{s,t}
\end{alignat*}
Given an element $x\in Z^r_{s,t}$, there is an element $a\in H_{s+t-1}(F_{s-r})$ such that $x$ and $a$ have the same image in $H_{s+t-1}(F_{s-1})$. Then $g'(a)$ is an element of $Z^r_{s-r,t+r-1}$. Different choices for $a$ differ by elements of $\ker(H_{s+t-1}(F_{s-r})\to H_{s+t-1}(F_{s-1}))$, so that the class of $g'(a)$ in $E^r_{s-r,t+r-1}$ is well defined. Then the map $\partial_r:E^r_{s,t}\to E^r_{s-r,t+r-1}$ is defined via this process, and $\partial_r^2=0$.

Now the elements of $E^1_{s,t}$ which represent an element of the kernel of $d_r$ are exactly the elements of $Z^{r+1}_{s,t}$, since $\ker(g')=\im(\alpha)$. Similarly, the elements of $E^1_{s,t}$ which represent an element of the image of $d_r$ are exactly the elements of $B^{r+1}_{s,t}$, since $\ker(\beta)=\im(\partial')$. Thus $H_{s,t}(E^r_{*,*},\partial_r)\simeq E^{r+1}_{s,t}$ as desired.

Finally, we should find an isomorphism $E^\infty_{s,t}\to F_s(H_{s+t}(C_*))/F_{s-1}(H_{s+t}(C_*))$. For this, note that $Z^\infty_{s,t}=\ker\partial=\im(g)$ and $B^\infty_{s,t}=g(\ker(H_{s+t}(F_s)\to H_{s+t}(C_*)))$. So given $x=g(y)\in Z^\infty_{s,t}$, $y$ maps via the vertical arrows to an element of $F_s(H_{s+t}(C_*))$. As $\ker(g)=\im(\gamma)$, this is well defined up to an element of $F_{s-1}(H_{s+t}(C_*))$, so we have defined a map $Z^\infty_{s,t}\to F_s(H_{s+t}(C_*))/F_{s-1}(H_{s+t}(C_*))$. A brief diagram chase reveals that the kernel is $B^\infty_{s,t}$ and that the map is onto.
%
%\begin{flushright}
 %{\small Transcribed  by Michael Donovan from a lecture of Mark Behrens, 29 March, 2011.}
%\end{flushright}


\myheading{A Derivation of the Leray-Serre-Atiyah-Hirzebruch Spectral Sequence}
Suppose that $\pi:E\to B$ is a Hurewicz fibration, with fiber $F$, where $B$ is a (finite?) connected CW-complex. Let $B^n$ be the $n$-skeleton of $B$, and define $E^n:=\pi^{-1}B^n$. Let $h_*$ be an unreduced homology theory. The long exact sequences for the pairs $(E^p,E^{p-1})$ can be written as in the following diagram. Here, wavy arrows indicate a map with degree $-1$, and each triangle is exact.
\[\xymatrix@C=.5cm{
0\ar[rr]|{\ \cdot\ \cdot\ \cdot\ }&&h_*(E^{p-1})\ar[rr]^i\ar[ld]_\rho&&h_*(E^{p})\ar[rr]^i \ar[ld]_\rho&&h_*(E^{p+1})\ar[ld]_\rho
\ar[rr]|{\ \cdot\ \cdot\ \cdot\ }&&h_*(E)\\
%&\cdots\ar[r]&h_*(E)\\
&\ \ \ \ \shrt{h_*(E^{p-1},E^{p-2})}\phantom{{5\choose3}}&&
\shrt{h_*(E^{p},E^{p-1})}\phantom{{5\choose3}}\ar@{~>}[lu]_\partial&&
\shrt{h_*(E^{p+1},E^{p})}\phantom{{5\choose3}}\ar@{~>}[lu]_\partial&&
\shrt{h_*(E^{p+2},E^{p+1})}\phantom{{5\choose3}}\ar@{~>}[lu]_\partial
}\]
Define, for $r\geq0$:
\begin{alignat*}{2}
Z_{pq}^r:=&\left\{x\in h_{p+q}(E^p,E^{p-1})\ \middle|\ \partial x\in\im(i^{r})\right\}\\
=&\left\{\text{relative classes $x$ whose boundary comes from $h_{p+q-1}(E^{p-r-1})$}\right\}.\\
B_{pq}^r:=&\left.\rho(\ker(i^r))\right.\\
=&\left\{\rho(y)\ \middle|\ y\in h_{p+q}(E^p),\ i^r(y)=0\in h_{p+q}(E^{p+r})\right\}\\
=&\left\{\text{restrictions $\rho(y)$ of classes $y\in h_{p+q}(E^p)$ which are nullhomologous in $E^{p+r}$}\right\}.\\
E_{pq}^{r+1}:=&\left.Z_{pq}^r/B_{pq}^r.\right.
\end{alignat*}
As each triangle is exact, $B_{pq}^r\subseteq E_{pq}^r$ for all $p,q,r$, so that the definition of $E_{pq}^{r+1}$ makes sense.
Note that $Z_{pq}^0=h_{p+q}(E^p,E^{p-1})$ and $B_{pq}^0=0$, so that $E_{pq}^1=h_{p+q}(E^p,E^{p-1})$.

For each $p,q\geq0$, let $i_{pq}:h_{p+q}(E^p)\to h_{p+q}(E)$ be the map on homology induced by the inclusion. Then there is an increasing filtration on $h_{p+q}(E)$ defined by $F_{p}h_{p+q}(E):=\im(i_{pq})$. Let $\Gr_{pq}$ be the $p^\text{th}$ subquotient $F_{p}h_{p+q}(E)/F_{p-1}h_{p+q}(E)$. Also, define $Z^\infty_{pq}$ and $B^\infty_{pq}$ as follows:
\begin{alignat*}{2}
Z_{pq}^\infty:=&\left.\bigcap_r Z_{pq}^r\right.=\left\{x\in h_{p+q}(E^p,E^{p-1})\ \middle|\ \partial x=0\right\}\\
%=&\left.\im(\rho:h_{p+q}(E^p)\to h_{p+q}(E^p,E^{p-1}))\right.\\
=&\left.\rho(h_{p+q}(E^p)).\right.\\
B_{pq}^\infty:=&\left.\bigcup_r B_{pq}^r\right.=\left\{\rho(y)\ \middle|\ y\in h_{p+q}(E^p),\ i_{pq}(y)=0\in h_{p+q}(E)\right\}\\
=&\left.\rho(\ker (i_{pq})).\right.\\
\Gr_{pq}=&\left.\im\bigl(h_{p+q}(E^p)\rightarrow h_{p+q}(E)\bigr)/\im\bigl(h_{p+q}(E^{p-1})\rightarrow h_{p+q}(E)\bigr).\right.
\end{alignat*}

Define a isomorphism $E^\infty_{pq}:=Z_{pq}^\infty/B_{pq}^\infty\to \Gr_{pq}$
by $[\rho(y)]\mapsto[i_{pq}(y)]$, for $y\in h_{p+q}(E^p)$. A diagram chase reveals that this is well defined and an isomorphism.

Finally, we can define differentials $d_{r+1}:E^{r+1}_{pq}\to E^{r+1}_{p-r-1,q+r}$ by the recipe $[x]\mapsto [\rho i^{-r}\partial(x)]$, for $x\in Z_{pq}^{r}$. A rather larger diagram chase reveals that this gives a differential on the groups $E^{r+1}_{pq}$ of bidegree $(-r-1,r)$, whose homology at $E^{r+1}_{pq}$ is $E^{r+2}_{pq}$, so that we have defined a spectral sequence converging to $h_{p+q}(E^p)$.



\myheading{Principal $G$-bundles on $\Sigma A$ --- clutching v.\ classifying}
Let $A$ be a pointed space, and $G$ a topological group. Suppose that $\beta:A\to \Omega BG$ and $\gamma:\Sigma A\to BG$ are adjoint maps. We can produce a principal $G$-bundle on $\Sigma A$ by either using a clutching construction with $\beta$, or pulling back $EG$ along $\gamma$. This note shows that the two bundles obtained are isomorphic.

Firstly, use the HLP to construct the map $L$ below (where $\star$ is the constant map to the basepoint of $EG$, and $\widetilde\epsilon$ is evaluation $(t,\sigma)\mapsto \sigma(t)$):
\[\xymatrix{
\{0\}\times\Omega BG\ar[r]^{\ \ \ \ \star}\ar[d]&EG\ar[d]^{\pi}\\
I\times\Omega BG\ar[r]^{\ \ \ \ \widetilde\epsilon}\ar@{-->}[ru]^L&BG
}\]
Then the equivalence $\phi:\Omega BG\to G$ is given by $\sigma\mapsto L(1,\sigma)$, where $G$ is identified with the fiber of $\pi$. Now define:
%\[E_cG:=G\times C(\Omega BG)\sqcup G / (g,1,\sigma)\sim gL(1,\sigma).\]
%There is a map $E_cG\to \Sigma\Omega BG$ defined by sending $(g,t,\sigma)\mapsto(t,\sigma)$. Moreover, there is a map $E_cG\to EG$ defined by $(g,t,\sigma)\mapsto gL(t,\sigma)$. Thus we have a morphism of principal $G$-bundles lying over the counit $\epsilon_{BG}$:
%\[\xymatrix{
%E_cG\ar[r]\ar[d]&EG\ar[d]\\
%\Sigma\Omega BG\ar[r]^{\ \ \epsilon_{BG}}&BG
%}\]%\text{\ \ which implies that }\]
%Thus $E_cG\simeq\epsilon_{BG}^*(EG)$. Now, let $E_c\beta$ be the space:
%\[E_c\beta:=G\times C(A)\sqcup G / (g,1,a)\sim gL(1,\beta(a)).\]
%Then $E_c\beta$ maps to $\Sigma A$ via $(g,t,a)\mapsto(t,a)$. Moreover, there is a map $E_c\beta\to E_cG$ defined by $(g,t,a)\mapsto (g,t,\beta(a))$.
\[E_c\beta:=G\times C(A)\sqcup G / (g,1,a)\sim gL(1,\beta(a)).\]
Then $E_c\beta$ maps to $\Sigma A$ via $(g,t,a)\mapsto(t,a)$. Moreover, there is a map $E_c\beta\to EG$ defined by $(g,t,a)\mapsto gL(t,\beta(a))$, which makes the following diagram commute (as is checked): 
\[\xymatrix{
E_c\beta\ar[rr]\ar[d]&&EG\ar[d]&
(g,t,a)\ar@{|->}[rr]\ar@{|->}[d]&&gL(t,\beta(a))\ar@{|->}[d]\\
\Sigma A\ar[r]^{\Sigma\beta}&\Sigma\Omega BG\ar[r]^{\epsilon_{BG}}&BG&
(t,a)\ar@{|->}[r]&(t,\beta(a))\ar@{|->}[r]&\beta(a)(t)
}\]
As $\gamma=\epsilon_{BG}\circ\Sigma\beta$, this shows that $E_cB$ (obtained from $\beta$ by clutching) is isomorphic to $\gamma^*EG$.


\myheading{The Projective Bundle Theorem}
Suppose that $\xi\downarrow B$ is a complex $n$-plane bundle. Let $\PP(\xi)\downarrow B$ be the associated $\PP^{n-1}$-bundle. Then $H^*(\PP(\xi))$ is a free $H^*(B)$-module on generators $1,e,e^2,\ldots,e^{n-1}$, where $e\in H^*(\PP(\xi))$ is the Euler class of the tautologous line bundle $L\downarrow \PP(\xi)$.

For each $x\in B$, let $F_x$ be the fiber above $x$, and let $L_x$ be the restriction of $L$ to that fiber. Thus $L_x$ is the tautologous line bundle over $F_x$, and $H^*(F_x)$ is generated by the euler class $e(L_x)=e|_{F_x}$. Thus the coefficient system is trivial. Moreover, the homomorphism $H^*(\PP(\xi))\to H^*(F)$ is surjective, where $F$ is the fiber over the basepoint of $B$.

Consider now the fibration $F \rightarrow\PP(\xi)\downarrow B$. The transgression is defined on all of $H^*(F)$, as we have the following diagram which describes the transgression:
\[\xymatrix{
H^*(\PP(\xi))\ar@{->>}[r]&H^*(F)\ar[r]^{0\ \ \ \ \ }&H^{*+1}(\PP(\xi),F)&E_\infty^{*+1,0}\\
&E_\infty^{0*}\ar@{^{(}->}[u]\ar@{.>}[rur]&H^{*+1}(B,*)\ar[u]\ar[r]&H^{*+1}(B)\ar@{->>}[u]
}\]
Moreover, this shows that $H^*(F)$ consists only of permanent cycles, so the Serre Spectral Sequence collapses at $E_2$. Thus the $E_\infty$ is a free $H^*(B)$-module, compatibly with the filtering on $B$, so that $H^*(\PP(\xi))$ is a free $H^*(B)$-module.

%To see this, c We should check first that the coefficient system $H^*(\PP(\xi)_x)$ on $B$ is trivial. It is enough to show that $\pi_1(B)$ acts trivially on $H^2(\PP(\xi))$, as each coefficient group is generated by $H^2$.



\myheading{Discussion of the EHP spectral sequence}
For every integer $p$, at the prime 2, there is a fibre sequence: $\xymatrix{S^p\ar[r]^e&\Omega S^{p+1}\ar[r]^h& \Omega S^{2p+1}}$. Stringing these together for $0\leq p<M$, we have a diagram of fibrations:
\[\xymatrix{
S^0\ar[r]&\Omega S^1\ar[d]\ar[r]&\Omega^2 S^2\ar[d]\ar[r]&\Omega^3 S^3\ar[d]\ar@{.>}[rr]&&\Omega^{p+1}S^{p+1}\ar[d]\ar@{.>}[rr]&&\Omega^{M}S^{M}\ar[d]\\
&\Omega S^1&\Omega^2 S^3& \Omega^3S^5&&\Omega^{p+1}S^{2p+1}&&\Omega^{M}S^{2M-1}
}\]
Applying $\pi_*$ yields an exact couple, and we obtain an exact sequence. This exact sequence has exactly $M$ nonzero columns ($0\leq p <M$), and converges to the stem $\pi_{p+q}(\Omega^{M}S^{M})$ of $S^{M}$. Now:
\[E_{pq}^1=\pi_{p+q}(\Omega^{p+1}S^{2p+1})=\pi_q(\Omega^{2p+1}S^{2p+1})\text{ which is stable for $q< 2p$}.\]
Moreover:
\[
\frac{\im\left(e^{M-p-1}:\pi_{p+q}(\Omega^{p+1}S^{p+1})\to\pi_{p+q}(\Omega^MS^M)\right)}
{\im\left(\phantom{e^{M-p-1}:}\makebox[0cm][r]{$e^{M-p}\,\ :$}
\,\makebox[0cm][l]{$\pi_{p+q}(\Omega^{p}S^{p})$}\phantom{\pi_{p+q}(\Omega^{p+1}S^{p+1})}
\to\pi_{p+q}(\Omega^MS^M)\right)}\simeq E_{pq}^\infty\text{\ \ via \ }e^{M-p-1}(\sigma)\mapsto h_*\sigma.\]%\text{ for }\sigma\in\pi_{p+q}(\Omega^{p+1}S^{p+1}) \]
Thus $E^\infty_{pq}$ records those classes in the $(p+q)$-stem of $S^M$ which are born on $S^{p+1}$. Moreover, the way such a class in recorded on the $E^\infty$ page is by its Hopf invariant. Thus, if a class $\sigma\in\pi_{p+q}(\Omega^{p+1}S^{p+1})$ is the birthplace of an element in the $(p+q)$-stem of $S^M$, it is recorded as $h_*\sigma\in\pi_q({\Omega^{2p+1}S^{2p+1}})$ in $E^\infty_{pq}$.

When $q$ is $0$, $h_*\sigma$ is an element of the stable $0$-stem, and thus a genuine integer, which happily equals the old-fashioned Hopf invariant. More generally, when $q\leq 2p-2$, $h_*\sigma$ lands in the stable $q$-stem, so terms on the $E^\infty$ page which lie below a line of slope two are recorded by a `Hopf invariant' which lies in a stable homotopy group of spheres.
% isomorphism is given by $e^{M-p-1}(\sigma)\mapsto h_*\sigma$, when $\sigma$ is in the $(p+q)$-stem of $S^{p+1}$.



\myheading{A couple of applications of the Steenrod squares}
\textbf{Application 1:} we show that $\eta^2:S^2\to S^0$ is not stably null, where $\eta$ is the
Hopf fibration. Whenever we have a homotopy commuting square, we can complete it
to a larger grid with rows and columns cofiber sequences. Note that we need to
use the same choice of homotopy to induce the two squiggly maps:
\[\xymatrix{
S^2\ar[r]^{\Sigma\eta}\ar[d]&S^1\ar[r]\ar[d]^\eta&\Sigma C\eta\ar@{~>}[d]\\
\text{$*$}\ar[r]\ar[d]&S^0\ar[r]\ar[d]&S^0\ar[d]\\
S^3\ar@{~>}[r]&C\eta\ar[r]^\beta&X\ar[d]^\gamma\\
&&\Sigma^2C\eta
}\]
Now it can be checked that $X$ has a cell in dimensions $0,2,4$. Moreover,
$\beta$ induces an isomorphism on $H^*(\DASH;\Z_2)$ for $*=0,2$, and gamma does
so for $*=2,4$. Thus if $\phi_0,\phi_2,\phi_4$ are the various generators, we
have $\Squ^2\phi_0=\phi_2$ and $\Squ^2\phi_2=\phi_4$. This contradicts the Adem
relation $\Squ^{22}=\Squ^{31}$.

\textbf{Application 2:}  let $M$ be the cofiber of the map $2:S^0\to S^0$ (of
spectra). We would like to see that $2\id_M:M\to M$ is not null. For this,
suppose that it is in fact null. Then, as $2\id_M=2\wedge\id_M$, we have
$C(2\id_M)=M\wedge M$.

Thus, to solve our problem, it is enough to check that $M\wedge M$ and $M\vee
\Sigma M$ have different $H^*(\DASH;\Z_2)$ (as $\Steen$-algebras). We calculate
first that $H^*(M;\Z_2)$ is $\Z_2$ in $*=0,1$. Thus $\Squ^2$ acts as zero on the
cohomology of $M\vee \Sigma M$. On the other hand, we can calculate directly
that $\Squ^2$ is nonzero on the base class in $M\wedge M$.



\myheading{The existence of an $A$-Adams resolution}
In this note we indicate why $A$-Adams resolutions exist. The context is the paper
``On relations between Adams spectral sequences'' by Haynes Miller, page 289.

Suppose that we have a chosen $A$-resolution $*\to X\to I^0\to I^1\to\cdots$ of
$X$. Suppose by induction on $s$ that we have a partial $A$-Adams resolution of
length $s$:
\[\xymatrix@C=.1cm{
&X^{s-2}\ar[dr]&&X^{s-1}\ar[dr]\ar[ll]&&X^s\ar[ll]\\
&&\Sigma^{s-2}I^{s-2}\ar@{-->}[rr]\ar@{-->}[ur]
&&\Sigma^{s-1}I^{s-1}\ar@{-->}[rr]\ar@{-->}[ur]&&\Sigma^{s}I^{s}
}\]
The dashed maps lower degree by one, in the sense that their target is the
desuspension of what is shown in the diagram. The bottom row should be the
resolution of $X$ (and $X=X_0\to I^0$ should be the initial map in the
resolution). Moreover, the sequences $*\to X^i\to\Sigma^iI^i\to\Sigma
X^{i+1}\to*$ should be $A$-exact, and of course, the top triangles must be
extended (co)-fiber sequences, and the bottom triangles should commute.

It is enough to show that there is a map $X^s\to \Sigma^sI^s$ making the
rightmost triangle commute, and to demonstrate that this map is $A$-monic. For
then, by lemma 1.5, we can define $X^{s+1}$ to be the fiber of this map. For
existence, we do a little diagram chasing:
\[\xymatrix{
\bullet\ar[rd]_5\ar[rrrrrd]_3&&\bullet\ar[rrrd]^1\ar[ll]_4&&\bullet\\
&\bullet\ar[ru]_2\ar@/_.31pc/[rrrr]^(.3){6}&&\bullet&&\bullet
}\]
The map exists iff the composite $1$ is null, by exactness in the upper right
triangle. Now $1\circ2$ is null, so that there exists a map $3$ such that
$1=3\circ4$. The map $5$ is $A$-monic, and the target is $A$-injective, so that
there exists a map $6$ such that $3=6\circ5$. Then $1=6\circ5\circ4$, and
$5\circ4$ is null.

To check that the map is $A$-monic, now that it is known to exist, we map
everything into an arbitrary $A$-injective object $I$, giving the following
purely algebraic situation (where numbers represent homsets), in which the
bottom row is exact, both triangles commute, and the middle vertical map is
injective:
\[\xymatrix{
&&&2\ar[lldl]_0\ar@{^{(}->}[ld]\\
3&&\ar[ll]4&&1\ar[ll]\ar[lu]_{\text{epi?}}
}\]
Now anything in $2$ maps to zero in 3, so $\im(2\to4)\subseteq\im(1\to4)$ by
exactness of the bottom row. Yet then, as $2\to4$ is monic, $1\to2$ is epic.



\myheading{The definition of an abelian category}
\noindent A category is \emph{additive} if:
\begin{itemise}
\itm the homsets are abelian groups, and composition is biadditive;
\itm there is a zero object; and
\itm finite biproducts exist.
\end{itemise}
A category is \emph{pre-abelian} if:
\begin{itemise}
\itm it is additive; and
\itm every morphism has a kernel and a cokernel
\end{itemise}
A category is \emph{abelian} if:
\begin{itemise}
\itm it is pre-abelian; and
\itm every monomorphism is normal and every epimorphism is conormal. That is,
every monomorphism is a kernel, and every epimorphism is a cokernel. This agrees
with the meaning of the term `normal subgroup'.
\end{itemise}
\myheading{$\Q$ and $\C$ are free $\Z$-modules}
Alternative definition: a flat $R$-module is an $R$-module $M$ such that for
every finitely generated ideal $I\subseteq R$, the induced morphism $I\otimes_R
M\to R\otimes_R M=M$ is injective.

If $R=\Z$ and $M=\Q$, we need to check $I=n\Z$ for any $n\in\Z$. If $n\neq\pm1$,
the induced morphism is $0\to\Q$, otherwise it is $\Q\to\Q$. Thus $\Q$ is flat.
To tensor with $\C$ over $\Z$, we may first tensor with $\Q$ over $\Z$ then with
$\C$ over $\Q$. As $\C$ is a free $\Q$-module, it is flat over $\Q$, and thus it
is flat over $\Z$.
\end{document}













